\chapter{Additional Simulation Results}\label{sec:extra_results}

In this section, we briefly explore additional simulations testing
variations in the halo concentration, cored halos, alternative orbits,
anisotropies, and ellipticities. While each of these variables
influences total dark matter evolution, the resulting inner structure is
similar across models. We conclude that consideration of these effects
likely would not change our conclusions.

\section{Alternative initial
conditions}\label{alternative-initial-conditions}

The following models aim to reproduce the stellar velocity dispersion of
Sculptor to within \(\lesssim 1\,\kms\) with a similar present-day
half-light radius.

\subsection{Halo concentration}\label{halo-concentration}

Changing the initial concentration (or \(\vmax\) and \(\rmax\))
primarily affects total dark matter evolution.
Fig.~\ref{fig:tidal_tracks_concentration} compares models on the
\smallperi{} orbit with different initial halo parameters. The heavier
halo has \(\vmax=43\) and \(\rmax=7\), whereas the lighter halo has
\(\vmax=25\) and \(\rmax=2.5\). While the less concentrated halo loses
more mass, the halo evolves to similar final structural parameters.

The heavier halo diverges from the initial point orbit more
substantially. We show the results of correcting the orbit (akin to Ursa
Minor's correction, see Section~\ref{sec:orbit_corrections}) in
Fig.~\ref{fig:tidal_tracks_concentration} as the ``heavier, new orbit''
halo. Since correcting the orbit has a relatively small effect on the
tidal evolution, we neglect these corrections in further comparisons for
simplicity.

\begin{figure}
\centering
\pandocbounded{\includegraphics[keepaspectratio]{figures/scl_mw_halo_boundmass.pdf}}
\caption[Tidal dependence on halo concentration]{A comparison of the
evolution of different N-body models for different halo concentrations
(heavier and lighter halo), and the heavy halo with the action-angle
corrected orbit. \textbf{Left:} The maximum circular velocity \(\vmax\)
of the dark matter halo is plotted as a function of simulation time.
\textbf{Right:} \(\vmax\) is instead plotted as a function of
\(\rmax\).}\label{fig:tidal_tracks_concentration}
\end{figure}

\subsection{Dark matter cores}\label{dark-matter-cores}

Many dwarf galaxies appear to have dark matter cores. In Scl, the
presence or absence of a core has been debated
\citep[e.g.,][]{battaglia+2008, walker+2009a, agnello+evans2012, breddels+helmi2013, amorisco+zavala+deboer2014, richardson+fairbairn2014}.
To simulate the evolution of a cored dark matter halo, we adopt a
``cored-NFW'' model, \begin{equation}{
\rho/\rho   _s = \frac{1}{(1+r/r_s)^2 (r_c/r_s + r/r_s)},
}\end{equation} where \(r_c\) is the core radius. We set
\(M_s=0.54\times10^{10}\,\Mo\) where \(M_s = 4\pi/3\ r_s^3 \rho_s\),
\(r_s=1.08\,\kpc\), \(r_c=r_s\), and use the same truncation as our
fiducial halo (Eq.~\ref{eq:trunc_nfw}). We plot the initial and final
density, as compared to a cuspy NFW, in Fig.~\ref{fig:cored_i_f}.

In Fig.~\ref{fig:tidal_tracks_structure}, we show the evolution of the
cored model as compared to a heavier halo. The cored halo appears to
evolve more mildly than the NFW halo. This is possibly because, to match
the velocity dispersion, the cored halo has \(\sim 50\%\) more mass
within \(1\,\kpc\) than a similar cuspy halo.

\begin{figure}
\centering
\pandocbounded{\includegraphics[keepaspectratio]{figures/cored_density_i_f.pdf}}
\caption[Tidal evolution of a cored density profile]{The initial (dotted
orange) and final (solid orange) 3D density profiles for the cored model
of Scl on the \smallperi{} orbit. The blue thin line represents an NFW
halo with the same \(M_s\) and \(r_s\).}\label{fig:cored_i_f}
\end{figure}

\begin{figure}
\centering
\pandocbounded{\includegraphics[keepaspectratio]{figures/scl_mw_structure_boundmass.pdf}}
\caption[Tidal tracks depending on halo substructure]{Similar to
Fig.~\ref{fig:tidal_tracks_concentration}, except testing the effects of
including a core, velocity anisotropy, and evolving an oblate halo.
While the normalization may differ, the tidal evolution is
similar.}\label{fig:tidal_tracks_structure}
\end{figure}

\subsection{Velocity anisotropy}\label{velocity-anisotropy}

Radial velocity anisotropy may cause halos to disrupt faster
\citep[e.g.][]{chiang+bosch+schive2024}. To test the effects of moderate
velocity anisotropy, we initialize a model with a velocity anisotropy
with an Osipkov-Merritt profile rising from \(\beta=0.2\) at the centre
to \(\beta=1\) and infinity, with scale length \(4r_s\).

Fig.~\ref{fig:anisotropy_i_f} shows the initial and final anisotropy
profiles after tidal evolution on Scl's \smallperi{} orbit. The dwarf
galaxy becomes more isotropic with tidal evolution. Particles on more
radially anisotropic orbits are more easily stripped as they have larger
apocentres than more circular orbits of the equivalent energy.
Regardless, the overall tidal evolution is very similar to our fiducial,
isotropic case (see Fig.~\ref{fig:tidal_tracks_structure}).

\begin{figure}
\centering
\pandocbounded{\includegraphics[keepaspectratio]{figures/anisotropy_i_f.pdf}}
\caption[Tidal evolution of anisotropy]{The initial and final anisotropy
profiles for the initially anisotropic model of Scl on the \smallperi{}
orbit. The final profile is more isotropic (closer to \(\beta=0\)) than
the initial.}\label{fig:anisotropy_i_f}
\end{figure}

\subsection{An ellipsoidal halo}\label{an-ellipsoidal-halo}

Sculptor and Ursa Minor may be highly elliptical in 3D, possibly
violating the assumption of spherical symmetry
\citep[e.g.,][]{an+koposov2022}. Although, the shape of the underlying
dark matter halo is unknown.

We create an oblate initial dark matter halo using \agama{}.
Fig.~\ref{fig:oblate_i_f} shows the initial equilibrium (after 5Gyr in
isolation) isodensity contours of our model. The initial snapshot is
sampled from the distribution function \begin{equation}{
\begin{split}
f = \frac{M_0}{(2\pi\,J_0)^3} \left[1 + \left(\frac{J_0}{h(J)}\right)^\eta\right]^{\Gamma / \eta} \ \left[1 + \left(\frac{g(J)}{J_0}\right)^\eta\right]^{-B/\eta} \\ \ \exp\left[-\left(\frac{g(J)}{J_{\rm cutoff}} \right)^\zeta\right] 
\end{split},
}\end{equation} (\texttt{doublePowerLaw} in \agama{}), with
\(g(J) = g_r J_r + g_z J_z + (3-g_r - g_z) |J_\phi|\) and
\(h(J) = h_rJ_r + h_zJ_z + (3-h_r-h_z) |J_\phi|\). The
\texttt{example\_doublepowerlaw.py} script in \agama{} solves for the
best parameters matching a given density profile. We chose to create a
model resembling an NFW but scaled by a factor of 0.5 in the \(z\)-axis.
For a scale-free halo (\(r_s=1\), \(M_s=1\)), the best-fit parameters
are \(J_0=0.890\), \(\Gamma=1.46\), \(\eta=0.568\), \(B=2.97\),
\(h_r=0.845\), \(h_z=1.66\), \(g_r=0.753\), \(g_z=1.69\), \(M_0=0.965\),
\(J_{\rm cutoff}=2.40\), \(\zeta=20\). The resulting initial conditions
are stable in isolation and produce a halo with near the desired density
profile and ellipticity. We scale the halo to have a major axis density
profile equivalent to an NFW with \(\rmax = 7\,\kpc\) and
\(\vmax=48\,\kpc\).

We find the evolution of the oblate halo to be nearly identical to the
spherical halo (see Fig.~\ref{fig:tidal_tracks_structure}), similar to
\citet{battaglia+sollima+nipoti2015}. The oblate halo becomes spherical
after tidal evolution (Fig.~\ref{fig:oblate_i_f}). If Scl's dark matter
halo is indeed elliptical today, this model may not be an adequate
description.

\begin{figure}
\centering
\pandocbounded{\includegraphics[keepaspectratio]{figures/oblate_projected_2d.pdf}}
\caption[Oblate halo projected density snapshots]{The initial and final
(after 9Gyr of tidal evolution) projected density profiles for the
oblate halo on Scl's \smallperi{} orbit, projected on with \(x'\) and
\(z'\) the major and minor axes. The contours are drawn assuming normal
smoothing of 0.4 kpc and are log-spaced with intervals of 0.1
dex.}\label{fig:oblate_i_f}
\end{figure}

\subsection{Orbital variation}\label{orbital-variation}

As discussed in Section~\ref{sec:scl_umi_orbit_uncert}, the long-term
orbit of Scl is uncertain. For a more massive LMC (e.g., the L3M11
model), Scl may have undergone an extreme pericentric passage with the
MW \(\sim 6\,\Gyr\) ago. We find that the 3\(\sigma\) smallest
pericentre is \(4\,\kpc\), and simulate this model to test if such a
pericentre may be sufficient.

Because of the strong tidal interaction with the MW, the trajectory is
substantially perturbed from a point orbit. We adjust the initial
conditions by comparing the change in actions (as calculated in the
static MW-only potential) before and after the pericentre. Our final
model, the \texttt{MW\ impact} model, can approximately reproduce the
observed position of Scl (see Fig.~\ref{fig:scl_mw_impact_orbit}). The
model has a Galactocentric initial position of
\([67.83, -352.2, 110.3]\,\kpc\) and velocity of
\([-3.68, 30.79, -22.77]\,\kpc\).

Fig.~\ref{fig:tidal_tracks_umi} compares the tidal evolution of Scl on
the mean, \texttt{MW\ impact}, and \smallperi{} orbit. The mean orbit
loses less mass than the \smallperi{} model. Instead, the
\texttt{MW\ impact} orbit experiences most tidal evolution during its
first MW pericentre. While evolving further along the tidal track, the
stars of this model nevertheless remain exponential
(Fig.~\ref{fig:scl_mw_impact_i_f}). We suggest that the impulsive
pericentric passage does not occur for long enough in this model to
produce the expected extended density profile. A yet more extreme
orbital history would be necessary to tidally transform Scl's stars.

\begin{figure}
\centering
\pandocbounded{\includegraphics[keepaspectratio]{figures/scl_mw_impact_orbits.pdf}}
\caption[Sculptor MW impact orbit]{The orbit of Sculptor for the
MW-impact model. The point (dotted) and n-body (solid) diverge by
\(\sim 50\,\kpc\) at early times.}\label{fig:scl_mw_impact_orbit}
\end{figure}

\begin{figure}
\centering
\pandocbounded{\includegraphics[keepaspectratio]{figures/scl_orbits_boundmass.pdf}}
\caption[Sculptor's tidal evolution for different orbits]{Similar to
Fig.~\ref{fig:tidal_tracks_concentration}, except testing the effects of
orbits in the LMC. Most LMC models evolve more weakly than the MW
models.}\label{fig:tidal_tracks_umi}
\end{figure}

\begin{figure}
\centering
\pandocbounded{\includegraphics[keepaspectratio]{figures/scl_impact_i_f.pdf}}
\caption[Sculptor MW-impact density profiles]{Similar to
Fig.~\ref{fig:scl_smallperi_i_f} except for the orbit of Scl passing
through the MW.}\label{fig:scl_mw_impact_i_f}
\end{figure}

\section{The formation of tidal
tails}\label{the-formation-of-tidal-tails}

As discussed in Section~\ref{sec:results}, any hints of a possible tidal
stream around Scl and UMi are beyond the reach of current observational
facilities. However, we can still predict the properties of such a
stream.

Fig.~\ref{fig:scl_tidal_stream} and Fig.~\ref{fig:umi_tidal_stream} show
the resulting distributions of velocities and distance along the stream
orbital axis \(\xi'\) for Scl and UMi. The tidal tails may have
detectible gradients in radial velocities (\(\sim 10\,\kms\)) and proper
motions (mostly for UMi, of \(\sim 0.1\,\masyr\)). However, detecting
such a gradient would require tracing stars across several degrees on
the sky.

\begin{figure}
\centering
\pandocbounded{\includegraphics[keepaspectratio]{figures/scl_sim_stream.pdf}}
\caption[Sculptor predicted stream]{The predicted properties of a tidal
tail in the Scl model. The panels are all as a function of \(\xi'\), the
distance along the stream as defined by the current GSR proper motion
vector. The top panels show the GSR proper motions in RA and Dec, and
the bottom two show the distance and GSR radial velocities. To sample
the stream, we randomly draw 100,000 samples from the snapshot based on
the stellar weights. A detectible gradient in \(\mu_{\alpha*}\) and LOS
velocity should be detectible if the stream is tracked across several
degrees.}\label{fig:scl_tidal_stream}
\end{figure}

\begin{figure}
\centering
\pandocbounded{\includegraphics[keepaspectratio]{/Users/daniel/thesis/figures/umi_sim_stream.pdf}}
\caption[Ursa Minor predicted stream]{The properties of the stream
around the UMi \smallperi{} orbit with Plummer
stars.}\label{fig:umi_tidal_stream}
\end{figure}

\section{Summary}\label{summary}

While we have compared the evolution of orbits in the MW potential for
simplicity, because the tidal evolution of Scl is very weak in the LMC
model, the differences between models become even more slight. And while
UMi has a lower pericentre than Scl, the differences between halo
structure should apply similarly to UMi.
