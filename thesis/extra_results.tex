\chapter{Additional simulation tests}\label{sec:extra_results}

In this section, we briefly explore additional simulations testing
assumptions not varied in the main text: variations in the halo
concentration, cored halos, alternative orbits, anisotropy, and
ellipsoidal models. While each of these variables influences resulting
dark matter evolution, the final stellar component and dark matter
remnant remains largely across alternate initial conditions. We conclude
that consideration of these effects likely would not change our
conclusions.

The following models aim to reproduce the velocity dispersion
\(\lesssim 1\,\kms\) with a similar present-day half-light radius.

\section{Alternative initial
conditions}\label{alternative-initial-conditions}

\subsection{Halo concentration}\label{halo-concentration}

Changing the initial concentration (or \(\vmax\) and \(\rmax\))
primarily affects total dark matter evolution.
Fig.~\ref{fig:tidal_tracks_concentration} compares models on the
\smallperi{} orbit with different initial halo parameters. The heavier
halo has \(\vmax=43\) and \(\rmax=7\), whereas the lighter halo has
\(\vmax=25\) and \(\rmax=2.5\). While the less concentrated halo loses
more mass, the halo evolves to similar final structural parameters.

\begin{figure}
\centering
\pandocbounded{\includegraphics[keepaspectratio]{figures/scl_mw_halo_boundmass.pdf}}
\caption[Tidal dependence on halo concentration]{A comparison of the
evolution of different N-body models for different halo concentrations
(heavier and lighter halo), and the mean orbit instead of the
\smallperi{} orbit. In the left, the maximum circular velocity \(\vmax\)
of the dark matter halo is plotted as a function of simulation time. In
the right, \(\vmax\) is instead plotted as a function of
\(\rmax\).}\label{fig:tidal_tracks_concentration}
\end{figure}

\pandocbounded{\includegraphics[keepaspectratio]{figures/anisotropy_i_f.pdf}}

Figure: The evolution of velocity anisotropy.

\subsection{Dark matter cores}\label{dark-matter-cores}

Many dwarf galaxies appear to have dark matter cores. In Scl, the
presence or absence of a core has been debated
\citep[e.g.,][]{battaglia+2008, walker+2009a, agnello+evans2012, breddels+helmi2013, amorisco+zavala+deboer2014, richardson+fairbairn2014}.
To simulate the evolution of a cored dark matter halo, we adopt a
``cored-NFW'' model, \begin{equation}{
\rho/\rho   _s = \frac{1}{(1+r/r_s)^2 (r_c/r_s + r/r_s)},
}\end{equation} where \(r_c\) is the core radius. We set
\(M_s=0.79\times10^{10}\,\Mo\) where \(M_s = 4\pi/3\ r_s^3 \rho_s\),
\(r_s=2.76\,\kpc\), \(r_c=0.1r_s\), and use the same truncation as our
fiducial halo (Eq.~\ref{eq:trunc_nfw}). While this is not the largest
possible core, the core is similar in size to the stellar component.
Larger cores would require a more massive halo in order to match the
velocity dispersion at a constant scale radius.

In Fig.~\ref{fig:tidal_tracks_structure}, we show the evolution of the
cored model as compared to a heavier halo. The cored halo loses more
total dark matter mass, evolving further along the tidal track as
expected. However, in order to reproduce the observed velocity
dispersion, the final halo reaches a \(\rmax\) and \(\vmax\) as the
fiducial halo. A cored dark matter halo likely cannot be completely
tidally disrupted and still recover the observed velocity dispersion.

\pandocbounded{\includegraphics[keepaspectratio]{figures/cored_density_i_f.pdf}}

Figure: Cored versus cuspy initial and final density (todo, fix)..

\subsection{An ellipsoidal halo}\label{an-ellipsoidal-halo}

Sculptor and Ursa Minor may be highly elliptical in 3D, possibly
violating our assumption of spherical symmetry
\citep[e.g.,][]{an+koposov2022}.

We create an oblate initial dark matter halo using \agama{}. The initial
snapshot is sampled from the distribution function \begin{equation}{
\begin{split}
f = \frac{M_0}{(2\pi\,J_0)^3} \left[1 + \left(\frac{J_0}{h(J)}\right)^\eta\right]^{\Gamma / \eta} \ \left[1 + \left(\frac{g(J)}{J_0}\right)^\eta\right]^{-B/\eta} \\ \ \exp\left[-\left(\frac{g(J)}{J_{\rm cutoff}} \right)^\zeta\right] 
\end{split},
}\end{equation} (\texttt{doublePowerLaw} in \agama{}), with
\(g(J) = g_r J_r + g_z J_z + (3-g_r - g_z) |J_\phi|\) and
\(h(J) = h_rJ_r + h_zJ_z + (3-h_r-h_z) |J_\phi|\). The
\texttt{example\_doublepowerlaw.py} script in \Agama{} solves for the
best parameters matching a given density profile. We chose to create a
model resembling an NFW but scaled by a factor 0.5 in the \(z\) axis.
For a scale-free halo (\(r_s=1\), \(M_s=1\)), the best-fit parameters
are \(J_0=0.890\), \(\Gamma=1.46\), \(\eta=0.568\), \(B=2.97\),
\(h_r=0.845\), \(h_z=1.66\), \(g_r=0.753\), \(g_z=1.69\), \(M_0=0.965\),
\(J_{\rm cutoff}=2.40\), \(\zeta=20\). The resulting initial conditions
are stable in isolation and produce a halo with near the desired density
profile and ellipticity. We scale the halo to have a major axis density
profile equivalent to an NFW with \(\rmax = 7\,\kpc\) and
\(\vmax=48\,\kpc\).

We find the evolution of the oblate halo to be nearly identical to the
spherical halo, similar to \citet{battaglia+sollima+nipoti2015}. The
oblate halo however becomes spherical as time progresses. If Scl's dark
matter halo is indeed elliptical, this model may not be an adequate
description. Possibly, internal rotation, alternative density structure,
or velocity anisotropy play a role in the stability of oblate initial
conditions to tidal stripping. A full investigation of these effects is
left to future work.

\pandocbounded{\includegraphics[keepaspectratio]{figures/oblate_density_i_f.pdf}}

Slices of the oblate density before and after tidal evolution.
\textbf{TODO: isodensity contours on RHS projected major/minor axis}.

\begin{figure}
\centering
\pandocbounded{\includegraphics[keepaspectratio]{figures/scl_mw_structure_boundmass.pdf}}
\caption[Tidal dependence on halo structure]{Similar to
Fig.~\ref{fig:tidal_tracks_concentration}, except testing the effects of
including a core, velocity anisotropy, and evolving an oblate halo.
While the normalization may differ, the tidal evolution is
similar.}\label{fig:tidal_tracks_structure}
\end{figure}

\subsection{Orbital variation}\label{orbital-variation}

In the main text, we focus on recent tidal impacts on Scl and UMi. But,
due to the long-term orbital uncertainties of Scl and UMi with an LMC
(see Fig.~\ref{fig:scl_umi_orbit_uncert}), tidally extreme orbits are
entirely possible. Here, we consider a few examples of extreme orbital
evolution in the long term case. While the actual orbit of either galaxy
almost certainly deviates from these examples (discussion in
Section~\ref{sec:orbital_uncertainties}), these orbits still represent
extreme past tidal histories.

For a more massive LMC (e.g., the L3M11 model), Scl may have undergone
an extreme pericentric passage with the MW \(\sim 6\,\Gyr\) ago. We find
that the 3\(\sigma\) smallest pericentre is \(4\,\kpc\). To test if this
model forms a different density profile, we simulate this model.

Because of the strong tidal interaction with the MW, the trajectory is
substantially perturbed from a point orbit. We adjust the initial
conditions by comparing the change in actions (as calculated in the
static MW-only potential) before and after the pericentre.
Table~\ref{tbl:orbit_adjustments} records the iterations we take. Our
final model is able to approximately reproduce the observed position of
Scl, but more refinement is required for a closer match.

\begin{figure}
\centering
\pandocbounded{\includegraphics[keepaspectratio]{figures/scl_orbits_boundmass.pdf}}
\caption[Ursa Minor tidal dependence on orbit]{Similar to
Fig.~\ref{fig:tidal_tracks_concentration}, except testing the effects of
orbits in the LMC. Most LMC models evolve more weakly than the MW
models, however these do not reach full agreement with the present-day
position so misrepresent recent tidal
evolution.}\label{fig:tidal_tracks_umi}
\end{figure}

\section{Present-day properties}\label{present-day-properties}

We find that this model also does not display strong signs of tidal
signatures. Fig.~\ref{fig:scl_mw_impact_i_f} shows the initial and final
stellar profiles for this model (integrated with a lower-resolution
\(10^6\) particles). The model evolves little more than the
\smallperi{}. Given that the Jacobi radius would be within \(R_h\), why
does this model not show strong density perturbations?

We propose that th case Scl undergoes a small MW pericentre is an
impulsive encounter. Because Scl's apocentre is still \(\sim 300\,\kpc\)
in this model, the galaxy passes through the MW with velocity
\(600\,\kms?\). Since the crossing time at the half-light radius is
about 90 Myr, in one crossing-time, the model moves from pericentre to a
Galactocentric distance of \(\sim20\,\kpc\), where tides forced have
dropped substantially. As a result, after \(6\Gyr\) of relatively
isolated evolution, the break radius falls well outside the observed
density profile, but the injection of energy uniformly throughout the
halo only results in moderate DM loss predominantly in the outskirts,
not a reshaping of the inner density profile. Note that the Jacobi
radius is derived on circular orbits, not highly elliptical orbits like
this case.

However, we only note that the stars appear to be consistent but are
well below the \citet{power+2003}'s numerically converged radius.

\section{The formation of tidal
tails}\label{the-formation-of-tidal-tails}

As we predict that any tidal tails formed in Scl and UMi would be
extremely faint, we deferred discussion of the properties of any such
tails to this section. We note that in extreme cases, \(\sim 2\%\) of
stars may become unbound in either galaxy, forming tidal tails which may
be detectible with substantially better observations.

\begin{figure}
\centering
\pandocbounded{\includegraphics[keepaspectratio]{/Users/daniel/thesis/figures/umi_sim_stream.pdf}}
\caption[Ursa Minor predicted stream]{The properties of the stream
around the UMi \smallperi{} orbit with Plummer stars. The panels are all
as a function of \(\xi'\), the distance along the stream as defined by
the current GSR proper motion vector. The top panels show the GSR proper
motions in RA and Dec, and the bottom two show the distance and GSR
radial velocities. To sample the stream, we randomly draw 100,000
samples from the snapshot based on the stellar weights. A detectible
gradient in \(\mu_{\alpha*}\) and LOS velocity should be detectible if
the stream is tracked across several
degrees.}\label{fig:umi_tidal_stream}
\end{figure}

\begin{figure}
\centering
\pandocbounded{\includegraphics[keepaspectratio]{figures/scl_sim_stream.pdf}}
\caption[Scl predicted stream]{The predicted properties of a tidal tail
in the Scl model. \textbf{TODO: titles}}\label{fig:scl_tidal_stream}
\end{figure}

\section{Summary}\label{summary}

While we have compared the evolution of orbits in the MW potential for
simplicity, because the tidal evolution of Scl is very weak in the LMC
model, the differences between models become even more slight. And while
UMi has a lower pericentre than Scl, the differences between halo
structure should apply similarly to UMi.
