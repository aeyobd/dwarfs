\chapter{Additional results an tests}\label{sec:extra_results}

In this section, we briefly explore additional simulations testing
assumptions not varied in the main text: variations in the halo
concentration, cored halos, alternative orbits, anisotropy, and
oblateness.

\section{Alternative initial
conditions}\label{alternative-initial-conditions}

The initial concentration (i.e.~choice of \(\vmax\) and \(\rmax\))
affects total dark matter evolution.
Fig.~\ref{fig:tidal_tracks_concentration} compares models on the
\smallperi{} orbit with different initial halo parameters. The heavier
halo has \(\vmax=43\) and \(\rmax=7\), whereas the lighter halo has
\(\vmax=25\) and \(\rmax=2.5\). While the less concentrated halo loses
more mass, the halo evolves to similar final structural parameters, as
expected in \citet{EN2021}. We note these choices were constrained to
match Scl's present-day velocity dispersion.

\begin{figure}
\centering
\pandocbounded{\includegraphics[keepaspectratio]{figures/scl_mw_halo_boundmass.pdf}}
\caption[Tidal dependence on halo concentration]{A comparison of the
evolution of different N-body models for different halo concentrations
(heavier and lighter halo), and the mean orbit instead of the
\smallperi{} orbit. In the left, the maximum circular velocity \(\vmax\)
of the dark matter halo is plotted as a function of simulation time. In
the right, \(\vmax\) is instead plotted as a function of
\(\rmax\).}\label{fig:tidal_tracks_concentration}
\end{figure}

We have good observational evidence for some dwarf galaxies to be cored.
Even Sculptor has debated claims for having a core
\citep[e.g.,][]{amorisco+zavala+deboer2014, breddels+helmi2013, battaglia+2008, walker+2009a, richardson+fairbairn2014, agnello+evans2012}.
To test if a core matter, we adopt a ``cored-NFW'' model,
\begin{equation}{
\rho/\rho   _s = \frac{1}{(1+r/r_s)^2 (r_c/r_s + r/r_s)},
}\end{equation} where \(r_c\) is the core radius. We adopt
\(r_c = 0.1r_s\) and adopt an extreme initial mass of \(\vmax=49\,\kms\)
and \(\rmax=7.1\,\kpc\).

Sculptor and Ursa Minor may also be highly elliptical in 3D
\citep[e.g.,][]{an+koposov2022}, especially given their projected
shapes. We consider

We find the evolution of the oblate halo to be nearly identical to the
spherical halo, similar to \citet{battaglia+sollima+nipoti2015}.

While we have compared the evolution of orbits in the MW potential for
simplicity, because the tidal evolution of Scl is very weak in the LMC
model, the differences between models become even more slight.

\begin{figure}
\centering
\pandocbounded{\includegraphics[keepaspectratio]{figures/scl_mw_structure_boundmass.pdf}}
\caption[Tidal dependence on halo structure]{Similar to
Fig.~\ref{fig:tidal_tracks_concentration}, except testing the effects of
including a core, velocity anisotropy, and evolving an oblate halo.
While the normalization may differ, the tidal evolution is
similar.}\label{fig:tidal_tracks_structure}
\end{figure}

\begin{figure}
\centering
\pandocbounded{\includegraphics[keepaspectratio]{figures/umi_orbits_boundmass.pdf}}
\caption[Ursa Minor tidal dependence on orbit]{Similar to
Fig.~\ref{fig:tidal_tracks_concentration}, except testing the effects of
orbits in the LMC. Most LMC models evolve more weakly than the MW
models, however these do not reach full agreement with the present-day
position so misrepresent recent tidal
evolution.}\label{fig:tidal_tracks_umi}
\end{figure}

\begin{figure}
\centering
\pandocbounded{\includegraphics[keepaspectratio]{figures/scl_impact_i_f.pdf}}
\caption[Scl MW impact stellar densities]{The initial and final density
profiles for the Sculptor model which passes through the MW previously.
The impact does not substantially affect the inner density region. The
hints of a tidal stream in this model are misaligned to the proper
motion, likely a result of the LMC.}
\end{figure}
