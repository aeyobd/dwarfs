\chapter{Radial velocity modeling}\label{sec:rv_obs}

\section{Data selection}\label{data-selection}

For both Sculptor and Ursa Minor, we construct literature samples of
radial velocity measurements. We combine these samples with J+24's
members to produce RV consistent stars and to compute velocity
dispersion, systematic velocities, and test for the appearance of
velocity gradients.

First, we crossmatch all catalogues to J+24 Gaia stars. If a study did
not report GaiaDR3 source ID's, we match to the nearest star within 1-3
arcseconds (see REF Table~\ref{tbl:rv_measurements}). We combine the
mean RV measurement from each study using the inverse-variance weighted
mean \(\bar v\), standard uncertainty \(\delta \bar v\), and (biased)
variance \(s^2\). We remove stars with significant velocity dispersions
as measured between observations in a study or between studies. By using
that \(\chi^2=\frac{s^2}{\delta \bar v^2}\), we remove stars with a
\(\chi^2\) larger than the 99.9th percentile of the \(\chi^2\)
distribution with \(N-1\) measurements. This cut typically removes stars
with reduced chi-squared values
\(\tilde\chi^2  = \frac{s^2}{\nu\,\delta \bar v^2}\gtrsim 7\) (since the
number of measurements is 1-3 typically).

Next, we need to correct the coordinate frames for the solar motion and
on-sky size of the galaxy. We transform the frame into the galactic
standard of rest (GSR). The next step is to account for the slight
differences in the direction of each radial velocity. Let the \(\hat z\)
be the direction from the sun to the dwarf galaxy. Then if \(\phi\) is
the angular distance between the centre of the galaxy and the individual
star, the corrected radial velocity is then \begin{equation}{
v_z = v_{\rm los, gsr}\cos\phi  - v_{\alpha}\cos\theta \sin\phi - v_\delta \sin\theta\sin\phi
}\end{equation} where \(v_{\rm los, gsr}\) is the line of sight velocity
in the GSR frame, \(v_\alpha\) and \(v_\delta\) are the tangental
velocities in RA and Dec, and \(\theta\) is the position angle of the
star with respect to the centre of the dwarf. The correction from both
effects induces an apparent gradient of about \(1.3\,\kmsdeg\) for
Sculptor and less for Ursa Minor \citep[see
also][]{WMO2008, strigari2010}. We add the uncertainty in \(v_z\) from
the distance uncertainty and velocity dispersion in quadrature to the RV
uncertainties for each star. We then use the \(v_z\) values for the
following modelling, however repeating with uncorrected, heliocentric
velocities does not significantly affect the results.

The combined likelihood, including RV information, becomes
\begin{equation}{
{\cal L} = {\cal L}_{\rm space} {\cal L}_{\rm CMD} {\cal L}_{\rm PM} {\cal L}_{\rm RV}
}\end{equation} where we assume that the satellite and background
distributions are Gaussian. Specifically, \begin{equation}{
\begin{split}
{\cal L}_{\rm RV, sat} &= f\left( \frac{v_i -\mu_{v}}{\sqrt{\sigma_{v}^2 + (\delta v_i)^2}}\right) \\
{\cal L}_{\rm RV, bg} &= f\left( v_i /  \sigma_{\rm halo} \right)
\end{split}
}\end{equation} where \(f\) is the probability density of a standard
normal distribution, \(\mu_v\) and \(\sigma_v\) are the systemic
velocity and dispersion of the satellite, and \(\delta v_i\) is the
individual measurement uncertainty. Typically, the velocity dispersion
will dominate the uncertainty budget here. We assume a halo/background
velocity dispersion of a constant \(\sigma_{\rm halo} = 100\,\kms\)
\citep[e.g.][]{brown+2010}.

Similar to above, we retain stars with the resulting membership
probability of greater than 0.2. Because of the additional information
from radial velocities, most stars have probabilities close to 1 or 0 so
the probability cut is not too significant.

We assume priors on systematic velocity and velocity dispersion of
\begin{equation}{
\begin{split}
\mu_{v} &= N(0\,\kms, \sigma_{\rm halo}^2) \\ 
\sigma_{v} &= U(0, 20\,\kms)
\end{split}
}\end{equation} where \(\sigma_{\rm halo} = 100\,{\rm km\,s^{-1}}\) is
the velocity dispersion of the MW halo adopted above, a reasonable
assumption for dwarfs in orbit around the MW.

\section{Results}\label{sec:rv_results}

\begin{figure}
\centering
\pandocbounded{\includegraphics[keepaspectratio]{figures/scl_umi_rv_fits.pdf}}
\caption[LOS velocity fit to Scl.]{Velocity histogram of Scl and UMi in
terms of \(v_z\) (REF). Orange points are from our crossmatched RV
membership sample.}
\end{figure}

For Sculptor, we combine radial velocity measurements from APOGEE,
\citet{sestito+2023a}, \citet{tolstoy+2023}, and \citet{WMO2009}.
\citet{tolstoy+2023} and \citet{WMO2009} provide the bulk of the
measurements. We find that there is no significant velocity shift in
crossmatched stars between catalogues. After crossmatching to high
quality Gaia stars and excluding significant stellar velocity
dispersions, we have a sample of 1918 members.

We derive a systemic velocity for Sculptor of \(111.3\pm0.2\,\kms\)with
velocity dispersion \(9.64\pm0.16\,\kms\). Our values are very
consistent with previous work \citep[e.g.][\citet{arroyo-polonio+2024},
\citet{battaglia+2008}]{walker+2009}. See appendix REF for a more
detailed comparison between different samples and additional tests.

We detect a moderately significant gradient of \(4.3\pm1.3\,\kmsdeg\) at
a position angle of \(-149_{-13}^{+17}\) degrees (see appendix REF).
Several past work has attempted to detect a gradient in Sculptor, but no
consensus has been reached. \citet{arroyo-polonio+2024} detect a
velocity gradient of \(4\pm1.5\,\kmsdeg\) in a similar direction using
the \citet{tolstoy+2023} sample, finding inconclusive statistical
evidence. They additionally suggest a third chemodynamical component of
the galaxy which may bias rotation measurements. \citet{battaglia+2008}
also detect a \(-7.6_{-2.2}^{+3.0}\,\kmsdeg\) velocity gradient along
the major axis, approximately the same direction. Instead,
\citet{strigari2010}; \citet{martinez-garcia+2023} detect no significant
gradient in Sculptor using \citet{WMO2009} sample. Note that
pre-\emph{Gaia} work did not have as strong of a constraint on the
proper motion of Scl, which limits conclusions of the intrinsic velocity
gradient in Scl.

For UMi, we collect radial velocities from, APOGEE,
\citet{sestito+2023b}, \citet{pace+2020}, and \citet{spencer+2018}. We
shifted the velocities of \citet{spencer+2018} (\(-1.1\,\kms\)) and
\citet{pace+2020} (\(+1.1\,\kms\) ) to reach the same scale. We found
183 crossmatched common stars (passing 3\(\sigma\) RV cut, velocity
dispersion cut, and PSAT J+24 \textgreater{} 0.2 w/o velocities). Since
the median difference in velocities in this crossmatch is about 2.2
km/s, we adopt 1 km/s as the approximate systematic error here. Our
final sample includes 831 members.

We derive a mean \(-245.8\pm0.3_{\rm stat}\,\kms\) and velocity
dispersion of \(8.8\pm0.2\,\kms\) for UMi. This is consistent with
\citet{pace+2020} and to a lesser extent with \citet{spencer+2018}. We
do not find evidence for a velocity gradient, consistent with past work
\citep{pace+2020, martinez-garcia+2023}.

\section{Discussion and limitations}\label{discussion-and-limitations}

Our model here is relatively simple. Some things which we note as
systematics or limitations:

\begin{itemize}
\tightlist
\item
  Inter-study systematics and biases. While basic crossmatches and a
  simple velocity shift, combining data from multiple instruments is
  challenging. This appears to be a minor issue (Sculptor) or is
  corrected for (Ursa Minor).
\item
  Misrepresentative uncertainties. Inspection of the variances compared
  to the standard deviations within a study seems to imply that errors
  are accurately reported. APOGEE notes that their RV uncertainties are
  known to be underestimates but are a small proportion of our sample.
\item
  Binarity. While not too large of a change for classical dwarfs, this
  could inflate velocity dispersions of about \(9\,\kms\) by about
  \(1\,\kms\)\citep{spencer+2017}. Thus, our measurement is likely
  slightly inflated given the high binarity fractions measured in these
  systems \citep[\citet{spencer+2018}]{arroyo-polonio+2023}.
\item
  Multiple populations. Both Sculptor and Ursa Minor likely contain
  multiple populations \citep[\citet{pace+2020},
  \citet{tolstoy+2004}]{arroyo-polonio+2024}. Since we only model a
  single population, and each population may have a different extent and
  velocity dispersion, this could result in biased velocity dispersions.
  However, it is unclear how to uniquely determine an overall velocity
  dispersion in a multi-population system.
\item
  Selection effects. RV studies each have their own selection effects,
  which may affect the resulting dispersion, especially if different
  populations or regions of the galaxy have different velocities or
  velocity dispersions. We do not attempt to correct for this.
\end{itemize}

For both Ursa Minor and Sculptor, we also fit models to only data from
individual surveys (see REF). Since the resulting parameters are very
similar, we conclude that many of the systematic uncertainties are
likely smaller than the present errors or that each large survey has
similar biases.

\section{Velocity modelling and
comparisons}\label{velocity-modelling-and-comparisons}

Here, we describe in additional detail, our methods and comparisons for
RV modelling between studies.

Savage-Dickey calculated Bayes factor using Silverman-bandwidth KDE
smoothed samples from posterior/prior.

\begin{table*}[t]
\centering
\caption[Spectroscopic LOS velocity measurements]{Summary of velocity measurements and derived properties. }
\label{tbl:rv_measurements}
\begin{tabular}{lllllllll}
\toprule
 & Study & Instrument & Nspec & Nstar & Ngood & Nmemb & $\delta v_{\rm med}$ & $R_{\rm xmatch}$/arcmin\\
\midrule
Scl & combined &  & 8945 & 2280 & 2096 & 1981 & 0.9 & \\
 & tolstoy+23 & FLAMES & 3311 & 1701 & 1522 & 1482 & 0.65 & –\\
 & sestito+23a & GMOS & 2 & 2 & 2 & 2 & 13 & –\\
 & walker+09 & MMFS & 1818 & 1522 & 1417 & 1328 & 1.8 & 3\\
 & APOGEE & APOGEE & 5082 & 253 & 170 & 164 & 0.5 & –\\
UMi & combined &  & 4714 & 1225 & 1148 & 863 & 2.1 & \\
 & sestito+23b & GRACES & 5 & 5 & 5 & 5 & 1.8 & –\\
 & pace+20 & DEIMOS & 1716 & 1538 & 829 & 678 & 2.5 & 1\\
 & spencer+18 & Hectoshell & 1407 & 970 & 596 & 406 & 0.9 & 2\\
 & APOGEE & APOGEE & 9500 & 279 & 37 & 67 & 0.6 & –\\
\bottomrule
\end{tabular}
\end{table*}

measurement

\begin{table*}[t]
\centering
\caption[Ursa Minor RV fits]{MCMC fits for UMi velocity dispersion. }
\label{tbl:umi_rv_mcmc}
\begin{tabular}{lllll}
\toprule
study & mean & sigma & $\log bf_{\rm sigma}$ & $\log bf_{\rm grad}$\\
\midrule
all & $-245.8\pm0.3$ & $8.8\pm0.2$ & +1.3 & +0.9\\
pace & $-244.6\pm0.4$ & $9.0\pm0.3$ & +0.3 & +0.5\\
spencer & $-246.9\pm0.4$ & $8.8\pm0.3$ & +1.8 & -0.06\\
apogee & $-245.6\pm1.2$ & $10.0_{-0.8}^{+1.0}$ & +1.0 & +0.5\\
\bottomrule
\end{tabular}
\end{table*}

\begin{figure}
\centering
\pandocbounded{\includegraphics[keepaspectratio]{figures/scl_rv_scatter.pdf}}
\caption[Scl velocity sample]{RV members of Sculptor plotted in the
tangent plane coloured by corrected velocity difference from mean
\(v_z - \bar v_z\) . The black ellipse marks the half-light radius in
Fig.~\ref{fig:scl_selection}. The black and green arrows mark the proper
motion (PM, GSR frame) and derived velocity gradient (rot) vectors (to
scale).}
\end{figure}

\begin{figure}
\centering
\pandocbounded{\includegraphics[keepaspectratio]{figures/scl_vel_gradient_scatter.pdf}}
\caption[Scl velocity gradient]{The corrected LOS velocity along the
best fit rotational axis. RV members are black points, the systematic
\(v_z\) is the horizontal grey line, blue lines represent the
(projected) gradient from MCMC samples, and the orange line is a rolling
median (with a window size of 50).}
\end{figure}
