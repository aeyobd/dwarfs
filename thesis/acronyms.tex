%%%%%%%%%%%%%%% Acronyms %%%%%%%%%%%%%%%%%%
\usepackage[acronym,automake,symbols,
nogroupskip,
accsupp,
counter=section,
stylemods={longextra}
]{glossaries-extra}
\makeglossaries


% Ordinary abbreviations
\newcommand{\kms}{{\rm km\,s^{-1}}}
\newcommand{\kmsdeg}{{\rm km\,s^{-1}\,deg^{-1}}}
\newcommand{\masyr}{{\rm mas^{-1}\,yr^{-1}}}
\newcommand{\Gyr}{{\rm Gyr}}
\newcommand{\Msun}{{\rm M_\odot}}
\newcommand{\V}{{\rm v}}
\newcommand{\LCDM}{$\Lambda$CDM}
\newcommand{\agama}{{\sc Agama}}

\setabbreviationstyle[acronym]{long-short-sc}
\GlsXtrEnablePreLocationTag{\S }{\S }





\newacronym
[description={Initial mass function. A function describing the mass distribution of newly formed stars. I use a \citet{kroupa01} {\sc imf}, which is described as a piecewise power-law function of $M$.}]
{imf}{imf}{initial mass function}

\newacronym
[description={
Apache Point Observatory Galactic Evolution Experiment. 
A large near-infrared spectroscopic survey of stars in the Milky Way. \citep{apogee17.}}]
{apogee}{apogee}{Apache Point Observatory Galactic Evolution Experiment}





%%%%%%%%%%%%%%%%%%%%%% terms %%%%%%%%%%%%%%%
\newglossaryentry{metallicity}{name={metallicity},
    description={the (mass) fraction of a star or gas which is not made of either H or He. For the sun, the metallicity is $\Zo = 0.014$}
}

\newglossaryentry{yield}{name={yield},
    description={The net production of a new element during a star's lifecycle divided by the star's mass (including winds and supernovae ejecta). }
}

\newglossaryentry{nucleosynthesis}{name={nucleosynthesis},
    description={The synthesis of new elements through fusion inside stars. See section \ref{sec:stel_evo}.}
}


