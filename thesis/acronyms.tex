%%%%%%%%%%%%%%% Acronyms %%%%%%%%%%%%%%%%%%
\usepackage[acronym,automake,symbols,
nogroupskip,
accsupp,
counter=section,
stylemods={longextra}
]{glossaries-extra}
\makeglossaries

\setabbreviationstyle[acronym]{long-short-sc}
\GlsXtrEnablePreLocationTag{\S }{\S }


\newacronym
[ description ={Core collapse supernovae. Massive star explosions. {\sc Ccsne} produce many elements including \gls{alpha}, \gls{fepeak}, and r and \gls{sproc}.}
]
{cc}{ccsne}{core collapse supernovae}

\newacronym
[description={Asymptotic giant branch stars. The {\sc agb} phase is the last phase of \glspl{lowmass}, before stars become white dwarfs. Produces C, N, and heavy \gls{sproc} elements. See \ref{sec:stel_evo}.}.]
{agb}{agb}{asymptotic giant branch}

\newacronym
[description={Type Ia Supernovae. Exploding white dwarfs. Produces \gls{fepeak} and has a long delay time.}]
{ia}{sne ia}{supernovae type Ia}


\newacronym
[description={Galactic chemical evolution.}]
{gce}{gce}{galactic chemical evolution}

\newacronym
[description={Star formation history.}]
{sfh}{sfh}{star formation history}

\newacronym
[description={Single stellar population. A group of stars formed all at the same time.}]
{ssp}{ssp}{single stellar population}

\newacronym
[description={Initial mass function. A function describing the mass distribution of newly formed stars. I use a \citet{kroupa01} {\sc imf}, which is described as a piecewise power-law function of $M$.}]
{imf}{imf}{initial mass function}

\newacronym
[description={
Apache Point Observatory Galactic Evolution Experiment. 
A large near-infrared spectroscopic survey of stars in the Milky Way. \citep{apogee17.}}]
{apogee}{apogee}{Apache Point Observatory Galactic Evolution Experiment}

\newacronym
[
description={Damped Lyman-alpha system. {\sc dla}s are clouds of gas from the 
early universe which are observed through their absorption of quasar spectra. The name 
comes from the strong Lyman-alpha lines ($1216\AA$) due to H absorption.}
]
{dla}{dla}{damped Lyman-alpha systems}





\newcommand{\cc}{\gls{cc}}
\newcommand{\Cc}{\Gls{cc}}
\newcommand{\agb}{\gls{agb}}
\newcommand{\ia}{\gls{ia}}
\newcommand{\sfh}{\gls{sfh}}
\newcommand{\dla}{\gls{dla}}
\newcommand{\ssp}{\gls{ssp}}
\newcommand{\imf}{\gls{imf}}
\newcommand{\gce}{\gls{gce}}
\newcommand{\Gce}{\Gls{gce}}
\newcommand{\apogee}{\gls{apogee}}




%%%%%%%%%%%%%%%%%%%%%% terms %%%%%%%%%%%%%%%
\newglossaryentry{metallicity}{name={metallicity},
    description={the (mass) fraction of a star or gas which is not made of either H or He. For the sun, the metallicity is $\Zo = 0.014$}
}

\newglossaryentry{yield}{name={yield},
    description={The net production of a new element during a star's lifecycle divided by the star's mass (including winds and supernovae ejecta). }
}

\newglossaryentry{nucleosynthesis}{name={nucleosynthesis},
    description={The synthesis of new elements through fusion inside stars. See section \ref{sec:stel_evo}.}
}

\newglossaryentry{multizone}{name={multi-zone},
    description={A chemical evolution model where a galaxy is divided into several zones, each with different stars, gas, and properties. }
}

\newglossaryentry{onezone}{name={one-zone},
    description={A chemical evolution model where the gas is all the same composition, i.e. neglecting spatial variations.}
}

\newglossaryentry{alpha}{name={$\alpha$-element},
    description={Elements which are made up of $\alpha$-particles (He nuclei) throuch the triple-$\alpha$ process (see Eq.~\ref{eq:triple_alpha}). Essentially light, even-numbered elements like O, Mg, and Na.}
}

\newglossaryentry{fepeak}{name={Fe-peak elements},
    description={Fe and nearby elements, produced in \gls{cc} and \gls{ia}.}
}

\newglossaryentry{sproc}{name={s-process elements},
    description={Elements produced through slow neutron captures, typically in \agb\ stars.}
}

\newglossaryentry{rgb}{name={red giant branch},
    description={Red giant branch stars are stars that have completed hydrogen core burning and have expanded in size. See Appendix \ref{sec:stel_evo}.}
}

\newglossaryentry{tdu}{name={third dredge up},
    description={Third dredge up occurs inside \gls{agb} stars. During each thermal pulse, material is {\it dredged up} from the core, changing the chemical abundances of the stellar atmosphere. (While nominally called {\it third dredge up}, there are typically several third dredge ups.). See Appendix \ref{sec:stel_evo}.}
}

\newglossaryentry{fdu}{name={first dredge up},
    description={First dredge up occurs when a \gls{lowmass} enters the \gls{rgb} phase. Material from the core is brought to the surface, increasing N and decreasing C abundances. See Appendix \ref{sec:stel_evo}.}
}

\newglossaryentry{hbb}{name={hot bottom burning},
    description={Hot bottom burning occurs inside \agb\ stars. The base of the convective envelope becomes hot enough for CNO burning to initiate.}
}

\newglossaryentry{subgiant}{name={subgiant},
    description={A star in the process of leaving the main sequence and becoming a \gls{rgb}.}
}

\newglossaryentry{imfave}{name={\gls{imf}-averaged},
    description={Averaged over the initial-mass function ({\sc imf}). The {\imf}-weighted yield is the mass of the newly produced element divided by the mass of star formation for a single stellar population (see \ssp).}
}


\newglossaryentry{dtd}{name={delay time distribution},
    description={The distribution in time of when an element is produced 
    after a star formation event.}
}

\newglossaryentry{massloading}{name={mass loading},
    description={The strength of outflows relative to star formation. See also $\eta$. }
}

\newglossaryentry{lowalpha}{name={low-$\alpha$},
    description={The low-$\alpha$ sequence, as described by Eq.~\ref{eq:high_alpha}. }
}

\newglossaryentry{highalpha}{name={high-$\alpha$},
    description={The high-$\alpha$ sequence, as described by Eq.~\ref{eq:high_alpha}. }
}


\newglossaryentry{lowmass}{name={low-mass star},
    description={Stars with masses $\lesssim 8\,M_\odot$ which end life as a white dwarf. }
}

\newglossaryentry{highmass}{name={high-mass star},
    description={Stars with masses $\lesssim 8\,M_\odot$, which end as a neutron star, black hole, or supernovae.}
}

\newglossaryentry{insideout}{name={{\it insideout}},
    description={Our fiducial star formation history. The rate of star formation is highest towards the center of the galaxy and at earlier times. See Eq.~\ref{eq:inside_out}.}
}

\newglossaryentry{cno}{name={CNO cycle},
    description = { A proton-fusion cycle which occurs in \gls{rgb} stars 
        consisting of a chain of proton captures releasing a He nucleus ($\alpha$-particle) and energy. See Eq.~\ref{eq:cno_cycle}.
}
}


