\section{Abstract}\label{abstract}

To provide an empirical benchmark for simulations in the next chapters,
we aim to understand the observed properties of Sculptor and Ursa Minor.
In this section, we first compile past measurements of each galaxy. We
describe the \citet{jensen+2024} Bayesian methodology to select high
probability member stars from \emph{Gaia}. The derived density profiles
are robust to changes in assumptions. Fornax is well represented by a 2D
exponential profile. Instead, Sculptor and Ursa Minor show an excess of
stars relative to an 2D exponential in the outer regions, better
described my a Plummer profile. Finally, we combine various radial
velocity samples with \citet{jensen+2024}'s members to derive systemic
velocities and dispersions for Sculptor and Ursa Minor.

\section{Introduction}\label{introduction}

Since the discovery of dwarf galaxies around the Milky Way,
observational work has attempted to measure and refine the basic
properties of these objects. While the Milky Way's satellites are close
(by extragalactic standards), their low numbers of stars and large areas
on the sky present challenges for observational work. The classical
systems discussed here each extend about 1-2 degrees across the sky
(figs.~\ref{fig:scl_selection}, \ref{fig:umi_selection}, \ref{fig:fornax_selection}).

In particular, \emph{Gaia} has provided a wealth of high quality data,
including the distances to nearby stars and proper motions. This allows
for some of the first measurements on the motions for many local group
dwarf galaxies {[}e.g.; \citet{MV2020}{]}. Accurate distances and
velocities are invaluable for understanding the orbital history of
satellites.

High-quality large spectroscopic samples combined with \emph{Gaia} has
opened new windows into understanding the internal kinematics and
history of dwarf galaxies. \textless Discuss battaglia, pace, etc. and
conclusions about dwarf galaxies.\textgreater{}

In tbls.~\ref{tbl:scl_obs_props}, \ref{tbl:umi_obs_props} we present
adopted observed properties for each galaxy.

\begin{table*}[t]
\centering
\caption[Observed Properties of Sculptor]{Observed properties of Sculptor. References are: 1. Ricardo R. Muñoz et al. (2018), 2. Tran et al. (2022), 3. McConnachie and Venn (2020b), 4 McConnachie and Venn (2020a). }
\label{tbl:scl_obs_props}
\begin{tabular}{lll}
\toprule
parameter & value & Source\\
\midrule
$\alpha$ & $15.0183 \pm 0.0012^\circ$ & 1\\
$\delta$ & $-33.7186 \pm 0.0007^\circ$ & 1\\
distance modulus & $19.60 \pm 0.05$ (RR lyrae) & 2\\
distance & $83.2 \pm 2$ kpc & 2\\
$\mu_{\alpha*}$ & $0.099 \pm 0.002 \pm 0.017$ mas yr$^{-1}$ & 3\\
$\mu_\delta$ & $-0.160 \pm 0.002_{\rm stat} \pm 0.017_{\rm sys}$ mas yr$^{-1}$ & 3\\
radial velocity & $111.3 \pm 0.2\ {\rm km\,s^{-1}}$ & sec. \ref{sec:rv_results}\\
$\sigma_v$ & $9.64\pm0.16\ {\rm km\,s^{-1}}$ & sec. \ref{sec:rv_results}\\
$R_h$ & $9.79 \pm 0.04$ arcmin & 4\\
$R_{h,inner}$ & $9.46 \pm 0.26$ arcmin & XREF\\
ellipticity & $0.36 \pm 0.01$ & 1\\
position angle & $92\pm1^\circ$ & 1\\
$M_V$ & $-10.82\pm0.14$ & 1\\
\bottomrule
\end{tabular}
\end{table*}

\begin{table*}[t]
\centering
\caption[Observed Properties of Ursa Minor]{Observed properties of Ursa Minor. References are: (1) Ricardo R. Muñoz et al. (2018), (2) Garofalo et al. (2025), (3) McConnachie and Venn (2020a), (4) average of Pace et al. (2020) and Spencer et al. (2018). }
\label{tbl:umi_obs_props}
\begin{tabular}{lll}
\toprule
parameter & value & Source\\
\midrule
$\alpha$ & $ 227.2420 \pm 0.0045$˚ & 1\\
$\delta$ & $67.2221 \pm 0.0016$˚ & 1\\
distance modulus & $19.23 \pm 0.11$ (RR lyrae) & 2\\
distance & $70.1 \pm 3.6$ kpc & 2\\
$\mu_\alpha*$ & $-0.124 \pm 0.004 \pm 0.017$ mas yr$^{-1}$ & 3\\
$\mu_\delta$ & $0.078 \pm 0.004_{\rm stat} \pm 0.017_{\rm sys}$ mas yr$^{-1}$ & 3\\
radial velocity & $-245.8 \pm 0.3_{\rm stat} \pm 1_{\rm sys}$ km s$^{-1}$ & sec. \ref{sec:rv_results}\\
$\sigma_v$ & $8.8 \pm 0.2$ & sec. \ref{sec:rv_results}\\
$R_h$ & $11.62 \pm 0.1$ arcmin & 1\\
$R_{h, \rm inner}$ & $11.7\pm0.5$ & \\
ellipticity & $0.55 \pm 0.01$ & 1\\
position angle & $50 \pm 1^\circ$ & 1\\
$M_V$ & $-9.03 \pm 0.05$ & 1\\
\bottomrule
\end{tabular}
\end{table*}

\subsection{\texorpdfstring{The \emph{Gaia}
mission}{The Gaia mission}}\label{the-gaia-mission}

Some of the most fundamental properties of astronomical objects are
their position and velocity. Unfortunately, determining distances to
stars is nontrivial. Additionally, while line-of-sight or radial
velocities (RVs) are easily determined from spectroscopy, the tangental
velocities, perpendicular to RV, are only measurable through proper
motions, typically requiring precise astrometry as well. \emph{Gaia}'s
mission is to produce extremely precise astrometry enabling measurements
of unprecedented accuracy and scale for proper motions, parallaxes, and
magnitudes.

\emph{Gaia} was designed to revolutionize proper motion and parallax
measurements. \emph{Gaia} is a space-based, all-sky survey telescope
with two primary 1.45x0.5m mirrors situated at the Sun-Earth L2 lagrange
point \citep{gaiacollaboration+2016}. \emph{Gaia} was launched in XXX,
completing its mission in 2025 (but with two more data releases
planned). By imagining two patches of sky on the same focal plane,
separated by a fixed angle of 106.5degrees, \emph{Gaia} is able to
measure absolute proper motions by comparing the apparent shifts of
stars in different regions of the sky. In addition to precise
astrometric information, \emph{Gaia} measures the magnitude of stars in
the very wide \emph{G} band (330-1050nm), blue and red colours using the
blue and red photometers (BP and RP, 330-680, 640-1050 respectively),
and takes low resolution BP-RP spectra and radial velocity measurements
of bright stars (magnitudes \textless16?).

\emph{Gaia} has revolutionized many astronomical disciplines, the least
of which is local group and Milky Way science. While proper motions of a
dwarf galaxies has been measured in a case by case basis by the Hubble
Space Telescope, a full systematic determination of proper motions for
most dwarf galaxies was unavailable until \emph{Gaia}
\citep{MV2020a, battaglia+2022}. \emph{Gaia} has furthermore allowed for
the detection of many substructures of the halo, streams,

\section{Gaia Membership Selection}\label{gaia-membership-selection}

Here, we briefly describe J+24's membership estimation method. J+24 use
a Bayesian framework incorporating proper motion (PM), colour-magnitude
diagram (CMD), and spatial information to determine the probability that
a given star belongs to the satellite. By accounting for PM in
particular, J+24 produces low contamination samples of candidate member
stars. J+24 extends the algorithm presented in \citet{MV2020a};
\citet{MV2020b} but additionally includes an optional secondary,
extended spatial component to find possible members as far as
\textasciitilde10 half-light radii \(R_h\) from some dwarf galaxies. See
also similar work by \citet{pace+li2019}; \citet{battaglia+2022};
\citet{pace+erkal+li2022}; \citet{qi+2022}.

To create a high-quality sample, J+24 select stars initially from Gaia
within a 2--4 degree circular region centred on the dwarf satisfying:

\begin{itemize}
\tightlist
\item
  Solved astrometry, magnitude, and colour.
\item
  Renormalized unit weight error, \({\rm ruwe} \leq 1.3\), ensuring high
  quality astrometry. \texttt{ruwe} is a measure of the excess
  astrometric noise on fitting a consistent parallax-proper motion
  solution \citep[see][]{lindegrenXXX}.
\item
  3\(\sigma\) consistency of measured parallax with dwarf's distance
  (dwarf parallax is very small; with \citet{lindegren+2021} zero-point
  correction).
\item
  Absolute proper motions, \(\mu_{\alpha*}\), \(\mu_\delta\), less than
  10\(\,{\rm mas\ yr^{-1}}\). (Corresponds to tangental velocities of
  \(\gtrsim 500\) km/s at distances larger than 10 kpc.)
\item
  Corrected colour excess is within expectations:
  \(|C^*| \leq 3\,\sigma_{C^*}(G)\), with \(C^*\) and \(\sigma_{C^*}\)
  from \citet{riello+2021}. Removes stars with unreliable photometry.
\item
  De-reddened \(G\) magnitude is between
  \(22 > G > G_{\rm TRGB} - 5\delta\rm {DM}\). Removes very faint stars
  and stars significantly brighter than the tip of the red giant branch
  (TRGB) magnitude plus the distance modulus uncertainty
  \(\delta {\rm DM}\).
\item
  Colour is between \(-0.5 < {\rm BP - RP} <  2.5\) (dereddened).
\end{itemize}

Photometry is dereddened with \citet{schlegel+finkbeiner+davis1998}
extinction maps.

J+24 define likelihoods \({\cal L}\) representing the probability
density that a star is consistent with either the MW stellar background
(\({\cal L}_{\rm bg}\)) or the satellite galaxy
(\({\cal L}_{\rm sat}\)). In either case, the likelihoods are the
product of a spatial, PM, and CMD term: \begin{equation}
{\cal L} = {\cal L}_{\rm space}\ {\cal L}_{\rm PM}\ {\cal L}_{\rm CMD}.
\end{equation}

Each likelihoods is normalized over their respective 2D parameter space
for both the satellite. Since the likelihoods are normalized,
\(f_{\rm sat}\), representing the fraction of member stars in the field,
controls the relative scaling between the satellite and background
likelihoods. The total likelihood for any star in this model is then
\begin{equation}\protect\phantomsection\label{eq:Ltot}{
{\cal L}_{\rm tot} = f_{\rm sat}{\cal L}_{\rm sat} + (1-f_{\rm sat}){\cal L}_{\rm bg}
}\end{equation} The probability that any star belongs to the satellite
is then given by \begin{equation}
P_{\rm sat} = \frac{f_{\rm sat}{\cal L}_{\rm sat}}{f_{\rm sat}{\cal L}_{\rm sat} + (1-f_{\rm sat}){\cal L}_{\rm bg}}.
\end{equation}

For the satellite's spatial likelihood, J+24 consider both one-component
and a two-component cases. The one component model is constructed as a
single exponential profile (\(\Sigma \propto e^{R_{\rm ell} / R_s}\)),
with \(R_s\) fixed to the value in the table in \citet{MV2020a}.
Additionally, structural uncertainties (for position angle, ellipticity,
and scale radius) are sampled over to construct the final likelihood
map. The two-component model instead adds a second exponential,
\(\Sigma_\star \propto e^{-R/R_s} + B\,e^{-R/R_{\rm outer}}\). The inner
scale radius is fixed, and the outer scale radius and magnitude of the
second component \(R_{\rm outer}\), \(B\) are solved for. Structural
properties are not accounted for.

The PM likelihood is a bivariate gaussian with variance and covariance
equal to each star's proper motions. J+24 assume the stellar PM errors
are the main source of uncertainty.

The satellite's CMD likelihood is based on a Padova isochrone
\citep{girardi+2002}. The isochrone has a matching metallicity and 12
Gyr age. The (gaussian) colour width is assumed to be 0.1 mag plus the
Gaia colour uncertainty at each magnitude. The HB is modelled as a
constant magnitude extending blue of the CMD (mean magnitude of -2.2, 12
Gyr HB stars and a 0.1 mag width plus the mean colour error). A
likelihood map is constructed by sampling the distance modulus in
addition to the CMD width, taking the maximum of RGB and HB likelihoods.

The background likelihoods are instead empirically constructed. Stars
stars outside of 5\(R_h\) passing the quality cuts estimate the
background density in PM and CMD space. The density is a sum of
bivariate gaussians with variances based on Gaia uncertainties (and
covariance for proper motions). The spatial background likelihood is
assumed to be constant.

J+24 derive \(\mu_{\alpha*}\), \(\mu_\delta\), \(f_{\rm sat}\) (and
\(B\), \(R_{\rm outer}\) for two-component) through an MCMC simulation
with likelihood from Eq.~\ref{eq:Ltot}. Priors are weakly informative or
uniform. The proper motion single component prior is same as
\citet{MV2020a}: a normal distribution with mean 0 and standard
deviation \(100\,\kms\). If 2-component spatial, instead is a uniform
distribution spanning 5\(\sigma\) of single component case w/ systematic
uncertainties. \(f_{\rm sat}\) (and \(B\)) has a uniform prior 0--1.
\(R_{\rm outer}\) has a uniform prior only restricting
\(R_{\rm outer} > R_s\). The mode of each parameter from the MCMC are
then reported and used to calculate the final \(P_{\rm sat}\) values.

We adopt a probability cut of \(P_{\rm sat} = 0.2\) as our fiducial
sample. Most stars are assigned probabilities close to either 0 or 1, so
the choice of probability threshold is only marginally significant.
Additionally, even for a probability cut of 0.2, the purity of the
resulting sample with RV measurements is very high (\textasciitilde90\%,
J+24). Note there is likely a systematic bias in using stars with RV
measurements to measure purity. Fainter stars are less likely to have
been targeted and have poorer astrometry and colour information. We find
no difference in the resulting density distributions when restricting
stars to be brighter than a specific magnitude.

\subsection{Selected samples}\label{selected-samples}

In
Figs.~\ref{fig:sculptor_selection}, \ref{fig:umi_selection}, \ref{fig:fornax_selection},
we illustrate the resulting samples from the algorithm in the tangent
plane (\(\xi\), \(\eta\), \emph{does this need defined?}), CMD, and
proper motion space. For all galaxies, each criteria plays a
commensurate role in sifting out nonmembers. The CMD is well defined and
probable members only extend a few times the colour uncertainty from the
CMD. In proper motion space, selected stars are within \(\masyr\) from
the systematic proper motion. The stars furthest away in proper motion
space typically have large uncertainties \(\sim 1\masyr\), so most
members are reasonably consistent. In any case, the algorithm does
enable removing many stars which do not have consistent proper motions
with the dwarf. Finally, the spatial likelihood reduces the
probabilities of stars distant from the dwarfs centres. Member stars are
rare outside of \(\sim5R_h\), resulting from both the likelihood
specification and the lower density of background stars consistent with
the CMD and PM of the satellite.

One limitation of the J+24 method is that it assumes a spatial form of
the dwarf galaxy, possibly limiting the detection of very distant stars
or features. Figs.~\ref{fig:sculptor_selection}, \ref{fig:umi_selection}
also illustrates the distribution of stars selected without a spatial
criterion. We define the \emph{CMD+PM} selection as
\begin{equation}\protect\phantomsection\label{eq:sel_cmd_pm}{
{\cal L}_{\rm CMD,\ sat}\ {\cal L}_{\rm PM,\ sat} > {\cal L}_{\rm CMD,\ bg}\ {\cal L}_{\rm PM,\ bg}
}\end{equation} with the likelihoods from J+24 as described above. These
stars are distributed similar to the fiducial (probable members) sample
but with a fainter uniform distribution across the entire field. This
illustrates the approximate background of stars which may be confused as
members. Additionally, since there is no clear spatial structure in the
\emph{CMD+PM} (or \emph{all}) sample outside several \(R_h\), it is
unlikely that there are additional faint, tidal features detectable with
\emph{Gaia}. Not shown here, we also try a variety of simpler, absolute
cuts and thresholds, finding no evidence of other structure in both
\emph{Gaia} and DELVE/UNIONS data.

Finally, we illustrate the location of RV-confirmed members from
sec.~\ref{sec:rv_obs} in each panel. Because RV targets tend to be
brighter than a typical \emph{Gaia} candidate, RV members typically have
more precise PMs. However, the RV members fill about the same area on
the CMD down to the magnitude limit. Additionally, while the number of
stars with RV measurements decreases at large radii, there are still
confirmed RV members as far as \(\sim 10R_h\) for both galaxies. These
stars illustrate the extended profiles of each galaxy. Note that for an
exponential profile, 99.95\% of stars should be within \(6R_h\), so the
discovery of any stars beyond \(6R_h\) hints at a deviation from an
exponential profile. These \emph{extratidal} stars hint at something
tidal in nature, and our goal is to explore possible interpretations.

\begin{figure}
\centering
\pandocbounded{\includegraphics[keepaspectratio]{figures/scl_selection.pdf}}
\caption[Sculptor sample selection]{The selection criteria for Scl
members. Field stars (satisfying quality criteria) are light grey,
CMD+PM selected stars are turqoise (eq.~\ref{eq:sel_cmd_pm}, probable
members (2-component) are blue diamonds, and RV confirmed members are
indigo crosses. We mark the two stars from \citet{sestito+2023a} with
orange-outlined indigo stars. \textbf{Top:} Tangent plane \(\xi, \eta\).
The ellipse represents 3 half-light radii. \textbf{Bottom left:} Colour
magnitude diagram in Gaia G versus BP - RP. We plot the Padova 12Gyr
MH=-1.68 isochone in orange. The black bar in the top left represents
the median colour error. \textbf{Bottom right:} Proper motion in
declination \(\mu_\delta\) vs RA \(\mu_{\alpha*}\) (corrected) the
orange circle represents the \citet{MV2020b} proper motion. The black
cross represents the median proper motion
error.}\label{fig:sculptor_selection}
\end{figure}

\begin{figure}
\centering
\pandocbounded{\includegraphics[keepaspectratio]{figures/umi_selection.pdf}}
\caption[Ursa Minor sample selection]{Similar to
Fig.~\ref{fig:sculptor_selection} except for Ursa Minor. We outline RV
members outside of \(6R_h\) in black stars (from \citet{sestito+2023b},
\citet{pace+2020} and \citet{spencer+2018}).}\label{fig:umi_selection}
\end{figure}

\begin{figure}
\centering
\pandocbounded{\includegraphics[keepaspectratio]{figures/fornax_selection.pdf}}
\caption[Fornax sample selection]{Similar to
Fig.~\ref{fig:sculptor_selection} except for
Fornax.}\label{fig:fornax_selection}
\end{figure}

\subsection{Density Profiles}\label{density-profiles}

Density profiles are an essential observational constraint for our later
simulations. To derive density profiles, we use 0.05 dex bins in log
radius. We remove bins at smaller (larger) radii than the first bin to
contain no stars, working outwards. We use symmetric poisson
uncertainties. As discussed below, these uncertainties are
straightforward but are likely under-representative.

Figs.~\ref{fig:scl_observed_profiles}, \ref{fig:umi_observed_profiles}
show the derived density profiles for Scl and UMi. We calculate density
profiles for different selections of stars from above: all quality
stars, CMD+PM only, and the probable member samples. In each case, all
samples are the same towards the inner regions of the satellite.
Classical dwarfs dominate the stellar density in their cores. However,
the density profile \emph{all stars} plateaus at the total background in
the field at radii of 30-60 arcminutes. By restricting stars to being
most likely satellite members by CMD + PM, the background is reduced by
1-2 dex. The CMD+PM background plateau likely represents the density of
background stars which could be mistaken as members. Finally, the
background (BG)-subtracted profile results from subtracting the apparent
background in the \emph{all} profile.

Note that the probable members (fiducial) density profile continues to
confidently estimate the density profile below the CMD+PM background.
These points are likely unreliable (see discussion below). However,
before this point, both the BG subtracted and probable members density
profiles are strikingly similar. Assumptions about the details of the
likelihood and spatial dependence have marginal influence on the
resulting density profile when the satellite is higher density than the
background. Thus the detection of excesses of stars in Sculptor (past 20
arcmin) and UMi (past 15-20 arcmin) are robust.

Nearby to UMi, there is a small, likely unassociated, ultrafaint star
cluster, Muñoz 1 \citep{munoz+2012}. The cluster is at a relative
position of \((\xi, \eta) \approx(-42, -15)\) arcminutes, corresponding
to an elliptical radius of 36 arcminutes. However this cluster does not
have a bright RGB, so has few stars brighter than a \(G\) mag of 22.
While Muñoz 1 may contribute \(\sim\) 5 stars to the density profile of
UMi, this would possibly only affect 1-2 bins due to its compact size.

\begin{figure}
\centering
\pandocbounded{\includegraphics[keepaspectratio]{figures/scl_umi_fnx_density_methods.pdf}}
\caption[Sculptor density profiles]{The density profile of Sculptor,
Ursa Minor, and Fornax for different selection criteria, ploted as log
surface dencity versus log elliptical radius. Residuals are with respect
to th interpolated \emph{probable members} density. \emph{Probable
members} selects stars with PSAT \textgreater{} 0.2 considering PM, CMD,
and spatial, \emph{CMD+PM} select stars more likely to be members
according to CMD and PM only, \emph{all} selects any high quality star,
and \emph{BG subtracted} is the background-subtracted density derived
from high-quality stars. We mark the half-light radius ( vertical dashed
line) and the break radius (black arrow,
REF).}\label{fig:scl_observed_profiles}
\end{figure}

\subsection{Caveats}\label{caveats}

The J+24 method was designed in particular to detect the presence of a
density excess and find individual stars at large radii to be followed
up. We are more interested in accurately quantifying the density profile
and size of any perturbations. One potential problem with using J+24's
candidate members is that the algorithm assumes the density is either
described by a single or double exponential. If this model does not
accurately match the actual density profile of the dwarf galaxy, we
would like to understand how strongly influenced the density profile is
by this assumption.

In particular, in Fig.~\ref{fig:umi_observed_profiles}, notice that the
PSAT method produces small errorbars, even when the density is \(>1\)dex
below the local background. These stars are likely selecting stars from
the statistical MW background consistent with UMi PM / CMD, recovering
the assumed density profile. As a result, the reliability of these
density profiles below the CMD+PM background may be questionable. A more
robust analysis, removing this particular density assumption, would be
required to more appropriately represent the knowledge of the density
profile as the background begins to dominate.

J+24 do not account for structural uncertainties in dwarfs for the two
component case. We assume constant ellipticity and position angle. Dwarf
galaxies, in reality, are not necessarily smooth and constant. They do
test an alternative method using circular radii for the extended density
component, and we find these density profiles are very similar. We also
assume a constant ellipticity and position angle here.

Finally, our approximation to the poisson uncertainties Additionally, a
more self consistent model would fit the density profile to the entire
field at once, eliminating possible misrepresentation of the
uncertainties.

While \emph{Gaia} has shown excellent performance, some notable
limitations may introduce problems in our interpretation and reliability
of density profiles.

Gaia systematics in proper motions and parallaxes are typically smaller
than the values for sources of magnitudes \(G\in[18,20]\). Since we use
proper motions and parallaxes as general consistency with the dwarf, and
factor in systematic uncertainties in each case, these effects should
not be too significant. However, the systematic proper motion
uncertainties becomes the dominant source of uncertainty in the derived
systemic proper motions of each galaxy (see
tbls.~\ref{tbl:scl_obs_props}, \ref{tbl:umi_obs_props}).

\emph{Gaia} shows high but imperfect completeness, particularly showing
limitations in crowded fields and for faint sources (\(G\gtrapprox20\)).
As discussed in \citet{fabricius+2021}, fir the high stellar densities
in globular clusters, the completeness relative to HST varies
significantly with the stellar density. However, the typical stellar
densities of dwarf galaxies are much lower, at about 20 stars/arcmin =
90,000 stars / degree, lower than the lowest globular cluster densities
and safely below the crowding limit of 750,000 objects/degree for BP/RP
photometry. , where the completeness down to \(G\approx 20\) is
\(\sim 80\%\). Closely separated stars pose problems for Gaia's on-board
processing, as the pixel size is 59x177 mas on the sky. This results in
a reduction of stars separated by less than 1.5'' and especially for
stars separated by less than 0.6 arcseconds. The astrometric parameters
of closely separated stars furthermore tends to be of lower quality
\citep{fabricius+2021}. However, even for the denser field of Fornax,
only about 3\% of stars have a neighbour within 2 arc seconds, so
multiplicity should not affect completeness too much (except for
unresolved binaries). One potential issue is that the previous analysis
do not account for our cuts on quality and number of astrometric
parameters. These could worsen completeness, particularly since the
BP-RP spectra are more sensitive to dense fields. In \textbf{REF}, we
test if magnitude cuts impact the resulting density profiles, finding
that this is likely not an issue.

\section{Radial velocity modeling}\label{sec:rv_obs}

\subsection{Data selection}\label{data-selection}

For both Sculptor and Ursa Minor, we construct literature samples of
radial velocity measurements. We combine these samples with J+24's
members to produce RV consistent stars and to compute velocity
dispersion, systematic velocities, and test for the appearance of
velocity gradients.

First, we crossmatch all catalogues to J+24 Gaia stars. If a study did
not report GaiaDR3 source ID's, we match to the nearest star within 1-3
arcseconds (see REF \citet{tab:rv_measurements}). We combine the mean RV
measurement from each study using the inverse-variance weighted mean
\(\bar v\), standard uncertainty \(\delta \bar v\), and (biased)
variance \(s^2\). We remove stars with significant velocity dispersions
as measured between observations in a study or between studies. By using
that \(\chi^2=\frac{s^2}{\delta \bar v^2}\), we remove stars with a
\(\chi^2\) larger than the 99.9th percentile of the \(\chi^2\)
distribution with \(N-1\) measurements. This cut typically removes stars
with reduced chi-squared values
\(\tilde\chi^2  = \frac{s^2}{\nu\,\delta \bar v^2}\gtrsim 7\) (since the
number of measurements is 1-3 typically).

Next, we need to correct the coordinate frames for the solar motion and
on-sky size of the galaxy. We transform the frame into the galactic
standard of rest (GSR). The next step is to account for the slight
differences in the direction of each radial velocity. Let the \(\hat z\)
be the direction from the sun to the dwarf galaxy. Then if \(\phi\) is
the angular distance between the centre of the galaxy and the individual
star, the corrected radial velocity is then \begin{equation}
v_z = v_{\rm los, gsr}\cos\phi  - v_{\alpha}\cos\theta \sin\phi - v_\delta \sin\theta\sin\phi
\end{equation} where \(v_{\rm los, gsr}\) is the line of sight velocity
in the GSR frame, \(v_\alpha\) and \(v_\delta\) are the tangental
velocities in RA and Dec, and \(\theta\) is the position angle of the
star with respect to the centre of the dwarf. The correction from both
effects induces an apparent gradient of about \(1.3\,\kmsdeg\) for
Sculptor and less for Ursa Minor \citep[see
also][]{WMO2008, strigari2010}. We add the uncertainty in \(v_z\) from
the distance uncertainty and velocity dispersion in quadrature to the RV
uncertainties for each star. We then use the \(v_z\) values for the
following modelling, however repeating with uncorrected, heliocentric
velocities does not significantly affect the results.

The combined likelihood, including RV information, becomes
\begin{equation}
{\cal L} = {\cal L}_{\rm space} {\cal L}_{\rm CMD} {\cal L}_{\rm PM} {\cal L}_{\rm RV}
\end{equation} where we assume that the satellite and background
distributions are Gaussian. Specifically, \begin{equation}
\begin{split}
{\cal L}_{\rm RV, sat} &= f\left( \frac{v_i -\mu_{v}}{\sqrt{\sigma_{v}^2 + (\delta v_i)^2}}\right) \\
{\cal L}_{\rm RV, bg} &= f\left( v_i /  \sigma_{\rm halo} \right)
\end{split}
\end{equation} where \(f\) is the probability density of a standard
normal distribution, \(\mu_v\) and \(\sigma_v\) are the systemic
velocity and dispersion of the satellite, and \(\delta v_i\) is the
individual measurement uncertainty. Typically, the velocity dispersion
will dominate the uncertainty budget here. We assume a halo/background
velocity dispersion of a constant \(\sigma_{\rm halo} = 100\,\kms\)
\citep[e.g.][]{brown+2010}.

Similar to above, we retain stars with the resulting membership
probability of greater than 0.2. Because of the additional information
from radial velocities, most stars have probabilities close to 1 or 0 so
the probability cut is not too significant.

We assume priors on systematic velocity and velocity dispersion of
\begin{equation}
\begin{split}
\mu_{v} &= N(0\,\kms, \sigma_{\rm halo}^2) \\ 
\sigma_{v} &= U(0, 20\,\kms)
\end{split}
\end{equation} where \(\sigma_{\rm halo} = 100\,{\rm km\,s^{-1}}\) is
the velocity dispersion of the MW halo adopted above, a reasonable
assumption for dwarfs in orbit around the MW.

\subsection{Results}\label{sec:rv_results}

\begin{figure}
\centering
\pandocbounded{\includegraphics[keepaspectratio]{figures/scl_umi_rv_fits.pdf}}
\caption[LOS velocity fit to Scl.]{Velocity histogram of Scl and UMi in
terms of \(v_z\) (REF). Orange points are from our crossmatched RV
membership sample.}
\end{figure}

For Sculptor, we combine radial velocity measurements from APOGEE,
\citet{sestito+2023a}, \citet{tolstoy+2023}, and \citet{WMO2009}.
\citet{tolstoy+2023} and \citet{WMO2009} provide the bulk of the
measurements. We find that there is no significant velocity shift in
crossmatched stars between catalogues. After crossmatching to high
quality Gaia stars and excluding significant stellar velocity
dispersions, we have a sample of 1918 members.

We derive a systemic velocity for Sculptor of \(111.3\pm0.2\,\kms\)with
velocity dispersion \(9.64\pm0.16\,\kms\). Our values are very
consistent with previous work \citep[e.g.][\citet{arroyo-polonio+2024},
\citet{battaglia+2008}]{walker+2009}. See appendix REF for a more
detailed comparison between different samples and additional tests.

We detect a moderately significant gradient of \(4.3\pm1.3\,\kmsdeg\) at
a position angle of \(-149_{-13}^{+17}\) degrees (see appendix REF).
Several past work has attempted to detect a gradient in Sculptor, but no
consensus has been reached. \citet{arroyo-polonio+2024} detect a
velocity gradient of \(4\pm1.5\,\kmsdeg\) in a similar direction using
the \citet{tolstoy+2023} sample, finding inconclusive statistical
evidence. They additionally suggest a third chemodynamical component of
the galaxy which may bias rotation measurements. \citet{battaglia+2008}
also detect a \(-7.6_{-2.2}^{+3.0}\,\kmsdeg\) velocity gradient along
the major axis, approximately the same direction. Instead,
\citet{strigari2010}; \citet{martinez-garcia+2023} detect no significant
gradient in Sculptor using \citet{WMO2009} sample. Note that
pre-\emph{Gaia} work did not have as strong of a constraint on the
proper motion of Scl, which limits conclusions of the intrinsic velocity
gradient in Scl.

For UMi, we collect radial velocities from, APOGEE,
\citet{sestito+2023b}, \citet{pace+2020}, and \citet{spencer+2018}. We
shifted the velocities of \citet{spencer+2018} (\(-1.1\,\kms\)) and
\citet{pace+2020} (\(+1.1\,\kms\) ) to reach the same scale. We found
183 crossmatched common stars (passing 3\(\sigma\) RV cut, velocity
dispersion cut, and PSAT J+24 \textgreater{} 0.2 w/o velocities). Since
the median difference in velocities in this crossmatch is about 2.2
km/s, we adopt 1 km/s as the approximate systematic error here. Our
final sample includes 831 members.

We derive a mean \(-245.8\pm0.3_{\rm stat}\,\kms\) and velocity
dispersion of \(8.8\pm0.2\,\kms\) for UMi. This is consistent with
\citet{pace+2020} and to a lesser extent with \citet{spencer+2018}. We
do not find evidence for a velocity gradient, consistent with past work
\citep{pace+2020, martinez-garcia+2023}.

\subsection{Discussion and
limitations}\label{discussion-and-limitations}

Our model here is relatively simple. Some things which we note as
systematics or limitations:

\begin{itemize}
\tightlist
\item
  Inter-study systematics and biases. While basic crossmatches and a
  simple velocity shift, combining data from multiple instruments is
  challenging. This appears to be a minor issue (Sculptor) or is
  corrected for (Ursa Minor).
\item
  Misrepresentative uncertainties. Inspection of the variances compared
  to the standard deviations within a study seems to imply that errors
  are accurately reported. APOGEE notes that their RV uncertainties are
  known to be underestimates but are a small proportion of our sample.
\item
  Binarity. While not too large of a change for classical dwarfs, this
  could inflate velocity dispersions of about \(9\,\kms\) by about
  \(1\,\kms\)\citep{spencer+2017}. Thus, our measurement is likely
  slightly inflated given the high binarity fractions measured in these
  systems \citep[\citet{spencer+2018}]{arroyo-polonio+2023}.
\item
  Multiple populations. Both Sculptor and Ursa Minor likely contain
  multiple populations \citep[\citet{pace+2020},
  \citet{tolstoy+2004}]{arroyo-polonio+2024}. Since we only model a
  single population, and each population may have a different extent and
  velocity dispersion, this could result in biased velocity dispersions.
  However, it is unclear how to uniquely determine an overall velocity
  dispersion in a multi-population system.
\item
  Selection effects. RV studies each have their own selection effects,
  which may affect the resulting dispersion, especially if different
  populations or regions of the galaxy have different velocities or
  velocity dispersions. We do not attempt to correct for this.
\end{itemize}

For both Ursa Minor and Sculptor, we also fit models to only data from
individual surveys (see REF). Since the resulting parameters are very
similar, we conclude that many of the systematic uncertainties are
likely smaller than the present errors or that each large survey has
similar biases.

\section{Comparison and conclusions}\label{comparison-and-conclusions}

To illustrate the differences between each dwarf galaxy, in
Fig.~\ref{fig:classical_dwarfs_densities}, we compare Sculptor, Ursa
Minor, and Fornax against 2D-exponential and Plummer density profiles
(REF). While all dwarfs appear similar in the inner regions, each dwarf
diverges in the outer regions relative to an exponential. Relative to an
exponential, Fornax is underdense but Sculptor and Ursa Minor are both
overdense. A Plummer profile instead provides a more reasonable fit to
Scl and UMi.

In summary, we have used J+24 data to derive the density profiles for
Fornax, Sculptor, and Ursa Minor. In each case, the density profile is
robust against different selection criteria. Both Sculptor and Ursa
Minor show strong evidence for deviations from an exponential profile.
We also compile velocity measurements to derive the systemic motions and
velocity dispersions of each galaxy. We find evidence for a velocity
gradient in Sculptor of \(4.3\pm1.3\,\kmsdeg\). We find no evidence of
additional (velocity or stellar) substructure in either galaxy. Our goal
in the following chapters is to test if tides provide a viable
explanation for the observed properties of Sculptor and Ursa Minor.

\begin{figure}
\centering
\pandocbounded{\includegraphics[keepaspectratio]{figures/scl_umi_fornax_exp_fit.pdf}}
\caption[Classical dwarf density profiles]{The density profiles of
Sculptor, Ursa Minor, and Fornax compared to Exp2D and Plummer density
profiles. Dwarf galaxies are scaled to the same half-light radius and
density at half-light radius (fit from the inner 3 scale radii
exponential recursively. ). Sculptor and Ursa Minor have an excess of
stars in the outer regions (past \(\log R/R_h \sim 0.3\)) compared with
Sculptor. Sculptor and Ursa Minor's density profiles are created with
the 2-component J+24 model, but the excess does not change significantly
for the 1-component model.}\label{fig:classical_dwarfs_densities}
\end{figure}

\section{Appendix}\label{appendix}

\subsection{Density profile tests}\label{density-profile-tests}

\begin{figure}
\centering
\pandocbounded{\includegraphics[keepaspectratio]{/Users/daniel/thesis/figures/scl_density_methods_extra.pdf}}
\caption[Density profiles]{Density profiles for various assumptions for
Sculptor. PSAT is our fiducial 2-component J+24 sample, circ is a
2-component bayesian model assuming circular radii, simple is the series
of simple cuts described, bright is the sample of the brightest half of
stars (scaled by 2), DELVE is a sample of RGB stars (background
subtracted and rescaled to match). \textbf{TODO} plot exponential fit
and add better labels}\label{fig:sculptor_observed_profiles}
\end{figure}

\begin{figure}
\centering
\pandocbounded{\includegraphics[keepaspectratio]{figures/umi_density_methods_extra.pdf}}
\caption[UMi Density profiles]{Similar to
fig.~\ref{fig:sculptor_observed_profiles} except for Ursa
Minor}\label{fig:umi_observed_profiles}
\end{figure}

Note that a full rigorous statistical analysis would require a
simulation study of injecting dwarfs into Gaia and assessing the
reliability of various methods of membership and density profiles. This
is beyond the scope of this thesis.

\begin{verbatim}
SELECT TOP 1000
       *
FROM delve_dr2.objects
WHERE 11 < ra
and ra < 19
and -37.7 < dec
and dec < -29.7
\end{verbatim}

\section{Velocity modelling and
comparisons}\label{velocity-modelling-and-comparisons}

Here, we describe in additional detail, our methods and comparisons for
RV modelling between studies.

Savage-Dickey calculated Bayes factor using Silverman-bandwidth KDE
smoothed samples from posterior/prior.

\begin{table*}[t]
\centering
\caption{Summary of velocity measurements and derived properties.}
\begin{tabular}{lllllllll}
\toprule
 & Study & Instrument & Nspec & Nstar & Ngood & Nmemb & $\delta v_{\rm med}$ & $R_{\rm xmatch}$/arcmin\\
\midrule
Scl & combined &  & 8945 & 2280 & 2096 & 1981 & 0.9 & \\
 & tolstoy+23 & FLAMES & 3311 & 1701 & 1522 & 1482 & 0.65 & –\\
 & sestito+23a & GMOS & 2 & 2 & 2 & 2 & 13 & –\\
 & walker+09 & MMFS & 1818 & 1522 & 1417 & 1328 & 1.8 & 3\\
 & APOGEE & APOGEE & 5082 & 253 & 170 & 164 & 0.5 & –\\
UMi & combined &  & 4714 & 1225 & 1148 & 863 & 2.1 & \\
 & sestito+23b & GRACES & 5 & 5 & 5 & 5 & 1.8 & –\\
 & pace+20 & DEIMOS & 1716 & 1538 & 829 & 678 & 2.5 & 1\\
 & spencer+18 & Hectoshell & 1407 & 970 & 596 & 406 & 0.9 & 2\\
 & APOGEE & APOGEE & 9500 & 279 & 37 & 67 & 0.6 & –\\
\bottomrule
\end{tabular}
\end{table*}

\begin{table*}[t]
\centering
\caption[Sculptor RV fits]{MCMC fits for different RV datasets for Sculptor among 3 different models. }
\label{tbl:scl_rv_mcmc}
\begin{tabular}{lllllll}
\toprule
study & mean & sigma & $\partial \log\sigma / \partial \log R$ & $\partial v_z / \partial x$ (km/s/deg) & $\theta_{\rm grad} / ^{\circ}$ & $\log B_2/B_1$\\
\midrule
all &  &  &  &  &  & \\
 & $111.3\pm0.2$ & $9.64\pm0.16$ & - & - & - & 0\\
 & $111.3 \pm 0.2$ & $9.61 \pm0.16$ & - & $4.3\pm1.3$ & $-147_{-13}^{+17}$ & -1.9\\
 & $111.2\pm0.2$ & $9.66\pm0.16$ & $0.07\pm0.02$ & - & - & -3.4\\
tolstoy+23 &  &  &  &  &  & \\
 & $111.3 \pm 0.3$ & $9.79 \pm 0.18$ & - & - & - & 0\\
 & $111.3\pm0.3$ & $9.77\pm0.19$ & – & $4.3\pm1.4$ & $-154_{-13}^{+19}$ & -1.3\\
 & $111.2 \pm 0.3$ & $9.73\pm0.19$ & $0.085 \pm 0.023$ & – & – & -4.6\\
walker+09 &  &  &  &  &  & \\
 & $111.1\pm0.3$ & $9.5\pm0.2$ & – & – & – & 0\\
 & $111.1\pm0.3$ & $9.5\pm0.2$ & - & $5.2_{-1.6}^{+1.7}$ & $-134_{-16}^{+22}$ & -1.9\\
 & $111.1\pm0.3$ & $9.6\pm0.2$ & $0.06\pm0.03$ & – & – & +0.3\\
apogee &  &  &  &  &  & \\
 & $111.2\pm0.7$ & $8.6\pm0.5$ & – & – & – & –\\
 & $111.2\pm0.7$ & $8.5\pm0.5$ & – & $6\pm3$ & $-126_{-33}^{+45}$ & +0.1\\
 & $111.1\pm0.7$ & $8.5\pm0.5$ & $0.07\pm0.06$ & – & – & +0.9\\
\bottomrule
\end{tabular}
\end{table*}

\begin{table*}[t]
\centering
\caption[Ursa Minor RV fits]{MCMC fits for UMi velocity dispersion. }
\label{tbl:umi_rv_mcmc}
\begin{tabular}{lllll}
\toprule
study & mean & sigma & $\log bf_{\rm sigma}$ & $\log bf_{\rm grad}$\\
\midrule
all & $-245.8\pm0.3$ & $8.8\pm0.2$ & +1.3 & +0.9\\
pace & $-244.6\pm0.4$ & $9.0\pm0.3$ & +0.3 & +0.5\\
spencer & $-246.9\pm0.4$ & $8.8\pm0.3$ & +1.8 & -0.06\\
apogee & $-245.6\pm1.2$ & $10.0_{-0.8}^{+1.0}$ & +1.0 & +0.5\\
\bottomrule
\end{tabular}
\end{table*}

\begin{figure}
\centering
\pandocbounded{\includegraphics[keepaspectratio]{figures/scl_rv_scatter.pdf}}
\caption[Scl velocity sample]{RV members of Sculptor plotted in the
tangent plane coloured by corrected velocity difference from mean
\(v_z - \bar v_z\) . The black ellipse marks the half-light radius in
fig.~\ref{fig:scl_selection}. The black and green arrows mark the proper
motion (PM, GSR frame) and derived velocity gradient (rot) vectors (to
scale).}
\end{figure}

\begin{figure}
\centering
\pandocbounded{\includegraphics[keepaspectratio]{/Users/daniel/thesis/figures/scl_vel_gradient_scatter.pdf}}
\caption[Scl velocity gradient]{The corrected LOS velocity along the
best fit rotational axis. RV members are black points, the systematic
\(v_z\) is the horizontal grey line, blue lines represent the
(projected) gradient from MCMC samples, and the orange line is a rolling
median (with a window size of 50).}
\end{figure}
