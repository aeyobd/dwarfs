As discussed in Section~\ref{sec:exponential_profiles}, the projected
luminosity/stellar mass density profile of dwarf galaxies is generally
well-described by an exponential law (Eq.~\ref{eq:exponential_law}). A
prototypical example is the Fornax dSph --- well-fit by an exponential
over 4.5 decades in surface density. On the other hand, dSphs like
Sculptor or Ursa Minor have profiles which deviate significantly from an
exponential profile fitted to the inner regions, with a clear excess of
stars/light in the outer regions. In this chapter, we critically review
the density profile of the MW classical dwarfs. This thesis concerns the
origin of this excess, with a tidal interpretation explored in
subsequent chapters.

\section{\texorpdfstring{Satellite stellar membership with
\emph{Gaia}}{Satellite stellar membership with Gaia}}\label{satellite-stellar-membership-with-gaia}

Measuring the light profile of a resolved galaxy requires careful
consideration of whether any given star belongs to the system or not.
Without removing contamination from foreground/background sources, faint
features may be lost in the noise or be of uncertain association. When
only photometric data was available, the membership of stars was
ascertained using the colour-magnitude diagram alone \citep[e.g.,
matched filter methods like those used by][]{rockosi+2002}. Now that
\emph{Gaia} data are available, stellar parallax and proper motion are
also available to improve membership assignment.

Here, we use the \citet{jensen+2024}'s (hereafter J+24) membership
probabilities from \emph{Gaia} data. J+24 used a Bayesian framework
incorporating proper motion (PM), colour-magnitude diagram (CMD), and
spatial information to determine the probability that a given star
belongs to the satellite or foreground/background. By accounting for PM
in particular, J+24 produced low contamination samples of candidate
member stars out to large distances from a dwarf galaxy. J+24 extended
the algorithm presented in \citet{MV2020a}; \citet{MV2020b} by
additionally including a secondary, extended spatial component. J+24
detected candidate members out to \textasciitilde10 half-light radii
from the centres of some galaxies (\(R_h\)). Similar recent work has
also included \citet{pace+li2019}; \citet{battaglia+2022};
\citet{pace+erkal+li2022}; \citet{qi+2022}.

J+24's method uses likelihoods, \({\cal L}\), representing the
probability density that a star is consistent with either the
foreground/background, \({\cal L}_{\rm bg}\), or the satellite galaxy,
\({\cal L}_{\rm sat}\). In either case, the likelihoods are the product
of a spatial, PM, and CMD component, \begin{equation}{
{\cal L} = {\cal L}_{\rm space}\ {\cal L}_{\rm PM}\ {\cal L}_{\rm CMD}.
}\end{equation}

For a satellite, the \emph{spatial likelihood} is specified either as a
one- or two-component elliptical exponential profile. A second component
is included only when the preferred amplitude of the second component is
non-zero. The \emph{proper-motion likelihood} quantifies the agreement
of a star's motion with the dwarf galaxy's systemic motion, accounting
for observational uncertainties. The \emph{CMD likelihood} measures the
consistency of a star's \emph{Gaia} \(G\) magnitude and
\(G_{\rm BP}- G_{\rm RP}\) colour with theoretical isochrones for the
galaxy. For the background likelihoods, the spatial likelihood is a
uniform distribution over the field, and the background CMD and PM
likelihoods are constructed empirically from the stars in an annulus far
from the satellite. Each likelihood is normalized as a probability
density over the respective parameter space.

The total likelihood is a mixture model of the satellite and background,
weighted by the fraction of stars in the field belonging to the
satellite, \(f_{\rm sat}\): \begin{equation}{
{\cal L}_{\rm tot} = f_{\rm sat}{\cal L}_{\rm sat} + (1-f_{\rm sat}){\cal L}_{\rm bg}.
}\end{equation} The probability that a given star belongs to the
satellite is then \begin{equation}{
P_{\rm sat} = 
\frac{f_{\rm sat}\,{\cal L}_{\rm sat}}{{\cal L}_{\rm tot}}
= \frac{f_{\rm sat}{\cal L}_{\rm sat}}{f_{\rm sat}{\cal L}_{\rm sat} + (1-f_{\rm sat}){\cal L}_{\rm bg}}.
}\end{equation}

J+24 fit this model using Monte Carlo Markov chain simulations, solving
for the proper motions, satellite membership fraction (\(f_{\rm sat}\)),
and structural properties of the second exponential density profile (if
included). The median parameters from the samples are then used to
calculate the \(P_{\rm sat}\) we use for sample selection.

For our fiducial sample, we adopt a minimum probability of
\(P_{\rm sat} = 0.2\). We do not filter on magnitudes explicitly, but
J+24's quality cuts typically only include stars with \(G < 21\). We use
the \(P_{\rm sat}\) values from the elliptical 2-component runs if a
galaxy shows evidence for an outer component, the 1-component run
otherwise. Most stars have \(P_{\rm  sat}\) values which are nearly 0 or
1, so the exact choice of probability threshold has little effect on the
resulting sample. Even at our relatively generous probability threshold
of 0.2, the purity remains high when validated against spectroscopic
line-of-sight (LOS) velocities (\textasciitilde90\%, J+24). We note,
however, that these purity estimates may be biased. Stars with LOS
velocities are typically brighter and, as a result, have more precise
\emph{Gaia} measurements. However, we find that our conclusions are
unchanged when limiting samples to only the brightest stars. Altogether,
the J+24 method provides a high-quality, low-contamination sample of
dwarf galaxy candidate member stars, which we will now investigate.

\section{The effects of membership
criteria}\label{the-effects-of-membership-criteria}

We analyze stellar distributions in the tangent plane, considering as
well the projected shape of a galaxy. The tangent plane coordinates
\(\xi\) and \(\eta\) are offsets in RA and declination as measured on
the plane tangent relative to the galaxy centre. To account for the
elliptical shape of the galaxy, we use \(R_{\rm ell}\), which we define
as the circularized elliptical radius, \begin{equation}{
R_{\rm ell}^2 = a\,b\,\left(\frac{{\xi'}^2}{a^2} + \frac{{\eta'}^2}{b^2} \right),
}\end{equation} where \(\xi'\) and \(\eta'\) are the tangent plane
coordinate rotated to align with a dwarf galaxy's major and minor axis,
and \(a\) and \(b\) are the semi-major and semi-minor axis of the
galaxy.

We illustrate the likelihood-based membership selection criteria through
a progressive tightening of criteria. First, we consider a
minimally-refined sample, \textbf{all}, which only excludes stars with
poor astrometry, unreliable photometry, or inconsistent parallaxes.
Next, incorporating CMD and PM information, we use the \textbf{CMD+PM}
to test a selection method agnostic to the spatial position. This sample
includes stars where the CMD and PM combined likelihood favours
satellite membership:
\begin{equation}\protect\phantomsection\label{eq:sel_cmd_pm}{
{\cal L}_{\rm CMD,\ sat}\ {\cal L}_{\rm PM,\ sat} > {\cal L}_{\rm CMD,\ bg}\ {\cal L}_{\rm PM,\ bg}.
}\end{equation} Next, our \textbf{fiducial} sample,
\(P_{\rm sat} > 0.2\), includes a spatial likelihood as described in
J+24. Finally, the \textbf{RV members} (for radial velocity) sample adds
line-of-sight velocity information (see Appendix \ref{sec:rv_obs} for
details). While the latter is the best motivated membership sample, we
do not use this sample for stellar density analysis due to its
incompleteness and complex selection function.

Figs.~\ref{fig:scl_selection}, \ref{fig:umi_selection}, \ref{fig:fornax_selection}
show our fiducial sample for three dSphs (Sculptor, Ursa Minor, and
Fornax), as well as impact of different samples, in tangent-plane
coordinates, the CMD, and proper-motion space. The ``all'' sample
extends uniformly across the tangent plane, but includes a substantial
background population. The ``CMD+PM'' sample has a much lower background
density in the tangent plane, revealing the satellite more clearly. The
fiducial sample appears similar to the ``CMD+PM'' sample in the CMD and
PM planes, but excludes improbably distant stars. The ``RV member''
sample also traces out similar distributions in CMD and PM space as the
fiducial sample, with less dispersion, likely reflecting the brighter
magnitudes of stars with spectroscopic follow-up. Each selection
criteria---spatial, CMD, and PM---contributes towards a high-quality
membership assignment.

Based on the fiducial sample's distribution in
Figs.~\ref{fig:scl_selection}, \ref{fig:umi_selection}, \ref{fig:fornax_selection},
Ursa Minor stands out because of its higher ellipticity. Sculptor and
Ursa Minor show a population of members past \(6R_h\) in radius, but
Fornax does not, even though Fornax has many more stars. Fornax also has
a bluer CMD, indicative of more recent star formation. Otherwise, there
are no major morphological differences between these galaxies.

One limitation of the J+24 method is the assumption of a specific
density profile for the dwarf galaxy (a one- or two-component
exponential), which may impact the membership probability of distant
stars. However, there are no clear extensions or overdensities in the
``CMD+PM'' selected sample, which does not include a spatial likelihood.
If even fainter, more extended features exist, they are not clearly
detectable with \emph{Gaia}.

To illustrate the extent of Sculptor and Ursa Minor, we highlight
distant, spectroscopically confirmed members with red-outlined stars in
Figs.~\ref{fig:scl_selection}, \ref{fig:umi_selection}. In particular,
\citet{sestito+2023a}; \citet{sestito+2023b} targeted distant, bright
stars from the J+24 candidate membership list. The most distant stars
confirmed in their work lie at radii \(\gtrsim 10 R_h\) from the centres
of each dwarf. For an exponential profile, 99.95\% of stars fall within
a radius of \(6R_h\), so the mere presence of very distant member stars
provides evidence for deviations from an exponential profile.

\begin{figure}
\centering
\includegraphics[width=0.7\linewidth,height=\textheight,keepaspectratio]{figures/scl_selection.png}
\caption[Sculptor sample selection]{The distributions of various samples
of \emph{Gaia} stars for Sculptor. We plot light grey points for ``all''
field stars (with consistent parallaxes and reliable photometry and
astrometry), turquoise points for ``CMD+PM'' selected stars
(Eq.~\ref{eq:sel_cmd_pm}), blue squares for the ``fiducial'' sample
(with membership probability \(P_{\rm sat} > 0.2\)), and indigo diamonds
for the ``RV members'' sample. We mark the two far-outlier stars from
\citet{sestito+2023a} with rust-outlined indigo stars. \textbf{Top:}
Tangent plane \(\xi, \eta\). The orange ellipses represents 3 and 6
half-light radii. \textbf{Bottom left:} Colour magnitude diagram in Gaia
\(G\) magnitude versus \(G_{\rm BP} - G_{\rm RP}\) colour. We plot a
Padova \(12\,\)Gyr, \({\rm [Fe/H] }=-1.68\) isochone in orange. The
black bar in the top left represents the median colour error.
\textbf{Bottom right:} Proper motion in declination \(\mu_\delta\) vs RA
\(\mu_{\alpha*}\) (corrected). The orange point marks the systemic
\citet{MV2020b} proper motion. The black cross represents the median
proper motion error.}\label{fig:scl_selection}
\end{figure}

\begin{figure}
\centering
\includegraphics[width=0.7\linewidth,height=\textheight,keepaspectratio]{figures/umi_selection.png}
\caption[Ursa Minor sample selection]{Similar to
Fig.~\ref{fig:scl_selection} except for Ursa Minor. We outline
``velocity confirmed'' members outside a radius of \(6R_h\) with red
stars \citep[from][]{sestito+2023b, pace+2020, spencer+2018}. We also
mark the location of Muñoz 1 with a pink
circle.}\label{fig:umi_selection}
\end{figure}

\begin{figure}
\centering
\includegraphics[width=0.7\linewidth,height=\textheight,keepaspectratio]{figures/fornax_selection.png}
\caption[Fornax sample selection]{Similar to
Fig.~\ref{fig:scl_selection} except for Fornax. RV measurements are from
\citet{WMO2009}. Fornax does not show the same extended outer halo of
probable members as Sculptor or Ursa Minor despite having many more
stars.}\label{fig:fornax_selection}
\end{figure}

\section{Density profiles}\label{density-profiles}

We derive density profiles by binning member stars in constant width
bins in \(\log R_{\rm ell}\) of 0.05 dex. We ignore bins interior or
exterior to the first empty bin in either direction. We use symmetric
Poisson uncertainties as ``error bars'' in the density estimate at each
bin.

Fig.~\ref{fig:scl_observed_profiles} show the derived density profiles
for Sculptor, Ursa Minor, and Fornax. We calculate density profiles for
three different samples from above: ``all'', ``CMD+PM'', and the
``fiducial'' sample. In each case, all samples coincide towards the
inner regions of the satellite. However, the density profile from
``all'' plateaus at the total background in the field at radii of 30-60
arcminutes, depending on the galaxy. By restricting stellar membership
with CMD + PM, the background is reduced by 1-2 dex. The ``CMD+PM''
background plateau represents the density of background stars which
could be mistaken as members because of their coincident colours and
proper motions. Finally, the ``all -- background'' profile results from
subtracting the apparent background in the ``all'' profile.

While the density profiles of all samples agree in the inner regions,
they begin to deviate outside \(6 R_h\). Outside this radius, the
fiducial profile extends 1--2 magnitudes below the CMD+PM selection
background. Because the fiducial profile in this region depends
critically on the spatial likelihood, its shape is vulnerable to the
assumed density profile (an exponential in J+24). We explore this in
more detail in Appendix \ref{sec:density_extra}.

Near UMi, there is a small (\(R_h\sim 0.5'\)), likely unassociated,
ultrafaint star cluster, Muñoz 1 \citep{munoz+2012}. The cluster is at a
relative position of \((\xi, \eta) \approx(-42, -15)\) arcminutes,
corresponding to an elliptical radius of 37 arcminutes. This cluster
does not have a bright RGB, so has few stars brighter than a magnitude
\(G=22\). The cluster has little effect on the elliptically-averaged
density profile (see location on Fig.~\ref{fig:scl_observed_profiles}).

Fig.~\ref{fig:classical_dwarfs_densities} compares the fiducial density
profiles of Sculptor, Ursa Minor, Fornax, and other classical dwarf
galaxies. Of the classicals, we exclude Antlia II, due to the extremely
high background, and Sagittarius, which was not included in J+24. The
density profiles are scaled to match at the half-light radius, taken
from \citet{munoz+2018}. All of the classical dwarfs appear to be well
described by an exponential profile in the inner regions. In the outer
regions, however, Sculptor and Ursa Minor deviate, and show a clear
outer excess over an exponential law (solid black line). These galaxies
are better fit by a Plummer law (dashed black line). The deviation from
an exponential grows outwards, and at \(\sim 8 R_h\), may reach 2 orders
of magnitude. The remainder of this thesis will be devoted to assessing
whether the outer excess shown by Scl and UMi are due to Galactic tides.

\begin{figure}
\centering
\pandocbounded{\includegraphics[keepaspectratio]{figures/scl_umi_fnx_density_methods.pdf}}
\caption[Sculptor density profiles]{The density profile of Sculptor,
Ursa Minor, and Fornax for different selection criteria, ploted as log
surface dencity versus log elliptical radius. The samples are: ``all''
selects any high quality and parallax-consistent star (grey
circles),``all--background'' subtracts a uniform background density from
the ``all'' profile (orange pentagons), ``CMD+PM'' select stars
according to CMD and PM only (teal triangles), and ``fiducial'' also
includes spatial information (blue squares). We mark the half-light
radius with a vertical dashed line and the background density with the
horizontal grey line. For Ursa Minor, we show the expected location of
Muñoz 1 stars as a horizontal bar (ranging from plus/minus 3 half-light
radii).}\label{fig:scl_observed_profiles}
\end{figure}

\begin{figure}
\centering
\pandocbounded{\includegraphics[keepaspectratio]{figures/classical_dwarf_profiles.pdf}}
\caption[Classical dwarf density profiles]{The density profiles of
Sculptor, Ursa Minor, and other classical dwarfs compared to Exp2D and
Plummer density profiles. Dwarf galaxies are scaled to the same
half-light radius and density at half-light radius. Sculptor and Ursa
Minor have an excess of stars in the outer regions (past
\(\log R/R_h \sim 0.3\)) compared with other classical
dwarfs.}\label{fig:classical_dwarfs_densities}
\end{figure}
