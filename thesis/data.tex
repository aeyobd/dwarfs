\section{Gaia Membership Selection}\label{gaia-membership-selection}

Gaia provides unprecedented accuracy in proper motions and magnitudes.
Gaia data is uniquely excellent to produce low-contamination samples of
likely member stars belonging to satellites. Here, we breifly describe
J+24's membership estimation and discuss how this informs our
observational knoledge of each galaxies density profile. In general,
J+24 use a Bayesian framework incorporating proper motion (PM),
colour-magnitude diagram (CMD), and spatial information to determine the
probability that a given star belongs to the satellite. J+24 extends
\citet{MV2020a} (see also \citet{pace+li2019}, \citet{battaglia+2022},
\citet{pace+erkal+li2022}, etc.).

J+24 select stars initially from Gaia within a 2--4 degree circular
region centred on the dwarf satisfying:

\begin{itemize}
\tightlist
\item
  Solved parallax, proper motions, colour, and magnitudes.
\item
  High quality astrometry (\texttt{ruwe\ \textless{}=\ 1.3})
\item
  3\(\sigma\) consistency of measured parallax with dwarf distance +
  uncertainty (typically near zero; with \citet{lindegren+2018}
  zero-point correction).
\item
  Absolute RA and Dec proper motions less than
  10\(\,{\rm mas\ yr^{-1}}\) (corresponding to tangental velocities of
  \(\gtrsim 500\) km/s at distances larger than 10 kpc.)
\item
  No colour excess (\citet{lindegren+2018} equation C.2)
\item
  G \textless{} 22 and less than 5\(\sigma\) above TRGB, and between
  -0.5 and 2.5 in Bp - Rp.
\end{itemize}

Photometry is dereddened using \citet{schlegel+1988} extenction.

J+24 calculate the probability that any star belongs to either the
satellite or the MW background as \[
P_{\rm sat} = \frac{f_{\rm sat}{\cal L}_{\rm sat}}{f_{\rm sat}{\cal L}_{\rm sat} + (1-f_{\rm sat}){\cal L}_{\rm bg}}
\]

where the satellite (sat) and background (bg) likelihoods are simply the
product of the PM, CMD, and spatial components: \[
{\cal L} = {\cal L}_{\rm space}\ {\cal L}_{\rm PM}\ {\cal L}_{\rm CMD}
\]

The satellite likelihood is constructed as

\begin{itemize}
\tightlist
\item
  CMD: The CMD is the lowest metallicity isochrone from Padova
  \citep{girardi+2002} with age 12 Gyr with a colour width of 0.1 mag
  plus the Gaia colour uncertainty at each magnitude. The HB is modelled
  as a constant magnitude extending blue of the CMD. The HB magnitude is
  the mean magnitude of HB stars from most metal poor isochrone with a
  0.1 mag width plus the mean colour error. A likelihood map is
  constructed by sampling the distance modulus in addition to the CMD
  width, taking the maximum of RGB and HB likelihoods.
\item
  Spatial: A single exponential
  (\(\Sigma \propto e^{R_{\rm ell} / R_s}\)) accounting for structural
  uncertainty (sampled over position angle, ellipticity, and half light
  radius).
\item
  Alternative spatial: For Scl and UMi, this is instead a double
  exponential
  \(\Sigma_\star \propto e^{-R/R_s} + B\,e^{-R/R_{\rm outer}}\) where
  the inner exponential remains fixed. Structural parameter
  uncertainties are not accounted for.
\item
  PM. A bivariate gaussian with variance and covariance equal to each
  star's proper motions. Each star's proper motions uncertainty are
  assumed to be the dominant uncertainty.
\end{itemize}

The background likelihood is constructed as:

\begin{itemize}
\tightlist
\item
  CMD : Constructed as a density map using the other quality-selected
  stars outside of \(5R_h\) in the catalogue. The map is a sum of
  bivariate gaussians for each star with standard deviations based on
  the Gaia uncertainties.
\item
  PM: same as CMD except in PM space.
\item
  Spatial: a constant likelihood.
\end{itemize}

Note that each likelihood map is normalized over the respective 2D
parameter space. In order to represent the difference in frequency of
background and forground stars, \(f_{\rm sat}\) represents the field
fraction of member stars.

In J+24, a MCMC simulation is ran using the above likelihood to solve
for the following parameters

\begin{itemize}
\tightlist
\item
  Systemic proper motions \(\mu_\alpha\), \(\mu_\delta\). Single
  component prior is same as \citet{MV2020}: a normal distribution with
  mean 0 and standard deviation 100 km/s. If 2-component spatial,
  instead is a uniform distribution spanning 5\(\sigma\) of single
  component case w/ systematic uncertainties.
\item
  \(f_{\rm sat}\) density normalization. Prior is a uniform distribution
  between 0 and 1.
\item
  Spatial component parameters \(B\) is uniform from 0-1 and
  \(R_{\rm outer}\) is uniform and greater than \(R_s\) for extended
  profiles (Scl and UMi here.)
\end{itemize}

The mode of each parameter from the MCMC are then used to calculate the
final \(P_{\rm sat}\) values we use here.

We adopt a probability cut of \(P_{\rm sat} = 0.2\) as our fiducial
sample. Most stars are assigned probabilities close to 0 or 1, so the
choice of probability threshhold is not too significant. Additionally,
even for a probability cut of 0.2, the purity of the resulting sample
with RV measurements is very high (\textasciitilde90\%, J+24). (Note:
there is likely a high systematic bias in using stars with RV
measurements to measure purity. Fainter stars have poorer measurements
and distant stars are less likely to have been targeted. )

\subsection{Resulting Samples}\label{resulting-samples}

In figures fig.~\ref{fig:sculptor_selection},
fig.~\ref{fig:umi_selction}, we illustrate the resulting samples from
the algorithm. In each case, each criteria plays a roll: proper motions
are centred around the dwarf systemic motion, the CMD is well defined,
and stars only within a few \(R_h\) are included. We also plot the RV
members found in general and in \citet{sestito+2024},
\citet{sestito+2024b}.

The tangent plane, \(\xi\), \(\eta\), is the projection that

We also illustrate the approximate result of removing the spatial
component from the likelihood. We define the CMD+PM selection as stars
satisfying \[
{\cal L}_{\rm CMD,\ sat}\ {\cal L}_{\rm PM,\ sat} > {\cal L}_{\rm CMD,\ bg}\ {\cal L}_{\rm PM,\ bg}
\]

These stars appear similar to the fiducial (probable members) sample,
but instead also appear as an approximately uniform distribution across
the entire field. This illustrates the approximate background of stars
which may be confused as members. Additionally, since there is no clear
spatial structure in the CMD+PM sample, it is unlikely that there are
tidal tails detectable with Gaia. Not shown here, we also try a variety
of simpler, absolute cuts and thresholds, finding no extended structure
beyond what is detected in J+24.

This means that at least at the level of where the background density
dominates, we can exclude models which produce tidal tails brighter than
a density of
\(\Sigma_\star \approx 10^{-2}\,\text{Gaia-stars\ arcmin}^{-2} \approx 10^{-6} \, {\rm M_\odot\ kpc^{-2}}\)
(TODO assuming a distance of \ldots{} and stellar mass of \ldots).

\begin{figure}
\centering
\pandocbounded{\includegraphics[keepaspectratio]{figures/scl_selection.pdf}}
\caption[Sculptor selection criteria]{The selection criteria for Scl
members. Probable members (2-component) are orange, and all field stars
(satisfying quality criteria) are in light grey. \textbf{Top:} Tangent
plane. \textbf{Bottom left:} Colour magnitude diagram. \textbf{Bottom
right:} Proper motion.}\label{fig:sculptor_selection}
\end{figure}

\begin{figure}
\centering
\pandocbounded{\includegraphics[keepaspectratio]{figures/umi_selection.pdf}}
\caption[Ursa Minor Selection]{Similar to
fig.~\ref{fig:sculptor_selection} except for Ursa Minor. UMi features a
very extended density profile with some stars \textasciitilde{} 6\(R_h\)
including a RV member. UMi is also highly elliptical compared to other
classical dwarfs.}\label{fig:umi_selection}
\end{figure}

\subsection{Density Profiles}\label{density-profiles}

Our primary observational constraint is the density profile of a dwarf
galaxy.

To derive density profiles, we use 0.05 dex bins in log radius (i.e.~the
bins are derived from 10\^{}(minimum(logR):0.05:maximum(logR))). The
density in each bin is then (from Poisson statistics) \[
\Sigma_b = N_{\rm memb} / A_{\rm bin} \pm \sqrt{N_{\rm memb}} / A_{\rm bin}
\] where \(N_{\rm memb}\) is the number of members in the bin and
\(A_{\rm bin}\) is the area of the bin's annulus in 2D. As discussed
below, these uncertainties underrepresent the true uncertainty on
multiple accounts. We retain Poisson errors for simplicity here.

In Figures fig.~\ref{fig:scl_observed_profiles},
fig.~\ref{fig:umi_observed_profiles}, we show the derived density
profiles for each galaxy for samples similar to in the selection plots
above. In each case, all samples are the same towards the inner regions
of the satellite, illustrating that these density profiles are dominated
by satellite stars in the centre. The sample containing all stars
reaches a plateau at the total background in the field. However,
restricting stars to being most likely satellite members by CMD + PM,
the background is much lower. This plateau likely represents the real
background of background stars which could be mistaken as members.
Finally, we have the probable members and bg-subtracted densities. BG
subtracted is based on the \texttt{all} density profile, subtracting the
mean background density for stars beyond the last point of the BG
subtracted profile.

Note that the probable members (fiducial) density profile continues to
confidently estimate the density profile below the CMD+PM likely star
background. These points are likely unreliable (see discussion below).
However, before this point, both the BG subtracted and probable members
density profiles are strikingly similar. Assumptions about the details
of the likelihood and spatial dependence have marginal influence on the
resulting density profile when the satellite is higher density than the
background.

\begin{figure}
\centering
\pandocbounded{\includegraphics[keepaspectratio]{figures/scl_density_methods.pdf}}
\caption[Sculptor density profiles]{The density profile of Sculptor for
different selection criteria. \emph{probable members} selects stars with
PSAT \textgreater{} 0.2 considering PM, CMD, and spatial, \emph{CMD+PM}
select stars more likely to be members according to CMD and PM only,
\emph{all} selects any high quality star, and \emph{BG subtracted} is
the background-subtracted density derived from high-quality
stars.}\label{fig:scl_observed_profiles}
\end{figure}

\begin{figure}
\centering
\pandocbounded{\includegraphics[keepaspectratio]{figures/umi_density_methods.pdf}}
\caption[Ursa Minor density profiles]{Similar to
fig.~\ref{fig:scl_observed_profiles} except for Ursa
Minor.}\label{fig:umi_observed_profiles}
\end{figure}

\begin{figure}
\centering
\pandocbounded{\includegraphics[keepaspectratio]{figures/fornax_density_methods.pdf}}
\caption[Fornax density profiles]{Similar to
fig.~\ref{fig:scl_observed_profiles} except for
Fornax.}\label{fig:fornax_observed_profiles}
\end{figure}

\section{Comparison of the Classical
dwarfs}\label{comparison-of-the-classical-dwarfs}

Classical dwarfs are often the brightest dwarfs in the sky.c The density
profiles of classical dwarf galaxies is thus well measured, enabling
detailed comparisons.

\begin{table*}[t]
\centering
\begin{tabular}{lll}
\toprule
galaxy & R\_h (exp inner 3Rs) & num cand\\
\midrule
Fornax & $17.8\pm0.6$ & 23,154\\
Leo I & $3.7\pm0.2$ & 1,242\\
Sculptor & $9.4\pm0.3$ & 6,888\\
Leo II & $2.4\pm0.3$ & 347\\
Carina & $8.7\pm0.4$ & 2,389\\
Sextans I & $20.2 \pm0.9$ & 1,830\\
Ursa Minor & $11.7 \pm 0.5$ & 2,122\\
Draco & $7.3\pm0.3$ & 1,781\\
\bottomrule
\end{tabular}
\end{table*}

Using the same methods above, we select members from J+24's stellar
probabilities. We use the one-component exponential density profiles
with

Figure: Density profiles for each dwarf galaxy. Here, we use the
1-component exponential stellar probabilities from J+24. Dwarf galaxies
are scaled by our derived \(R_h\) values.

\subsection{Density Profile Reliability and
Uncertainties}\label{density-profile-reliability-and-uncertainties}

\begin{itemize}
\tightlist
\item
  How well do we know the density profiles?
\item
  What uncertainties affect derived density profiles?
\item
  Can we determine if Gaia, structural, or algorithmic systematics
  introduce important errors in derived density profiles?
\item
  Using J+24 data, we validate

  \begin{itemize}
  \tightlist
  \item
    Check that PSAT, magnitude, no-space do not affect density profile
    shape too significantly
  \end{itemize}
\item
  Our ``high quality'' members all have \textgreater{} 50 member stars
  and do not depend too highly on the spatial component, mostly
  corresponding to the classical dwarfs
\end{itemize}

J+24's algorithm takes spatial position into account, assuming either a
one or two component exponential density profile. When deriving a
density profile, this assumption may influence the derived density
profile, especially when the galaxy density is fainter than the
background of similar appearing stars. To remedy this and estimate where
the background begins to take over, we also explore a cut based on the
likelihood ratio of only the CMD and PM components. This is in essence
assuming that the spatial position of a star contains no information on
it's membership probability (a uniform distribution like the background)

\subsection{Caveats}\label{caveats}

The J+24 method was designed to determine high probability members for
spectroscopic followup in particular. Note that we instead care about
retrieving a reliable density profile.

In particular, in fig.~\ref{fig:umi_observed_profiles}, notice that the
PSAT method produces artifically small errorbars even when the density
is \textgreater1dex below the local background. These stars are likely
selecting stars from the statistical MW background consistent with UMi
PM / CMD, recovering the assumed density profile. As a result, the
reliability of faint features in these density profiles is questionable
and a more robust analysis, removing this particular density assumtion,
would be required to more appropriately represent the knowledge of the
density profile as the background begins to dominate.

J+24 do not account for structural uncertainties in dwarfs. This is a
not insignificant source of uncertainty in the derived density profile

We assume constant ellipticity and position angle. Dwarf galaxies, in
reality, are not necessarily smooth and constant.

\section{Comparison and conclusions}\label{comparison-and-conclusions}

To illustrate the differences between each dwarf galaxy, in
fig.~\ref{fig:classical_dwarfs_densities}, we compare Scl, UMi, and Fnx
against exponential and plummer density profiles (\textbf{TODO: state
these somewhere}). While all dwarfs have marginal differences in the
inner regions, each dwarf diverges in the outer regions relative to an
exponential. In particular, while Fnx is underdense, Scl and UMi are
both overdense, approximately fitting a Plummer density profile instead.

In summary, we have used J+24 data to derive the density profiles for
Fornax, Sculptor, and Ursa Minor. In each case, the density profile is
robust against different selection criteria. Both Sculptor (Ursa Minor)
show strong (weak) evidence for deviations from an exponential profile.
We will explore a tidal explanation for these features in this work.

\begin{figure}
\centering
\pandocbounded{\includegraphics[keepaspectratio]{figures/scl_umi_fornax_exp_fit.pdf}}
\caption[Classical dwarf density profiles]{The density profiles of
Sculptor, Ursa Minor, and Fornax compared to Exp2D and Plummer density
profiles. Dwarf galaxies are scaled to the same half-light radius and
density at half-light radius (fit from the inner 3 scale radii
exponential recursively. )}\label{fig:classical_dwarfs_densities}
\end{figure}

\section{Radial velocity modeling}\label{radial-velocity-modeling}

For both Sculptor and Ursa Minor, we construct samples of radial
velocity measurements and cross match each sample to produce estimates
of the total velocity dispersion and line of sight velocity for each
galaxy.

\subsection{Methodology}\label{methodology}

First, we crossmatch all catalogues to J+24 Gaia stars. If a study did
not report GaiaDR3 source ID's, we match to the nearest star within 2
arcseconds. We exclude stars not matched to Gaia for simplicity.

We combine the mean RV measurement from each study using the
inverse-variance weighted mean and standard uncertainty. \[
\bar v = \frac{1}{\sum w_i}\sum_i w_i\ v_{i} \\
\delta v = \sqrt{\frac{1}{\sum_i w_{i}^2}}
\] where \(w_i = 1/s_i^2\), and we estimate the inter-study standard
deviation \[
s^2 = \frac{1}{\sum w_i} \sum_i w_i (v_{i} - \bar v)^2
\] We remove stars with significant velocity dispersions as measured
between or within a study: \[
s < 5\,\delta v\,\sqrt{n}
\] where \(s, \delta v, n\) are the standard deviation, standard, error,
and number of measurements for both a study (with multiple epochs) and
inter-study comparisons.

The combined RV likelihood is then \[
{\cal L} = {\cal L}_{\rm space} {\cal L}_{\rm CMD} {\cal L}_{\rm PM} {\cal L}_{\rm RV}
\] where \[
{\cal L}_{\rm RV, sat} = N(\mu_{v}, \sigma_{v}^2 + (\delta v_i)^2) \\
{\cal L}_{\rm RV, bg} = N(0, \sigma_{\rm halo}^2)
\] where \(\mu_v\) and \(\sigma_v\) are the systemic velocity and
dispersion, and \(\delta v_i\) is the individual measurement
uncertainty. Typically, the velocity dispersion will dominate the
uncertainty budget here. We assume a halo/background velocity dispersion
of a constant \$\sigma\_\{\rm halo\} = 100 \$ km/s
\citep[e.g.][]{brown+2010}.

Similar to above, we retain stars with the resulting membership
probability of greater than 0.2.

Finally, we need to correct the coordinate frames for the solar motion
and on-sky size of the galaxy. The first step is to subtract out the
solar motion from each radial velocity, corresponding to a typical
gradient of \textasciitilde3 km/s across the field. The next step is to
account for the slight differences in the direction of each radial
velocity. Define the \(\hat z\) direction to point parallel to the
direction from the sun to the centre of the dwarf galaxy. Then if
\(\phi\) is the angular distance between the centre of the galaxy and
the individual star, the corrected radial velocity is then \[
v_z = v_{\rm los, gsr}\cos\phi  - v_{\alpha}\cos\theta \sin\phi - v_\delta \sin\theta\sin\phi
\] where
\(v_{\rm tan, R} = d(\mu_{\alpha*}\cos\theta + \mu_\delta \sin \theta)\)
is the radial component of the proper motion with respect to the centre
of the galaxy.This correction is of the order \(v_{\rm tan}\theta\) so
induces a gradient of about \(1 km/s/degree\) for sculptor
\citep[see][]{WMO2008}. The uncertainty is then the velocity uncertainty
plus the distance uncertainties times the PM uncertainty from above. We
then use the \(v_z\) values for the following modelling, however
repeating with plain, solar-frame velocities does not substantially
affect the results too much .

For the priors on the satellite velocity dispersion and systematic
velocity, we use \[
\mu_{v} = N(0, \sigma_{\rm halo}^2) \\
\sigma_{v} = U(0, 20\,{\rm km\,s^{-1}})
\] where \(\sigma_{\rm halo} = 100\,{\rm km\,s^{-1}}\) is the velocity
dispersion of the MW halo adopted above, a reasonable assumption for
dwarfs in orbit around the MW.

\subsection{Sculptor}\label{sculptor}

\begin{table*}[t]
\centering
\caption{Summary of velocity measurements and derived properties.}
\label{tbl:Summary-of-velocity-measurements-and-derived-properties}
\begin{tabular}{llllllll}
\toprule
 & Study & Instrument & Memb & Rep. & N$\sigma > s$ & $\bar v$ & $\sigma_v$\\
\midrule
Scl &  &  &  &  &  &  & \\
 & tolstoy+23 & VLT/FLAMES & 200 (1604) & 500 &  &  & 9.6\\
 & sestito+23a & GMOS & 2000 &  &  &  & -\\
 & walker+09 & MMFS & 2000 &  &  &  & 9.5\\
 & APOGEE & APOGEE & 200 & 200 &  & 8 & 9\\
UMi &  &  &  &  &  &  & \\
 & sestito+23b & GRACES & 6 &  &  &  & \\
 & pace+20 & Keck/DEIMOS &  &  &  & -245 & 8.6\\
 & spencer+18 & MMT/Hectoshell &  &  &  & -247 & \\
 & APOGEE & APOGEE & 100 & 100 &  &  & \\
\bottomrule
\end{tabular}
\end{table*}

For Sculptor, we combine radial velocity measurements from APOGEE,
\citet{sestito+2023a}, \citet{tolstoy+2023}, and \citet{WMO2009}.
\citet{tolstoy+2023} and \citet{WMO2009} provide the bulk of the
measurements. We find that there is no significant velocity shift in
crossmatched stars between catalogues. After crossmatching to Gaia and
excluding significant inter-study dispersions, we have a sample of XXXX
members.

We derive a systemic velocity of \(111.2\pm0.2\) km/s with velocity
dispersion \(9.67\pm0.16\) km/s. Our values are very consistent with
previous work \citep[e.g.][\citet{arroyo-polonio+2024},
\citet{tolstoy+2023}]{WMO2009}.

Finally, we attempt to fit a linear velocity gradient to Scl by adding
parameters for the gradient in \(\xi\) ande \(\eta\). We derive a
gradient of xxx and xxx (see Figure\textasciitilde FIG.), noting that
this direction is different and higher in magnitude than the proper
motion of Scl. Comparing the density difference, we find a bayes factor
of -15. Compared to past work, \citet{battaglia+2008},
\citet{arroyo-polonio+2021}.

\begin{figure}
\centering
\pandocbounded{\includegraphics[keepaspectratio]{figures/scl_vlos_xi_eta.pdf}}
\caption[Scl velocity sample]{A plot of the corrected los velocities for
Scl binned in tangent plane coordinates. We detect a slight rotational
gradient towards the bottom right. \textbf{TODO} add a histogram of
velocities here too with fits, maybe with respect to distance.}
\end{figure}

\begin{figure}
\centering
\pandocbounded{\includegraphics[keepaspectratio]{figures/scl_vel_gradient.pdf}}
\caption[Scl velocity gradient]{A velocity gradient in Sculptor! The
arrow marks the gradient induced by Scl's proper motion on the sky.
\textbf{TODO}: plot of velocity of Scl in direction of gradient.}
\end{figure}

\emph{Is it interesting that the velocity dispersion of Scl seems to
increase significantly with Rell?}

\subsection{Ursa Minor}\label{ursa-minor}

For UMi, we collect radial velocities from, APOGEE,
\citet{sestito+2024b}, \citet{pace+2020}, and \citet{spencer+2018}.

We shifted the velocities of \citet{spencer+2018} (\(-0.9\) km/s) and
\citet{pace+2020} (\(+1.0\) km/s) to reach the same scale based on a
crossmatch of about 200 common stars. Since the mean difference in
velocities in this crossmatch is of order 1.9 km/s, we adopt this as the
systematic LOS velocity error.

UMi interestingly has a more structured observational pattern, but does
not appear to have any significant velocity substructure.

\begin{figure}
\centering
\pandocbounded{\includegraphics[keepaspectratio]{figures/umi_vlos_xi_eta.pdf}}
\caption[UMi velocity sample]{A plot of the corrected los velocities for
UMi binned in tangent plane coordinates. There is no clear rotation or
velocity gradient here. Interestingly, many velocity members are as far
as 100 arcmin away.}
\end{figure}

\subsection{Discussion and
limitations}\label{discussion-and-limitations}

Our model here is relatively simple. Some things which we note as
systematics and are challenging to account fully for are

\begin{itemize}
\tightlist
\item
  Inter-study systematics and biases. While basic crossmatches and a
  simple velocity shift, combining data from multiple instruments is
  challenging.
\item
  Inappropriate uncertainty reporting. Inspection of the variances
  compared to the standard deviations within a study seems to imply that
  errors are accurately reported. APOGEE notes that their RV
  uncertainties are known to be underestimates.
\item
  Binarity. While not too large of a change for classical dwarfs, this
  could inflate velocity dispersions of about 9 km/s by about 1 km/s
  \citet{spencer+2017}. Thus, our measurement is likely slightly
  inflated given the high binarity fractions measured in classical
  dwarfs \citep[\citet{spencer+2018}]{arroyo-polonio+2023}.
\item
  Selection effects. RV studies each have their own selection effects. I
  do not know how to correct for this.
\end{itemize}

Because the derived parameters are similar for the two different larger
surveys we consider for UMi and Scl, we note that many of these effects
are likely not too significant (with the exception of the systemic
motion of UMi.)

\section{Appendix / Extra Notes}\label{appendix-extra-notes}

\subsection{Additional density profile
tests}\label{additional-density-profile-tests}

\begin{figure}
\centering
\pandocbounded{\includegraphics[keepaspectratio]{figures/scl_density_methods_extra.pdf}}
\caption[Density profiles]{Density profiles for various assumptions for
Sculptor. PSAT is our fiducial 2-component J+24 sample, circ is a
2-component bayesian model assuming circular radii, simple is the series
of simple cuts described in Appendix ?, bright is the sample of the
brightest half of stars (scaled by 2), DELVE is a sample of RGB stars
(background subtracted and rescaled to
match).}\label{fig:sculptor_observed_profiles}
\end{figure}

Note that a full rigorous statistical analysis would require a
simulation study of injecting dwarfs into Gaia and assessing the
reliability of various methods of membership and density profiles. This
is beyond the scope of this thesis.

\begin{verbatim}
SELECT TOP 1000
       *
FROM delve_dr2.objects
WHERE 11 < ra
and ra < 19
and -37.7 < dec
and dec < -29.7
\end{verbatim}
