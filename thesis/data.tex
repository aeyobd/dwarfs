\section{Abstract}\label{abstract}

To provide an empirical benchmark for simulations in the next chapters,
we aim to understand the observed properties and density profiles of
Sculptor and Ursa Minor. In this section, we first compile past
measurements of each galaxy. We then describe the \citet{jensen+2024}
Bayesian methodology to select high probability member stars from
\emph{Gaia}. We show the derived density profiles are robust to changes
in assumptions. While Fornax is well represented by a 2D exponential
profile, Sculptor and Ursa Minor show an excess of stars relative to an
2D exponential in the outer regions, better described by a Plummer
profile. Future chapters assess possible explanations for these excess.

\section{Introduction}\label{introduction}

Since the discovery of dwarf galaxies around the Milky Way,
observational work has attempted to measure and refine their basic
properties. While the Milky Way's satellites are close (by extragalactic
standards), their low numbers of stars and large apparent sizes present
challenges for observational work. In particular, the properties of the
faintest dwarf galaxy systems and especially \emph{ambiguous} systems
remains a pressing challenge as a unique probe into the nature of the
early universe and dark matter.

Among local dwarf galaxies, classical systems are perhaps the systems
best studied and with the best derived properties.
Tables~\ref{tbl:scl_obs_props}, \ref{tbl:umi_obs_props} present our
adopted observed properties for Sculptor and Ursa Minor. While extending
1-2 degrees across the sky (e.g.
Figs.~\ref{fig:scl_selection}, \ref{fig:umi_selection}, \ref{fig:fornax_selection}),
classical dwarfs larger number of bright (giant) stars allow for
thousands of stars to be observed with deep photometry and spectroscopy
(e.g. \citet{tolstoy+2023}, \citet{pace+2020}). As a result, the
determination of fundamental properties such as the position, size,
orientation, distance (from 100s of RRL stars, e.g. \citet{tran+2022};
\citet{garofalo+2025}), proper motions (from Gaia, described in next
section), line-of-sight (LOS) velocity, and velocity dispersion
\(\sigma_v\) are relatively well constrained today and show excellent
convergence among different works. However, ongoing research still is
refining our understanding of the detailed structure of Milky Way
satellites, hinting that these objects may be more complex than at first
glance.

A number of works have previously speculated that dwarf galaxies such as
Sculptor and Ursa Minor may have signatures of tidal perturbations,
however not without contention. While \citet{hodge1961} and
\citet{demers+krautter+kunkel1980} were some of the earliest work to
derive the density profile of Sculptor, \citet{innanen+papp1979} was
perhaps the first to speculate that Sculptor may harbour a substantial
population of ``extratidal stars'' (stars beyond the tidal radius),
finding candidate members out to 180' in an elongated distribution from
\citet{vanagt1978}`s catalogue of variable stars.\footnote{Interestingly,
  \citet{innanen+papp1979} also speculate about Uranus's satellite
  distribution, covering gravitational dynamics at two very different
  scales.} Additionally, \citet{eskridge1988} showed a possible excess
of stars in Sculptor relative to a \citet{king1962} and Exponential
density profile beginning around 50', but suggest that this excess may
not be unusual. Later work by \citet{IH1995}, \citet{walcher+2003}, and
\citet{westfall+2006} also showed evidence of an extended component of
Sculptor's density profile, interpreting these stars as either evidence
of tidal debris or a dwarf galaxy halo. On the other hand,
\citet{coleman+dacosta+bland-hawthorn2005} show that Sculptor is well
described by a two component density profile, they additionally mention
that it is unlikely many of Sculptor's stars (less than 2\%) are
extratidal. While \citet{munoz+2018} is the most recent and deepest
photometric study of Sculptor, they only cover an area of out to radii
of about 30', so they are unable to study the possible extended
components of Sculptor detected in past works. In addition to the
presence of ``extratidal stars'', many studies note that Sculptor's
stars appear to become more elliptical with radius, consistent with
tidal effects (\citet{IH1995}; \citet{westfall+2006}).

Ursa Minor was often noted not for a density excess but for unusual
features. Starting with the first density profile from
\citet{hodge1964}, Ursa Minor is known to be highly elliptical. Studies
by \citet{olszewski+aaronson1985}, \citet{demers+1995}, \citet{IH1995},
\citet{kleyna+1998}, and \citet{bellazzini+2002} note that Ursa Minor
appears to contain substructure along its major axis, but without strong
interpretation for the causes. As one interpretation \citet{kleyna+2003}
suggest that this feature is a long-lived star cluster residing in a
cored dark matter halo to be dynamically stable. One of the first direct
suggestions of tidal features came from \citet{martinez-delgado+2001},
who find that stars extend far beyond the nominal tidal radius for Ursa
Minor in the direction of the galaxies elongation, interpreting this as
evidence for tidal effects. \citet{palma+2003} corroborate many of these
earlier findings, showing further evidence for Ursa Minor's peculiar
morphology including S-shaped contours, a possible extended extratidal
population of stars, and a failure for a King density profile to
adequately capture the actual stellar distribution. Each of these
characteristics appear to indicate strong tidal disruption. Using
velocity observations, \citet{pace+2014} additionally show that there
are multiple components in spatial-velocity information. However,
\citet{munoz+2018}'s more modern photometry show a more regular
distribution of stars. Irregardless, Ursa Minor has had enough work
suggesting peculiarities that a deeper investigation into the
possibility of tidal effects is worthwhile.

However, everything discussed so far is before \emph{Gaia}. Thus, the
knowledge of the orbits of these systems is largely unknown and most
surveys could only filter out stars using photometry. As a result, the
tidal radius could only be guessed based on either the density profile
or current distance of the dwarf galaxy. With \emph{Gaia} (discussed
more below), recent work has used Bayesian frameworks to derive
systematic proper motions of many dwarf galaxies (e.g. \citet{MV2020a})
and filter away foreground stars using proper motions and parallax to
detect distant members, and study the 6D internal kinematics of these
galaxies (e.g. \citet{tolstoy+2023}). Most relevant here,
\citet{jensen+2024} present a bayesian algorithm to determine likely
members in \emph{Gaia} (described below), and detect extended secondary
components for 9 dwarf galaxies, including Sculptor and Ursa Minor.
\citet{sestito+2023a} and \citet{sestito+2023b} followed up a few of the
most distant stars detected in \citet{jensen+2024} confirming that
members exist in each galaxy as far ass 9-12 \(R_h\) from the centres.
In this chapter, we discuss the origin and reliability of these
detections, confirming that the density profiles are indeed robust.
These density profiles then provide the scientific motivation to explore
the tidal interpretation in the next chapters.

\emph{Maybe define \(R_{\rm ell}\)} \emph{and} \(\xi, \eta\)
\emph{here}? \emph{We will discuss multiple populations in future
chapters}

\begin{table*}[t]
\centering
\caption[Observed Properties of Sculptor]{Observed properties of Sculptor. References are: 1. Ricardo R. Muñoz et al. (2018) Sérsic fits, 2. Tran et al. (2022), 3. McConnachie and Venn (2020b), 4. Arroyo-Polonio et al. (2024), 5. McConnachie and Venn (2020a). }
\label{tbl:scl_obs_props}
\begin{tabular}{lll}
\toprule
parameter & value & Source\\
\midrule
$\alpha$ & $15.0183 \pm 0.0012^\circ$ & 1\\
$\delta$ & $-33.7186 \pm 0.0007^\circ$ & 1\\
distance modulus & $19.60 \pm 0.05$ (RR lyrae) & 2\\
distance & $83.2 \pm 2$ kpc & 2\\
$\mu_{\alpha*}$ & $0.099 \pm 0.002 \pm 0.017$ mas yr$^{-1}$ & 3\\
$\mu_\delta$ & $-0.160 \pm 0.002_{\rm stat} \pm 0.017_{\rm sys}$ mas yr$^{-1}$ & 3\\
LOS velocity & $111.2 \pm 0.3\ {\rm km\,s^{-1}}$ & 4\\
$\sigma_v$ & $9.7\pm0.2\ {\rm km\,s^{-1}}$ & 4\\
$R_h$ & $9.79 \pm 0.04$ arcmin & 1\\
ellipticity & $0.37 \pm 0.01$ & 1\\
position angle & $94\pm1^\circ$ & 1\\
$M_V$ & $-10.82\pm0.14$ & 1\\
\bottomrule
\end{tabular}
\end{table*}

\begin{table*}[t]
\centering
\caption[Observed Properties of Ursa Minor]{Observed properties of Ursa Minor. References are: (1) Ricardo R. Muñoz et al. (2018) Sérsic fits, (2) Garofalo et al. (2025), (3) McConnachie and Venn (2020a), (4) Pace et al. (2020), average of MMT and Keck results. }
\label{tbl:umi_obs_props}
\begin{tabular}{lll}
\toprule
parameter & value & Source\\
\midrule
$\alpha$ & $ 227.2420 \pm 0.0045$˚ & 1\\
$\delta$ & $67.2221 \pm 0.0016$˚ & 1\\
distance modulus & $19.23 \pm 0.11$ (RR lyrae) & 2\\
distance & $70.1 \pm 3.6$ kpc & 2\\
$\mu_\alpha*$ & $-0.124 \pm 0.004 \pm 0.017$ mas yr$^{-1}$ & 3\\
$\mu_\delta$ & $0.078 \pm 0.004_{\rm stat} \pm 0.017_{\rm sys}$ mas yr$^{-1}$ & 3\\
LOS velocity & $-245.9 \pm 0.3_{\rm stat} \pm 1_{\rm sys}$ km s$^{-1}$ & 4\\
$\sigma_v$ & $8.6 \pm 0.3$ & 4\\
$R_h$ & $11.62 \pm 0.1$ arcmin & 1\\
ellipticity & $0.55 \pm 0.01$ & 1\\
position angle & $50 \pm 1^\circ$ & 1\\
$M_V$ & $-9.03 \pm 0.05$ & 1\\
\bottomrule
\end{tabular}
\end{table*}

\subsection{\texorpdfstring{The \emph{Gaia}
mission}{The Gaia mission}}\label{the-gaia-mission}

Some of the most fundamental properties of astronomical objects are
their positions and velocities. Unfortunately, determining distances to
stars is nontrivial. Additionally, while line-of-sight (LOS) velocities
are easily determined from spectroscopy, the tangental velocities,
perpendicular to the LOS, are best measured through proper motions,
typically requiring accurate determinations of small changes in a star's
position. \emph{Gaia}'s mission is to produce extremely precise
astrometry enabling measurements of unprecedented accuracy and scale for
proper motions, parallaxes, and magnitudes.

\emph{Gaia} was designed to revolutionize proper motion and parallax
measurements. \emph{Gaia} is a space-based, all-sky survey telescope
with two primary 1.45x0.5m mirrors situated at the Sun-Earth L2 Lagrange
point \citep{gaiacollaboration+2016}. \emph{Gaia} was launched in 2013,
completing its mission in 2025 (but with two more data releases
planned). By imagining two patches of sky on the same focal plane,
separated by a fixed angle of 106.5 degrees, \emph{Gaia} is able to
measure absolute proper motions. Stars in different regions of the sky
are affected by parallax differently, so by observing the relative
positions of stars separated by large angles during multiple epochs over
a year, an absolute all-sky reference frame can be derived
\citep{gaiacollaboration+2016}. In addition to precise astrometric
information, \emph{Gaia} measures the magnitude of stars in the very
wide \emph{G} band (330-1050nm), blue and red colours using the blue and
red photometers (BP and RP, 330-680, 640-1050 respectively), and takes
low resolution BP-RP spectra and radial velocity measurements of bright
stars (\emph{Gaia} radial velocity magnitudes \textless16)
\citep{gaiacollaboration+2016}. For our purposes, most useful are
\emph{Gaia}'s measurements of \(G\) magnitude,
\(G_{\rm BP} - G_{\rm RP}\) colour, the position \(\alpha, \delta\), and
the proper motions in RA and declination \(\mu_{\alpha*}\) and
\(\mu_\delta\) (where \(\mu_{\alpha*} = \mu_\alpha \cos \delta\)
corrects for projection effects).

\emph{Gaia} has revolutionized many astronomical disciplines, the least
of which is local group and Milky Way science. While proper motions of a
dwarf galaxies has been measured in a case by case basis by the Hubble
Space Telescope, a full systematic determination of proper motions for
most dwarf galaxies was unavailable until \emph{Gaia}
\citep{MV2020a, battaglia+2022}. \emph{Gaia} has furthermore allowed for
the detection and improved measurements of halo substructures
(CITATIONS), streams (CITAITONS), and Milky Way structure and
kinematics, among many other applications.

\section{Gaia Membership Selection}\label{gaia-membership-selection}

\subsection{Quality cuts and membership
model}\label{quality-cuts-and-membership-model}

Here, we briefly describe \citet{jensen+2024}'s (hereafter J+24)
membership estimation method. J+24 use a Bayesian framework
incorporating proper motion (PM), colour-magnitude diagram (CMD), and
spatial information to determine the probability that a given star
belongs to the satellite. By accounting for PM in particular, J+24
produces low contamination samples of candidate member stars out to
large distances from a dwarf galaxy. J+24 extends the algorithm
presented in \citet{MV2020a}; \citet{MV2020b} but additionally includes
an optional secondary, extended spatial component to find possible
members as far as \textasciitilde10 half-light radii \(R_h\) from
several dwarf galaxies. See also similar work by \citet{pace+li2019};
\citet{battaglia+2022}; \citet{pace+erkal+li2022}; \citet{qi+2022}.

To create a high-quality sample, J+24 select stars initially from Gaia
within a 2--4 degree circular region centred on the dwarf satisfying:

\begin{itemize}
\tightlist
\item
  Solved astrometry, magnitude, and colour.
\item
  Renormalized unit weight error, \({\rm ruwe} \leq 1.3\), ensuring high
  quality astrometry. \texttt{ruwe} is a measure of the excess
  astrometric noise on fitting a consistent parallax-proper motion
  solution \citep[see][]{lindegren+2021}.\\
\item
  3\(\sigma\) consistency of measured parallax with dwarf's distance
  (dwarf parallax is very small; with \citet{lindegren+2021} zero-point
  correction).
\item
  Absolute proper motions, \(\mu_{\alpha*}\), \(\mu_\delta\), less than
  10\(\,{\rm mas\ yr^{-1}}\). (Corresponds to tangental velocities of
  \(\gtrsim 500\) km/s at distances larger than 10 kpc.)
\item
  Corrected colour excess is within 3\(\sigma\) of the expected
  distribution from \citet{riello+2021}. Removes stars with unreliable
  photometry.
\item
  De-reddened \(G\) magnitude is between
  \(22 > G > G_{\rm TRGB} - 5\sigma_{\rm DM}\). Removes very faint stars
  and stars significantly brighter than the tip of the red giant branch
  (TRGB) magnitude plus the distance modulus uncertainty
  \(\sigma_{\rm DM}\).
\item
  Colour is between \(-0.5 < {\rm BP - RP} <  2.5\) (dereddened).
  Removes stars substantially outside the expected CMD.
\end{itemize}

Photometry is dereddened with \citet{schlegel+finkbeiner+davis1998}
extinction maps.

J+24 define likelihoods \({\cal L}\) representing the probability
density that a star is consistent with either the MW stellar background
(\({\cal L}_{\rm bg}\)) or the satellite galaxy
(\({\cal L}_{\rm sat}\)). In either case, the likelihoods are the
product of a spatial, PM, and CMD term: \begin{equation}{
{\cal L} = {\cal L}_{\rm space}\ {\cal L}_{\rm PM}\ {\cal L}_{\rm CMD}.
}\end{equation}

Each likelihood is normalized over their respective 2D parameter space
for both the satellite. To control the relative frequency of member and
background stars, \(f_{\rm sat}\) representing the fraction of member
stars in the field. The total likelihood for any star in this model is
the sum of the satellite and background likelihoods, weighted by their
relative frequencies,
\begin{equation}\protect\phantomsection\label{eq:Ltot}{
{\cal L}_{\rm tot} = f_{\rm sat}{\cal L}_{\rm sat} + (1-f_{\rm sat}){\cal L}_{\rm bg}.
}\end{equation} The probability that any star belongs to the satellite
is then given by \begin{equation}{
P_{\rm sat} = 
\frac{f_{\rm sat}\,{\cal L}_{\rm sat}}{{\cal L}_{\rm tot}}
= \frac{f_{\rm sat}{\cal L}_{\rm sat}}{f_{\rm sat}{\cal L}_{\rm sat} + (1-f_{\rm sat}){\cal L}_{\rm bg}}.
}\end{equation}

For the satellite's spatial likelihood, J+24 consider both one-component
and a two-component density models. The one component model is
constructed as a single exponential profile ( surface density
\(\Sigma \propto e^{R_{\rm ell} / R_s}\)), with scale radius \(R_s\)
fixed to the value in table 1 of \citet{MV2020a} from \citet{munoz+2018}
(for a Sérsic fit). Additionally, structural uncertainties (for position
angle, ellipticity, and scale radius) are sampled over to construct the
final likelihood map. The two-component model instead adds a second
exponential,
\(\Sigma_\star \propto e^{-R/R_s} + B\,e^{-R/R_{\rm outer}}\). The inner
scale radius is fixed, and the outer scale radius and magnitude of the
second component \(R_{\rm outer}\), \(B\) are free parameters.
Structural property uncertainties are not included in the two-component
model.

The PM likelihood is a bivariate gaussian with variance and covariance
equal to each star's proper motions. J+24 assume the stellar PM errors
are the main source of uncertainty.

The satellite's CMD likelihood is based on a Padova isochrone
\citep{girardi+2002}. The isochrone has a matching metallicity and 12
Gyr age (except 2 Gyr is used for Fornax). The (gaussian) colour width
is assumed to be 0.1 mag plus the Gaia colour uncertainty at each
magnitude. The horizontal branch is modelled as a constant magnitude
extending blue of the CMD (mean magnitude of -2.2, 12 Gyr HB stars and a
0.1 mag width plus the mean colour error). A likelihood map is
constructed by sampling the distance modulus in addition to the CMD
width, taking the maximum of RGB and HB likelihoods.

The background likelihoods are instead empirically constructed. Stars
stars outside of 5\(R_h\) passing the quality cuts estimate the
background density in PM and CMD space. The density is a sum of
bivariate gaussians with variances based on Gaia uncertainties (and
covariance for proper motions). The spatial background likelihood is
assumed to be constant.

J+24 derive \(\mu_{\alpha*}\), \(\mu_\delta\), \(f_{\rm sat}\) (and
\(B\), \(R_{\rm outer}\) for two-component) through an MCMC simulation
with likelihood from Eq.~\ref{eq:Ltot}. Priors are only weakly
informative. The proper motion single component prior is same as
\citet{MV2020a}: a normal distribution with mean 0 and standard
deviation \(100\ \kms\). If 2-component spatial, instead is a uniform
distribution spanning 5\(\sigma\) of single component case w/ systematic
uncertainties. \(f_{\rm sat}\) (and \(B\)) has a uniform prior 0--1.
\(R_{\rm outer}\) has a uniform prior only restricting
\(R_{\rm outer} > R_s\). The mode of each parameter from the MCMC are
then reported and used to calculate the final \(P_{\rm sat}\) values.

\subsection{Selected samples}\label{selected-samples}

In
Figs.~\ref{fig:scl_selection}, \ref{fig:umi_selection}, \ref{fig:fornax_selection},
we illustrate the resulting samples from the algorithm in the tangent
plane, CMD, and proper motion space. The tangent plane coordinates
\(\xi\) and \(\eta\) are the distances in RA and declination as measured
on a plane tangent to the dwarf galaxies centre.

For our fiducial sample, we adopt a probability cut of
\(P_{\rm sat} = 0.2\). Most stars are assigned probabilities close to
either 0 or 1, so the choice of probability threshold is only marginally
significant. Additionally, even for a probability cut of 0.2, the purity
of the resulting sample with RV measurements is very high
(\textasciitilde90\%, J+24). Note there is likely a systematic bias in
using stars with RV measurements to measure purity. Fainter stars are
less likely to have been targeted and have poorer astrometry and colour
information. We find no difference in the resulting density
distributions when restricting stars to be brighter than a specific
magnitude.

For all galaxies, each criteria is commensurate in sifting out
nonmembers. The CMD is well defined, and probable members only extend a
few times the colour uncertainty from the CMD. In proper motion space,
selected stars are within \(\masyr\) from the systematic proper motion.
The stars furthest away in proper motion space typically have large
uncertainties \(\sim 1\masyr\), so most members are reasonably
consistent. In any case, the algorithm removes stars with inconsistent
PMs. Finally, the spatial likelihood reduces the probabilities of stars
distant from the dwarfs centres. Member stars are rare outside of
\(\sim5R_h\), resulting from both the likelihood specification and the
lower density of background stars consistent with the CMD and PM of the
satellite.

One limitation of the J+24 method is that it assumes a spatial form of
the dwarf galaxy, possibly limiting the detection of very distant stars
or features. Figs.~\ref{fig:scl_selection}, \ref{fig:umi_selection} also
illustrates the distribution of stars selected without a spatial
criterion. We define the \emph{CMD+PM} selection as
\begin{equation}\protect\phantomsection\label{eq:sel_cmd_pm}{
{\cal L}_{\rm CMD,\ sat}\ {\cal L}_{\rm PM,\ sat} > {\cal L}_{\rm CMD,\ bg}\ {\cal L}_{\rm PM,\ bg}
}\end{equation} with the likelihoods from J+24 as described above. These
stars are distributed similar to the fiducial (probable members) sample
but with a background uniform distribution across the entire field. This
illustrates the approximate background of stars which may be confused as
members. Additionally, since there is no clear spatial structure in the
\emph{CMD+PM} (or \emph{all}) sample outside several \(R_h\), it is
unlikely that there are additional faint, tidal features detectable with
\emph{Gaia}. Not shown here, we also try a variety of simpler, absolute
cuts and thresholds, finding no evidence of other structure in both
\emph{Gaia} and DELVE/UNIONS data.

Finally, we illustrate the location of RV-confirmed members from
Section~\ref{sec:rv_obs} in each panel. Because RV targets tend to be
brighter than a typical \emph{Gaia} candidate, RV members typically have
more precise PMs. However, the RV members fill about the same area on
the CMD down to the magnitude limit. Additionally, while the number of
stars with RV measurements decreases at large radii, there are still
confirmed RV members as far as \(\sim 10R_h\) for both galaxies. These
stars illustrate the extended profiles of each galaxy. Note that for an
exponential profile, 99.95\% of stars should be within \(6R_h\), so the
discovery of any stars beyond \(6R_h\) hints at a deviation from an
exponential profile. These \emph{extratidal} stars hint at something
tidal in nature, and our goal is to explore possible interpretations.

\begin{figure}
\centering
\pandocbounded{\includegraphics[keepaspectratio]{figures/scl_selection.pdf}}
\caption[Sculptor sample selection]{The selection criteria for Scl
members. Light grey points represent field stars (satisfying quality
criteria), turquoise points CMD+PM selected stars
(Eq.~\ref{eq:sel_cmd_pm}), blue xs probable members (2-component), and
RV confirmed members indigo diamonds. We mark the two stars from
\citet{sestito+2023a} with rust-outlined indigo stars. \textbf{Top:}
Tangent plane \(\xi, \eta\). The orange ellipse represents 3 half-light
radii. \textbf{Bottom left:} Colour magnitude diagram in Gaia \(G\)
magnitude versus \(G_{\rm BP} - G_{\rm RP}\) colour. We plot the Padova
12Gyr {[}Fe/H{]}=-1.68 isochone in orange. The black bar in the top left
represents the median colour error. \textbf{Bottom right:} Proper motion
in declination \(\mu_\delta\) vs RA \(\mu_{\alpha*}\) (corrected) the
orange circle represents the \citet{MV2020b} proper motion. The black
cross represents the median proper motion
error.}\label{fig:scl_selection}
\end{figure}

\begin{figure}
\centering
\pandocbounded{\includegraphics[keepaspectratio]{figures/umi_selection.pdf}}
\caption[Ursa Minor sample selection]{Similar to
Fig.~\ref{fig:scl_selection} except for Ursa Minor. We outline RV
members outside of \(6R_h\) in black stars (from \citet{sestito+2023b},
\citet{pace+2020} and \citet{spencer+2018}).}\label{fig:umi_selection}
\end{figure}

\begin{figure}
\centering
\pandocbounded{\includegraphics[keepaspectratio]{figures/fornax_selection.pdf}}
\caption[Fornax sample selection]{Similar to
Fig.~\ref{fig:scl_selection} except for Fornax. RV members are from
\citet{WMO2009}. While possibly a limitation of RV sample selection,
Fornax does not show the same extended outer halo of probable members as
Sculptor or Ursa Minor despite having many more
stars.}\label{fig:fornax_selection}
\end{figure}

\subsection{Density Profiles}\label{density-profiles}

Density profiles are an essential observational constraint for our later
simulations. To derive density profiles, we use 0.05 dex bins in log
radius. We remove bins at smaller (larger) radii than the first empty
bin working outwards. We use symmetric poisson uncertainties. As
discussed below, these uncertainties are straightforward but are likely
under-representative. We use the values from \citet{munoz+2018}'s Sérsic
maximum likelihood fits for \(R_h\), which are more precisely derived
given their deeper photometry.

Fig.~\ref{fig:scl_observed_profiles} show the derived density profiles
for Sculptor, Ursa Minor, and Fornax. We calculate density profiles for
different selections of stars from above: all quality stars, CMD+PM
only, and the probable member samples. In each case, all samples are the
same towards the inner regions of the satellite. Classical dwarfs
dominate the stellar density in their cores. However, the density
profile \emph{all stars} plateaus at the total background in the field
at radii of 30-60 arcminutes. By restricting stars to being most likely
satellite members by CMD + PM, the background is reduced by 1-2 dex. The
CMD+PM background plateau likely represents the density of background
stars which could be mistaken as members. Finally, the background
(BG)-subtracted profile results from subtracting the apparent background
in the \emph{all} profile.

Note that the probable members (fiducial) density profile continues to
confidently estimate the density profile below the CMD+PM background.
These points are likely unreliable (see discussion below). However,
before this point, both the BG subtracted and probable members density
profiles are strikingly similar. Assumptions about the details of the
likelihood and spatial dependence have marginal influence on the
resulting density profile when the satellite is higher density than the
background. Thus the detection of excesses of stars in Sculptor (past 20
arcmin) and UMi (past 15-20 arcmin) are robust.

Nearby to UMi, there is a small, likely unassociated, ultrafaint star
cluster, Muñoz 1 \citep{munoz+2012}. The cluster is at a relative
position of \((\xi, \eta) \approx(-42, -15)\) arcminutes, corresponding
to an elliptical radius of 36 arcminutes. However this cluster does not
have a bright RGB, so has few stars brighter than a \(G\) mag of 22.
While Muñoz 1 may contribute \(\sim\) 5 stars to the density profile of
UMi, this would possibly only affect 1-2 bins due to its compact size.

To illustrate the differences between each dwarf galaxy, in
Fig.~\ref{fig:classical_dwarfs_densities}, we compare Sculptor, Ursa
Minor, and other classical dwarfs against 2D-exponential and Plummer
density profiles (REF). All dwarf galaxies appear to be well described
by an exponential profile in the inner regions. However, while an
exponential continues to be a good description for other dwarf galaxies,
Sculptor and Ursa Minor diverge, showing a substantial excess over an
exponential, better fit by a Plummer instead. In fact, Sculptor and Ursa
Minor show excesses of between 2-3 orders of magnitude beyond what is
expected of an exponential at the very outer regions!

\begin{figure}
\centering
\pandocbounded{\includegraphics[keepaspectratio]{figures/scl_umi_fnx_density_methods.pdf}}
\caption[Sculptor density profiles]{The density profile of Sculptor,
Ursa Minor, and Fornax for different selection criteria, ploted as log
surface dencity versus log elliptical radius. \emph{CMD+PM} select stars
more likely to be members according to CMD and PM only, \emph{all}
selects any high quality star, and \emph{BG subtracted} is the
background-subtracted density derived from \emph{all} stars. We mark the
half-light radius (vertical dashed line) and the break radius (black
arrow, REF).}\label{fig:scl_observed_profiles}
\end{figure}

\begin{figure}
\centering
\pandocbounded{\includegraphics[keepaspectratio]{figures/classical_dwarf_profiles.pdf}}
\caption[Classical dwarf density profiles]{The density profiles of
Sculptor, Ursa Minor, and other classical dwarfs compared to Exp2D and
Plummer density profiles. Dwarf galaxies are scaled to the same
half-light radius and density at half-light radius. Sculptor and Ursa
Minor have an excess of stars in the outer regions (past
\(\log R/R_h \sim 0.3\)) compared with other classical
dwarfs.}\label{fig:classical_dwarfs_densities}
\end{figure}

\subsection{Caveats}\label{caveats}

The J+24 method was designed in particular to detect the presence of a
density excess and find individual stars at large radii to be followed
up. We are more interested in accurately quantifying the density profile
and size of any perturbations. One potential problem with using J+24's
candidate members is that the algorithm assumes the density is either
described by a single or double exponential. If this model does not
accurately match the actual density profile of the dwarf galaxy, we want
to understand the impact of this assumption.

In particular, in Fig.~\ref{fig:scl_observed_profiles}, notice that the
\(P_{\rm sat}\) selection method produces small errorbars, even when the
density is more than 1 dex below the local background. These stars are
likely selecting stars from the statistical MW background consistent
with UMi PM / CMD, recovering the assumed density profile. As a result,
the reliability of these density profiles below the CMD+PM background
may be questionable. A more robust analysis, removing this particular
density assumption, would be required to more appropriately represent
the knowledge of the density profile as the background begins to
dominate.

J+24 do not account for structural uncertainties in dwarfs for the two
component case. We assume constant ellipticity and position angle. Dwarf
galaxies, in reality, are not necessarily smooth or have constant
ellipticity. J+24 test an alternative method using circular radii for
the extended density component, and we find these density profiles are
very similar to the fully elliptical case, even when assuming circular
bins for the circular outer component. As such, even reducing the
assumed ellipticity from \(0.37-0.55\) to 0 does not substantially
impact the density profiles.

Finally, our uncertainties are likely underrepresented. A more self
consistent model would fit the density profile to the entire field at
once, eliminating possible misrepresentation of the uncertainties. In
the Appendix to this section, we \emph{will} also discuss an alternative
method which runs a MCMC model using the likelihoods above to solve for
the density in each elliptical bin. From these tests, we note that the
density profile and uncertainties derived from the J+24 sample are
reliable insofar as the dwarf density is above the background from MW
CMD+PM-consistent interlopers. We estimate that this effect comprimises
the density profiles past \(\log R / {\rm arcmin} = 1.8\) for Sculptor
and Ursa Minor, but agreement is good before then.

While \emph{Gaia} has shown excellent performance, some notable
limitations may introduce problems in our interpretation and reliability
of density profiles. Gaia systematics in proper motions and parallaxes
are typically smaller than the values for sources of magnitudes
\(G\in[18,20]\). Since we use proper motions and parallaxes as general
consistency with the dwarf, and factor in systematic uncertainties in
each case, these effects should not be too significant. However, the
systematic proper motion uncertainties becomes the dominant source of
uncertainty in the derived systemic proper motions of each galaxy (see
Tables~\ref{tbl:scl_obs_props}, \ref{tbl:umi_obs_props}).

\emph{Gaia} shows high but imperfect completeness, particularly showing
limitations in crowded fields and for faint sources (\(G\gtrapprox20\)).
As discussed in \citet{fabricius+2021}, for the high stellar densities
in globular clusters, the completeness relative to HST varies
significantly with the stellar density. However, the typical stellar
densities of dwarf galaxies are much lower, at about 20 stars/arcmin =
90,000 stars / degree, lower than the lowest globular cluster densities
and safely below the crowding limit of 750,000 objects/degree for BP/RP
photometry. In \citet{fabricius+2021}, for the lowest density globular
clusters, the completeness down to \(G\approx 20\) is \(\sim 80\%\).
Closely separated stars pose problems for Gaia's on-board processing, as
the pixel size is 59x177 mas on the sky. This results in a reduction of
stars separated by less than 1.5'' and especially for stars separated by
less than 0.6 arcseconds. The astrometric parameters of closely
separated stars furthermore tends to be of lower quality
\citep{fabricius+2021}. However, even for the denser field of Fornax,
only about 3\% of stars have a neighbour within 2 arc seconds, so
multiplicity should not affect completeness too much (except for
unresolved binaries). One potential issue is that the previous analyses
do not account for our cuts on quality and number of astrometric
parameters. These could worsen completeness, particularly since the
BP-RP spectra are more sensitive to dense fields. In the appendix to
this section, we test if magnitude cuts impact the resulting density
profiles, finding that this is likely not an issue.

Finally, an excellent test of systematics and methods in \emph{Gaia} is
to compare against another survey. In the appendix, we show that
profiles derived from J+24 agree with DELVE or UNIONS data within
uncertainties. Systematics, completion effects, and selection methods
are unlikely to substantially change the density profiles presented
here.

\section{Discussion and Conclusions}\label{discussion-and-conclusions}

In summary, we have used J+24 data to derive the density profiles for
the classical dwarf galaxies. In each case, the density profile is
robust against different selection criteria. Both Sculptor and Ursa
Minor show strong evidence for deviations from an exponential profile.
We find no evidence of additional (velocity or stellar) substructure in
either galaxy. Our density profiles are consistent with many previous
works, all indicating divergence from a King or Exponential density
profile with a break between 30 and 60 arcminutes. Our goal in the
following chapters is to test if tides provide a viable explanation for
the observed properties of Sculptor and Ursa Minor.

Are dwarf galaxies indeed expected to be one-component exponential-like
density profiles? The suggestion of exponential density profiles dates
back to \citet{faber+lin1983}. Expand\ldots{}

\section{Additional density profile
tests}\label{additional-density-profile-tests}

In this section, we discuss additional tests and verification of the
derived density profiles. In particular, we check that methodology
(simpler cuts, circularized radii, algorithm version) do not
substantially affect the density profile. We also compile density
profiles presented in the literature as reference. In all cases, the
density profiles appear to have excellent convergence out to
\(\log R_{\rm ell} / {\rm arcmin} \approx 1.8\), about the distance
where the background dominates.

Discuss selection criteria for DELVE and UNIONS samples, literature
comparison, simple selection criteria, MCMC density profiles and when
\citet{jensen+2024} becomes background-limited.

\begin{figure}
\centering
\pandocbounded{\includegraphics[keepaspectratio]{figures/scl_density_methods_extra.pdf}}
\caption[Density profiles]{Density profiles for various assumptions for
Sculptor. PSAT is our fiducial 2-component J+24 sample, circ is a
2-component bayesian model assuming circular radii, simple is the series
of simple cuts described, bright is the sample of the brightest half of
stars (scaled by 2), DELVE is a sample of RGB stars (background
subtracted and rescaled to match).}\label{fig:scl_density_extras}
\end{figure}

\begin{figure}
\centering
\pandocbounded{\includegraphics[keepaspectratio]{figures/scl_density_methods_j24.pdf}}
\caption[Density profiles]{Comparison of density profiles for each J+24
method. The fiducial is a 2-component elliptical model. However, the
1-component is still elliptical but only contains 1 component and the
circular model assumes a circular outer density profile and bins in
circular bins instead of elliptical
bins.}\label{fig:scl_density_j24_methods}
\end{figure}

\begin{figure}
\centering
\pandocbounded{\includegraphics[keepaspectratio]{figures/umi_density_methods_extra.pdf}}
\caption[UMi Density profiles]{Similar to
Fig.~\ref{fig:scl_observed_profiles} except for Ursa
Minor}\label{fig:umi_density_extras}
\end{figure}

\begin{figure}
\centering
\pandocbounded{\includegraphics[keepaspectratio]{figures/umi_density_methods_j24.pdf}}
\caption[UMi density methods]{Similar to
Fig.~\ref{fig:scl_density_j24_methods} except for Ursa
Minor.}\label{fig:umi_density_j24_methods}
\end{figure}
