To provide an empirical benchmark for simulations in the next chapters,
we aim to understand the observed properties and density profiles of
Sculptor and Ursa Minor. In this section, we first compile past
measurements of each galaxy. We then describe the \citet{jensen+2024}
Bayesian methodology to select high probability member stars from
\emph{Gaia}. We show the derived density profiles are robust to changes
in assumptions. While Fornax is well represented by a 2D exponential
profile, Sculptor and Ursa Minor show an excess of stars relative to an
2D exponential in the outer regions, better described by a Plummer
profile. Future chapters assess possible explanations for these excess.

\section{\texorpdfstring{\emph{Gaia} data}{Gaia data}}\label{gaia-data}

Here, we briefly describe \citet{jensen+2024}'s (hereafter J+24)
membership estimation method. J+24 use a Bayesian framework
incorporating proper motion (PM), colour-magnitude diagram (CMD), and
spatial information to determine the probability that a given star
belongs to the satellite. By accounting for PM in particular, J+24
produces low contamination samples of candidate member stars out to
large distances from a dwarf galaxy. J+24 extends the algorithm
presented in \citet{MV2020a}; \citet{MV2020b} but additionally includes
an optional secondary, extended spatial component to find possible
members as far as \textasciitilde10 half-light radii \(R_h\) from
several dwarf galaxies. See also similar work by \citet{pace+li2019};
\citet{battaglia+2022}; \citet{pace+erkal+li2022}; \citet{qi+2022}.

J+24 define likelihoods \({\cal L}\) representing the probability
density that a star is consistent with either the MW stellar background
(\({\cal L}_{\rm bg}\)) or the satellite galaxy
(\({\cal L}_{\rm sat}\)). In either case, the likelihoods are the
product of a spatial, PM, and CMD term: \begin{equation}{
{\cal L} = {\cal L}_{\rm space}\ {\cal L}_{\rm PM}\ {\cal L}_{\rm CMD}.
}\end{equation}

Each likelihood is normalized over their respective 2D parameter space
for both the satellite. To control the relative frequency of member and
background stars, \(f_{\rm sat}\) representing the fraction of member
stars in the field. The total likelihood for any star in this model is
the sum of the satellite and background likelihoods, weighted by their
relative frequencies, \begin{equation}{
{\cal L}_{\rm tot} = f_{\rm sat}{\cal L}_{\rm sat} + (1-f_{\rm sat}){\cal L}_{\rm bg}.
}\end{equation} The probability that any star belongs to the satellite
is then given by \begin{equation}{
P_{\rm sat} = 
\frac{f_{\rm sat}\,{\cal L}_{\rm sat}}{{\cal L}_{\rm tot}}
= \frac{f_{\rm sat}{\cal L}_{\rm sat}}{f_{\rm sat}{\cal L}_{\rm sat} + (1-f_{\rm sat}){\cal L}_{\rm bg}}.
}\end{equation}

\section{Selected stars}\label{selected-stars}

In
Figs.~\ref{fig:scl_selection}, \ref{fig:umi_selection}, \ref{fig:fornax_selection},
we illustrate the resulting samples from the algorithm in the tangent
plane, CMD, and proper motion space. The tangent plane coordinates
\(\xi\) and \(\eta\) are the distances in RA and declination as measured
on a plane tangent to the dwarf galaxies centre. \(R_{\rm ell}\) is
defined as the circularized elliptical radius, so
\(R_{\rm ell}^2 = a\,b\,({\xi'}^2 / a^2 + {\eta'}^2 / b^2)\) where
\(\xi'\) and \(\eta'\) are rotated to align with the position angle
direction \(\theta\).

For our fiducial sample, we adopt a probability cut of
\(P_{\rm sat} = 0.2\). Most stars are assigned probabilities close to
either 0 or 1, so the choice of probability threshold is only marginally
significant. Additionally, even for a probability cut of 0.2, the purity
of the resulting sample with RV measurements is very high
(\textasciitilde90\%, J+24). Note there is likely a systematic bias in
using stars with RV measurements to measure purity. Fainter stars are
less likely to have been targeted and have poorer astrometry and colour
information. We find no difference in the resulting density
distributions when restricting stars to be brighter than a specific
magnitude.

For all galaxies, each criteria is commensurate in sifting out
nonmembers. The CMD is well defined, and probable members only extend a
few times the colour uncertainty from the CMD. In proper motion space,
selected stars are within \(\masyr\) from the systematic proper motion.
The stars furthest away in proper motion space typically have large
uncertainties \(\sim 1\masyr\), so most members are reasonably
consistent. In any case, the algorithm removes stars with inconsistent
PMs. Finally, the spatial likelihood reduces the probabilities of stars
distant from the dwarfs centres. Member stars are rare outside of
\(\sim5R_h\), resulting from both the likelihood specification and the
lower density of background stars consistent with the CMD and PM of the
satellite.

One limitation of the J+24 method is that it assumes a spatial form of
the dwarf galaxy, possibly limiting the detection of very distant stars
or features. Figs.~\ref{fig:scl_selection}, \ref{fig:umi_selection} also
illustrates the distribution of stars selected without a spatial
criterion. We define the \emph{CMD+PM} selection as
\begin{equation}\protect\phantomsection\label{eq:sel_cmd_pm}{
{\cal L}_{\rm CMD,\ sat}\ {\cal L}_{\rm PM,\ sat} > {\cal L}_{\rm CMD,\ bg}\ {\cal L}_{\rm PM,\ bg}
}\end{equation} with the likelihoods from J+24 as described above. These
stars are distributed similar to the fiducial (probable members) sample
but with a background uniform distribution across the entire field. This
illustrates the approximate background of stars which may be confused as
members. Additionally, since there is no clear spatial structure in the
\emph{CMD+PM} (or \emph{all}) sample outside several \(R_h\), it is
unlikely that there are additional faint, tidal features detectable with
\emph{Gaia}. Not shown here, we also try a variety of simpler, absolute
cuts and thresholds, finding no evidence of other structure in both
\emph{Gaia} and DELVE/UNIONS data.

Finally, we illustrate the location of RV-confirmed members from
Section~\ref{sec:rv_obs} in each panel. Because RV targets tend to be
brighter than a typical \emph{Gaia} candidate, RV members typically have
more precise PMs. However, the RV members fill about the same area on
the CMD down to the magnitude limit. Additionally, while the number of
stars with RV measurements decreases at large radii, there are still
confirmed RV members as far as \(\sim 10R_h\) for both galaxies. These
stars illustrate the extended profiles of each galaxy. Note that for an
exponential profile, 99.95\% of stars should be within \(6R_h\), so the
discovery of any stars beyond \(6R_h\) hints at a deviation from an
exponential profile. These \emph{extratidal} stars hint at something
tidal in nature, and our goal is to explore possible interpretations.

\begin{figure}
\centering
\includegraphics[width=0.7\linewidth,height=\textheight,keepaspectratio]{figures/scl_selection.png}
\caption[Sculptor sample selection]{The selection criteria for Scl
members. Light grey points represent field stars (satisfying quality
criteria), turquoise points CMD+PM selected stars
(Eq.~\ref{eq:sel_cmd_pm}), blue xs probable members (2-component), and
RV confirmed members indigo diamonds. We mark the two stars from
\citet{sestito+2023a} with rust-outlined indigo stars. \textbf{Top:}
Tangent plane \(\xi, \eta\). The orange ellipse represents 3 half-light
radii. \textbf{Bottom left:} Colour magnitude diagram in Gaia \(G\)
magnitude versus \(G_{\rm BP} - G_{\rm RP}\) colour. We plot the Padova
12Gyr {[}Fe/H{]}=-1.68 isochone in orange. The black bar in the top left
represents the median colour error. \textbf{Bottom right:} Proper motion
in declination \(\mu_\delta\) vs RA \(\mu_{\alpha*}\) (corrected) the
orange circle represents the \citet{MV2020b} proper motion. The black
cross represents the median proper motion
error.}\label{fig:scl_selection}
\end{figure}

\begin{figure}
\centering
\includegraphics[width=0.7\linewidth,height=\textheight,keepaspectratio]{figures/umi_selection.png}
\caption[Ursa Minor sample selection]{Similar to
Fig.~\ref{fig:scl_selection} except for Ursa Minor. We outline RV
members outside of \(6R_h\) in black stars (from \citet{sestito+2023b},
\citet{pace+2020} and \citet{spencer+2018}).}\label{fig:umi_selection}
\end{figure}

\begin{figure}
\centering
\includegraphics[width=0.7\linewidth,height=\textheight,keepaspectratio]{figures/fornax_selection.png}
\caption[Fornax sample selection]{Similar to
Fig.~\ref{fig:scl_selection} except for Fornax. RV members are from
\citet{WMO2009}. While possibly a limitation of RV sample selection,
Fornax does not show the same extended outer halo of probable members as
Sculptor or Ursa Minor despite having many more
stars.}\label{fig:fornax_selection}
\end{figure}

\section{Density profiles}\label{density-profiles}

Density profiles are an essential observational constraint for our later
simulations. To derive density profiles, we use 0.05 dex bins in log
radius. We remove bins at smaller (larger) radii than the first empty
bin working outwards. We use symmetric poisson uncertainties. As
discussed below, these uncertainties are straightforward but are likely
under-representative. We use the values from \citet{munoz+2018}'s Sérsic
maximum likelihood fits for \(R_h\), which are more precisely derived
given their deeper photometry.

Fig.~\ref{fig:scl_observed_profiles} show the derived density profiles
for Sculptor, Ursa Minor, and Fornax. We calculate density profiles for
different selections of stars from above: all quality stars, CMD+PM
only, and the probable member samples. In each case, all samples are the
same towards the inner regions of the satellite. Classical dwarfs
dominate the stellar density in their cores. However, the density
profile \emph{all stars} plateaus at the total background in the field
at radii of 30-60 arcminutes. By restricting stars to being most likely
satellite members by CMD + PM, the background is reduced by 1-2 dex. The
CMD+PM background plateau likely represents the density of background
stars which could be mistaken as members. Finally, the background
(BG)-subtracted profile results from subtracting the apparent background
in the \emph{all} profile.

Note that the probable members (fiducial) density profile continues to
confidently estimate the density profile below the CMD+PM background.
These points are likely unreliable (see discussion below). However,
before this point, both the BG subtracted and probable members density
profiles are strikingly similar. Assumptions about the details of the
likelihood and spatial dependence have marginal influence on the
resulting density profile when the satellite is higher density than the
background. Thus the detection of excesses of stars in Sculptor (past 20
arcmin) and UMi (past 15-20 arcmin) are robust.

Nearby to UMi, there is a small, likely unassociated, ultrafaint star
cluster, Muñoz 1 \citep{munoz+2012}. The cluster is at a relative
position of \((\xi, \eta) \approx(-42, -15)\) arcminutes, corresponding
to an elliptical radius of 36 arcminutes. However this cluster does not
have a bright RGB, so has few stars brighter than a \(G\) mag of 22.
While Muñoz 1 may contribute \(\sim\) 5 stars to the density profile of
UMi, this would possibly only affect 1-2 bins due to its compact size.

To illustrate the differences between each dwarf galaxy, in
Fig.~\ref{fig:classical_dwarfs_densities}, we compare Sculptor, Ursa
Minor, and other classical dwarfs against 2D-exponential and Plummer
density profiles (REF). All dwarf galaxies appear to be well described
by an exponential profile in the inner regions. However, while an
exponential continues to be a good description for other dwarf galaxies,
Sculptor and Ursa Minor diverge, showing a substantial excess over an
exponential, better fit by a Plummer instead. In fact, Sculptor and Ursa
Minor show excesses of between 2-3 orders of magnitude beyond what is
expected of an exponential at the very outer regions!

\begin{figure}
\centering
\pandocbounded{\includegraphics[keepaspectratio]{figures/scl_umi_fnx_density_methods.pdf}}
\caption[Sculptor density profiles]{The density profile of Sculptor,
Ursa Minor, and Fornax for different selection criteria, ploted as log
surface dencity versus log elliptical radius. \emph{CMD+PM} select stars
more likely to be members according to CMD and PM only, \emph{all}
selects any high quality star, and \emph{BG subtracted} is the
background-subtracted density derived from \emph{all} stars. We mark the
half-light radius (vertical dashed line) and the break radius (black
arrow, REF).}\label{fig:scl_observed_profiles}
\end{figure}

\begin{figure}
\centering
\pandocbounded{\includegraphics[keepaspectratio]{figures/classical_dwarf_profiles.pdf}}
\caption[Classical dwarf density profiles]{The density profiles of
Sculptor, Ursa Minor, and other classical dwarfs compared to Exp2D and
Plummer density profiles. Dwarf galaxies are scaled to the same
half-light radius and density at half-light radius. Sculptor and Ursa
Minor have an excess of stars in the outer regions (past
\(\log R/R_h \sim 0.3\)) compared with other classical
dwarfs.}\label{fig:classical_dwarfs_densities}
\end{figure}

\section{Summary}\label{summary}

In summary, we have used J+24 data to derive the density profiles for
the classical dwarf galaxies. In each case, the density profile is
robust against different selection criteria. Both Sculptor and Ursa
Minor show strong evidence for deviations from an exponential profile.
We find no evidence of additional (velocity or stellar) substructure in
either galaxy. Our density profiles are consistent with many previous
works, all indicating divergence from a King or Exponential density
profile with a break between 30 and 60 arcminutes. Our goal in the
following chapters is to test if tides provide a viable explanation for
the observed properties of Sculptor and Ursa Minor.
