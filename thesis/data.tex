\section{Abstract}\label{abstract}

To provide an empirical benchmark for simulations in the next chapters,
we would like to understand and determine the observed properties of
Sculptor and Ursa Minor. In this section, we first review past
measurements of each galaxy and the observed properties of each dwarf
galaxy. We describe the \citet{jensen+2024} Bayesian methodology to
select high probability member stars from \emph{Gaia}. We test the
derived density profiles for Sculptor, Ursa Minor, and Fornax, finding
that \citet{jensen+2024}'s methodology successfully reduces the
background and produces similar results to a simple
background-subtracted density model. We show that Fornax is
approximately represented by a 2-D exponential profile. Instead,
Sculptor and Ursa Minor show an excess of stars relative to an 2-D
exponential in the outer regions. A Plummer profile provides a better
fit instead. Finally, we combine various radial velocity samples with
\citet{jensen+2024}'s members to derive systemic velocities and
dispersions for Sculptor and Ursa Minor.

\section{Introduction}\label{introduction}

The determination of basic properties of astronomical objects is an
ongoing and challenging endeavour.

\subsection{Gaia}\label{gaia}

Some of the most fundamental properties of astronomical objects are
their position and velocity. Unfortunately, determining positions to
stars is nontrivial, . Additionally, while line-of-sight or radial
velocities (RVs) are easily determined from spectroscopy, the tangental
velocities, perpendicular to RV, are only measurable through proper
motions, typically requiring precise astrometry as well. \emph{Gaia}'s
mission is to produce extremely precise astrometry enabling measurements
of unprecedented accuracy and scale for proper motions, parallaxes, and
magnitudes.

\emph{Gaia} was designed to revolutionize proper motion and parallax
measurements. \emph{Gaia} is a space-based, all-sky survey telescope
with two primary 1.3x0.75m mirrors situated at the Sun-Earth L2 lagrange
point \citep{gaiacollaboration+2016}. \emph{Gaia} was launched in XXX,
completing its mission in 2025 (but with two more data releases
planned). By imagining two patches of sky on the same focal plane,
separated by a fixed angle of 106.5degrees, Gaia is able to measure
absolute proper motions by comparing the apparent shifts of stars in
different regions of the sky. In addition to precise astrometric
information, \emph{Gaia} measures the magnitude of stars in the very
wide \emph{G} band (330-1050nm), blue and red colours using the blue and
red photometers (BP and RP), and takes low resolution BP-RP spectra and
radial velocity measurements of bright stars (magnitudes \textgreater{}
16?).

\emph{Gaia} has revolutionized many astronomical disciplines, the least
of which is local group and Milky Way science. While proper motions of a
dwarf galaxies has been measured in a case by case basis by the Hubble
Space Telescope, a full systematic determination of proper motions for
most dwarf galaxies was unavailable until \emph{Gaia}
\citep{MV2020a, battaglia+2022}. \emph{Gaia} has furthermore allowed for
the detection of many substructures of the halo, streams,

While \emph{Gaia} has performed excellently, a few limitations and
systematic uncertainties could become problematic. Gaia does show
systematic variations in its derived proper motions and parallaxes, as
measured by background quasars \citep{lindegren+2021}. These errors
limit the ability to determine absolute proper motions to some degree,
but since we would like to measure if a star is \emph{consistent} with a
dwarf galaxy, these only factor into the uncertainty budget for orbital
motion and are included in J+24. Additionally, while \emph{Gaia} was
designed to be complete down to about a magnitude of \(G=20\),
overcrowding or dense fields can limit this. \emph{Gaia} can only
successfully separate stars at a distance of about \textbf{X} (a few
times the pixel length of \ldots; pixel width is in scanning direction).
For the galaxies Ursa Minor, Fornax, and Sculptor, the highest density
(in FOrnax), is only \textbf{xxx} which appears to not significantly
affect \emph{Gaia}'s survey completeness. We test for brightness affects
in \textbf{REF}.

For our purposes, \emph{Gaia}

\citet[hereafter J+24]{jensen+2024} utilize \emph{Gaia} data with a
Bayesian framework to determine samples of high-probability members for
many Milky Way group dwarf galaxies.

Here, we breifly describe J+24's membership estimation and discuss how
this informs our observational knoledge of each galaxies density
profile. In general, J+24 use a Bayesian framework incorporating proper
motion (PM), colour-magnitude diagram (CMD), and spatial information to
determine the probability that a given star belongs to the satellite.
J+24 extends \citet{MV2020a}; \citet{MV2020b}. \citep[See also similar
work by][]{pace+li2019, battaglia+2022, pace+erkal+li2022, qi+2022}

\subsection{Summary of properties}\label{summary-of-properties}

In tbls.~\ref{tbl:scl_obs_props}, \ref{tbl:umi_obs_props} we present
adopted observed properties for each galaxy.

\begin{table*}[t]
\centering
\caption[Observed Properties of Sculptor]{\label{tbl:scl_obs_props}Observed properties of Sculptor. References are: 1. Ricardo R. Muñoz et al. (2018), 2. Tran et al. (2022), 3. McConnachie and Venn (2020b), 4 McConnachie and Venn (2020a). }
\label{tbl:scl_obs_props}
\begin{tabular}{lll}
\toprule
parameter & value & Source\\
\midrule
$\alpha$ & $15.0183 \pm 0.0012^\circ$ & 1\\
$\delta$ & $-33.7186 \pm 0.0007^\circ$ & 1\\
distance modulus & $19.60 \pm 0.05$ (RR lyrae) & 2\\
distance & $83.2 \pm 2$ kpc & 2\\
$\mu_{\alpha*}$ & $0.099 \pm 0.002 \pm 0.017$ mas yr$^{-1}$ & 3\\
$\mu_\delta$ & $-0.160 \pm 0.002_{\rm stat} \pm 0.017_{\rm sys}$ mas yr$^{-1}$ & 3\\
radial velocity & $111.18 \pm 0.23\ {\rm km\,s^{-1}}$ & sec. \ref{sec:rv_results}\\
$\sigma_v$ & $9.71\pm0.17\ {\rm km\,s^{-1}}$ & sec. \ref{sec:rv_results}\\
$R_h$ & $12.33 \pm 0.05$ arcmin & 4\\
$R_{h,inner}$ & $8.69 \pm 0.2$ arcmin & XREF\\
ellipticity & $0.36 \pm 0.01$ & 1\\
position angle & $92\pm1^\circ$ & 1\\
$M_V$ & $-10.82\pm0.14$ & 1\\
\bottomrule
\end{tabular}
\end{table*}

\begin{table*}[t]
\centering
\caption[Observed Properties of Ursa Minor]{\label{tbl:umi_obs_props}Observed properties of Ursa Minor. References are: (1) Ricardo R. Muñoz et al. (2018), (2) Garofalo et al. (2025), (3) McConnachie and Venn (2020a), (4) average of Pace et al. (2020) and Spencer et al. (2018). }
\label{tbl:umi_obs_props}
\begin{tabular}{lll}
\toprule
parameter & value & Source\\
\midrule
$\alpha$ & $ 227.2420 \pm 0.0045$˚ & 1\\
$\delta$ & $67.2221 \pm 0.0016$˚ & 1\\
distance modulus & $19.23 \pm 0.11$ (RR lyrae) & 2\\
distance & $70.1 \pm 3.6$ kpc & 2\\
$\mu_\alpha*$ & $-0.124 \pm 0.004 \pm 0.017$ mas yr$^{-1}$ & 3\\
$\mu_\delta$ & $0.078 \pm 0.004_{\rm stat} \pm 0.017_{\rm sys}$ mas yr$^{-1}$ & 3\\
radial velocity & $-245.9 \pm 0.3_{\rm stat} \pm 1_{\rm sys}$ km s$^{-1}$ & sec. \ref{sec:rv_results}\\
$\sigma_v$ & $8.8 \pm 0.2$ & sec. \ref{sec:rv_results}\\
$r_h$ & $18.2 \pm 0.1$ arcmin & 1\\
ellipticity & $0.55 \pm 0.01$ & 1\\
position angle & $50 \pm 1^\circ$ & 1\\
$M_V$ & $-9.03 \pm 0.05$ & 1\\
\bottomrule
\end{tabular}
\end{table*}

\section{Gaia Membership Selection}\label{gaia-membership-selection}

\subsection{Data}\label{data}

J+24 select stars initially from Gaia within a 2--4 degree circular
region centred on the dwarf. J+24 only consider stars with:

\begin{itemize}
\tightlist
\item
  Solved astrometry, magnitude, and colour.
\item
  Renormalized unit weight error, \({\rm ruwe} \leq 1.3\), so astrometry
  is high quality.
\item
  3\(\sigma\) consistency of measured parallax with dwarf distance
  (typically near zero; with \citet{lindegren+2021} zero-point
  correction).
\item
  Absolute RA and Dec proper motions, \(\mu_{\alpha*}\), \(\mu_\delta\),
  less than 10\(\,{\rm mas\ yr^{-1}}\). (Corresponds to tangental
  velocities of \(\gtrsim 500\) km/s at distances larger than 10 kpc.)
\item
  Corrected colour excess is within expectations:
  \(|C^*| \leq 3\,\sigma_{C^*}(G)\), with \(C^*\) and \(\sigma_{C^*}\)
  from \citet{riello+2021}. Removes stars with unreliable photometry.
\item
  \(G\) magnitude is between \(22 > G > G_{\rm TRGB} - 5\delta\rm {DM}\)
  (dereddened magnitudes). Removes very faint stars and stars
  significantly brighter than the tip of the red giant branch (TRGB)
  magnitude plus the distance modulus uncertainty \(\delta {\rm DM}\).
\item
  Colour is between \(-0.5 < {\rm BP - RP} <  2.5\) (dereddened).
\end{itemize}

Photometry is dereddened using \citet{schlegel+finkbeiner+davis1998}
extinction maps.

J+24 calculate the probability that any star belongs to either the
satellite or the MW background with \begin{equation}
P_{\rm sat} = \frac{f_{\rm sat}{\cal L}_{\rm sat}}{f_{\rm sat}{\cal L}_{\rm sat} + (1-f_{\rm sat}){\cal L}_{\rm bg}}
\end{equation}

where \(f_{\rm sat}\) is the fraction of candidate members in the field,
and \({\cal L}_{\rm sat}\) and \({\cal L}_{\rm bg}\) are the satellite
(sat) and background (bg) likelihoods respectively. The likelihoods are
the product of a spatial, proper motion, and CMD term: \begin{equation}
{\cal L} = {\cal L}_{\rm space}\ {\cal L}_{\rm PM}\ {\cal L}_{\rm CMD}.
\end{equation}

Each likelihood map is normalized over the respective 2D parameter
space. \(f_{\rm sat}\) scales the likelihoods by the abundance of stars
in the satellite compared to the background.

J+24 consider both a one-component and a two-component spatial
likelihood. The one component model is constructed as a single
exponential profile (\(\Sigma \propto e^{R_{\rm ell} / R_s}\)), with
\(R_s\) fixed to the value in the table in \citet{MV2020a}.
Additionally, structural uncertainties (for position angle, ellipticity,
and scale radius) are sampled over to construct the final likelihood
map. The two-component model instead adds a second exponential,
\(\Sigma_\star \propto e^{-R/R_s} + B\,e^{-R/R_{\rm outer}}\). The inner
scale radius is fixed, and the outer scale radius and magnitude of the
second component \(R_{\rm outer}\), \(B\) are solved for. Structural
properties are not accounted for. The PM likelihood is a bivariate
gaussian with variance and covariance equal to each star's proper
motions. (The stellar PM uncertainties is assumed and typically is the
dominant PM uncertainty.)

The satellite's CMD likelihood is based on a Padova isochrone
\citep{girardi+2002}. The isochrone has a matching metallicity and 12
Gyr age. The (gaussian) colour width is assumed to be 0.1 mag plus the
Gaia colour uncertainty at each magnitude. The HB is modelled as a
constant magnitude extending blue of the CMD (mean magnitude of
isochone's HB stars and a 0.1 mag width plus the mean colour error). A
likelihood map is constructed by sampling the distance modulus in
addition to the CMD width, taking the maximum of RGB and HB likelihoods.

The background likelihoods are instead empirically constructed. Stars
stars outside of 5\(R_h\) passing the quality cuts estimate the
background density in PM and CMD space. The density is a sum of
bivariate gaussians with variances based on Gaia uncertainties (and
covariance for proper motions). The spatial background likelihood is
assumed to be constant.

In J+24, a MCMC simulation is ran using the above total likelihood to
solve \(\mu_{\alpha*}\), \(\mu_\delta\), and \(f_{\rm sat}\). For the
two component models, the amplitude and scale radius of the excented
compopnet is also solved for. The proper motion single component prior
is same as \citet{MV2020a}: a normal distribution with mean 0 and
standard deviation 100 km/s. If 2-component spatial, instead is a
uniform distribution spanning 5\(\sigma\) of single component case w/
systematic uncertainties. Priors on \(f_{\rm sat}\) and \(B\) (if 2
component) are uniform from 0--1. \(R_{\rm outer}\) has a uniform prior
only restricting \(R_{\rm outer} > R_s\). The mode of each parameter
from the MCMC are then used to calculate the final \(P_{\rm sat}\)
values we use here.

We adopt a probability cut of \(P_{\rm sat} = 0.2\) as our fiducial
sample. Most stars are assigned probabilities close to either 0 or 1, so
the choice of probability threshold is only marginally significant.
Additionally, even for a probability cut of 0.2, the purity of the
resulting sample with RV measurements is very high (\textasciitilde90\%,
J+24). Note there is likely a systematic bias in using stars with RV
measurements to measure purity. Fainter stars are less likely to have
been targeted and have poorer astrometry and colour information. We find
no difference in the resulting density distributions when restricting
stars to be brighter than a specific magnitude.

\subsection{Selected samples}\label{selected-samples}

In Figs.~\ref{fig:sculptor_selection}, \ref{fig:umi_selection}, we
illustrate the resulting samples from the algorithm in the tangent plane
(\(\xi\), \(\eta\), \emph{does this need defined?}), CMD, and proper
motion space. For both galaxies, each criteria plays a commensurate role
in sifting out nonmembers. Proper motions are clustered around the dwarf
systemic motion, the CMD is well defined including the horizontal
branch, and stars only within a few \(R_h\) are included. We also mark
stars with consistent radial velocities (RVs) (see below), which trace
each feature albeit more cautiously. We additionally mark the furthest
confirmed RV members of each galaxy, illustrating the large extent of
each galaxy past \(R_h\).

Figs.~\ref{fig:sculptor_selection}, \ref{fig:umi_selection} also
illustrates the distribution of stars selected without a spatial
criterion. We define the CMD+PM selection as stars satisfying
\begin{equation}
{\cal L}_{\rm CMD,\ sat}\ {\cal L}_{\rm PM,\ sat} > {\cal L}_{\rm CMD,\ bg}\ {\cal L}_{\rm PM,\ bg}
\end{equation} These stars are distributed similar to the fiducial
(probable members) sample but with a fainter uniform distribution across
the entire field. This illustrates the approximate background of stars
which may be confused as members. Additionally, since there is no clear
spatial structure in the CMD+PM sample outside several \(R_h\), it is
unlikely that there are additional faint, tidal features detectable with
Gaia. Not shown here, we also try a variety of simpler, absolute cuts
and thresholds, finding no extended structure beyond what is detected in
J+24. (\emph{Is it worth using this to calculate an OOM upper limit on
the density of tidal tails?}).

\begin{figure}
\centering
\pandocbounded{\includegraphics[keepaspectratio]{figures/scl_selection.pdf}}
\caption[Sculptor selection criteria]{The selection criteria for Scl
members. Field stars (satesfying quality criteria) are light grey,
CMD+PM selected stars are blue, probable members (2-component) are
orange, and RV confirmed members are red. \textbf{Top:} Tangent plane
\(\xi, \eta\). We outline in star symbols the two stars from
\citet{sestito+2023a}. \textbf{Bottom left:} Colour magnitude diagram in
Gaia G versus BP - RP. \textbf{Bottom right:} Proper motion in
declination \(\mu_\delta\) vs RA \(\mu_{\alpha*}\)
(corrected).}\label{fig:sculptor_selection}
\end{figure}

\begin{figure}
\centering
\pandocbounded{\includegraphics[keepaspectratio]{figures/umi_selection.pdf}}
\caption[Ursa Minor Selection]{Similar to
Fig.~\ref{fig:sculptor_selection} except for Ursa Minor. We outline RV
members outside of \(3R_h\) in black stars (from \citet{sestito+2023b},
\citet{pace+2020} and \citet{spencer+2018}).}\label{fig:umi_selection}
\end{figure}

\subsection{Density Profiles}\label{density-profiles}

Density profiles are an essential observational constraint for our later
simulations. To derive density profiles, we use 0.05 dex bins in log
radius. We remove bins at smaller (larger) radii than the first bin to
contain no stars, working outwards. We use symmetric poisson
uncertainties. As discussed below, these uncertainties are
straightforward but are likely under-representative.

Figs.~\ref{fig:scl_observed_profiles}, \ref{fig:umi_observed_profiles}
show the derived density profiles for Scl and UMi. We calculate density
profiles for different selections of stars from above: all quality
stars, CMD+PM only, and the probable member samples. In each case, all
samples are the same towards the inner regions of the satellite.
Classical dwarfs dominate the stellar density in their cores. However,
the density profile \emph{all stars} plateaus at the total background in
the field at radii of 30-60 arcminutes. By restricting stars to being
most likely satellite members by CMD + PM, the background is reduced by
1-2 dex. The CMD+PM background plateau likely represents the density of
background stars which could be mistaken as members. Finally, the
background (BG)-subtracted profile results from subtracting the apparent
background in the \emph{all} profile.

Note that the probable members (fiducial) density profile continues to
confidently estimate the density profile below the CMD+PM background.
These points are likely unreliable (see discussion below). However,
before this point, both the BG subtracted and probable members density
profiles are strikingly similar. Assumptions about the details of the
likelihood and spatial dependence have marginal influence on the
resulting density profile when the satellite is higher density than the
background. Thus the detection of excesses of stars in Sculptor (past 20
arcmin) and UMi (past 15-20 arcmin) are robust.

Nearby to UMi, there is a small, likely unassociated, ultrafaint star
cluster, Muñoz 1 \citep{munoz+2012}. The cluster is at a relative
position of \((\xi, \eta) \approx(-42, -15)\) arcminutes, corresponding
to an elliptical radius of 36 arcminutes. However this cluster does not
have a bright RGB, so has few stars brighter than a \(G\) mag of 22.
While Muñoz 1 may contribute \(\sim\) 5 stars to the density profile of
UMi, this would possibly only affect 1-2 bins due to its compact size.

\begin{figure}
\centering
\pandocbounded{\includegraphics[keepaspectratio]{figures/scl_density_methods.pdf}}
\caption[Sculptor density profiles]{The density profile of Sculptor for
different selection criteria, ploted as log surface dencity versus log
elliptical radius. Residuals are with respect to th interpolated
\emph{probable members} density. \emph{Probable members} selects stars
with PSAT \textgreater{} 0.2 considering PM, CMD, and spatial,
\emph{CMD+PM} select stars more likely to be members according to CMD
and PM only, \emph{all} selects any high quality star, and \emph{BG
subtracted} is the background-subtracted density derived from
high-quality stars. We mark the half-light radius ( vertical dashed
line) and the break radius (black arrow,
REF).}\label{fig:scl_observed_profiles}
\end{figure}

\begin{figure}
\centering
\pandocbounded{\includegraphics[keepaspectratio]{figures/umi_density_methods.pdf}}
\caption[Ursa Minor density profiles]{Similar to
fig.~\ref{fig:scl_observed_profiles} except for Ursa
Minor.}\label{fig:umi_observed_profiles}
\end{figure}

\begin{figure}
\centering
\pandocbounded{\includegraphics[keepaspectratio]{figures/fornax_density_methods.pdf}}
\caption[Fornax density profiles]{Similar to
fig.~\ref{fig:scl_observed_profiles} except for
Fornax.}\label{fig:fornax_observed_profiles}
\end{figure}

\subsection{Caveats}\label{caveats}

The J+24 method was designed in particular to detect the presence of a
density excess and find individual stars at large radii to be followed
up. We are more interested in accurately quantifying the density profile
and size of any perturbations. One potential problem with using J+24's
candidate members is that the algorithm assumes the density is either
described by a single or double exponential. If this model does not
accurately match the actual density profile of the dwarf galaxy, we
would like to understand how strongly influenced the density profile is
by this assumption.

In particular, in Fig.~\ref{fig:umi_observed_profiles}, notice that the
PSAT method produces small errorbars, even when the density is \(>1\)dex
below the local background. These stars are likely selecting stars from
the statistical MW background consistent with UMi PM / CMD, recovering
the assumed density profile. As a result, the reliability of these
density profiles below the CMD+PM background may be questionable. A more
robust analysis, removing this particular density assumption, would be
required to more appropriately represent the knowledge of the density
profile as the background begins to dominate.

J+24 do not account for structural uncertainties in dwarfs for the two
component case. We assume constant ellipticity and position angle. Dwarf
galaxies, in reality, are not necessarily smooth and constant. They do
test an alternative method using circular radii for the extended density
component, and we find these density profiles are very similar. We also
assume a constant ellipticity and position angle here.

Finally, our approximation to the poisson uncertainties Additionally, a
more self consistent model would fit the density profile to the entire
field at once, eliminating possible misrepresentation of the
uncertainties.

\section{Radial velocity modeling}\label{radial-velocity-modeling}

\subsection{Data selection}\label{data-selection}

For both Sculptor and Ursa Minor, we construct literature samples of
radial velocity measurements. We combine these samples with J+24's
members to produce RV consistent stars and to compute velocity
dispersion, systematic velocities, and test for the appearance of
velocity gradients.

First, we crossmatch all catalogues to J+24 Gaia stars. If a study did
not report GaiaDR3 source ID's, we match to the nearest star within 1-3
arcseconds (see REF \citet{tab:rv_measurements}). We combine the mean RV
measurement from each study using the inverse-variance weighted mean
\(\bar v\), standard uncertainty \(\delta \bar v\), and (biased)
variance \(s^2\). We remove stars with significant velocity dispersions
as measured between observations in a study or between studies. By using
that \(\chi^2=\frac{s^2}{\delta \bar v^2}\), we remove stars with a
\(\chi^2\) larger than the 99.9th percentile of the \(\chi^2\)
distribution with \(N-1\) measurements. This cut typically removes stars
with reduced chi-squared values
\(\tilde\chi^2  = \frac{s^2}{\nu\,\delta \bar v^2}\gtrsim 7\) (since the
number of measurements is 1-3 typically).

Next, we need to correct the coordinate frames for the solar motion and
on-sky size of the galaxy. We transform the frame into the galactic
standard of rest (GSR). The next step is to account for the slight
differences in the direction of each radial velocity. Let the \(\hat z\)
be the direction from the sun to the dwarf galaxy. Then if \(\phi\) is
the angular distance between the centre of the galaxy and the individual
star, the corrected radial velocity is then \begin{equation}
v_z = v_{\rm los, gsr}\cos\phi  - v_{\alpha}\cos\theta \sin\phi - v_\delta \sin\theta\sin\phi
\end{equation} where \(v_{\rm los, gsr}\) is the line of sight velocity
in the GSR frame, \(v_\alpha\) and \(v_\delta\) are the tangental
velocities in RA and Dec, and \(\theta\) is the position angle of the
star with respect to the centre of the dwarf. The correction from both
effects induces an apparent gradient of about \(1.3\,\kmsdeg\) for
Sculptor and less for Ursa Minor \citep[see
also][]{WMO2008, strigari2010}. We add the uncertainty in \(v_z\) from
the distance uncertainty and velocity dispersion in quadrature to the RV
uncertainties for each star. We then use the \(v_z\) values for the
following modelling, however repeating with uncorrected, heliocentric
velocities does not significantly affect the results.

The combined likelihood, including RV information, becomes
\begin{equation}
{\cal L} = {\cal L}_{\rm space} {\cal L}_{\rm CMD} {\cal L}_{\rm PM} {\cal L}_{\rm RV}
\end{equation} where we assume that the satellite and background
distributions are Gaussian. Specifically, \begin{equation}
\begin{split}
{\cal L}_{\rm RV, sat} &= f\left( \frac{v_i -\mu_{v}}{\sqrt{\sigma_{v}^2 + (\delta v_i)^2}}\right) \\
{\cal L}_{\rm RV, bg} &= f\left( v_i /  \sigma_{\rm halo} \right)
\end{split}
\end{equation} where \(f\) is the probability density of a standard
normal distribution, \(\mu_v\) and \(\sigma_v\) are the systemic
velocity and dispersion of the satellite, and \(\delta v_i\) is the
individual measurement uncertainty. Typically, the velocity dispersion
will dominate the uncertainty budget here. We assume a halo/background
velocity dispersion of a constant \(\sigma_{\rm halo} = 100\,\kms\)
\citep[e.g.][]{brown+2010}.

Similar to above, we retain stars with the resulting membership
probability of greater than 0.2. Because of the additional information
from radial velocities, most stars have probabilities close to 1 or 0 so
the probability cut is not too significant.

We assume priors on systematic velocity and velocity dispersion of
\begin{equation}
\begin{split}
\mu_{v} &= N(0\,\kms, \sigma_{\rm halo}^2) \\ 
\sigma_{v} &= U(0, 20\,\kms)
\end{split}
\end{equation} where \(\sigma_{\rm halo} = 100\,{\rm km\,s^{-1}}\) is
the velocity dispersion of the MW halo adopted above, a reasonable
assumption for dwarfs in orbit around the MW.

\subsection{Results}\label{sec:rv_results}

\begin{figure}
\centering
\pandocbounded{\includegraphics[keepaspectratio]{figures/scl_umi_rv_fits.pdf}}
\caption[LOS velocity fit to Scl.]{Velocity histogram of Scl and UMi in
terms of \(v_z\) (REF). Orange points are from our crossmatched RV
membership sample.}
\end{figure}

For Sculptor, we combine radial velocity measurements from APOGEE,
\citet{sestito+2023a}, \citet{tolstoy+2023}, and \citet{WMO2009}.
\citet{tolstoy+2023} and \citet{WMO2009} provide the bulk of the
measurements. We find that there is no significant velocity shift in
crossmatched stars between catalogues. After crossmatching to high
quality Gaia stars and excluding significant stellar velocity
dispersions, we have a sample of 1918 members.

We derive a systemic velocity for Sculptor of
\(111.18\pm0.23\,\kms\)with velocity dispersion \(9.71\pm0.17\,\kms\).
Our values are very consistent with previous work
\citep[e.g.][\citet{arroyo-polonio+2024},
\citet{battaglia+2008}]{walker+2009}. See appendix REF for a more
detailed comparison between different samples and additional tests.

We detect a moderately significant gradient of \(4.3\pm1.4\,\kmsdeg\) at
a position angle of \(-149_{-13}^{+17}\) degrees (see appendix REF).
Several past work has attempted to detect a gradient in Sculptor, but no
consensus has been reached. \citet{arroyo-polonio+2024} detect a
velocity gradient of \(4\pm1.5\,\kmsdeg\) in a similar direction using
the \citet{tolstoy+2023} sample, finding inconclusive statistical
evidence. They additionally suggest a third chemodynamical component of
the galaxy which may bias rotation measurements. \citet{battaglia+2008}
also detect a \(-7.6_{-2.2}^{+3.0}\,\kmsdeg\) velocity gradient along
the major axis, approximately the same direction. Instead,
\citet{strigari2010}; \citet{martinez-garcia+2023} detect no significant
gradient in Sculptor using \citet{WMO2009} sample. Note that
pre-\emph{Gaia} work did not have as strong of a constraint on the
proper motion of Scl, which limits conclusions of the intrinsic velocity
gradient in Scl.

For UMi, we collect radial velocities from, APOGEE,
\citet{sestito+2023b}, \citet{pace+2020}, and \citet{spencer+2018}. We
shifted the velocities of \citet{spencer+2018} (\(-1.1\,\kms\)) and
\citet{pace+2020} (\(+1.1\,\kms\) ) to reach the same scale. We found
183 crossmatched common stars (passing 3\(\sigma\) RV cut, velocity
dispersion cut, and PSAT J+24 \textgreater{} 0.2 w/o velocities). Since
the median difference in velocities in this crossmatch is about 2.2
km/s, we adopt 1 km/s as the approximate systematic error here. Our
final sample includes 831 members.

We derive a mean \(-245.9\pm0.3_{\rm stat}\,\kms\) and velocity
dispersion of \(8.76\pm0.24\,\kms\) for UMi. This is consistent with
\citet{pace+2020} and to a lesser extent with \citet{spencer+2018}. We
do not find evidence for a velocity gradient, consistent with past work
\citep{pace+2020, martinez-garcia+2023}.

\subsection{Discussion and
limitations}\label{discussion-and-limitations}

Our model here is relatively simple. Some things which we note as
systematics or limitations:

\begin{itemize}
\tightlist
\item
  Inter-study systematics and biases. While basic crossmatches and a
  simple velocity shift, combining data from multiple instruments is
  challenging. This appears to be a minor issue (Sculptor) or is
  corrected for (Ursa Minor).
\item
  Misrepresentative uncertainties. Inspection of the variances compared
  to the standard deviations within a study seems to imply that errors
  are accurately reported. APOGEE notes that their RV uncertainties are
  known to be underestimates but are a small proportion of our sample.
\item
  Binarity. While not too large of a change for classical dwarfs, this
  could inflate velocity dispersions of about \(9\,\kms\) by about
  \(1\,\kms\)\citep{spencer+2017}. Thus, our measurement is likely
  slightly inflated given the high binarity fractions measured in these
  systems \citep[\citet{spencer+2018}]{arroyo-polonio+2023}.
\item
  Multiple populations. Both Sculptor and Ursa Minor likely contain
  multiple populations \citep[\citet{pace+2020},
  \citet{tolstoy+2004}]{arroyo-polonio+2024}. Since we only model a
  single population, and each population may have a different extent and
  velocity dispersion, this could result in biased velocity dispersions.
  However, it is unclear how to uniquely determine an overall velocity
  dispersion in a multi-population system.
\item
  Selection effects. RV studies each have their own selection effects,
  which may affect the resulting dispersion, especially if different
  populations or regions of the galaxy have different velocities or
  velocity dispersions. We do not attempt to correct for this.
\end{itemize}

For both Ursa Minor and Sculptor, we also fit models to only data from
individual surveys (see REF). Since the resulting parameters are very
similar, we conclude that many of the systematic uncertainties are
likely smaller than the present errors or that each large survey has
similar biases.

\section{Comparison and conclusions}\label{comparison-and-conclusions}

To illustrate the differences between each dwarf galaxy, in
Fig.~\ref{fig:classical_dwarfs_densities}, we compare Sculptor, Ursa
Minor, and Fornax against 2D-exponential and Plummer density profiles
(REF). While all dwarfs appear similar in the inner regions, each dwarf
diverges in the outer regions relative to an exponential. Relative to an
exponential, Fornax is underdense but Sculptor and Ursa Minor are both
overdense. A Plummer profile instead provides a more reasonable fit to
Scl and UMi.

In summary, we have used J+24 data to derive the density profiles for
Fornax, Sculptor, and Ursa Minor. In each case, the density profile is
robust against different selection criteria. Both Sculptor and Ursa
Minor show strong evidence for deviations from an exponential profile.
We also compile velocity measurements to derive the systemic motions and
velocity dispersions of each galaxy. We find evidence for a velocity
gradient in Sculptor of \(4.3\pm1.3\,\kmsdeg\). We find no evidence of
additional (velocity or stellar) substructure in either galaxy. Our goal
in the following chapters is to test if tides provide a viable
explanation for the observed properties of Sculptor and Ursa Minor.

\begin{figure}
\centering
\pandocbounded{\includegraphics[keepaspectratio]{figures/scl_umi_fornax_exp_fit.pdf}}
\caption[Classical dwarf density profiles]{The density profiles of
Sculptor, Ursa Minor, and Fornax compared to Exp2D and Plummer density
profiles. Dwarf galaxies are scaled to the same half-light radius and
density at half-light radius (fit from the inner 3 scale radii
exponential recursively. ). Sculptor and Ursa Minor have an excess of
stars in the outer regions (past \(\log R/R_h \sim 0.3\)) compared with
Sculptor. Sculptor and Ursa Minor's density profiles are created with
the 2-component J+24 model, but the excess does not change significantly
for the 1-component model.}\label{fig:classical_dwarfs_densities}
\end{figure}
