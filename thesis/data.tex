\section{Gaia Membership Selection}\label{gaia-membership-selection}

Gaia provides unprecedented accuracy in proper motions and magnitudes.
As such, Gaia data is uniquely excellent to produce low-contamination
samples of likely member stars belonging to satellites. Here, we breifly
describe J+24's membership estimation and discuss how this informs our
observational knoledge of each galaxies density profile. In general,
J+24 use a Bayesian framework incorporating proper motion (PM,
colour-magnitude diagram (CMD), and spatial information to determine the
probability that a given star belongs to the satellite. J+24 extends
\citet{MV2020a} (see also \citet{pace+li2019}, etc.).

J+24 select stars initially from Gaia satisfying:

\begin{itemize}
\tightlist
\item
  High quality astrometry (\texttt{ruwe\ \textless{}=\ 1.3})
\item
  3\(\sigma\) consistency of measured parallax with dwarf distance +
  uncertainty (near zero with \citet{lindegren+2018} zero-point
  correction)
\item
  Absolute RA and Dec proper motions less than
  10\(\,{\rm mas\ yr^{-1}}\)
\item
  Good photometry
\item
  No colour excess (\citet{lindegren+2018} equation C.2)
\item
  G \textgreater{} 22 and less than 5\(\sigma\) above TRGB, and between
  -0.5 and 2.5 in Bp - Rp.
\end{itemize}

J+24 calculate the probability that any star belongs to either the
satellite or the MW background as \[
P_{\rm sat} = \frac{f_{\rm sat}{\cal L}_{\rm sat}}{f_{\rm sat}{\cal L}_{\rm sat} + (1-f_{\rm sat}){\cal L}_{\rm bg}}
\]

where the satellite (sat) and background (bg) likelihoods are simply the
product of the PM, CMD, and spatial components: \[
{\cal L} = {\cal L}_{\rm space}\ {\cal L}_{\rm PM}\ {\cal L}_{\rm CMD}
\]

The satellite likelihood is constructed as

\begin{itemize}
\tightlist
\item
  CMD: The CMD is from Padova \citep{girardi+2002} isochrone at 12 Gyr
  with a colour width of 0.1 mag plus the Gaia colour uncertanty and the
  observed metallicity of the dwarf. The HB is modelled as a constant
  magnitude extending blue of the CMD with a 0.1 mag width plus the mean
  colour error. A likelihood map is constructed by sampling the distance
  modulus in addition to the CMD width, taking the maximum of RGB and HB
  likelihoods.
\item
  Spatial: A single exponential
  (\(\Sigma \propto e^{R_{\rm ell} / R_s}\)) using mean values of the
  structural parameters (\textbf{verify}). For Scl and UMi, this is
  instead a double exponential
  \(\Sigma_\star \propto e^{-R/R_s} + B\,e^{-R/R_{\rm outer}}\) where
  the inner exponential remains fixed and structural parameters are not
  accounted for. Structural parameter uncertainties are not accounted
  for.
\item
  PM. A bivariate gaussian with variance and covariance equal to each
  star's proper motions. Each star's proper motions uncertainty are
  assumed to be the dominant uncertainty.
\end{itemize}

The background likelihood is constructed as:

\begin{itemize}
\tightlist
\item
  CMD : Constructed as a KDE using the other quality-selected stars
  outside of \(5R_h\) in the catalogue. The width of each star is
  projected as a bivariate Gaussian.
\item
  PM: same as CMD except in PM space.
\item
  Spatial: a constant likelihood.
\end{itemize}

Note that each likelihood map is normalized over the respective
parameter space. In order to represent the difference in frequency of
background and forground stars, \(f_{\rm sat}\) represents the field
fraction of member stars.

In J+24, a MCMC simulation is ran using the above likelihood to solve
for the following parameters

\begin{itemize}
\tightlist
\item
  Systemic proper motions \(\mu_\alpha\), \(\mu_\delta\). Prior: within
  5\(\sigma\) of single component case w/ systematic uncertainties.
  Single component prior is based on MV2020 (RV members / expected
  velocity dispersion of halo at distance?)
\item
  \(f_{\rm sat}\) density normalization (uniform between 0 and 1)
\item
  Spatial component parameters \(B\) and \(R_{\rm outer}\) for extended
  profiles (Scl and UMi here.)
\end{itemize}

The median parameters from the simulation are then used to calculate the
final \(P_{\rm sat}\) values we use here.

We adopt a probability cut of \(P_{\rm sat} = 0.2\) as our fiducial
sample. Most stars are assigned probabilities close to 0 or 1, so the
choice of probability threshhold is not too significant. Additionally,
even for a probability cut of 0.2, the purity of the resulting sample
with RV measurements is very high (\textasciitilde90\%, J+24). (Note:
there is likely a high systematic bias in using stars with RV
measurements to measure purity. Fainter stars have poorer measurements
and distant stars are less likely to have been targeted. )

\subsection{Resulting Samples}\label{resulting-samples}

\begin{figure}
\centering
\includegraphics{figures/scl_selection.pdf}
\caption[Sculptor selection criteria]{The selection criteria for Scl
members. Probable members (2-component) are orange, and all field stars
(satisfying quality criteria) are in light grey. \textbf{Top:} Tangent
plane. \textbf{Bottom left:} Colour magnitude diagram. \textbf{Bottom
right:} Proper motion.}\label{fig:sculptor_selection}
\end{figure}

\begin{figure}
\centering
\includegraphics{figures/umi_selection.pdf}
\caption[Ursa Minor Selection]{Similar to
fig.~\ref{fig:sculptor_selection} except for Ursa Minor. UMi features a
very extended density profile with some stars \textasciitilde{} 6\(R_h\)
including a RV member. UMi is also highly elliptical compared to other
classical dwarfs.}\label{fig:umi_selection}
\end{figure}

Fornax(figures/fornax\_selection.png)\{\#fig:fornax\_selection\}

Figure: Similar to fig.~\ref{fig:sculptor_selection} except for Fornax.
Note that we only use one-component probabilities here. Fornax is the
quintessential unperturbed dwarf galaxy.

\subsubsection{Searches for tidal tails}\label{searches-for-tidal-tails}

\begin{itemize}
\item
  There are no apparent overdensities in the PM \& CMD only selected
  stars to suggest the presence of a tidal tail
\item
  We have tried selective matched filter
\item
  This means that at least at the level of where the background density
  dominates, we can exclude models which produce tidal tails brighter
  than a density of
  \(\Sigma_\star \approx 10^{-2}\,\text{Gaia-stars\ arcmin}^{-2} \approx 10^{-6} \, {\rm M_\odot\ kpc^{-2}}\)
  (TODO assuming a distance of \ldots{} and stellar mass of \ldots).
\end{itemize}

\subsection{Density Profiles}\label{density-profiles}

Our primary observational constraint is the density profile of a dwarf
galaxy.

To derive density profiles, we use 0.05 dex bins in log radius (i.e.~the
bins are derived from 10\^{}(minimum(logR):0.05:maximum(logR))). The
density in each bin is then (from Poisson statistics) \[
\Sigma_b = N_{\rm memb} / A_{\rm bin} \pm \sqrt{N_{\rm memb}} / A_{\rm bin}
\] where \(N_{\rm memb}\) is the number of members in the bin and
\(A_{\rm bin}\) is the area of the bin. As discussed below, these
uncertainties underrepresent the true uncertainty on multiple accounts.
We retain Poisson errors for simplicity here.

In Figures fig.~\ref{fig:scl_observed_profiles},
fig.~\ref{fig:umi_observed_profiles},
\citet{fig_fornax_observed_profiles}, we show the derived density
profiles for each galaxy for samples similar to in the selection plots
above. In each case, all samples are the same towards the inner regions
of the satellite, illustrating that these density profiles are dominated
by satellite stars in the centre. The sample containing all stars
reaches a plateau at the total background in the field. However,
restricting stars to being most likely satellite members by CMD + PM,
the background is much lower. This plateau likely represents the real
background of background stars which could be mistaken as members.
Finally, we have the probable members and bg-subtracted densities. BG
subtracted is based on the \texttt{all} density profile, subtracting the
mean background density for stars beyond the last point of the BG
subtracted profile.

Note that the probable members (fiducial) density profile continues to
confidently estimate the density profile below the CMD+PM likely star
background. These points are likely unreliable (see discussion below).
However, before this point, both the BG subtracted and probable members
density profiles are strikingly similar. Assumptions about the details
of the likelihood and spatial dependence have marginal influence on the
resulting density profile when the satellite is higher density than the
background.

\begin{figure}
\centering
\includegraphics{figures/scl_density_methods.pdf}
\caption[Sculptor density profiles]{The density profile of Sculptor for
different selection criteria. \emph{probable members} selects stars with
PSAT \textgreater{} 0.2 considering PM, CMD, and spatial, \emph{CMD+PM}
select stars more likely to be members according to CMD and PM only,
\emph{all} selects any high quality star, and \emph{BG subtracted} is
the background-subtracted density derived from high-quality
stars.}\label{fig:scl_observed_profiles}
\end{figure}

\begin{figure}
\centering
\includegraphics{figures/umi_density_methods.pdf}
\caption[Ursa Minor density profiles]{Similar to
fig.~\ref{fig:scl_observed_profiles} except for Ursa
Minor.}\label{fig:umi_observed_profiles}
\end{figure}

\begin{figure}
\centering
\includegraphics{figures/fornax_density_methods.pdf}
\caption[Fornax density profiles]{Similar to
fig.~\ref{fig:scl_observed_profiles} except for
Fornax.}\label{fig:fornax_observed_profiles}
\end{figure}

\section{Comparison of the Classical
dwarfs}\label{comparison-of-the-classical-dwarfs}

Classical dwarfs are often the brightest dwarfs in the sky.c The density
profiles of classical dwarf galaxies is thus well measured, enabling
detailed comparisons.

\begin{table*}[t]
\centering
\begin{tabular}{lll}
\toprule
galaxy & R\_h (exp inner 3Rs) & num cand\\
\midrule
Fornax & $17.8\pm0.6$ & 23,154\\
Leo I & $3.7\pm0.2$ & 1,242\\
Sculptor & $9.4\pm0.3$ & 6,888\\
Leo II & $2.4\pm0.3$ & 347\\
Carina & $8.7\pm0.4$ & 2,389\\
Sextans I & $20.2 \pm0.9$ & 1,830\\
Ursa Minor & $11.7 \pm 0.5$ & 2,122\\
Draco & $7.3\pm0.3$ & 1,781\\
\bottomrule
\end{tabular}
\end{table*}

Using the same methods above, we select members from J+24's stellar
probabilities. We use the one-component exponential density profiles
with

Figure: Density profiles for each dwarf galaxy. Here, we use the
1-component exponential stellar probabilities from J+24. Dwarf galaxies
are scaled by our derived \(R_h\) values.

\subsection{Density Profile Reliability and
Uncertainties}\label{density-profile-reliability-and-uncertainties}

\begin{itemize}
\tightlist
\item
  How well do we know the density profiles?
\item
  What uncertainties affect derived density profiles?
\item
  Can we determine if Gaia, structural, or algorithmic systematics
  introduce important errors in derived density profiles?
\item
  Using J+24 data, we validate

  \begin{itemize}
  \tightlist
  \item
    Check that PSAT, magnitude, no-space do not affect density profile
    shape too significantly
  \end{itemize}
\item
  Our ``high quality'' members all have \textgreater{} 50 member stars
  and do not depend too highly on the spatial component, mostly
  corresponding to the classical dwarfs
\end{itemize}

J+24's algorithm takes spatial position into account, assuming either a
one or two component exponential density profile. When deriving a
density profile, this assumption may influence the derived density
profile, especially when the galaxy density is fainter than the
background of similar appearing stars. To remedy this and estimate where
the background begins to take over, we also explore a cut based on the
likelihood ratio of only the CMD and PM components. This is in essence
assuming that the spatial position of a star contains no information on
it's membership probability (a uniform distribution like the background)

\subsection{Caveats}\label{caveats}

The J+24 method was designed to determine high probability members for
spectroscopic followup in particular. Note that we instead care about
retrieving a reliable density profile.

In particular, in fig.~\ref{fig:umi_observed_profiles}, notice that the
PSAT method produces artifically small errorbars even when the density
is \textgreater1dex below the local background. These stars are likely
selecting stars from the statistical MW background consistent with UMi
PM / CMD, recovering the assumed density profile. As a result, the
reliability of faint features in these density profiles is questionable
and a more robust analysis, removing this particular density assumtion,
would be required to more appropriately represent the knowledge of the
density profile as the background begins to dominate.

J+24 do not account for structural uncertainties in dwarfs. This is a
not insignificant source of uncertainty in the derived density profile

We assume constant ellipticity and position angle. Dwarf galaxies, in
reality, are not necessarily smooth and constant.

\section{Comparison and conclusions}\label{comparison-and-conclusions}

To illustrate the differences between each dwarf galaxy, in
fig.~\ref{fig:classical_dwarfs_densities}, we compare Scl, UMi, and Fnx
against exponential and plummer density profiles (\textbf{TODO: state
these somewhere}). While all dwarfs have marginal differences in the
inner regions, each dwarf diverges in the outer regions relative to an
exponential. In particular, while Fnx is underdense, Scl and UMi are
both overdense, approximately fitting a Plummer density profile instead.

In summary, we have used J+24 data to derive the density profiles for
Fornax, Sculptor, and Ursa Minor. In each case, the density profile is
robust against different selection criteria. Both Sculptor (Ursa Minor)
show strong (weak) evidence for deviations from an exponential profile.
We will explore a tidal explanation for these features in this work.

\begin{figure}
\centering
\includegraphics{figures/scl_umi_fornax_exp_fit.pdf}
\caption[Classical dwarf density profiles]{The density profiles of
Sculptor, Ursa Minor, and Fornax compared to Exp2D and Plummer density
profiles. Dwarf galaxies are scaled to the same half-light radius and
density at half-light radius (fit from the inner 3 scale radii
exponential recursively. )}\label{fig:classical_dwarfs_densities}
\end{figure}

\section{Appendix / Extra Notes}\label{appendix-extra-notes}

\subsection{Additional density profile
tests}\label{additional-density-profile-tests}

\begin{figure}
\centering
\includegraphics{figures/scl_density_methods_extra.pdf}
\caption[Density profiles]{Density profiles for various assumptions for
Sculptor. PSAT is our fiducial 2-component J+24 sample, circ is a
2-component bayesian model assuming circular radii, simple is the series
of simple cuts described in Appendix ?, bright is the sample of the
brightest half of stars (scaled by 2), DELVE is a sample of RGB stars
(background subtracted and rescaled to
match).}\label{fig:sculptor_observed_profiles}
\end{figure}

Note that a full rigorous statistical analysis would require a
simulation study of injecting dwarfs into Gaia and assessing the
reliability of various methods of membership and density profiles. This
is beyond the scope of this thesis.

\begin{verbatim}
SELECT TOP 1000
       *
FROM delve_dr2.objects
WHERE 11 < ra
and ra < 19
and -37.7 < dec
and dec < -29.7
\end{verbatim}
