\section{Gaia Membership Selection}\label{gaia-membership-selection}

Gaia provides unprecedented accuracy in proper motions and magnitudes.
Gaia data is uniquely excellent to produce low-contamination samples of
likely member stars belonging to satellites. Here, we breifly describe
J+24's membership estimation and discuss how this informs our
observational knoledge of each galaxies density profile. In general,
J+24 use a Bayesian framework incorporating proper motion (PM),
colour-magnitude diagram (CMD), and spatial information to determine the
probability that a given star belongs to the satellite. J+24 extends
\citet{MV2020a} (see also \citet{pace+li2019}, \citet{battaglia+2022},
\citet{pace+erkal+li2022}, etc.).

J+24 select stars initially from Gaia within a 2--4 degree circular
region centred on the dwarf. J+24 only consider stars with:

\begin{itemize}
\tightlist
\item
  Solved parallax, proper motions, colour, and magnitudes.
\item
  High quality astrometry (\texttt{ruwe\ \textless{}=\ 1.3}).
\item
  3\(\sigma\) consistency of measured parallax with dwarf distance +
  uncertainty (typically near zero; with \citet{lindegren+2018}
  zero-point correction).
\item
  Absolute RA and Dec proper motions less than
  10\(\,{\rm mas\ yr^{-1}}\) (corresponding to tangental velocities of
  \(\gtrsim 500\) km/s at distances larger than 10 kpc.).
\item
  No colour excess (\citet{lindegren+2018} equation C.2).
\item
  22 \textgreater{} G \textgreater{} 5\(\sigma\) brighter than TRGB (in
  practice astrometry ruwe =\textgreater{} 21 \textgreater{} G).
\item
  Between -0.5 and 2.5 in Bp - Rp.
\end{itemize}

Photometry is dereddened using \citet{schlegel+1988} extinction maps.

J+24 calculate the probability that any star belongs to either the
satellite or the MW background with \[
P_{\rm sat} = \frac{f_{\rm sat}{\cal L}_{\rm sat}}{f_{\rm sat}{\cal L}_{\rm sat} + (1-f_{\rm sat}){\cal L}_{\rm bg}}
\]

where \(f_{\rm sat}\) is the fraction of candidate members in the field,
and \({\cal L}_{\rm sat}\) and \({\cal L}\)\_\{\rm bg\}\$ are the
satellite (sat) and background (bg) likelihoods respectively. Each
likelihood is the product of a spatial, proper motion, and CMD term: \[
{\cal L} = {\cal L}_{\rm space}\ {\cal L}_{\rm PM}\ {\cal L}_{\rm CMD}.
\]

The satellite likelihoods are specified as

\begin{itemize}
\tightlist
\item
  CMD: The CMD is the lowest metallicity isochrone from Padova
  \citep{girardi+2002} with age 12 Gyr with a colour width of 0.1 mag
  plus the Gaia colour uncertainty at each magnitude. The HB is modelled
  as a constant magnitude extending blue of the CMD. The HB magnitude is
  the mean magnitude of HB stars from most metal poor isochrone with a
  0.1 mag width plus the mean colour error. A likelihood map is
  constructed by sampling the distance modulus in addition to the CMD
  width, taking the maximum of RGB and HB likelihoods.
\item
  Spatial: A single exponential
  (\(\Sigma \propto e^{R_{\rm ell} / R_s}\)) accounting for structural
  uncertainty (sampled over position angle, ellipticity, and half light
  radius).
\item
  Alternative spatial: For Scl and UMi, this is instead a double
  exponential
  \(\Sigma_\star \propto e^{-R/R_s} + B\,e^{-R/R_{\rm outer}}\) where
  the inner exponential remains fixed. Structural parameter
  uncertainties are not accounted for.
\item
  PM. A bivariate gaussian with variance and covariance equal to each
  star's proper motions. Each star's proper motions uncertainty are
  assumed to be the dominant uncertainty.
\end{itemize}

The background likelihoods are constructed as:

\begin{itemize}
\tightlist
\item
  CMD : Constructed as a density map using the other quality-selected
  stars outside of \(5R_h\) in the catalogue. The map is a sum of
  bivariate gaussians for each star with standard deviations based on
  the Gaia uncertainties.
\item
  PM: same as CMD except in PM space (\(\mu_{\alpha*}\) and
  \(\mu_\delta\) with respecive uncertainties and covariance.)
\item
  Spatial: a constant likelihood.
\end{itemize}

Each likelihood map is normalized over the respective 2D parameter
space. \(f_{\rm sat}\) thus is the only term controlling the total
abundance of satellite stars relative to the background.

In J+24, a MCMC simulation is ran using the above total likelihood to
solve for the following parameters:

\begin{itemize}
\tightlist
\item
  Systemic proper motions \(\mu_\alpha\), \(\mu_\delta\). Single
  component prior is same as \citet{MV2020}: a normal distribution with
  mean 0 and standard deviation 100 km/s. If 2-component spatial,
  instead is a uniform distribution spanning 5\(\sigma\) of single
  component case w/ systematic uncertainties.
\item
  \(f_{\rm sat}\) density normalization. Prior is a uniform distribution
  between 0 and 1.
\item
  Spatial component parameters \(B\) is uniform from 0-1 and
  \(R_{\rm outer}\) is uniform and greater than \(R_s\) for extended
  profiles (Scl and UMi here.)
\end{itemize}

The mode of each parameter from the MCMC are then used to calculate the
final \(P_{\rm sat}\) values we use here.

We adopt a probability cut of \(P_{\rm sat} = 0.2\) as our fiducial
sample. Most stars are assigned probabilities close to 0 or 1, so the
choice of probability threshhold is not too significant. Additionally,
even for a probability cut of 0.2, the purity of the resulting sample
with RV measurements is very high (\textasciitilde90\%, J+24). However,
there is likely a systematic bias in using stars with RV measurements to
measure purity. Fainter stars have poorer astrometry and are less likely
to have been targeted. We find no difference in the resulting density
distributions when restricting stars to be brighter than a specific
magnitude.

\subsection{Selected samples}\label{selected-samples}

In figures fig.~\ref{fig:sculptor_selection},
fig.~\ref{fig:umi_selction}, we illustrate the resulting samples from
the algorithm in the tangent plane (\(\xi\), \(\eta\), \emph{does this
need defined?}), CMD, and proper motion space. For both galaxies, each
criteria plays a commensurate role in sifting out members. Proper
motions are clustered around the dwarf systemic motion, the CMD is well
defined including the horizontal branch, and stars only within a few
\(R_h\) are included. We also mark stars with consistent radial
velocities (RVs) (see below), which trace each feature albeit more
cautiously.

Figure REF also illustrats the distribution of stars selected without a
spatial criterion. We define the CMD+PM selection as stars satisfying \[
{\cal L}_{\rm CMD,\ sat}\ {\cal L}_{\rm PM,\ sat} > {\cal L}_{\rm CMD,\ bg}\ {\cal L}_{\rm PM,\ bg}
\] These stars are distributed similar to the fiducial (probable
members) sample but with a fainter uniform distribution across the
entire field. This illustrates the approximate background of stars which
may be confused as members. Additionally, since there is no clear
spatial structure in the CMD+PM sample, it is unlikely that there are
additional faint, tidal features detectable with Gaia. Not shown here,
we also try a variety of simpler, absolute cuts and thresholds, finding
no extended structure beyond what is detected in J+24. (\emph{Is it
worth using this to calculate an OOM upper limit on the density of tidal
tails?}).

\begin{figure}
\centering
\pandocbounded{\includegraphics[keepaspectratio]{figures/scl_selection.pdf}}
\caption[Sculptor selection criteria]{The selection criteria for Scl
members. Probable members (2-component) are orange, and all field stars
(satisfying quality criteria) are in light grey. \textbf{Top:} Tangent
plane. We outline in star symbols the two stars from
\citet{sestito+2023a}. \textbf{Bottom left:} Colour magnitude diagram.
\textbf{Bottom right:} Proper motion.}\label{fig:sculptor_selection}
\end{figure}

\begin{figure}
\centering
\pandocbounded{\includegraphics[keepaspectratio]{figures/umi_selection.pdf}}
\caption[Ursa Minor Selection]{Similar to
fig.~\ref{fig:sculptor_selection} except for Ursa Minor. We outline RV
members outside of \(3R_h\) in black stars (from \citet{sestito+2023b}
and \citet{pace+2020} and
\citet{spencer+2018}).}\label{fig:umi_selection}
\end{figure}

\subsection{Density Profiles}\label{density-profiles}

Density profiles are an essential observational constraint for our later
simulations. To derive density profiles, we use 0.05 dex bins in log
radius (i.e.~the bins are derived from
10\^{}(minimum(logR):0.05:maximum(logR))). We remove bins at smaller
(larger) radii than the first bin to contain no stars, working outwards.
We use poisson uncertainties
(\(\delta \Sigma_i / \Sigma_i = 1/\sqrt{N_i}\) for a derived density
\(\Sigma_i\) in a bin with \(N_i\) stars). As discussed below, these
uncertainties are straightforward but are likely under-represented.

fig.~\ref{fig:scl_observed_profiles},
fig.~\ref{fig:umi_observed_profiles} show the derived density profiles
for Scl and UMi. We calculate density profiles for different selections
of stars from above: all quality stars, CMD+PM only, and the probable
member samples. In each case, all samples are the same towards the inner
regions of the satellite. Classical dwarfs dominate the stellar density
in their cores. However, the density profile \emph{all stars} plateaus
at the total background in the field at radii of 30-60 arcminutes. By
restricting stars to being most likely satellite members by CMD + PM,
the background is reduced by 1-2 dex. The CMD+PM background plateau
likely represents the density of background stars which could be
mistaken as members. Finally, the background (BG)-subtracted profile
results from subtracting the apparent background in the \emph{all}
profile.

Note that the probable members (fiducial) density profile continues to
confidently estimate the density profile below the CMD+PM background.
These points are likely unreliable (see discussion below). However,
before this point, both the BG subtracted and probable members density
profiles are strikingly similar. Assumptions about the details of the
likelihood and spatial dependence have marginal influence on the
resulting density profile when the satellite is higher density than the
background. Thus the detection of excesses of stars in Sculptor (past 20
arcmin) and UMi (past 15-20 arcmin) are robust.

One interesting note is for UMi, there is a small ultrafaint star
cluster at a position of (-100, -15) arcmin \citet{munoz+2012}. However
this cluster does not have a bright RGB so has few stars brighter than g
mag of 22. While munoz 1 may contribute about 4 stars to the density
profile of UMi, this would only apply to the bin at XXX because of the
small half light radius. We do not subtract this out.

\begin{figure}
\centering
\pandocbounded{\includegraphics[keepaspectratio]{figures/scl_density_methods.pdf}}
\caption[Sculptor density profiles]{The density profile of Sculptor for
different selection criteria. \emph{probable members} selects stars with
PSAT \textgreater{} 0.2 considering PM, CMD, and spatial, \emph{CMD+PM}
select stars more likely to be members according to CMD and PM only,
\emph{all} selects any high quality star, and \emph{BG subtracted} is
the background-subtracted density derived from high-quality
stars.}\label{fig:scl_observed_profiles}
\end{figure}

\begin{figure}
\centering
\pandocbounded{\includegraphics[keepaspectratio]{figures/umi_density_methods.pdf}}
\caption[Ursa Minor density profiles]{Similar to
fig.~\ref{fig:scl_observed_profiles} except for Ursa
Minor.}\label{fig:umi_observed_profiles}
\end{figure}

\begin{figure}
\centering
\pandocbounded{\includegraphics[keepaspectratio]{figures/fornax_density_methods.pdf}}
\caption[Fornax density profiles]{Similar to
fig.~\ref{fig:scl_observed_profiles} except for
Fornax.}\label{fig:fornax_observed_profiles}
\end{figure}

\subsection{Caveats}\label{caveats}

The J+24 method was designed in particular to detect the presence of a
density excess and find individual stars at large radii to be followed
up. We are more interested in accurately quantifying the density profile
and size of any perturbations. One potential problem with using J+24's
candidate members is that the algorithm assumes the density is either
described by a single or double exponential. If this model does not
accurately match the actual density profile of the dwarf galaxy, we
would like to understand how strongly influenced the density profile is
by this assumption.

In particular, in fig.~\ref{fig:umi_observed_profiles}, notice that the
PSAT method produces artifically small errorbars even when the density
is \textgreater1dex below the local background. These stars are likely
selecting stars from the statistical MW background consistent with UMi
PM / CMD, recovering the assumed density profile. As a result, the
reliability of these density profiles below the CMD+PM background may be
questionable. A more robust analysis, removing this particular density
assumption, would be required to more appropriately represent the
knowledge of the density profile as the background begins to dominate.

J+24 do not account for structural uncertainties in dwarfs for the two
component case. We assume constant ellipticity and position angle. Dwarf
galaxies, in reality, are not necessarily smooth and constant. They do
test an alternative method using circular radii for the extended density
component, and we find these density profiles are very similar. We also
assume a constant ellipticity and position angle here.

\section{Radial velocity modeling}\label{radial-velocity-modeling}

For both Sculptor and Ursa Minor, we construct literature samples of
radial velocity measurements. We combine these samples with J+24's
members to produce RV consistent stars and to compute velocity
dispersion, systematic velocities, and test for the appearance of
velocity gradients.

First, we crossmatch all catalogues to J+24 Gaia stars. If a study did
not report GaiaDR3 source ID's, we match to the nearest star within 2
arcseconds. We exclude stars not matched to Gaia for simplicity.

We combine the mean RV measurement from each study using the
inverse-variance weighted mean and standard uncertainty. \[
\bar v = \frac{1}{\sum w_i}\sum_i w_i\ v_{i} 
\] \[
\delta v = \sqrt{\frac{1}{\sum_i w_{i}^2}}
\]

where \(w_i = 1/s_i^2\), and we estimate the inter-study standard
deviation \[
s^2 = \frac{1}{\sum w_i} \sum_i w_i (v_{i} - \bar v)^2
\] We remove stars with significant velocity dispersions as measured
between or within a study: \[
P\left(\chi2(N-1) < \frac{s^2}{\delta v^2}\right) > 0.001
\] where \(s, \delta v, N\) are the weighted standard deviation,
weighted standard error, and number of observations. We apply this cut
to both within a study and between studies, which typically removes
stars with reduced \(\chi^2 \gtrsim 2\).

The combined RV likelihood is then \[
{\cal L} = {\cal L}_{\rm space} {\cal L}_{\rm CMD} {\cal L}_{\rm PM} {\cal L}_{\rm RV}
\] where \[
{\cal L}_{\rm RV, sat} = N(\mu_{v}, \sigma_{v}^2 + (\delta v_i)^2)
\] \[
{\cal L}_{\rm RV, bg} = N(0, \sigma_{\rm halo}^2)
\]

where \(\mu_v\) and \(\sigma_v\) are the systemic velocity and
dispersion of the satellite, and \(\delta v_i\) is the individual
measurement uncertainty. Typically, the velocity dispersion will
dominate the uncertainty budget here. We assume a halo/background
velocity dispersion of a constant \(\sigma_{\rm halo} = 100\) km/s
\citep[e.g.][]{brown+2010}.

Similar to above, we retain stars with the resulting membership
probability of greater than 0.2.

Finally, we need to correct the coordinate frames for the solar motion
and on-sky size of the galaxy. The first step is to subtract out the
solar motion from each radial velocity, corresponding to a typical
gradient of \textasciitilde3 km/s across the field. The next step is to
account for the slight differences in the direction of each radial
velocity. Define the \(\hat z\) direction to point parallel to the
direction from the sun to the centre of the dwarf galaxy. Then if
\(\phi\) is the angular distance between the centre of the galaxy and
the individual star, the corrected radial velocity is then \[
v_z = v_{\rm los, gsr}\cos\phi  - v_{\alpha}\cos\theta \sin\phi - v_\delta \sin\theta\sin\phi
\] where
\(v_{\rm tan, R} = d(\mu_{\alpha*}\cos\theta + \mu_\delta \sin \theta)\)
is the radial component of the proper motion with respect to the centre
of the galaxy.This correction is of the order \(v_{\rm tan}\theta\) so
induces a gradient of about \(1 km/s/degree\) for sculptor
\citep[see][]{WMO2008}. The uncertainty is then the velocity uncertainty
plus the distance uncertainties times the PM uncertainty from above. We
then use the \(v_z\) values for the following modelling, however
repeating with uncorrected, heliocentric velocities does not
significantly affect the results .

For the priors on the satellite velocity dispersion and systematic
velocity, we use \[
\mu_{v} = N(0, \sigma_{\rm halo}^2) \\
\sigma_{v} = U(0, 20\,{\rm km\,s^{-1}})
\] where \(\sigma_{\rm halo} = 100\,{\rm km\,s^{-1}}\) is the velocity
dispersion of the MW halo adopted above, a reasonable assumption for
dwarfs in orbit around the MW.

\subsection{}\label{section}

\begin{table*}[t]
\centering
\caption{Summary of velocity measurements and derived properties. sestito+2023a number of members depends on spatial model used.}
\label{tbl:Summary-of-velocity-measurements-and-derived-properties-sestito+2023a-number-of-members-depends-on-spatial-model-used}
\begin{tabular}{llllllll}
\toprule
 & Study & Instrument & Nspec & Nstar & Ngood & Nmemb & $\delta v_{\rm med}$\\
\midrule
Scl & combined &  & 8945 & 2280 & 2034 & 1920 & 0.9\\
 & tolstoy+23 & FLAMES & 3311 & 1701 & 1522 & 1481 & 0.65\\
 & sestito+23a & GMOS & 2 & 2 & 2 & 2 & 13\\
 & walker+09 & MMFS & 1818 & 1522 & 1417 & 1330 & 1.8\\
 & APOGEE & APOGEE & 5082 & 253 & 102 & 98 & 0.6\\
UMi & combined &  & 4714 & 1225 & 1148 & 831 & 2.3\\
 & sestito+23b & GRACES & 5 & 5 & 5 & 5 & 1.8\\
 & pace+20 & DEIMOS & 1716 & 1538 & 829 & 682 & 2.5\\
 & spencer+18 & Hectoshell & 1407 & 970 & 596 & 406 & 0.9\\
 & APOGEE & APOGEE & 9500 & 279 & 37 & 32 & 0.9\\
\bottomrule
\end{tabular}
\end{table*}

\subsection{Sculptor}\label{sculptor}

\begin{figure}
\centering
\pandocbounded{\includegraphics[keepaspectratio]{figures/scl_rv_2dhist.pdf}}
\caption[Scl velocity sample]{A plot of the corrected los velocities for
Scl binned in tangent plane coordinates. We detect a slight rotational
gradient towards the bottom right.}
\end{figure}

\begin{figure}
\centering
\pandocbounded{\includegraphics[keepaspectratio]{figures/scl_vel_gradient_binned.pdf}}
\caption[Scl velocity gradient]{A velocity gradient in Sculptor! The
arrow marks the gradient induced by Scl's proper motion on the sky.
\textbf{TODO}: plot of velocity of Scl in direction of gradient.}
\end{figure}

For Sculptor, we combine radial velocity measurements from APOGEE,
\citet{sestito+2023a}, \citet{tolstoy+2023}, and \citet{WMO2009}.
\citet{tolstoy+2023} and \citet{WMO2009} provide the bulk of the
measurements. We find that there is no significant velocity shift in
crossmatched stars between catalogues. After crossmatching to Gaia and
excluding significant inter-study dispersions, we have a sample of XXXX
members.

We derive a systemic velocity of \(111.2\pm0.2\) km/s with velocity
dispersion \(9.67\pm0.16\) km/s. Our values are very consistent with
previous work \citep[e.g.][\citet{arroyo-polonio+2024},
\citet{tolstoy+2023}]{WMO2009}.

Finally, we attempt to fit a linear velocity gradient to Scl by adding
parameters for the gradient in \(\xi\) ande \(\eta\). We derive a
gradient of xxx and xxx (see Figure\textasciitilde FIG.), noting that
this direction is different and higher in magnitude than the proper
motion of Scl. Comparing the density difference, we find a bayes factor
of -15. Compared to past work, \citet{battaglia+2008},
\citet{arroyo-polonio+2021}.

Bayes factor found using relative density via KDE with Silverman's
bandwith to prior (XXX theorem).

\emph{Is it interesting that the velocity dispersion of Scl seems to
increase significantly with Rell?}

\begin{table*}[t]
\centering
\caption{MCMC fits for different RV datasets for Scl amoung 3 different models.}
\label{tbl:MCMC-fits-for-different-RV-datasets-for-Scl-amoung-3-different-models}
\begin{tabular}{lllllll}
\toprule
study & mean & sigma & $\partial \log\sigma / \partial \log R$ & $\partial v_z / \partial x$ (km/s/deg) & $\theta_{\rm grad} / ^{\circ}$ & $\log bf$\\
\midrule
all &  &  &  &  &  & \\
 & $111.17\pm0.22$ & $9.68\pm0.17$ & - & - & - & 0\\
 & $111.18 \pm 0.23$ & $9.64\pm0.16$ & - & $4.8\pm1.3$ & $-147_{-12}^{+15}$ & -3.7*\\
 & $111.15\pm0.22$ & $9.70\pm0.16$ & $0.055\pm0.021$ & - & - & -0.8\\
tolstoy+23 &  &  &  &  &  & \\
 & $111.2 \pm 0.3$ & $9.77 \pm 0.18$ & - & - & - & 0\\
 & $111.2\pm0.3$ & $9.73\pm0.18$ & – & $4.8\pm1.4$ & $-154_{-12}^{+16}$ & -2.5\\
 & $111.2 \pm 0.3$ & $9.71\pm0.18$ & $0.081 \pm 0.023$ & – & – & -3.3\\
walker+09 &  &  &  &  &  & \\
 & $111.0\pm0.3$ & $9.57\pm0.21$ & – & – & – & 0\\
 & $111.1\pm0.3$ & $9.54\pm0.21$ & - & $5.3_{-1.6}^{+1.8}$ & $-134_{-16}^{+22}$ & -2.0\\
 & $111.0\pm0.3$ & $9.61\pm0.21$ & $0.03\pm0.03$ & – & – & +1.6\\
apogee &  &  &  &  &  & \\
 & $109.9\pm0.8$ & $8.3\pm0.6$ & – & – & – & –\\
 & $109.9\pm0.8$ & $8.3\pm0.6$ & – & $6\pm3$ & $-151_{-36}^{+44}$ & +0.3\\
 & $109.9\pm0.8$ & $8.3\pm0.7$ & $0.05\pm0.08$ & – & – & +1.1\\
\bottomrule
\end{tabular}
\end{table*}

\subsection{Ursa Minor}\label{ursa-minor}

\begin{figure}
\centering
\pandocbounded{\includegraphics[keepaspectratio]{figures/umi_rv_2dhist.pdf}}
\caption[UMi velocity sample]{UMi velocities}
\end{figure}

Figure: A plot of the corrected los velocities for UMi binned in tangent
plane coordinates. There is no clear rotation or velocity gradient here.
Interestingly, many velocity members are as far as 100 arcmin away.

For UMi, we collect radial velocities from, APOGEE,
\citet{sestito+2023b}, \citet{pace+2020}, and \citet{spencer+2018}.

We shifted the velocities of \citet{spencer+2018} (\(-1.1\) km/s) and
\citet{pace+2020} (\(+1.1\) km/s) to reach the same scale. We found 183
crossmatched common stars (passing 3\(\sigma\) RV cut, velocity
dispersion cut, and PSAT J+24 \textgreater{} 0.2 w/o velocities). Since
the median difference in velocities in this crossmatch is about 2.2
km/s, we adopt 1 km/s as the approximate systematic error here.

UMi interestingly has a more structured observational pattern, but does
not appear to have any significant velocity substructure.

\begin{table*}[t]
\centering
\caption{MCMC fits for UMi velocity dispersion.}
\label{tbl:MCMC-fits-for-UMi-velocity-dispersion}
\begin{tabular}{lllll}
\toprule
study & mean & sigma & $\log bf_{\rm sigma}$ & $\log bf_{\rm grad}$\\
\midrule
all & $-245.9\pm0.3$ & $8.76\pm0.24$ & +1.5 & +2.2\\
pace & $-244.5\pm0.4$ & $9.1\pm0.3$ & +0.2 & +1.1\\
spencer & $-246.9\pm0.4$ & $8.8\pm0.3$ & +1.8 & -0.3\\
apogee & $-248.2\pm1.6$ & $9.0_{-1.1}^{+1.3}$ & +0.8 & +0.8\\
\bottomrule
\end{tabular}
\end{table*}

\subsection{Discussion and
limitations}\label{discussion-and-limitations}

Our model here is relatively simple. Some things which we note as
systematics and are challenging to account fully for are

\begin{itemize}
\tightlist
\item
  Inter-study systematics and biases. While basic crossmatches and a
  simple velocity shift, combining data from multiple instruments is
  challenging.
\item
  Inappropriate uncertainty reporting. Inspection of the variances
  compared to the standard deviations within a study seems to imply that
  errors are accurately reported. APOGEE notes that their RV
  uncertainties are known to be underestimates but are a small
  proportion of our sample.
\item
  Binarity. While not too large of a change for classical dwarfs, this
  could inflate velocity dispersions of about 9 km/s by about 1 km/s
  \citet{spencer+2017}. Thus, our measurement is likely slightly
  inflated given the high binarity fractions measured in classical
  dwarfs \citep[\citet{spencer+2018}]{arroyo-polonio+2023}.
\item
  Selection effects. RV studies each have their own selection effects. I
  do not know how to correct for this.
\end{itemize}

Because the derived parameters are similar for the two different larger
surveys we consider for UMi and Scl, we note that many of these effects
are likely not too significant (with the exception of the systemic
motion of UMi.)

\section{Comparison and conclusions}\label{comparison-and-conclusions}

To illustrate the differences between each dwarf galaxy, in
fig.~\ref{fig:classical_dwarfs_densities}, we compare Scl, UMi, and Fnx
against exponential and plummer density profiles (\textbf{TODO: state
these somewhere}). While all dwarfs have marginal differences in the
inner regions, each dwarf diverges in the outer regions relative to an
exponential. In particular, while Fnx is underdense, Scl and UMi are
both overdense, approximately fitting a Plummer density profile instead.

In summary, we have used J+24 data to derive the density profiles for
Fornax, Sculptor, and Ursa Minor. In each case, the density profile is
robust against different selection criteria. Both Sculptor (Ursa Minor)
show strong (weak) evidence for deviations from an exponential profile.
We also compile velocity measurements to derive the systemic motions and
velocity dispersions of each galaxy. We find evidence for a velocity
gradient in Scl of XXX. We find no evidence of additional (velocity or
stellar) substructure in either galaxy. Our goal is thus to explain why
Scl and UMi have an excess of stars in their outer regions and why Scl
may have a velocity gradient.

\begin{figure}
\centering
\pandocbounded{\includegraphics[keepaspectratio]{figures/scl_umi_fornax_exp_fit.pdf}}
\caption[Classical dwarf density profiles]{The density profiles of
Sculptor, Ursa Minor, and Fornax compared to Exp2D and Plummer density
profiles. Dwarf galaxies are scaled to the same half-light radius and
density at half-light radius (fit from the inner 3 scale radii
exponential recursively. )}\label{fig:classical_dwarfs_densities}
\end{figure}

\section{Appendix / Extra Notes}\label{appendix-extra-notes}

\subsection{Additional density profile
tests}\label{additional-density-profile-tests}

\begin{figure}
\centering
\pandocbounded{\includegraphics[keepaspectratio]{figures/scl_density_methods_extra.pdf}}
\caption[Density profiles]{Density profiles for various assumptions for
Sculptor. PSAT is our fiducial 2-component J+24 sample, circ is a
2-component bayesian model assuming circular radii, simple is the series
of simple cuts described in Appendix ?, bright is the sample of the
brightest half of stars (scaled by 2), DELVE is a sample of RGB stars
(background subtracted and rescaled to
match).}\label{fig:sculptor_observed_profiles}
\end{figure}

Note that a full rigorous statistical analysis would require a
simulation study of injecting dwarfs into Gaia and assessing the
reliability of various methods of membership and density profiles. This
is beyond the scope of this thesis.

\begin{verbatim}
SELECT TOP 1000
       *
FROM delve_dr2.objects
WHERE 11 < ra
and ra < 19
and -37.7 < dec
and dec < -29.7
\end{verbatim}
