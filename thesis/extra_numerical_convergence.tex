\chapter{Numerical convergence and
parameters}\label{sec:extra_convergence}

Here, we describe some convergence tests to ensure our methods and
results are minimally impacted by numerical limitations and assumptions.
See \citet{power+2003} for a detailed discussion of various assumptions
and parameters used in N-body simulations.

\section{Softening}\label{softening}

One challenge of N-body integration is close, collisional encounters,
assuming point particles, cause divergences in the local force. However,
this should not occur in a \emph{collisionless} simulation. As a result,
most collisionless N-body codes adopt a gravitational softening, a
length scale below which the force of gravity begins to decrease between
point particles.

\citet{power+2003} empirically suggest that the ideal softening is
\begin{equation}{
h_{\rm grav} = 4 \frac{R_{200}}{\sqrt{N_{200}}},
}\end{equation}

where \(h_{\rm grav}\) is the softening length, and \(N_{200}\) is the
number of particles within \(R_{200}\). This choice balances integration
time and only compromises resolution in collisional regime.

For our isolation halo (\(M_s=2.7\), \(r_s=2.76\)) and with \(10^7\)
particles, this works out to be \(0.044\,{\rm kpc}\).We adpoted the
slightly smaller softening which was reduced by a factor of
\(\sqrt{10}\) which appears to improve agreement slightly in the
innermost regions.

\begin{figure}
\centering
\pandocbounded{\includegraphics[keepaspectratio]{figures/iso_converg_softening.png}}
\caption{Softening convergence}\label{fig:softening_convergence}
\end{figure}

\section{Time stepping and force
accuracy}\label{time-stepping-and-force-accuracy}

In general, we use adaptive timestepping and relative opening criteria
for gravitational force computations. To verify that these choices and
associated accuracy parameters minimally impact convergence or speed, we
show a few more isolation runs (using only 1e5 particles)

\begin{itemize}
\tightlist
\item
  constant timestep (\ldots), approximantly half of minimum timestep
  with adaptive timestepping
\item
  geometric opening, with \(\theta = 0.5\).
\item
  strict integration accuracy, (facc = \ldots.)
\end{itemize}

\begin{figure}
\centering
\pandocbounded{\includegraphics[keepaspectratio]{figures/iso_converg_methods.png}}
\caption{Isolation method convergence}\label{fig:methods_convergence}
\end{figure}
