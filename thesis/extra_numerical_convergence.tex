\chapter{Numerical convergence and
parameters}\label{sec:extra_convergence}

Here, we describe the numerical specifics of our simulation and
analysis. We present convergence tests supporting our choices of
softening and additional parameters. We find alternate choices of
numerical parameters are unlikely to improve our convergence or result
in significantly different evolution.

\section{Softening}\label{softening}

One challenge of N-body integration is close, collisional encounters,
assuming point particles, cause divergences in the local force. However,
this should not occur in a \emph{collisionless} simulation. As a result,
most collisionless N-body codes adopt a gravitational softening, a
length scale below which the force of gravity begins to decrease between
point particles.

\citet{power+2003} empirically suggest that the ideal softening is
\begin{equation}{
h_{\rm grav} = 4 \frac{R_{200}}{\sqrt{N_{200}}},
}\end{equation}

where \(h_{\rm grav}\) is the softening length, and \(N_{200}\) is the
number of particles within \(R_{200}\). This choice balances integration
time and only compromises resolution in collisional regime. For our
isolation halo (\(\rmax=6\,\kpc\), \(\vmax=31\,\kms\)) with \(10^7\)
particles, this works out to be \(0.044\,{\rm kpc}\), about
\(\sqrt{10}\) times larger than our adopted softening
(Eq.~\ref{eq:softening_length}).

\begin{figure}
\centering
\pandocbounded{\includegraphics[keepaspectratio]{figures/iso_converg_softening.png}}
\caption[Softening convergence]{Similar to
Fig.~\ref{fig:numerical_convergence} except for simulations with
different softening lengths, \(h_{\rm grav}\). \(0.044\,\kpc\) is our
fiducial softening length for this halo. All simulations here were ran
in isolation of 5 Gyr with \(10^{6}\)
particles.}\label{fig:softening_convergence}
\end{figure}

Fig.~\ref{fig:softening_convergence} illustrates the influence of
softening length on isolation evolution. The simulation with
\(h=0.14\,\kpc\) deviates the most from the expected profile. Instead,
the final profiles for \(h=0.044\,\kpc\) and \(h=0.014\,\kpc\) are
nearly identical. Smaller values of softening than \(h=0.044\,\kpc\)
likely will not affect the evolution in the inner regions. As
computation time increases moderately with decreasing softening length,
\(h=0.044\,\kpc\) is an ideal choice balancing efficiency and accuracy.
Our simulations therefore adopt this choice (as in
Eq.~\ref{eq:softening_length}).

\section{Time-stepping and force
accuracy}\label{time-stepping-and-force-accuracy}

In general, we use adaptive time-stepping and relative opening criteria
for gravitational force computations. To verify that these choices and
associated accuracy parameters minimally impact convergence or speed, we
show a few more isolation runs (using only \(10^5\) particles).

We use in \gadget{} the relative tree opening criterion with the
accuracy parameter set to \(\alpha =0.005\) (so nodes are opened when
\(M\,l/r^3 < \alpha |a|\) for a distance from the node of mass \(M\) of
\(r\), \(l\) is the side length of the node, and \(a\) is the particle's
total acceleration), and adaptive time stepping with integration
accuracy set to \(\eta=0.01\) (particles must take time-steps smaller
than \(dt < \sqrt{2\,\eta\,h_{\rm grav} / a}\) for acceleration \(a\)).

Fig.~\ref{fig:methods_convergence} tests various additional simulation
parameters in isolation. We compare against an equivalent model ran
instead with \texttt{GADGET-2}, a model ran with constant timestep of
0.5Myr (\texttt{dt0.1} named after our dimensionless units), a
simulation with the time-step accuracy set to \(\eta=0.003\) (high
accuracy), and a simulation using a geometric opening criterian instead
of the force (geometric). In all cases, the final isolation velocity
profile is nearly the same after 5Gyr. The numerical details likely do
not limit our convergence (as marked by the arrow).

\begin{figure}
\centering
\pandocbounded{\includegraphics[keepaspectratio]{figures/iso_converg_methods.png}}
\caption[Isolation method convergence]{A comparison against different
integration methods. See text for details on each
model.}\label{fig:methods_convergence}
\end{figure}
