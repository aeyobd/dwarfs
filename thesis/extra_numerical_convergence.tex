\chapter{Numerical convergence and
parameters}\label{sec:extra_convergence}

Here, we describe the numerical specifics of our simulation and
analysis. We present convergence tests supporting our choices of
softening and additional parameters. We find that alternate choices of
numerical parameters neither improve convergence nor the resulting
evolution.

Due to the finite resolution of N-body simulations, collisional (close)
encounters between particles are inevitable. Such collisional encounters
are both unphysical (with possible arbitrary acceleration, violating the
``collisionless'' assumption) and computationally expensive (requiring
many small time-steps). As a remedy, many N-body codes use a
``softened'' gravitational force law, where the force weakens when
closer to a particle than a ``softening length''. The choice of
softening length ideally balances resolution and computational speed.

Empirically, \citet{power+2003} suggest that the ideal softening is
\begin{equation}\protect\phantomsection\label{eq:power_softening}{
h_{\rm grav} = 4 \frac{R_{200}}{\sqrt{N_{200}}},
}\end{equation}

where \(h_{\rm grav}\) is the softening length, and \(N_{200}\) is the
number of particles within \(R_{200}\). This choice balances integration
time and only compromises resolution in collisional regime. For our
isolation halo (\(\rmax=6\,\kpc\), \(\vmax=31\,\kms\)) with \(10^7\)
particles, this works out to be \(0.044\,{\rm kpc}\). Next we consider
the effect of softening as applied to our isolation halo.

The two other key numerical parameters in \gadget{} are the tree-force
and time-step accuracy parameters. Specifically, \gadget{} opens a node
if \(M\,l/r^3 < \alpha |a|\), the node's mass is \(M\), distance from
the particle is \(r\), side-length is \(l\), and the particle's total
acceleration is \(a\). We adopt \(\alpha =0.005\). We also elect to use
adaptive time stepping with integration accuracy set to \(\eta=0.01\).
(particles must take time-steps smaller than
\(dt < \sqrt{2\,\eta\,h_{\rm grav} / a}\) for acceleration \(a\)).

The top panel of Fig.~\ref{fig:methods_convergence} illustrates the
influence of softening length for simulations of Scl's \smallperi{}
orbit with \(10^5\) particles. We show a model with the fiducial
softening length (\(h=0.14\,\kpc\)), and larger and smaller softenings
by a factor \(\sqrt{10}\). The larger softening length is consistent
with Eq.~\ref{eq:power_softening}'s prediction. With a larger softening
length, the halo diverges from the expectation (using \(100\times\) more
particles) across most radii. On the other hand, the fiducial and
smaller softening lengths predict similar final results, although the
smaller softening length simulation deviates slightly more than the
fiducial, likely a result of more collisional encounters. As computation
time increases moderately with decreasing softening length, the fiducial
softening length balances efficiency and accuracy for this simulation.
While smaller than \citet{power+2003}'s suggestion, our choice is
designed for more compact, dwarf satellite halos instead of large,
galaxy-mass isolated halos.

The lower panel of Fig.~\ref{fig:methods_convergence} tests changes to
these parameters. We include a model with smaller integration accuracy
(``small timestep'') and a stricter tree-force tolerance (``high acc.
force''). More precise tolerances on these parameters does not affect
the evolution within the uncertainties of the final profile.

As demonstrated in this subsection, stricter numerical accuracy and
modified softening lengths do not affect our results. We thus conclude
that our simulations are numerically well-converged (up to the
``convergence'' radius).

\begin{figure}
\centering
\pandocbounded{\includegraphics[keepaspectratio]{figures/orbit_converg_methods.png}}
\caption[Isolation method convergence]{A comparison of the final
profiles using different simulation methods for Sculptor's \smallperi{}
model. The benchmark model is our fiducial \(10^7\) particle run, and
all other models use \(10^5\) paritcles, with their ``converged radius''
marked by the black arrow and softening by the vertical bar.
\textbf{Top:} Models with \(\sqrt{10}\) larger and smaller softening
lengths. \textbf{Bottom:} Models with more precise timestep accuracy
(\(\eta=0.001\) compared to fiducial \(\eta=0.005\)) and gravitational
force accuracy (\(\alpha=0.003\) versus \(\alpha=0.01\)).
\textbf{Summary:} Except for the model with a larger softening length,
all simulations agree within uncertainties. \textbf{Todo: softening
along lower axis always\ldots, changing
linewidth?}}\label{fig:methods_convergence}
\end{figure}
