\section{Reliability and comparison of observed density
profiles}\label{reliability-and-comparison-of-observed-density-profiles}

\subsection{Caveats}\label{caveats}

While the J+24 method is excellent and has been verified, there are
several possible limitations we discuss here. In general, while these
limitations are likely real, they do not substantially affect our
conclusions up to where we suggest the density profiles are unreliable.

\textbf{Spatial likelihood.} J+24 method was designed in particular to
detect the presence of a density excess and find individual stars at
large radii to be followed up. We are more interested in accurately
quantifying the density profile and size of any perturbations. One
potential problem with using J+24's candidate members is that the
algorithm assumes the density is either described by a single or double
exponential. If this model does not accurately match the actual density
profile of the dwarf galaxy, we want to understand the impact of this
assumption.In particular, in Fig.~\ref{fig:scl_observed_profiles},
notice that the \(P_{\rm sat}\) selection method produces small
errorbars, even when the density is more than 1 dex below the local
background. These stars are likely selecting stars from the statistical
MW background consistent with UMi PM / CMD, recovering the assumed
density profile. As a result, the reliability of these density profiles
below the CMD+PM background may be questionable. A more robust analysis,
removing this particular density assumption, would be required to more
appropriately represent the knowledge of the density profile as the
background begins to dominate.

\textbf{Uncertainty misrepresentation.} A more self consistent model
would fit the density profile to the entire field at once, eliminating
possible misrepresentation of the uncertainties. In the Appendix to this
section, we \emph{will} also discuss an alternative method which runs a
MCMC model using the likelihoods above to solve for the density in each
elliptical bin. From these tests, we note that the density profile and
uncertainties derived from the J+24 sample are reliable insofar as the
dwarf density is above the background from MW CMD+PM-consistent
interlopers. We estimate that this effect comprimises the density
profiles past \(\log R / {\rm arcmin} = 1.8\) for Sculptor and Ursa
Minor, but agreement is good before then.

\textbf{\emph{Gaia} systematics}. While \emph{Gaia} has shown excellent
performance, some notable limitations may introduce problems in our
interpretation and reliability of density profiles. Gaia systematics in
proper motions and parallaxes are typically smaller than the values for
sources of magnitudes \(G\in[18,20]\). Since we use proper motions and
parallaxes as general consistency with the dwarf, and factor in
systematic uncertainties in each case, these effects should not be too
significant. However, the systematic proper motion uncertainties becomes
the dominant source of uncertainty in the derived systemic proper
motions of each galaxy (see
Tables~\ref{tbl:scl_obs_props}, \ref{tbl:umi_obs_props}).

\textbf{Completeness}. \emph{Gaia} shows high but imperfect
completeness, particularly showing limitations in crowded fields and for
faint sources (\(G\gtrapprox20\)). As discussed in
\citet{fabricius+2021}, for the high stellar densities in globular
clusters, the completeness relative to HST varies significantly with the
stellar density. However, the typical stellar densities of dwarf
galaxies are much lower, at about 20 stars/arcmin = 90,000 stars /
degree, lower than the lowest globular cluster densities and safely
below the crowding limit of 750,000 objects/degree for BP/RP photometry.
In \citet{fabricius+2021}, for the lowest density globular clusters, the
completeness down to \(G\approx 20\) is \(\sim 80\%\). Closely separated
stars pose problems for Gaia's on-board processing, as the pixel size is
59x177 mas on the sky. This results in a reduction of stars separated by
less than 1.5'' and especially for stars separated by less than 0.6
arcseconds. The astrometric parameters of closely separated stars
furthermore tends to be of lower quality \citep{fabricius+2021}.
However, even for the denser field of Fornax, only about 3\% of stars
have a neighbour within 2 arc seconds, so multiplicity should not affect
completeness too much (except for unresolved binaries). One potential
issue is that the previous analyses do not account for our cuts on
quality and number of astrometric parameters. These could worsen
completeness, particularly since the BP-RP spectra are more sensitive to
dense fields. In the appendix to this section, we test if magnitude cuts
impact the resulting density profiles, finding that this is likely not
an issue.

\textbf{Structural uncertainties}. J+24 do not account for structural
uncertainties in dwarfs for the two component case. We assume constant
ellipticity and position angle. Dwarf galaxies, in reality, are not
necessarily smooth or have constant ellipticity. J+24 test an
alternative method using circular radii for the extended density
component, and we find these density profiles are very similar to the
fully elliptical case, even when assuming circular bins for the circular
outer component. As such, even reducing the assumed ellipticity from
\(0.37-0.55\) to 0 does not substantially impact the density profiles.

\textbf{Outlook.} Finally, an excellent test of systematics and methods
in \emph{Gaia} is to compare against another survey. In the appendix, we
show that profiles derived from J+24 agree with DELVE or UNIONS data
within uncertainties. Systematics, completion effects, and selection
methods are unlikely to substantially change the density profiles
presented here.

\subsection{Comparison to literature}\label{comparison-to-literature}

A number of works have previously speculated that dwarf galaxies such as
Sculptor and Ursa Minor may have signatures of tidal perturbations,
however not without contention. While \citet{hodge1961} and
\citet{demers+krautter+kunkel1980} were some of the earliest work to
derive the density profile of Sculptor, \citet{innanen+papp1979} was
perhaps the first to speculate that Sculptor may harbour a substantial
population of ``extratidal stars'' (stars beyond the tidal radius),
finding candidate members out to 180' in an elongated distribution from
\citet{vanagt1978}`s catalogue of variable stars.\footnote{Interestingly,
  \citet{innanen+papp1979} also speculate about Uranus's satellite
  distribution, covering gravitational dynamics at two very different
  scales.} Additionally, \citet{eskridge1988} showed a possible excess
of stars in Sculptor relative to a \citet{king1962} and Exponential
density profile beginning around 50', but suggest that this excess may
not be unusual. Later work by \citet{IH1995}, \citet{walcher+2003}, and
\citet{westfall+2006} also showed evidence of an extended component of
Sculptor's density profile (among other dwarfs), interpreting these
stars as either evidence of tidal debris or a dwarf galaxy halo. On the
other hand, \citet{coleman+dacosta+bland-hawthorn2005} show that
Sculptor is well described by a two component density profile, they
additionally mention that it is unlikely many of Sculptor's stars (less
than 2\%) are extratidal. While \citet{munoz+2018} is the most recent
and deepest photometric study of Sculptor, they only cover an area of
out to radii of about 30', so they are unable to study the possible
extended components of Sculptor detected in past works. In addition to
the presence of ``extratidal stars'', many studies note that Sculptor's
stars appear to become more elliptical with radius, consistent with
tidal effects (\citet{IH1995}; \citet{westfall+2006}).

Ursa Minor was often noted not for a density excess but for unusual
features. Starting with the first density profile from
\citet{hodge1964}, Ursa Minor is known to be highly elliptical. Studies
by \citet{olszewski+aaronson1985}, \citet{demers+1995}, \citet{IH1995},
\citet{kleyna+1998}, and \citet{bellazzini+2002} note that Ursa Minor
appears to contain substructure along its major axis, but without strong
interpretation for the causes. As one interpretation \citet{kleyna+2003}
suggest that this feature is a long-lived star cluster residing in a
cored dark matter halo to be dynamically stable. One of the first direct
suggestions of tidal features came from \citet{martinez-delgado+2001},
who find that stars extend far beyond the nominal tidal radius for Ursa
Minor in the direction of the galaxies elongation, interpreting this as
evidence for tidal effects. \citet{palma+2003} corroborate many of these
earlier findings, showing further evidence for Ursa Minor's peculiar
morphology including S-shaped contours, a possible extended extratidal
population of stars, and a failure for a King density profile to
adequately capture the actual stellar distribution. Each of these
characteristics appear to indicate strong tidal disruption. Using
velocity observations, \citet{pace+2014} additionally show that there
are multiple components in spatial-velocity information. However,
\citet{munoz+2018}'s more modern photometry show a more regular
distribution of stars. Irregardless, Ursa Minor has had enough work
suggesting peculiarities that a deeper investigation into the
possibility of tidal effects is worthwhile.

However, everything discussed so far is before \emph{Gaia}. Thus, the
knowledge of the orbits of these systems is largely unknown and most
surveys could only filter out stars using photometry. As a result, the
tidal radius could only be guessed based on either the density profile
or current distance of the dwarf galaxy. With \emph{Gaia} (discussed
more below), recent work has used Bayesian frameworks to derive
systematic proper motions of many dwarf galaxies (e.g. \citet{MV2020a})
and filter away foreground stars using proper motions and parallax to
detect distant members, and study the 6D internal kinematics of these
galaxies (e.g. \citet{tolstoy+2023}). Most relevant here,
\citet{jensen+2024} present a bayesian algorithm to determine likely
members in \emph{Gaia} (described below), and detect extended secondary
components for 9 dwarf galaxies, including Sculptor and Ursa Minor.
\citet{sestito+2023a} and \citet{sestito+2023b} followed up a few of the
most distant stars detected in \citet{jensen+2024} confirming that
members exist in each galaxy as far ass 9-12 \(R_h\) from the centres.
In this chapter, we discuss the origin and reliability of these
detections, confirming that the density profiles are indeed robust.
These density profiles then provide the scientific motivation to explore
the tidal interpretation in the next chapters.

\section{Limitations of simulations}\label{limitations-of-simulations}

Variations to the potential of the inner disk (exclusion of a bar,
spiral arms) should minimally affect our results as no orbit we consider
reaches less than \textasciitilde15 kpc of the MW centre. We exclude the
mass evolution of the halo from this analysis. Over \(10\,\)Gyr, this
would be fairly significant (factor of \(\sim 2\)in MW mass, REF) but
since we want to determine the upper limit of tidal effects, it is safe
to neglect this.

\subsection{Long term orbital history}\label{long-term-orbital-history}

\citet{dsouza+bell2022}, \citet{santistevan+2024}. challenges to
backwards time integration

\subsection{Deviations from the NfW}\label{deviations-from-the-nfw}

\begin{itemize}
\tightlist
\item
  \citet{dicintio+2013}
\end{itemize}

\section{Comparison to other work}\label{comparison-to-other-work}

\subsection{Sculptor}\label{sculptor}

Theoretical work on Sculptor

\begin{itemize}
\tightlist
\item
  \citet{battaglia+2008}
\item
  \citet{iorio+2019}
\item
  \citet{amorisco+zavala+deboer2014}
\item
  \citet{battaglia+2008}
\item
  \citet{breddels+2013}
\item
  \citet{breddels+helmi2013}
\item
  \citet{richardson+fairbairn2014}
\item
  \citet{SFW2017}
\item
  \citet{innanen+papp1979}
\item
  \citet{wilkinson+2002}
\item
  \citet{yang+2025}: chemical evolution in Scl.
\item
  \citet{skuladottir+2024}; \citet{skuladottir+2021};
  \citet{lee+jeon+bromm2024}, pop III high res spectro?
\item
  \citet{wang+2024a} hydrosim of dwarf galaxies like sculptor
\item
  \citet{tang+2023}: apogee modeling of Scl.
\item
  \citet{pandey+west2022} chem evo OMEGA and isotopes.
\item
  \citet{an+koposov2022}: distance gradients?
\item
  \citet{kawata+2006}, 2pop origins?
\item
  \citet{grcevich+putman2009}, \citet{carignan+1998} ambiguous HI clouds
  near Scl ?
\item
  \citet{agnello+evans2012}, claim that Sculptor cannot hold two
  populations in NFW halo?
\end{itemize}

Observational work on Scl

\begin{itemize}
\tightlist
\item
  \citet{sestito+2023a}
\item
  \citet{westfall+2006} wide degree survey for extended structure.
\item
  \citet{tolstoy+2023}, \citet{arroyo-polonio+2023},
  \citet{arroyo-polonio+2024}
\item
  \citet{eskridge1988}, \citet{eskridge1988a}, \citet{eskridge1988b}
\item
  \citet{coleman+dacosta+bland-hawthorn2005}
\item
  \citet{DQ1994}
\item
  \citet{WMO2009}
\item
  \citet{IH1995},
\item
  \citet{munoz+2018}: Using Megacam to derive density profiles and
  structural properties of many dwarf spheroidal galaxies.
\item
  \citet{kirby+2009}
\item
  \citet{martinez-vazquez+2015}, \citet{pietrzynski+2008}
\item
  \citet{grebel1996}
\item
  \citet{barbosa+2025}: Using DECam to derive narrowband photometric
  metallicity gradient and search for metal poor stars in Scl.
\end{itemize}

Future ideas:

\begin{itemize}
\tightlist
\item
  \citet{evslin2016}: measuring ellipticities of halos w TMT.
\end{itemize}

\subsection{Ursa Minor}\label{ursa-minor}

\begin{itemize}
\tightlist
\item
  \citet{sestito+2023b}
\item
  \citet{pace+2020}
\item
  \citet{pace+2014} substructure
\item
  \citet{bellazzini+2002}
\item
  \citet{hargreaves+1994}
\item
  \citet{martinez-delgado+2001}
\item
  \citet{munoz+2005}: velocity dispersion profile
\item
  \citet{palma+2003}
\item
  \citet{spencer+2018}
\item
  \citet{vitral+2023}
\item
  \citet{piatek+2005}: old HST Pm meas.
\item
  \citet{gallagher+2003}: no ionizaed gas.
\item
  \citet{shetrone+cote+stetson2001}: RGBs
\item
  \citet{wilkinson+2004}
\item
  \citet{kleyna+2003}: dynamical fossil.
\item
  \citet{pryor+kormendy1990}, \citet{lake1990} MLR / velocity dispersion
\item
  \citet{armandroff+olszewski+pryor1995};
  \citet{olszewski+pryor+armandroff1996} MLR
\item
  \citet{aaronson+olszewski+hodge1983}; \citet{aaronson1983} carbon
  stars and velocities
\item
  \citet{stetson1984} early spectroscopy.
\item
  \citet{tsujimoto+shigeyama2002} chem evo.
\end{itemize}

Theoretical work

\begin{itemize}
\tightlist
\item
  robles+bullock2021
\item
  \citet{caproni+lanfranchi2021}, \citet{caproni+2015}: simulation of
  gass loss.
\item
  \citet{bajkova+bobylev2017}, local orbits
\item
  \citet{gomez-flechoso+martinez-delgado2003}: MLR of UMi
\item
  \citet{lynden-bell1976} early discussion of orbits and LMC plane.
\end{itemize}

\section{Are Sculptor and Ursa Minor
typical?}\label{are-sculptor-and-ursa-minor-typical}

\subsection{The formation of Exponential
profiles}\label{the-formation-of-exponential-profiles}

As mentioned earlier, many dwarf galaxies appear to have exponential
stellar density profiles.

Disk galaxies have long been known to be exponential-like across a wide
range of scales. As such, several theoretical works have aimed to
undertsand the formation of exponential stellar disks. One prevailing
theory posits that scattering of stars in a disk naturally forms an
exponential \citep[\citet{wu+2022}]{elmegreen+struck2013}, or that the
disk is due to conservation of angular momentum during spherical
collapse.

Given that dwarf galaxies were noticed to be exponential like above,
\citet{faber+lin1983} proposed that dwarf spheroidals formed from disky
galaxies maintaining the typical exponential density profile

\citet{read+gilmore2005} propsed that exponential dSph form through
impusive mass loss, redistributing the stellar component into an
exponential like profile.

\begin{itemize}
\tightlist
\item
  disk galaxy formation \citet{fall+efstathiou1980}, \citet{mestel1963}
\item
  valcke+derijcke+dejonghe2008 with early idealized hydro simulations
  show Sérsic values from simulations between 0.8 and 1 for dSph,
  indicating a natural formation of Exponential or slightly steeper
  density profiles.
\end{itemize}

One possible explanation is that \citet{mayer+2001a}

Other explanations range from the effects of mass loss, feedback,
angular momentum,

\citet{klimentowski+2007}, \citet{klimentowski+2009}.

In summary, while there is some theoretical reasons why exponential
spheroidal stellar profiles may form, the emperical sucess of an
exponential law still requires explanations. Deviations from this
exponential thus test wether this exponential law is indeed
near-universal or if other ingredients in the formation and history of
dwarf galaxies may be responsible for changes from the empirical
expectation.

\subsection{Alternatives to exponential
profiles}\label{alternatives-to-exponential-profiles}

\begin{itemize}
\tightlist
\item
  elmegreen+hunter2006: Discuss the formation of double exponentials in
  disk galaxies. While a simpler idealized / Semi-analytic model of star
  formation, show that assuming exponentail gas and KS-relation with SF
  threshhold, that reduction of turbulance (driven by a number of
  processes) in outer disk leads to a steeper decline in SFR as crossing
  the critical SF threshold becomes less likely.
\end{itemize}

\section{Understanding the extended density profiles of dwarf
galaxies}\label{understanding-the-extended-density-profiles-of-dwarf-galaxies}

Are dwarf galaxies indeed expected to be one-component exponential-like
density profiles? The suggestion of exponential density profiles dates
back to \citet{faber+lin1983}. Expand\ldots{}

A number of recent work has also confirmed the presence of extended,
likely non-tidal, density profiles. For example, \citet{chiti+2021};
\citet{chiti+2023} spectroscopically confirm members out to \(~9 R_h\)
in Tucana II.

\begin{itemize}
\tightlist
\item
  \citet{revaz+jablonka2018}
\item
  \citet{tau+vivas+martinez-vazquez2024} (RRL stars )
\item
  \citet{roderick+2016} Boötes I
\item
  \citet{mcconnachie+penarrubia+navarro2007} (velocity space \& multiple
  populations).
\item
  \citet{cicuendez+battaglia2018}
\end{itemize}

\subsection{Multi-epoch star formation and
feedback}\label{multi-epoch-star-formation-and-feedback}

The star formation history of dwarf galaxies is typically thought to be
\emph{bursty}: a series of discrete episodes of intense star formation
separated by periods of quiescence.

For example, in the simulations of \citet{wheeler+2019}, etc., dwarf
galaxies exhibit strongly bursty SFHs.

\begin{itemize}
\tightlist
\item
  \citet{maxwell+2012}
\item
  \citet{wright+2019}
\item
  \citet{azartash-namin+2024}
\end{itemize}

\subsubsection{Multi-component density
profiles}\label{multi-component-density-profiles}

A related note is that several dwarf galaxies show evidence for multiple
chemodynamical components in their stars. While spectra tend to focus on
the inner regions (high probability members), the existence of
multiple-component stellar populations in the inner regions of dwarf
galaxies hints at a complex star formation history capable of creating
the observed profiles.

\begin{itemize}
\tightlist
\item
  \citet{benitez-llambay+2016}
\item
  \citet{mercado+2021}: formation of metallicity gradients from SFH
\item
  \citet{revaz+jablonka2018} natural formation of multi-component and
  gradients from cosmological simulations of dwarf galaxies. Originate
  from dynamical heating of metal rich population and outside in star
  formation?
\item
  \citet{el-badry+2016}. Feedback drives fluctuations in size + radial
  migration =\textgreater{} population gradients.
\item
\end{itemize}

Observational evidence:

\begin{itemize}
\tightlist
\item
  \citet{arroyo-polonio+2024}
\item
  \citet{pace+2020}
\item
  \citet{fabrizio+2016}, \citet{kordopatis+2016} (multi-component in
  Carina)
\item
  \citet{battaglia+2006}, \citet{amorisco+evans2012} for fornax
\end{itemize}

\subsection{Past mergers and
accretion}\label{past-mergers-and-accretion}

\begin{itemize}
\tightlist
\item
  \citet{deason+2014}
\item
  \citet{deason+2022}
\item
  \citet{ricotti+polisensky+cleland2022}
\item
  \citet{querci+2025}
\item
  \citet{amorisco+evans+vandeven2014}
\item
  \citet{tarumi+yoshida+frebel2021}: Shows cosmological simulations in
  context of Tuc II, merger of two galaxies in the early universe
  produces a more extended density profile and metallicity gradient.
\item
  \citet{lokas+2014}: And II from merger,
\item
  rotation from merger: \citet{cardona-barrero+2021}
\end{itemize}

\subsection{Preprocessing}\label{preprocessing}

Another alternative is that the dwarf galaxies have been
``preprocessed'' by another dwarf/LMC like galaxy earlier on and this
has resulted in their unique properties

\subsection{Tidal dwarfs}\label{tidal-dwarfs}

On e possibility is that Sculptor and Ursa Minor are not formed from
dark matter halos but instead due to the collision of gas. This
possibility is not a convincing explanation because we know the star
formations histories are extended and relatively old, and the galaxies
are close to apocentre, making achieving the observed velocity
dispersion challenging in this framework.

\subsection{\texorpdfstring{Alternatives to
\(\Lambda\)CDM}{Alternatives to \textbackslash LambdaCDM}}\label{alternatives-to-lambdacdm}

Alternatives to the prevailing cosmology we assume may help explain the
nature of these dwarf galaxies. t

\begin{itemize}
\tightlist
\item
  WDM: forms cores, doesn't help
\item
  SIDM: can cause faster disruption, doesn't fix orbital path or change
  tidal effects?
\item
  MOND: \citet{sanchez-salcedo+hernandez2007}: mond in dsph
\end{itemize}
