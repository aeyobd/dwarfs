In the previous chapters, we have shown that Sculptor and Ursa Minor
have more extended density profiles compared to other MW classical
satellites, and that recent tides are an unlikely explanation for their
extended outer structure. We therefore suggest that Scl and UMi may host
an additional, extended population of stars, such as a ``stellar halo''.
With our data, other classical dwarfs lack evidence for similar extended
populations. Thus, Scl and UMi have followed a divergent evolutionary
path, motivating our discussion of extended density formation pathways
here.

In this chapter, we place our results in context with past work. We
review both works, considering tidal effects on Scl and UMi and
detections of peculiarities. We then consider possible origins of
extended stellar populations for these galaxies. Scenarios include
multi-epoch star formation, past mergers depositing stars into the
outskirts, and even departures from \LCDM{}. We later discuss how future
observations and simulations may distinguish among formation scenarios.
Finally, we summarize the main conclusions of this thesis.

\section{Comparison with prior work}\label{comparison-with-prior-work}

\subsection{The role of tides}\label{the-role-of-tides}

Tidal effects have long been invoked to explain features of dwarf
galaxies \citep[e.g.,
Section~\ref{sec:scl_umi_obs_tides},][]{mayer+2001a, tsujimoto+shigeyama2002}.
Yet, growing work suggests tides may not be as commonplace as suspected.
For example, \citet{read+2006} suggested that the absence of rising
radial velocity dispersion profiles in dwarfs implies a lack of tidal
disruption. \citet{penarrubia+2009}, introducing the break radius as a
diagnostic (Eq.~\ref{eq:r_break}), showed that most satellite orbits
were inconsistent with observable tidal effects. Recently,
\citet{pace+erkal+li2022} used a Jacobi radius-based criterion
(Eq.~\ref{eq:r_jacobi}) to reach a similar conclusion. Except for the
Sagittarius dSph, these results suggest tides have a mild influence on
the stellar components of classical dwarfs.

A few works have modelled Scl or UMi specifically, finding similar
conclusions. \citet{iorio+2019} applied idealized N-body simulations to
study tidal effects on Scl. They similarly found weak tidal effects,
even for a dark-matter-free model. Most recently,
\citet{tchiorniy+genina2025} also used idealized simulations tuned to
five classical dwarfs with a focus on the inner density, concluding that
tides do not strongly affect equilibrium assumptions or the stellar
component.

Our work extends these models, adding updated structural properties, a
broader range of orbital histories, and the influence of the LMC.
Despite these considerations, we still reach similar conclusions: tidal
effects do not shape Scl or UMi's stellar component. A non-tidal
explanation likely underlies the origin of these galaxies' outer density
excess.

\subsection{Peculiarities in the galaxies}\label{sec:peculiarities}

\begin{figure}
\centering
\pandocbounded{\includegraphics[keepaspectratio]{/Users/daniel/thesis/figures/scl_umi_fe_h_gradient.pdf}}
\caption[Metallicity gradients in Sculptor and Ursa Minor]{Density
profiles and metallicity gradients in Scl (left) and UMi (right).
\textbf{Top}: Black points show the J+24 density profiles as a function
of elliptical radius, and transparent lines represent the two components
of a double-exponential density fit. The vertical lines represent the
transition from the inner to the outer exponential profile.
\textbf{Bottom}: Metallicities as a function of elliptical radius for
stars crossmatched against J+24 as described in Appendix
\ref{sec:extra_rv_models}. Larger, red stars show the members from
\citet{sestito+2023a, sestito+2023b}.}\label{fig:metallicity_gradients}
\end{figure}

Besides extended stellar density profiles, Scl and UMi display other
peculiarities that may hold clues to their formation. Scl and UMi both
host at least two chemodynamic populations, as revealed through their
photometric or metallicity-velocity structure\footnote{Other examples of
  galaxies with multiple populations include Carina
  \citep{battaglia+2012, fabrizio+2016, kordopatis+2016}, Fornax
  \citep{battaglia+2006, amorisco+evans2012, delpino+aparicio+hidalgo2015},
  Sextans
  \citep{battaglia+2011, cicuendez+battaglia2018, roederer+2023}, and
  Andromeda II
  \citep{mcconnachie+arimoto+irwin2007, ho+2012, delpino+2017}.}
\citep{tolstoy+2004, battaglia+2008, pace+2020}. The inner population is
younger, higher metallicity, and dynamically colder, whereas the outer
population is older, lower metallicity, and dynamically hotter.

Could these internal stellar populations map to the observed density
excess? In Scl, both populations have scale radii likely smaller than
the half-light radius. But, the third, more tentatively detected
population in \citet{arroyo-polonio+2024} would reside within the outer
density excess. In Ursa Minor, the transition between the metal-rich and
metal-poor stellar components occurs at \(\sim 30\,\)arcmin
\citep{pace+2020}, which coincides with the density excess. This
metal-poor population may thus be related to the extended stellar
density of Ursa Minor.

Fig.~\ref{fig:metallicity_gradients} shows the radial metallicity
distribution in both galaxies. To aid in comparison, we fit a
2-component exponential to the density profiles. We also mark where the
densities of each component become equal, the ``transition'' radius. Scl
and UMi both show evidence of a metallicity gradient. But, the relation
between the metallicity gradient and the transition from the inner to
the outer density profile is unclear. UMi also has few measurements past
the transition radius. Further observations would be necessary to
understand if the extended stellar component represents a chemically
distinct population.

Ursa Minor has also shown evidence for possible inner substructure, such
as stellar or kinematic ``clumps''
\citep[e.g.,][]{olszewski+aaronson1985, demers+1995, kleyna+1998, battinelli+demers1999, bellazzini+2002}.
While one clump was re-detected kinematically in \citet{pace+2014},
\citet{munoz+2018} find no evidence of substructure with modern
photometry. In any case, the survival of a cold clump depends on the
inner structure of the halo, disrupting quickly in halos with
substructure or inner density cusps but surviving longer in smooth,
cored halos \citep{kleyna+2003, lora+2012}.

Given their multiple stellar populations and possible inner
substructure, this hints that Scl and UMi's distinct structure today
arises from an unusual past. For example, a merger may naturally explain
both extended outer and multiple stellar populations.

\section{Forming an extended stellar
population}\label{sec:stellar_halos}

As we disfavour a tidal origin for Scl and UMi's extended density
profiles, we now consider six other processes that may form an extended
stellar population in these dwarfs.

\textbf{Episodic star formation.} Star formation may quench and
reignite, creating successive stellar generations with differing
distributions. External star formation triggers include tidal
compression \citep{mayer+2001a, dong+lin+murray2003}, collisions with
gaseous filaments \citep{genina+2019}, perturbations from dark halos
\citep{starkenburg+helmi+sales2016}, or shocks with the MW corona
\citep{wright+2019}. More common mechanisms, like feedback or
reionization-driven quenching, may also form multiple stellar
generations
\citep{kawata+2006, benitez-llambay+2015, revaz+jablonka2018}. However,
such processes would not explain why extended stellar populations appear
to be non-universal. If an extended population was formed in an older,
induced burst of star formation, then the star formation history should
contain evidence of a corresponding burst.

\textbf{Major mergers.} Dwarf galaxy mergers may be relatively common
\citep{deason+wetzel+garrison-kimmel2014}. After a merger, stars from
the lower-mass galaxy are preferentially dispersed, forming an extended
stellar component and population gradient
\citep{benitez-llambay+2016, deason+2022}. If the galaxies contain gas,
the merger can trigger new star formation, forming a younger population
of stars \citep[e.g.,][]{genina+2019}. A few local dwarfs are suspected
to have undergone a major merger, including Tucana II, Andromeda II, and
Phoenix
\citep{lokas+2014, fouquet+2017, tarumi+yoshida+frebel2021, cardona-barrero+2021, querci+2025}.
A galaxy having undergone a major merger should harbour at least two or
three populations from distinct origins.

Just as the Milky Way's halo is believed to be built from many minor
mergers, dwarf galaxies may build accreted ``stellar halos'' composed of
even fainter satellites \citep{ricotti+polisensky+cleland2022}. Few
examples of dwarf-dwarf accretion are known, but a possible stream
around And II or chemical peculiarities around Sextans I may be such
instances \citep{amorisco+evans+vandeven2014, roederer+2023}. In this
scenario, the extended stellar populations would contain chemical
signatures from several distinct ultra-faint dwarf galaxies.

\textbf{Tidal preprocessing}. Some dwarf galaxies may have been tidally
``preprocessed'' by a massive satellite like the LMC
\citep[e.g.,][]{santistevan+2023, riley+2024}. Tidal preprocessing
redistributes already-present stellar populations and may mimic a
stellar halo. Key evidence suggestive of preprocessing may include
distant dwarf stars or stellar streams.

\textbf{Dark-matter-free dwarfs.} Tidal dwarfs form in gas-rich tidal
streams created during the merger of two galaxies
\citep[e.g.,][]{mirabel+dottori+lutz1992, bournaud+duc2006}. Without a
dark matter halo, tidal dwarfs may be more susceptible to tides, forming
extended density profiles and inflating the velocity dispersion
\citep{casas+2012, yang+2014, wang+2024a}.

Similarly, in Modified Newtonian Dynamics (MOND), galaxies do not
contain dark matter. Instead, the gravitational force law is altered to
explain rotation curves. Dwarfs would similarly experience stronger
tidal effects in MOND \citep{mcgaugh+wolf2010, brada+milgrom2000}.

In both cases, tides more plausibly produce extended density profiles.
However, recovering the observed velocity dispersions often requires
ongoing tidal disruption
\citetext{\citealp{mcgaugh+wolf2010}; \citealp[but see
also][]{sanchez-salcedo+hernandez2007}}. Current data do not show
evidence of such features (see Appendix
\ref{sec:extra_rv_models})---future data would need to uncover signs of
disruption to support this hypothesis.

\textbf{Dynamical heating.} Old stars in dwarf galaxies may be hotter
than younger stars due to processes including stellar feedback
\citep{stinson+2009, maxwell+2012, el-badry+2016, mercado+2021},
sub-subhalo interactions \citep{penarrubia+2025}, or even fuzzy dark
matter interference fringes
\citep[e.g.,][]{el-zant+2020, duttachowdhury+2023}. However, most of
these processes should operate similarly across dwarf
galaxies---extended stellar populations should be more common if these
processes are important.

\section{Disentangling the origin of extended stellar
populations}\label{disentangling-the-origin-of-extended-stellar-populations}

We reviewed a number of different, possibly concurrent explanations for
the formation of extended populations. While we leave the nature of Scl
and UMi's extended stellar densities an open question, we can discuss
possible clues to different formation scenarios.

\emph{Precise chemistry}. Chemistry, particularly comparing the inner
and outer regions, could test if the halo appears to originate from a
distinct system than the dwarf. This would test if the halo is a
separate chemical population, may reveal the properties of any merged
galaxies, or show if the halo is consistent with originating from the
same galaxy.

\emph{Detailed kinematics}. For models relying on recent tidal
disruption, kinematic disequilibrium features should be visible. These
would appear as velocity gradients, increasing velocity dispersions,
outward-biased moving stars, or non-phase mixed structures
\citep[e.g,][]{kroupa1997, read+2006, sanchez-salcedo+hernandez2007}. In
addition, measurements of the ellipticity and anisotropy would help
verify or alter our understanding of the mass structure and whether
DM-free models may be permissible.

\emph{Deep photometry} may find or rule out signs of dynamical
disequilibrium and tidal tails for tidally susceptible models (e.g.,
\emph{pre-processing} and \emph{DM-free dwarfs}). In addition,
photometry will help constrain the prevalence and nature of extended
features in dwarf galaxies.

\emph{Star formation histories} can be derived through photometry or
chemistry. Evidence of significant star formation episodes or lack
thereof may differentiate scenarios that rely on strong star formation
bursts (e.g., \emph{episodic star formation history, induced star
formation,} and \emph{gas-rich mergers}).

Ongoing and future observational surveys will be instrumental for
testing dwarf galaxy halo formation. Because dSph stars are usually
faint, detailed chemical analysis will rely on medium--high resolution
multi-object spectroscopy on large telescopes, such as 4MOST
\citep{skuladottir+2023} or with the extremely large telescope
\citep{jagourel+2018}. Mapping the 3D kinematics across a dwarf may
require a \emph{Gaia} successor, but ongoing surveys with JWST and HST
can derive precise proper motions within select regions of dwarf
galaxies, able to reconstruct internal structure and anisotropies
\citep[e.g.,][]{vitral+2025}. Current wide-field photometric surveys by
Rubin Observatory \citep{ivezic+2019} and the Euclid mission
\citep{euclidcollaboration+2025} will trace even fainter features around
dwarf galaxies, possibly finding halos, tidal streams, and unusual
stellar populations. Together, these observational programs promise a
new window into the detailed structure, evolution, and outskirts of
dwarf galaxies.

The next generation of dwarf simulations will likewise continue to
refine our understanding of dwarf galaxy formation. Notable recent
simulation projects include EDGE \citep{agertz+2020} and LYRA
\citep{gutcke+2021}, both of which can simulate the cosmological
formation of isolated dwarf galaxies while resolving individual
(massive) stars. Next steps for simulations include improved realism
(e.g., magnetic fields), refining physical models (including feedback
and star formation), and improving numerical resolution. With these
continuing advances, future simulations may shed light on the viability
of extended stellar population formation scenarios and the details of
dwarf galaxy assembly.

\section{Conclusion and outlook}\label{conclusion-and-outlook}

In this thesis, we have investigated the extended, outer density excess
of Sculptor and Ursa Minor and its possible tidal origin. We show that
the density profiles are robust, implying that Scl and UMi contain a
true density excess relative to other classical dwarfs.

We then investigated whether tides are a possible explanation. By
modelling each galaxy based on cosmological initial conditions, we
showed that tides do not strongly affect either galaxy. The LMC changes
the orbital history of Scl and UMi, but tides become even weaker in a
combined LMC and MW potential. We conclude that recent tides are
unlikely to shape the outer radial distribution of stars in Scl and UMi.

Finally, we discuss the implications of a non-tidal density excess in
Scl and UMi. We review mechanisms ranging from episodic star formation
histories, to accretion, to departures from \LCDM{}. Future simulations
and observational programs will be able to investigate the nature of
dwarfs with ever-increasing detail. Abundant opportunities await to
uncover the cosmic origins and history of dwarf galaxies, the building
blocks of galaxies like ours.
