In the previous chapters, we have shown that (1) Sculptor and Ursa Minor
have unusually extended density profiles for MW classical satellites and
(2) recent tides are an unlikely explanation. Here, we first consider
the reliability and assumptions supporting our conclusions. We then
consider possible formation scenarios of extended density profiles,
including multi-epoch star formation, dwarf mergers and subhalo
encounters, and alternative theories of dwarf galaxy structure. We
finish with an outlook on prospects for disentangling various stellar
halo formation scenarios.

While tides have been a popular explanation for some features of dwarf
galaxies \citep[discussion above, and
e.g.,][]{tsujimoto+shigeyama2002, mayer+2001a}, our work, and many
others, have disfavoured this for Scl, UMi, and many other MW dwarfs.
For example, \citet{read+2006} suggest that the lack of observational
evidence for a rising velocity dispersion profile with radii indicates
most dwarfs are not tidally affected. Later, \citet{penarrubia+2009}
define and use the break radius, Eq.~\ref{eq:r_break}, to show that the
(then understood) orbits of most satellites are inconsistent with a
tidal feature. In addition, \citet{pace+erkal+li2022} use a criterion
based on the observed mean density and orbital pericentre (like the
Jacobi radius) to show most dwarfs are unlikely to be undergoing
significant tidal disruption. Most recently,
\citet{tchiorniy+genina2025} use a similar idealized framework to ours,
with a focus on the inner density, showing tides do not strongly affect
equilibrium assumptions for several classical dwarfs. The general tidal
evolution of our simulations is furthermore consistent with much of the
literature \citep[e.g.,][]{robles+bullock2021, EN2021}. Tides seem to
now play a subdominant role in the evolution of most Milky Way
satellites.

\section{Comparison with prior work}\label{comparison-with-prior-work}

For both Scl and UMi, these galaxies have been studied extensively in
both a theoretical and observational context. While many works
considered mass modelling, star formation histories, chemistry, and
other observational properties, we focus on the observations and models
concerning tides and dynamical evolution here.

However, some cosmological simulations have suggested that tidal effects
may be ubiquitous. \citep{riley+2024, shipp+2023} find that a majority
of satellites form substantial streams, and not due to resolution
effects \citep[see also][]{panithanpaisal+2021}. This result appears to
be in tension with ours and many previous works. Some reasons
cosmological and idealized simulations may give opposite results could
include resolution effects \citep[e.g.,][]{santos-santos+2025},
additional perturbations from dark matter substructure, tidal
``pre-processing'' by other infalling satellites, or the streams may be
below detection limits \citep[e.g.,][]{shipp+2023}. More work is needed
to understand why streams appear to form more readily in cosmological
simulations than is observed in the Milky Way or through idealized
simulations.

Closely related to our work, \citet{iorio+2019} applied idealized N-body
simulations to study tidal effects on Scl. They similarly found weak
tidal effects, even for a dark-matter-free model. However, their orbits
are less well-constrained and they did not consider the LMC.

\subsection{Sculptor}\label{sculptor}

Scl has long been speculated to be disturbed. \citet{innanen+papp1979}
found RR Lyrae candidate Scl members \citep[from][]{vanagt1978} out to
180' in an elongated distribution, speculating this to be tidal
disruption. Many later density profile determinations noted Scl's
elongation and apparent outer density excess similar to J+24's detection
\citep[but see also
\citet{coleman+dacosta+bland-hawthorn2005}]{eskridge1988, IH1995, walcher+2003, westfall+2006}.
Scl's ellipticity and extended density profile were often interpreted as
evidence of tidal debris or sometimes a dwarf galaxy halo.

In addition, Scl hosts at least two populations, as revealed through
photometry \citep{tolstoy+2004}, kinematics
\citep{battaglia+2008, tolstoy+2023, arroyo-polonio+2024}, dynamical
structure \citep{breddels+helmi2014}, age gradients \citep{deboer+2011},
and chemistry \citep{kirby+2009}. These two populations both have radii
smaller than the outer density excess. Scl challenges the idea of a
simple, single-component dwarf spheroidal population and hints at
episodic star formation or hierarchical assembly.

\subsection{Ursa Minor}\label{ursa-minor}

UMi has garnered claims of inner substructure such as stellar or
kinematic ``clumps''. Studies by \citet{olszewski+aaronson1985},
\citet{demers+1995}, \citet{IH1995}, \citet{kleyna+1998},
\citet{battinelli+demers1999}, and \citet{bellazzini+2002} note that
Ursa Minor appears to contain ``clumps'' along its major axis. One clump
has furthermore been shown to be kinematically distinct and colder
\citep[e.g.,][]{pace+2014}. If a star cluster, than the cluster should
disolve in \(\sim 3\) Gyrs provided the UMi's halo is not cored and did
not interact with dark (sub)subhalos---the survival of such substructure
depends on the nature of dark matter \citep{kleyna+2003, lora+2012}.
\citet{wilkinson+2004} additionally find a puzzling drop in the velocity
dispersion in the outskirts. The nature of any clumps or substructure
remains unclear, but \citet{munoz+2018}'s are unable to find signs of
any substructure with modern, deeper photometry.

In addition, several works have found supposed tidal features in UMi.
\citet{hargreaves+1994} first detected a velocity gradient in UMi,
suggestive of tidal disruption. Later, \citet{martinez-delgado+2001}
find that stars extend far beyond the ``tidal radius'' (from a King
profile fit) for Ursa Minor, in the direction of the galaxy's elongation
\citep[see corresponding simulations
by][]{gomez-flechoso+martinez-delgado2003}. \citet{palma+2003} further
showed evidence for S-shaped contours, and an extended population of
``extratidal'' stars. Our density profiles are consistent with these
works. However, strong evidence of tidal disruption has not yet been
found.

\section{Forming a stellar halo}\label{sec:stellar_halos}

As a tidal origin of Scl and UMi's extended density profiles is
disfavoured, we consider alternative processes which may form a dwarf
galaxy ``stellar halo''. Some recent works have also confirmed the
presence of extended, likely non-tidal, density profiles. For example,
\citet{chiti+2021}; \citet{chiti+2023} spectroscopically confirm members
out to \(~9 R_h\) in Tucana II. Given these stars are misaligned with
the orbit and the lack of a velocity gradient, tides seem to be an
unlikely explanation for Tucana II as well. The prevalence of stellar
halos around dwarf galaxies remains an active research topic.

Many dwarf galaxies also host multiple chemodynamical stellar
populations. Typically, older populations are extended, kinematically
hotter, and metal-poo , whereas the younger populations are more
compact, colder, and metal-rich. Examples include Carina
\citep[\citet{fabrizio+2016}, \citet{kordopatis+2016}]{battaglia+2012},
Fornax \citep[\citet{amorisco+evans2012},
\citet{delpino+aparicio+hidalgo2015}]{battaglia+2006}, Sextans
\citep{battaglia+2011, cicuendez+battaglia2018, roederer+2023}, and
Andromeda II
\citep{mcconnachie+arimoto+irwin2007, ho+2012, delpino+2017}. Evidence
of multiple-populations in dwarf galaxies suggests that effects like
galaxy mergers or episodic star formation shape the stellar structure of
dwarf galaxies.

We now explore possible origins of a ``stellar halo'' which may resemble
the extended density profiles in Scl and UMi.

\subsection{Internal processess}\label{internal-processess}

\textbf{Dynamical heating of old stars}. In a galaxy, older stars
generally have higher random velocities (i.e., kinematically ``hotter'')
than younger stars. In dwarf galaxies, several mechanism may heat
stellar components, including supernovae kicks in forming stars,
feedback-driven potential fluctuations
\citep{stinson+2009, maxwell+2012, el-badry+2016, mercado+2021}, or
heating by dark sub-subhalos \citep{penarrubia+2025}. Over time, these
processes naturally produce a more extended, older stellar population.

\textbf{Episodic star formation and feedback.} Dwarf galaxies are
thought to experience bursty star formation--i.e., consisting of several
episodes of intense star formation separated by periods of quiescence
\citep[e.g.,][]{salvadori+ferrara+schneider2008, valcke+derijcke+dejonghe2008, wheeler+2019, azartash-namin+2024}.
Stellar feedback in a dwarf galaxy's shallow potential well drives
oscillations in the star formation rate. Alternatively, re-ionization
may temporarily suspending star formation \citep{benitez-llambay+2015}.
In many simulations, shrinking gas reservoirs form
centrally-concentrated later generations of stars, naturally spawning
multiple populations \citep{kawata+2006, revaz+jablonka2018}. However,
star formation is also sensitive to a dwarf's environment.

\subsection{External processes}\label{external-processes}

\textbf{Induced star formation.} Star formation may quench and reignite
due to an external perturbation. Examples include tidal compression
\citep{mayer+2001a, dong+lin+murray2003}, gaseous filaments
\citep{genina+2019}, dark halos \citep{starkenburg+helmi+sales2016}, or
shocks with the MW corona \citep{wright+2019}. Induced burst would be
more stochastic than internal bursts, possibly explaining diversity in
dwarf galaxy structure.

\textbf{Major mergers.} When galaxies merge, they may leave signatures
such as population gradients and stellar halos. Classical dwarfs have a
\(\sim 10\%\) chance of undergoing a major merger since redshift \(z=1\)
\citep{deason+wetzel+garrison-kimmel2014}. Mergers may preferentially
disperse the lower-mass galaxy's stars, forming a halo and a metallicity
gradient \citep{benitez-llambay+2016}. Intermediate-mass mergers (with
mass-ratios of \(\sim\) 1:5) most effectively populate halos, balancing
stellar mass from larger galaxies with better dispersion of lower-mass
galaxies \citep{deason+2022}. Tuc II's properties are suggested to
originate from a similar merger
\citep{tarumi+yoshida+frebel2021, querci+2025}.

\textbf{Gas-rich mergers.} If a merger occurs between gas-rich galaxies,
triggered star formation may occur in the aftermath
\citep[e.g.,][]{genina+2019}. And II and Phoenix have steep metallicity
gradients and unusual prolate rotation, theorized to result from mergers
of disky dwarfs \citep{lokas+2014, fouquet+2017, cardona-barrero+2021}.

\textbf{Minor mergers / accretion}. Like how the Milky Way's halo is
believed to be built from many minor mergers, dwarf galaxies may form
halos through accretion of yet fainter dwarfs.
\citet{ricotti+polisensky+cleland2022} demonstrated that accretion of
ultra-faint dwarfs may create dwarf stellar halos. A stream detected
around And II further supports the occurrence of mergers among dwarfs
\citep{amorisco+evans+vandeven2014, roederer+2023}. In this scenario,
the halo chemistry would resemble a population of ultra-faint dwarfs.
The occurance of dwarf mergers would further constrain small-scale
galaxy formation.

\textbf{Tidal preprocessing}. Dwarf galaxies may have been
``preprocessed'' by a larger satellite like the LMC
\citep[e.g.,][]{santistevan+2023, riley+2024}. From the orbit
integrations above, it is possible that UMi was once an LMC satellite.
However, the prevalence of preprocessing remains uncertain. Like stellar
heating, preprocessing redistributes already-present stellar
populations.

\textbf{Tidal dwarf galaxies} are cluster-like objects which form in
gas-rich tidal streams created during the merger of two galaxies
\citep[e.g.,][]{mirabel+dottori+lutz1992, bournaud+duc2006}. Tidal
dwarfs may be more susceptible to tides, forming extended density
profiles and appearing to have dark matter
\citep{casas+2012, yang+2014, wang+2024a}. If Scl and UMi are tidal
dwarfs, a stronger velocity (dispersion) gradient and tidal tails should
be detectible.

\subsection{\texorpdfstring{Beyond \LCDM{}}{Beyond }}\label{beyond}

\textbf{Modified Newtonian Dynamics (MOND)}. MOND modifies gravity
instead of using dark matter to explain the rotation curves of galaxies.
In MOND, tides may more strongly affect dwarf galaxies owing to the lack
of dark matter mass and the stronger MW tidal field
\citep{mcgaugh+wolf2010, brada+milgrom2000}. A tidal origin of the
density excess is more likely in this case. If MOND is to recover the
observed velocity dispersions of satellites, then many more dwarfs may
be actively tidally disrupting as well
\citep{casas+2012, yang+2014, wang+2024a}.

\textbf{Self-interacting dark matter (SIDM).} SIDM significantly
complicates tidal evolution. SIDM halos in isolation are not static---by
transferring heat through collisions, these halos first undergo core
formation and then core collapse. Besides structural changes, SIDM adds
that pressure from the host DM halo can change the structure
(e.g.~aiding core collapse) or remove material from the inner galaxy
(analogous to ram pressure stripping) \citep[e.g.,][]{cartonzeng+2024}.
An SIDM halo may be more strongly affected by tides, but the velocity
dispersion makes a tidal disruption in SIDM still unlikely.

\textbf{Fuzzy dark matter} can also heat up stars owing to density
perturbations in the form of interference fringes
\citep[e.g.,][]{el-zant+2020, duttachowdhury+2023}. Similar to other
intrinsic heating methods, this model would likely affect all dwarfs
similarly, so we should detect similar halos around similar dwarfs.

\subsection{Disentangling the origin of a stellar
halo.}\label{disentangling-the-origin-of-a-stellar-halo.}

We just reviewed a number of different, possibly-concurrent explanations
for the formation of a stellar halo. While we leave the nature of Scl
and UMi's extended stellar densities an open question, we can discuss
possible clues to different formation scenarios.

\textbf{Chemistry}. Large samples of detailed chemical abundances have
been invaluable for understanding MW substructure (e.g.~for
\emph{Gaia}-Sausage Enceladus). Chemistry, particularly comparing the
inner and outer regions could test if the halo appears to originate from
a distinct system than the dwarf, differentiating many \emph{internal}
versus \emph{external} scenarios.

\textbf{Star formation histories}. Evidence of significant star
formation episodes or lack thereof may differentiate scenarios which
rely on strong star formation bursts (e.g.~\emph{episodic star formation
history, induced star formation, and gas-rich mergers.})

\textbf{Kinematics.} Particularly, for models relying on recent tidal
disruption, kinematic disequilibrium features should be visible. These
would appear as velocity gradients, increasing velocity dispersions,
outward-biased moving stars, or non-phase mixed structures
\citep[e.g,][]{kroupa1997, read+2006, sanchez-salcedo+hernandez2007}.

\textbf{Deep photometry} may find or rule out signs of dynamical
disequilibrium and tidal tails for tidally susceptible models
\emph{(e.g., MOND and tidal dwarfs.)}

Ongoing, upcoming, and future facilities will be essential for testing
these theories. For example, given the typical faint magnitudes of dSph
stars, chemistry is best accessible through large field-of-view
multi-object spectroscopic instruments on large telescopes (e.g., 4MOST
and extremely large telescopes \citet{skuladottir+2023}). 3D internal
kinematics of dwarfs may require a successor to \emph{Gaia}. However,
using JWST and HST enables precise proper motions for small regions
within dwarf galaxies \citep[e.g.,][]{vitral+2025}, which may be
sufficient to constrain vastly different internal dynamic structures and
ongoing tidal disruption. Finaly, Photometric surveys by Rubin
Observatory and Euclid will most-emminently probe yet fainter magnitudes
around dwarf galaxies and possibly find or constrain stellar halos and
their star formation histories. Altogether, new surveys will likely
uncover novel aspects about the inner workings and outer halos of dwarf
galaxies.

In addition to new observing facilities, the next generation of dwarf
simulations are beginning to answer questions and unravel processes in
the formation of dwarf galaxies. With improved resolution, realism, and
physics, the frequency and effects of various mechanisms discussed can
be better constrained.

\section{Conclusion and outlook}\label{conclusion-and-outlook}

In this thesis, we have investigated the extended, outer density excess
of Sculptor and Ursa Minor and its possible tidal origin. We first
verified that Scl and UMi have unusually extended density profiles,
compared to other dwarf spheroidals. We show that the density profiles
are robust to alternate data criteria, implying that this ``density
excess'' is likely a real feature of each galaxy.

We then investigated if tides were a permissible explanation. By
modelling each galaxy based on cosmological initial conditions, we
showed that tides do not strongly affect either galaxy. The LMC changes
the orbital history of Scl and UMi, and tides become even weaker in a
combined LMC and MW potential. While UMi may form a stellar stream, the
stream is far fainter than is presently detectible. We conclude that
recent tides are unlikely to shape the observed stellar distributions of
Scl and UMi.

Finally, we consider alternative scenarios forming extended density
distribution. We review mechanisms ranging from episodic star formation
histories, to accretion, to departures from \LCDM{}. While a more
precise explanation awaits, upcoming and ongoing surveys will uncover
the stellar populations within dwarf galaxies in unprecedented detail.
Future work will the Milky Way, uncovering the cosmic origins of the
smallest structures forming building blocks of galaxies like ours.
