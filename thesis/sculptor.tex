\section{Sculptor (MW only)}\label{sculptor-mw-only}

\subsection{Orbital estimation}\label{orbital-estimation}

fig.~\ref{fig:unref} contains example orbits

Given the parameters in table@REF, we

\subsection{Simulations}\label{simulations}

\begin{itemize}
\tightlist
\item
  dark matter loss
\item
  little influence to stars
\item
  smallest pericentre does not help
\end{itemize}

\section{Sculptor (LMC orbits)}\label{sculptor-lmc-orbits}

As discussed in \citet{battaglia+2022}, Sculptor's orbit is strongly
influenced by the presence of an LMC. Figure
fig.~\ref{fig:scl_lmc_orbit_effect}, the addition of an LMC reduces
Scl's pericentre with the MW and implies that Scl may be on its first
infall with the MW, in contrast to the discussion above. Thus, the LMC
has a critical impact on the evolution of Scl. Additionally, Scl passes
rapidly but relatively close to the LMC in the past 100 Myr.

We now consider 3 orbits under the L3M11 potential model from
\citet{vasiliev2024}. We focus on this LMC potential as a lighter LMC or
MW should only reduce tidal impacts and the recent orbit of Scl is
minimally affected by the LMC potential.

\begin{figure}
\centering
\pandocbounded{\includegraphics[keepaspectratio]{/Users/daniel/Library/Application Support/typora-user-images/image-20250415112706417.png}}
\caption{image-20250415112706417}
\end{figure}

\begin{figure}
\centering
\pandocbounded{\includegraphics[keepaspectratio]{/Users/daniel/Library/Application Support/typora-user-images/image-20250415113122385.png}}
\caption{image-20250415113122385}
\end{figure}

\begin{figure}
\centering
\pandocbounded{\includegraphics[keepaspectratio]{/Users/daniel/Library/Application Support/typora-user-images/image-20250415113133353.png}}
\caption{image-20250415113133353}
\end{figure}
