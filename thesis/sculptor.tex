\section{Sculptor}\label{sculptor}

\subsection{Derived properties}\label{derived-properties}

\begin{table*}[t]
\centering
\begin{tabular}{lll}
\toprule
parameter & value & reference\\
\midrule
$L_\star$ & $1.82_{-0.22}^{+0.25}\times10^6\ L_\odot$ & \\
$M_\star$ & $2.7_{-0.6}^{+0.7} \times10^6\ {\rm M}_\odot$ & \\
$M_{200}$ & $0.48_{-0.25}^{+0.52}\ M_0$ & \\
$c_{\rm NFW}$ & $13.1_{-2.8}^{+3.6}$ & \\
$v_{\rm circ, max}$ & $31\pm8\,\kms$ & \\
$r_{\rm circ, max}$ & $5.9 \pm 2.9$ kpc & \\
$r_{\rm break, obs}$ & $25 \pm 5$ arcmin & \\
$t_{\rm break, obs}$ & $110\pm30$ Myr & \\
\bottomrule
\end{tabular}
\end{table*}

\subsection{Milky Way tides}\label{milky-way-tides}

\subsubsection{Orbital properties}\label{orbital-properties}

Fig.~\ref{fig:unref} contains example orbits

Given the parameters in table@REF, we

\subsubsection{Tidal effects}\label{tidal-effects}

\begin{itemize}
\tightlist
\item
  dark matter loss
\item
  little influence to stars
\item
  smallest pericentre does not help
\end{itemize}

\subsection{Effects of the LMC}\label{effects-of-the-lmc}

\subsubsection{Orbital effects}\label{orbital-effects}

As discussed in \citet{battaglia+2022}, Sculptor's orbit is strongly
influenced by the presence of an LMC. Figure
Fig.~\ref{fig:scl_lmc_orbit_effect}, the addition of an LMC reduces
Scl's pericentre with the MW and implies that Scl may be on its first
infall with the MW, in contrast to the discussion above. Thus, the LMC
has a critical impact on the evolution of Scl. Additionally, Scl passes
rapidly but relatively close to the LMC in the past 100 Myr.

Finally, we also consider the influence of the large Milky Cloud on the
orbits and evolution of Scl. We adopt the \citet{vasiliev2024} multipole
approximation of an N-body simulation of the LMC and MW. Their initial
conditions are

\begin{itemize}
\tightlist
\item
  MW halo:
\item
  MW bulge (static):
\item
  MW disk (static):
\item
  LMC halo:
\end{itemize}

We now consider 3 orbits under the L3M11 potential model from
\citet{vasiliev2024}. We focus on this LMC potential as a lighter LMC or
MW should only reduce tidal impacts and the recent orbit of Scl is
minimally affected by the LMC potential.

\begin{figure}
\centering
\includegraphics[width=1\linewidth,height=\textheight,keepaspectratio]{/Users/daniel/Library/Application Support/typora-user-images/image-20250415112706417.png}
\caption[Effect of LMC on Sculptor's Orbit]{Orbits of Sculptor with and
without the LMC. The potential is that of \citet{vasiliev+2021} in each
case.}
\end{figure}

\begin{figure}
\centering
\includegraphics[width=1\linewidth,height=\textheight,keepaspectratio]{/Users/daniel/Library/Application Support/typora-user-images/image-20250415113122385.png}
\caption[Selected orbits of Sculptor with an LMC]{Selected orbits of
Sculptor in the MW frame. \textbf{TODO}: include LMC distance or frame,
maybe xyz instead?}
\end{figure}

\subsubsection{Tidal effects}\label{tidal-effects-1}

\section{Ursa Minor}\label{ursa-minor}

\subsection{Derived properties}\label{derived-properties-1}

\begin{table*}[t]
\centering
\begin{tabular}{lll}
\toprule
parameter & value & reference\\
\midrule
$L_\star$ &  & \\
$M_\star$ &  & \\
$M_{200}$ &  & \\
$c_{\rm NFW}$ &  & \\
$v_{\rm circ, max}$ & $\,\kms$ & \\
$r_{\rm circ, max}$ & kpc & \\
$r_{\rm break, obs}$ & $30 \pm 5$ arcmin & \\
$t_{\rm break, obs}$ & $120\pm30$ Myr & \\
\bottomrule
\end{tabular}
\end{table*}

\subsection{Orbital properties}\label{orbital-properties-1}

\begin{table*}[t]
\centering
\begin{tabular}{lllllllll}
\toprule
Orbit & Mean &  & smallperi &  &  &  &  & \\
\midrule
ra &  &  &  &  &  &  &  & \\
dec &  &  &  &  &  &  &  & \\
distance &  &  &  &  &  &  &  & \\
pmra &  &  &  &  &  &  &  & \\
pmdec &  &  &  &  &  &  &  & \\
rv &  &  &  &  &  &  &  & \\
x &  &  &  &  &  &  &  & \\
y &  &  &  &  &  &  &  & \\
z &  &  &  &  &  &  &  & \\
vx &  &  &  &  &  &  &  & \\
vy &  &  &  &  &  &  &  & \\
vz &  &  &  &  &  &  &  & \\
\bottomrule
\end{tabular}
\end{table*}

\subsection{Tidal effects}\label{tidal-effects-2}
