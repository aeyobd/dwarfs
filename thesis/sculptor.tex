In this chapter, we present our results from our simulations of Sculptor
and Ursa Minor. In each case, we first discuss the range of reasonable
orbits and initial dark matter halos for each galaxy. Then, we describe
how tides affect the evolution of each galaxy. We furthermore consider
how the effects of the LMC changes the results, including using
additional N-body simulations in evolving potentials. Finally, we review
the predictions and characteristics of the present-day stellar
populations and properties from our models.

\section{Sculptor}\label{sculptor}

Before running N-body simulations, we need to understand both the
typical halos which could host a sculptor like galaxy and the possible
orbits Sculptor.

Fig.~\ref{fig:scl_halos} presents approximate constraints based on

\subsubsection{Derived properties and
setup}\label{derived-properties-and-setup}

\begin{table*}[t]
\centering
\caption[Derived Properties of Sculptor]{To derive total mass, we use the absolute magnitude from Muñoz et al. (2018) (see also Table \ref{tbl:scl_obs_props}) along with the stellar mass to light ratio (1.7 with 0.17 dex uncertainty) from Woo, Courteau, and Dekel (2008). Sculptor’s total stellar mass is then $M_\star \sim 3.1_{-1.0}^{1.6} \times 10^6\,\Mo$. From the stellar mass to dark matter halo’s characteristic velocity $v_{\rm circ}$ in Fattahi et al. (2018), we expect Sculptor to have $v_{\rm circ} \approx 31 \pm 3 \kms$. }
\label{tbl:scl_derived_props}
\begin{tabular}{lll}
\toprule
parameter & value & reference\\
\midrule
$L_\star$ & $1.8\pm0.2\times10^6\ L_\odot$ & \\
$M_\star$ & $3.1_{-1.0}^{+1.6} \times10^6\ {\rm M}_\odot$ & \\
$M_\star / L_\star$ & $1.7\times 10^{\pm 0.17}$ & Woo, Courteau, and Dekel (2008)\\
$M_{200}$ & $0.5 \pm 0.2\times10^{10}\ M_0$ & \\
$c_{\rm NFW}$ & $13.1_{-2.8}^{+3.6}$ & \\
$v_{\rm circ, max}$ & $31\pm 3\,\kms$ & relation from Fattahi et al. (2018)\\
$r_{\rm circ, max}$ & $6 \pm 2$ kpc & $v_{\rm circ}$ with Ludlow et al. (2016) mass-concentration relation\\
$r_{\rm break, obs}$ & $25 \pm 5$ arcmin & (sestito+2024a?)\\
$t_{\rm break, obs}$ & $110\pm30$ Myr & $\sigma_v$, $r_{\rm break}$ with @\\
\bottomrule
\end{tabular}
\end{table*}

For the mass-to-light ratio, we use the values from
\citet{woo+courteau+dekel2008}. Note that we assume a scatter of 0.17
dex for the mass-to-light ratios as reported for mass uncertainties.
\citet{delosreyes+2024} also indicate that other methods for stellar
mass-to-light ratios result in similar scatters and systematic biases.
However these values are just approximate.

\begin{figure}
\centering
\pandocbounded{\includegraphics[keepaspectratio]{figures/scl_initial_halos.pdf}}
\caption[Sculptor initial halos]{The suggested halos of Sculptor
compared to cosmological predictions.}\label{fig:scl_halos}
\end{figure}

\begin{table*}[t]
\centering
\caption[Initial halos of Sculptor]{The parameters for our initial Sculptor halos. }
\label{tbl:scl_ini_halos}
\begin{tabular}{lllll}
\toprule
Halo name & Rmax & Vmax & M200 & c\\
\midrule
compact & 3.2 & 31 & 0.33 & 21\\
lmc & 4.2 & 31 & 0.39 & 17\\
small & 2.5 & 25 &  & \\
\bottomrule
\end{tabular}
\end{table*}

\subsection{Milky Way tides}\label{milky-way-tides}

\subsubsection{Orbital properties}\label{orbital-properties}

\begin{figure}
\centering
\pandocbounded{\includegraphics[keepaspectratio]{figures/scl_xyzr_orbits.pdf}}
\caption[Sculptor Orbits]{The orbits of Sculptor in a static Milky Way
potential in galactocentric \(x\), \(y\), and \(z\) coordinates. The
Milky Way is at the centre with the disk lying in the \(x\)--\(y\)
plane. Our selected \texttt{smallperi} orbit is plotted in black and
light blue transparent orbits represent the past 5Gyr orbits sampled
over Sculptor observables in Table~\ref{tbl:scl_obs_props}. The orbit of
sculptor is well-constrained in this potential and it is unlikely to
achieve a smaller pericentre than our selected orbit.
\the\textwidth}\label{fig:scl_orbits}
\end{figure}

To select an orbit with \textasciitilde{} maximum possible
observationally-consistent tidal forces, we take the median parameters
for orbits with a pericentre less than 2x the 3\(\sigma\) minimum
pericentre and derive the following orbit. The initial conditions are
from the

\begin{table*}[t]
\centering
\caption[Sculptor Selected Orbits]{Properties of selected orbits for Sculptor. The mean orbit represents the observational mean from Table \ref{tbl:scl_obs_props}. The Smallperi represents instead the $3\sigma$ smallest pericentre, which we use to provide an upper limit on tidal effects. }
\label{tbl:scl_obrits}
\begin{tabular}{llll}
\toprule
Property & Mean & SmallPeri & LMC\\
\midrule
distance & 83.2 & 82.6 & \\
pmra & 0.099 & 0.134 & \\
pmdec & -0.160 & -0.198 & \\
Vlos & 111.2 & 111.2 & \\
$t_i$ & -8.74 & -9.43 & \\
$\hat{x}_{i}$ & [16.13, 92.47, 39.63] & [-2.49, -42.78, 86.10] & \\
$\vec{v}_i$ & [-2.37, -54.70, 128.96] & [-20.56, -114.83, -57.29] & \\
pericentre & 53 & 43 & \\
apocentre & 102 & 96 & \\
$t_{\rm last\ peri}$ & -0.45 & -0.46 & \\
Numer of peris &  &  & \\
\bottomrule
\end{tabular}
\end{table*}

\subsection{Tidal effects}\label{tidal-effects}

\begin{itemize}
\tightlist
\item
  dark matter loss
\item
  little influence to stars
\item
  smallest pericentre does not help
\end{itemize}

\begin{figure}
\centering
\pandocbounded{\includegraphics[keepaspectratio]{figures/scl_sim_images.png}}
\caption[Sculptor simulation snapshots]{Images of the dark matter
evolution over a selection of past apocentres and the present day
position. Limits range from -150 to 150 kpc in the \(y\)-\(z\)
(approximately orbital) plane and the colourscale is logarithmic
spanning 5 orders of magnitude between the maximum and minimum values.
In this image, stars occupy only ever a few pixels so are not plotted.}
\end{figure}

\begin{figure}
\centering
\pandocbounded{\includegraphics[keepaspectratio]{figures/scl_tidal_track.pdf}}
\caption[Sculptor Tidal Tracks]{The tidal tracks for the smallperi
orbit. Todo: add velocity dispersion plot to RHS}
\end{figure}

\begin{figure}
\centering
\pandocbounded{\includegraphics[keepaspectratio]{figures/scl_sigma_v_time.pdf}}
\caption[Sculptor velocity dispersion evolution]{Evolution of stellar
velocity dispersion within 1 kpc for different Scl models. In all cases,
the evolution is mild. Note that binarity may reduce the inflate the
observed velocity dispersion by \textasciitilde{} 1 km/s, so the
conservative lower limit is around 8 km/s.}
\end{figure}

\begin{figure}
\centering
\pandocbounded{\includegraphics[keepaspectratio]{figures/scl_smallperi_i_f.pdf}}
\caption[Sculptor initial and final density profiles]{Effects on
exponential initial stars. TODO: plot 2D sky proj. stars}
\end{figure}

\begin{figure}
\centering
\pandocbounded{\includegraphics[keepaspectratio]{figures/scl_plummer_i_f.pdf}}
\caption[Sculptor Plummer initial and final density profiles]{effects on
Plummer initial stars.}
\end{figure}

\subsection{Effects of the LMC}\label{effects-of-the-lmc}

\subsubsection{Orbital effects}\label{orbital-effects}

As discussed in \citet{battaglia+2022}, Sculptor's orbit is strongly
influenced by the presence of an LMC. Figure
Fig.~\ref{fig:scl_lmc_orbit_effect}, the addition of an LMC reduces
Scl's pericentre with the MW and implies that Scl may be on its first
infall with the MW, in contrast to the discussion above. Thus, the LMC
has a critical impact on the evolution of Scl. Additionally, Scl passes
rapidly but relatively close to the LMC in the past 100 Myr.

Finally, we also consider the influence of the large Milky Cloud on the
orbits and evolution of Scl. We adopt the \citet{vasiliev2024} multipole
approximation of an N-body simulation of the LMC and MW. Their initial
conditions are

\begin{itemize}
\tightlist
\item
  MW halo:
\item
  MW bulge (static):
\item
  MW disk (static):
\item
  LMC halo:
\end{itemize}

We now consider 3 orbits under the L3M11 potential model from
\citet{vasiliev2024}. We focus on this LMC potential as a lighter LMC or
MW should only reduce tidal impacts and the recent orbit of Scl is
minimally affected by the LMC potential.

\begin{figure}
\centering
\pandocbounded{\includegraphics[keepaspectratio]{figures/scl_lmc_xyzr_orbits.pdf}}
\caption[Sculptor Orbits with LMC]{This figure is similar to
Fig.~\ref{fig:scl_orbits} except that we are showing the orbits with and
without an LMC. In the bottom row, the distance from Sculptor (or the
LMC) to the MW is plotted (left), and the Sculptor - LMC distance
(right.)}\label{fig:scl_lmc_orbit_effect}
\end{figure}

\textbf{tinyperilmc}

\begin{itemize}
\item
  ra = 15.0183 dec = -33.7186 distance = 73.1 pmra = 0.137 pmdec =
  -0.156 radial\_velocity = 111.2

  t\_i = -2.00 pericentre = 38.82 apocentre = 187.50 t last peri = -0.33
  x\_i = {[}4.30 138.89 125.88{]} v\_i = {[}6.89 -56.83 52.09{]}
\end{itemize}

Smallperi

\begin{itemize}
\item
  ra = 15.0183 dec = -33.7186 distance = 84.0 pmra = 0.166 pmdec =
  -0.237 radial\_velocity = 111.2

  t\_i = -2.00 pericentre = 28.00 apocentre = 190.45 t last peri = -0.39
  x\_i = {[}12.79 168.76 87.33{]} v\_i = {[}0.37 -57.69 61.40{]}
\end{itemize}

\subsubsection{Tidal effects}\label{tidal-effects-1}

\begin{figure}
\centering
\pandocbounded{\includegraphics[keepaspectratio]{figures/scl_lmc_sim_images.pdf}}
\caption{Sculptor Simulation Snapshots with LMC}
\end{figure}

\begin{figure}
\centering
\pandocbounded{\includegraphics[keepaspectratio]{figures/scl_lmc_i_f.pdf}}
\caption[Sculptor initial and final density with LMC]{The tidal effects
on the stellar surface density due to the LMC today.}
\end{figure}

\section{Ursa Minor}\label{ursa-minor}

\subsection{Derived properties}\label{derived-properties}

\begin{table*}[t]
\centering
\caption[Derived Properties of Ursa Minor]{Derived properties of Ursa Minor. }
\label{tbl:umi_derived_props}
\begin{tabular}{lll}
\toprule
parameter & value & reference\\
\midrule
$L_\star$ & $3.5 \pm 0.1 \times 10^5\,L_\odot$ & \\
$M_\star$ & $7_{-2}^{+3} \times 10^5\,\Mo$ & \\
$M_\star / L_\star$ & 1.9 (pm 0.17 dex) & Woo, Courteau, and Dekel (2008)\\
$M_{200}$ & $3_{-2}^{+4} \times 10^9\,\Mo$ & \\
$c$ & 14? & \\
$v_{\rm circ, max}$ & $27_{-6}^{+7}\,\kms$ & \\
$r_{\rm circ, max}$ & $5_{-2}^{+1}$ kpc & \\
$r_{\rm break, obs}$ & $30 \pm 5$ arcmin & \\
$t_{\rm break, obs}$ & $120\pm30$ Myr & \\
\bottomrule
\end{tabular}
\end{table*}

\begin{table*}[t]
\centering
\caption[Ursa Minor Initial Halos]{Initial halos for Ursa Minor. }
\label{tbl:umi_ini_halos}
\begin{tabular}{lllll}
\toprule
Halo name & Rmax & Vmax & M200 & c\\
\midrule
fiducial & 5 & 37 & 0.67 & 17\\
compact & 4 & 38 & 0.62 & 21\\
\bottomrule
\end{tabular}
\end{table*}

\begin{figure}
\centering
\pandocbounded{\includegraphics[keepaspectratio]{figures/umi_initial_halos.pdf}}
\caption[Ursa Minor initial halos]{The suggested halos of Ursa Minor
compared to cosmological predictions.}
\end{figure}

\subsection{Orbital properties}\label{orbital-properties-1}

\begin{figure}
\centering
\pandocbounded{\includegraphics[keepaspectratio]{figures/umi_xyzr_orbits.pdf}}
\caption{Ursa Minor Orbits}
\end{figure}

\subsection{Tidal Effects}\label{tidal-effects-2}

ra = 227.242 dec = 67.2221 distance = 64.6 pmra = -0.158 pmdec = 0.05
radial\_velocity = -245.75

t\_i = -9.53 pericentre = 29.64 apocentre = 74.88 t last peri = -0.80
x\_i = {[}-16.48 69.92 21.05{]} v\_i = {[}16.32 39.86 -116.99{]}

\begin{figure}
\centering
\pandocbounded{\includegraphics[keepaspectratio]{figures/umi_sim_images.png}}
\caption[Ursa Minor simulation snapshots]{Ursa Minor simulation images.}
\end{figure}

Figure: Velocity dispersion evolution of Ursa Minor

\begin{figure}
\centering
\pandocbounded{\includegraphics[keepaspectratio]{figures/umi_smallperi_i_f.pdf}}
\caption[Ursa Minor simulated density profiles]{The tidal effects on the
stellar surface density.}
\end{figure}

\subsection{Effects of the LMC}\label{effects-of-the-lmc-1}

\begin{figure}
\centering
\pandocbounded{\includegraphics[keepaspectratio]{figures/umi_lmc_xyzr_orbits.pdf}}
\caption[Ursa Minor orbits with LMC]{Orbits of Ursa Minor with (orange)
and without (green) an LMC. The final positions of Ursa Minor and the
LMC are plotted as scatter points and the solid blue line represents the
LMC trajectory. Note that the LMC only increases Ursa Minor's peri and
apo-centres, weakening any tidal effect. Interestingly, there is a
change that Ursa Minor may have once been bound to the LMC (diverging
orange lines at top left of middle panel.)}
\end{figure}
