\section{Abstract}\label{abstract}

\section{Introduction}\label{introduction}

Sculptor (Scl) is one of the first discovered dwarf galaxies of the
Milky Way (Shapley 1938; only preceded by the SMC and LMC!). As a
classical dwarf spheroidal, Scl is relatively bright and compact.

Since the initial discovery of Scl, most authors have noted that Scl has
a slight ellipticity (\(0.36\)), often attributed to tidal effects.
However, this does not align with the absolute proper motion or orbital
path of the galaxy.

While Scl has a relatively large pericentre (greater than 50kpc),
\citet{sestito+2023a} detect that the galaxy likely has an excess of
stars in the outskirts (past about 60 arcminutes). This excess is
perhaps one of the clearest indications that Scl may be affected by
tides. Here, our goal is to determine if under a \(\Lambda\) CMD
paradigm with DM-only simulations, tidal effects of the Milky Way (or
LMC) may indeed be consistent with these observations, or if these
observations may reveal instead a extended stellar ``halo'' or second
component of the galaxy---illustrating a complex formation for the
galaxy.

Theoretical work on Sculptor

\begin{itemize}
\tightlist
\item
  \citet{battaglia+2008}
\item
  \citet{iorio+2019}
\item
  \citet{amorisco+zavala+deboer2014}
\item
  \citet{battaglia+2008}
\item
  \citet{breddels+2013}
\item
  \citet{breddels+helmi2013}
\item
  \citet{richardson+fairbairn2014}
\item
  \citet{SFW2017}
\item
  \citet{innanen+papp1979}
\item
  \citet{wilkinson+2002}
\item
  \citet{yang+2025}: chemical evolution in Scl.
\item
  \citet{skuladottir+2024}; \citet{skuladottir+2021};
  \citet{lee+jeon+bromm2024}, pop III high res spectro?
\item
  \citet{wang+2024a} hydrosim of dwarf galaxies like sculptor
\item
  \citet{tang+2023}: apogee modeling of Scl.
\item
  \citet{pandey+west2022} chem evo OMEGA and isotopes.
\item
  \citet{an+koposov2022}: distance gradients?
\item
  \citet{kawata+2006}, 2pop origins?
\end{itemize}

Observational work on Scl

\begin{itemize}
\tightlist
\item
  \citet{sestito+2023a}
\item
  \citet{westfall+2006} wide degree survey for extended structure.
\item
  \citet{tolstoy+2023}, \citet{arroyo-polonio+2023},
  \citet{arroyo-polonio+2024}
\item
  \citet{eskridge1988}, \citet{eskridge1988a}, \citet{eskridge1988b}
\item
  \citet{coleman+dacosta+bland-hawthorn2005}
\item
  \citet{DQ1994}
\item
  \citet{WMO2009}
\item
  \citet{IH1995},
\item
  \citet{munoz+2018}: Using Megacam to derive density profiles and
  structural properties of many dwarf spheroidal galaxies.
\item
  \citet{kirby+2009}
\item
  \citet{martinez-vazquez+2015}, \citet{pietrzynski+2008}
\item
  \citet{grebel1996}
\item
  \citet{barbosa+2025}: Using DECam to derive narrowband photometric
  metallicity gradient and search for metal poor stars in Scl.
\end{itemize}

Future ideas:

\begin{itemize}
\tightlist
\item
  \citet{evslin2016}: measuring ellipticities of halos w TMT.
\end{itemize}

\subsection{Derived properties}\label{derived-properties}

\begin{table*}[t]
\centering
\begin{tabular}{lll}
\toprule
parameter & value & reference\\
\midrule
$L_\star$ & $1.82_{-0.22}^{+0.25}\times10^6\ L_\odot$ & \\
$M_\star$ & $2.7_{-0.6}^{+0.7} \times10^6\ {\rm M}_\odot$ & \\
$M_{200}$ & $0.48_{-0.25}^{+0.52}\ M_0$ & \\
$c_{\rm NFW}$ & $13.1_{-2.8}^{+3.6}$ & \\
$v_{\rm circ, max}$ & $31\pm8\,\kms$ & \\
$r_{\rm circ, max}$ & $5.9 \pm 2.9$ kpc & \\
$r_{\rm break, obs}$ & $25 \pm 5$ arcmin & \\
$t_{\rm break, obs}$ & $110\pm30$ Myr & \\
\bottomrule
\end{tabular}
\end{table*}

\section{Sculptor (MW only)}\label{sculptor-mw-only}

\subsection{Orbital estimation}\label{orbital-estimation}

Fig.~\ref{fig:unref} contains example orbits

Given the parameters in table@REF, we

\subsection{Simulations}\label{simulations}

\begin{itemize}
\tightlist
\item
  dark matter loss
\item
  little influence to stars
\item
  smallest pericentre does not help
\end{itemize}

\section{Sculptor (LMC orbits)}\label{sculptor-lmc-orbits}

As discussed in \citet{battaglia+2022}, Sculptor's orbit is strongly
influenced by the presence of an LMC. Figure
Fig.~\ref{fig:scl_lmc_orbit_effect}, the addition of an LMC reduces
Scl's pericentre with the MW and implies that Scl may be on its first
infall with the MW, in contrast to the discussion above. Thus, the LMC
has a critical impact on the evolution of Scl. Additionally, Scl passes
rapidly but relatively close to the LMC in the past 100 Myr.

We now consider 3 orbits under the L3M11 potential model from
\citet{vasiliev2024}. We focus on this LMC potential as a lighter LMC or
MW should only reduce tidal impacts and the recent orbit of Scl is
minimally affected by the LMC potential.

\begin{figure}
\centering
\includegraphics[width=1\linewidth,height=\textheight,keepaspectratio]{/Users/daniel/Library/Application Support/typora-user-images/image-20250415112706417.png}
\caption[Effect of LMC on Sculptor's Orbit]{Orbits of Sculptor with and
without the LMC. The potential is that of \citet{vasiliev+2021} in each
case.}
\end{figure}

\begin{figure}
\centering
\includegraphics[width=1\linewidth,height=\textheight,keepaspectratio]{/Users/daniel/Library/Application Support/typora-user-images/image-20250415113122385.png}
\caption[Selected orbits of Sculptor with an LMC]{Selected orbits of
Sculptor in the MW frame. \textbf{TODO}: include LMC distance or frame,
maybe xyz instead?}
\end{figure}
