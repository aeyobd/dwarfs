In this chapter, we present our results from our simulations of Sculptor
and Ursa Minor. In each case, we first discuss the range of reasonable
orbits and initial dark matter halos for each galaxy. Then, we describe
how tides affect the evolution of each galaxy. We furthermore consider
how the effects of the LMC changes the results, including using
additional N-body simulations in evolving potentials. Finally, we review
the predictions and characteristics of the present-day stellar
populations and properties from our models.

\section{Sculptor}\label{sculptor}

\subsubsection{Derived properties and
setup}\label{derived-properties-and-setup}

From the observed properties of Sculptor, we can infer reasonable
\(\Lambda\)CMD dark matter halo hosts. Table~\ref{tbl:scl_derived_props}
reports our inferred halo and kinematic properties of Sculptor. First,
taking the absolute magnitude from \citet{munoz+2018} with the
mass-to-light ratio from \citet{woo+courteau+dekel2008} (1.7 with
\(\sim\) 0.17 dex uncertainty), the total current stellar mass of
Sculptor is \(M_\star \sim 3.1 \times 10^6 \Mo\). Based on the stellar
mass-\(v_{\rm max}\) relation \citep[from][]{fattahi+2018}, see also
\ref{fig:SMHM}, the halo should have
\(v_{\rm max} \sim 31 \pm 3 \,\kms\). Finally, using the
\citet{ludlow+2016} \(z=0\) mass-concentration relation, this constraint
translates into a \(r_{\rm max} \sim 6 \pm 2\,{\rm kpc}\), or
alternatively \(M_{200} = 5\times10^9\) with \(c=13\).

Fig.~\ref{fig:scl_halos} illustrates these estimates visually. The
stellar-mass \(v_{\rm max}\) constraint translates to an estimate of
\(v_{\rm max}\) only (horizontal band). The mass-concentration relation
describes the relationship between \(v_{\rm max}\) and \(r_{\rm max}\),
an approximately diagonal linear band. And finally, we include a curved
line which illustrates where the initial velocity dispersion is
approximantly larger than the present-day observed dispersion
\(v_{\rm circ}(R_h) / \sqrt{3} \sim \sigma_v \gtrsim 9\,\kms\).

The intersection of the constraints on the halo leads us to select an
initial halo, \texttt{compact}, which is more compact than the mean but
with a observationally consistent initial velocity dispersion. We also
have the \texttt{light} halo which occupies the intersection between
3\(\sigma\) lowest mass and the velocity dispersion constraints, which
we come back to in Section~\ref{sec:scl_lmc}.

While there is some range in the choice of initial halo, reasonable
changes to the initial halo do not substantially affect the tidal
evolution for Sculptor. The velocity dispersion, in particular,
translates to an estimate of the mass within \(r_h\). As a result, halos
with the same velocity dispersion only differ in total mass, but the
density slope and circular velocity near the stellar component is
similar. These halos will have a similarly affected stellar component.

\begin{table*}[t]
\centering
\caption[Derived Properties of Sculptor]{To derive total mass, we use the absolute magnitude from Muñoz et al. (2018) (see also Table \ref{tbl:scl_obs_props}) along with the stellar mass to light ratio (1.7 with 0.17 dex uncertainty) from Woo, Courteau, and Dekel (2008). Sculptor’s total stellar mass is then $M_\star \sim 3.1_{-1.0}^{1.6} \times 10^6\,\Mo$. From the stellar mass to dark matter halo’s characteristic velocity $v_{\rm circ}$ in Fattahi et al. (2018), we expect Sculptor to have $v_{\rm circ} \approx 31 \pm 3 \kms$. }
\label{tbl:scl_derived_props}
\begin{tabular}{lll}
\toprule
parameter & value & reference\\
\midrule
$L_\star$ & $1.8\pm0.2\times10^6\ L_\odot$ & \\
$M_\star$ & $3.1_{-1.0}^{+1.6} \times10^6\ {\rm M}_\odot$ & \\
$M_\star / L_\star$ & $1.7\times 10^{\pm 0.17}$ & Woo, Courteau, and Dekel (2008)\\
$v_{\rm circ, max}$ & $31\pm 3\,\kms$ & relation from Fattahi et al. (2018)\\
$r_{\rm circ, max}$ & $6 \pm 2$ kpc & $v_{\rm circ}$ with Ludlow et al. (2016) mass-concentration relation\\
$M_{200}$ & $0.5 \pm 0.2\times10^{10}\ M_0$ & \\
$c_{\rm NFW}$ & $13.1_{-2.8}^{+3.6}$ & \\
$r_{\rm break, obs}$ & $25 \pm 5$ arcmin & (sestito+2024a?)\\
$t_{\rm break, obs}$ & $110\pm30$ Myr & $\sigma_v$, $r_{\rm break}$ with @\\
\bottomrule
\end{tabular}
\end{table*}

\begin{figure}
\centering
\pandocbounded{\includegraphics[keepaspectratio]{figures/scl_initial_halos.pdf}}
\caption[Sculptor initial halos]{The suggested halos of Sculptor
compared to cosmological predictions. \textbf{TODO: simplify
\(\sigma_v\) calculation}.}\label{fig:scl_halos}
\end{figure}

\begin{table*}[t]
\centering
\caption[Initial halos of Sculptor]{The parameters for our initial Sculptor halos. }
\label{tbl:scl_ini_halos}
\begin{tabular}{lllll}
\toprule
Halo name & Rmax & Vmax & M200 & c\\
\midrule
compact & 3.2 & 31 & 0.33 & 21\\
lmc & 4.2 & 31 & 0.39 & 17\\
small & 2.5 & 25 &  & \\
\bottomrule
\end{tabular}
\end{table*}

\subsection{Milky Way tides}\label{milky-way-tides}

\subsubsection{Orbital properties}\label{orbital-properties}

Next, we calculate the range of possible orbits of Sculptor in our
fiducial Milky Way potential. Fig.~\ref{fig:scl_orbits} illustrates
point particle orbits for 100 samples of Sculptor's observed kinematics
integrated backwards 5 Gyr in both galactocentric coordinate slices (x,
y, z) and in galactocentric radius with time. In terms of the x, y, z
planes, note that all sampled orbits of Sculptor have nearly the same
morphology---they each orbit mostly in the y-z plane, passing through
the same number of pericentres and maintaining similar orbits across
5Gyr. Sculptor's orbital history is relatively well-constrained.

We select an orbit with the \(\sim 3\sigma\) smallest pericentre among
all possible orbits. We achieve this by taking the median parameters of
all orbits with a pericentre less than the 0.0027th quantile pericentre,
yielding a pericentre of 43 kpc.\footnote{only slightly smaller than the
  0.00135th quantile pericentre} It is unlikely given the current
observations that Sculptor has a smaller pericentre. Since tidal effects
depend most strongly on the pericentre of an orbit, this orbit should
maximize the tidal force from the Milky Way.

We take the first apocentre after a lookback time of 10 Gyr, or at 9.43
Gyr, as the initial conditions for our model of Sculptor.
Table~\ref{tbl:scl_orbits} notes the initial conditions for our
simulation.

\begin{figure}
\centering
\pandocbounded{\includegraphics[keepaspectratio]{figures/scl_xyzr_orbits.pdf}}
\caption[Sculptor Orbits]{The orbits of Sculptor in a static Milky Way
potential in galactocentric \(x\), \(y\), and \(z\) coordinates. The
Milky Way is at the centre with the disk lying in the \(x\)--\(y\)
plane. Our selected \texttt{smallperi} orbit is plotted in black and
light blue transparent orbits represent the past 5Gyr orbits sampled
over Sculptor observables in Table~\ref{tbl:scl_obs_props}. The orbit of
sculptor is well-constrained in this potential and it is unlikely to
achieve a smaller pericentre than our selected
orbit.}\label{fig:scl_orbits}
\end{figure}

\begin{table*}[t]
\centering
\caption[Sculptor Selected Orbits]{Properties of selected orbits for Sculptor. The mean orbit represents the observational mean from Table \ref{tbl:scl_obs_props}. The Smallperi represents instead the $3\sigma$ smallest pericentre, which we use to provide an upper limit on tidal effects. }
\label{tbl:scl_obrits}
\begin{tabular}{llll}
\toprule
Property & Mean & SmallPeri & LMC\\
\midrule
distance & 83.2 & 82.6 & \\
pmra & 0.099 & 0.134 & \\
pmdec & -0.160 & -0.198 & \\
Vlos & 111.2 & 111.2 & \\
$t_i$ & -8.74 & -9.43 & \\
$\hat{x}_{i}$ & [16.13, 92.47, 39.63] & [-2.49, -42.78, 86.10] & \\
$\vec{v}_i$ & [-2.37, -54.70, 128.96] & [-20.56, -114.83, -57.29] & \\
pericentre & 53 & 43 & \\
apocentre & 102 & 96 & \\
$t_{\rm last\ peri}$ & -0.45 & -0.46 & \\
Numer of peris &  &  & \\
\bottomrule
\end{tabular}
\end{table*}

\subsection{Tidal effects}\label{tidal-effects}

The Milky Way tides indeed affect Sculptor's dark matter halo.
Fig.~\ref{fig:scl_sim_images} shows projected dark matter distribution
of our N-body simulation for Sculptor in the \(y\)-\(z\) plane for 5
previous pericentres. This simulation uses the \texttt{compact} initial
halo on the \texttt{smallperi} orbit. The dark matter forms large
streams which orbit the Milky Way several times. However, note that the
central dark matter cusp is similar in appearance across all simulation
snapshots.

More quantitatively, the halo loses about 99\% of its initial dark
matter mass. This corresponds in a reduction from the initial
\(v_{\rm max} = 31\,\kms\) to \(v_{\rm max} = 22\,\kms\). However, the
stellar component of Sculptor is relatively unaffected.

While mass loss leads to adiabatic expansion of the stellar component,
no features of tidal disruption or disequilibrium are currently
observable. Fig.~\ref{fig:scl_smallperi_i_f} shows the final projected
density of stars on the sky, and the initial and final radially averaged
density profiles. No non-spherical density features are apparent, even
at 5 decades fainter in surface density than the central density. In
addition, the initial and final density profiles look identical up to
some scale, with a tidally-induced excess of stars only appearing
\(\sim 3\) orders of magnitude fainter than the faintest our density
profile measures.

The analytically motivated break and tidal radii corroborate a weak
tidal effect on stars. In both cases, these radii work out to be
\(\gtrsim 100\) arcminutes, outside where we can measure the density
profile. In particular, because we have chosen the most extreme
observationally permissible orbit, it is unlikely, for similar
\(\Lambda\)CDM motivated initial conditions, that these radii would ever
approach where the break in Sculptor's density profile appears.

\begin{figure}
\centering
\pandocbounded{\includegraphics[keepaspectratio]{figures/scl_sim_images.png}}
\caption[Sculptor simulation snapshots]{Images of the dark matter
evolution over a selection of past apocentres and the present day
position. Limits range from -150 to 150 kpc in the \(y\)-\(z\)
(approximately orbital) plane and the colourscale is logarithmic
spanning 5 orders of magnitude between the maximum and minimum values.
In this image, stars occupy only ever a few pixels so are not
plotted.}\label{fig:scl_sim_images}
\end{figure}

\begin{figure}
\centering
\pandocbounded{\includegraphics[keepaspectratio]{figures/scl_tidal_track.pdf}}
\caption[Sculptor Tidal Tracks]{The tidal tracks for the smallperi
orbit. Todo: add velocity dispersion plot to
RHS}\label{fig:scl_tidal_track}
\end{figure}

\begin{figure}
\centering
\pandocbounded{\includegraphics[keepaspectratio]{figures/scl_sigma_v_time.pdf}}
\caption[Sculptor velocity dispersion evolution]{Evolution of stellar
velocity dispersion within 1 kpc for different Scl models. In all cases,
the evolution is mild. Note that binarity may reduce the inflate the
observed velocity dispersion by \textasciitilde{} 1 km/s, so the
conservative lower limit is around 8 km/s. \textbf{TODO}, add bound mass
with time. Maybe combine with tidal tracks and radius / time orbit.}
\end{figure}

\begin{figure}
\centering
\pandocbounded{\includegraphics[keepaspectratio]{figures/scl_smallperi_i_f.pdf}}
\caption[Sculptor initial and final density profiles]{Effects on
exponential initial stars. TODO: plot 2D sky proj.
stars}\label{fig:scl_smallperi_i_f}
\end{figure}

\begin{figure}
\centering
\pandocbounded{\includegraphics[keepaspectratio]{figures/scl_plummer_i_f.pdf}}
\caption[Sculptor Plummer initial and final density profiles]{effects on
Plummer initial stars.}\label{fig:scl_smallperi_plummer_i_f}
\end{figure}

\subsection{Effects of the LMC}\label{effects-of-the-lmc}

The Milky Way isn't the only galaxy in town. Becoming more apparent, the
infall of the LMC substantially affects the Milky Way system. The mass
of the LMC may be as high as 1/5 of the Milky Way mass, so including an
LMC in the Milky Way potential may change conclusions about properties
and orbits of the Milky Way satellite system.

\subsubsection{Orbital effects}\label{orbital-effects}

As discussed in \citet{battaglia+2022}, Sculptor's orbit is strongly
influenced by the presence of an LMC.

To explore the effects of the LMC on Sculptor, we use the publically
available L3M11 model from \citet{vasiliev2024}. This model stars with a
lighter Milky Way than our fiducial \citet{EP2020}, \citet{mcmillan2011}
like model. The L3M11 potential is an evolving multipole approximation
of an N-body simulation with both a live MW and LMC dark matter halo.
The MW intital conditions were an NFW with \(r_s=16.5\,\)kpc and mass
\(M_{\rm 100}= 11\times10^{11}\Mo\), and the LMC was a NFW halo with
\(r_s=11.7\) and \(M_{100} = 2.76 \times 10^{11} \Mo\). This model had a
previous LMC pericentre at about 6 Gyr ago.

Fig.~\ref{fig:scl_lmc_orbit_effect} displays the effect of including an
LMC in the potential. The green samples are in the initial MW-only
potential in the \texttt{L3M11} model, and the orange samples are
integrated in the evolving MW and LMC \texttt{L3M11} model. The past 1
Gyr is similar in both cases, but the orbits diverge significantly
afterwards. The recent passage of Sculptor with the LMC around 0.1 Gyr
ago allows for Sculptor to begin as far as 300 kpc from the Milky Way
centre. The evolving potential also adds significant long term
variability in the possible orbits of Sculptor. Sculptor, however, is
orbiting in the opposite direction of the LMC so is likely not
associated with the LMC system.

Given the large uncertainties of the LMC model, we conservatively double
all of the observational parameters of Sculptor. This has a similar
effect to including LMC mass and orbital uncertainties but is
considerably simpler. From this larger range of orbits, we once again
select the orbit with the median final observables of all orbits with
pericentres less than the 0.0027th quantile. This orbit, the
\texttt{smallperilmc} orbit is plotted in black and is only integrated
up to 2 Gyr ago, to isolate recent tidal effects.

\begin{figure}
\centering
\pandocbounded{\includegraphics[keepaspectratio]{figures/scl_lmc_xyzr_orbits.pdf}}
\caption[Sculptor Orbits with LMC]{This figure is similar to
Fig.~\ref{fig:scl_orbits} except that we are showing the orbits with and
without an LMC. In the bottom row, the distance from Sculptor (or the
LMC) to the MW is plotted (left), and the Sculptor - LMC distance
(right.)}\label{fig:scl_lmc_orbit_effect}
\end{figure}

\textbf{tinyperilmc}

\begin{itemize}
\item
  ra = 15.0183 dec = -33.7186 distance = 73.1 pmra = 0.137 pmdec =
  -0.156 radial\_velocity = 111.2

  t\_i = -2.00 pericentre = 38.82 apocentre = 187.50 t last peri = -0.33
  x\_i = {[}4.30 138.89 125.88{]} v\_i = {[}6.89 -56.83 52.09{]}
\end{itemize}

\subsubsection{Tidal effects}\label{tidal-effects-1}

Unlike the previous mdoel, this model only has one pericentre with the
Milky Way. Fig.~\ref{fig:scl_lmc_sim_images} shows snapshots of Sculptor
over the past 2 Gyr while marking the position of the LMC. With only one
pericentre, and a larger one than the \texttt{smallperi} orbit,
Sculptor's dark matter is substantially less disrupted. And while, based
on the tidal tensor values, the LMC induces a greater instantaneous
tidal effect than the Milky Way, Sculptor's dark matter component does
not show strong effects due to the LMC.

Finally, Fig.~\ref{fig:scl_lmc_i_f} shows the projected on-sky final
stellar distribution and the initial and final stellar density profile
for this model with exponential stars. The stellar component does not
change at all. While the break radius set by the time since the last LMC
pericentre would agree with the location of the break in the observed
density profile, no stellar effect would be observable.

The LMC flyby encounter is an approximately impulsive encounter, in
contrast with more adiabadic mass loss due to the Milky Way. Impulsive
encounters tend to inject energy into the stellar and dark matter
distribution, and can initially cause the galaxy to contract. In
addition, the tidal force is required to be far larger than for slower,
adiabadic mass loss, because the galaxy experiences this tidal field for
far less time. Even in models where Sculptor passes through the the LMC,
tidal tails do not form immediately after this encounter.

A final note is that this model only considers the recent tidal effects
.As illustrated in the range of possible orbits in
Fig.~\ref{fig:scl_lmc_orbits_effect}, there is a chance that Sculptor
experienced an extremely small pericentre with the Milky Way. This
pericentre has the potential to substantially rearrange the stellar
component, drive large tidal mass loss, and create a Plummer-like
stellar density profile. However, this encounter is highly dependent on
the choice of the MW-LMC potential model, and may not occur at all. Long
time integration in dynamic potentials amplifies uncertainties in the
inputs and may not be reliable. \textbf{TODO: I would love to get this
model to work\ldots{}}

\begin{figure}
\centering
\pandocbounded{\includegraphics[keepaspectratio]{figures/scl_lmc_sim_images.pdf}}
\caption{Sculptor Simulation Snapshots with
LMC}\label{fig:scl_lmc_sim_images}
\end{figure}

\begin{figure}
\centering
\pandocbounded{\includegraphics[keepaspectratio]{figures/scl_lmc_i_f.pdf}}
\caption[Sculptor initial and final density with LMC]{The tidal effects
on the stellar surface density due to the LMC today.}
\end{figure}

\section{Ursa Minor}\label{ursa-minor}

\subsection{Derived properties}\label{derived-properties}

\begin{table*}[t]
\centering
\caption[Derived Properties of Ursa Minor]{Derived properties of Ursa Minor. }
\label{tbl:umi_derived_props}
\begin{tabular}{lll}
\toprule
parameter & value & reference\\
\midrule
$L_\star$ & $3.5 \pm 0.1 \times 10^5\,L_\odot$ & \\
$M_\star$ & $7_{-2}^{+3} \times 10^5\,\Mo$ & \\
$M_\star / L_\star$ & 1.9 (pm 0.17 dex) & Woo, Courteau, and Dekel (2008)\\
$M_{200}$ & $3_{-2}^{+4} \times 10^9\,\Mo$ & \\
$c$ & 14? & \\
$v_{\rm circ, max}$ & $27_{-6}^{+7}\,\kms$ & \\
$r_{\rm circ, max}$ & $5_{-2}^{+1}$ kpc & \\
$r_{\rm break, obs}$ & $30 \pm 5$ arcmin & \\
$t_{\rm break, obs}$ & $120\pm30$ Myr & \\
\bottomrule
\end{tabular}
\end{table*}

\begin{table*}[t]
\centering
\caption[Ursa Minor Initial Halos]{Initial halos for Ursa Minor. }
\label{tbl:umi_ini_halos}
\begin{tabular}{lllll}
\toprule
Halo name & Rmax & Vmax & M200 & c\\
\midrule
fiducial & 5 & 37 & 0.67 & 17\\
compact & 4 & 38 & 0.62 & 21\\
\bottomrule
\end{tabular}
\end{table*}

\begin{figure}
\centering
\pandocbounded{\includegraphics[keepaspectratio]{figures/umi_initial_halos.pdf}}
\caption[Ursa Minor initial halos]{The suggested halos of Ursa Minor
compared to cosmological predictions.}
\end{figure}

\subsection{Orbital properties}\label{orbital-properties-1}

\begin{figure}
\centering
\pandocbounded{\includegraphics[keepaspectratio]{figures/umi_xyzr_orbits.pdf}}
\caption{Ursa Minor Orbits}
\end{figure}

\subsection{Tidal Effects}\label{tidal-effects-2}

ra = 227.242 dec = 67.2221 distance = 64.6 pmra = -0.158 pmdec = 0.05
radial\_velocity = -245.75

t\_i = -9.53 pericentre = 29.64 apocentre = 74.88 t last peri = -0.80
x\_i = {[}-16.48 69.92 21.05{]} v\_i = {[}16.32 39.86 -116.99{]}

\begin{figure}
\centering
\pandocbounded{\includegraphics[keepaspectratio]{figures/umi_sim_images.png}}
\caption[Ursa Minor simulation snapshots]{Ursa Minor simulation images.}
\end{figure}

Figure: Velocity dispersion evolution of Ursa Minor

\begin{figure}
\centering
\pandocbounded{\includegraphics[keepaspectratio]{figures/umi_smallperi_i_f.pdf}}
\caption[Ursa Minor simulated density profiles]{The tidal effects on the
stellar surface density.}
\end{figure}

\subsection{Effects of the LMC}\label{effects-of-the-lmc-1}

\begin{figure}
\centering
\pandocbounded{\includegraphics[keepaspectratio]{figures/umi_lmc_xyzr_orbits.pdf}}
\caption[Ursa Minor orbits with LMC]{Orbits of Ursa Minor with (orange)
and without (green) an LMC. The final positions of Ursa Minor and the
LMC are plotted as scatter points and the solid blue line represents the
LMC trajectory. Note that the LMC only increases Ursa Minor's peri and
apo-centres, weakening any tidal effect. Interestingly, there is a
change that Ursa Minor may have once been bound to the LMC (diverging
orange lines at top left of middle panel.)}
\end{figure}
