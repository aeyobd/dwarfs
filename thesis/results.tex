As discussed in Chapters \ref{sec:introduction} and
\ref{sec:observations}, this thesis aims to test whether Galactic tides
are responsible for the extended density profiles of Scl and UMi. In
this Chapter, we analyze tailored N-body simulations, using the methods
described in Chapter \ref{sec:methods}, to assess the tidal impact of
the Galactic potential. To anticipate our main conclusion, we find that
tides drive dark matter loss in both systems but leave their compact
stellar components largely unaffected. The Large Magellanic Cloud (LMC)
has the potential of substantially perturbing Scl's and even UMi's
orbit, yet the resulting tidal effects are still too weak to account for
the extended outer profiles. Our simulations demonstrate that recent
tides are unlikely to have altered the stellar structure of Scl or UMi.

In this Chapter, we consider Scl first, describing tidal effects from
the MW on its dark matter and stellar components. Next, we consider how
accounting for the LMC may affect our conclusions. We then analyze UMi
in a similar manner, considering in turn the dark matter evolution,
stellar evolution, and orbital effects of the LMC.

\section{Tidal effects on Sculptor}\label{tidal-effects-on-sculptor}

\subsection{Evolution of Sculptor's dark matter
halo}\label{evolution-of-sculptors-dark-matter-halo}

As a representation of an extreme tidal history, we initially
investigate the \smallperi{} orbit, described in
Section~\ref{sec:scl_smallperi}, chosen to maximize possible effects of
Galactic tides.

Scl experiences moderate tidal mass loss after 10 Gyrs of evolution.
Fig.~\ref{fig:scl_sim_images} shows the stripping of dark matter and the
formation of diffuse streams trailing and leading Sculptor's orbit.
Because tidal stripping can be described as a gradual removal of the
least bound particles, most mass loss occurs in the outer halo. Instead,
the inner regions of the galaxy may be relatively unaffected.

N-body models may deviate from a point-particle trajectory due to
dynamical self-friction \citep[e.g.,][]{white1983, miller+2020}.
However, this effect is slight for Scl, which ends near the observed
position, without adjusting the initial conditions (the green point in
Fig.~\ref{fig:scl_sim_images}).

The inner density cusp is tidally resilient.
Fig.~\ref{fig:scl_tidal_track} shows the initial and final circular
velocity profiles, and the evolution of the maximum circular velocity.
The maximum velocity drops from \(31\,\kms\) to \(22\,\kms\), evolving
along the tidal track from \citet{EN2021}. The final circular velocity
profile resembles the initial with an inner cusp, but has a sharper
outer truncation. Quantitatively, the halo loses \(>90\%\) its initial
mass (see Table~\ref{tbl:scl_sim_results}). However, the inner structure
is not expected to be affected, as the Jacobi radius is over 3kpc,
outside of the initial and final \(\rmax\) (see
Table~\ref{tbl:scl_sim_results} and Fig.~\ref{fig:scl_tidal_track}).
Thus, tides may remove significant amounts of mass, but mostly from the
outer halo.

\begin{figure}
\centering
\pandocbounded{\includegraphics[keepaspectratio]{figures/scl_sim_images.png}}
\caption[Sculptor simulation snapshots]{Images of the dark matter
evolution over a selection of past apocentres and the present day.
Limits range from -150 to 150 kpc in the \(y\)--\(z\) (\(\sim\)orbital)
plane, and the colourscale is logarithmic spanning 5 orders of magnitude
between the maximum and minimum values. The green dot represents the
final expected position of the galaxy and the solid and dotted grey
curves represents the orbit over one previous or future radial
oscillation respectively.}\label{fig:scl_sim_images}
\end{figure}

\begin{figure}
\centering
\pandocbounded{\includegraphics[keepaspectratio]{figures/scl_tidal_track.png}}
\caption[Sculptor tidal tracks]{Dynamical evolution for the \smallperi{}
model of Sculptor. Dotted and solid lines show the initial and final
circular velocity profiles, and blue and orange lines show the dark
matter and stellar (2D exponential) profiles. The points represent the
evolution of the maximum circular velocity, and the dotted black line
shows the tidal track from \citet{EN2021}. To calculate the velocity
profiles, unbound particles are iteratively removed, recalculating the
potential at each step assuming spherical
symmetry.}\label{fig:scl_tidal_track}
\end{figure}

\begin{table*}[t]
\centering
\caption[Simulation results for Sculptor’s dark matter]{The orbital and dark matter properties for the simulation of Sculptor. The random samples column shows the distributions from point orbits, and the \smallperi{} column contains the results from the N-body simulation. }
\label{tbl:scl_sim_results}
\begin{tabular}{lll}
\toprule
Property & random samples & \smallperi{}\\
\midrule
pericentre & $53\pm3$ & 42\\
apocentre & $102\pm3$ & 94.4\\
time of last pericentre / Gyr & $-0.45 \pm 0.2$ & -0.47\\
number of pericentres & 6 & 6\\
Jacobi radius / kpc & $4.5 \pm 0.3$ & 3.5\\
Jacobi radius / arcmin & $186\pm12$ & 101\\
final heliocentric distance / kpc & $83.2\pm2$ & 81.6\\
$\V_\textrm{max, f} / \V_\textrm{max, i}$ &  & 0.695\\
$r_\textrm{max, f} / r_\textrm{max, i}$ &  & 0.406\\
fractional final bound mass &  & 0.0893\\
\bottomrule
\end{tabular}
\end{table*}

\subsection{Evolution of Sculptor's stars}\label{sec:scl_sim_stars}

Tides minimally affect the stellar component of Sculptor in the
\smallperi{} orbit. In Fig.~\ref{fig:scl_smallperi_i_f}, the projected
stellar distribution displays no prominent distortions, and the radial
density profile is nearly unchanged. Only at a surface density
\(\sim10^8\) times fainter than the centre do some faint tidal features
emerge. The total stellar mass lost corresponds to \(\sim 10\) stars in
total (see Table~\ref{tbl:scl_sim_results})---a formidable challenge
with the best of observations.

This result implies that Scl's extended profile cannot be reproduced by
Galactic tides operating on an initially exponential profile. The weak
effect of tides suggests that the outer profile of Sculptor is innate,
and not the result of tidal evolution. We check this assertion by
choosing a different initial stellar profile which matches the observed
profile and assessing how it evolves on the smallperi orbit. We show in
Fig.~\ref{fig:scl_smallperi_plummer_i_f} that a Plummer profile (instead
of an exponential) provides an adequate fit to Scl's observed profile.
The Plummer model loses more stellar mass and forms more luminous tidal
tails. Observations reaching surface densities \(\sim10\) times fainter
than our data could reveal a stream in this case. Nevertheless, over the
radial extent probed by our data, the stellar profile remains nearly
unchanged by tidal evolution.

The Jacobi and break radii further support that tidal effects should not
be apparent in the observed stellar component. As calculated for this
model, (see
Tables~\ref{tbl:scl_sim_results}, \ref{tbl:scl_sim_stars_results}), the
break and Jacobi radii both fall outside of \(\sim 100\) arcminutes for
either stellar component. Indeed, the stellar component only begins to
deviate from an exponential profile around this break radius
(Figs.~\ref{fig:scl_smallperi_i_f}, \ref{fig:scl_smallperi_plummer_i_f}).
Since no orbits of Scl produce significantly smaller break or Jacobi
radii, it is unlikely any orbit would produce an observable density
excess.

Table~\ref{tbl:scl_sim_stars_results} quantifies the evolution of
stellar properties. The stellar velocity dispersion decreases by only
\(\sim1\,\kms\) and the half-light radius expands by \(\sim 10\%\). This
is consistent with adiabatic expansion due to the reduction of the total
mass \citep[e.g.,][]{stucker+2023}. In addition, the break and Jacobi
radii are \(\gtrsim 100\) arcminutes on the sky---tidal signatures would
be beyond the reach of our data. Altogether, Galactic tides negligibly
impact Scl's stellar component.

\begin{figure}
\centering
\pandocbounded{\includegraphics[keepaspectratio]{figures/scl_smallperi_i_f.pdf}}
\caption[Sculptor initial and final density profiles]{The tidal effects
on Scl's stellar component, for the \smallperi{} orbit with the fiducial
halo and exponential stars with \(R_s=0.10\,\kpc\). \textbf{Top:} the
initial (left) and final (right) 2D projected density of stars on the
sky. The solid circle marks \(6R_h\), the dotted circle the break
radius, and the blue arrow the orbital direction. \textbf{Bottom:} The
initial (dotted) and final (solid) stellar density profiles as compared
to the observed stellar density profile. Arrows mark the half-light
(\(R_h\)), break, and Jacobi radii
(Eqs.~\ref{eq:r_break}, \ref{eq:r_jacobi})
.}\label{fig:scl_smallperi_i_f}
\end{figure}

\begin{figure}
\centering
\pandocbounded{\includegraphics[keepaspectratio]{figures/scl_plummer_i_f.pdf}}
\caption[Sculptor Plummer initial and final density profiles]{Similar to
Fig.~\ref{fig:scl_smallperi_i_f} except for Plummer initial stars with
\(R_h = 0.20\,\kpc\). While a faint stream may be visible with deeper
observations, effects on the stellar profile are
minimal.}\label{fig:scl_smallperi_plummer_i_f}
\end{figure}

\begin{table*}[t]
\centering
\caption[Simulation results for Sculptor’s stars]{The present-day stellar properties for the simulations of Sculptor. In each row, we have the initial stellar velocity dispersion (within 1kpc), the final velocity dispersion, the fraction of stellar mass unbound, the initial half-light radius, the final half-light radius, and the break radius in arcmin and kpc (Eq. \ref{eq:r_break}). }
\label{tbl:scl_sim_stars_results}
\begin{tabular}{lll}
\toprule
Property & Exponential & Plummer\\
\midrule
$\sigma_{\V, i}\,/\,\kms$ & 9.8 & 10.7\\
$\sigma_{\V, f} \,/\,\kms$ & 8.8 & 9.4\\
fractional stellar mass loss & $2.1\times 10^{-6}$ & $0.024$\\
$R_{h, i}\,/\,\kpc$ & 0.169 & 0.202\\
$R_{h, f}\,/\,\kpc$ & 0.189 & 0.227\\
break radius / arcmin & $98$ & $105$\\
break radius / kpc & 2.3 & 2.5\\
\bottomrule
\end{tabular}
\end{table*}

\subsection{Orbital effects of the LMC}\label{sec:scl_lmc}

The Milky Way isn't the only galaxy in town. Recently, work has shown
that the infall of the LMC may substantially affect the Milky Way system
\citep[e.g.,][]{erkal+2019, cautun+2019, garavito-camargo+2021, vasiliev2023}.
With a mass up to one fifth of the MW \citep[e.g.,][]{penarrubia+2015},
the LMC infall affects conclusions about the MW properties and the
orbits of satellites \citep[see
e.g.,][]{patel+2020, battaglia+2022, correamagnus+vasiliev2022}. In this
section, we examine how the LMC may affect the orbital history of
Sculptor.

We use the \texttt{L3M11} model of the MW and LMC potential from
\citet{vasiliev2024}. The \texttt{L3M11} potential is an evolving
multipole approximation of an N-body simulation including a live MW and
LMC dark matter halo. The potential includes a static MW bulge and disk,
evolving MW and LMC halos, and the MW reflex motion. In their
simulation, the MW was initially a NFW halo with \(r_s=16.5\,\)kpc and
\(M_{\rm 200}= 98.4\times10^{10}\Mo\) , and the LMC a NFW halo with
\(r_s=11.7\) and \(M_{200} = 24.6 \times 10^{10} \Mo\). The total
\texttt{L3M11} MW mass is lighter than our initial \citet{EP2020}
potential.

The inclusion of the LMC reshapes Scl's orbital history, as shown in
Fig.~\ref{fig:scl_lmc_orbits_effect}. In the MW-only potential, Scl's
orbit is typical of a long-term MW satellite. However, Scl's closest
approach to the LMC \(~\sim0.1\,\Gyr\) ago affects the long-term
orbit---Scl is inferred to reach an apocentre of nearly \(300\,\kpc\).
Scl may even be on first infall, depending on the MW and LMC mass. Scl's
is orbiting the Milky Way on a similar plane to the LMC, but in the
opposite direction---Scl is unlikely to be a LMC satellite.

Promisingly, the timing of the LMC encounter implies a break radius
(\(\sim 25'\), from Table~\ref{tbl:scl_lmc_sim_stars}) consistent with
the beginning of Scl's observed density excess (see
Fig.~\ref{fig:classical_dwarfs_densities}, and
Section~\ref{sec:data_density_profiles}). To probe this further, we
select an orbit with the smallest LMC-Scl pericentre (20,kpc) in the
\texttt{L3M11} model, consistent with Scl's present-day positions and
velocities. The orbit is selected following the procedure in
Section~\ref{sec:orbital_estimation} (with uncertainties doubled). This
\texttt{LMC-flyby} orbit, is integrated back in time \(2\,\Gyr\) ago to
isolate recent tidal effects. We modify Scl's initial halo to have
\(\rmax = 2.5\,\kpc\) and \(\vmax = 25\,\kms\), slightly reducing the
initial stellar velocity dispersion.
Fig.~\ref{fig:scl_lmc_orbits_effect} shows this selected orbit in black
and Table~\ref{tbl:orbit_ics} records the initial conditions.

\begin{table*}[t]
\centering
\caption[Orbits and results for Scl in the LMC potential.]{The orbital properties and dark matter evolution for the models including an LMC. Similar to Table \ref{tbl:scl_sim_results} except quantities with respect to the LMC are in parentheses. }
\label{tbl:scl_lmc_sims}
\begin{tabular}{lll}
\toprule
Property & random samples & \texttt{LMC-flyby}\\
\midrule
pericentre & $44\pm 3$ ($29 \pm 2$) & 39 (20)\\
apocentre & $218 \pm 8$ & –\\
time of last pericentre / Gyr & $-0.38\pm0.01$ (-0.11) & -0.33 (-0.10)\\
number of pericentres & 1 (2) & 1 (1)\\
Jacobi radius / kpc & $4.1\pm0.3$ ($4.5\pm0.2$) & 2.8 (2.6)\\
Jacobi radius / arcmin & $168 \pm 11$ ($186\pm6$) & 132 (121)\\
final heliocentric distance / kpc & $83.2\pm2$ & 72.9\\
$\V_\textrm{max, f} / \V_\textrm{max, i}$ &  & 0.928\\
$r_\textrm{max, f} / r_\textrm{max, i}$ &  & 0.763\\
fractional final bound mass &  & 0.5402\\
\bottomrule
\end{tabular}
\end{table*}

\begin{figure}
\centering
\includegraphics[width=1\linewidth,height=\textheight,keepaspectratio]{figures/scl_lmc_xyzr_orbits.png}
\caption[Sculptor orbits with LMC]{Similar to Fig.~\ref{fig:scl_orbits}
except for orbits with (orange) and without (green lines) the inclusion
of an LMC (blue line) in the potential. The bottom row additionally
shows the distance between Scl and the LMC over
time.}\label{fig:scl_lmc_orbits_effect}
\end{figure}

\subsection{Tidal effects from the
LMC}\label{tidal-effects-from-the-lmc}

Perhaps surprisingly, the combined tidal effect of the MW and LMC are
weaker for Scl than in the MW-only case.
Fig.~\ref{fig:scl_lmc_sim_images} shows the dark matter evolution of Scl
and the passage of the LMC. With only one MW pericentre, Scl's dark
matter is less disrupted than the previous MW-only model. The subsequent
LMC passage modifies Scl's orbit but has otherwise little effect. As a
result, the dark matter structure evolves mildly and \(\sim 50\%\) of
mass remains bound (Table~\ref{tbl:scl_lmc_sims}).

JFN: This is mainly because of the altered orbit of Sculptor, which is
now inferred to come from a much larger apocentre, and, thus, to have
completed fewer orbits than in the previous MW-only ``smallperi'' orbit.
This reduces the number of pericentric passages, and therefore the net
tidal effect.

Correspondingly, the stellar component is nearly unchanged by the
combined MW and LMC tides. Fig.~\ref{fig:scl_lmc_i_f} shows the
projected stellar distributions and density profiles of this model.
While the break radius is within the observed density profile, tidal
effects are too weak to be detectable. Structural properties of the
stars similarly evolve little (Table~\ref{tbl:scl_lmc_sim_stars}).

While the instantaneous tidal force from the LMC is larger than the MW,
Scl does not experience the LMC tidal field long enough to display
disturbances. Scl and the LMC have a similar orbital plane, but orbit in
opposite directions (Fig.~\ref{fig:scl_lmc_orbits_effect}). As a result,
the interaction between Scl and the LMC is brief as their relative
velocity is \(\sim 400\,\kms\) at closest approach. Furthermore, the
Jacobi radius due to the LMC still falls outside the observed density
profile (Fig.~\ref{fig:scl_lmc_i_f}) and the MW Jacobi radius is even
larger. As a result, tides in a MW and LMC potential are even weaker
overall than for the MW-only orbit.

\begin{figure}
\centering
\pandocbounded{\includegraphics[keepaspectratio]{figures/scl_lmc_sim_images.png}}
\caption[Sculptor simulation snapshots with LMC]{Similar to
Fig.~\ref{fig:scl_sim_images} except for the case where the potential
includes an LMC. The current position and path of the LMC are
represented by the green dot and line respectively. We also plot the
full orbit (over the past 2Gyr) for both Scl and the LMC, as less than
one radial period happens over this time
frame.}\label{fig:scl_lmc_sim_images}
\end{figure}

\begin{figure}
\centering
\pandocbounded{\includegraphics[keepaspectratio]{figures/scl_lmc_i_f.pdf}}
\caption[Sculptor initial and final density with LMC]{Similar to
Fig.~\ref{fig:scl_smallperi_i_f} except for the \texttt{LMC-flyby}
model. The Jacobi and break radii here are calculated with respect to
the LMC, the corresponding radii with respect to the MW are larger. With
only one MW pericentre and a recent, rapid LMC encounter, tidal forces
do not appear to affect the stellar
distribution.}\label{fig:scl_lmc_i_f}
\end{figure}

\begin{table*}[t]
\centering
\caption[Simulation results for Sculptor’s stars in the LMC+MW potential]{Similar to Table \ref{tbl:scl_sim_stars_results}, but for the properties of the stellar components of the \texttt{LMC-flyby} model of Sculptor. }
\label{tbl:scl_lmc_sim_stars}
\begin{tabular}{lll}
\toprule
Property & Scl: LMC-exponential & LMC-Plummer\\
\midrule
$\sigma_{\V, i}\,/\,\kms$ & 9.0 & 9.4\\
$\sigma_{\V, f} \,/\,\kms$ & 8.8 & 9.2\\
fractional stellar mass loss & $<10^{-12}$ & 0.0013\\
$R_{h, i}$ / kpc & 0.186 & 0.201\\
$R_{h, f}$ / kpc & 0.189 & 0.205\\
break radius & $78'$, 1.6 kpc & $81'$, 1.7 kpc\\
LMC break radius & $23'$, 0.49 kpc & $24'$, 0.52 kpc\\
\bottomrule
\end{tabular}
\end{table*}

\subsection{Summary}\label{summary}

We find, including only the MW potential, that tides only remove dark
matter from the outskirts of Scl. The central cusp and compact stellar
distribution are resilient to tides and any tidal effects are
well-outside the reach of current observations. We have also found that
the LMC strongly perturbs Scl's orbit---in this case, Scl may be on
first infall. However, with only 1 pericentre each for the LMC and MW,
the combined tides are weaker than for our initial model. In either
case, we conclude that tides are unlikely to affect Sculptor's stellar
component.

\section{Tidal effects on Ursa Minor}\label{tidal-effects-on-ursa-minor}

\subsection{Evolution of Ursa Minor's dark matter
halo}\label{evolution-of-ursa-minors-dark-matter-halo}

\begin{figure}
\centering
\pandocbounded{\includegraphics[keepaspectratio]{figures/umi_sim_images.png}}
\caption[Ursa Minor simulation snapshots]{Similar to
Fig.~\ref{fig:scl_sim_images} except for Ursa Minor on the \smallperi{}
orbit. Dark matter evolution is more dramatic than for
Scl.}\label{fig:umi_sim_images}
\end{figure}

\begin{table*}[t]
\centering
\caption[Simulation results for Ursa Minor’s dark matter]{The present-day properties for Ursa Minor’s final dark matter halo. See Table \ref{tbl:scl_sim_results} for details. }
\label{tbl:umi_sim_results}
\begin{tabular}{lll}
\toprule
Property & Random orbits & \smallperi{}\\
\midrule
pericentre & $37\pm3$ & 30\\
apocentre & $83 \pm 4$ & 75\\
time of last pericentre & $-0.97 \pm 0.07$ & -0.80\\
number of pericentres & $\sim 8$ & 8\\
Jacobi radius / kpc & $3.7 \pm 0.3$ & 2.9\\
Jacobi radius / arcmin & $184 \pm 12$ & 156\\
final heliocentric distance & $70.1 \pm 3.6$ & 64.7\\
${\vmax}_f / {\vmax}_i$ &  & 0.511\\
${\rmax}_f / {\rmax}_i$ &  & 0.249\\
fractional dm final mass &  & 0.035\\
\bottomrule
\end{tabular}
\end{table*}

\begin{figure}
\centering
\pandocbounded{\includegraphics[keepaspectratio]{figures/umi_tidal_track.png}}
\caption[Ursa Minor tidal tracks]{Similar to
Fig.~\ref{fig:scl_tidal_track} except for Ursa Minor. Ursa Minor looses
substantially more mass than Sculptor. \textbf{todo: thinner and
truncated tidal track, thinner stellar
component}}\label{fig:umi_tidal_track}
\end{figure}

The tidal evolution of Ursa Minor is similar to Sculptor in the MW-only
potential. Fig.~\ref{fig:umi_sim_images} shows snapshots of the DM
evolution. UMi loses significantly more DM mass than Scl, forming
substantial dark matter streams encircling the MW several times.

UMi only retains 3\% of its total mass after 9 Gyr
(Table~\ref{tbl:umi_sim_results}). As a result, the final dark matter
component is much smaller than the initial, but still evolves along the
predicted tidal track (Fig.~\ref{fig:umi_tidal_track}). Despite the more
substantial tidal evolution, the Jacobi radius is still large, lying at
\(\sim 4\,\kpc\), well beyond the final \(\rmax\).

Because of UMi's mass loss, the orbit deviates substantially from a
point orbit. Through our orbit-adjustment procedure in
Section~\ref{sec:orbit_corrections}, we recover nearly exactly the
present-day position of Ursa Minor by changing the initial positions by
\(20\,\kpc\) and \(\sim 9\,\kms\). These adjustments do not
significantly affect the qualitative structure or pericentre of the
orbit.

\subsection{Evolution of Ursa Minor's
stars}\label{evolution-of-ursa-minors-stars}

\begin{table*}[t]
\centering
\caption[Simulation results for Ursa Minor’s stars]{Similar to Table \ref{tbl:scl_sim_stars_results}, the present-day stellar properties for the simulation of Ursa Minor for exponential and Plummer stars. }
\label{tbl:umi_sim_stars_results}
\begin{tabular}{lll}
\toprule
Property & smallperi-exp & smallperi-Plummer\\
\midrule
$\sigma_{\V, i}\,/\,\kms$ & 10.0 & 10.9\\
$\sigma_{\V, f}\,/\,\kms$ & 8.2 & 8.5\\
fractional stellar mass loss & $0.00015$ & 0.039\\
$R_{h, i}\,/\,\kpc$ & 0.135 & 0.151\\
$R_{h, f}\,/\,\kpc$ & 0.169 & 0.191\\
break radius & 197 arcmin, 3.7 kpc & 204 arcmin, 3.8 kpc\\
\bottomrule
\end{tabular}
\end{table*}

\begin{figure}
\centering
\pandocbounded{\includegraphics[keepaspectratio]{figures/umi_smallperi_i_f.pdf}}
\caption[Ursa Minor simulated density profiles]{The tidal effects on the
stellar surface density of Ursa Minor for exponential stars on the
\smallperi{} orbit.}\label{fig:umi_smallperi_i_f}
\end{figure}

\begin{figure}
\centering
\pandocbounded{\includegraphics[keepaspectratio]{figures/umi_plummer_i_f.pdf}}
\caption[Ursa Minor Plummer model density]{The tidal effects on the
stellar surface density of Ursa Minor for Plummer stars on the
\smallperi{} orbit.}\label{fig:umi_plummer_i_f}
\end{figure}

Tidal features in UMi's stellar component are still extremely faint,
becoming apparent only outside 100 arcminutes in
Fig.~\ref{fig:umi_smallperi_i_f}. The observed size and velocity
dispersion of Ursa Minor evolve little
(Table~\ref{tbl:umi_sim_stars_results}). For exponential initial
conditions, tidal effects are unlikely to ever be observable in the near
future.

The break and Jacobi radii fall well outside the observed stellar
profile. Tides would have to be far stronger to affect the observed
stellar component. As a result, the minimal tidal evolution of this
model is not unexpected.

As for Scl (Section~\ref{sec:scl_sim_stars}), we also consider a model
where UMi's stars are initially a more extended Plummer profile,
resembling the present-day density profile. The stellar evolution of
this Plummer stellar component is similar
(Fig.~\ref{fig:umi_plummer_i_f}). However, because there are more
loosely-bound stars, the Plummer model looses nearly 4\% of its initial
stellar mass to tides (Table~\ref{tbl:umi_sim_stars_results}), and tidal
features may be detectable if we measure densities 2 orders of magnitude
fainter than our present data. We show the properties of a stream in the
Appendix (Fig.~\ref{fig:umi_tidal_stream}), but such a stream is
unlikely to be observable in the near future. We conclude that tides do
not strongly affect the stellar component of this model.

\subsection{Effects of the LMC}\label{effects-of-the-lmc}

\begin{figure}
\centering
\pandocbounded{\includegraphics[keepaspectratio]{figures/umi_lmc_xyzr_orbits.png}}
\caption[Ursa Minor orbits with LMC]{Orbits of Ursa Minor with (orange)
and without (green) an LMC. The final positions of Ursa Minor and the
LMC are plotted as scatter points and the solid blue line represents the
LMC trajectory. Note that the LMC mostly increases Ursa Minor's
pericentres and apocentres.}\label{fig:umi_orbits_lmc}
\end{figure}

Fig.~\ref{fig:umi_orbits_lmc} shows the effects of including an LMC on
the orbit of Ursa Minor. Predominantly, the effect is to increase the
orbital period, apocentre and pericentre. Yet, the orbit remains in a
similar plane and with similar morphology. As UMi is on the opposite
side as the Galaxy of the LMC, and has a closest LMC approach of
\(\gtrsim 100\,\kpc\), this is not surprising.

The deviation from the MW-only orbit is mostly due to the LMC-induced
reflex motion of the Milky Way. Because the MW centre is accelerated
towards the LMC and away from UMi, UMi's orbit to increase in
characteristic radius.

\subsection{Summary}\label{summary-1}

While tides affect UMi more strongly than Scl, the tidal effects are
insufficient to reshape the observed stellar density profile. Faint
tidal tails may be observable with deeper data. Finally, including the
LMC in the potential further weakens the tides experienced by UMi.

\section{Modelling uncertainties}\label{modelling-uncertainties}

\subsection{Modelling assumptions}\label{modelling-assumptions}

As the above results show, tides only marginally affected the stellar
components of Scl and UMi, even orbit chosen to have the smallest
observationally-consistent pericentres. While we have only presented
select models, alternative initial conditions do not affect our
qualitative conclusions on tidal effects expected for Scl and UMi in the
Galactic (and LMC) potential.

Although our analysis neglects baryonic physics, Scl and UMi have
predominantly stars older than \(\sim 9\) Gyr
\citep{carrera+2002, deboer+2011, weisz+2014, delosreyes+2022, sato+2025}.
So, gas dynamics are unlikely to affect recent evolution. A
collisionless dark-matter only simulation should therefore be an
excellent approximation.

Cored or less concentrated dark matter halos disrupt quicker
\citep[e.g.,][]{stucker+2023}. Our fiducial UMi halo in particular is
among the least concentrated halos consistent with UMi's velocity
dispersion. Although Scl's fiducial halo is more concentrated, less
concentrated and cored halos evolve similarly (see Appendix
\ref{sec:extra_results}).

Galaxies are rarely perfect isotropic spheres. Sculptor and Ursa Minor
are elliptical, and halos are expected to be radially anisotropic
\citep[e.g.,][]{navarro+2010}. We test non-spherical and anisotropic
models in Appendix \ref{sec:extra_results}, finding that these
assumptions likely do not alter our conclusions.

Alternative initial conditions may influence the total mass evolution
but should produce a similar final stellar structure. A system's
observed velocity dispersion directly constrains the mean density within
\(R_h\) \citep[e.g.,][]{wolf+2010}. Thus, the tidal force required to
disrupt the stellar component does not strongly depend on the inner halo
structure (via the Jacobi radius).

\subsection{Orbital uncertainties}\label{sec:scl_umi_orbit_uncert}

The long-term orbits of satellites are uncertain. Analytic Milky Way
potentials neglect many unknown details, including triaxiality, halo
twisting, mass evolution, and substructure. Due to these inadequacies,
calculated orbits may diverge significantly from the true orbits of
satellites \citep[e.g.,][]{dsouza+bell2022}. The mass-growth of the
Milky Way and dynamical friction imply that orbits are typically less
bound in the past (less affected by tides). Orbital energy and angular
momentum of subhalos are not conserved in cosmological N-body
simulations. Consequently, orbits in analytic potentials may
overestimate the pericentre and underestimate the maximum tidal stress
\citep[although typically not by enough to change our
conclusions,][]{santistevan+2023, santistevan+2024}.

As an example, Fig.~\ref{fig:scl_orbit_lmc_uncert} illustrates how
changes to the LMC potential modifies the long-term orbital trajectories
of Scl and UMi. More than 4 Gyr ago, orbits of Scl diverge
substantially. Some orbits are near apocentres of \(\sim 300\,\kpc\)
when others approach pericentres as small as \(\sim 10\,\kpc\). Ursa
Minor's orbit is more stable until the possible previous LMC pericentre.
In some cases, Ursa Minor may have been bound to the LMC.

Motivated by Fig.~\ref{fig:scl_orbit_lmc_uncert}, we examine an extreme
pericentre of \(4\,\kpc\) of Scl with the MW in Appendix
\ref{sec:extra_results}, finding it still insufficient to produce the
observed density profile. Regardless, we conclude our simulated orbits
represent reasonable extremes for \emph{recent} tidal effects. Past
encounters with the LMC are revisited below as a form of
``pre-processing'' in Section~\ref{sec:stellar_halos}.

\begin{figure}
\centering
\pandocbounded{\includegraphics[keepaspectratio]{/Users/daniel/thesis/figures/scl_lmc_orbits_mass_effect.png}}
\caption[Sculptor Orbits with LMC]{The long-term orbital history of
Sculptor (\textbf{top}) and UMi (\textbf{bottom}) are uncertain. In both
panels, light, transparent lines represent randomly-sampled orbit of the
satellites (alla ref) in three different LMC/MW mass models from
\citet{vasiliev2024}. The LMC orbits are in solid, thick lines of the
corresponding colour. The L2M11 has a lighter LMC mass, and the L3M10
model has a lighter MW mass than our fiducial L3M11 LMC
model.}\label{fig:scl_orbit_lmc_uncert}
\end{figure}

\subsection{Summary}\label{summary-2}

While the long-term tidal evolution is unconstrained, we conclude that
our models are reasonable representations of recent tidal effects. As a
result, recent tides are unlikely to affect the stellar distributions of
Sculptor and Ursa Minor.
