As discussed in Chapters \ref{sec:introduction}, \ref{sec:data},
Sculptor and Ursa Minor have unusually extended stellar light profiles.
Using our simulation setup introduced in Chapter \ref{sec:methods}, we
present our simulation results for each galaxy in this Chapter. We find
that tides may drive significant evolution of each galaxies total mass.
However, the compact stellar distribution for each galaxy is nearly
unaffected. We also show that the LMC strongly perturbs Scl's orbit.
But, the tidal effects when considering an LMC are yet weaker than in
the Milky Way only case. We conclude that tides are unlikely to affect
the stellar component of either galaxy.

\subsection{Tidal evolution in the Milky
Way}\label{tidal-evolution-in-the-milky-way}

The Milky Way tides drive significant mass loss for Sculptor.
Fig.~\ref{fig:scl_sim_images} shows projected dark matter distribution
of our N-body simulation for Sculptor in the \(y\)-\(z\) plane for 5
previous pericentres. This simulation uses the fiducial initial halo on
the \smallperi{} orbit. The dark matter forms large streams which orbit
the Milky Way several times. Note that, in contrast to the outer
regions, the central dark matter cusp is similar in appearance across
all simulation snapshots.

Fig.~\ref{fig:scl_tidal_track} shows the initial and final circular
velocity profile of the dark matter, and the evolution of the maximum
circular velocity. Sculptor's final dark matter halo evolves from
\(\vmax=31\,\kms\) at \(\rmax=3.2\,\kms\) to \(\vmax = 22\,\kms\) at
\(\rmax = 2\,\kpc\). In terms of bound mass, the mass evolves from
\(0.40 \times 10^{10}\,\Mo\) to \(0.026\times10^{10}\,\Mo\). The
circular velocity maximum evolves along the tidal track predicted by
\citet{EN2021}.

\begin{figure}
\centering
\pandocbounded{\includegraphics[keepaspectratio]{figures/scl_sim_images.png}}
\caption[Sculptor simulation snapshots]{Images of the dark matter
evolution over a selection of past apocentres and the present day
position. Limits range from -150 to 150 kpc in the \(y\)-\(z\)
(approximately orbital) plane and the colourscale is logarithmic
spanning 5 orders of magnitude between the maximum and minimum values.
The green dot represents the final (observed) position of the galaxy and
the grey curve represents the orbit over one radial
oscillation.}\label{fig:scl_sim_images}
\end{figure}

\begin{figure}
\centering
\pandocbounded{\includegraphics[keepaspectratio]{figures/scl_tidal_track.png}}
\caption[Sculptor Tidal Tracks]{The tidal tracks for the smallperi
orbit. Todo: add velocity dispersion plot to RHS, label initial and
final with times}\label{fig:scl_tidal_track}
\end{figure}

\subsection{}\label{section}

While mass loss leads to adiabatic expansion of the stellar component,
no features of tidal disruption or disequilibrium are currently
observable. Fig.~\ref{fig:scl_smallperi_i_f} shows the final projected
density of stars on the sky, and the initial and final radially averaged
density profiles. No non-spherical density features are apparent, even
at 5 decades fainter in surface density than the central density. In
addition, the initial and final density profiles look identical up to
some scale, with a tidally-induced excess of stars only appearing
\(\sim 3\) orders of magnitude fainter than the faintest our density
profile measures.

The analytically motivated break and tidal radii also support a weak
tidal effect on stars. In both cases, these radii work out to be
\(\gtrsim 100\) arcminutes, outside where we can measure the density
profile. In particular, because we have chosen the most extreme
observationally permissible orbit, it is unlikely, for similar
\(\Lambda\)CDM motivated initial conditions, that these radii would ever
approach where the break in Sculptor's density profile appears.

\begin{figure}
\centering
\pandocbounded{\includegraphics[keepaspectratio]{figures/scl_smallperi_i_f.pdf}}
\caption[Sculptor initial and final density profiles]{The tidal effects
on Scl's stellar component, for the \smallperi{} orbit with the fiducial
halo and exponential stars with \(R_s=0.10\,\kpc\). \textbf{Left:} the
final 2D projected density of stars on the sky, with colours
representing logarithmically increasing density. The circle marks the
break radius. \textbf{Right:} The initial (dotted) and final (solid)
stellar density profiles as compared to the observed stellar density
profile. Arrows mark the break and Jacobi (Eq.~\ref{eq:jacobi_radius})
radii. \textbf{add initial sky-projected density, t=xxx in legend,
sculptor in figure title}}\label{fig:scl_smallperi_i_f}
\end{figure}

\begin{figure}
\centering
\pandocbounded{\includegraphics[keepaspectratio]{figures/scl_plummer_i_f.pdf}}
\caption[Sculptor Plummer initial and final density profiles]{Similar to
Fig.~\ref{fig:scl_smallperi_i_f} except for Plummer initial stars with
\(R_h = 0.20\,\kpc\). While a faint stream may be visible with deeper
observations, effects on the stellar profile are minimal. \emph{could
overlay orbit on left}}\label{fig:scl_smallperi_plummer_i_f}
\end{figure}

\subsection{Effects of the LMC}\label{sec:scl_lmc}

The Milky Way isn't the only galaxy in town. Recently, work has shown
that the infall of the LMC may substantially affects the Milky Way
system
\citep[e.g.,][]{erkal+2019, cautun+2019, garavito-camargo+2021, vasiliev2023}.
The mass of the LMC may be as high as 1/5 of the Milky Way mass.
Including an LMC in the Milky Way potential may change conclusions about
properties and orbits of the Milky Way satellite system, including Scl
\citep[e.g.,][]{patel+2020, battaglia+2022}.

\subsubsection{Orbital effects}\label{orbital-effects}

To explore the effects of the LMC on Sculptor, we use the publicly
available L3M11 model from \citet{vasiliev2024}. While
\citet{vasiliev2024} present other models, the L3M11 model has the
highest MW and LMC masses, and the (recent) orbital history appears to
be independent of the potential details. This model stars with a lighter
Milky Way than our fiducial, \citet{mcmillan2011-like} model. The L3M11
potential is an evolving multipole approximation of an N-body simulation
with both a live MW and LMC dark matter halo. The potential additionally
includes the reflex motion of the MW due to the LMC. The MW was
initialized as an NFW dark matter halo with \(r_s=16.5\,\)kpc and mass
\(M_{\rm 100}= 11\times10^{11}\Mo\), and the LMC was a NFW halo with
\(r_s=11.7\) and \(M_{100} = 2.76 \times 10^{11} \Mo\). This model had a
previous LMC pericentre at about 6 Gyr ago.

For our orbit, we select the orbit with the \(3\sigma\)-smallest LMC
pericentre. Results are similar when selecting for a small MW pericentre
instead. We also use, instead, the \texttt{small} halo for Sculptor, as
the tidal effects do not reduce the velocity dispersion, unlike for the
MW-only model.

Fig.~\ref{fig:scl_lmc_orbits_effect} displays the effect of including an
LMC in the potential. The green samples are in the initial MW-only
potential in the \texttt{L3M11} model, and the orange samples are
integrated in the evolving MW and LMC \texttt{L3M11} model. The past 1
Gyr is similar in both cases, but the orbits diverge significantly
afterwards. The recent passage of Sculptor with the LMC around 0.1 Gyr
ago allows for Sculptor to begin as far as 300 kpc from the Milky Way
centre. The evolving potential also adds significant long term
variability in the possible orbits of Sculptor. Sculptor, however, is
orbiting in the opposite direction of the LMC so is likely not
associated with the LMC system.

Given the large uncertainties of the LMC model, we conservatively double
all of the observational parameters of Sculptor. This has a similar
effect to including LMC mass and orbital uncertainties but is
considerably simpler. From this larger range of orbits, we once again
select the orbit with the median final observables of all orbits with
pericentres less than the 0.0027th quantile. This orbit, the
\texttt{smallperilmc} orbit is plotted in black and is only integrated
up to 2 Gyr ago, to isolate recent tidal effects.

\begin{figure}
\centering
\pandocbounded{\includegraphics[keepaspectratio]{figures/scl_lmc_xyzr_orbits.png}}
\caption[Sculptor Orbits with LMC]{This figure is similar to
Fig.~\ref{fig:scl_orbits} except that we are showing the orbits with and
without an LMC. In the bottom row, the distance from Sculptor (or the
LMC) to the MW is plotted (left), and the Sculptor - LMC distance
(right.) \textbf{cut scale on buttom
panel}}\label{fig:scl_lmc_orbits_effect}
\end{figure}

\begin{figure}
\centering
\pandocbounded{\includegraphics[keepaspectratio]{figures/scl_lmc_orbits_mass_effect.png}}
\caption[Sculptor Orbits with LMC]{The long-term orbital history of
Sculptor is uncertain. In light, transparant lines, the orbits of
Sculptor in three different LMC/MW mass models from \citet{vasiliev2024}
are shown for random samples of the observed properties. The LMC orbits
are in solid, thick lines of the corresponding colour. The L2M11 has a
lighter LMC mass, and the L3M10 model has a lighter MW mass than our
fiducial L3M11 LMC model.}\label{fig:scl_lmc_orbits_mass}
\end{figure}

\begin{table*}[t]
\centering
\caption[Sculptor Selected Orbits]{Properties of selected orbits for Sculptor. The mean orbit represents the observational mean from Table \ref{tbl:scl_obs_props}. The \smallperi{} represents instead the $3\sigma$ smallest pericentre, which we use to provide an upper limit on tidal effects. }
\label{tbl:scl_orbits}
\begin{tabular}{ll}
\toprule
Property & smallperi-LMC\\
\midrule
distance / kpc & 73.1\\
$\pmra / \masyr$ & 0.137\\
$\pmdec / \masyr$ & –0.156\\
LOS velocity / $\kms$ & 111.2\\
$t_i / \Gyr$ & -2\\
${x}_{i} / \kpc$ & 4.30\\
${y}_{i} / \kpc$ & 138.89\\
${z}_{i} / \kpc$ & 125.88\\
$\V_{x\,i} / \kms$ & 6.89\\
$\V_{y\,i} / \kms$ & -56.83\\
$\V_{z\,i} / \kms$ & 52.09\\
pericentre / kpc & 38.82\\
peri-LMC / kpc & ?\\
apocentre / kpc & 187.50\\
$t_{\rm last\ peri} / {\rm Gyr}$ & -0.46\\
$t_{\rm last\ peri-LMC} / \Gyr$ & -0.1?\\
Number of pericentres & 1-2\\
\bottomrule
\end{tabular}
\end{table*}

\subsubsection{Tidal effects}\label{tidal-effects}

Unlike the previous mdoel, this model only has one pericentre with the
Milky Way. Fig.~\ref{fig:scl_lmc_sim_images} shows snapshots of Sculptor
over the past 2 Gyr while marking the position of the LMC. With only one
pericentre, and a larger one than the \texttt{smallperi} orbit,
Sculptor's dark matter is substantially less disrupted. And while, based
on the tidal tensor values, the LMC induces a greater instantaneous
tidal effect than the Milky Way, Sculptor's dark matter component does
not show strong effects due to the LMC.

Finally, Fig.~\ref{fig:scl_lmc_i_f} shows the projected on-sky final
stellar distribution and the initial and final stellar density profile
for this model with exponential stars. The stellar component does not
change at all. While the break radius set by the time since the last LMC
pericentre would agree with the location of the break in the observed
density profile, no stellar effect would be observable.

The LMC flyby encounter is an approximately impulsive encounter, in
contrast with more adiabadic mass loss due to the Milky Way. Impulsive
encounters tend to inject energy into the stellar and dark matter
distribution, and can initially cause the galaxy to contract. In
addition, the tidal force is required to be far larger than for slower,
adiabadic mass loss, because the galaxy experiences this tidal field for
far less time. Even in models where Sculptor passes through the the LMC,
tidal tails do not form immediately after this encounter.

\begin{figure}
\centering
\pandocbounded{\includegraphics[keepaspectratio]{figures/scl_lmc_sim_images.png}}
\caption{Sculptor Simulation Snapshots with
LMC}\label{fig:scl_lmc_sim_images}
\end{figure}

\begin{figure}
\centering
\pandocbounded{\includegraphics[keepaspectratio]{figures/scl_lmc_i_f.pdf}}
\caption[Sculptor initial and final density with LMC]{The tidal effects
on the stellar surface density due to the LMC
today.}\label{fig:scl_lmc_i_f}
\end{figure}

\subsubsection{A possible second MW
pericentre}\label{a-possible-second-mw-pericentre}

A final note is that this model only considers the recent tidal effects
.As illustrated in the range of possible orbits in
Fig.~\ref{fig:scl_lmc_orbits_effect}, there is a chance that Sculptor
experienced an extremely small pericentre with the Milky Way. This
pericentre has the potential to substantially rearrange the stellar
component, drive large tidal mass loss, and create a Plummer-like
stellar density profile. However, this encounter is highly dependent on
the choice of the MW-LMC potential model, and may not occur at all. Long
time integration in dynamic potentials amplifies uncertainties in the
inputs and may not be reliable.

To illustrate the possible effects of this previous pericentre, we
create a model with a goal of matching the \(3\sigma\) smallest
pericentre with the MW. Because of strong tidal shocking, the final
conditions depend strongly on the initial conditions. In addition,
actions are not conserved in this evolving potential, so we use the
\citet{vasiliev2024} method of iteratively updating the initial
conditions by empirically estimating the Jacobian. Our final position
reaches adequate, but not perfect, agreement with the intended final
position.

As a result of this proceedure, while the point particle reaches a
pericentre of \(\sim 4\,\kpc\), the adjusted N-body orbit instead only
reaches a pericentre of \(\sim 12\,\kpc\).

\section{Ursa Minor}\label{ursa-minor}

\begin{table*}[t]
\centering
\caption{Given the observed break radii, the required time of last pericentre and pericentre to produce the observed tidal effect.}
\begin{tabular}{llll}
\toprule
parameter & Scl & UMi & \\
\midrule
$r_{\rm break}$ observed & $25 \pm 5$ arcmin & $30 \pm 5$ arcmin & (sestito+2024a?); (sestito+2024b?)\\
$t_{\rm last\ peri}$ required & $110\pm30$ Myr & $120\pm30$ Myr & Eq. \ref{eq:r_break}\\
$r_{\rm peri}$ required & 16kpc & 14kpc & Eq. \ref{eq:r_jacobi}\\
\bottomrule
\end{tabular}
\end{table*}

\begin{figure}
\centering
\pandocbounded{\includegraphics[keepaspectratio]{figures/umi_sim_images.png}}
\caption[Ursa Minor simulation snapshots]{Ursa Minor simulation images.}
\end{figure}

Figure: Velocity dispersion evolution of Ursa Minor

\begin{figure}
\centering
\pandocbounded{\includegraphics[keepaspectratio]{figures/umi_smallperi_i_f.pdf}}
\caption[Ursa Minor simulated density profiles]{The tidal effects on the
stellar surface density.}
\end{figure}

\subsection{Effects of the LMC}\label{effects-of-the-lmc}

\begin{figure}
\centering
\pandocbounded{\includegraphics[keepaspectratio]{figures/umi_lmc_xyzr_orbits.png}}
\caption[Ursa Minor orbits with LMC]{Orbits of Ursa Minor with (orange)
and without (green) an LMC. The final positions of Ursa Minor and the
LMC are plotted as scatter points and the solid blue line represents the
LMC trajectory. Note that the LMC only increases Ursa Minor's peri and
apo-centres, weakening any tidal effect. Interestingly, there is a
change that Ursa Minor may have once been bound to the LMC (diverging
orange lines at top left of middle panel.)}
\end{figure}
