As discussed in Chapters \ref{sec:introduction}, \ref{sec:observations},
we test whether Galactic tides can form the extended density profiles of
Scl and UMi. In this Chapter, we analyze tailored N-body simulations,
using the methods described in Chapter \ref{sec:methods}, to assess
tidal impacts on each galaxy. We find that tides drive dark matter loss
in both systems but leave their compact stellar components largely
unaffected. The Large Magellanic Cloud (LMC) perturbs the orbit of Scl,
and somewhat UMi, yet the resulting tidal effects are weaker than in a
Milky Way-only potential. Our simulations demonstrate that recent tides
are unlikely to alter the stellar structure of Scl or UMi.

In this Chapter, we consider Scl first, describing tidal effects from
the MW on dark matter and stellar components, and then consider the
effects from the LMC. We then analyze UMi in a similar manner,
considering in turn the dark matter evolution, stellar evolution, and
orbital effects of the LMC.

\section{Tidal effects on Sculptor}\label{tidal-effects-on-sculptor}

\subsection{Dark matter evolution}\label{dark-matter-evolution}

As a representation of the most extreme tidal history, we initially
investigate the \smallperi{} Scl model.

Scl experiences moderate tidal mass loss. Fig.~\ref{fig:scl_sim_images}
shows the stripping of dark matter and formation of diffuse streams
trailing and leading Sculptor's orbit. Because tidal stripping can be
described as removal of the lowest energy particles, most mass loss
occurs in the outer halo.

N-body models may deviate from a point-particle trajectory due to
dynamical self-friction \citep[e.g.,][]{miller+2020}. However, this
effect is slight for Scl, which ends near the observed position, without
adjusting the initial conditions (the green point in
Fig.~\ref{fig:scl_sim_images}).

The inner density cusp is tidally resilient.
Fig.~\ref{fig:scl_tidal_track} shows the initial and final circular
velocity profiles, and the evolution of the maximum circular velocity.
The maximum velocity drops from \(31\,\kms\) to \(22\,\kms\), evolving
along the tidal track from \citet{EN2021}. The final circular velocity
profile resembles the initial with an inner cusp, but has a sharper
outer truncation. Quantitatively, the halo loses \(>90\%\) its initial
mass, as listed in Table~\ref{tbl:scl_sim_results}. However, the inner
structure is not expected to be affected, as the Jacobi radius is over
3kpc, outside of the initial and final \(\rmax\). Thus, tides may remove
significant mass, but mostly from the outer halo.

\begin{figure}
\centering
\pandocbounded{\includegraphics[keepaspectratio]{figures/scl_sim_images.png}}
\caption[Sculptor simulation snapshots]{Images of the dark matter
evolution over a selection of past apocentres and the present day.
Limits range from -150 to 150 kpc in the \(y\)--\(z\) (approximately
orbital) plane and the colourscale is logarithmic spanning 5 orders of
magnitude between the maximum and minimum values. The green dot
represents the final expected position of the galaxy and the solid and
dotted grey curves represents the orbit over one previous or future
radial oscillation respectively.}\label{fig:scl_sim_images}
\end{figure}

\begin{figure}
\centering
\pandocbounded{\includegraphics[keepaspectratio]{figures/scl_tidal_track.png}}
\caption[Sculptor tidal tracks]{Dark matter evolution for the
\smallperi{} model of Sculptor. Blue solid and orange dashed lines show
the initial and final circular velocity profiles. The points represent
the evolution of the maximum circular velocity, and the dotted black
line shows the tidal track from \citet{EN2021}. To calculate the
velocity profiles, unbound particles are iteratively removed,
recalculating the potential at each step assuming spherical
symmetry.}\label{fig:scl_tidal_track}
\end{figure}

\begin{table*}[t]
\centering
\caption[Simulation results for Sculptor’s dark matter]{The orbital and dark matter properties for the simulation of Sculptor. The random samples column shows the distributions from point orbits, and the \smallperi{} column contains the results from the N-body simulation. }
\label{tbl:scl_sim_results}
\begin{tabular}{lll}
\toprule
Property & random samples & \smallperi{}\\
\midrule
pericentre & $53\pm3$ & 29\\
apocentre & $102\pm3$ & 94.4\\
time of last pericentre / Gyr & $-0.45 \pm 0.2$ & -0.47\\
number of pericentres & 6 & 6\\
Jacobi radius / kpc & $4.5 \pm 0.3$ & 3.5\\
Jacobi radius / arcmin & $186\pm12$ & 101\\
final heliocentric distance / kpc & $83.2\pm2$ & 81.6\\
$\V_\textrm{max, f} / \V_\textrm{max, i}$ &  & 0.695\\
$r_\textrm{max, f} / r_\textrm{max, i}$ &  & 0.406\\
fractional final bound mass &  & 0.0893\\
\bottomrule
\end{tabular}
\end{table*}

\subsection{Stellar evolution}\label{stellar-evolution}

Tides minimally affect the stellar component of Sculptor in the
\smallperi{} orbit. In Fig.~\ref{fig:scl_smallperi_i_f}, the projected
stellar distribution displays no prominent distortions, and the radial
density profile is nearly unchanged. Only at a surface density
\(\sim10^8\) times fainter than the centre may faint tidal features
emerge. The total stellar mass lost corresponds to \(\sim 10\) stars in
total (see Table~\ref{tbl:scl_sim_results})---a formidable challenge
with the best of observations.

More extended stellar profiles are more sensitive to tides. With Plummer
initial conditions, as shown in
Fig.~\ref{fig:scl_smallperi_plummer_i_f}, the model loses more stellar
mass and forms more luminous tidal tails. Observations reaching surface
densities \(\sim10\) times fainter than our data could reveal a stream
in this case. Nevertheless, the radial extent probed by our data remains
nearly unchanged by tidal evolution.

Table~\ref{tbl:scl_sim_stars_results} quantifies the evolution of
stellar properties. The stellar velocity dispersion decreases by only
\(\sim1\,\kms\) and the half-light radius expands by \(\sim 10\%\). This
is consistent with adiabatic expansion due to the reduction of the total
mass \citep[e.g.,][]{stucker+2023}. In addition, the break and Jacobi
radii are \(\gtrsim 100\) arcminutes on the sky---tidal signatures would
be beyond the reach of our data. Altogether, Galactic tides negligibly
impact Scl's stellar component.

\begin{figure}
\centering
\pandocbounded{\includegraphics[keepaspectratio]{figures/scl_smallperi_i_f.pdf}}
\caption[Sculptor initial and final density profiles]{The tidal effects
on Scl's stellar component, for the \smallperi{} orbit with the fiducial
halo and exponential stars with \(R_s=0.10\,\kpc\). \textbf{Top:} the
initial (left) and final (right) 2D projected density of stars on the
sky. The solid circle marks \(6R_h\), the dotted circle the break
radius, and the (dotted) line the past (future) orbit. \textbf{Bottom:}
The initial (dotted) and final (solid) stellar density profiles as
compared to the observed stellar density profile. Arrows mark the break
and Jacobi radii (Eqs.~\ref{eq:r_break}, \ref{eq:r_jacobi})
.}\label{fig:scl_smallperi_i_f}
\end{figure}

\begin{figure}
\centering
\pandocbounded{\includegraphics[keepaspectratio]{figures/scl_plummer_i_f.pdf}}
\caption[Sculptor Plummer initial and final density profiles]{Similar to
Fig.~\ref{fig:scl_smallperi_i_f} except for Plummer initial stars with
\(R_h = 0.20\,\kpc\). While a faint stream may be visible with deeper
observations, effects on the stellar profile are
minimal.}\label{fig:scl_smallperi_plummer_i_f}
\end{figure}

\begin{table*}[t]
\centering
\caption[Simulation results for Sculptor’s stars]{The present-day stellar properties for the simulations of Sculptor. In each row, we have the initial stellar velocity dispersion (within 1kpc), the final velocity dispersion, the fraction of stellar mass unbound, the initial half-light radius, the final half-light radius, and the break radius in arcmin and kpc (Eq. \ref{eq:r_break}). }
\label{tbl:scl_sim_stars_results}
\begin{tabular}{lll}
\toprule
Property & Exponential & Plummer\\
\midrule
$\sigma_{\V, i}\,/\,\kms$ & 9.8 & 10.7\\
$\sigma_{\V, f} \,/\,\kms$ & 8.8 & 9.4\\
fractional stellar mass loss & $2.1\times 10^{-6}$ & $0.024$\\
$R_{h, i}\,/\,\kpc$ & 0.169 & 0.202\\
$R_{h, f}\,/\,\kpc$ & 0.189 & 0.227\\
break radius / arcmin & $98$ & $105$\\
break radius / kpc & 2.3 & 2.5\\
\bottomrule
\end{tabular}
\end{table*}

\subsection{Orbital effects of the LMC}\label{sec:scl_lmc}

The Milky Way isn't the only galaxy in town. Recently, work has shown
that the infall of the LMC may substantially affect the Milky Way system
\citep[e.g.,][]{erkal+2019, cautun+2019, garavito-camargo+2021, vasiliev2023}.
With a mass up to one fifth of the MW \citep[e.g.,][]{penarrubia+2015},
the LMC infall affects conclusions about the MW properties and the
orbits of satellites \citep[see
e.g.,][]{patel+2020, battaglia+2022, correamagnus+vasiliev2022}. In this
section, we examine how the LMC affects the orbital history of Sculptor.

We use the \texttt{L3M11} model of the MW and LMC potential from
\citet{vasiliev2024}. The \texttt{L3M11} potential is an evolving
multipole approximation of an N-body simulation including a live MW and
LMC dark matter halo. The potential includes a static MW bulge and disk,
evolving MW and LMC halos, and the MW reflex motion. In their
simulation, the MW was initially a NFW halo with \(r_s=16.5\,\)kpc and
\(M_{\rm 200}= 98.4\times10^{10}\Mo\) , and the LMC a NFW halo with
\(r_s=11.7\) and \(M_{200} = 24.6 \times 10^{10} \Mo\). The total
\texttt{L3M11} MW mass is lighter than our initial \citet{EP2020}
potential. Notably, this model has a previous LMC pericentre
\(\sim6\,\Gyr\) ago.

The inclusion of the LMC reshapes Scl's orbital history, as shown in
Fig.~\ref{fig:scl_lmc_orbits_effect}. In the MW-only potential, Scl's
orbit is typical of a long-term MW satellite. However, Scl's passage by
the LMC \(~\sim0.1\,\Gyr\) ago affects the long-term orbit---Scl instead
reaches an apocentre of nearly \(300\,\kpc\). Scl may even be on first
infall, depending on the MW and LMC mass. Scl's orbit is
counter-rotating to the LMC, so Scl is unlikely to be a LMC satellite.

The timing of the LMC encounter implies a break radius consistent with
the observed density kink. We select the orbit with the
\(6\sigma\)-smallest LMC pericentre. The orbit is selected following the
procedure in Section~\ref{sec:orbital_estimation} except with
observational uncertainties doubled. Results are similar when selecting
for a small MW pericentre instead. This \texttt{LMC-flyby} orbit, is
integrated back in time \(2\,\Gyr\) ago to isolate recent tidal effects.
We modify Scl's initial halo to have \(\rmax = 2.5\,\kpc\) and
\(\vmax = 25\,\kms\), slightly reducing the initial stellar velocity
dispersion. Fig.~\ref{fig:scl_lmc_orbits_effect} shows this selected
orbit in black and Table~\ref{tbl:orbit_ics} records the initial
conditions.

\begin{table*}[t]
\centering
\caption[Orbits and results for Scl in the LMC potential.]{The orbital properties and dark matter evolution for the models including an LMC. Similar to Table \ref{tbl:scl_sim_results} except quantities with respect to the LMC are in parentheses. }
\label{tbl:scl_lmc_sims}
\begin{tabular}{lll}
\toprule
Property & random samples & \texttt{LMC-flyby}\\
\midrule
pericentre & $44\pm 3$ ($29 \pm 2$) & 39 (20)\\
apocentre & $218 \pm 8$ & –\\
time of last pericentre / Gyr & $-0.38\pm0.01$ (-0.11) & -0.33 (-0.10)\\
number of pericentres & 1 (2) & 1 (1)\\
Jacobi radius / kpc & $4.1\pm0.3$ ($4.5\pm0.2$) & 2.8 (2.6)\\
Jacobi radius / arcmin & $168 \pm 11$ ($186\pm6$) & 132 (121)\\
final heliocentric distance / kpc & $83.2\pm2$ & 72.9\\
$\V_\textrm{max, f} / \V_\textrm{max, i}$ &  & 0.928\\
$r_\textrm{max, f} / r_\textrm{max, i}$ &  & 0.763\\
fractional final bound mass &  & 0.5402\\
\bottomrule
\end{tabular}
\end{table*}

\begin{figure}
\centering
\pandocbounded{\includegraphics[keepaspectratio]{figures/scl_lmc_xyzr_orbits.png}}
\caption[Sculptor orbits with LMC]{Similar to Fig.~\ref{fig:scl_orbits}
except for orbits with (orange) and without (green lines) the inclusion
of an LMC (blue line) in the potential. The bottom row additionally
shows the distance between Scl and the LMC over
time.}\label{fig:scl_lmc_orbits_effect}
\end{figure}

\subsection{Tidal effects from the
LMC}\label{tidal-effects-from-the-lmc}

The combined tidal force of the MW and LMC are even weaker for Scl than
in the MW-only case. Fig.~\ref{fig:scl_lmc_sim_images} shows the dark
matter evolution of Scl and the passage of the LMC. With only one MW
pericentre, Scl's dark matter is less disrupted than the previous
MW-only model. The subsequent LMC passage has little effect. As a
result, the dark matter structure evolves mildly and \(\sim 50\%\) of
mass remains bound (Table~\ref{tbl:scl_lmc_sims}).

Correspondingly, the stellar component is nearly unchanged.
Fig.~\ref{fig:scl_lmc_i_f} shows the projected stellar distributions and
density profiles of this model. While the break radius is within the
observed density profile, tidal effects are too weak to be detectible.
Structural properties of the stars similarly evolve little
(Table~\ref{tbl:scl_lmc_sim_stars}). While the instantaneous tidal force
from the LMC is larger than the MW, Scl does not experience the LMC
tidal field long enough to display disturbances.

\begin{figure}
\centering
\pandocbounded{\includegraphics[keepaspectratio]{figures/scl_lmc_sim_images.png}}
\caption[Sculptor simulation snapshots with LMC]{Similar to
Fig.~\ref{fig:scl_sim_images} except for the case where the potential
includes an LMC. The current position and path of the LMC are
represented by the green dot and line respectively. We also plot the
full orbit (over the past 2Gyr) for both Scl and the LMC, as less than
one radial period happens over this time
frame.}\label{fig:scl_lmc_sim_images}
\end{figure}

\begin{figure}
\centering
\pandocbounded{\includegraphics[keepaspectratio]{figures/scl_lmc_i_f.pdf}}
\caption[Sculptor initial and final density with LMC]{Similar to
Fig.~\ref{fig:scl_smallperi_i_f} except for the \texttt{LMC-flyby}
model. With only one MW pericentre and a recent, rapid LMC encounter,
tidal forces do not appear to affect the stellar
distribution.}\label{fig:scl_lmc_i_f}
\end{figure}

\begin{table*}[t]
\centering
\caption[Simulation results for Sculptor’s stars in the LMC+MW potential]{Similar to Table \ref{tbl:scl_sim_stars_results}, but for the properties of the stellar components of the \texttt{LMC-flyby} model of Sculptor. }
\label{tbl:scl_lmc_sim_stars}
\begin{tabular}{lll}
\toprule
Property & Scl: LMC-exponential & LMC-Plummer\\
\midrule
$\sigma_{\V, i}\,/\,\kms$ & 9.0 & 9.4\\
$\sigma_{\V, f} \,/\,\kms$ & 8.8 & 9.2\\
fractional stellar mass loss & $<10^{-12}$ & 0.0013\\
$R_{h, i}$ / kpc & 0.186 & 0.201\\
$R_{h, f}$ / kpc & 0.189 & 0.205\\
break radius & $77'$, 1.6 kpc & $80'$, 1.7 kpc\\
LMC break radius & $23'$, 0.49 kpc & $24'$, 0.52 kpc\\
\bottomrule
\end{tabular}
\end{table*}

\subsection{Robustness to initial
conditions}\label{robustness-to-initial-conditions}

Some initial conditions may be slightly more susceptible to tides. Dark
matter halos which are cored or less concentrated disrupt or lose mass
faster \citep[e.g.,][]{stucker+2023}. We consider alternative dark
matter halo initial conditions in the Appendix. We find that while the
details of the strength of tidal impacts depend on the halo, our models
here remain representative of the tidal effects.

\subsection{Summary}\label{summary}

We find, including only the MW, that tides only remove dark matter from
the outskirts of Scl. The central cusp and compact stellar distribution
are resilient to tides and any tidal effects are well-outside the reach
of current observations. We then find that the LMC strongly perturbs
Scl's orbit---in this case, Scl may be on first infall. However, with
only 2 pericentres, the combined tides of the LMC and MW tides are
weaker than for our initial model. In either case, tides are unlikely to
affect Sculptor's stellar component.

\section{Tidal effects on Ursa Minor}\label{tidal-effects-on-ursa-minor}

The tidal evolution of Ursa Minor is similar to Sculptor in the MW-only
case. Fig.~\ref{fig:umi_sim_images} shows snapshots of the DM evolution.
UMi loses more DM mass, with only about 3\% of the total mass remaining
({[}Table~\ref{tbl:umi_sim_results}; {]}). However, tidal features in
the observed stellar components are still extremely faint, appearing
outside 100 arcminutes in Fig.~\ref{fig:umi_smallperi_i_f}. Tides only
moderately affect the observed size and velocity dispersion of Ursa
Minor (Table~\ref{tbl:umi_sim_stars_results}).

A stream may just be beyond observational limits if Ursa Minor's stars
were initially more extended. As seen in Fig.~\ref{fig:umi_plummer_i_f},
the tidal tails are more luminous than for Sculptor and would begin near
the outermost density measurement. The properties of the stream are
shown in Fig.~\ref{fig:umi_tidal_stream}. The stream stars become
apparent outside of about 2 degrees, or the Jacobi radius. As expected
for tidal tails, the stream shows gradients in each observable, with a
dispersion in each observable similar to that of the dwarf progenitor.

\begin{table*}[t]
\centering
\caption[Simulation results for Ursa Minor’s dark matter]{The present-day properties for Ursa Minor’s final dark matter halo. See Table \ref{tbl:scl_sim_results} for details. }
\label{tbl:umi_sim_results}
\begin{tabular}{lll}
\toprule
Property & Random orbits & \smallperi{}\\
\midrule
pericentre & $37\pm3$ & 30\\
apocentre & $83 \pm 4$ & 75\\
time of last pericentre & $-0.97 \pm 0.07$ & -0.80\\
number of pericentres & $\sim 8$ & 8\\
Jacobi radius / kpc & $3.7 \pm 0.3$ & 2.9\\
Jacobi radius / arcmin & $184 \pm 12$ & 156\\
final heliocentric distance & $70.1 \pm 3.6$ & 64.7\\
${\vmax}_f / {\vmax}_i$ &  & 0.511\\
${\rmax}_f / {\rmax}_i$ &  & 0.249\\
fractional dm final mass &  & 0.035\\
\bottomrule
\end{tabular}
\end{table*}

\begin{table*}[t]
\centering
\caption[Simulation results for Ursa Minor’s stars]{Similar to Table \ref{tbl:scl_sim_stars_results}, the present-day stellar properties for the simulation of Ursa Minor for exponential and Plummer stars. }
\label{tbl:umi_sim_stars_results}
\begin{tabular}{lll}
\toprule
Property & smallperi-exp & smallperi-Plummer\\
\midrule
$\sigma_{\V, i}\,/\,\kms$ & 10.0 & 10.9\\
$\sigma_{\V, f}\,/\,\kms$ & 8.2 & 8.5\\
fractional stellar mass loss & $0.00015$ & 0.039\\
$R_{h, i}\,/\,\kpc$ & 0.135 & 0.151\\
$R_{h, f}\,/\,\kpc$ & 0.169 & 0.191\\
break radius & 197 arcmin, 3.7 kpc & 204 arcmin, 3.8 kpc\\
\bottomrule
\end{tabular}
\end{table*}

\begin{figure}
\centering
\pandocbounded{\includegraphics[keepaspectratio]{figures/umi_sim_images.png}}
\caption[Ursa Minor simulation snapshots]{Similar to
Fig.~\ref{fig:scl_sim_images} except for Ursa Minor on the \smallperi{}
orbit.}\label{fig:umi_sim_images}
\end{figure}

\begin{figure}
\centering
\pandocbounded{\includegraphics[keepaspectratio]{figures/umi_smallperi_i_f.pdf}}
\caption[Ursa Minor simulated density profiles]{The tidal effects on the
stellar surface density of Ursa Minor for exponential stars on the
\smallperi{} orbit.}\label{fig:umi_smallperi_i_f}
\end{figure}

\begin{figure}
\centering
\pandocbounded{\includegraphics[keepaspectratio]{figures/umi_plummer_i_f.pdf}}
\caption[Ursa Minor simulated density profiles]{The tidal effects on the
stellar surface density of Ursa Minor for Plummer stars on the
\smallperi{} orbit.}\label{fig:umi_plummer_i_f}
\end{figure}

\begin{figure}
\centering
\pandocbounded{\includegraphics[keepaspectratio]{figures/umi_sim_stream.pdf}}
\caption[Ursa Minor Predicted stream]{The properties of the stream
around the UMi \smallperi{} orbit with Plummer stars. The panels are all
as a function of \(\xi'\), the distance along the stream as defined by
the current GSR proper motion vector. The top panels show the GSR proper
motions in RA and Dec, and the bottom two show the distance and GSR
radial velocities. To sample the stream, we randomly draw 100,000
samples from the snapshot based on the stellar weights. A detectible
gradient in \(\mu_{\alpha*}\) and LOS velocity should be detectible if
the stream is tracked across several
degrees.}\label{fig:umi_tidal_stream}
\end{figure}

\subsection{Effects of the LMC}\label{effects-of-the-lmc}

Fig.~\ref{fig:umi_orbits_lmc} shows the effects of including an LMC on
the orbit of Ursa Minor. Predominantly, the effect is to increase the
orbital period, apocentre and pericentre. However, the orbit remains in
a similar plane and with similar morphology. As UMi is on the opposite
side of the Galaxy of the LMC, this is not unexpected. The deviation
from the MW-only orbit is mostly due to the LMC-induced reflex motion of
the Milky Way.

\begin{figure}
\centering
\pandocbounded{\includegraphics[keepaspectratio]{figures/umi_lmc_xyzr_orbits.png}}
\caption[Ursa Minor orbits with LMC]{Orbits of Ursa Minor with (orange)
and without (green) an LMC. The final positions of Ursa Minor and the
LMC are plotted as scatter points and the solid blue line represents the
LMC trajectory. Note that the LMC mostly increases Ursa Minor's
pericentres and apocentres.}\label{fig:umi_orbits_lmc}
\end{figure}

\subsection{Summary}\label{summary-1}

While tides affect UMi more strongly than Scl, the tidal effects are
insufficient to reshape the observed stellar density profile. Faint
tidal tails may be observable with deeper data. Finally, including the
LMC in the potential further weakens the tides experience by UMi.
