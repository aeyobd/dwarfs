The satellite galaxies of the Milky Way (MW), except the Magellanic
Clouds, are all dwarf spheroidals (dSphs)---gas-free, non-rotating,
dark-matter-dominated stellar systems. The light profiles of dSphs
typically follow an exponential law, with a sharply declining outer
density. Yet systems like the Sculptor (Scl) and Ursa Minor (UMi) dSphs
host member stars out to \({\sim}10\) effective radii, indicating outer
deviations from an exponential law. The origin of these extended
profiles is unclear, possibly arising from Galactic tides or intrinsic
properties. In this thesis, we review the evidence for extended light
profiles in Scl and UMi, validating the Bayesian membership catalogue of
\citet{jensen+2024}. To evaluate if tides can produce these density
features, we conduct idealized N-body simulations of both galaxies in
the tidal field of the Milky Way. We find that neither dwarf experiences
tides strong enough to affect their stellar distribution. One
complication is that Sculptor's orbit is strongly influenced by the
presence of the Large Magellanic Cloud (LMC). Our N-body models still
suggest that the combined tides of the LMC and MW are unable to explain
Scl's outer profile. We conclude that the extended light profiles of Scl
and UMi are not of tidal origin. They are instead likely innate,
possibly explained by past mergers, accretion events, or episodic star
formation.
