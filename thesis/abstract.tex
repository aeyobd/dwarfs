Dwarf galaxies are unique cosmological probes, testing our understanding
of the smallest scales of structure formation in the Universe. Among the
classical dwarf galaxies, Sculptor and Ursa Minor appear unusual, with
more stars than expected in the outer regions compared with the typical
Exponential density profile, rememnescent of tidal interaction. I verify
the validity of this excess, checking \citet{jensen+2024}'s bayesian
membership catalogue for completeness, simple cuts, and other surveys.
To evaluate if tides indeed can explain these density features, I
conduct idealized N-body simulation of both galaxies in the tidal field
of the Milky Way. In both cases, the orbits are well constrained,
producing pericenters larger than expected to produce substantial tidal
impacts. In addition, the time since the last pericenter does not agree
with a tidal interpretation. However, Sculptor's orbit is strongly
influenced by the presence of an LMC. With an LMC, Sculptor's long term
orbital history is highly uncertain, however recent tidal effects due to
the Milky Way and LMC only minimally affect this galaxy. Because tides
likely do not explain the presence of these density features, I review
other theories about the formation of dwarf galaxy ``halos,'' including
mergers / accretion events and complex multi-component star formation
histories. The density profiles of Sculptor and Ursa Minor, with their
deviations from the classically expected exponential, present an
interesting puzzle, likely with a solution involving their formation and
history in the early universe.
