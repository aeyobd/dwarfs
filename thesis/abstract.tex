The satellite galaxies of the Milky Way (MW), with the exception of the
Magellanic clouds, are all dwarf spheroidals (dSphs)---gas-free,
non-rotating stellar systems who's gravity is dominated by dark matter.
The light profiles of dSphs are usually well-approximated by an
exponential law, with a sharp outer cutoff. Yet systems like the
Sculptor (Scl) and Ursa Minor (UMi) dSphs have been found to host member
stars far from the main body, out to \textasciitilde10 effective radii.
The origin of these extended profiles is not clear, and may reflect
either the influence of Galactic tides or instead an innate feature. In
this thesis, I review the excestence of an extended light profiles in
Scl and UMi, validating the Bayesian membership catalogue of
\citet{jensen+2024}. To evaluate if tides indeed can explain these
density features, I conduct idealized N-body simulation of both galaxies
in the tidal field of the Milky Way. I find that neigher dwarf has been
subjected to tidal forces strong enough to create the observed extended
light profiles from initially exponential profiles. One complication is
that Sculptor's orbit is strongly influenced by the presence of the
Large Magellanic Cloud (LMC). Because the LMC's mass is poorly
constrained, Sculptor's long term orbital history is highly uncertain.
Despite this uncertainty, our models still suggest that the combined
tides of the LMC and MW are still unable to explain Scl's outer profile
over the past 5 Gyr. I conclude that the extended light profiles of Scl
and UMi are innate, likely reflecting their formation histories.
Mergers, accretion events, or complex, multi-component star formation
episodes are possible explanations for such an extended outer profile.
