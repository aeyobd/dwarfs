The satellite galaxies of the Milky Way (MW), except the Magellanic
Clouds, are all dwarf spheroidals (dSphs)---gas-free, non-rotating
stellar systems dynamically dominated by dark matter. The light profiles
of dSphs are usually well-approximated by an exponential law, with a
steep outer cutoff. Yet systems like the Sculptor (Scl) and Ursa Minor
(UMi) dSphs have distant member stars, out to \(\sim10\) effective radii
from the centre, indicative of outer deviations from an exponential law.
The origin of these extended profiles is unclear, and may reflect either
the influence of Galactic tides or an innate feature. In this thesis, we
review the existence of extended light profiles in Scl and UMi,
validating the Bayesian membership catalogue of \citet{jensen+2024}. To
evaluate if tides indeed can explain these density features, we conduct
idealized N-body simulation of both galaxies in the tidal field of the
Milky Way. We find that neither dwarf has been subjected to tidal forces
strong enough to create the observed extended light profiles from
initially exponential profiles. One complication is that Sculptor's
orbit is strongly influenced by the presence of the Large Magellanic
Cloud (LMC). Our models still suggest that the combined tides of the LMC
and MW are unable to explain Scl's outer profile. We conclude that the
extended light profiles of Scl and UMi are innate, likely reflecting
their assembly histories. Mergers, accretion events, or episodic star
formation are possible explanations for the observations of extended
light profiles.
