Dwarf galaxies host, in many ways, the most extreme galactic
environments in the universe. These little galaxies are typically
defined to fainter than the LMC or SMC \citep[\(M_V \gtrsim -18\),
e.g.][\citet{hodge1971}]{mcconnachie2012}. As the smallest class of
galaxies, dwarfs are the most numerous galaxies in the universe. Dwarf
galaxies are highly \emph{dark-matter dominated}, with mass to light
ratios sometimes exceeding 100 to 1000 \(M_\odot/ L_\odot\). Because of
their small total masses, many dwarf galaxies are \emph{quenched}, with
little to no recent star formation. Their stellar populations are
\emph{relics} from the early universe, consisting of many of the oldest
and most metal poor stars. Understanding the properties of dwarf
galaxies thus has relevance across astronomy, from cosmological
structure formation on the smallest scale to extremely metal-poor
stellar populations to chemical evolution to galactic dynamics. Of the
classical dwarfs, the Sculptor and Ursa Minor dwarf spheroidal galaxies
stand out with a more extended density profile relative to an
exponential, hinting at tidal effects. As a case study in tidal effects
and our understanding of dwarf galaxy formation and evolution, we aim to
understand the dynamical history of these galaxies.

In this section, we first describe the observational history of dwarf
galaxies, what \emph{Gaia} is and how this telescope has changed our
understanding of the local group, the present state of dwarf galaxy
knowledge, and why Sculptor and Ursa Minor stand out. We then describe
the role of dwarf galaxies in cosmology, their formation and our current
theory of hierarchichal structure formation. We later discuss idealized
simulations, tidal effects, break radii, and recent developments in
tidal simulations. Finally, we provide a roadmap to the thesis as a
whole at the end of this section.

\section{Observational context}\label{observational-context}

\subsection{Early observations}\label{early-observations}

The Magellanic clouds are the most prominent of the Milky Way satellites
and were likely always known to humans in the southern hemisphere.
However, the first discovered dwarf galaxies in 1938, Sculptor and
Fornax, heralds i n an era of discovery and investigation into a novel
astronomical system. \citet{shapley1938} present Sculptor and Fornax as
some kind of \emph{stellar system} resembling the Magellanic clouds and
globular clusters. The first direct hints that these stellar systems are
galaxies originates from early spectroscopic work
\citep[e.g.,][\citet{gallagher+wyse1994}, \citet{mateo1994},
\citet{pryor1994}, \citet{pryor1996}, \citet{gerhard1994}, and
\citet{olszewski1998}.]{aaronson1983}. Based on their velocity
dispersions given their physical sizes, the mass to light ratios of
these dwarf galaxies is far larger than expected for star clusters
without dark matter. In general, these objects have reached a consensus
that they have high velocity dispersion, old spheroidal stellar
populations, and low stellar masses. While we now know the basic
properties of dwarf galaxies, tremendous progress has been made in the
past 10-20 years with large area deep photometric surveys, multi-object
spectrographs, high resolution chemistry, and theoretical work,
unveiling that dwarf galaxies are each unique systems with different
formation pathways \citep[e.g.][]{simon2019}.

Dwarf galaxies span a large range of size and morphology.
Fig.~\ref{fig:galaxy_images} displays several images of dwarf galaxies.
The largest in the local group but not always considered a dwarf galaxy,
the Large Magellanic Cloud (LMC), to classical systems Sculptor and Ursa
Minor dwarf spheroidal, to an ultra-faint dwarf galaxy, Boötes V. From
this figure, we notice that while large LMC-like dwarf galaxies exhibit
complex structure such as a bar, young blue stars, and irregular shapes,
the classical and faint dwarf galaxies are composed mostly of very old,
faint, red stars, very diffuse and often large on the sky if local.
These galaxies are typically classified as dwarf spheroidals (dSph),
however we typically omit the dSph designation and refer to these
galaxies by their constellation hereafter.

Since the discovery of dwarf galaxies around the Milky Way,
observational work has attempted to measure and refine their basic
properties \citep[e.g.][]{mateo1998}. While the Milky Way's satellites
are close (by extragalactic standards), their low numbers of stars and
large apparent sizes present challenges for observational work. In
particular, the properties of the faintest dwarf galaxy systems and
especially \emph{ambiguous} systems remains a pressing challenge as a
unique probe into the nature of the early universe and dark matter.

\textbf{Classical systems}, the brightest of the local group
(non-Magellanic) dwarf galaxies, are perhaps the systems best studied
and with the most stringingly derived properties. Defined\footnote{Note
  that some work like \citet{simon2019} instead define classical dwarfs
  based on luminosity as compared to ultra faint dwarfs. This may
  include galaxies such as Antlia I, Crater II, Canes Venatici I, and
  Sagittarius. We stick with the historical definition for simplicity
  and since the Gaia observations are most numerous for the historical
  classicals.} as the first six discovered dwarf spheroidal galaxies,
(Sculptor, Ursa Minor, Fornax, Draco, Leo I, Leo II, and Carina),
classical systems are typically old, spheroidal, and gas-poor. While
extending 1-2 degrees across the sky (e.g.
Figs.~\ref{fig:scl_selection}, \ref{fig:umi_selection}, \ref{fig:fornax_selection}),
these dwarfs contain large numbers of bright (giant) stars allow for
thousands of stars to be observed with deep photometry and spectroscopy
\citep[e.g.][\citet{pace+2020}]{tolstoy+2023}. As a result, the
determination of fundamental properties such as the position, size,
orientation, distance \citep[from 100s of RRL stars,
e.g.][]{tran+2022, garofalo+2025}, proper motions (from \emph{Gaia},
e.g. \citet{battaglia+2022}, \citet{MV2020a}), line-of-sight (LOS)
velocity and dispersion \(\sigma_v\), are all relatively well
constrained today and show excellent convergence among different works.
However, ongoing research still is refining our understanding of the
detailed structure of Milky Way satellites, hinting that these objects
may be more complex than at first glance.

\begin{figure}
\centering
\includegraphics[width=5.41667in,height=5.41667in]{figures/galaxy_pictures.pdf}
\caption[Dwarf Galaxy Pictures]{Images of the LMC (DSS2), Sculptor (DES
2), Ursa Minor (UNWISE with Gaia point sources over-plotted), and Bootes
V (SDSS). While the LMC is very prominent in the sky, even the classical
dwarf galaxies are not obvious except in terms of star counts (e.g.
Fig.~\ref{fig:scl_selection}). All (non-LMC) dwarfs occupy the central
third to sixth of the image.}\label{fig:galaxy_images}
\end{figure}

\begin{figure}
\centering
\includegraphics[width=5.41667in,height=\textheight,keepaspectratio]{figures/mw_satellites_1.jpg}
\caption[Dwarf galaxies sky position]{The location of MW dwarf galaxies
on the sky. We label the classical dwarf galaxies (green diamonds),
fainter dwarfs (blue squares), globular clusters (orange circles), and
ambiguous systems (pink open hexagons). Globular clusters are more
centrally concentrated, but dwarf galaxies are preferentially found away
from the MW disk. Sculptor and Ursa Minor are highlighted as two dwarfs
we study later. The background image is from ESA/Gaia/DPAC
(https://www.esa.int/ESA\_Multimedia/Images/2018/04/Gaia\_s\_sky\_in\_colour2).
Dwarf galaxies (confirmed), globular clusters, and ambiguous systems are
from the \citet{pace2024} catalogue (version
1.0.3).}\label{fig:mw_satellite_system}
\end{figure}

\subsection{\texorpdfstring{The era of
\emph{Gaia}}{The era of Gaia}}\label{the-era-of-gaia}

Some of the most fundamental properties of astronomical objects are
their positions and velocities. Unfortunately, determining distances to
stars is nontrivial. Additionally, while line-of-sight (LOS) velocities
are easily determined from spectroscopy, the tangental velocities,
perpendicular to the LOS, are best measured through proper motions,
typically requiring accurate determinations of small changes in a star's
position. \emph{Gaia}'s mission addresses these problems by making
extremely precise measurements of stellar positions over time.

\emph{Gaia} was designed to revolutionize proper motion and parallax
measurements. \emph{Gaia} is a space-based, all-sky survey telescope
with two primary 1.45x0.5m mirrors situated at the Sun-Earth L2 Lagrange
point \citep{gaiacollaboration+2016}. \emph{Gaia} was launched in 2013,
completing its space-based mission in 2025 (with two more data releases
planned). By imagining two patches of sky on the same focal plane,
separated by a fixed angle of 106.5 degrees, \emph{Gaia} is able to
measure absolute proper motions. Stars in different regions of the sky
are affected by parallax differently, so by precisely observing the
relative positions of stars separated by large angles during multiple
epochs over a year, an absolute all-sky reference frame can be derived
\citep{gaiacollaboration+2016}. In addition to precise astrometric
information, \emph{Gaia} measures the magnitude of stars in the very
wide \emph{G} band (330-1050nm), blue and red colours using the blue and
red photometers (BP and RP, 330-680 nm and 640-1050 nm respectively),
and takes low resolution BP-RP spectra and radial velocity measurements
of bright stars (\emph{Gaia} radial velocity magnitudes \textless16)
\citep{gaiacollaboration+2016}. For our purposes, most useful are
\emph{Gaia}'s measurements of \(G\) magnitude,
\(G_{\rm BP} - G_{\rm RP}\) colour, the position \(\alpha, \delta\), and
the proper motions in RA and declination \(\mu_{\alpha*}\) and
\(\mu_\delta\).\footnote{\(\mu_{\alpha*}\) allows where
  \(\mu_{\alpha*} = \mu_\alpha \cos \delta\) corrects for projection
  effects{]}.}

\emph{Gaia} has revolutionized many astronomical disciplines, the least
of which is local group and Milky Way science. While proper motions of a
dwarf galaxies has been measured in a case by case basis by the Hubble
Space Telescope, a full systematic determination of proper motions for
most dwarf galaxies was unavailable until \emph{Gaia}
\citep{MV2020a, battaglia+2022}. \emph{Gaia} has furthermore allowed for
the detection and improved measurements of halo substructures and
streams \citep{bonaca+price-whelan2025}, Milky Way structure and
kinematics \citep{hunt+vasiliev2025}.

For Local group dwarf spheroidals, \emph{Gaia} has allowed for many of
the first proper motion measurements. While the proper motion
uncertainty on a typical dwarf member star is often with large
uncertainties, by combining the proper motions for 100s to 1000s of
stars in \emph{Gaia}, incredibly precise proper motion measurements can
be determined \citep[e.g.][]{MV2020a}, sometimes only limited by
\emph{Gaia}'s systematic error floor. Proper motions have ushered in a
new dynamical era for MW satellites studies, where we often have precise
determinations of the dwarf galaxy's (short term) orbit under a given MW
potential.

\subsection{Dwarf galaxies today}\label{dwarf-galaxies-today}

\subsubsection{Exponential stellar density
profiles}\label{exponential-stellar-density-profiles}

For a long time, exponential surface density profiles have been used as
an description for spiral galaxies {[}REFS{]}.

Since \citet{faber+lin1983} demonstrated that an Exponential density
profile is a reasonable empirical fit to dwarf galaxies, many works have
fitted an Exponential profile (\citet{Eskridge} 1988a,c; Hodge et al
1991a,b; \citet{IH1995}, ) in addition to Sérsic, King and various other
profiles. \citet{lelli+2014} (example of using exponential.)

\citet{wang+2019} for fornax, \citet{mcconnachie+irwin2006} for Exp,
Plummer, King in other galaxies.

However, even starting from \citet{aparicio1997}, some people have noted
deviations from this simple empirical rule. Additionally,
\citet{herrmann+hunter+elmegreen2013},
\citet{herrmann+hunter+elmegreen2016} note that at least in a
photometric sample of more distant dwarf disky/Irr/blue compact dwarfs,
that many dwarfs show deviations from exponential profiles.

\begin{itemize}
\tightlist
\item
  \citet{makarov+2012} isolated local volume dwarf galaxy with central
  deprivation compared to exponential, spheroidal
\item
  \citet{martin+2016}, exponential fits to many pandas dwarf galaxies
\item
  \citet{moskowitz+walker2020}, generalized plummer fits.
\end{itemize}

\subsubsection{Hints of tides and
disruptions}\label{hints-of-tides-and-disruptions}

\subsubsection{Kinematics and structure}\label{kinematics-and-structure}

\subsection{Sculptor and Ursa Minor: Hints of tidal
signatures?}\label{sculptor-and-ursa-minor-hints-of-tidal-signatures}

Sculptor and Ursa Minor may appear to be typical dwarf galaxies at first
glance. Table~\ref{tbl:scl_obs_props}. describe the present-day
properties of each galaxy. Indeed, Sculptor as the first discovered
dwarf galaxy, is often described as a ``prototypical'' dSph. However,
both galaxies have a long history of speculation that they may be
influenced by the Milky Way's gravity (see discussion). Only recently,
with \emph{Gaia} data and spectroscopic followup, has the detection of a
density excess become more unmistakable. Using \citet{jensen+2024}'s
algorithm (described briefly below), \citet{sestito+2023a},
\citet{sestito+2023b} report that a ``kink'' in the density profile,
beginning around 30 arcmin for both Sculptor and Ursa Minor. They
spectroscopically followup some of the most distant stars, finding
multiple members between 6-12 half-light radii from the centre of each
dwarf (REF fig.~XXX). If dwarfs initially begin with exponential
profiles like Fornax, then these stars should be much rarer.

Perhaps the most straighforward explanation for the density profiles of
Sculptor and Ursa Minor are tidal interactions. \citet{sestito+2023a},
\citet{sestito+2023b} conclude that tides are likely the explanation.
Fig.~\ref{fig:scl_umi_fnx_vs_penarrubia} reproduces their comparison,
showing that Sculptor and Ursa Minor's density profiles deviate
substantially from fornax's in the outskirts, and are well-described by
the tidal model from \citet{PNM2008}. Our goal is to determine, assuming
\(\Lambda\)CMD, if the effects of the Milky Way (or other satellites)
may indeed create observable tidal signatures in these galaxies. If
tides cannot explain these features, Sculptor and Ursa Minor may instead
contain a extended stellar ``halo'' or second component of the
galaxy---illustrating evidence of complex history or formation in each
galaxy and forcing formation models to confront a diversity of density
profiles in the local group.

\begin{figure}
\centering
\pandocbounded{\includegraphics[keepaspectratio]{./figures/scl_umi_vs_penarrubia.pdf}}
\caption[Sculptor and Ursa Minor match tidal models]{A plot of the
surface density profiles of Sculptor, Ursa Minor, and Fornax scaled to
their half-light radius and the density at half-light radius (data
described in Section~\ref{sec:data}). The lines represent the initial
and final models from \citet{PNM2008}'s model (Fig. 4 top left, stellar
segregation of 0.20 moving on a highly eccentric orbit peri:apo = 1 :
100)).}\label{fig:scl_umi_vs_penarrubia}
\end{figure}

\begin{table*}[t]
\centering
\caption[Observed Properties of Sculptor]{Observed properties of Sculptor. References are: 1. Muñoz et al. (2018) Sérsic fits, 2. Tran et al. (2022) RR lyrae distance, 3. Alan W. McConnachie and Venn (2020b), 4. Arroyo-Polonio et al. (2024). }
\label{tbl:scl_obs_props}
\begin{tabular}{lll}
\toprule
parameter & value & Source\\
\midrule
$\alpha$ & $15.0183 \pm 0.0012^\circ$ & 1\\
$\delta$ & $-33.7186 \pm 0.0007^\circ$ & –\\
distance modulus & $19.60 \pm 0.05$ & 2\\
distance & $83.2 \pm 2$ kpc & –\\
$\mu_{\alpha*}$ & $0.099 \pm 0.002 \pm 0.017$ mas yr$^{-1}$ & 3\\
$\mu_\delta$ & $-0.160 \pm 0.002_{\rm stat} \pm 0.017_{\rm sys}$ mas yr$^{-1}$ & –\\
LOS velocity & $111.2 \pm 0.3\ {\rm km\,s^{-1}}$ & 4\\
$\sigma_v$ & $9.7\pm0.2\ {\rm km\,s^{-1}}$ & –\\
$R_h$ & $9.79 \pm 0.04$ arcmin & 1\\
ellipticity & $0.37 \pm 0.01$ & –\\
position angle & $94\pm1^\circ$ & –\\
$M_V$ & $-10.82\pm0.14$ & –\\
\bottomrule
\end{tabular}
\end{table*}

\begin{table*}[t]
\centering
\caption[Observed Properties of Ursa Minor]{Observed properties of Ursa Minor. References are: (1) Muñoz et al. (2018) Sérsic fits, (2) Garofalo et al. (2025) RR lyrae distance, (3) Alan W. McConnachie and Venn (2020a), (4) Pace et al. (2020), average of MMT and Keck results. }
\label{tbl:umi_obs_props}
\begin{tabular}{lll}
\toprule
parameter & value & Source\\
\midrule
$\alpha$ & $ 227.2420 \pm 0.0045$˚ & 1\\
$\delta$ & $67.2221 \pm 0.0016$˚ & –\\
distance modulus & $19.23 \pm 0.11$ & 2\\
distance & $70.1 \pm 3.6$ kpc & –\\
$\mu_\alpha*$ & $-0.124 \pm 0.004 \pm 0.017$ mas yr$^{-1}$ & 3\\
$\mu_\delta$ & $0.078 \pm 0.004_{\rm stat} \pm 0.017_{\rm sys}$ mas yr$^{-1}$ & –\\
LOS velocity & $-245.9 \pm 0.3_{\rm stat} \pm 1_{\rm sys}$ km s$^{-1}$ & 4\\
$\sigma_v$ & $8.6 \pm 0.3$ & –\\
$R_h$ & $11.62 \pm 0.1$ arcmin & 1\\
ellipticity & $0.55 \pm 0.01$ & –\\
position angle & $50 \pm 1^\circ$ & –\\
$M_V$ & $-9.03 \pm 0.05$ & –\\
\bottomrule
\end{tabular}
\end{table*}

\section{Cosmological and theoretical
context}\label{cosmological-and-theoretical-context}

The leading theory of cosmology, \(\Lambda\)CMD (cold dark matter) holds
that the universe is composed of about 68\% dark energy (\(\Lambda\)),
27\% dark matter (DM), and 5\% regular baryons
\citep{planckcollaboration+2020}. While baryonic matter is the only
matter we can directly understand or observe, we still know the general
properties of dark matter and dark energy. Dark energy causes the
acceleration of the expansion of the universe on large scales. We do not
discuss dark energy here as it does not substantially affect the local
group today. Dark matter makes up the vast majority of mass in galaxies.
Typically, galaxies have baryonic to dark matter ratios of between 1:5
to beyond 1:100. The faintest dwarf galaxies harbouring the largest
proportion of dark matter. In \(\Lambda\)CMD, dark matter is assumed to
only interact gravitationally with itself and other matter (transparent
to light, or \emph{dark}). Dark matter is also created \emph{cold},
i.e.~typical velocities much smaller than the speed of light in the
early universe. Dark matter is a key ingredient in cosmological
structural formation, dynamical galaxy structure, and galaxy-galaxy
interactions.

\subsection{Formation of dwarf
galaxies}\label{formation-of-dwarf-galaxies}

The very early universe was almost featureless. As observed from the
cosmic microwave background created at redshift \(z\sim 1100\), the
temperature is a near perfect blackbody with fluctuations only at 1 part
in 10,000. These tiny perturbations began to grow as the universe ages
and expands. Initially hot and dense, baryonic matter was prevented from
collapsing by thermal / radiation pressure. However, dark matter was
uninhibited, freely passing through the universe and collapsing into
denser regions. Each self-gravitating overdensity of dark matter is
known as a \emph{halo}. As the universe continued to expand and cool
down, baryonic matter began to collapse into dark matter halos.
Eventually, the densest regions of baryons situated near the centres of
halos would become the first stars of the first galaxies.

Halos rarely evolve as isolated systems. Instead, structure formation is
\emph{hierarchical}, where smaller \emph{subhalos} are accreted into
larger halos which may accrete into yet larger halos. The most massive
halos are made up of thousands of subhalos (verify), with each subhalo
becoming entire galaxies making up a galaxy cluster. But, even the
smallest galaxies reside in a \emph{halo}, with the dwarf galaxies
representing entities with simpler \emph{assembly histories}.

\subsection{Cosmological dark matter density
profiles}\label{cosmological-dark-matter-density-profiles}

From cosmological simulations, we know that across many orders of
magnitude, dark matter halos appear self-similar. In
\citet{NFW1996};\citet{NFW1997}, hereafter NFW, the authors observe that
the radial density profiles \(\rho(r)\) follow a two-parameter law:
\begin{equation}{
\rho/\rho_s= \frac{1}{(r/r_s)(1+r/r_s)^2}
}\end{equation} where \(r_s\) and \(\rho_s\) are the scale radius and
density (see below). This density profile has shown remarkable success
as a description of \(\Lambda\)CMD halos \citep[REFS, but see
also][darkEXP, ect.]{navarro2010}. This density profile is \emph{cuspy},
where the density continuously rises like \(\rho \sim 1/r\) at the
centre. Also note that the total mass of the NFW profile logarithmically
diverges. Typically, the virial mass of NFW halos is discussed instead,
defined as \(M_{200}\), the mass within a region \(r_{200}\) containing
a mean density 200 times the critical density of the universe
\(\rho_{\rm crit}\), so
\(M_{200} = M(r<r_{200}) = 200\ (4\pi/3) \ r_{200}^3\ \rho_{\rm crit}\).

There are several parameterizations of the NFW profile. \citet{NFW1996}
initially write the halo in terms of \(r_s\) and a concentration
\(c= r_{200} / r_s\) relating the virial radius \(r_{200}\) to \(r_s\).
. In this case,
\(\rho_s = \rho_{\rm crit} c^3 / [\log(1+c) - c/(1+c)]\). However, we
chose to discuss halos in terms of velocity space. The circular velocity
of the NFW halo, \(v_{\rm circ}(r) = \sqrt{G M(r) / r}\) related to the
enclosed mass \(M(r)\), reaches a maximum of \(v_{\rm max}\) at
\(r_{\rm max} = \alpha\,r_s\) where \(\alpha\approx2.16258\). Thus
\(r_{\rm max}\) is directly related to the scale radius and
\(v_{\rm max}\) is related to the total mass.

\begin{figure}
\centering
\pandocbounded{\includegraphics[keepaspectratio]{/Users/daniel/Library/Application Support/typora-user-images/image-20250715095831443.png}}
\caption[The NFW density profile]{showing the fits to the NFW halos for
several different halos from \citet{NFW1996}.}
\end{figure}

\subsection{Typical properties}\label{typical-properties}

A key prediction of \(\Lambda\)CDM is the stellar mass halo mass
relation, describing the amount of stellar mass forms in a (sub)halo of
a given size. In particular, the SMHM relation grows especially steep in
the dwarf galaxy regime---many dwarf galaxies are formed in halos of
similar masses. \citet{fattahi+2013}, \citet{fattahi+2018}

\begin{itemize}
\tightlist
\item
  \citet{battaglia+nipoti2022}
\end{itemize}

\begin{figure}
\centering
\pandocbounded{\includegraphics[keepaspectratio]{/Users/daniel/Library/Application Support/typora-user-images/image-20250715100059332.png}}
\caption[stellar mass halo mass relation]{The stellar mass halo mass
relation, where \(v_{\rm max}\) is related to the halo mass. Taken from
\citet{fattahi2018}.}
\end{figure}

\subsection{Challenges to cosmology or modelling
limitations?}\label{challenges-to-cosmology-or-modelling-limitations}

One of the open questions in dwarf galaxy evolution is the formation of
dark matter \emph{cores} where the density becomes constant instead
\(\rho \sim 1\). A leading theory is cores form through of baryonic
effects and feedback. If dark matter deviates from \(\Lambda\)CDM, then
the NFW density profile may not describe halos anymore. \emph{Warm} dark
matter, relativistic in the early universe but cooler now, smooths out
small-scale features and softens the cusps of dark matter halos.
\emph{Self-interacting} dark matter instead can form cores but also the
cores can collapse into a density peak.

\begin{itemize}
\tightlist
\item
  \citet{walker+penarrubia2011}: measuring cores and cusps
\item
  \citet{dicintio+2013}, \citet{navarro+2010}, NFW profiles may actually
  be more Einasto like
\end{itemize}

\section{Simulating tidal effects}\label{simulating-tidal-effects}

Since the discovery of dwarf galaxies, a large body of work has
speculated, considered, or simulated tidal effects. However, before
\emph{Gaia}, the orbits of most systems were essentially unconstrained.
The theory developed still relates to today's work.

Because we only have finite compute resources, major assumptions or
limitations are common in simulations. Cosmological simulations,
starting from the very early universe, create a realistic environment,
structure formation, and merger history. But, cosmological simulations
typically barely resolve dwarf galaxies {[}REFS{]}, leading to numerical
artifacts and artificial disruption. On the other hand, idealized
simulations are able to reach high resolution and numerical convergence
for a single dwarf galaxy. But, idealization may not be the most
realistic environment, neglecting mergers, assembly history, and often
baryonic physics. We use idealized simulations here which are more
powerful in accurately assessing tidal effects, given that the
idealizations do not impact these galaxies's recent history too much.

\subsection{Simulating large gravitational systems: The N-body
method}\label{simulating-large-gravitational-systems-the-n-body-method}

While the dominant force on large scales in the universe, calculating
the gravitational interaction between a large number of objects becomes
computationally prohibitive. So, how do we manage to create simulations
with billions of particles across vast volumes of the universe?

Perhaps the simplest method to compute the evolution of dark matter is
through \emph{N-body simulations}. A dark matter halo is represented as
a large number of dark matter particles (bodies).

Naively, the newtonian gravitational force requires adding together the
forces from each particle at each particle, causing a computational cost
that scales quadratically with the number of particles, or \(O(N^2)\).
With this method, simulating a large number of particles, such as
10\^{}6, would require about 10\^{}12 force evaluations at each time
step, making cosmological and high-resolution studies unfeasible.
However, only long-range gravitational interactions tend to be important
for CMD, so we can utilize the \emph{tree method} to compute the
gravitational force vastly more efficiently.

The first gravitational tree code was introduced in
\citet{barnes+hut1986}, and is still in use today. We utilize the
massively parallel code \emph{Gadget 4} \citep{gadget4}. Particles are
spatially split into an \emph{octotree}. The tree construction stars
with one large node, a box containing all of the particles. If there is
more than one particle in a box/node , the box is then divided into 8
more nodes (halving the side length in each dimension) and this step is
repeated until each node only contains 1 particle. With this
heirarchichal organization, if a particle is sufficiently far away from
a node, then the force is well approximated by the force from the centre
of mass of the node. As such, each force calculation only requires a
walk through the tree, only descending farther into the tree as
necessary to retain accuracy. The total force calculations reduce from
\(N^2\) to \(O(N\,\log N)\), representing orders of magnitude speedup.
Modern codes such as \emph{Gadget} utilize many other performance
tricks, such as splitting particles across many supercomputer nodes,
efficient memory storage, adaptive time stepping, and parallel file
writing to retain fast performance for large scale simulations, forming
the foundation for many cosmological simulation codes.

\subsection{Galactic tides and idealized
simulations}\label{galactic-tides-and-idealized-simulations}

Despite the power of cosmological simulations for understanding large
scale structure, a fundamental limitation is their ability to resolve
small scales such as dwarf galaxies. Dwarf galaxies, without sufficient
resolution, are likely to artificially disrupt as a result of numerical
noise. For understanding the tidal evolution of a dwarf galaxy, this
tool has limited accuracy then.

Idealized simulations take a different approach. Instead of starting
from the universe as a whole, just a single subhalo is simulating,
allowing resolution several orders of magnitude than cosmological
simulations. For instance, the highest resolution dark matter only
simulation, the \emph{Aquarious Project} only reached a mass resolution
of XXX. In constrast, our dark matter mass resolution in later chapters
is of order 1000s of solar masses.

As such, a long history of work has resolved to understand the tidal
evolution of dwarf galaxies using these \emph{idealized} simulations.

Some of the earliest theory work on tidal stripping of dwarf galaxies
originate from \citep[\citet{moore+davis1994},
\citet{johnston+spergel+hernquist1995}, \citet{oh+lin+aarseth1995},
\citet{piatek+pryor1995}, \citet{velazquez+white1995},
\citet{kroupa1997}]{allen+richstone1988}. Many of these works used
similar techniques as we continue to use today.

More recently, \citet{PNM2008} run idealized simulations of a local
group dwarf galaxy in a milky way halo.

\begin{itemize}
\tightlist
\item
  \citet{mayer+2001} theory of tidal stripping
\end{itemize}

\subsubsection{Break and tidal radii}\label{break-and-tidal-radii}

Analytic approximations are powerful tools to generally determine the
influence of tides given a dwarf galaxy and a host.

The Jacobi radius represents the approximate radius where stars become
unbound for a galaxy in a circular orbit around a host galaxy.
Calculated from an approximation of the location of the L1 and L2
lagrange points, the Jacobi radius is where the mean density of the
dwarf galaxy is three times the mean interior density of the host galaxy
at pericentre, or \begin{equation}{
3\bar \rho_{\rm MW}(r_{\rm peri}) = \bar \rho_{\rm dwarf}(r_J).
}\end{equation} If \(r_J\) occurs within the visible extent of a galaxy,
we should expect to find unbound, \emph{extratidal} stars. While most
valid for circular orbits, assuming \(r_{\rm peri}\) for the host-dwarf
distance works as most stars are lost near pericentre.

We use the break radius as defined in \citet{penarrubia+2009}, marking
where the galaxy is still in disequilibrium. The break radius
\(r_{\rm break}\) is proportional to the velocity dispersion
\(\sigma_v\) and time since pericentre \(\Delta t\), \begin{equation}{
r_{\rm break} = C\,\sigma_{v}\,\Delta t
}\end{equation} where the scaling constant \(C \approx 0.55\) is a fit.
\(r_{\rm break}\) describes where the dynamical timescale is longer than
the time since the perturbation, i.e.~the radius within which the galaxy
should have dynamically relaxed. As illustrated in
Fig.~\ref{fig:toy_profiles}, for an idealized model with exponential
stars in a NFW halo, after some time past pericentre, the stellar
component is smooth but contains a change in slope around
\(r_{\rm break}\), where the radial velocities of the stars becomes
positive as they still readjust to a new equilibrium.

From this argument, we note that the following properties must be
approximately true for tides to occur:

\begin{itemize}
\tightlist
\item
  Close enough pericentre. The other break radius \(r_J\) implies that
  if the host density is 3x the satellite, stars will be lost
\item
  Corresponding time since last pericentre: If the time since last
  pericentre is not \(\sim\) consistent with an observed break in the
  density profile, then tides
\item
  Halo evolution. As found in \citet{EN2021}, galaxies evolve along well
  defined tidal tracks (assuming spherical, isotropic, NFW halo, which
  may not be true, see \ldots). These tracks tend to ``puff up'' the
  stellar component while also removing dark matter mass, leaving a
  smaller, compacter DM halo with a more extended stellar component.

  \begin{itemize}
  \tightlist
  \item
    This information is mostly related to the statistical initial
    distribution of satellites from cosmology {[}ludlow+2016;
    \citet{fattahi+2018}{]}
  \end{itemize}
\end{itemize}

\begin{figure}
\centering
\pandocbounded{\includegraphics[keepaspectratio]{/Users/daniel/thesis/figures/idealized_break_radius.png}}
\caption[Break radius validation]{The break radius of the simulations is
set by the time since pericentre. The initial conditions are
Sculptor-like, exponential stars embedded in NFW, evolved in
\citet{EP2020} potential with a pericentre of 15 kpc and apocentre of
100 kpc. See Section~\ref{sec:methods} for a description of our
simulation setup. \textbf{TODO:} mark \(r_J\), plot initial, make sure
1st pericentre, simplify models.}\label{fig:idealized_break_radius}
\end{figure}

\subsubsection{Tidal evolution}\label{tidal-evolution}

While the break and jacobi radii help with understanding when tidal
effects matter, we still need to know what these effects do and how a
galaxy evolves.

As a galaxy evolves in a tidal field, several changes happen.

\begin{enumerate}
\def\labelenumi{\arabic{enumi}.}
\tightlist
\item
  \emph{Mass is lost}. In particular, particles and stars on weakly
  bound orbits are most likely to be removed by tides.
\item
  \emph{Steams form}. If tides are strong enough, mass lost from the
  galaxy becomes part of tidal tails, evolving along a similar orbit but
  leading or trailing the galaxy.
\item
  \emph{The galaxy reshapes}. Bound mass of the galaxy redistributes. As
  mentioned in the last section, this is visible as a wave of outward
  moving material, with the outermost material reaching equilibrium
  last.
\item
  \emph{A new equilibrium}. With mass loss, the gravitational potential
  decreases, resulting in a more compact dark matter halo and stars
  which adiobadically expand to a larger scale radius.
\end{enumerate}

Assuming that galaxies are spherical, isotropic, and evolve in a
constant tidal field, most dwarf galaxies should evolve along similar
tidal tracks.

\citet{EN2021} derive tidal tracks for dwarf galaxies, showing that NFW
halos all evolve along a similar trajectory in terms of \(r_{\rm max}\)
and \(v_{\rm max}\). \begin{equation}{
\frac{v_{\rm max}}{v_{\rm max, 0}} = 
2^\alpha 
\left(\frac{r_{\rm max}}{r_{\rm max, 0}}\right)^{\beta}\left[1 + \left(r_{\rm max} / r_{\rm max, 0}\right)^2\right]^{-\beta}
}\end{equation} \{eq:tidal\_track\} where \(\alpha=0.4\), \(\beta=0.65\)
are empirical fits. As illustrated in Fig.~\ref{fig:tidal_tracks}, this
formula works for both circular and elliptical orbits and is independent
of the initial subhalo size or distance to the host.

\begin{figure}
\centering
\pandocbounded{\includegraphics[keepaspectratio]{/Users/daniel/Library/Application Support/typora-user-images/image-20250715095615423.png}}
\caption[Tidal tracks of dwarf galaxies]{Tidal tracks of dwarf galaxies.
Fig. 6 from \citet{EN2021}.}\label{fig:tidal_tracks}
\end{figure}

\subsection{Notable related works}\label{notable-related-works}

Case studies:

\begin{itemize}
\tightlist
\item
  \citet{dicintio+2024}
\item
  \citet{iorio+2019}
\end{itemize}

Dynamical

\begin{itemize}
\tightlist
\item
  \citet{read+2006}
\item
  \citet{bullock+johnston+2005}
\item
  \citet{PNM2008}
\item
  \citet{borukhovetskaya+2022a}, \citet{borukhovetskaya+2022}
\item
  \citet{errani+2023a}, \citet{fattahi+2018}
\item
  \citet{wang+2017},
\item
  \citet{pryor1996}
\item
  \citet{klimentowski+2009}
\end{itemize}

\section{Thesis Outline}\label{thesis-outline}

In chapter 2, we describe how we compute observational density profiles
from \citet{jensen+202}. In chapter 3, we review our simulation methods.
Next, we present our results for the tidal stripping of Sculptor and
Ursa Minor in Chapter 4. We discuss our results, limitations, and
implications in Chapter 5. Finally, in Chapter 6, we summarize this work
and discuss future directions for similar work and the field of dwarf
galaxies.
