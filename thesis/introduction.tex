\subsection{Background \& Past Work}\label{background-past-work}

\begin{itemize}
\tightlist
\item
  What is dark matter? Why do we look at dwarfs?
\item
  Forms of dark matter, lambda-CDM, and dwarf galaxies
\item
  How does gravity affect dwarfs, theory of tidal perturbations

  \begin{itemize}
  \tightlist
  \item
    EN21, peñarrubia+09, etc.
  \end{itemize}
\item
  Instances of dwarfs undergoing weird processess
\item
  Alternative processes and uncertainties in the evolution of dwarfs
\end{itemize}

\subsection{Gaia Membership Selection}\label{gaia-membership-selection}

\begin{figure}
\centering
\includegraphics{figures/scl_selection.png}
\caption{Sculptor selection criteria.}\label{fig:sculptor_selection}
\end{figure}

We use J+24 data, updated PM from the MV 2020a catalogue.

J+24 select members using a multi-component Baysian algorithm.

\begin{itemize}
\tightlist
\item
  First, stars with poor Gaia astrometry, color excess nearby
  parallaxes, and proper motions \textgreater{} 10 mas/yr are removed
\item
  Stars are assigned a likelihood based on the location on the CMD
  (assuming)
\end{itemize}

Simple cuts, such as selecting PM, CMD based on purely geometric cuts
reveal a similar density profile.

One caveat is the original algorithm takes spatial position into
account. When deriving a density profile, this assumption may influence
the derived density profile, especially when the galaxy density is
fainter than the background of similar appearing stars. To remidy this
and estimate where the background begins to take over, we also explore a
cut based on the likelihood ratio of only the CMD and PM components.

\subsection{Searches for tidal tails}\label{searches-for-tidal-tails}

\begin{figure}
\centering
\includegraphics{/Users/daniel/Library/Application Support/typora-user-images/image-20250313132102118.png}
\caption{image-20250313132102118}
\end{figure}

\subsection{Density Profile Reliability and
Uncertainties}\label{density-profile-reliability-and-uncertainties}

Given a set of probable members, the density profile is given by
EQUATION, where we use \ldots{}

\begin{itemize}
\tightlist
\item
  Using J+24 data, we validate

  \begin{itemize}
  \tightlist
  \item
    Check that PSAT, magnitude, no-space do not affect density profile
    shape too significantly
  \end{itemize}
\item
  Our ``high quality'' members all have \textgreater{} 50 member stars
  and do not depend too highly on the spatial component, mostly
  corresponding to the classical dwarfs
\item
  We fit Sérsic profiles to each galaxy

  \begin{itemize}
  \tightlist
  \item
    The Sérsic index, \(n\), is a measure of the deviation from an
    exponential. Exponentials have \(n=1\), whereas more extended dwarf
    galaxies will have higher \(n\)
  \end{itemize}
\item
  To better estimate the uncertanties due to unknown galaxy properties
  and flexibility in the likelihood cut, we can
\end{itemize}

\begin{figure}
\centering
\includegraphics{figures/density_profiles_medley.png}
\caption{Density profiles}\label{fig:sculptor_observed_profiles}
\end{figure}

Dark Matter is one of the biggest open questions in modern astronomy.
Hello fig.~\ref{fig:sculptor}.

\subsection{Comparison of the Classical
dwarfs}\label{comparison-of-the-classical-dwarfs}

\begin{figure}
\centering
\includegraphics{/Users/daniel/Library/Application Support/typora-user-images/image-20250313130050775.png}
\caption{image-20250313130050775}
\end{figure}

\begin{figure}
\centering
\includegraphics{/Users/daniel/Library/Application Support/typora-user-images/image-20250313130110550.png}
\caption{image-20250313130110550}
\end{figure}

\begin{figure}
\centering
\includegraphics{/Users/daniel/Library/Application Support/typora-user-images/image-20250313130043114.png}
\caption{image-20250313130043114}
\end{figure}

\subsection{Summary}\label{summary}

\begin{itemize}
\tightlist
\item
  Of the classical dwarfs, UMi \& Scl stand out statistically, with high
  \(n\) given their luminosity
\item
  Including fainter dwarf galaxies, Boo 3 and Boo 1 appear to also have
  extended density profiles

  \begin{itemize}
  \tightlist
  \item
    Deeper data would be required to robustly measure this
  \end{itemize}
\end{itemize}
