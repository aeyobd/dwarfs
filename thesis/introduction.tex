\section{Background \& Past Work}\label{background-past-work}

\begin{itemize}
\tightlist
\item
  What is dark matter? Why do we look at dwarfs?
\item
  Forms of dark matter, lambda-CDM, and dwarf galaxies
\item
  How does gravity affect dwarfs, theory of tidal perturbations

  \begin{itemize}
  \tightlist
  \item
    \citet{EN2021}, \citet{PNM2008}, etc.
  \end{itemize}
\item
  Instances of dwarfs undergoing weird processess
\item
  Alternative processes and uncertainties in the evolution of dwarfs
\end{itemize}

\subsubsection{Sculptor}\label{sculptor}

\begin{itemize}
\tightlist
\item
  \citet{sestito+2023a}
\item
  \citet{tolstoy+2023}, \citet{arroyo-polonio+2023},
  \citet{arroyo-polonio+2024}
\item
  \citet{battaglia+2008}
\item
  \citet{iorio+2019}
\end{itemize}

\begin{table*}[t]
\centering
\caption{Measured properties of Sculptor}
\label{tbl:Measured-properties-of-Sculptor}
\begin{tabular}{lll}
\toprule
parameter & value & Source\\
\midrule
$\alpha$ & $15.0183 \pm 0.0012$˚ & M+18\\
$\delta$ & $-33.7186 \pm 0.00072$˚ & M+18\\
distance & $83.2 \pm 2$ kpc & Tran+22\\
$\mu_\alpha \cos \delta$ & $0.099 \pm 0.002 \pm 0.017$ mas yr$^{-1}$ & MV20a\\
$\mu_\delta$ & $-0.160 \pm 0.002_{\rm stat} \pm 0.017_{\rm sys}$ mas yr$^{-1}$ & MV20a\\
RV & $111.03 \pm 0.23$ & This work\\
$\sigma_v$ & $9.61\pm0.16$ & This work\\
$r_h$ & $12.33 \pm 0.05$ arcmin & MV20*\\
ell & $0.36 \pm 0.01$ & M+18\\
PA & $92\pm1$ & M+18\\
$M_V$ & $-10.82\pm0.14$ & M+18\\
$\Upsilon_\star$ & $1.5 \pm 0.3$ & assumed\\
\bottomrule
\end{tabular}
\end{table*}

\subsection{Gaia Membership Selection}\label{gaia-membership-selection}

\begin{figure}
\centering
\includegraphics{figures/scl_selection.pdf}
\caption[Sculptor selection criteria]{The selection criteria for Scl
members. Members are red and all field stars (satisfying quality
criteria) are in light grey. \textbf{Top left:} tangent plane.
\textbf{Top Right:} Colour magnitude
diagram.}\label{fig:sculptor_selection}
\end{figure}

We use J+24 data. J+24 select members using a multi-component Baysian
algorithm:

\begin{itemize}
\item
  Remove stars with poor astrometry or photometry, no colour excess
  (\citet{lindegren+2018} equation C.2), 3\(\sigma\) consistency of
  measured parallax with dwarf distance (near zero with
  \citet{lindegren+2018} zero-point correction), and absolute RA and Dec
  proper motions less than 10\(\,{\rm mas\ yr^{-1}}\).
\item
  Spatial likelihood based on a double exponential
  \(\Sigma_\star \propto e^{-r/r_s} + B\,e^{-r/r_{\rm outer}}\) where
  the inner scale radius is fixed.
\item
  Stars are assigned a likelihood based on the location on the CMD
  (using Padova isochrones including an intrinsic 0.1 CMD width in
  colour convolved with colour and distance modulus)
\item
  Background KDE density maps for the CMD and PM are constructed using
  the other quality-selected stars outside of \(5R_h\), where the
  satellite density would be orders of magnitude less than the
  background (even in the presence of extended tidal features).
\item
  Likelihoods normalized to unity to represnt a PDF

  Membership probabilities are then given by
\end{itemize}

\[
P_{\rm sat} = \frac{f_{\rm sat}{\cal L}_{\rm sat}}{f_{\rm sat}{\cal L}_{\rm sat} + (1-f_{\rm sat}){\cal L}_{\rm bg}} = \frac{1}{1 + \frac{(1-f_{\rm sat}){\cal L}_{\rm bg}}{f_{\rm sat}{\cal L}_{\rm sat} }}
\]

where \(f_{\rm sat}\) is the fraction of stars belonging to the system
inside the given field, \({\cal L}_{\rm sat}\) is the likelihood of a
star belonging to the satellite, and \({\cal L}_{\rm bg}\) is the
likelihood of the star belonging to the background. Each likelihood is
calculated as a product of the CMD, PM, and spatial likelihoods: \[
{\cal L} = {\cal L}_{\rm space}\ {\cal L}_{\rm PM}\ {\cal L}_{\rm CMD}
\]

The above formula suggests that a cut in \(P_{\rm sat}\) is equivalent
to the cut in likelihoods \[
\frac{{\cal L}_{\rm sat}}{{\cal L}_{\rm bg}} > \frac{(1-f_{\rm sat})/f_{\rm sat}}{1/P_{\rm sat}- 1}
\]

Note that if we remove the spatial component of the likelihood, then
\(f_{\rm sat}\) represents a global normalization.

Not shown here, we explore simple cuts of the stars, using absolute cuts
in parallax, proper motions, and the CMD. The results are similar to the
nospace model.

\subsubsection{Searches for tidal tails}\label{searches-for-tidal-tails}

\begin{figure}
\centering
\includegraphics{figures/scl_tidal_tails.pdf}
\caption[Tidal tails]{The distribution of member stars (orange), PM \&
CMD only selected stars (blue) and all stars (passing quality cuts,
black).}\label{fig:sculptor_tidal_tails}
\end{figure}

\begin{itemize}
\item
  There are no apparent overdensities in the PM \& CMD only selected
  stars to suggest the presence of a tidal tail
\item
  This means that at least at the level of where the background density
  dominates, we can exclude models which produce tidal tails brighter
  than a density of
  \(\Sigma_\star \approx 10^{-2}\,\text{Gaia-stars\ arcmin}^{-2} \approx 10^{-6} \, {\rm M_\odot\ kpc^{-2}}\)
  (TODO assuming a distance of \ldots{} and stellar mass of \ldots).
\end{itemize}

\subsection{Density Profile Reliability and
Uncertainties}\label{density-profile-reliability-and-uncertainties}

\begin{itemize}
\tightlist
\item
  How well do we know the density profiles?
\item
  What uncertainties affect derived density profiles?
\item
  Can we determine if Gaia, structural, or algorithmic systematics
  introduce important errors in derived density profiles?
\end{itemize}

J+24's algorithm takes spatial position into account, assuming either a
one or two component exponential density profile. When deriving a
density profile, this assumption may influence the derived density
profile, especially when the galaxy density is fainter than the
background of similar appearing stars. To remedy this and estimate where
the background begins to take over, we also explore a cut based on the
likelihood ratio of only the CMD and PM components. This is in essence
assuming that the spatial position of a star contains no information on
it's membership probability (a uniform distribution like the background)

To incorporate the structural uncertainties and robustly model the
sampling uncertainty, we construct the following bootstrap model

\begin{itemize}
\item
  Centre is varied by a centring error, estimated from the standard
  normal error of the positions plus the systematic shift of the mean
\item
  Position angle and ellipticity are sampled from a normal distribution
  given the reported uncertainties
\item
  \(f_{\rm sat}\) is sampled from \ldots{}

  \textbf{TODO}: Look into normalization of Likelihoods and check how
  \(f_{\rm sat}\) matters. If \(f_{\rm sat}\) is related to the
  normalizations of fg / bg densities, and other likelihoods are
  area-normalized to 1, then this makes life much easier.

  Test if psat weighted density profiles are similar

  Save MC density profile outputs
\end{itemize}

\begin{figure}
\centering
\includegraphics{figures/scl_density_methods.pdf}
\caption{Density profiles}\label{fig:sculptor_observed_profiles}
\end{figure}

Figure:

\section{Comparison of the Classical
dwarfs}\label{comparison-of-the-classical-dwarfs}

\begin{itemize}
\tightlist
\item
  Using J+24 data, we validate

  \begin{itemize}
  \tightlist
  \item
    Check that PSAT, magnitude, no-space do not affect density profile
    shape too significantly
  \end{itemize}
\item
  Our ``high quality'' members all have \textgreater{} 50 member stars
  and do not depend too highly on the spatial component, mostly
  corresponding to the classical dwarfs
\item
  We fit Sérsic profiles to each galaxy

  \begin{itemize}
  \tightlist
  \item
    The Sérsic index, \(n\), is a measure of the deviation from an
    exponential. Exponentials have \(n=1\), whereas more extended dwarf
    galaxies will have higher \(n\)
  \end{itemize}
\item
  To better estimate the uncertainties due to unknown galaxy properties
  and flexibility in the likelihood cut, we can
\end{itemize}

\begin{figure}
\centering
\includegraphics{figures/classical_dwarf_profiles.pdf}
\caption[Sculptor and UMi versus classical dwarfs]{Density profiles for
each dwarf galaxy.}
\end{figure}

\textbf{TODO}: Use updated density profiles (nospace) with uncertainties
included to MCMC fit Sérsic profiles to every dwarf galaxy.

\begin{table*}[t]
\centering
\caption{Sérsic profile fits and basic structural parameters for classical and bright dwarfs around the MW.}
\label{tbl:Sérsic-profile-fits-and-basic-structural-parameters-for-classical-and-bright-dwarfs-around-the-MW}
\begin{tabular}{llllll}
\toprule
galaxy & Ell. & Sersic Rh & sersic n & Sersic mass & \\
\midrule
Fornax &  &  &  &  & \\
Leo I &  &  &  &  & \\
Sculptor &  &  &  &  & \\
Leo II &  &  &  &  & \\
Carina &  &  &  &  & \\
Sextans I &  &  &  &  & \\
Ursa Minor &  &  &  &  & \\
Draco &  &  &  &  & \\
Canes Venatici I &  &  &  &  & \\
Crater II &  &  &  &  & \\
Bootes I &  &  &  &  & \\
Bootes III &  &  &  &  & \\
Antlia II &  &  &  &  & \\
\bottomrule
\end{tabular}
\end{table*}

Cannot resolve

\begin{itemize}
\tightlist
\item
  reticulum II yyy
\end{itemize}

\subsection{Radial Velocity
Measurements}\label{radial-velocity-measurements}

\section{Simulation-based
motivations}\label{simulation-based-motivations}

To motivate why a tidal interaction may give rise to the observed
density profiles, we create a toy simulation following \citet{PNM2008}.

\begin{itemize}
\item
  NFW initial conditions (sculptor like, vcm, rcm)
\item
  Evolved in x-y plane using \citet{EP2020} potential for
  \textasciitilde{} 5Gyr with pericentre of 15 kpc and apocentre of 100
  kpc.
\item
  Exponential initial stellar profile.
\end{itemize}

As a dark matter halo is perturbed on a pericentric passage with the
milky way,

\begin{itemize}
\tightlist
\item
  Tidal stress heats halo slightly
\item
  Mass loss, particularly of loosely bound particles
\end{itemize}

The stellar component tracers will similarly follow the behaviour of the
dark matter.

An emperical estimate of where the simulation's stars are becoming
unbound is, as stated in \citet{PNM2008}, the break radius \[
R_b = C\,\sigma_{v}\,\Delta t
\] where \(\sigma_v\) is the present line of sight velocity dispersion ,
\(\delta t\) is the time since pericentre, and \(C \approx 0.55\) is a
fit. The idea motivating this equation is stars in the inner regions
will have dynamically equilibriated to the new potential (phase mixed),
however the outer regions are no longer in steady state, so we have to
wait until the crossing time reaches them as well.

As illustrated in fig.~\ref{fig:toy_profiles}, the density profile
initially stars off exponential. At increasing times since the first
pericentric passage, the break radius, appearing as an apparent
separation between the slopes of the inner and outer profile, increases.

\begin{figure}
\centering
\includegraphics{figures/scl_umi_vs_idealized.pdf}
\caption[Idealized simulations match Scl and UMi]{Sculptor and UMi's
profiles are well-matched to an idealized
simulation}\label{fig:toy_profiles}
\end{figure}

\begin{figure}
\centering
\includegraphics{figures/idealized_break_radius.pdf}
\caption[Break radius validation]{The break radius of the simulations is
set by}\label{fig:idealized_break_radius}
\end{figure}

From this argument, we note that the following properties must be
approximately true for tides to occur:

\begin{itemize}
\tightlist
\item
  Close enough pericentre. The other break radius \(r_J\) implies that
  if the host density is 3x the satellite, stars will be lost
\item
  Corresponding time since last pericentre: If the time since last
  pericentre is not \(\sim\) consistent with an observed break in the
  density profile, then tides
\item
  Halo evolution. As found in \citet{EN2021}, galaxies evolve along well
  defined tidal tracks (assuming spherical, isotropic, NFW halo, which
  may not be true, see \ldots). These tracks tend to ``puff up'' the
  stellar component while also removing dark matter mass, leaving a
  smaller, compacter DM halo with a more extended stellar component.

  \begin{itemize}
  \tightlist
  \item
    This information is mostly related to the statistical initial
    distribution of satellites from cosmology {[}ludlow+2016;
    \citet{fattahi+2018}{]}
  \end{itemize}
\end{itemize}

\section{Summary}\label{summary}

\begin{itemize}
\tightlist
\item
  Of the classical dwarfs, UMi \& Scl stand out statistically, with high
  \(n\) given their luminosity
\item
  Including fainter dwarf galaxies, Boo 3 and Boo 1 appear to also have
  extended density profiles

  \begin{itemize}
  \tightlist
  \item
    Deeper data would be required to robustly measure this
  \end{itemize}
\end{itemize}
