Dwarf galaxies host, in many ways, the most extreme galactic
environments in the universe. These galaxies are typically defined to be
fainter than the Large Magellanic Cloud (LMC), with \(M_V \gtrsim -18\)
or similarly \(M_\star \lesssim 10^9 M_\odot\) {[}e.g.,
\citet{hodge1971}; \citet{mcconnachie2012}). Because the galaxy
luminosity function increases towards fainter objects, dwarfs are the
most numerous galaxies in the Universe. Dwarf galaxies are also highly
\emph{dark-matter dominated}, with mass to light ratios which may exceed
1000 \(M_\odot/ L_\odot\). With the exception of the Magellanic Cloud,
all dwarf galaxy satellites of the Milky Way (MW) are \emph{quenched},
with little to no recent star formation. Indeed, most faint MW
satellites contain stellar populations which are \emph{relics} from the
early universe, consisting of many of the oldest and most metal poor
stars.

Understanding the properties of dwarf galaxies thus has implications
across astronomy, from cosmological structure formation on small scales
to the formation of metal-poor stellar populations.

\emph{Remove this paragraph?} Of the classical dwarfs, we find that the
Sculptor and Ursa Minor dwarf spheroidal galaxies stand out with a more
extended density profile relative to an exponential, hinting at tidal
effects. As a case study in tidal effects and our understanding of dwarf
galaxy formation and evolution, we aim to understand the dynamical
history of these galaxies.

In this section, we first describe the general observed properties of
local dwarf galaxies. Next, we summarize the cosmological origin and
role of dwarf galaxies. We then review recent updates to the observed
properties of dwarf galaxies made on new facilities. We then discuss how
tides influence satellite galaxies in general. Finally, we provide a
roadmap to the thesis. \textbf{Look at this paragraph again}

\section{Observational history of dwarf
galaxies}\label{observational-history-of-dwarf-galaxies}

Dwarf galaxies have long raised conundrums for theories of galaxy
formation. The discovery of Fornax and Sculptor in 1938
\citep{shapley1938}\footnote{Technically, the Large and Small Magellanic
  Clouds (LMC, SMC) are also classified as dwarf galaxies, but these
  were likely always known to humans at southern latitudes.}, with no
known analogues, already presented such an enigma. Shapley presented
these dwarfs as a new type of \emph{stellar system} resembling the
Magellanic Clouds and globular clusters but did not attempt to speculate
on their exact nature. While dwarf galaxies were quickly understood to
be galaxies based on the inferred luminosities and sizes, their nature
and formation remained unclear for decades
\citep[e.g.,][]{hodge1971, gallagher+wyse1994}.

The earliest spectroscopic work hinted that dwarf galaxies may contain
substantial amounts of dark matter. From early determinations of the
velocity dispersion for the Sculptor and Ursa Minor dwarf spheroidal
(dSph) galaxies \citep[e.g.,][]{aaronson1983, aaronson+olszewski1987},
inferred mass-to-light ratios were at least 10 times larger than the
values of globular clusters (GCs). While rather uncertain initially,
these values have been corroborated with larger and more precise samples
\citep[e.g.,][]{hargreaves+1994}. At the time, several theories were
proposed to explain these unusually high mass-to-light ratios. Examples
include: ongoing tidal disruption artificially inflating the
line-of-sight velocity dispersions \citep[e.g.,][]{kuhn+miller1989}, the
presence of massive central black holes
\citep[e.g.,][]{strobel+lake1994}, or modified theories of gravity
\citep{milgrom1995}. Over time, however, a consensus developed where the
high mass-to-light ratios of dwarf galaxies was due to the presence of a
dark matter halo, which was lacking in GCs
\citep[e.g.,][\textbf{REFS}]{dekel+silk1986}. Since then, the properties
of dwarf galaxies have played an increasingly important role in our
understanding of the clustering of dark matter on small scales {[}e.g.,
sales+2022, bullock+boylan-kolchyn2017{]}.

Dwarf galaxies span a large range of physical sizes, luminosities, and
morphologies. Broadly, there are three classes of dwarf galaxies based
on luminosity, as illustrated by Fig.~\ref{fig:galaxy_images}. Local
\textbf{bright dwarfs} with magnitudes
\(-18 \lesssim M_V \lesssim  -14\), often exhibit irregular morphologies
and recent star formation. Fig.~\ref{fig:galaxy_images} shows the Large
Magellanic Cloud (LMC) as an example of an irregular, bright dwarf
galaxy, where most stars are in a rotationally-supported thin disk (seen
nearly face-on) with a prominent bar. \textbf{Classical dwarf
spheroidals} occupy intermediate luminosities (
\(-14 \lesssim M_V  \lesssim -7.7\)). Typically, these systems are old,
non-star forming, gas-poor, and spheroidal. All Milky Way satellites
discovered before digital sky surveys are classicals. The 12 classical
dwarf satellites of our Galaxy are (in order of decreasing luminosity)
Sagittarius, Fornax, Leo I, Sculptor, Antlia II, Leo II, Carina, Draco,
Ursa Minor, Canes Venatici I, Sextans I, and Crater II.\footnote{While
  formally the dwarf galaxy names we discuss contain ``dwarf
  spheroidal'' (dSph), e.g.~Sculptor dSph, we omit this suffix for
  brevity.} The \textbf{ultra-faint}s dwarfs (UFDs) occupy the very
faintest magnitude range (\(-7.7 \lesssim M_V\)). These galaxies have
minuscule stellar masses, tend to be more compact, and include the
faintest known galaxies (\textbf{ref}). Altogether, known dwarf galaxies
span more than 15 orders in magnitude, or over 7 decades in stellar
mass. \textbf{Rewrite to include stellar mass ranges as well}

A common definition today for a dwarf galaxy is a stellar system with
evidence for dark matter (high mass-to-light ratios, e.g.~\textbf{REF}).
In contrast, star clusters (like GCs) have no clear evidence for dark
matter.

\begin{figure}
\centering
\includegraphics[width=5.41667in,height=5.41667in]{figures/galaxy_pictures.png}
\caption[Dwarf Galaxy Pictures]{Images of the LMC (DSS2), Fornax (DES
DR2), Sculptor (DES DR2), and Ursa Minor (UNWISE with Gaia point sources
over-plotted). Each image includes the galaxy's luminosity and a 200 pc
scale bar. \emph{TODO: Mark half-light radius with
circle?}}\label{fig:galaxy_images}
\end{figure}

\section{Cosmological context}\label{cosmological-context}

We only understand a tiny fraction of the universe's composition. The
leading theory of cosmology, \LCDM (Lambda Cold Dark Matter), posits
that the universe is composed of about 68\% dark energy (\(\Lambda\)),
27\% dark matter (DM), and 5\% regular baryons\footnote{Astronomers like
  to change definitions of words. \emph{Baryons} here means
  baryons+leptons, i.e.~any standard model massive fermion.}
\citep{planckcollaboration+2020}. While the composition of dark matter
and dark energy remains elusive, we know their general properties. Dark
energy causes the acceleration of the expansion of the universe on large
scales. We do not discuss dark energy here---it does not substantially
affect the Local Group today. Dark matter, instead, makes up the vast
majority of mass in galaxies. Typically, galaxies have baryonic to dark
matter ratios of between 1:5 to beyond 1:1000 for faint dwarf galaxies.
In \LCDM, dark matter is assumed to interact only gravitationally. Light
passes through dark matter unimpeded---in this sense dark matter is
transparent. Dark matter is also commonly assumed to be \emph{cold},
i.e.~typical velocities much smaller than the speed of light in the
early universe. If dark matter is cold, then it should condense on all
scales, forming non-linear structures (or \emph{halos}) from the size of
galaxy clusters to smaller than the faintest dwarf galaxies.
Implications of dark matter properties range from cosmological
structural formation, galaxy structure, and galaxy interactions.

\textbf{is this ordered logically?}

\subsection{\texorpdfstring{Dwarf galaxy formation in
\LCDM}{Dwarf galaxy formation in }}\label{dwarf-galaxy-formation-in}

\textbf{In the next section, we consider galaxy formatin in LCDM}

The very early universe was almost featureless. Our earliest
observations of the universe stem from the cosmic microwave background
(CMB)---revealing a uniform, isotropic, near-perfect blackbody emission.
But tiny perturbations in the CMB, temperature fluctuations of 1 part in
10,000, reveal the underlying seeds of large-scale cosmological
structure. In an expanding universe, gravitational instability makes CDM
overdensities grow and collapse hierarchically onto larger structures.
Initially, baryonic matter was coupled to radiation and resisted
collapse. Dark matter, only influenced by gravity instead, freely
collapsed into the first structures. For mass perturbations both smaller
than the horizon and overdense enough to collapse, these overdensities
of dark matter become self-gravitating, known as \emph{halos}. After
recombination, where electrons combined with atomic nuclei to form
atoms, baryons decoupled from radiation and fell into the dark matter
halos. The densest pockets of baryons later formed the first stars and
galaxies.

Dark matter halos, and their associated galaxies, rarely evolve in
isolation. Instead, structure formation is \emph{hierarchical}. Small
dark matter halos collapse first and hierarchically merge into
progressively larger halos
\citep[e.g.,][]{blumenthal+1984, white+rees1978, white+frenk1991}.
Structure formation happens on a wide range of scales. The largest
structures become the cosmic web---composed of voids, filaments, and
sheets---and the smallest non-linear structures (halos, \textbf{define
non-linear}) host dwarf galaxies. Hierarchical assembly is evident
through the large scale structure of the universe, remnants of past
mergers within the Milky Way, and tidal disruption of dwarf galaxies and
their streams around nearby galaxies.

Small-scale structure formation is sensitive to deviations from
\LCDM cosmology \citep[e.g.,][]{bechtol+2022}. One key prediction of
\LCDM is that mass perturbations are expected to exist on all scales,
and are largest on the smallest scales, so we would expect the formation
of halos on all scales. Many alternative models, such as warm dark
matter, may smooth out small-scale features and reduce the abundance of
small halos or change their structure \citep[e.g.,][]{lovell+2014}.
Dwarf galaxies, which occupy the smallest dark matter halos, are
promising windows into small-scale structures. To understand the
evolution of these objects, we need to understand the general
predictions of \LCDM for the properties of dwarf galaxies.

\subsection{Typical structure of dark matter
halos}\label{typical-structure-of-dark-matter-halos}

In \LCDM cosmological simulations, dark matter halos are remarkably
self-similar. In \citet{NFW1996, NFW1997}, hereafter NFW, the authors
observe that the spherically-averaged density profiles \(\rho(r)\) are
universally well described by a two-parameter law:
\begin{equation}\protect\phantomsection\label{eq:nfw}{
\rho/\rho_s= \frac{1}{(r/r_s)(1+r/r_s)^2},
}\end{equation} where \(r_s\) is a scale radius and \(\rho_s\) a scale
density . This profile has shown remarkable success in describing
\LCDM halos across several orders of magnitude in mass. NFW profiles are
\emph{cuspy}, where the density continuously rises like
\(\rho \sim 1/r\) at small radii \(r \ll r_s\). The steepness of the
density profile increases gradually with radius, and at large radii the
density falls off like \(\rho \sim 1/r^3\). Fig.~\ref{fig:nfw_density}
shows an example NFW halo.

The total mass of an NFW profile formally diverges, so halo masses are
commonly defined using an overdensity criterion. The virial mass,
\(M_{200}\), is defined as the mass within a radius, \(r_{200}\),
containing a mean enclosed density 200 times\footnote{For the collapse
  of a uniform spherical density, the virialized overdensity would be
  \(\Delta = 18\pi^2\approx 178\) for a critical universe
  \(\Omega_m = 1\). This is commonly rounded to \(\Delta = 200\). While
  this parameter may be closer to \(\Delta \approx 100\) for our
  universe, \(\Delta\) also increases with redshift \citep[using eq. 6
  from][]{bryan+norman1998}.} the critical density of the universe:
\begin{equation}{
M_{200} =200\,\frac{4\pi}{3} \ r_{200}^3\ \rho_{\rm crit}, \qquad {\rm where} \quad \rho_{\rm crit}(z) = 3H(z)^2 / 8\pi G
}\end{equation} A standard second parameter of NFW halos is the halo
concentration, \(c=r_{200} / r_s\) which describes how the
characteristic radius scale of the halo compares to the virial radius.
Using this parameter, the scale density is a function of \(c\) alone,
\(\rho_s = (200/3)\,\rho_{\rm crit} c^3 / [\log(1+c) - c/(1+c)]\)
\citep{NFW1996}.

An equivalent, alternative characterization of NFW halos uses their
circular velocity profiles. The circular velocity,
\(\V_{\rm circ}(r) = \sqrt{G M(r) / r}\), reaches a maximum
\(\V_{\rm max}\) at radius \(r_{\rm max} \approx 2.16258\,r_s\).
\(\V_{\rm max}\) and \(r_{\rm max}\), like \(M_{200}\) and \(c\), fully
specify an NFW halo,

The two parameters of an NFW profile are not independent. Lower mass
dark matter halos often collapse earlier, when the universe was denser.
As a result, low mass subhalos tend to be more concentrated
\citep[e.g.,][]{NFW1997}. The relationship between \(M_{200}\) and c, or
the mass-concentration relation, describes the mean trend of
concentration with mass \citep[e.g.,][]{bullock+2001, ludlow+2016}. The
left panel of Fig.~\ref{fig:smhm} illustrates the present-day
mass-concentration relation in terms of \(r_{\rm max}\) and
\(\V_{\rm max}\). While there is a general trend for concentration to
increase with mass, the concentration values still have substantial
scatter. Other parameters such as the halo spin or shape may affect the
scatter of the mass-concentration relation, but their effect is
typically expected to be small
\citep{navarro+2010, dicintio+2013, dutton+maccio2014}.

\begin{figure}
\centering
\pandocbounded{\includegraphics[keepaspectratio]{figures/example_density_profiles.pdf}}
\caption[Example density profiles]{Density profiles in log density
versus log radius for stars and dark matter in a Sculptor-like galaxy.
The dark matter is more extended and massive than the star across the
entire galaxy.}\label{fig:nfw_density}
\end{figure}

\begin{figure}
\centering
\pandocbounded{\includegraphics[keepaspectratio]{figures/cosmological_means.pdf}}
\caption[Stellar-mass halo-mass relation]{\textbf{Left} Radius of
maximum circular velocity \(r_{\rm max}\) as a function of maximum
circular velocity. The solid line with 1-\(\sigma\) shaded region and
dashed line are the relations from \citet{ludlow+2016} for \(z=0\) and
\(z=2\) respectively. \textbf{Right} Stellar mass (top) as a function of
maximum circular velocity. The solid line with the 1-\(\sigma\) shaded
region is the relation from \citet{fattahi+2018} with scatter points
simulated central galaxies from APOSTLE in
\citet{fattahi+2018}.}\label{fig:smhm}
\end{figure}

\subsection{\texorpdfstring{Formation of stars within
\LCDM halos}{Formation of stars within halos}}\label{formation-of-stars-within-halos}

The observed abundance of galaxies may be compared with the abundance of
\LCDM halos to derive constraints regarding which galaxies inhabit which
halos. This technique, dubbed ``abundance matching,'' implies a tight
relation between the stellar mass of a galaxy and the mass of the halo
it inhabits \citep{li+white, moster+naab+white2013}.

Fig.~\ref{fig:smhm} shows the stellar mass versus halo mass relation
(SMHM, with halo mass represented by \(\V_{\rm max}\)) predicted by the
\LCDM cosmological hydrodynamical simulations from the the APOSTLE
project \citep{sawala+2016}, which uses the hydrodynamical setup from
the EAGLE project \citep{crain+2015, schaye+2015} to simulate Local
Group analogues in a \LCDM cosmological context \textbf{(db: reorder
these sentences)}. While there is some scatter, the range of predicted
\({\rm v}_{\rm max}\) is fairly narrow across \(\sim 4\) decades in
stellar mass. This figure indicates that the SMHM relation becomes
increasingly steep in the dwarf galaxy regime---many dwarf galaxies are
formed in relatively massive halos. Because lower mass galaxies have
increasingly shallow potential wells, feedback becomes more effective at
removing gas. Re-ionization additionally suppresses late star formation
in the faintest galaxies. As a result, the resulting stellar mass of a
galaxy is highly sensitive to the dark matter mass, especially for faint
dwarf galaxies. If we know the initial stellar mass of a dwarf galaxy,
we also know, at some level, the properties of its dark matter halo.
\textbf{fix this paragraph}

In \LCDM galaxy formation, the majority of mass in dwarf galaxies comes
from Dark Matter. Given the SMHM, Fig.~\ref{fig:nfw_density} shows an
example stellar component for a classical dwarf galaxy with the expected
underlying dark matter halo. Even when the stars are most dense in the
inner regions, the dark matter remains nearly an order of magnitude
higher in density. Stars have a small contribution to the gravitational
structure of dwarf galaxies--instead stars are reasonably approximated
as tracer particles of the underlying dark matter halo. \textbf{add
something about stellar compactness}

Several challenges complicate the SMHM trend, including environment,
assembly history, tidal effects, and the details of galaxy formation. In
particular, effects like ram-pressure stripping (removal of gas in the
dwarf galaxy due to pressure from host's circumgalactic medium) and
tidal removal of gas cause star formation to quench {[}\textbf{REFS}{]}.
Additionally, the time of formation (relative to reionization) can
influence the resulting stellar content \citep{kim+2024}. Finally, tides
influence both the dark matter and stellar mass but in different amounts
\citep[e.g.,][]{PNM2008}. Consequently, tides may reduce the halo mass
more than the stellar mass, adding additional scatter to the SMHM trend,
particularly for low-mass satellites \citep[e.g.,][]{fattahi+2018}.
Understanding the effects of tides on Local Group dwarf galaxies is
essential to determine where and how these galaxies formed in a
cosmological context.

\section{Observing faint features of dwarf
galaxies}\label{observing-faint-features-of-dwarf-galaxies}

\subsection{\texorpdfstring{The era of
\emph{Gaia}}{The era of Gaia}}\label{the-era-of-gaia}

Since Local Group dwarfs are nearby, they are resolved into individual
stars, and we can study these galaxies on a star-by-star basis. As a
result, it is possible to measure the 3D velocity and position of a star
if we can measure the its position, distance, line-of-sight (LOS)
velocity, and proper motions. Unfortunately, determining distances and
full 3D velocities is challenging. The most direct measurement of
distance, parallax, requires precise tracking of a star's sky position
across a year. And while line-of-sight (LOS) velocities are relatively
easily determined from spectroscopy, tangential velocities, derived from
proper motions and distances, are much more challenging. Typically,
measuring proper motions requires accurate (much less than arcsecond)
determinations of small changes in a star's position over baselines of
several years to decades. The full 6D position and velocity information
for stars has, until recently, been known for only a handful of stars.

Launched in 2013, \emph{Gaia} is a space-based, all-sky survey telescope
situated at the Sun-Earth L2 Lagrange point
\citep{gaiacollaboration+2016}. \emph{Gaia} has redefined astrometry,
providing photometry, positions, proper motions, and parallaxes for over
1 billion stars \citep{gaiacollaboration+2021}. While \emph{Gaia}
completed its space-based mission in 2025, two forthcoming data releases
are expected.

Determining absolute parallax measurement is facilitated by the
observation that stars in different regions of the sky are affected by
parallax motion with different phases. By imagining two regions
separated by 106.5 degrees on the same focal plane, \emph{Gaia} measures
tiny changes in relative positions of stars across small and large
angles. Combining measurements from multiple epochs across several
years, an absolute all-sky reference frame is derived from which
parallax and proper motions are derived. In addition to astrometry,
\emph{Gaia} measures photometry in the wide \emph{G} band (330-1050nm)
and colours from the blue photometer (BP, 330-680 nm) and red photometer
(RP, 640-1050 nm). \emph{Gaia} additionally provides low resolution
BP-RP spectra and radial velocity measurements of bright stars (of
magnitudes \(G_{\rm RVS} < 16\)) \citep{gaiacollaboration+2016}. For our
work, \emph{Gaia}'s most relevant measurements are \(G\) magnitude,
\(G_{\rm BP} - G_{\rm RP}\) colour, \((\alpha, \delta)\) position, and
\((\mu_{\alpha*}, \mu_\delta)\) proper motions.\footnote{The proper
  motions \(\mu_\alpha\) and \(\mu_\delta\) are the apparent rates of
  change in right ascension, \(\alpha\), and declination, \(\delta\),
  typically in units of mili-arcsecond (mas) per year.
  \(\mu_{\alpha*} = \mu_\alpha \cos \delta\) corrects for projection
  effects in \(\alpha\).}

\subsection{The Local Group of
Galaxies}\label{the-local-group-of-galaxies}

The Local Group (LG) is defined as galaxies which are within \(\sim 1\)
Mpc from the Milky Way-Andromeda system \citep[e.g.,][ and references
therein]{mcconnachie2012}. \emph{Gaia} has revolutionized LG studies.
For example, the 6D dynamical measurements and metallicities of MW stars
led to the (re)discovery of past mergers or Milky Way building blocks
like \emph{Gaia}-Sausage Enceladus
\citetext{\citealp[e.g.,][]{helmi+2018}; \citealp{belokurov+2018}; \citealp[but
see also][]{mesa+2005}}, out-of-equilibrium structures like the
\emph{Gaia} snail \citep[e.g.,][]{antoja+2018}, and dynamical effects of
the Milky Way's spiral arms and the bar in the solar neighbourhood
\citep{hunt+vasiliev2025}. Moving to the Milky Way halo, \emph{Gaia} has
helped find and constrain numerous stellar streams
\citep[\citet{bonaca+price-whelan2025}]{ibata+malhan+martin2019}.
Altogether, \emph{Gaia} has revealed the hierarchical formation and
complex, evolving structure of our own Galaxy.

For Milky Way satellites, \emph{Gaia} has enabled well-constrained
orbital analysis and facilitated precise stellar membership
determination. Before, proper motions of dwarf galaxies were ill-known
or undetermined. Few galaxies had precisely measured proper motions,
more recently from Hubble Space Telescope observations
\citep[e.g.,][]{sohn+2017}. \emph{Gaia} allowed the first systematic
determinations of Milky Way satellite proper motions of unprecedented
precision \citep{pace+li2019, MV2020a}. While the proper motion
uncertainty of a typical dwarf member star is often large, by combining
the proper motions of 100s or 1000s of stars from \emph{Gaia}, average
precise proper motion measurements can be determined, sometimes only
limited by \emph{Gaia}'s systematic error floor
\citep[e.g.,][]{MV2020a}. Proper motions have thus ushered in a new
dynamical era for MW satellite studies, where we can derive precise
orbits for any satellite, assuming a given MW potential. In addition,
\emph{Gaia} helps establish membership by separate out contaminating MW
foreground stars. By measuring parallaxes and/or proper motions, many
background and foreground stars can be classified as non-members
\citep[e.g.,][\citet{jensen+2024}]{battaglia+2022}.

\begin{figure}
\centering
\includegraphics[width=5.41667in,height=\textheight,keepaspectratio]{figures/mw_satellites_1.jpg}
\caption[Dwarf galaxies sky position]{The location of MW dwarf galaxies
on the sky. We label the classical dwarf galaxies (green diamonds),
fainter dwarfs (blue squares), globular clusters (orange circles), and
ambiguous systems (pink open hexagons). Globular clusters are more
centrally concentrated, but dwarf galaxies are preferentially found away
from the MW disk. Sculptor and Ursa Minor are highlighted as two dwarfs
we study later. The background image is from ESA/Gaia/DPAC
(https://www.esa.int/ESA\_Multimedia/Images/2018/04/Gaia\_s\_sky\_in\_colour2).
Dwarf galaxies (confirmed), globular clusters, and ambiguous systems are
from the \citet{pace2024} catalogue (version
1.0.3).}\label{fig:mw_satellite_system}
\end{figure}

\subsection{Dwarf galaxies today}\label{dwarf-galaxies-today}

Today, we know that the Milky Way system is teeming with satellites.
Fig.~\ref{fig:mw_satellite_system} shows the MW satellite system,
including dwarf galaxies, globular clusters, and ambiguous systems,
whose nature as a galaxy or star cluster remains uncertain. Advances in
observational techniques has accelerated progress in our understanding
of dwarf galaxies and introduced new questions. Deep digital photometric
surveys have more than quadrupled the number of known Milky Way dwarf
galaxies since 2005 \citep{simon2019}. Upcoming surveys, like the Vera
Rubin Observatory's Legacy Survey for Space and Time
\citep[LSST,][]{ivezic+2019}, will likely identify even fainter dwarf
galaxies. In addition, large aperture multi-object spectrographs have
revealed the detailed and complex inner chemodynamical structure of
dwarf galaxies \textbf{REFS}. Beyond precise structural and kinematic
properties of dwarf galaxies, modern observations allow for the
separation of multiple stellar populations, detailed constraints on the
dark matter density profiles, and hints of tidal disruption or stellar
halos.

Of local dwarfs, the classical systems still remain the most studied
galaxies with the best constrained parameters. These dwarfs also contain
large numbers of bright (giant) stars which can be followed up
spectroscopically \citep[e.g.,][]{tolstoy+2023, pace+2020}. As a result,
the determination of fundamental properties such as the position, size,
orientation, distances, proper motions, line-of-sight (LOS) velocities,
and velocity dispersions \(\sigma_\V\), are all relatively well
constrained today.

Observations of dwarf galaxies have been the origin of several disputes
or \emph{small-scale} problems for
\LCDM [see reviews by @bullock+boylan-kolchin2017; sales+2022]. For
example, the mismatch between the expected number of dwarf galaxies from
simulations and the observed abundance has been known as the
\emph{missing satellites problem}. Additionally, a number of
observations suggested that some dwarf galaxies, although not all,
possess dark matter ``cores,''
\citep[e.g.,][]{moore1994, adams+2014, oh+2015, walker+penarrubia2011, read+walker+steger2019},\footnote{In
  detail, gas-phase rotation curves are better able to differentiate
  between cores and cusps, whereas stellar kinematics is less
  constraining.} contrary to the expectation from \LCDM of ``cuspy''
inter dark matter profiles \citep{NFW1996, NFW1997}. As a result,
alternative forms of dark matter have been advocated as solutions, such
as Warm or Self-Interacting Dark Matter.

However, these tensions have eased as a result of improved understanding
of baryonic physics. For example, recent hydrodynamic simulations in
particular have shown that strong feedback can produce dark matter cores
\citetext{\citealp[e.g.,][\citet{tollet+2016}]{navarro+eke+frenk1996}; \citealp{fitts+2017}; \citealp{benitez-llambay+2019}; \citealp{orkney+2021}}.
Altogether, the numerous past challenges for \LCDM in the dwarf galaxy
regime illustrate the opportunity for dwarf galaxies to test the
understanding of galaxy formation and dark matter physics.

\subsection{Dwarf galaxy light profiles}\label{sec:exponential_profiles}

Projected luminosity / stellar density profiles efficiently characterize
the radial structure of a galaxy. At its most basic, light profiles
synthesize properties such as the shape, location, size, and orientation
of a dwarf galaxy. In addition, the details of a stellar density profile
can help interpret a galaxy's assembly and dynamical history
\citep[e.g.,][, \citet{lee+2018}, \citet{querci+2025}]{penarrubia+2009}.
We note that for resolved galaxies, these profiles are in terms of
stellar count density instead of surface brightness.

Four different surface density laws are frequently used to parameterize
dwarf galaxy profiles: Exponential, Plummer, King, or Sérsic profiles
\citep[e.g.,][]{munoz+2018}. The exponential profile is perhaps the
simplest, defined in terms of the central surface density, \(\Sigma_0\),
and scale radius, \(R_s\):
\begin{equation}\protect\phantomsection\label{eq:exponential_law}{
\Sigma_{\rm exp} = \Sigma_0\exp(-R / R_s).
}\end{equation}

This profile is often applied to the radial light distribution of galaxy
disks {[}\citet{freeman1970}; other classic paper?{]}.

To fit globular cluster density profiles, \citet{plummer1911} proposed a
profile based on a self-gravitating polytrope,\footnote{where density
  and pressure are assumed to be related by a power law}
\begin{equation}{
\Sigma_{\rm Pl} = \frac{\Sigma_0}{(1 + (R/R_h)^2)^2},
}\end{equation}

where \(\Sigma_0\) is the central surface density and \(R_h\) is the
projected half-light radius. Now mostly superseded by the King profile
for globular clusters, the Plummer model is still a good fit to many
dwarf spheroidals \citep[e.g.,][]{moskowitz+walker2020}.

The \citet{king1962} profile, also an empirical fit to globular
clusters, is also used to describe dwarf galaxies, especially in older
literature. Using three parameters, a core radius \(R_c\), a truncation
radius \(R_t\), and a characteristic density, \(\Sigma_0\), the King
profile may be written as \begin{equation}{
\Sigma_{\rm K} = \Sigma_0\left(\frac{1}{\sqrt{1 + (R/R_c)^2}} - \frac{1}{\sqrt{1+(R_t/R_c)^2}}\right).
}\end{equation}

In much of the older literature, \(R_t\) was often interpreted as a
``tidal radius'', after an analogous interpretation for globular
clusters \citep[e.g.,][\citet{hodge1961}]{IH1995}.

Finally, the \citet{sersic1963} profile represents a generalization of
an exponential profile, and describes most dwarf galaxy light profiles
well. Typically parameterized in terms of a half-light radius \(R_h\),
the density at half-light radius \(\Sigma_h\) and a Sérsic index \(n\),
the profile's equation is \begin{equation}{
\Sigma_{\rm S} = \Sigma_h \exp\left[-b_n \,  \left((R/R_h)^{1/n} - 1\right)\right]
}\end{equation} where \(b_n\) is a constant that depends on \(n\). A
Sérsic profile with \(n=1\) is equivalent to an exponential profile,
while \(n=4\) recovers \citet{devaucouleurs1948}'s profile for
elliptical galaxies. Although a Sérsic profile is not always used for
dwarf galaxies, \citet{munoz+2018} advocate for the Sérsic profile since
the added flexibility allows for more profiles to be fit with a single
law.

While there are no clear theoretical preferences for any of these
profiles, exponential density profiles are commonly used for dwarf
spheroidal galaxies. \citet{faber+lin1983} were among the first to
demonstrate that an exponential law is a reasonable empirical fit to
dwarf galaxies, theorizing that dwarf spheroidals may have evolved from
exponential disk galaxies and maintained a similar light profile. Later,
\citet{read+gilmore2005} showed that exponential profiles may originate
from mass loss during the evolution of dwarf galaxies. Many subsequent
photometric works for dwarf spheroidal galaxies have used exponential
fits, finding that exponential and King profiles both provide good
descriptions in many cases
\citetext{\citealp[\citet{mateo1998}]{binggeli+sandage+tarenghi1984}; \citealp{mcconnachie+irwin2006}; \citealp{cicuendez+2018}}.
As a result, it has become conventional to assume an exponential density
profile to describe dwarf galaxies in theoretical or observational
modelling \citetext{\citealp[e.g.,][
\citet{MV2020a}]{martin+2016}; \citealp{battaglia+2022}; \citealp[ but
is for disk]{kowalczyk+2013}}.

Outside Local Group dwarf spheroidals, exponential profiles are fairly
common, but sometimes with significant modifications. For example, many
extragalactic dwarf elliptical and blue compact / Irr galaxies are fit
well with an exponential + central cusp / nuclear region
\citep[\citet{noeske+2003}]{caldwell+bothun1987}. In constrast, some
studies find an inner decrement in density compared to exponentials
\citep[e.g.,][]{caldwell+1992, makarov+2012}.

However, exponential profiles do not fit every dwarf galaxy. Even
starting from \citet{aparicio+1997}, and \citet[for Coma cluster dwarf
ellipticals]{graham+guzman2003}, some people have noted deviations from
this simple empirical rule. Additionally, \citet{hunter+elmegreen2006};
\citet{herrmann+hunter+elmegreen2013};
\citet{herrmann+hunter+elmegreen2016}; \citet{lee+2018} note that at
least in a photometric sample of more distant dwarf disky/irregular/blue
compact dwarfs, many or most dwarfs are best fit with two nested
exponential profiles. Dwarf irregular galaxies do represent a
substantially different class though so it is unclear how these
conclusions apply to dwarf spheroidals. \citet{caldwell+1992}'s
photometric study of M31 dwarfs show that the outer density profile is
typically well fit by an exponential, although deviations are evident in
the inner regions.

More recently, \citet{moskowitz+walker2020}, have used a Plummer profile
and modified Plummer profiles with steeper outer slopes
(\(\Gamma = d\log \Sigma / d \log R = -8\) rather than \(\Gamma = -4\)).
For most galaxies, they do not have convincing statistical evidence to
prefer one model over another, but show that for 8 / 15 of dwarfs with
sufficient data, a Plummer or steeper profile is (slightly) preferred.
\textbf{This sentence is problematic}

Altogether, while there is some variation in the density profiles of
dwarf galaxies, an exponential is an excellent first-order
approximation. Typically, deviations from exponentials are in a
direction of a steeper outer cutoff or changes to the inner slope of a
dwarf galaxy (due, for example, to a nuclear star cluster.). Flattened
density profiles in the outer regions are more unusual. Explaining the
origin, similarity, and diversity of dwarf galaxy density profiles is a
pressing question for theories of dwarf galaxy formation and evolution.
Whether dwarf galaxies are indeed exponential or not, the diversity of
density profile shapes of dwarf galaxies needs to be explained.

\subsection{The extended light profiles of Sculptor and Ursa Minor:
Hints of tidal
signatures?}\label{the-extended-light-profiles-of-sculptor-and-ursa-minor-hints-of-tidal-signatures}

Sculptor and Ursa Minor appear to be typical dwarf spheroidal galaxies
at first glance. Tables~\ref{tbl:scl_obs_props}, \ref{tbl:umi_obs_props}
describe the present-day properties of each galaxy. Sculptor, as the
first discovered classical dSph, is often described as a
``prototypical'' dSph. There has long been speculation that both
Sculptor and Ursa Minor may have been influenced by the Milky Way's
tidal field (see Section~\ref{sec:discussion}). \citet{sestito+2023a};
\citet{sestito+2023b} report a ``kink'' in the density profile,
beginning around 30 arcmin for both Sculptor and Ursa Minor, that they
interpret as caused by the effects of Galactic tides. They
spectroscopically follow up some of the most distant stars, finding
multiple members between 6-12 half-light radii from the centre of each
dwarf. If dwarfs had exponential profiles, like Fornax, then these stars
should be much rarer.

Sculptor and Ursa Minor are not well-described by an exponential
profile. The left panel of Fig.~\ref{fig:scl_umi_vs_penarrubia} shows
the density profiles of Sculptor, Ursa Minor, and Fornax (see
Section~\ref{sec:observations} for details on how these profiles are
measured). Compared to Fornax, both Sculptor and Ursa Minor show an
excess of stars beginning around \(\log R/R_h\approx 0.4\). The right
panel of Fig.~\ref{fig:scl_umi_vs_penarrubia} compares the same density
profiles to an initial and final density profile from a simple tidal
model of an exponential dwarf spheroidal galaxy embedded in a NFW halo
in the Milky Way field, described in Section~\ref{sec:tidal_theory}
(\textbf{Move to later or describe?}). Fornax agrees well with the
exponential initial conditions. However, Sculptor and Ursa Minor are
better described by the more extended, tidally evolved density profile.

A goal of this work is to determine, assuming \LCDM, if the effects of
the Milky Way (or other satellites) are indeed responsible for the
extended outer light profiles of Sculptor and Ursa Minor. If tides
cannot explain these features, these features may instead be due to an
extended stellar ``halo'' or second component of the galaxy---suggestive
of a complex star formation or accretion history.

\begin{figure}
\centering
\pandocbounded{\includegraphics[keepaspectratio]{./figures/scl_umi_vs_penarrubia.pdf}}
\caption[Sculptor and Ursa Minor match tidal models]{A plot of the
surface density profiles of Sculptor, Ursa Minor, and Fornax scaled to
their half-light radius and the density at half-light radius (data
described in Section~\ref{sec:observations}). The right panel adds an
idealized model with initial, final, and break radius as a dotted line,
solid line and an arrow respectively see \ref{sec:break_radii}.
\textbf{Move RHS to later\ldots{}}}\label{fig:scl_umi_vs_penarrubia}
\end{figure}

\begin{table*}[t]
\centering
\caption[Observed Properties of Sculptor]{Observed properties of Sculptor. References are: 1. Muñoz et al. (2018) Sérsic fits, 2. Tran et al. (2022) RR lyrae distance, 3. Alan W. McConnachie and Venn (2020b), 4. Arroyo-Polonio et al. (2024). }
\label{tbl:scl_obs_props}
\begin{tabular}{lll}
\toprule
parameter & value & Source\\
\midrule
$\alpha$ & $15.0183 \pm 0.0012^\circ$ & 1\\
$\delta$ & $-33.7186 \pm 0.0007^\circ$ & ”\\
distance modulus & $19.60 \pm 0.05$ & 2\\
distance & $83.2 \pm 2$ kpc & ”\\
$\mu_{\alpha*}$ & $0.099 \pm 0.002 \pm 0.017$ mas yr$^{-1}$ & 3\\
$\mu_\delta$ & $-0.160 \pm 0.002_{\rm stat} \pm 0.017_{\rm sys}$ mas yr$^{-1}$ & ”\\
LOS velocity & $111.2 \pm 0.3\ {\rm km\,s^{-1}}$ & 4\\
$\sigma_v$ & $9.7\pm0.2\ {\rm km\,s^{-1}}$ & ”\\
$R_h$ & $9.79 \pm 0.04$ arcmin & 1\\
ellipticity & $0.37 \pm 0.01$ & ”\\
position angle & $94\pm1^\circ$ & ”\\
$M_V$ & $-10.82\pm0.14$ & ”\\
\bottomrule
\end{tabular}
\end{table*}

\begin{table*}[t]
\centering
\caption[Observed Properties of Ursa Minor]{Observed properties of Ursa Minor. References are: (1) Muñoz et al. (2018) Sérsic fits, (2) Garofalo et al. (2025) RR lyrae distance, (3) Alan W. McConnachie and Venn (2020a), (4) Pace et al. (2020), average of MMT and Keck results. }
\label{tbl:umi_obs_props}
\begin{tabular}{lll}
\toprule
parameter & value & Source\\
\midrule
$\alpha$ & $ 227.2420 \pm 0.0045$˚ & 1\\
$\delta$ & $67.2221 \pm 0.0016$˚ & ”\\
distance modulus & $19.23 \pm 0.11$ & 2\\
distance & $70.1 \pm 3.6$ kpc & ”\\
$\mu_\alpha*$ & $-0.124 \pm 0.004 \pm 0.017$ mas yr$^{-1}$ & 3\\
$\mu_\delta$ & $0.078 \pm 0.004_{\rm stat} \pm 0.017_{\rm sys}$ mas yr$^{-1}$ & ”\\
LOS velocity & $-245.9 \pm 0.3_{\rm stat} \pm 1_{\rm sys}$ km s$^{-1}$ & 4\\
$\sigma_v$ & $8.6 \pm 0.3$ & ”\\
$R_h$ & $11.62 \pm 0.1$ arcmin & 1\\
ellipticity & $0.55 \pm 0.01$ & ”\\
position angle & $50 \pm 1^\circ$ & ”\\
$M_V$ & $-9.03 \pm 0.05$ & ”\\
\bottomrule
\end{tabular}
\end{table*}

\section{Interpreting tidal signatures}\label{sec:tidal_theory}

Simulating dwarf galaxies accurately in a cosmological context remains a
substantial challenge. As discussed above, cosmological simulations can
predict the overall abundance of the most massive dwarf galaxies
{[}e.g., \textbf{references}{]} and broadly examine the effects of tides
\citep[e.g.,][]{riley+2024}. However, dwarf galaxies are often near the
resolution limit. Insufficient resolution can lead to artificial
disruption of dwarf galaxies and overestimation of tidal effects
\citep[e.g.,][]{santos-santos+2025}. To address this challenge,
idealized simulations are often used to simulate a single subhalo in an
approximate host potential. As a result, idealized simulations can reach
excellent numerical convergence For instance, our simulations we
describe later reach 3x higher resolution than Aquarius
\citep{springel+2008} at a fraction of the computational cost (400x
fewer particles). Idealized simulations make numerous simplifications:
neglecting mergers, cosmological context and evolution, mass assembly,
and often baryonic physics. We use idealized simulations here to assess
tidal effects after infall.

Some of the earliest simulation work on tidal mass loss of dwarf
galaxies originate from \citet{oh+lin+aarseth1995};
\citet{piatek+pryor1995}; \citet{moore+davis1994};
\citet{johnston+spergel+hernquist1995}. Already, these works used
techniques similar to what we continue to use today and laid the
foundation for understanding tidal effects. While the detailed
assumptions have evolved (e.g.~no longer assuming mass-follows-light),
many of their conclusions still hold true. More recent work expanding
this theory include \citet{read+2006}; \citet{bullock+johnston2005};
\citet{PNM2008}; \citet{penarrubia+2009}; \citet{klimentowski+2009};
\citet{errani+2023a}; \citet{fattahi+2018}; \citet{stucker+2023};
\citet{wang+2017}.

With precise orbits and a better understanding of the Milky Way
potential and system, more recent work began to directly probe the
dynamical histories of individual dwarf galaxies (although early work
began this for Sagittarius / etc). Examples include \citet{iorio+2019}
for Sculptor, \citet{borukhovetskaya+2022}; \citet{dicintio+2024} for
Fornax; \citet{borukhovetskaya+2022a} for Antlia II. Our goal is to
apply a similar framework to Sculptor and Ursa Minor.

*\textbf{Finish last two paragraphs}

\subsection{A simple tidal simulation}\label{a-simple-tidal-simulation}

To illustrate the effects of tides on an NFW halo in a Milky-Way like
potential, we consider a toy model. See also \citet{PNM2008}?.

We evolve an NFW profile in a logarithmic host potential along an orbit
where the pericentre to apocentre is 1:50.

As illustrated in Fig.~\ref{fig:idealized_break_radius}, for an
idealized model with exponential stars in a NFW halo, shortly after
pericentre, the stellar component is smooth but contains a change in
slope around \(r_{\rm break}\). This radius is visible in the stellar
distribution as non-spherical S-shaped overdensities of stars. Also,
\(r_{\rm break}\) is where where the mean radial velocities of the stars
becomes positive, i.e.~the system is out of equilibrium and adjusting to
the new density profile.

\begin{figure}
\centering
\pandocbounded{\includegraphics[keepaspectratio]{figures/idealized_break_radius.pdf}}
\caption[Break radius validation]{Example density and velocity
distributions of an idealized simulation shortly after pericentre.
\textbf{Top left}: The 2D density profile for the initial and final
simulation with the break radius marked. The break radius of the
simulations is set by the time since pericentre. \textbf{Top right}: The
projected 2D stellar density in the \(x\)-\(y\) plane. The green circle
represents the break radius and the grey arrow points towards the host
centre. \textbf{Bottom left}: the mean radial velocity (dot product of
relative position and velocity relative to dwarf centre) as a function
of 2D radius. \textbf{Bottom right}: The mean radial velocity as
projected into 2D bins. The initial conditions are Sculptor-like,
exponential stars embedded in NFW, evolved in \citet{EP2020} potential
with a pericentre of 10 kpc and apocentre of 100 kpc shortly after the
3rd pericentre. See Section~\ref{sec:methods} for a description of our
simulation setup.}\label{fig:idealized_break_radius}
\end{figure}

\subsection{Tidal and ``break'' radii}\label{sec:break_radii}

For a given orbit in a given potential, there are characteristic radii
which help gauge the effects of tides on a dwarf galaxy model.

The \textbf{Jacobi radius} represents the approximate radius where stars
become unbound for a galaxy in a circular orbit around a host galaxy.
Calculated from an approximation of the location of the \(L_1\) and
\(L_2\) Lagrange points, the Jacobi radius is where the mean density of
the dwarf galaxy is roughly three times the mean interior density of the
host galaxy at pericentre, or \begin{equation}{
3\bar \rho_{\rm MW}(r_{\rm peri}) \approx \bar \rho_{\rm dwarf}(r_J).
}\end{equation} If \(r_J\) occurs within the visible extent of a galaxy,
we should expect to find relatively clear signs of tidal disturbance.
While strictly valid for circular orbits, assuming \(r_{\rm peri}\) for
the host-dwarf distance works as most stars are lost near pericentre.

We also use the \textbf{break radius} as defined in
\citet{penarrubia+2009}, marking where the galaxy is still in
disequilibrium after pericentre in a highly-eccentric orbit. The break
radius \(r_{\rm break}\) is proportional to the velocity dispersion
,\(\sigma_v\), and time elapsed since pericentre ,\(\Delta t\),
\begin{equation}\protect\phantomsection\label{eq:r_break}{
r_{\rm break} = C\,\sigma_{v}\,\Delta t
}\end{equation} where the scaling constant \(C \approx 0.55\) was
derived empirically. \(r_{\rm break}\) describes where the dynamical
timescale is longer than the time since the perturbation, i.e.~the
radius within which the galaxy has had enough time to dynamically relax.

\section{Thesis outline}\label{thesis-outline}

In this thesis, our goal is to review the evidence for an extended
density profile in Ursa Minor and Sculptor, to assess the impact of
tidal effects on each galaxy, and to discuss possible interpretations
for the structure of these galaxies.

In chapter \ref{sec:observations}, we describe how we compute
observational density profiles from \citet{jensen+2024}. In chapter
\ref{sec:methods}, we review our simulation methods. Next, we present
our results for the tidal effects on Sculptor and Ursa Minor in chapter
\ref{sec:results}, We discuss our results, limitations, and implications
in chapter \ref{sec:discussion}. Finally, chapter \ref{sec:summary}
summarizes this work and discuss future directions for similar work and
the field of dwarf galaxies.
