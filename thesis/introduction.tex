Dwarf galaxies host, in many ways, the most extreme galactic
environments in the universe. These little galaxies are typically
defined to be fainter than the LMC \citep[\(M_V \gtrsim -18\) or
similarly \(L \lesssim 10^9 L_\odot\),
e.g.,][]{hodge1971, mcconnachie2012}. As the smallest class of galaxies,
dwarfs are the most cosmologically numerous galaxies. Dwarf galaxies are
highly \emph{dark-matter dominated}, with mass to light ratios exceeding
100 to 1000 \(M_\odot/ L_\odot\). Because of their small total masses,
many dwarf galaxy satellites of the Milky Way are \emph{quenched}, with
little to no recent star formation. Milky Way dwarfs contain stellar
populations which are \emph{relics} from the early universe, consisting
of many of the oldest and most metal poor stars. Understanding the
properties of dwarf galaxies thus has implications across astronomy,
from cosmological structure formation on the smallest scale to extremely
metal-poor stellar populations to chemical evolution to galactic
dynamics. Of the classical dwarfs, we find that the Sculptor and Ursa
Minor dwarf spheroidal galaxies stand out with a more extended density
profile relative to an exponential, hinting at tidal effects. As a case
study in tidal effects and our understanding of dwarf galaxy formation
and evolution, we aim to understand the dynamical history of these
galaxies.

In this section, we first describe cosmological structure formation and
the role of dwarf galaxies. Then, we explore the observational history
of dwarf galaxies, the \emph{Gaia} telescope, open questions, and why
Sculptor and Ursa Minor stand out. We discuss idealized simulations,
tidal effects, break radii, and recent developments in tidal
simulations. Finally, we provide a roadmap to the thesis as a whole at
the end of this section.

\section{Observational context}\label{observational-context}

Dwarf galaxies have long raised conundrums in theories of galaxy
formation. The discovery of Fornax and Sculptor in 1938
\citep{shapley1938}\footnote{Technically, the Large and Small Magellanic
  Clouds (LMC, SMC) may be classified as dwarf galaxies, but these were
  likely always known to humans at southern latitudes.}, with no known
analogues, already presented such an enigma. Shapley presented these
dwarfs as a new type of \emph{stellar system} resembling the Magellanic
Clouds and globular clusters but did not attempt to speculate on the
exact nature. While dwarf galaxies were quickly understood to be
galaxies based on the inferred luminosities and sizes, their nature and
formation remained unclear for decades
\citep[e.g.,][]{hodge1971, gallagher+wyse1994}.

The earliest spectroscopic work hinted that dwarf galaxies may contain
substantial amounts of dark matter. From early determinations of the
velocity dispersion for the Sculptor and Ursa Minor dwarf spheroidal
(dSph) galaxies
\citep[e.g.,][\citet{aaronson+olszewski1987}]{aaronson1983}, inferred
mass-to-light ratios were at least 10 times larger than the values from
globular clusters scaled to the same sizes. While rather uncertain
initially, these values have corroborated with larger and more precise
samples \citep[e.g.,][]{hargreaves+1994}. Subsequently, several theories
attempting to understand the formation and observed properties of these
objects were proposed. Examples include: dwarf galaxies are undergoing
tidal dissolution resulting in extreme mass-to-light ratios
\citep[e.g.,][]{kuhn+miller1989}, presence of massive central black
holes \citep[e.g.,][]{strobel+lake1994}, the formation of dark
matter-free ``tidal dwarfs'' from past mergers
\citep[e.g.,][\citet{kroupa1997}]{lynden-bell1982}, or modified theories
of gravity \citep{milgrom1995}. However, consistency with CDM galaxy
formation was also not out of the question
\citep[e.g.,][]{dekel+silk1986}. Since then, we have known that dwarf
galaxies are among the darkest objects in the universe, and
understanding their properties is critical to understanding dark matter.

Dwarf galaxies span a large range of physical sizes, luminosities, and
morphologies. Broadly, there are three classes of dwarf galaxies based
on luminosity, as illustrated by Fig.~\ref{fig:galaxy_images}. Local
\textbf{bright dwarfs} with magnitudes
\(-18 \lesssim M_V \lesssim  -14\), often exhibit irregular morphologies
and recent star formation. Fig.~\ref{fig:galaxy_images} shows the Large
Magellanic Cloud (LMC) as an example of an irregular, bright dwarf
galaxy displaying a bar. \textbf{Classical dwarfs} occupy intermediate
luminosities ( \(-14 \lesssim M_V  \lesssim -7.7\)). Typically, these
systems are old, gas-poor, and spheroidal. All Milky Way satellites
discovered before digital sky surveys are classicals. The 12 classical
dwarfs satellites of our Galaxy are Sagittarius, Fornax, Leo I,
Sculptor, Antlia II, Leo II, Carina, Draco, Ursa Minor, Canes Venatici
I, Sextans I, and Crater II.\footnote{While formally the dwarf galaxy
  names we discuss contain ``dwarf spheroidal'' (dSph), e.g.~Sculptor
  dSph, we omit this suffix for brevity.} The \textbf{ultra-faint}s
occupy the very faintest magnitudes (\(-7.7 \lesssim M_V\)). These
galaxies have minuscule stellar masses, tend to be more compact, and are
the darkest known galaxies. Altogether, dwarf galaxies span more than 15
orders in absolute magnitude.

Since Local Group dwarfs are nearby, we can study these galaxies on a
star-by-star basis. As a result, it is possible to measure the 3D
velocity and position of a star if we can measure the stars position,
line-of-sight (LOS) velocity, and proper motions. Unfortunately,
determining distances and full 3D velocities is challenging. The most
direct measurement of distance, parallax, requires precise tracking of a
star's sky position across a year. And while line-of-sight (LOS)
velocities are easily determined from spectroscopy, tangential
velocities, derived from proper motions and distances, are much more
challenging. Typically, measuring proper motions requires accurate (much
less than arcsecond) determinations of small changes in a star's
position over baselines of several years to decades. The full 6D
position and velocity information for stars has historically been known
for only a handful of stars.

\begin{figure}
\centering
\includegraphics[width=5.41667in,height=5.41667in]{figures/galaxy_pictures.png}
\caption[Dwarf Galaxy Pictures]{Images of the LMC (DSS2), Fornax (DES
DR2), Sculptor (DES DR2), and Ursa Minor (UNWISE with Gaia point sources
over-plotted). Each image includes the galaxy's luminosity and a 200 pc
scale bar. \emph{TODO: Mark half-light radius with
circle?}}\label{fig:galaxy_images}
\end{figure}

\begin{figure}
\centering
\includegraphics[width=5.41667in,height=\textheight,keepaspectratio]{figures/mw_satellites_1.jpg}
\caption[Dwarf galaxies sky position]{The location of MW dwarf galaxies
on the sky. We label the classical dwarf galaxies (green diamonds),
fainter dwarfs (blue squares), globular clusters (orange circles), and
ambiguous systems (pink open hexagons). Globular clusters are more
centrally concentrated, but dwarf galaxies are preferentially found away
from the MW disk. Sculptor and Ursa Minor are highlighted as two dwarfs
we study later. The background image is from ESA/Gaia/DPAC
(https://www.esa.int/ESA\_Multimedia/Images/2018/04/Gaia\_s\_sky\_in\_colour2).
Dwarf galaxies (confirmed), globular clusters, and ambiguous systems are
from the \citet{pace2024} catalogue (version
1.0.3).}\label{fig:mw_satellite_system}
\end{figure}

\subsection{\texorpdfstring{The era of
\emph{Gaia}}{The era of Gaia}}\label{the-era-of-gaia}

\emph{(db: these paragraphs start the same\ldots)}

\emph{Gaia} has redefined astrometry, providing photometry, positions,
proper motions, and parallaxes for over 1 billion stars
\citep{gaiacollaboration+2021}. Launched in 2013, \emph{Gaia} is a
space-based, all-sky survey telescope situated at the Sun-Earth L2
Lagrange point \citep{gaiacollaboration+2016}. While \emph{Gaia}
completed its space-based mission in 2025, two forthcoming data releases
remain. Determining absolute parallax measurement is facilitated by the
observation that stars in different regions of the sky are affected by
parallax motion with different phases. By imagining two regions
separated by 106.5 degrees on the same focal plane, \emph{Gaia} measures
tiny changes in relative positions of stars across small and large
angles. Combining measurements from multiple epochs across several
years, an absolute all-sky reference frame is derived from which
parallax and proper motions are derived. In addition to astrometry,
\emph{Gaia} measures photometry in the wide \emph{G} band (330-1050nm)
and colours from the blue photometer (BP, 330-680 nm) and red photometer
(RP, 640-1050 nm). \emph{Gaia} additionally provides low resolution
BP-RP spectra and radial velocity measurements of bright stars
(\emph{Gaia} radial velocity magnitudes \textless16)
\citep{gaiacollaboration+2016}. For this work, \emph{Gaia}'s most
relevant measurement are \(G\) magnitude, \(G_{\rm BP} - G_{\rm RP}\)
colour, \((\alpha, \delta)\) position, and
\((\mu_{\alpha*}, \mu_\delta)\) proper motions.\footnote{The proper
  motions \(\mu_\alpha\) and \(\mu_\delta\) are the apparent rates of
  change in right ascension, \(\alpha\), and declination, \(\delta\),
  typically in units of mili-arcsecond (mas) per year.
  \(\mu_{\alpha*} = \mu_\alpha \cos \delta\) corrects for projection
  effects in \(\alpha\).}

\emph{Gaia} has revolutionized many fields of astronomy including
studies of the Milky Way and the Local Group. The Local Group is defined
as galaxies which are approximately bound to the Milky Way-Andromeda
system, or within about 1 Mpc from the Milky Way \citep[e.g.,][ and
references therein]{mcconnachie2012}. For example, 6D kinematic
classification of Milky Way stars led to the discovery of past mergers
or Milky Way building blocks like \emph{Gaia}-Sausage Enceladus
\citep[e.g.,][]{helmi+2018}, out-of-equilibrium structures like the
\emph{Gaia} snail {[}e.g., \citet{antoja+2018}, and dynamical effects of
the Milky Way's spiral arms and the bar in the solar neighbourhood
\citep{hunt+vasiliev2025}. Moving to the Milky Way halo, \emph{Gaia} has
helped find and constrain numerous stellar streams
\citep{bonaca+price-whelan2025}. Altogether, \emph{Gaia} has revealed
the hierarchical formation and complex, evolving structure of our own
Galaxy.

For Milky Way satellites, \emph{Gaia} has enabled well-constrained
orbital analysis and facilitated precise stellar membership
determination. Before, proper motions of dwarf galaxies were sparse or
undetermined. Few galaxies had precisely measured proper motions, often
from Hubble Space Telescope observations \citep[e.g.,][]{sohn+2017}.
\emph{Gaia} allowed for the first systematic determinations of Milky Way
satellite proper motions of unprecedented precision
\citep{MV2020a, pace+li2019}. While the proper motion uncertainty on a
typical dwarf member star is often large, by combining the proper
motions for 100s to 1000s of stars in \emph{Gaia}, precise proper motion
measurements can be determined, sometimes only limited by \emph{Gaia}'s
systematic error floor \citep[e.g.,][]{MV2020a}. Proper motions have
ushered in a new dynamical era for MW satellite studies, where we can
derive precise orbits assuming a given MW potential. In addition,
\emph{Gaia} helps separate out contaminating MW foreground stars. By
measuring parallaxes or proper motions, many background and foreground
stars can be removed as non-members
\citep[e.g.,][\citet{jensen+2024}]{battaglia+2022}.

\subsection{Dwarf galaxies today}\label{dwarf-galaxies-today}

Today, we know that the Milky Way system is teeming with satellites.
Fig.~\ref{fig:mw_satellite_system} shows the MW satellite system,
including dwarf galaxies, ambiguous systems (whose nature as a
dwarf-like or cluster-like system remains uncertain), and globular
clusters. Advances in observational techniques has accelerated progress
in our understanding of dwarf galaxies and introduced new questions.
Deep digital photometric surveys have more than quadrupled the number of
known Milky Way dwarf galaxies \citep{simon2019}. Upcoming and ongoing
surveys, like the Vera Rubin Observatory's Legacy Survey for Space and
Time \citep{ivezic+2019}, will likely identify even fainter dwarf
galaxies. In addition, large aperture multi-object spectrographs have
revealed the detailed and complex inner chemodynamical structure of
dwarf galaxies. Beyond precise structural and kinematic properties of
dwarf galaxies, modern observations allow for the separation of multiple
stellar populations, detailed constraints on the dark matter density
profiles, and hints of tidal disruption or stellar halos.

Of local dwarfs, the classical systems still remain the most studied
galaxies with the best constrained parameters. Because many Milky Way
dwarfs extend 1-2 degrees across the sky, measurements of some of their
properties can be challenging \citep[e.g.,][]{mateo1998}. These dwarfs
also contain large numbers of bright (giant) stars which can be followed
up spectroscopically \citep[e.g.,][]{tolstoy+2023, pace+2020}. As a
result, the determination of fundamental properties such as the
position, size, orientation, distances, proper motions, line-of-sight
(LOS) velocities, and velocity dispersions \(\sigma_v\), are all
relatively well constrained today. However, ongoing research continues
to redefine our understanding of the detailed structure of Milky Way
satellites, hinting that these objects may be more complex than expected
at first glance.

Observations of dwarf galaxies have been the origin of several disputes
or \emph{small-scale} problems for \(\Lambda\)CDM \citep[see
review][]{bullock+boylan-kolchin2017}. For example, the mismatch between
the expected number of dwarf galaxies from simulations and the observed
abundance was known as the \emph{missing satellites problem}.
Additionally, a number of observations suggested that some dwarf
galaxies, although not all, possess dark matter ``cores''
\citep[e.g.,][]{moore1994, adams+2014, oh+2015, walker+penarrubia2011, read+walker+steger2019}.\footnote{In
  detail, gas-phase rotation curves are better able to differentiate
  between cores and cusps, whereas stellar kinematics is less
  constraining.} As a result, alternative forms of dark matter have been
advocated as solutions, such as Warm and Self-Interacting Dark Matter.

However, these tensions have eased as a result of improved understanding
of baryonic physics. For example, recent hydrodynamic simulations in
particular have shown that strong feedback can produce dark matter cores
\citep[e.g.,][]{tollet+2016, fitts+2017, benitez-llambay+2019, orkney+2021}.
Altogether, the numerous past challenges for \(\Lambda\)CDM in the dwarf
galaxy regime illustrates the opportunity for dwarf galaxies to test the
understanding of galaxy formation and dark matter physics.

\subsection{Typical stellar density
profiles}\label{typical-stellar-density-profiles}

Projected density profiles efficiently characterize the shape of a
galaxy. At its most basic, density profiles provide properties such as
the shape, location, size, and orientation of a dwarf galaxy. In
addition, the details of a stellar density profile can help interpret
the total mass of dwarf galaxies, their assembly and dynamical history,
and to understand scaling laws between dwarf galaxy properties
\citep[e.g.,][\citet{penarrubia+2009}, \citet{querci+2025},
\citet{lee+2018}]{herrmann+hunter+elmegreen2013}. We note that for
resolved galaxies, these profiles are in terms of stellar count density
instead of surface magnitudes.

Four different surface density laws are frequently used to parameterize
dwarf galaxy profiles: Exponential, Plummer, King, or Sérsic profiles
\citep[e.g.,][]{munoz+2018}. The exponential profile is perhaps the
simplest, defined in terms of the central surface density, \(\Sigma_0\),
and scale radius, \(R_s\): \begin{equation}{
\Sigma_{\rm exp} = \Sigma_0\exp(-R / R_s).
}\end{equation}

This profile is often applied to the radial light distribution of galaxy
disks {[}\citet{freeman1970}; other classic paper?{]}.

To fit globular cluster density profiles, \citet{plummer1911} proposed a
profile based on a polytropic solution\footnote{where density and
  pressure are assumed to be related by a power law}, \begin{equation}{
\Sigma_{\rm Plummer} = \frac{\Sigma_0}{(1 + (R/R_h)^2)^2},
}\end{equation}

where \(\Sigma_0\) is the central surface density and \(R_h\) is the 2D
half-light radius. Now mostly superseded by the King profile for
globular clusters, the Plummer model is still a good fit to many dwarf
spheroidals \citep[e.g.,][]{moskowitz+walker2020}.

The \citet{king1962} profile, also an empirical fit to globular
clusters, is also often applied to dwarf galaxies, especially in older
literature. Using three parameters, a core radius \(R_c\), a truncation
radius \(R_t\), and a characteristic density, \(\Sigma_0\), the King
profile is \begin{equation}{
\Sigma_{\rm King} = \Sigma_0\left(\frac{1}{\sqrt{1 + (R/R_c)^2}} - \frac{1}{\sqrt{1+(R_t/R_c)^2}}\right).
}\end{equation}

In much of the older literature, \(R_t\) was often called and
interpreted as a ``tidal radius'', after the similar interpretation for
globular clusters \citep[e.g.,][\citet{hodge1961}]{IH1995}.

Finally, the \citet{sersic1963} profile represents a generalization of
an exponential profile, and describes most galaxy light profiles well.
Typically parameterized in terms of a half-light radius \(R_h\), the
density at half-light radius \(\Sigma_h\) and a Sérsic index \(n\), the
profile's equation is \begin{equation}{
\Sigma_{\rm S\acute ersic} = \Sigma_h \exp\left[-b_n \,  \left((R/R_h)^{1/n} - 1\right)\right]
}\end{equation} where \(b_n\) is a constant depending on \(n\). \(n=1\)
provides an exponential profile and \(n=4\) recovers
\citet{devaucouleurs1948}'s profile for elliptical galaxies. While a
Sérsic profile is not always used for dwarf galaxies, \citet{munoz+2018}
advocate for the Sérsic profile since the added flexibility allows for
more profiles to be fit.

While there are no clear theoretical preferences for any of these
profiles, exponential density profiles are commonly used for dwarf
spheroidal galaxies. \citet{faber+lin1983} were among the first to
demonstrate that an exponential law is a reasonable empirical fit to
dwarf galaxies, theorizing that dwarf spheroidal may have evolved from
exponential disk galaxies and maintained a similar light profile. Later,
\citet{read+gilmore2005} showed that exponential profiles may originate
from general mass loss during the evolution of dwarf galaxies. Many
later photometric works for dwarf spheroidal galaxies have used
exponential fits, finding that exponential and King profiles both
provide good descriptions in many cases
\citetext{\citealp[\citet{mateo1998}]{binggeli+sandage+tarenghi1984}; \citealp{mcconnachie+irwin2006}; \citealp{cicuendez+2018}}.
As a result, many studies assume an exponential density profile for
dwarf galaxies in theoretical or observational modelling
\citetext{\citealp[e.g.,][
\citet{MV2020a}]{martin+2016}; \citealp{battaglia+2022}; \citealp[ but
is for disk]{kowalczyk+2013}}.

An exponential is far from the only density profile advocated for. Some
authors fitting Sérsic or King profiles (often in addition to
Exponential), find that the additional parameter somewhat improves fits
relative to an exponential
\citep{IH1995, vanzee+barton+skillman2004, munoz+2018, wang+2019}. Note
that the Sérsic indices are typically \(n \lesssim 1\), implying that
most galaxies tend to be an exponential or slightly steeper in the outer
regions.

Beyond Local Group dwarf spheroidals, exponential profiles appear to be
common, but sometimes with significant modifications. In particular,
many extragalactic dwarf elliptical and blue compact / Irr galaxies are
fit well with an exponential + central cusp / nuclear region
\citep[\citet{noeske+2003}]{caldwell+bothun1987}. In constrast, some
studies find an inner decrement in density compared to exponentials
\citep[e.g.,][]{caldwell+1992, makarov+2012}.

However, exponential profiles do not fit every dwarf galaxy. Even
starting from \citet{aparicio+1997}, and \citet[for coma cluster
dE]{graham+guzman2003}, some people have noted deviations from this
simple empirical rule. Additionally, \citet{hunter+elmegreen2006};
\citet{herrmann+hunter+elmegreen2013};
\citet{herrmann+hunter+elmegreen2016}; \citet{lee+2018} note that at
least in a photometric sample of more distant dwarf disky/Irr/blue
compact dwarfs, that many or most dwarfs show two-part exponential
profiles. Irr do represent a substantially different class though so it
is unclear if these conclusions apply. \citet{caldwell+1992}'s
photometric study of M31 dwarfs show that the outer density profile
typically fits well to an exponential but with inner deviations. Note
that while diversity is also commented on here, dwarf galaxies with
outer flattening profiles are often a minority in these studies.

Alternatively, \citet{moskowitz+walker2020}, use a Plummer profile and
modified Plummer profiles with steeper outer slopes
(\(\Gamma = d\log \Sigma / d \log R = -8\) compared to \(\Gamma = -4\)).
For most galaxies, they do not have convincing statistical evidence to
prefer one model over another, but show that for 8 / 15 of dwarfs with
sufficient data, a plummer or steeper profile is preferred, with only
the stepper one ever strongly preferred.

Altogether, while there is some natural variation in the density
profiles of dwarf galaxies, an exponential is an excellent first-order
approximation. Typically, deviations from exponentials are in a
direction of steeper outer cutoff or changes to the inner slope of a
dwarf galaxy (due to a nuclear region/etc.). Flattened density profiles
in the outer regions are more unusual. Explaining the origin,
similarity, and diversity of dwarf galaxy density profiles is a pressing
question for theories of dwarf galaxy formation and evolution. Whether
dwarf galaxies are indeed exponential or not, the diversity of density
profile shapes of dwarf galaxies needs to be explained.

\subsection{Sculptor and Ursa Minor: Hints of tidal
signatures?}\label{sculptor-and-ursa-minor-hints-of-tidal-signatures}

Sculptor and Ursa Minor may appear to be typical dwarf galaxies at first
glance. Tables~\ref{tbl:scl_obs_props}, \ref{tbl:umi_obs_props} describe
the present-day properties of each galaxy. Sculptor, as the first
discovered classical dSph, is often described as a ``prototypical''
dSph. There has long been speculation that both Sculptor and Ursa Minor
may have been influenced by the Milky Way's tidal field (see
Section~\ref{sec:discussion}). \citet{sestito+2023a};
\citet{sestito+2023b} report that a ``kink'' in the density profile,
beginning around 30 arcmin for both Sculptor and Ursa Minor. They
spectroscopically follow up some of the most distant stars, finding
multiple members between 6-12 half-light radii from the centre of each
dwarf. If dwarfs had exponential profiles like Fornax, then these stars
should be much rarer.

Sculptor and Ursa Minor are not well-described by an exponential
profile. The left panel of Fig.~\ref{fig:scl_umi_vs_penarrubia} shows
our density profiles of Sculptor, Ursa Minor, and Fornax (see
Section~\ref{sec:observations} for our methods). Compared to Fornax,
both Sculptor and Ursa Minor show an excess of stars beginning around
\(\log R/R_h\approx 0.4\). The right panel of
Fig.~\ref{fig:scl_umi_vs_penarrubia} compares the same density profiles
to an initial and final density profile from a simple tidal model of an
exponential dwarf spheroidal galaxy embedded in a NFW halo in the Milky
Way field, described in Section~\ref{sec:tidal_theory}. Fornax agrees
well with the exponential initial conditions. However, Sculptor and Ursa
Minor are better described by the more extended, tidally evolved density
profile.

A goal of this work is to determine, assuming \(\Lambda\)CMD, if the
effects of the Milky Way (or other satellites) may indeed create
observable tidal signatures in Sculptor and Ursa Minor. If tides cannot
explain these features, these features may instead be a extended stellar
``halo'' or second component of the galaxy---suggestive of a complex
history or formation in each galaxy, which needs to be explained by
galaxy formation models of Local Group dwarf galaxies.

\begin{figure}
\centering
\pandocbounded{\includegraphics[keepaspectratio]{./figures/scl_umi_vs_penarrubia.pdf}}
\caption[Sculptor and Ursa Minor match tidal models]{A plot of the
surface density profiles of Sculptor, Ursa Minor, and Fornax scaled to
their half-light radius and the density at half-light radius (data
described in Section~\ref{sec:observations}). The right panel adds an
idealized model with initial, final, and break radius as a dotted line,
solid line and an arrow respectively see
\ref{sec:break_radii}.}\label{fig:scl_umi_vs_penarrubia}
\end{figure}

\begin{table*}[t]
\centering
\caption[Observed Properties of Sculptor]{Observed properties of Sculptor. References are: 1. Muñoz et al. (2018) Sérsic fits, 2. Tran et al. (2022) RR lyrae distance, 3. Alan W. McConnachie and Venn (2020b), 4. Arroyo-Polonio et al. (2024). }
\label{tbl:scl_obs_props}
\begin{tabular}{lll}
\toprule
parameter & value & Source\\
\midrule
$\alpha$ & $15.0183 \pm 0.0012^\circ$ & 1\\
$\delta$ & $-33.7186 \pm 0.0007^\circ$ & ”\\
distance modulus & $19.60 \pm 0.05$ & 2\\
distance & $83.2 \pm 2$ kpc & ”\\
$\mu_{\alpha*}$ & $0.099 \pm 0.002 \pm 0.017$ mas yr$^{-1}$ & 3\\
$\mu_\delta$ & $-0.160 \pm 0.002_{\rm stat} \pm 0.017_{\rm sys}$ mas yr$^{-1}$ & ”\\
LOS velocity & $111.2 \pm 0.3\ {\rm km\,s^{-1}}$ & 4\\
$\sigma_v$ & $9.7\pm0.2\ {\rm km\,s^{-1}}$ & ”\\
$R_h$ & $9.79 \pm 0.04$ arcmin & 1\\
ellipticity & $0.37 \pm 0.01$ & ”\\
position angle & $94\pm1^\circ$ & ”\\
$M_V$ & $-10.82\pm0.14$ & ”\\
\bottomrule
\end{tabular}
\end{table*}

\begin{table*}[t]
\centering
\caption[Observed Properties of Ursa Minor]{Observed properties of Ursa Minor. References are: (1) Muñoz et al. (2018) Sérsic fits, (2) Garofalo et al. (2025) RR lyrae distance, (3) Alan W. McConnachie and Venn (2020a), (4) Pace et al. (2020), average of MMT and Keck results. }
\label{tbl:umi_obs_props}
\begin{tabular}{lll}
\toprule
parameter & value & Source\\
\midrule
$\alpha$ & $ 227.2420 \pm 0.0045$˚ & 1\\
$\delta$ & $67.2221 \pm 0.0016$˚ & ”\\
distance modulus & $19.23 \pm 0.11$ & 2\\
distance & $70.1 \pm 3.6$ kpc & ”\\
$\mu_\alpha*$ & $-0.124 \pm 0.004 \pm 0.017$ mas yr$^{-1}$ & 3\\
$\mu_\delta$ & $0.078 \pm 0.004_{\rm stat} \pm 0.017_{\rm sys}$ mas yr$^{-1}$ & ”\\
LOS velocity & $-245.9 \pm 0.3_{\rm stat} \pm 1_{\rm sys}$ km s$^{-1}$ & 4\\
$\sigma_v$ & $8.6 \pm 0.3$ & ”\\
$R_h$ & $11.62 \pm 0.1$ arcmin & 1\\
ellipticity & $0.55 \pm 0.01$ & ”\\
position angle & $50 \pm 1^\circ$ & ”\\
$M_V$ & $-9.03 \pm 0.05$ & ”\\
\bottomrule
\end{tabular}
\end{table*}

\section{Cosmological context}\label{cosmological-context}

We only understand a tiny fraction of the universe's composition. The
leading theory of cosmology, \(\Lambda\)CDM (Lambda Cold Dark Matter),
posits that the universe is composed of about 68\% dark energy
(\(\Lambda\)), 27\% dark matter (DM), and 5\% regular baryons\footnote{Astronomers
  like to change definitions of words. \emph{Baryons} here means
  baryons+leptons, i.e.~any standard model massive fermion.}
\citep{planckcollaboration+2020}. While the composition of dark matter
and dark energy remains elusive, we know their general properties. Dark
energy causes the acceleration of the expansion of the universe on large
scales. We do not discuss dark energy here---it does not substantially
affect the Local Group today. Dark matter, instead, makes up the vast
majority of mass in galaxies. Typically, galaxies have baryonic to dark
matter ratios of between 1:5 to beyond 1:1000 for faint dwarf galaxies.
In \(\Lambda\)CMD, dark matter is assumed to interact only
gravitationally. Light passes through dark matter unimpeded---in this
sense dark matter is transparent. Dark matter is also commonly assumed
to be \emph{cold}, i.e.~typical velocities much smaller than the speed
of light in the early universe. Implications of dark matter properties
range from cosmological structural formation, galaxy structure, and
galaxy interactions.

\subsection{Structure formation and dwarf
galaxies}\label{structure-formation-and-dwarf-galaxies}

The very early universe was almost featureless. Our earliest
observations of the universe stem from the cosmic microwave background
(CMB)---revealing a uniform, isotropic, near-perfect blackbody emission.
But tiny perturbations in the CMB, temperature fluctuations of 1 part in
10,000, reveal the underlying seeds of large-scale cosmological
structure.\footnote{The power spectrum from the CMB is complicated by
  baryons, which lead to oscillations in power based on the
  sound-crossing timescale.} Governed by cosmological expansion,
gravitational collapse, and baryonic physics, each overdensity grows
into larger structures. Initially, baryonic matter was coupled to
radiation and resisted collapse. Dark matter, only influenced by gravity
instead, freely collapsed into the first structures. For mass
perturbations both smaller than the horizon and overdense enough to
collapse, these overdensities of dark matter become self-gravitating,
known as \emph{halos}. After recombination, where electrons combined
with atomic nuclei to form atoms, baryons decoupled from radiation and
fell into the dark matter halos. The densest pockets of baryons later
formed the first stars and galaxies.

Dark matter halos, and their associated galaxies, rarely evolve in
isolation. Instead, structure formation is \emph{hierarchical}. Small
dark matter halos collapse first and hierarchically merge into
progressively larger halos
\citep[e.g.,][]{blumenthal+1984, white+rees1978, white+frenk1991}.
Structure formation happens on a wide range of scales. The largest
structures become the cosmic web---composed of voids, filaments, and
clusters---and the smallest structures directly observable host dwarf
galaxies. Hierarchical assembly is evident through the large scale
structure of the universe, remnants of past mergers within the Milky
Way, and tidal disruption of dwarf galaxies and their streams around
nearby galaxies.

Small-scale structure formation is sensitive to deviations from
\(\Lambda\)CDM cosmology \citep[e.g.,][]{bechtol+2022}. One key
prediction of \(\Lambda\)CDM is that mass perturbations are expected to
exist on all scales, and are largest on the smallest scales, so we would
expect the formation of halos on all scales. Many alternative models,
such as warm dark matter or self-interacting dark matter, may smooth out
small-scale features and reduce the abundance of small halos or change
their structure \citep[e.g.,][]{lovell+2014}. Dwarf galaxies, which
occupy the smallest dark matter halos, are promising windows into
small-scale cosmological features. Alternative dark matter properties
can influence dwarf galaxy abundance, formation, structure, and tidal
evolution. The nearby dwarf galaxies of the Local Group, with detailed
observations, present a promising opportunity to understand the
evolution of these objects and test our understanding of cosmology. To
understand the evolution of these objects, we need to understand the
general predictions of \(\Lambda\)CDM for the properties of dwarf
galaxies.

\emph{Warm} dark matter, relativistic in the early universe but cooler
now, smooths out small-scale features and softens the cusps of dark
matter halos. \emph{Self-interacting} dark matter instead can form cores
but also the cores can collapse into a density peak. \textbf{TODO:
incorporate this here.}

\subsection{Typical structure of dark matter
halos}\label{typical-structure-of-dark-matter-halos}

In \(\Lambda\)CDM cosmological simulations, dark matter halos are
remarkably self-similar. In \citet{NFW1996, NFW1997}, hereafter NFW, the
authors observe that the spherically-averaged density profiles
\(\rho(r)\) are universally well described by a two-parameter law:
\begin{equation}\protect\phantomsection\label{eq:nfw}{
\rho/\rho_s= \frac{1}{(r/r_s)(1+r/r_s)^2},
}\end{equation} where \(r_s\) is a scale radius and \(\rho_s\) a scale
density . This profile has shown remarkable success in describing
\(\Lambda\)CMD halos across several orders of magnitude in mass. NFW
profiles are \emph{cuspy}, where the density continuously rises like
\(\rho \sim 1/r\) at small radii \(r \ll r_s\). The steepness of the
density profile increases gradually with radius, and at large radii the
density falls off like \(\rho \sim 1/r^3\). Fig.~\ref{fig:nfw_density}
shows an example of a NFW halo.

The total mass of an NFW profile diverges, so halos are commonly defined
using an overdensity criterion. The virial mass, \(M_{200}\), is defined
as the mass within a radius \(r_{200}\) containing a mean enclosed
density 200 times\footnote{For the collapse of a uniform spherical
  density, the virialized overdensity would be
  \(\Delta = 18\pi^2\approx 178\) for a critical universe
  \(\Omega_m = 1\). This is commonly rounded to \(\Delta = 200\). While
  this parameter may be closer to \(\Delta \approx 100\) for our
  universe, \(\Delta\) also increases with redshift \citep[using eq. 6
  from][]{bryan+norman1998}.} the critical density of the universe:
\begin{equation}{
M_{200} =200\,\frac{4\pi}{3} \ r_{200}^3\ \rho_{\rm crit}, \qquad {\rm where} \quad \rho_{\rm crit}(z) = 3H(z)^2 / 8\pi G
}\end{equation} A standard second parameter is the halo concentration,
\(c=r_{200} / r_s\) describing how the characteristic size of the halo
compares to its virial radius. In this case, the scale density is a
function of \(c\) alone,
\(\rho_s = (200/3)\,\rho_{\rm crit} c^3 / [\log(1+c) - c/(1+c)]\)
\citep{NFW1996}. However, we characterize halos by their circular
velocity profiles. The circular velocity,
\(v_{\rm circ}(r) = \sqrt{G M(r) / r}\), reaches a maximum of
\(v_{\rm max}\) at radius \(r_{\rm max} \approx 2.16258\,r_s\).
\(v_{\rm max}\) is related to both the total halo mass and the observed
line of sight velocity dispersion, and \(r_{\rm max}\) relates to the
scale radius.

While an NFW profile has two free parameters (\(M_{200}\) and \(c\)),
these are not independent. Lower mass dark matter halos often collapse
earlier, when the universe was denser. As a result, low mass subhalos
tend to be more concentrated \citep[e.g.,][]{NFW1997}. The relationship
between \(M_{200}\) and c, or the mass-concentration relation, describes
the mean trend of concentration with mass
\citep[e.g.,][]{bullock+2001, ludlow+2016}. The left panel of
Fig.~\ref{fig:smhm} illustrates the present-day mass-concentration
relation in terms of \(r_{\rm max}\) and \(v_{\rm max}\). While there is
a general trend for concentration to increase with mass, the
concentration values still have substantial scatter. Other parameters
such as the halo spin or shape may affect the scatter of the
mass-concentration relation, but their effect is typically expected to
be small \citep{navarro+2010, dicintio+2013, dutton+maccio2014}.

\begin{figure}
\centering
\pandocbounded{\includegraphics[keepaspectratio]{figures/example_density_profiles.pdf}}
\caption[Example density profiles]{Density profiles in log density
versus log radius for the stars and dark matter of a Sculptor-like
galaxy. The dark matter is more extended and massive than the star
across the entire galaxy---dynamical evolution is driven by the dark
matter mass distribution.}\label{fig:nfw_density}
\end{figure}

\begin{figure}
\centering
\pandocbounded{\includegraphics[keepaspectratio]{figures/cosmological_means.pdf}}
\caption[Stellar-mass halo-mass relation]{\textbf{Left} Radius of
maximum circular velocity \(r_{\rm max}\) as a function of maximum
circular velocity. The solid line with 1-\(\sigma\) shaded region and
dashed line are the relations from \citet{ludlow+2016} for \(z=0\) and
\(z=2\) respectively. \textbf{Right} Stellar mass (top) as a function of
maximum circular velocity. The solid line with the 1-\(\sigma\) shaded
region is the relation from \citet{fattahi+2018} with scatter points
simulated central galaxies from APOSTLE in
\citet{fattahi+2018}.}\label{fig:smhm}
\end{figure}

\subsection{Connecting dark matter to
stars}\label{connecting-dark-matter-to-stars}

Unfortunately, directly observing dark matter remains infeasible (lest
dark matter wouldn't be dark). Fortunately, dwarf spheroidal galaxies
come with a convenient population of visible stars. Stars typically
represent a small fraction of the total mass of a dwarf galaxy. For
example, Fig.~\ref{fig:nfw_density} shows the typical density profile
for the stars and dark matter for a Sculptor-like galaxy. At all radii,
dark matter are at least \textasciitilde10 times more dense, and the
total virial mass is often \textasciitilde1000 times higher than the
stellar mass alone. As a result, the dynamics and formation of dwarf
galaxies is governed by the underlying dark matter halo.

Both cosmology and observations find fundamental relationships between
the dynamical or total mass and stellar masses of
galaxies.\textbf{Improve last sentence.} A prediction of cosmological
simulations is the Stellar mass Halo Mass relation (SMHM), describing
the mass of stars which form in a halo of a given size. In particular,
the SMHM relation grows especially steep in the dwarf galaxy
regime---many dwarf galaxies are formed in halos of similar mass.
Fig.~\ref{fig:smhm} shows the stellar mass versus \(v_{\rm max}\)
maximum circular velocity (proxy for halo mass) for APOSTLE dwarfs from
\citet{fattahi+2018}. The APOSTLE project \citep{sawala+2016} uses the
hydrodynamical setup from the EAGLE project
\citep{crain+2015, schaye+2015} to simulate Local Group analogues in a
\(\Lambda\)CDM cosmological context (db: reorder these sentences). While
there is some scatter, the range of predicted \(v_{\rm max}\) is fairly
narrow across \(\sim 4\) decades in stellar mass. Because lower mass
galaxies have increasingly shallow potential wells, feedback becomes
more effective at removing gas. Re-ionization additionally suppresses
late star formation in the faintest galaxies. As a result, the resulting
stellar mass of a galaxy is highly sensitive to the dark matter mass,
especially for faint dwarf galaxies. If we know the initial stellar mass
of a dwarf galaxy, we also know, at some level, the properties of its
dark matter halo.

Several challenges complicate a simple SMHM trend including environment,
assembly history, and tidal effects. Additionally, the detailed physics
of galaxy formation can impact the SMHM relationship. By being closer to
a massive host, many dwarfs quench earlier, resulting in lower stellar
masses \citep[e.g.,][]{christensen+2024}. In particular, effects like
ram-pressure stripping (removal of gas in the dwarf galaxy due to
pressure from host's circumgalactic medium) and tidal removal of gas
cause star formation to quench {[}REFS{]}. Additionally, the time of
formation (relative to reionization) can influence the resulting stellar
content \citep{kim+2024}. Finally, tides influence both the dark matter
and stellar mass but in different amounts \citep[e.g.,][]{PNM2008}.
Consequently, tides may reduce the halo mass more than the stellar mass,
adding additional scatter to the SMHM trend, particularly for low-mass
satellites \citep[e.g.,][]{fattahi+2018}. Understanding the effects of
tides on Local Group dwarf galaxies is essential to determine where and
how these galaxies formed in a cosmological context.

\section{Simulating tidal effects}\label{sec:tidal_theory}

Since the discovery of dwarf galaxies, a large body of work has either
considered or simulated tidal effects. While pre-\emph{Gaia} work often
had few constraints on orbits of dwarf galaxies, the theory of tidal
mass loss remains largely the same.

Simulating dwarf galaxies accurately in a cosmological context remains a
substantial challenge. Currently, cosmological simulations can predict
the overall abundance of larger mass? dwarf galaxies {[}e.g., {]} and
broadly predict effects of tides \citep[e.g.,][]{riley+2024}. But, dwarf
galaxies are often near the resolution limit. Insufficient resolution
can lead to artificial disruption of dwarf galaxies and overestimation
of tidal effects \citep[e.g.,][]{santos-santos+2025}. For example, the
highest resolution cosmological simulation of a Milky Way-size dark
matter halo, the Aquarius project \citep{springel+2008}, achieved a DM
resolution of \(1.712 \times10^3 \Mo\), enough to barely resolve the
stellar components of Sculptor like halos (see chapter
\ref{sec:methods}). To address this challenge, idealized simulations
only simulate a single subhalo in an approximate host potential. As a
result, idealized simulations can reach excellent numerical convergence
at the cost of realism. For instance, our simulations later are about 3x
higher resolution than Aquarius at a fraction of the computational cost
(400x less particles). Idealized simulations make numerous
simplifications: neglecting mergers, cosmological context and evolution,
mass assembly, and often baryonic physics. We use idealized simulations
here which are more powerful in accurately assessing tidal effects after
infall, given that the idealizations do not impact these galaxies's
recent history too much.

Some of the earliest theory work on tidal mass loss of dwarf galaxies
originate from \citet{oh+lin+aarseth1995}; \citet{piatek+pryor1995};
\citet{moore+davis1994}; \citet{johnston+spergel+hernquist1995}.
Already, these works used techniques similar to what we continue to use
today and laid the foundation for understanding tidal effects. While the
detailed assumptions have evolved (e.g.~no longer assuming
mass-follows-light), many of their conclusions still hold true. More
recent work expanding this theory include \citet{read+2006};
\citet{bullock+johnston2005}; \citet{PNM2008}; \citet{penarrubia+2009};
\citet{klimentowski+2009}; \citet{errani+2023a}; \citet{fattahi+2018};
\citet{stucker+2023}; \citet{wang+2017}.

With precise orbits and a better understanding of the Milky Way
potential and system, more recent work began to directly probe the
dynamical histories of individual dwarf galaxies (although early work
began this for Sagittarius / etc). Examples include \citet{iorio+2019}
for Sculptor, \citet{borukhovetskaya+2022}; \citet{dicintio+2024} for
Fornax; \citet{borukhovetskaya+2022a} for Antlia II. Our goal is to
apply a similar framework to Sculptor and Ursa Minor.

\textbf{\emph{These past three paragraphs need revisions\ldots{}}}

\subsection{Simulating large gravitational systems: The N-body
method}\label{simulating-large-gravitational-systems-the-n-body-method}

Modelling gravitational evolution for large systems requires special
methods. Perhaps the simplest method to compute the evolution of dark
matter is through \emph{N-body simulations}. A dark matter halo is
represented as a large number of dark matter particles (bodies). Each
body is essentially a Monte Carlo sample of the underlying phase-space
distribution. Note that dark matter (and galaxies) are often assumed to
be \emph{collisionless}---particles are not strongly affected by close,
\emph{collisional} gravitational encounters which substantially change
the momenta of involved bodies. In contrast, star clusters are often
collisional so neglecting these encounters may not be a reasonable
approximation. While we use individual gravitating bodies in N-body
simulations, the Newtonian gravitational force is softened to be a
Plummer sphere as to limit strongly collisional encounters.

Naively, the Newtonian gravitational force requires adding together the
forces from each particle on each particle, with a computational cost
that scales quadratically with the number of particles, or \(O(N^2)\).
With this method, simulating a large number of particles, such as
\(10^6\), would require \(10^{12}\) force evaluations at each time step,
making cosmological and high-resolution studies unfeasible. However,
only long-range gravitational interactions tend to be important for CMD,
so we can utilize the \emph{tree method} to compute the gravitational
force vastly more efficiently.

The first gravitational tree code was introduced in
\citet{barnes+hut1986}, and is still in use today. We utilize the
massively parallel code \emph{Gadget 4} \citep{gadget4}. Particles are
spatially split into an \emph{octotree}. The tree construction stars
with one large node, a box containing all of the particles. If there is
more than one particle in a box/node , the box is then divided into 8
more nodes (halving the side length in each dimension) and this step is
repeated until each node only contains 1 particle. With this
heirarchichal organization, if a particle is sufficiently far away from
a node, then the force is well approximated by the force from the centre
of mass of the node. As such, each force calculation only requires a
walk through the tree, only descending farther into the tree as
necessary to retain accuracy. The total force calculations reduce from
\(O(N^2)\) to \(O(N\,\log N)\), representing orders of magnitude
speedup. Modern codes such as \emph{Gadget} utilize other performance
tricks, such as splitting particles across many supercomputer nodes,
efficient memory storage, adaptive time stepping, and parallel file
writing to retain fast performance for large scale simulations, forming
the foundation for many cosmological simulation codes.

\subsection{Tidal evolution}\label{tidal-evolution}

A galaxy in equilibrium will remain in equilibrium, unless acted upon by
an external force. As an example, Fig.~\ref{fig:lagrange_points}
illustrates the effective potential around a Sculptor-like NFW galaxy in
a circular orbit around a Milky Way-like galaxy. The effective potential
accounts for the centrifugal force in the rotating frame. There are two
key saddle points in Fig.~\ref{fig:lagrange_points}, labeled \(L_1\) and
\(L_2\) (as Lagrange points, there are another 3 not as relevant). Since
a particles trajectory is constrained to be within the equipotential
surface equal to the particle's total energy, the easiest (requiring the
lowest energy) path for a particle to escape the satellites
gravitational influence is through the windows around \(L_1\) and
\(L_2\). Otherwise, the potential steeply increases. \(L_1\) and \(L_2\)
are co-linear with the satellite and host-origin. In general, particles
which have higher energy than the energy at \(L_1\) and \(L_2\) are most
commonly unbound from the galaxy.

The full evolution of a dwarf galaxy in a tidal follows an approximate
sequence (although somewhat concurrent as well.)

\begin{enumerate}
\def\labelenumi{\arabic{enumi}.}
\tightlist
\item
  \emph{Mass is lost}. In particular, particles and stars on weakly
  bound orbits are most likely to be removed by tides. Particles escape
  through \(L_1\) and \(L_2\)
\item
  \emph{Steams form}. Unbound mass becomes part of a stream or tidal
  tails. These particles follow similar orbits but are slightly higher
  or lower energy (depending on which side the particle escaped from).
\item
  \emph{Mass redistributes}. Bound mass of the galaxy redistributes. As
  mentioned in the last section, this is visible as a wave of outward
  moving material, with the outermost material reaching equilibrium
  last.
\item
  \emph{A new equilibrium}. With mass loss, the gravitational potential
  decreases, resulting in a more compact dark matter halo and stars
  which adiabadically expand to a larger scale radius.
\end{enumerate}

A tidally affected galaxy contains predictable observational clues to
this process. Nearby to the galaxy, the density profile is flattened as
a result of newly unbound material. Further away from the galaxy, this
would be visible as a stream. Additionally, a satellite stream contains
a velocity gradient along the path. Finally, due to mass loss, a
satellite will have a reduced velocity dispersion, larger stellar size,
and

Dwarf galaxies evolve along \emph{tidal tracks}. From N-body simulations
across a numerous initial conditions and orbits, NFW halos evolve along
a narrow track in terms of maximum circular velocity and radius
\(v_{\rm max}\) and \(r_{\rm max}\) . \citet{EN2021} derive an empirical
fit to these tidal tracks, finding that

\begin{equation}{
\frac{v_{\rm max}}{v_{\rm max, 0}} = 
2^\alpha 
\left(\frac{r_{\rm max}}{r_{\rm max, 0}}\right)^{\beta}\left[1 + \left(r_{\rm max} / r_{\rm max, 0}\right)^2\right]^{-\beta},
}\end{equation} where \(\alpha=0.4\) and \(\beta=0.65\). As illustrated
in Fig.~\ref{fig:tidal_tracks}, this formula works for both circular and
elliptical orbits and is independent of the initial subhalo size or
distance to the host. \textbf{\emph{Rewrite this paragrah}}

\begin{figure}
\centering
\pandocbounded{\includegraphics[keepaspectratio]{./figures/lagrange_points.pdf}}
\caption[Lagrange points]{The potential contours for a 2-body galaxy
system in the rotating frame. \(L_1\) and and \(L_2\) is most efficient
as the required energy (the effective potential) is the
smallest.}\label{fig:lagrange_points}
\end{figure}

\begin{figure}
\centering
\includegraphics[width=0.8\linewidth,height=\textheight,keepaspectratio]{/Users/daniel/Library/Application Support/typora-user-images/image-20250715095615423.png}
\caption[Tidal tracks of dwarf galaxies]{Tidal tracks of dwarf galaxies,
the logarithm of maximum circular velocity and radius of relative to the
initial conditions for satellites on a variety of orbits. Almost all
satellites follow the tidal track suggested by \citet{EN2021}. Adapted
from fig.~6 of \citet{EN2021}.}\label{fig:tidal_tracks}
\end{figure}

\subsection{Break and tidal radii}\label{sec:break_radii}

Following the detailed discussion of tidal evolution, we can interpret
two simple analytic approximations of where tidal effects should be
visible.

The \textbf{Jacobi radius} represents the approximate radius where stars
become unbound for a galaxy in a circular orbit around a host galaxy.
Calculated from an approximation of the location of the \(L_1\) and
\(L_2\) Lagrange points, the Jacobi radius is where the mean density of
the dwarf galaxy is roughly three times the mean interior density of the
host galaxy at pericentre, or \begin{equation}{
3\bar \rho_{\rm MW}(r_{\rm peri}) \approx \bar \rho_{\rm dwarf}(r_J).
}\end{equation} If \(r_J\) occurs within the visible extent of a galaxy,
we should expect to find relatively clear signs of tidal disturbance.
While strictly valid for circular orbits, assuming \(r_{\rm peri}\) for
the host-dwarf distance works as most stars are lost near pericentre.

We also use the \textbf{break radius} as defined in
\citet{penarrubia+2009}, marking where the galaxy is still in
disequilibrium. The break radius \(r_{\rm break}\) is proportional to
the velocity dispersion ,\(\sigma_v\), and time elapsed since pericentre
,\(\Delta t\), \begin{equation}{
r_{\rm break} = C\,\sigma_{v}\,\Delta t
}\end{equation} where the scaling constant \(C \approx 0.55\) is derived
empirically. \(r_{\rm break}\) describes where the dynamical timescale
is longer than the time since the perturbation, i.e.~the radius within
which the galaxy should have dynamically relaxed. As illustrated in
Fig.~\ref{fig:idealized_break_radius}, for an idealized model with
exponential stars in a NFW halo, shortly after pericentre, the stellar
component is smooth but contains a change in slope around
\(r_{\rm break}\). This radius is visible in the stellar distribution as
non-spherical S-shaped overdensities of stars. Also, \(r_{\rm break}\)
is where where the mean radial velocities of the stars becomes positive,
i.e.~the system is out of equilibrium and adjusting to the new density
profile.

\begin{figure}
\centering
\pandocbounded{\includegraphics[keepaspectratio]{figures/idealized_break_radius.pdf}}
\caption[Break radius validation]{Example density and velocity
distributions of an idealized simulation shortly after pericentre.
\textbf{Top left}: The 2D density profile for the initial and final
simulation with the break radius marked. The break radius of the
simulations is set by the time since pericentre. \textbf{Top right}: The
projected 2D stellar density in the \(x\)-\(y\) plane. The green circle
represents the break radius and the grey arrow points towards the host
centre. \textbf{Bottom left}: the mean radial velocity (dot product of
relative position and velocity relative to dwarf centre) as a function
of 2D radius. \textbf{Bottom right}: The mean radial velocity as
projected into 2D bins. The initial conditions are Sculptor-like,
exponential stars embedded in NFW, evolved in \citet{EP2020} potential
with a pericentre of 10 kpc and apocentre of 100 kpc shortly after the
3rd pericentre. See Section~\ref{sec:methods} for a description of our
simulation setup. \emph{Note to self:} This figure may be too complex,
not sure the best way to present everything
yet.}\label{fig:idealized_break_radius}
\end{figure}

\section{Thesis outline}\label{thesis-outline}

In this thesis, our goal is to review the evidence for an extended
density profile in Ursa Minor and Sculptor, to assess the impact of
tidal effects on each galaxy, and to discuss possible interpretations
for the structure of these galaxies.

In chapter \ref{sec:observations}, we describe how we compute
observational density profiles from \citet{jensen+2024}. In chapter
\ref{sec:methods}, we review our simulation methods. Next, we present
our results for the tidal effects on Sculptor and Ursa Minor in chapter
\ref{sec:results}, We discuss our results, limitations, and implications
in chapter \ref{sec:discussion}. Finally, chapter \ref{sec:summary}
summarizes this work and discuss future directions for similar work and
the field of dwarf galaxies.
