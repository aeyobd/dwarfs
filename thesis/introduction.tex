Dwarf galaxies host, in many ways, the most extreme galactic
environments in the universe. These little galaxies are typically
defined to fainter than the LMC or SMC \citep[\(M_V \gtrsim -18\),
e.g.][\citet{hodge1971}]{mcconnachie2012}. As the smallest class of
galaxies, dwarfs are the most numerous galaxies. Dwarf galaxies are
highly \emph{dark-matter dominated}, with mass to light ratios sometimes
exceeding 100 to 1000 \(M_\odot/ L_\odot\). Because of their small total
masses, many dwarf galaxies are \emph{quenched}, with little to no
recent star formation. Their stellar populations are \emph{relics} from
the early universe, consisting of many of the oldest and most metal poor
stars. Understanding the properties of dwarf galaxies thus has
implications across astronomy, from cosmological structure formation on
the smallest scale to extremely metal-poor stellar populations to
chemical evolution to galactic dynamics. Of the classical dwarfs, the
Sculptor and Ursa Minor dwarf spheroidal galaxies stand out with a more
extended density profile relative to an exponential, hinting at tidal
effects. As a case study in tidal effects and our understanding of dwarf
galaxy formation and evolution, we aim to understand the dynamical
history of these galaxies.

In this section, we first describe cosmological structure formation and
the role of dwarf galaxies. Then, we explore the observational history
of dwarf galaxies, the \emph{Gaia} telescope, open questions, and why
Sculptor and Ursa Minor stand out. We discuss idealized simulations,
tidal effects, break radii, and recent developments in tidal
simulations. Finally, we provide a roadmap to the thesis as a whole at
the end of this section.

\section{Cosmological context}\label{cosmological-context}

We only understand a tiny fraction of the composition of the universe.
The leading theory of cosmology, \(\Lambda\)CDM (cold dark matter),
posits that the universe is composed of about 68\% dark energy
(\(\Lambda\)), 27\% dark matter (DM), and 5\% regular baryons\footnote{Astronomers
  like to change definitions of words. \emph{Baryons} here means
  baryons+leptons, i.e.~any standard model massive fermion. The photon
  energy density is negligible today} \citep{planckcollaboration+2020}.
While the composition of dark matter and dark energy remains elusive, we
know their general properties. Dark energy causes the acceleration of
the expansion of the universe on large scales. We do not discuss dark
energy here---it does not substantially affect the local group today.
Dark matter, instead, makes up the vast majority of mass in galaxies.
Typically, galaxies have baryonic to dark matter ratios of between 1:5
to beyond 1:1000 for faint dwarf galaxies. In \(\Lambda\)CMD, dark
matter is assumed to only interact gravitationally, passing through
matter without effect (transparent to light, or \emph{dark}). Dark
matter is also \emph{cold}, i.e.~typical velocities much smaller than
the speed of light in the early universe. Implications of dark matter
properties range from cosmological structural formation, galaxy
structure, and galaxy interactions.

\subsection{Structure formation and dwarf
galaxies}\label{structure-formation-and-dwarf-galaxies}

The very early universe was almost featureless. Our earliest
observations of the universe stem from the cosmic microwave background
(CMB)---revealing a uniform, isotropic, near-perfect blackbody emission.
But tiny perturbations in the CMB, temperature fluctuations of 1 part in
10,000, became the seeds of large-scale cosmological structure. Governed
by cosmological expansion, gravitational collapse, and baryonic physics,
each perturbation grows into larger structures. Initially, baryonic
matter was coupled to radiation and resisted collapse. Dark matter, only
influenced by gravity instead, freely collapsed into the first
structures. Each self-gravitating overdensity of dark matter is known as
a \emph{halo}. After recombination, where electrons combined with atomic
nuclei to form atoms, baryons decoupled from radiation and fell into the
dark matter halos. The densest pockets of baryons later formed the first
stars and galaxies.

Dark matter halos, and their associated galaxies, rarely evolve in
isolation. Instead, structure formation is \emph{hierarchical}. Small
dark matter halos collapse first and hierarchically merge into
progressively larger halos \citep[e.g.][\citet{blumenthal+1986},
\citet{white+rees1978}, \citet{white+frenk1991},
\citet{somerville+dave2015}]{blumenthal+1984}. Structure formation
happens on a wide range of scales. The largest structures become the
cosmic web---composed of voids, filaments, and clusters---and the
smallest structures directly observable host dwarf galaxies.
Hierarchical assembly is evident through the large scale structure of
the universe, remnants of past mergers within the milky way, and tidal
disruption of dwarf galaxies and their streams around nearby galaxies.

Small-scale structure formation is sensitive to deviations from
\(\Lambda\)CDM cosmology \citep[e.g.][]{bechtol+2022}. Many alternative
models, such as warm dark matter or self-interacting dark matter, may
smooth out small-scale features and reduce the abundance of small halos
or change their structure \citep[e.g.][]{lovell+2014}. Dwarf galaxies,
occupying the smallest dark matter halos, are promising windows into
small-scale cosmological features. Alternative dark matter properties
can influence dwarf galaxy structure, formation, and tidal evolution.
The nearby dwarf galaxies of the local group, with detailed
observations, present a promising opportunity to understand the
evolution of these objects and test our understanding of cosmology. To
understand the evolution of these objects, we need to understand the
general predictions from \(\Lambda\)CDM for the properties of dwarf
galaxies.

\subsection{Cosmological dark matter density
profiles}\label{cosmological-dark-matter-density-profiles}

In \(\Lambda\)CDM cosmological simulations, dark matter halos are
remarkably self-similar. In \citet{NFW1996};\citet{NFW1997}, hereafter
NFW, the authors observe that the spherical radial density profiles
\(\rho(r)\) are well described by a two-parameter law: \begin{equation}{
\rho/\rho_s= \frac{1}{(r/r_s)(1+r/r_s)^2},
}\end{equation} where \(r_s\) is a scale radius and \(\rho_s\) a scale
density . This profile has shown remarkable success in describing
\(\Lambda\)CMD halos across several orders of magnitude in mass. NFW
profiles are \emph{cuspy}, where the density continuously rises like
\(\rho \sim 1/r\) at small radii \(r \ll r_s\). The steepness of the
density profile increases at \(r \sim r_s\) and at large radii, the
density falls off like \(\rho \sim 1/r^3\). Fig.~\ref{fig:nfw_density}
shows an example of a NFW halo and a cored dark matter halo.

The total mass of an NFW profile diverges, so halos are commonly
characterized by an overdensity criterion. The virial mass, \(M_{200}\),
is defined as the mass within a region \(r_{200}\) containing a mean
enclosed density 200 times\footnote{200 smells like an arbitrary value.
  From approximate analytic arguments about the mean density of a
  self-gravitating spherical collapse halo in an expanding universe, the
  expected virial overdensity is XXX, rounding up to 200 (REFS).} the
critical density of the universe: \begin{equation}{
M_{200} =(4\pi/3) \ r_{200}^3\ 200\rho_{\rm crit}, \qquad {\rm where} \quad \rho_{\rm crit}(z) = 3H(z) / 8\pi G
}\end{equation} A standard second parameter is the halo concentration,
\(c=r_{200} / r_s\) describing how the characteristic size of the halo
compares to its virial radius. In this case, the scale density is a
function of \(c\) alone,
\(\rho_s = (200/3)\,\rho_{\rm crit} c^3 / [\log(1+c) - c/(1+c)]\)
\citep{NFW1996}. However, we characterize halos by their circular
velocity profiles. The circular velocity,
\(v_{\rm circ}(r) = \sqrt{G M(r) / r}\), reaches a maximum of
\(v_{\rm max}\) at radius \(r_{\rm max} = \alpha\,r_s\) where
\(\alpha\approx2.16258\). \(v_{\rm max}\) is related to both the total
halo mass and the observed line of sight velocity disperion, and
\(r_{\rm max}\) relates to the scale radius.

While two free parameters characterize a NFW halo, \(M_{200}\) and
\(c\), these parameters are not independent. Smaller dark matter halos
often collapse earlier, when the universe was denser. As a result, small
subhalos tend to be more concentrated \citep[e.g.][]{NFW1997}. The
relationship between \(M_{200}\) and c, or the mass-concentration
relation, how more massive halos become less concentrated in a
predictable way \citep[e.g.][\citet{ludlow+2016}]{bullock+2001}. While
there is a general trend in concentration with mass, the concentration
values still have substantial scatter. Other parameters such as the halo
spin or shape or slight changes in the halo density profile add
additional variability to the present-day distribution of halos
\citep[see e.g.][\citet{dutton+maccio2014}, darkEXP,
ect.]{navarro+2010}.

\begin{figure}
\centering
\pandocbounded{\includegraphics[keepaspectratio]{figures/example_density_profiles.pdf}}
\caption[Example density profiles]{Example stellar and dark matter
density profiles for a Sculptor-like galaxy. The dark matter is more
extended and massive than the star across the entire
galaxy.}\label{fig:nfw_density}
\end{figure}

\subsection{Connecting stars and dark
matter}\label{connecting-stars-and-dark-matter}

Both cosmology and observations find fundamental relationships between
properties of galaxies, particularly for the total mass and luminosity.
A key prediction of \(\Lambda\)CDM is the stellar mass halo mass
relation, describing the amount of stellar mass forms in a (sub)halo of
a given size. In particular, the SMHM relation grows especially steep in
the dwarf galaxy regime---many dwarf galaxies are formed in halos of
similar masses. Fig.~\ref{fig:smhm} shows the stellar mass versus
\(v_{\rm max}\) maximum circular velocity (proxy for halo mass) for
apostle dwarfs from \citet{fattahi+2018}. While there is some scatter,
the range of possible \(v_{\rm max}\) is fairly narrow across \(\sim 5\)
decades in stellar mass. Thus, if we know the initial stellar mass of a
dwarf galaxy, we also know, at some level, its dark matter mass.

Several challenges complicate a simple SMHM trend including environment,
assembly history, and tidal effects. By being closer to a massive host,
most dwarfs quench earlier, resulting in lower stellar masses
\citep[e.g.][]{christensen+2024}. Additionally, the time of formation
(relative to reionization) can influence the resulting stellar content
\citep{kim+2024}. Finally, tides influence both the dark matter and
stellar mass but in different amounts \citep[e.g.][]{PNM2008}.
Consequently, tides may reduce the halo mass more than the stellar mass,
adding additional scatter to the SMHM trend, particularly for Milky Way
satellites {[}@.e.g fattahi+2018{]}. Understanding the effects of tides
on local group dwarf galaxies is essential to determining where and how
these galaxies formed in a cosmological context.

\begin{figure}
\centering
\includegraphics[width=1\linewidth,height=\textheight,keepaspectratio]{/Users/daniel/Library/Application Support/typora-user-images/image-20250715100059332.png}
\caption[Stellar-mass halo-mass relation]{The stellar mass halo mass
relation, where \(v_{\rm max}\) is related to the halo mass. Taken from
\citet{fattahi+2018}. Right: the mass-concentration relation for NFW
halos, but parameterized in terms of \(r_{\rm max}\) and
\(v_{\rm max}\). Together, these plots represent the cosmologically
expected properties of underlying dark matter halos in any dwarf galaxy.
The velocity dispersion directly constrains the mass contained within a
half-light radius, so the underlying halo in \(\Lambda\)CDM is
well-constrained.}\label{fig:smhm}
\end{figure}

\section{Observational context}\label{observational-context}

\subsection{Early observations}\label{early-observations}

Dwarf galaxies have often puzzled astronomers and our understanding of
the universe. While we now generally understand how dwarf galaxies fit
into the universe, this was not always the case. The first discovered
dwarf galaxies in 1938\footnote{technically, the LMC and SMC may be
  classified as dwarf galaxies, but these were likely always known to
  humans at southern latitudes.}, Sculptor and Fornax, heralds an era of
discovery and investigation into these astronomical system.
\citet{shapley1938} present Sculptor and Fornax as a new type of
\emph{stellar system} resembling the Magellanic clouds and globular
clusters, but do not attempt to speculate on the exact nature. The first
direct hints that these systems are galaxies originates from early
spectroscopic work deriving the first velocity dispersions
\citep[e.g.,][\citet{gallagher+wyse1994}, \citet{mateo1994},
\citet{pryor1994}, \citet{pryor1996}, \citet{gerhard1994}, and
\citet{olszewski1998}.]{aaronson1983}. Based on the physical size and
velocity dispersions, the mass to light ratios of these dwarf galaxies
is far larger than expected for star clusters without dark matter. In
general, these objects have reached a consensus that they have high
velocity dispersion, old spheroidal stellar populations, and low stellar
masses.

Dwarf galaxies span a large range of size and morphology.
Fig.~\ref{fig:galaxy_images} displays several images of dwarf galaxies.
The largest in the local group but not always considered a dwarf galaxy,
the Large Magellanic Cloud (LMC), to classical systems Sculptor and Ursa
Minor dwarf spheroidal, to an ultra-faint dwarf galaxy, Boötes V. Large
LMC-like dwarf galaxies exhibit complex structure such as a bar, young
blue stars, and irregular shapes. On the other hand, the classical and
faint dwarf galaxies are composed mostly of old, faint, and red stars.
These galaxies are typically classified as dwarf spheroidals (dSph)
(although we typically omit the dSph designation and refer to these
galaxies by their constellation alone).

Since the discovery of dwarf galaxies around the Milky Way,
observational work has attempted to measure and refine their basic
properties \citep[e.g.][]{mateo1998}. While the Milky Way's satellites
are close (by extragalactic standards), their low numbers of stars and
large apparent sizes present challenges for observational work. In
particular, the properties of the faintest dwarf galaxy systems and
especially \emph{ambiguous} systems remains a pressing challenge as a
unique probe into the nature of the early universe and dark matter.

\emph{Classical systems}, the brightest of the local group
(non-Magellanic) dwarf galaxies, are perhaps the systems best studied
and with the most stringently derived properties. Defined\footnote{Note
  that some work like \citet{simon2019} instead define classical dwarfs
  based on luminosity as compared to ultra faint dwarfs. This may
  include galaxies such as Antlia I, Crater II, Canes Venatici I, and
  Sagittarius. We stick with the historical definition for simplicity
  and since the Gaia observations are most numerous for the historical
  classicals.} as the first six discovered dwarf spheroidal galaxies,
(Sculptor, Ursa Minor, Fornax, Draco, Leo I, Leo II, and Carina),
classical systems are typically old, spheroidal, and gas-poor. While
extending 1-2 degrees across the sky (e.g.
Figs.~\ref{fig:scl_selection}, \ref{fig:umi_selection}, \ref{fig:fornax_selection}),
these dwarfs contain large numbers of bright (giant) stars, allowing
thousands of stars to be observed with deep photometry and spectroscopy
\citep[e.g.][\citet{pace+2020}]{tolstoy+2023}. As a result, the
determination of fundamental properties such as the position, size,
orientation, distance \citep[from 100s of RRL stars,
e.g.][]{tran+2022, garofalo+2025}, proper motions (from \emph{Gaia},
e.g. \citet{battaglia+2022}, \citet{MV2020a}), line-of-sight (LOS)
velocity and dispersion \(\sigma_v\), are all relatively well
constrained today and show excellent convergence among different works.
However, ongoing research continues to redefine our understanding of the
detailed structure of Milky Way satellites, hinting that these objects
may be more complex than at first glance.

\begin{figure}
\centering
\includegraphics[width=5.41667in,height=5.41667in]{figures/galaxy_pictures.pdf}
\caption[Dwarf Galaxy Pictures]{Images of the LMC (DSS2), Sculptor (DES
2), Ursa Minor (UNWISE with Gaia point sources over-plotted), and Bootes
V (SDSS). While the LMC is very prominent in the sky, even the classical
dwarf galaxies are not obvious except in terms of star counts (e.g.
Fig.~\ref{fig:scl_selection}). All (non-LMC) dwarfs occupy the central
third to sixth of the image.}\label{fig:galaxy_images}
\end{figure}

\begin{figure}
\centering
\includegraphics[width=5.41667in,height=\textheight,keepaspectratio]{figures/mw_satellites_1.jpg}
\caption[Dwarf galaxies sky position]{The location of MW dwarf galaxies
on the sky. We label the classical dwarf galaxies (green diamonds),
fainter dwarfs (blue squares), globular clusters (orange circles), and
ambiguous systems (pink open hexagons). Globular clusters are more
centrally concentrated, but dwarf galaxies are preferentially found away
from the MW disk. Sculptor and Ursa Minor are highlighted as two dwarfs
we study later. The background image is from ESA/Gaia/DPAC
(https://www.esa.int/ESA\_Multimedia/Images/2018/04/Gaia\_s\_sky\_in\_colour2).
Dwarf galaxies (confirmed), globular clusters, and ambiguous systems are
from the \citet{pace2024} catalogue (version
1.0.3).}\label{fig:mw_satellite_system}
\end{figure}

\subsection{\texorpdfstring{The era of
\emph{Gaia}}{The era of Gaia}}\label{the-era-of-gaia}

Some of the most fundamental properties of astronomical objects are
their positions and velocities. Unfortunately, determining distances to
stars is nontrivial. Additionally, while line-of-sight (LOS) velocities
are easily determined from spectroscopy, the tangental velocities,
perpendicular to the LOS, are best measured through proper motions,
typically requiring accurate determinations of small changes in a star's
position. \emph{Gaia}'s mission addresses these problems by making
extremely precise measurements of stellar positions over time.

\emph{Gaia} was designed to revolutionize proper motion and parallax
measurements. \emph{Gaia} is a space-based, all-sky survey telescope
with two primary 1.45x0.5m mirrors situated at the Sun-Earth L2 Lagrange
point \citep{gaiacollaboration+2016}. \emph{Gaia} was launched in 2013,
completing its space-based mission in 2025 (with two more data releases
planned). By imagining two patches of sky on the same focal plane,
separated by a fixed angle of 106.5 degrees, \emph{Gaia} is able to
measure absolute proper motions. Stars in different regions of the sky
are affected by parallax differently, so by precisely observing the
relative positions of stars separated by large angles during multiple
epochs over a year, an absolute all-sky reference frame can be derived
\citep{gaiacollaboration+2016}. In addition to precise astrometric
information, \emph{Gaia} measures the magnitude of stars in the very
wide \emph{G} band (330-1050nm), blue and red colours using the blue and
red photometers (BP and RP, 330-680 nm and 640-1050 nm respectively),
and takes low resolution BP-RP spectra and radial velocity measurements
of bright stars (\emph{Gaia} radial velocity magnitudes \textless16)
\citep{gaiacollaboration+2016}. For our purposes, most useful are
\emph{Gaia}'s measurements of \(G\) magnitude,
\(G_{\rm BP} - G_{\rm RP}\) colour, the position \(\alpha, \delta\), and
the proper motions in RA and declination \(\mu_{\alpha*}\) and
\(\mu_\delta\).\footnote{\(\mu_{\alpha*}\) allows where
  \(\mu_{\alpha*} = \mu_\alpha \cos \delta\) corrects for projection
  effects{]}.}

\emph{Gaia} has revolutionized many astronomical disciplines, the least
of which is local group and Milky Way science. While proper motions of a
dwarf galaxies has been measured in a case by case basis by the Hubble
Space Telescope, a full systematic determination of proper motions for
most dwarf galaxies was unavailable until \emph{Gaia}
\citep{MV2020a, battaglia+2022}. \emph{Gaia} has furthermore allowed for
the detection and improved measurements of halo substructures and
streams \citep{bonaca+price-whelan2025}, Milky Way structure and
kinematics \citep{hunt+vasiliev2025}.

For Local group dwarf spheroidal, \emph{Gaia} has allowed for many of
the first proper motion measurements. While the proper motion
uncertainty on a typical dwarf member star is often with large
uncertainties, by combining the proper motions for 100s to 1000s of
stars in \emph{Gaia}, incredibly precise proper motion measurements can
be determined \citep[e.g.][]{MV2020a}, sometimes only limited by
\emph{Gaia}'s systematic error floor. Proper motions have ushered in a
new dynamical era for MW satellites studies, where we often have precise
determinations of the dwarf galaxy's (short term) orbit under a given MW
potential.

\subsection{Dwarf galaxies today}\label{dwarf-galaxies-today}

With the help of \emph{Gaia}, high resolution spectroscopy, high-quality
photometry, and large aperture multi-object spectroscopic instruments,
we can observe dwarf galaxies in unprecedented detail. Today, dwarf
galaxies are used to search for dark matter cores (REFS), test our
understanding of early nucleosynthesis (REFS), the details of galaxy
formation (REFS), and \ldots.

While we now know the basic properties of dwarf galaxies, tremendous
progress has been made in the past 10-20 years with large area deep
photometric surveys, multi-object spectrographs, high resolution
chemistry, and theoretical work, unveiling that dwarf galaxies are each
unique systems with different formation pathways
\citep[e.g.][]{simon2019}.

\citet{battaglia+nipoti2022}

One of the open questions in dwarf galaxy evolution is the formation of
dark matter \emph{cores} where the density becomes constant instead
\(\rho \sim 1\). A leading theory is cores form through of baryonic
effects and feedback. If dark matter deviates from \(\Lambda\)CDM, then
the NFW density profile may not describe halos anymore. \emph{Warm} dark
matter, relativistic in the early universe but cooler now, smooths out
small-scale features and softens the cusps of dark matter halos.
\emph{Self-interacting} dark matter instead can form cores but also the
cores can collapse into a density peak.

\begin{itemize}
\tightlist
\item
  \citet{walker+penarrubia2011}: measuring cores and cusps
\item
  \citet{dicintio+2013}, \citet{navarro+2010}, NFW profiles may actually
  be more Einasto like
\end{itemize}

\subsection{Stellar density profiles}\label{stellar-density-profiles}

Given star counts for a local group dwarf, one of the main quantities of
interest is the surface density profile. Long used to characterize and
understand larger galaxies, these profiles can provide hints as to the
formation of dwarf galaxies, determine the physical extent of the
stellar component, and the total stellar mass. Previous works tend to
focus on fitting 3 different surface brightness profiles: an
Exponential, a Plummer, or a King profile.

The exponential profile is perhaps the simplest, as a 1-parameter
profile \begin{equation}{
\Sigma_{\rm exp} = \Sigma_0\exp(-R / R_s)
}\end{equation}

For a long time, exponential surface density profiles have been used as
an description for spiral galaxies {[}REFS{]}.

\citet{plummer1911} proposed a 1-parameter solution for a polytropic
potential to fit globular cluster density profiles:

\begin{equation}{
\Sigma_{\rm Plummer} = \frac{M}{\pi R_h^2}\frac{1}{(1 + (R/R_h)^2)^2}
}\end{equation}

The \citet{king1962} profile, originating as an empirical fit to
globular clusters, also sometimes reasonably fits dwarf galaxies. This
density profile includes an additional parameter \(R_t\), a truncation
radius where the density sharply drops to 0. \begin{equation}{
\Sigma_{\rm King} = \Sigma_0\left(\frac{1}{\sqrt{1 + (R/R_c)^2}} - \frac{1}{\sqrt{1+(R_t/R_c)^2}}\right)
}\end{equation}

Finally, the Sérsic profile represents an extension of the exponential.
Typically parameterized in terms of a half-light radius \(R_h\), the
density at half-light radius \(\Sigma_h\) and a Sérsic index \(n\), the
profile's equation is \begin{equation}{
\Sigma_{\rm Sérsic} = \Sigma_0 \exp\left[-b_n \,  \left((R/R_h)^{1/n} - 1\right)\right]
}\end{equation} where \(b_n\) is a constant depending on \(n\). Less
commonly used in studies of dwarf galaxies, \citet{munoz+2018} advocate
for using the Sérsic profile since the added flexibility allows for more
profiles to be fit.

While there are not clear theoretical explanations why any profile is
best, the Exponential stellar density profile is commonly assumed as a
single parameter density profile which fits many dwarf galaxies well
\citep[e.g.][\citet{eskridge1988a}, \citet{hodge1991a},
\citet{hodge1991b}, \citet{IH1995}, etc.]{MV2020a} \citet{faber+lin1983}
demonstrated that an Exponential density profile is a reasonable
empirical fit to dwarf galaxies, theorizing that dwarf spheroidal evolve
from exponential disks, maintaining a similar light profile.

in addition to Sérsic, King and other profiles. \citet{lelli+2014}
(example of using exponential.) \citet{wang+2019} for fornax,
\citet{mcconnachie+irwin2006} for Exp, Plummer, King in other galaxies.

However, an exponential profile was often noted to not fit every dwarf
galaxy. Even starting from \citet{aparicio1997}, some people have noted
deviations from this simple empirical rule. Additionally,
\citet{herrmann+hunter+elmegreen2013},
\citet{herrmann+hunter+elmegreen2016} note that at least in a
photometric sample of more distant dwarf disky/Irr/blue compact dwarfs,
that many dwarfs show deviations from exponential profiles.

\begin{itemize}
\tightlist
\item
  \citet{makarov+2012} isolated local volume dwarf galaxy with central
  deprivation compared to exponential, spheroidal
\item
  \citet{martin+2016}, exponential fits to many pandas dwarf galaxies
\item
  \citet{moskowitz+walker2020}, generalized plummer fits.
\end{itemize}

\subsection{Sculptor and Ursa Minor: Hints of tidal
signatures?}\label{sculptor-and-ursa-minor-hints-of-tidal-signatures}

Sculptor and Ursa Minor may appear to be typical dwarf galaxies at first
glance. Table~\ref{tbl:scl_obs_props}. describe the present-day
properties of each galaxy. Indeed, Sculptor as the first discovered
dwarf galaxy, is often described as a ``prototypical'' dSph. However,
both galaxies have a long history of speculation that they may be
influenced by the Milky Way's gravity (see discussion). Only recently,
with \emph{Gaia} data and spectroscopic followup, has the detection of a
density excess become more unmistakable. Using \citet{jensen+2024}'s
algorithm (described briefly below), \citet{sestito+2023a},
\citet{sestito+2023b} report that a ``kink'' in the density profile,
beginning around 30 arcmin for both Sculptor and Ursa Minor. They
spectroscopically followup some of the most distant stars, finding
multiple members between 6-12 half-light radii from the centre of each
dwarf (REF fig.~XXX). If dwarfs initially begin with exponential
profiles like Fornax, then these stars should be much rarer.

Perhaps the most straighforward explanation for the density profiles of
Sculptor and Ursa Minor are tidal interactions. \citet{sestito+2023a},
\citet{sestito+2023b} conclude that tides are likely the explanation.
Fig.~\ref{fig:scl_umi_fnx_vs_penarrubia} reproduces their comparison,
showing that Sculptor and Ursa Minor's density profiles deviate
substantially from fornax's in the outskirts, and are well-described by
the tidal model from \citet{PNM2008}. Our goal is to determine, assuming
\(\Lambda\)CMD, if the effects of the Milky Way (or other satellites)
may indeed create observable tidal signatures in these galaxies. If
tides cannot explain these features, Sculptor and Ursa Minor may instead
contain a extended stellar ``halo'' or second component of the
galaxy---illustrating evidence of complex history or formation in each
galaxy and forcing formation models to confront a diversity of density
profiles in the local group.

\begin{figure}
\centering
\pandocbounded{\includegraphics[keepaspectratio]{./figures/scl_umi_vs_penarrubia.pdf}}
\caption[Sculptor and Ursa Minor match tidal models]{A plot of the
surface density profiles of Sculptor, Ursa Minor, and Fornax scaled to
their half-light radius and the density at half-light radius (data
described in Section~\ref{sec:data}). The lines represent the initial
and final models from \citet{PNM2008}'s model (Fig. 4 top left, stellar
segregation of 0.20 moving on a highly eccentric orbit peri:apo = 1 :
100)).}\label{fig:scl_umi_vs_penarrubia}
\end{figure}

\begin{table*}[t]
\centering
\caption[Observed Properties of Sculptor]{Observed properties of Sculptor. References are: 1. Muñoz et al. (2018) Sérsic fits, 2. Tran et al. (2022) RR lyrae distance, 3. Alan W. McConnachie and Venn (2020b), 4. Arroyo-Polonio et al. (2024). }
\label{tbl:scl_obs_props}
\begin{tabular}{lll}
\toprule
parameter & value & Source\\
\midrule
$\alpha$ & $15.0183 \pm 0.0012^\circ$ & 1\\
$\delta$ & $-33.7186 \pm 0.0007^\circ$ & ”\\
distance modulus & $19.60 \pm 0.05$ & 2\\
distance & $83.2 \pm 2$ kpc & ”\\
$\mu_{\alpha*}$ & $0.099 \pm 0.002 \pm 0.017$ mas yr$^{-1}$ & 3\\
$\mu_\delta$ & $-0.160 \pm 0.002_{\rm stat} \pm 0.017_{\rm sys}$ mas yr$^{-1}$ & ”\\
LOS velocity & $111.2 \pm 0.3\ {\rm km\,s^{-1}}$ & 4\\
$\sigma_v$ & $9.7\pm0.2\ {\rm km\,s^{-1}}$ & ”\\
$R_h$ & $9.79 \pm 0.04$ arcmin & 1\\
ellipticity & $0.37 \pm 0.01$ & ”\\
position angle & $94\pm1^\circ$ & ”\\
$M_V$ & $-10.82\pm0.14$ & ”\\
\bottomrule
\end{tabular}
\end{table*}

\begin{table*}[t]
\centering
\caption[Observed Properties of Ursa Minor]{Observed properties of Ursa Minor. References are: (1) Muñoz et al. (2018) Sérsic fits, (2) Garofalo et al. (2025) RR lyrae distance, (3) Alan W. McConnachie and Venn (2020a), (4) Pace et al. (2020), average of MMT and Keck results. }
\label{tbl:umi_obs_props}
\begin{tabular}{lll}
\toprule
parameter & value & Source\\
\midrule
$\alpha$ & $ 227.2420 \pm 0.0045$˚ & 1\\
$\delta$ & $67.2221 \pm 0.0016$˚ & ”\\
distance modulus & $19.23 \pm 0.11$ & 2\\
distance & $70.1 \pm 3.6$ kpc & ”\\
$\mu_\alpha*$ & $-0.124 \pm 0.004 \pm 0.017$ mas yr$^{-1}$ & 3\\
$\mu_\delta$ & $0.078 \pm 0.004_{\rm stat} \pm 0.017_{\rm sys}$ mas yr$^{-1}$ & ”\\
LOS velocity & $-245.9 \pm 0.3_{\rm stat} \pm 1_{\rm sys}$ km s$^{-1}$ & 4\\
$\sigma_v$ & $8.6 \pm 0.3$ & ”\\
$R_h$ & $11.62 \pm 0.1$ arcmin & 1\\
ellipticity & $0.55 \pm 0.01$ & ”\\
position angle & $50 \pm 1^\circ$ & ”\\
$M_V$ & $-9.03 \pm 0.05$ & ”\\
\bottomrule
\end{tabular}
\end{table*}

\section{Simulating tidal effects}\label{simulating-tidal-effects}

Since the discovery of dwarf galaxies, a large body of work has
speculated, considered, or simulated tidal effects. While
pre-\emph{Gaia} work often did not know the specific orbits of dwarf
galaxies, the theory for satellites serves as an excellent framework for
understanding specific systems.

While cosmological simulations predict that mergers and satellite-galaxy
interactions are common \citep[e.g.,][]{riley+2024}, they struggle to
understand the evolution of individual dwarf galaxies. In particular,
dwarf galaxies are often barely resolved. The highest resolution
simulation of a Milky Way dark matter halo, the Aquarius project
\citep{springel+2008}, acchieved a DM resolution of
\(1.712 \times10^3 \Mo\), enough to barely resolve Sculptor like halos
(see methods). On the other hand, idealized simulations are able to
reach high resolution and numerical convergence for a single dwarf
galaxy. But, idealization may not be the most realistic environment,
neglecting mergers, assembly history, and often baryonic physics. We use
idealized simulations here which are more powerful in accurately
assessing tidal effects, given that the idealizations do not impact
these galaxies's recent history too much. As such, a long history of
work has resolved to understand the tidal evolution of dwarf galaxies
using these \emph{idealized} simulations.

Some of the earliest theory work on tidal stripping of dwarf galaxies
originate from \citep[\citet{moore+davis1994},
\citet{johnston+spergel+hernquist1995}, \citet{oh+lin+aarseth1995},
\citet{piatek+pryor1995}, \citet{velazquez+white1995},
\citet{kroupa1997}]{allen+richstone1988}. Many of these works used
similar techniques as we continue to use today, setting up a dark matter
halo in a static Milky Way potential and predicting the evolution of
quantities of interest.

More recently, \citet{PNM2008} run idealized simulations of a local
group dwarf galaxy in a milky way halo.

\begin{itemize}
\tightlist
\item
  \citet{mayer+2001} theory of tidal stripping
\end{itemize}

With precise orbits and a better understanding of the Milky Way
potential and system, more recent work began to directly probe the
dynamical histories of individual dwarf galaxies. Examples include
\citet{iorio+2019} for Sculptor, \citet{borukhovetskaya+2022};
\citet{dicintio+2024} for Fornax, \citet{borukhovetskaya+2022a} for
Antlia II,

Our goal is to apply a similar framework to Sculptor and Ursa Minor.

\subsection{Simulating large gravitational systems: The N-body
method}\label{simulating-large-gravitational-systems-the-n-body-method}

Modelling gravitational evolution is essential for understanding
properties of galaxies. Perhaps the simplest method to compute the
evolution of dark matter is through \emph{N-body simulations}. A dark
matter halo is represented as a large number of dark matter particles
(bodies). Each body represents a monte-carlo sample from the underlying
matter distribution. However, galaxies are often assumed to be
\emph{collisionless}---particles are not strongly affected by
small-scale gravitational encounters. In contrast, star clusters evolve
differently as star-star gravitational collisions changes the dynamics
of the system. While we use individual gravitating bodies in N-body
simulations, the Newtonian gravitational force is softened to be a
Plummer sphere, so encounters closer than a softening length do not
substantially impact the dynamical evolution.

Naively, the newtonian gravitational force requires adding together the
forces from each particle at each particle, causing a computational cost
that scales quadratically with the number of particles, or \(O(N^2)\).
With this method, simulating a large number of particles, such as
10\^{}6, would require about 10\^{}12 force evaluations at each time
step, making cosmological and high-resolution studies unfeasible.
However, only long-range gravitational interactions tend to be important
for CMD, so we can utilize the \emph{tree method} to compute the
gravitational force vastly more efficiently.

The first gravitational tree code was introduced in
\citet{barnes+hut1986}, and is still in use today. We utilize the
massively parallel code \emph{Gadget 4} \citep{gadget4}. Particles are
spatially split into an \emph{octotree}. The tree construction stars
with one large node, a box containing all of the particles. If there is
more than one particle in a box/node , the box is then divided into 8
more nodes (halving the side length in each dimension) and this step is
repeated until each node only contains 1 particle. With this
heirarchichal organization, if a particle is sufficiently far away from
a node, then the force is well approximated by the force from the centre
of mass of the node. As such, each force calculation only requires a
walk through the tree, only descending farther into the tree as
necessary to retain accuracy. The total force calculations reduce from
\(O(N^2)\) to \(O(N\,\log N)\), representing orders of magnitude
speedup. Modern codes such as \emph{Gadget} utilize many other
performance tricks, such as splitting particles across many
supercomputer nodes, efficient memory storage, adaptive time stepping,
and parallel file writing to retain fast performance for large scale
simulations, forming the foundation for many cosmological simulation
codes.

\subsection{Break and tidal radii}\label{break-and-tidal-radii}

Analytic approximations are powerful tools to generally determine the
influence of tides given a dwarf galaxy and a host.

The Jacobi radius represents the approximate radius where stars become
unbound for a galaxy in a circular orbit around a host galaxy.
Calculated from an approximation of the location of the L1 and L2
lagrange points, the Jacobi radius is where the mean density of the
dwarf galaxy is three times the mean interior density of the host galaxy
at pericentre, or \begin{equation}{
3\bar \rho_{\rm MW}(r_{\rm peri}) = \bar \rho_{\rm dwarf}(r_J).
}\end{equation} If \(r_J\) occurs within the visible extent of a galaxy,
we should expect to find unbound, \emph{extratidal} stars. While most
valid for circular orbits, assuming \(r_{\rm peri}\) for the host-dwarf
distance works as most stars are lost near pericentre.

We use the break radius as defined in \citet{penarrubia+2009}, marking
where the galaxy is still in disequilibrium. The break radius
\(r_{\rm break}\) is proportional to the velocity dispersion
\(\sigma_v\) and time since pericentre \(\Delta t\), \begin{equation}{
r_{\rm break} = C\,\sigma_{v}\,\Delta t
}\end{equation} where the scaling constant \(C \approx 0.55\) is a fit.
\(r_{\rm break}\) describes where the dynamical timescale is longer than
the time since the perturbation, i.e.~the radius within which the galaxy
should have dynamically relaxed. As illustrated in
Fig.~\ref{fig:toy_profiles}, for an idealized model with exponential
stars in a NFW halo, after some time past pericentre, the stellar
component is smooth but contains a change in slope around
\(r_{\rm break}\), where the radial velocities of the stars becomes
positive as they still readjust to a new equilibrium.

From this argument, we note that the following properties must be
approximately true for tides to occur:

\begin{itemize}
\tightlist
\item
  Close enough pericentre. The other break radius \(r_J\) implies that
  if the host density is 3x the satellite, stars will be lost
\item
  Corresponding time since last pericentre: If the time since last
  pericentre is not \(\sim\) consistent with an observed break in the
  density profile, then tides
\item
  Halo evolution. As found in \citet{EN2021}, galaxies evolve along well
  defined tidal tracks (assuming spherical, isotropic, NFW halo, which
  may not be true, see \ldots). These tracks tend to ``puff up'' the
  stellar component while also removing dark matter mass, leaving a
  smaller, compacter DM halo with a more extended stellar component.

  \begin{itemize}
  \tightlist
  \item
    This information is mostly related to the statistical initial
    distribution of satellites from cosmology {[}ludlow+2016;
    \citet{fattahi+2018}{]}
  \end{itemize}
\end{itemize}

\begin{figure}
\centering
\pandocbounded{\includegraphics[keepaspectratio]{figures/idealized_break_radius.pdf}}
\caption[Break radius validation]{The break radius of the simulations is
set by the time since pericentre. The initial conditions are
Sculptor-like, exponential stars embedded in NFW, evolved in
\citet{EP2020} potential with a pericentre of 15 kpc and apocentre of
100 kpc. In this model, the jacobi radius is close to the break radius.
See Section~\ref{sec:methods} for a description of our simulation
setup.}\label{fig:idealized_break_radius}
\end{figure}

\subsection{Tidal evolution}\label{tidal-evolution}

While the break and jacobi radii help with understanding when tidal
effects matter, we still need to know what these effects do and how a
galaxy evolves.

As a galaxy evolves in a tidal field, several changes happen.

\begin{enumerate}
\def\labelenumi{\arabic{enumi}.}
\tightlist
\item
  \emph{Mass is lost}. In particular, particles and stars on weakly
  bound orbits are most likely to be removed by tides.
\item
  \emph{Steams form}. If tides are strong enough, mass lost from the
  galaxy becomes part of tidal tails, evolving along a similar orbit but
  leading or trailing the galaxy.
\item
  \emph{The galaxy reshapes}. Bound mass of the galaxy redistributes. As
  mentioned in the last section, this is visible as a wave of outward
  moving material, with the outermost material reaching equilibrium
  last.
\item
  \emph{A new equilibrium}. With mass loss, the gravitational potential
  decreases, resulting in a more compact dark matter halo and stars
  which adiobadically expand to a larger scale radius.
\end{enumerate}

Assuming that galaxies are spherical, isotropic, and evolve in a
constant tidal field, most dwarf galaxies should evolve along similar
tidal tracks.

\citet{EN2021} derive tidal tracks for dwarf galaxies, showing that NFW
halos all evolve along a similar trajectory in terms of \(r_{\rm max}\)
and \(v_{\rm max}\). \begin{equation}{
\frac{v_{\rm max}}{v_{\rm max, 0}} = 
2^\alpha 
\left(\frac{r_{\rm max}}{r_{\rm max, 0}}\right)^{\beta}\left[1 + \left(r_{\rm max} / r_{\rm max, 0}\right)^2\right]^{-\beta}
}\end{equation} where \(\alpha=0.4\), \(\beta=0.65\) are empirical fits.
As illustrated in Fig.~\ref{fig:tidal_tracks}, this formula works for
both circular and elliptical orbits and is independent of the initial
subhalo size or distance to the host.

Case studies:

Dynamical

\begin{itemize}
\tightlist
\item
  \citet{read+2006}
\item
  \citet{bullock+johnston2005}
\item
  \citet{PNM2008}
\item
  \citet{errani+2023a}, \citet{fattahi+2018}
\item
  \citet{wang+2017},
\item
  \citet{pryor1996}
\item
  \citet{klimentowski+2009}
\end{itemize}

\begin{figure}
\centering
\includegraphics[width=1\linewidth,height=\textheight,keepaspectratio]{/Users/daniel/Library/Application Support/typora-user-images/image-20250715095615423.png}
\caption[Tidal tracks of dwarf galaxies]{Tidal tracks of dwarf galaxies.
Fig. 6 from \citet{EN2021}.}\label{fig:tidal_tracks}
\end{figure}

\section{Thesis Outline}\label{thesis-outline}

In this thesis, our goal is to review the evidence for an extended
density profile in Ursa Minor and Sculptor, to assess the impact of
tidal effects on each galaxy, and to discuss possible interpretations
for the structure of these galaxies.

In chapter 2, we describe how we compute observational density profiles
from \citet{jensen+2024}. In chapter 3, we review our simulation
methods. Next, we present our results for the tidal stripping of
Sculptor and Ursa Minor in Chapter 4. We discuss our results,
limitations, and implications in Chapter 5. Finally, in Chapter 6, we
summarize this work and discuss future directions for similar work and
the field of dwarf galaxies.
