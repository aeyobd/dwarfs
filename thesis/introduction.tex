Dwarf galaxies host, in many ways, the most extreme galactic
environments in the universe. These galaxies are typically defined to be
fainter than the Large Magellanic Cloud (LMC), with \(M_V \gtrsim -18\)
or similarly \(M_\star \lesssim 10^9 M_\odot\)
\citep[e.g.,][]{hodge1971, mcconnachie2012}. Because the galaxy
luminosity function increases towards fainter objects, dwarfs are the
most numerous galaxies in the Universe
\citep[e.g.,][]{blanton+2005, mao+2021}. Dwarf galaxies are also highly
\emph{dark-matter dominated}, with mass to light ratios which may exceed
1000 \(M_\odot/ L_\odot\) \citep[e.g.,][]{simon+geha2007, hayashi+2023}.

With the exception of the Magellanic Clouds, most dwarf galaxy
satellites of the Milky Way (MW) are \emph{quenched}, with little to no
recent star formation \citep[e.g.,][]{weisz+2014}. Indeed, most faint MW
satellites contain old stellar populations which are \emph{relics} from
the early universe, consisting of many of the oldest and most metal poor
stars known \citep{simon2019}. Understanding the properties of dwarf
galaxies thus has implications across astronomy, from cosmological
structure formation on small scales to the formation of metal-poor
stellar populations.

In this Chapter, we first describe the general observed properties of
local dwarf galaxies. Next, we summarize our understanding of the
cosmological origin of dwarf galaxies. We later review recent
advancements and pending questions concerning dwarf galaxies, and
introduce the puzzle posed by the extended stellar density profiles of
Sculptor and Ursa Minor. We end with a brief roadmap to the remainder of
this dissertation.

\section{Observations of dwarf
galaxies}\label{observations-of-dwarf-galaxies}

Dwarf galaxies have long raised conundrums for theories of galaxy
formation. The discovery of Fornax and Sculptor in 1938
\citep{shapley1938}\footnote{Technically, the Large and Small Magellanic
  Clouds (LMC, SMC) are also classified as dwarf galaxies, but these
  were likely always known to humans at southern latitudes.}, with no
known analogues, already presented such an enigma. H. Shapley presented
these dwarfs as a new type of \emph{stellar system} resembling the
Magellanic Clouds and globular clusters but did not attempt to speculate
on their exact nature. While dwarf galaxies were soon understood to be
galaxies based on the inferred luminosities and sizes, their nature
remained unclear for decades
\citep[e.g.,][]{hodge1971, gallagher+wyse1994}.

The earliest spectroscopic work hinted that dwarf galaxies may contain
substantial amounts of dark matter. From early velocity dispersion
measurements for dwarf spheroidal (dSph) galaxies, inferred
mass-to-light ratios were at least 10 times larger than for globular
clusters \citep[GCs, e.g.,][]{aaronson1983, aaronson+olszewski1987}.
While uncertain initially, these values were later corroborated with
larger and more precise samples \citep[e.g.,][]{hargreaves+1994}. At the
time, several theories were proposed to explain these unusually high
mass-to-light ratios. Examples include: ongoing tidal disruption
inflating inferred velocity dispersions
\citep[e.g.,][]{kuhn+miller1989}, the presence of massive central black
holes \citep[e.g.,][]{strobel+lake1994}, or modified theories of gravity
\citep{milgrom1995}. Over time, however, a consensus developed where the
high mass-to-light ratios of dwarf galaxies was due to the presence of a
dark matter halo \citep[e.g.,][]{dekel+silk1986, wechsler+tinker2018}.
Since then, the properties of dwarf galaxies have played an increasingly
important role in our understanding of the clustering of dark matter on
small scales \citep[e.g.,][]{sales+2022, bullock+boylan-kolchin2017}.

Today, a common definition for a dwarf galaxy is a gravitationally bound
stellar system with dark matter.\footnote{Or, more generally, systems
  inconsistent with Newtonian dynamics of visible matter alone,
  \citet{willman+strader2012}} In contrast, star clusters (like GCs)
have no clear evidence for dark matter. The boundary between these two
classes blurs for faint, compact stellar associations, known as
``ambiguous'' systems.

Dwarf galaxies span a large range of sizes, luminosities, and
morphologies. Broadly, there are three classes of dwarf galaxies based
on luminosity. Local \textbf{bright dwarfs} with magnitudes
\(-14 \gtrsim M_V \gtrsim  -18\) or stellar masses
\(3\times10^7\,\Mo \lesssim M_\star \lesssim 10^9\,\Mo\) \footnote{assuming
  stellar mass-to-light ratio of 1, may be \(\sim 2\) for older
  populations}, often exhibit irregular morphologies and recent star
formation. Fig.~\ref{fig:galaxy_images} shows the Large Magellanic Cloud
(LMC) as an example of an irregular, bright dwarf galaxy, where most
stars are in a rotationally-supported thin disk (seen nearly face-on)
with a prominent bar. \textbf{Classical dwarf spheroidals}\footnote{While
  formally the dwarf galaxy names we discuss contain ``dwarf
  spheroidal'' (dSph), e.g.~Sculptor dSph, we omit this suffix for
  brevity. Additionally, the 12 classical dwarf satellites of our Galaxy
  are (in order of decreasing luminosity) Sagittarius, Fornax, Leo I,
  Sculptor, Antlia II, Leo II, Carina, Draco, Ursa Minor, Canes Venatici
  I, Sextans I, and Crater II. Antlia II, Crater II, and Canes Venatici
  I are the only post digital sky surveys additions.} occupy
intermediate luminosities ( \(-7.7 \gtrsim M_V  \gtrsim -14\) or
\(10^5\,\Mo \lesssim M_\star \lesssim 3\times10^7\,\Mo\)). Typically,
these systems are old, non-star forming, gas-poor, and spheroidal. All
Milky Way satellites discovered before digital sky surveys are
classicals, and these systems remain among the best studied. The
\textbf{ultra-faint}s dwarfs (UFDs) occupy the very faintest magnitude
range (\(M_V \gtrsim -7.7\) or \(M_\star \lesssim 10^5\,\Mo\)). These
galaxies have minuscule stellar masses, typically compact sizes, and
very metal-poor stellar populations \citep[see review][]{simon2019}.
Altogether, known dwarf galaxies span more than 15 orders in magnitude,
or over 7 decades in stellar mass.

Most well-studied dwarf galaxies lie near the Milky Way or in its
vicinity, the Local Group (LG) of galaxies. The LG is defined as the
group consisting of galaxies within \(\sim 1\) Mpc from the Milky
Way-Andromeda centre \citep[e.g.,][ and references
therein]{mcconnachie2012}. Today, we know that the Milky Way system is
teeming with dwarfs, many of which are satellites of either the Milky
Way or(MW) or Andromeda (M31). Fig.~\ref{fig:mw_satellite_system} shows
the MW satellite system, including dwarf galaxies, globular clusters,
and ambiguous systems. Such a nearby population of dwarf galaxies are
useful for resolved, detailed studies aiming to understand the nature of
these systems.

\begin{figure}
\centering
\includegraphics[width=5.41667in,height=5.41667in]{figures/galaxy_pictures.png}
\caption[Dwarf Galaxy Pictures]{Images of the LMC (DSS2), Fornax
\citep[DES DR2,][]{abbott+2021}, Sculptor (DES DR2), and Ursa Minor
\citep[UNWISE,][with \textit{Gaia} point sources
over-plotted]{lang2014, meisner+lang+schlegel2017, meisner+lang+schlegel2017a}.
The grey ellipse represents the half-light radius for the three dwarf
spheroidals, and the luminosity is derived from the absolute V-band
magnitude of each galaxy.}\label{fig:galaxy_images}
\end{figure}

\begin{figure}
\centering
\includegraphics[width=5.41667in,height=\textheight,keepaspectratio]{figures/mw_satellites_1.jpg}
\caption[Dwarf galaxies sky position]{The location of MW dwarf galaxies
on the sky. We label the classical dwarf galaxies (green diamonds),
fainter dwarfs (blue squares), globular clusters (orange circles), and
ambiguous systems (pink open hexagons). Globular clusters are more
centrally concentrated, but dwarf galaxies are preferentially found away
from the MW disk. Sculptor and Ursa Minor are highlighted as two dwarfs
we study later. The background image is from ESA/Gaia/DPAC
(https://www.esa.int/ESA\_Multimedia/Images/2018/04/Gaia\_s\_sky\_in\_colour2).
Dwarf galaxies (confirmed), globular clusters, and ambiguous systems are
from the \citet{pace2024} catalogue (version
1.0.3).}\label{fig:mw_satellite_system}
\end{figure}

\section{Dwarf galaxies in a cosmological
context}\label{dwarf-galaxies-in-a-cosmological-context}

We only understand a fraction of the universe's composition. The leading
theory of cosmology, Lambda Cold Dark Matter (\LCDM{}), posits that the
universe is composed of about 68\% dark energy (\(\Lambda\)), 27\% dark
matter (DM), and 5\% regular old baryons\footnote{\emph{Baryons} here
  means baryons+leptons, i.e.~any standard model massive fermion.}
\citep{planckcollaboration+2020}. While the composition of dark matter
and dark energy remains elusive, we know their general properties. Dark
energy drives the acceleration of the expansion of the universe on large
scales. We do not discuss dark energy here---it does not substantially
affect the Local Group today. Dark matter, instead, makes up the vast
majority of mass in galaxies. Typically, galaxies have baryonic to dark
matter ratios of between 1:5 to beyond 1:1000 for faint dwarf galaxies
\citep[e.g.,][]{hayashi+2023}.

In \LCDM{}, dark matter is assumed to interact only gravitationally.
Light passes through dark matter unimpeded---in this sense, dark matter
is transparent. Dark matter is also commonly assumed to be \emph{cold},
i.e.~typical velocities much smaller than the speed of light in the
early universe. If dark matter is cold, then it should condense on all
scales, forming non-linear structures (or \emph{halos}) from the size of
galaxy clusters to smaller than the faintest dwarf galaxies.
Implications of dark matter properties include cosmological structural
formation, galaxy structure, and galaxy interactions.

\subsection{\texorpdfstring{Structure formation in
\LCDM{}}{Structure formation in }}\label{structure-formation-in}

The very early universe was almost featureless. Our earliest
observations of the universe stem from the cosmic microwave background
(CMB)---revealing a uniform, isotropic, near-perfect blackbody emission.
But tiny perturbations in the CMB, temperature fluctuations of 1 part in
10,000, reveal the underlying seeds of large-scale cosmological
structure. In an expanding universe, gravitational instability makes CDM
overdensities grow and collapse hierarchically onto larger structures.
Initially, baryonic matter was coupled to radiation and resisted
collapse. Dark matter, only influenced by gravity instead, freely
collapsed into the first structures. Mass perturbations sufficiently
small and overdense become self-gravitating structures, known as
\emph{halos}. After recombination, where electrons combined with atomic
nuclei to form atoms, baryons decoupled from radiation and fell into the
dark matter halos. The densest pockets of baryons later formed the first
stars and galaxies.

Dark matter halos, and their associated galaxies, rarely evolve in
isolation. Instead, \LCDM{} structure formation is \emph{hierarchical}.
Small dark matter halos collapse first and hierarchically merge into
progressively larger halos
\citep[e.g.,][]{blumenthal+1984, white+rees1978, white+frenk1991}.
Hierarchical assembly is evident through the large scale structure of
the universe, remnants of past mergers within the Milky Way, and tidal
disruption of dwarf galaxies and their streams around nearby galaxies.

Small-scale structure formation is sensitive to deviations from \LCDM{}
cosmology \citep[e.g.,][]{bechtol+2022}. One key prediction of \LCDM{}
is that mass perturbations are expected to exist on all scales, and are
largest on the smallest scales, so we would expect the formation of
halos on all scales. Many alternative models, such as warm dark matter,
may smooth out small-scale features and reduce the abundance of small
halos or change their structure \citep[e.g.,][]{lovell+2014}. Dwarf
galaxies, which occupy the smallest dark matter halos, are promising
windows into the behaviour of dark matter on small scales.

\subsection{The structure of cold dark matter
halos}\label{the-structure-of-cold-dark-matter-halos}

In \LCDM{} cosmological simulations, dark matter halos are remarkably
self-similar. In \citet{NFW1996, NFW1997}, hereafter NFW, the authors
observe that the spherically-averaged density profiles \(\rho(r)\) are
universally well described by a two-parameter law,
\begin{equation}\protect\phantomsection\label{eq:nfw}{
\rho/\rho_s= \frac{1}{(r/r_s)(1+r/r_s)^2},
}\end{equation} where \(r_s\) is a scale radius and \(\rho_s\) a scale
density. This profile has shown remarkable success in describing \LCDM{}
halos across several orders of magnitude in mass. NFW profiles are
\emph{cuspy}, where the density rises like \(\rho \sim 1/r\) at small
radii \(r \ll r_s\). The steepness of the density profile increases
gradually with radius, and at large radii the density falls off like
\(\rho \sim 1/r^3\). The blue solid curve in Fig.~\ref{fig:nfw_density}
shows an example NFW halo.

The total mass of an NFW profile formally diverges, so halo masses are
conventionally defined using an overdensity criterion. The virial mass,
\(M_{200}\), is defined as the mass within a radius, \(r_{200}\),
containing a mean enclosed density 200 times\footnote{For the collapse
  of a uniform spherical density, the virialized overdensity would be
  \(\Delta = 18\pi^2\approx 178\) for a critical universe
  \(\Omega_m = 1\). This is commonly rounded to \(\Delta = 200\). While
  this parameter may be closer to \(\Delta \approx 100\) for our
  universe, \(\Delta\) also increases with redshift \citep[using eq. 6
  from][]{bryan+norman1998}.} the critical density of the universe:
\begin{equation}{
M_{200} =200\,\frac{4\pi}{3} \ r_{200}^3\ \rho_{\rm crit}, \qquad {\rm where} \quad \rho_{\rm crit}(z) = 3H(z)^2 / 8\pi G,
}\end{equation} and \(H(z)\) is the Hubble constant as a function of
redshift. Another way of characterizing NFW halos is through the halo
concentration, \(c=r_{200} / r_s\), which describes how the
characteristic radius scale of the halo compares to the virial radius.
Using this parameter, the scale density is a function of \(c\) alone,
\(\rho_s = (200/3)\,\rho_{\rm crit} c^3 / [\log(1+c) - c/(1+c)]\)
\citep{NFW1996}.

An equivalent, alternative characterization of NFW halos uses their
circular velocity profiles. The circular velocity,
\(\vcirc(r) = \sqrt{G M(r) / r}\), reaches a maximum \(\vmax\) at radius
\(\rmax \approx 2.16258\,r_s\). \(\vmax\) and \(r_{\rm max}\), like
\(M_{200}\) and \(c\), fully specify an NFW halo.

The two parameters of an NFW profile are not independent. Lower-mass
dark matter halos often collapse earlier, when the universe was denser.
As a result, low mass subhalos tend to be more concentrated
\citep[e.g.,][]{NFW1997}. The relationship between \(M_{200}\) and c, or
the mass-concentration relation, describes the mean trend of
concentration with mass or, equivalently, the dependence of \(\vmax\) on
\(\rmax\) \citep[e.g.,][]{bullock+2001, ludlow+2016}. The left panel of
Fig.~\ref{fig:smhm} illustrates the present-day mass-concentration from
\citet{ludlow+2016}. While concentration tends to decrease with
increasing mass, the relation has substantial scatter. Other parameters
such as the halo spin or shape may affect the scatter of the
mass-concentration relation, but their effect is typically expected to
be small \citep{navarro+2010, dicintio+2013, dutton+maccio2014}. The
mass-concentration relation informs our assumed structure of Scl and UMi
later.

\begin{figure}
\centering
\includegraphics[width=0.8\linewidth,height=\textheight,keepaspectratio]{figures/example_density_profiles.png}
\caption[Example density profiles]{Density profiles in log density
versus log radius for stars and dark matter in a Fornax-like galaxy. The
dark matter is more extended and massive than the star across the entire
galaxy. The halo has \(\vmax=40\,\kpc\) and \(\rmax=8\,\kpc\), or
\(M_{200} = 1\times10^{10}\,\Mo\) and \(c=12.5\), the cosmological mean
for a Fornax stellar mass galaxy. The galaxy has a stellar mass
\(M_\star \approx 2.5\times10^7\,\Mo\) with half-light radius
\(0.65\,\kpc\)
\citep{munoz+2018, woo+courteau+dekel2008}.}\label{fig:nfw_density}
\end{figure}

\begin{figure}
\centering
\pandocbounded{\includegraphics[keepaspectratio]{figures/cosmological_means.pdf}}
\caption[Stellar-mass halo-mass relation]{\textbf{Left} The NFW halo
concentration \(c=r_{200} / r_s\) as a function of virial mass
\(M_{200}\). The solid line with 1-\(\sigma\) shaded region is the
mass-concentration relation from \citet{ludlow+2016} for \(z=0\).
\textbf{Right} Stellar mass (top) as a function of maximum circular
velocity. The solid line with the 1-\(\sigma\) shaded region is the
relation from \citet{fattahi+2018} with scatter points simulated central
galaxies from \apostle{} in \citet{fattahi+2018}. The pink star
illustrates the location of the Fornax galaxy, whose density profiles
are shown in Fig.~\ref{fig:nfw_density}.}\label{fig:smhm}
\end{figure}

\subsection{\texorpdfstring{Galaxies formation in
\LCDM{}}{Galaxies formation in }}\label{galaxies-formation-in}

The observed abundance of galaxies may be compared with the abundance of
\LCDM{} halos to derive constraints regarding which galaxies inhabit
which halos. This technique, dubbed ``abundance matching,'' implies a
tight relation between the stellar mass of a galaxy and the mass of the
halo it inhabits \citep{li+white2009, moster+naab+white2013}.

Fig.~\ref{fig:smhm} shows the stellar mass versus halo mass relation
(SMHM, with halo mass represented by \(\vmax\)) predicted by the \LCDM{}
cosmological hydrodynamical simulations of Local Group analogues from
the \apostle{} project \citep{sawala+2016}.\footnote{\apostle{}
  simulated Local Group analogues in a \LCDM{} cosmological context with
  the hydrodynamical setup from the \eagle{} simulations
  \citep{crain+2015, schaye+2015}}. While there is some scatter, the
range of predicted \(\vmax\) is fairly narrow across \(\sim 4\) decades
in stellar mass. This figure indicates that the SMHM relation becomes
increasingly steep in the dwarf galaxy regime---many dwarf galaxies are
formed in relatively massive halos. Because lower mass galaxies have
increasingly shallow potential wells, feedback becomes more effective at
removing gas. Re-ionization additionally suppresses late star formation
in the faintest galaxies. As a result, the resulting stellar mass of a
dwarf galaxy is highly sensitive to the dark matter halo mass.

In \LCDM{} galaxy formation, the majority of mass in a dwarf galaxy
comes from the extended dark matter halo. Fig.~\ref{fig:nfw_density}
shows an example exponential stellar component for the Fornax dwarf
galaxy with the expected underlying dark matter halo. Even when the
stars are densest in the inner regions, the dark matter remains nearly
an order of magnitude higher in density. Stars make a small contribution
to the gravitational structure of dwarf galaxies---indeed, stars are
reasonably approximated as tracer particles of the underlying dark
matter halo. In addition, the stellar component is confined to the
central regions of the dark matter halo.

Several factors affect the SMHM trend, including environment, assembly
history, tidal effects, and the details of galaxy formation. For
example, effects like ram-pressure stripping (removal of gas in the
dwarf galaxy due to pressure from the host's circumgalactic medium) and
tidal removal of gas cause star formation to quench
\citep[e.g.,][]{christensen+2024}. Additionally, the time of formation
(relative to reionization) can influence the resulting stellar content
\citep{kim+2024}. Finally, tides influence both the dark matter and
stellar mass but in different amounts, adding additional scatter to the
SMHM trend for satellites \citep[e.g.,][]{PNM2008, fattahi+2018}.
Understanding the effects of tides on Local Group dwarf galaxies aids in
revealing where and how these galaxies formed in a cosmological context.

\subsection{Challenges and questions concerning dwarf
galaxies}\label{challenges-and-questions-concerning-dwarf-galaxies}

Observations of dwarf galaxies have been the origin of several disputes
or \emph{small-scale} problems for \LCDM{} \citep[see reviews
by][]{bullock+boylan-kolchin2017, sales+2022}. For example, the mismatch
between the expected number of dwarf galaxies from simulations and the
observed abundance has been known as the \emph{missing satellites
problem}. Additionally, a number of observations suggested that some
dwarf galaxies, although not all, possess dark matter ``cores,''
\citep[e.g.,][]{moore1994, adams+2014, oh+2015, walker+penarrubia2011, read+walker+steger2019},\footnote{In
  detail, gas-phase rotation curves are better able to differentiate
  between cores and cusps, whereas stellar kinematics is less
  constraining.} contrary to the expectation from \LCDM{} of ``cuspy''
inter dark matter profiles \citep{NFW1996, NFW1997}. As a result,
alternative forms of dark matter have been advocated as solutions, such
as Warm or Self-Interacting Dark Matter.

However, some of these tensions have eased as a result of improved
understanding of baryonic physics. For example, recent hydrodynamic
simulations in particular have shown that strong feedback can produce
dark matter cores
\citetext{\citealp[e.g.,][\citet{tollet+2016}]{navarro+eke+frenk1996}; \citealp{fitts+2017}; \citealp{benitez-llambay+2019}; \citealp{orkney+2021}}.
Several open questions remain, concerning e.g.~the plane of satellites,
and the details of the sizes and rotation curves of dwarf galaxies, the
existence and nature of stellar halos and accretion in dwarf galaxies
\citep[e.g.,][]{sales+2022}. Altogether, the numerous past and ongoing
challenges for \LCDM{} in the dwarf galaxy regime illustrate the
opportunity for dwarf galaxies to test the understanding of galaxy
formation and dark matter physics.

\section{The structure of nearby dwarf
galaxies}\label{the-structure-of-nearby-dwarf-galaxies}

\subsection{\texorpdfstring{The \emph{Gaia} space
telescope}{The Gaia space telescope}}\label{the-gaia-space-telescope}

Since Local Group dwarfs are nearby, they are resolved into individual
stars, and therefore we can study these galaxies on a star-by-star
basis. As a result, it is possible to measure the 3D velocity and
position of a star if we can measure its position, distance,
line-of-sight (LOS) velocity, and proper motion. Unfortunately,
determining distances and full 3D velocities is challenging. The most
direct measurement of distance, parallax, requires precise tracking of a
star's sky position across a year. And while line-of-sight (LOS)
velocities are relatively easily determined from spectroscopy,
tangential velocities, derived from proper motions and distances, are
much more challenging. Typically, measuring proper motions requires
accurate (much less than arcsecond) determinations of small changes in a
star's position over baselines of years to decades. The full 6D position
and velocity information for stars has, until recently, been known for
only a handful of stars.

Launched in 2013, \emph{Gaia} is a space-based, all-sky survey telescope
situated at the Sun-Earth L2 Lagrange point
\citep{gaiacollaboration+2016}. \emph{Gaia} has redefined astrometry,
providing photometry, positions, proper motions, and parallaxes for over
1 billion stars \citep{gaiacollaboration+2021}. While \emph{Gaia}
completed its space-based mission in 2025, two further data releases are
expected.

Determining absolute parallax measurement is facilitated by the
observation that stars in different regions of the sky are affected by
parallax motion with different phases. By imaging two regions separated
by 106.5 degrees on the same focal plane, \emph{Gaia} measures changes
in relative positions of stars across small and large angles. Combining
measurements from multiple epochs across several years, an absolute
all-sky reference frame is derived from which parallax and proper
motions are derived. In addition to astrometry, \emph{Gaia} measures
photometry in the wide \emph{G} band (330--1050nm) and colours from the
blue photometer (BP, 330--680 nm) and red photometer (RP, 640--1050 nm).
\emph{Gaia} additionally provides low resolution BP-RP spectra and
radial velocity measurements of bright stars \citep[of magnitudes
\(G_{\rm RVS} < 16\),][]{gaiacollaboration+2016}. For our work,
\emph{Gaia}'s most relevant measurements are \(G\) magnitude,
\(G_{\rm BP} - G_{\rm RP}\) colour, \((\alpha, \delta)\) position, and
\((\mu_{\alpha*}, \mu_\delta)\) proper motions.\footnote{The proper
  motions \(\mu_\alpha\) and \(\mu_\delta\) are the apparent rates of
  change in right ascension, \(\alpha\), and declination, \(\delta\),
  typically in units of mili-arcsecond (mas) per year.
  \(\mu_{\alpha*} = \mu_\alpha \cos \delta\) corrects for projection
  effects in \(\alpha\).}

\subsection{\texorpdfstring{\emph{Gaia}'s impact on Milky Way
studies}{Gaia's impact on Milky Way studies}}\label{gaias-impact-on-milky-way-studies}

\emph{Gaia} has revolutionized our understanding of Milky Way structure.
For example, the 6D dynamical measurements and metallicities of MW stars
led to the (re)discovery of past mergers or Milky Way building blocks
like \emph{Gaia}-Sausage Enceladus
\citetext{\citealp[e.g.,][]{helmi+2018}; \citealp{belokurov+2018}; \citealp[but
see also][]{meza+2005}}, out-of-equilibrium structures like the
\emph{Gaia} snail \citep[e.g.,][]{antoja+2018}, and dynamical effects of
the Milky Way's spiral arms and the bar in the solar neighbourhood
\citep[ and references therein]{hunt+vasiliev2025}. In the Milky Way
halo, \emph{Gaia} has helped find and constrain numerous stellar streams
\citep{ibata+malhan+martin2019, bonaca+price-whelan2025}. Altogether,
\emph{Gaia} has revealed the hierarchical formation and complex,
evolving structure of our own Galaxy.

For Milky Way satellites, \emph{Gaia} has improved orbital analysis and
facilitated robust stellar membership determination. Before \emph{Gaia},
few galaxies had precisely measured proper motions \citep[e.g.~using
Hubble Space Telescope,][]{sohn+2017}. \emph{Gaia} allowed the first
systematic and precise determinations of Milky Way satellite proper
motions \citep{pace+li2019, MV2020a}. While the proper motion
uncertainty of a typical dwarf member star is often large, by combining
the proper motions of 100s or 1000s of stars from \emph{Gaia}, precise
average proper motion measurements can be determined, sometimes only
limited by \emph{Gaia}'s systematic error floor
\citep[e.g.,][]{MV2020a}. Proper motions have thus ushered in a new era
for MW satellite dynamical studies, where we can derive precise orbits
for any satellite, assuming a given MW potential. In addition,
\emph{Gaia} helps establish membership by separating out contaminating
MW foreground stars. By measuring parallaxes and/or proper motions, many
more background and foreground stars can be classified as non-members
\citep[e.g.,][]{battaglia+2022, jensen+2024}.

\subsection{Dwarf galaxy light profiles}\label{sec:exponential_profiles}

Projected luminosity / stellar density profiles efficiently characterize
the radial structure of a galaxy. At its most basic, light profiles
synthesize properties such as the shape, size, and orientation of a
dwarf galaxy. In addition, the details of a stellar density profile can
help interpret a galaxy's assembly and dynamical history
\citep[e.g.,][]{penarrubia+2009, lee+2018, querci+2025}. Note that for
resolved galaxies, these profiles are expressed in stellar count
densities instead of surface brightness.

Four different surface density laws are frequently used to parameterize
dwarf galaxy profiles: Exponential, Plummer, King, or Sérsic profiles
\citep[e.g.,][]{munoz+2018}. The exponential profile is perhaps the
simplest, defined in terms of the central surface density, \(\Sigma_0\),
and scale radius, \(R_s\):
\begin{equation}\protect\phantomsection\label{eq:exponential_law}{
\Sigma_{\rm exp} = \Sigma_0\exp(-R / R_s).
}\end{equation} This profile is often applied to the radial light
distribution of galaxy disks
\citep{devaucouleurs1959a, freeman1970, kent1985}.

To fit globular cluster density profiles, \citet{plummer1911} proposed a
profile based on a self-gravitating polytrope,\footnote{where density
  and pressure are assumed to be related by a power law}
\begin{equation}{
\Sigma_{\rm Pl} = \frac{\Sigma_0}{(1 + (R/R_h)^2)^2},
}\end{equation} where \(\Sigma_0\) is the central surface density and
\(R_h\) is the projected half-light radius. Now mostly superseded by the
King profile for globular clusters, the Plummer model is still a good
fit to many dwarf spheroidals \citep[e.g.,][]{moskowitz+walker2020}.

The \citet{king1962} profile, also a fit to globular clusters, is also
used to describe dwarf galaxies, more so in older literature. Using
three parameters, a core radius \(R_c\), a truncation radius \(R_t\),
and a characteristic density, \(\Sigma_0\), the King profile may be
written as \begin{equation}{
\Sigma_{\rm K} = \Sigma_0\left(\frac{1}{\sqrt{1 + (R/R_c)^2}} - \frac{1}{\sqrt{1+(R_t/R_c)^2}}\right).
}\end{equation} In much of the older literature, \(R_t\) was interpreted
as a ``tidal radius'', after an analogous interpretation for globular
clusters \citep[e.g.,][]{hodge1961, IH1995}.

Finally, the \citet{sersic1963} profile represents a generalization of
an exponential profile, and describes most dwarf galaxy light profiles
well. Typically parameterized in terms of a half-light radius \(R_h\),
the density at half-light radius \(\Sigma_h\) and a Sérsic index \(n\),
the profile's equation is \begin{equation}{
\Sigma_{\rm S} = \Sigma_h \exp\left[-b_n \,  \left((R/R_h)^{1/n} - 1\right)\right]
}\end{equation} where \(b_n\) is a constant that depends on \(n\). A
Sérsic profile with \(n=1\) is equivalent to an exponential profile,
while \(n=4\) recovers \citet{devaucouleurs1948}'s profile for
elliptical galaxies. Although a Sérsic profile is less commonly applied
to dwarf galaxies, \citet{munoz+2018} advocate for the Sérsic profile
since the added flexibility allows more profiles to be fit with a single
law.

While there are no clear theoretical preferences for any of these
profiles, exponential density profiles have been commonly used for dwarf
spheroidal galaxies. \citet{faber+lin1983} were among the first to
demonstrate that an exponential law is a reasonable empirical fit,
theorizing that dwarf spheroidals may have evolved from exponential disk
galaxies and maintained a similar light profile. Later,
\citet{read+gilmore2005} showed that exponential profiles may originate
from mass loss during the evolution of dwarf galaxies. Many subsequent
photometric studies of dwarf spheroidal galaxies have used exponential
fits, finding that exponential and King profiles both provide good
descriptions in many cases
\citep{binggeli+sandage+tarenghi1984, mateo1998, mcconnachie+irwin2006, cicuendez+2018}.
More recently, \citet{moskowitz+walker2020} fit instead generalized
Plummer profiles, but most of their fits would be consistent with a
single-component exponential. As a result, it has become conventional to
assume an exponential density profile to describe dwarf galaxies in
theoretical or observational modelling
\citep[e.g.,][]{kowalczyk+2013, martin+2016, MV2020a, battaglia+2022}.

Outside the MW system, exponential profiles are common, but sometimes
with modifications. For example, many extragalactic dwarf elliptical,
blue compact, and irregular dwarf galaxies are better described with an
exponential profile to which is added a central cusp or nuclear region
\citep{caldwell+bothun1987, noeske+2003}. On the other hand, some
studies find an inner density decrement compared to exponentials
\citep[e.g.,][]{caldwell+1992, makarov+2012}, or that dwarfs are better
fit by two nested exponentials
\citep[e.g.,][]{aparicio+1997, graham+guzman2003, hunter+elmegreen2006, lee+2018}.
It is unclear how these conclusions apply to dwarf spheroidals.

Altogether, while there is some variation in the density profiles of
dwarf galaxies, an exponential is an excellent first-order
approximation. Typically, deviations from exponentials are in the
direction of a steeper outer cutoff or changes to the inner slope of a
dwarf galaxy (due, for example, to a nuclear star cluster). Flattened
density profiles in the outer regions are more unusual. Explaining in
detail the origin, similarity, and diversity of dwarf galaxy density
profiles is an open question for theories of dwarf galaxy formation and
evolution.

\subsection{The extended light profiles of Sculptor and Ursa Minor:
Hints of tidal
signatures?}\label{the-extended-light-profiles-of-sculptor-and-ursa-minor-hints-of-tidal-signatures}

Sculptor and Ursa Minor appear to be typical dwarf spheroidal galaxies
at first glance. Tables~\ref{tbl:scl_obs_props}, \ref{tbl:umi_obs_props}
describe the present-day properties of each galaxy. Sculptor, as the
first discovered classical dSph, is often described as a
``prototypical'' dSph. There has long been speculation that both
Sculptor and Ursa Minor may have been influenced by the Milky Way's
tidal field (see Section~\ref{sec:discussion}).
\citet{sestito+2023a, sestito+2023b}, for example, have recently
reported a ``kink'' in the density profile, beginning around 30 arcmin
for from the centre of each of these systems, that they interpret as
potentially caused by the effects of Galactic tides. They
spectroscopically follow up some of the most distant stars, finding
members as far as 6 and 12 half-light radii from the centre of each
dwarf. If these dwarfs had exponential profiles, like Fornax, then these
far-outlying stars should be much rarer.

Sculptor and Ursa Minor are poorly described by an exponential profile.
The left panel of Fig.~\ref{fig:scl_umi_vs_fornax} shows the density
profiles of Sculptor, Ursa Minor, and Fornax (see
Section~\ref{sec:observations} for details on how these profiles are
measured). Compared to Fornax, both Sculptor and Ursa Minor show an
excess of stars outside \(\log R/R_h\approx 0.4\), which exceeds 100
times the density of the exponential fit at large radii.

A goal of this work is to determine if tidal effects are indeed
responsible for the extended outer light profiles of Sculptor and Ursa
Minor. If tides cannot explain these features, these features may
instead be due to an extended stellar ``halo'' or second component of
the galaxy---suggestive of a complex star formation or assembly history.

\begin{figure}
\centering
\pandocbounded{\includegraphics[keepaspectratio]{./figures/scl_umi_vs_fornax.pdf}}
\caption[Sculptor and Ursa Minor match tidal models]{A plot of the
surface density profiles of Sculptor (orange squares), Ursa Minor (red
triangles), and Fornax (green circles) scaled to their half-light radius
and the density at half-light radius (data described in
Section~\ref{sec:observations}). The solid black line is an exponential
profile (Eq.~\ref{eq:exponential_law}).}\label{fig:scl_umi_vs_fornax}
\end{figure}

\begin{table*}[t]
\centering
\caption[Observed Properties of Sculptor]{Observed properties of Sculptor. References are: 1. Muñoz et al. (2018) Sérsic fits, 2. Tran et al. (2022) RR lyrae distance, 3. Alan W. McConnachie and Venn (2020b), 4. Arroyo-Polonio et al. (2024). }
\label{tbl:scl_obs_props}
\begin{tabular}{lll}
\toprule
parameter & value & Source\\
\midrule
$\alpha$ & $15.0183 \pm 0.0012^\circ$ & 1\\
$\delta$ & $-33.7186 \pm 0.0007^\circ$ & ”\\
distance modulus & $19.60 \pm 0.05$ & 2\\
distance & $83.2 \pm 2$ kpc & ”\\
$\mu_{\alpha*}$ & $0.099 \pm 0.002 \pm 0.017$ mas yr$^{-1}$ & 3\\
$\mu_\delta$ & $-0.160 \pm 0.002_{\rm stat} \pm 0.017_{\rm sys}$ mas yr$^{-1}$ & ”\\
LOS velocity & $111.2 \pm 0.3\ {\rm km\,s^{-1}}$ & 4\\
$\sigma_v$ & $9.7\pm0.2\ {\rm km\,s^{-1}}$ & ”\\
$R_h$ & $9.79 \pm 0.04$ arcmin & 1\\
ellipticity & $0.37 \pm 0.01$ & ”\\
position angle & $94\pm1^\circ$ & ”\\
$M_V$ & $-10.82\pm0.14$ & ”\\
\bottomrule
\end{tabular}
\end{table*}

\begin{table*}[t]
\centering
\caption[Observed Properties of Ursa Minor]{Observed properties of Ursa Minor. References are: (1) Muñoz et al. (2018) Sérsic fits, (2) Garofalo et al. (2025) RR lyrae distance, (3) Alan W. McConnachie and Venn (2020a), (4) Pace et al. (2020), average of MMT and Keck results. }
\label{tbl:umi_obs_props}
\begin{tabular}{lll}
\toprule
parameter & value & Source\\
\midrule
$\alpha$ & $ 227.2420 \pm 0.0045$˚ & 1\\
$\delta$ & $67.2221 \pm 0.0016$˚ & ”\\
distance modulus & $19.23 \pm 0.11$ & 2\\
distance & $70.1 \pm 3.6$ kpc & ”\\
$\mu_\alpha*$ & $-0.124 \pm 0.004 \pm 0.017$ mas yr$^{-1}$ & 3\\
$\mu_\delta$ & $0.078 \pm 0.004_{\rm stat} \pm 0.017_{\rm sys}$ mas yr$^{-1}$ & ”\\
LOS velocity & $-245.9 \pm 0.3_{\rm stat} \pm 1_{\rm sys}$ km s$^{-1}$ & 4\\
$\sigma_v$ & $8.6 \pm 0.3$ & ”\\
$R_h$ & $11.62 \pm 0.1$ arcmin & 1\\
ellipticity & $0.55 \pm 0.01$ & ”\\
position angle & $50 \pm 1^\circ$ & ”\\
$M_V$ & $-9.03 \pm 0.05$ & ”\\
\bottomrule
\end{tabular}
\end{table*}

\section{Interpreting tidal signatures}\label{sec:tidal_theory}

Simulating dwarf galaxies accurately in a cosmological context remains a
challenge. Cosmological simulations can predict the overall abundance of
the most massive dwarf galaxies \citep[e.g.,][]{sawala+2016} and broadly
examine the effects of tides \citep[e.g.,][]{riley+2024}. However,
simulated dwarf galaxies are often near the resolution limit.
Insufficient resolution can lead to artificial disruption of dwarf
galaxies and over-prediction of tidal effects
\citep[e.g.,][]{vandenbosch+2018, santos-santos+2025}. To address this
challenge, idealized simulations are often used to simulate a single
subhalo in an approximate host potential, achieving excellent numerical
convergence. For instance, the simulations we describe later reach 3
times higher resolution than Aquarius \citep{springel+2008} at a
fraction of the computational cost (400 times fewer particles).
Idealized simulations make numerous simplifications, neglecting mergers,
cosmological context, mass assembly, and often baryonic physics
\citep[e.g.,][]{hayashi+2003, bullock+johnston2005, klimentowski+2009, ogiya+2019}.
We shall use idealized simulations here to assess tidal effects after
infall.

Conclusions from early work on tidal evolution remain mostly relevant
today. A galaxy undergoing tidal effects loses most mass near the
Lagrange points at pericentre, forms symmetric tidal streams with a
velocity gradient, and retains its central structure
\citep{oh+lin+aarseth1995, piatek+pryor1995, moore+davis1994, johnston+spergel+hernquist1995, read+2006}.

CDM halos also are found to be highly resilient to full disruption
\citep{EP2020}. In addition, NFW halos evolve along ``tidal tracks''
\citep[e.g,][]{PNM2008, green+vandenbosch2019, EN2021}. On the other
hand, cored dark matter halos likely disrupt faster
\citep[e.g.,][]{penarrubia+2010, errani+2023a}.

\citet{drakos+taylor+benson2020, drakos+taylor+benson2022, amorisco2021}
showed that tidal effects are nearly entirely described as the removal
of particles below a certain energy \citep[see
also][]{choi+weinberg+katz2009}. \citet{stucker+2023} generalized this
idea, creating a model for adiabatic tidal mass loss in an isotropic
tidal field, confirming that an NFW halo is likely unable to be
disrupted and explaining the evolution of NFW halos along similar tidal
tracks.

With precise orbits and a better understanding of the Milky Way
potential, recent work continues to directly probe the dynamical
histories of individual dwarf galaxies.
\citet{borukhovetskaya+2022, dicintio+2024} both ran simulations tuned
to Fornax, showing that this galaxy's stellar component or globular
clusters are likely not affected by tides. \citet{borukhovetskaya+2022a}
also used N-body simulations to analyze Crater II, showing that the
present-day properties of the galaxy are challenging to reconcile with
\LCDM{} initial conditions with tidal evolution. Most relevantly,
\citet{iorio+2019} conducted simulations of Sculptor, concluding tides
likely do not affect this galaxy. Our goal is to apply a similar
framework to Sculptor and Ursa Minor.

\subsection{Tidal and ``break'' radii}\label{sec:break_radii}

For a given orbit in a given potential, there are characteristic radii
which help gauge the effects of tides on a dwarf galaxy model.

The \textbf{Jacobi radius} represents the approximate radius where stars
become unbound for a galaxy in a circular orbit around a host galaxy.
Calculated from an approximation of the location of the \(L_1\) and
\(L_2\) Lagrange points, the Jacobi radius is where the mean density of
the dwarf galaxy is roughly three times the mean interior density of the
host galaxy at pericentre, or
\begin{equation}\protect\phantomsection\label{eq:r_jacobi}{
3\bar \rho_{\rm MW}(r_{\rm peri}) \approx \bar \rho_{\rm dwarf}(r_J),
}\end{equation} \citep[ eq. 7-84]{BT1987}. If \(r_J\) occurs within the
visible extent of a galaxy, we should expect to find relatively clear
signs of tidal disturbance. While strictly valid for circular orbits,
assuming \(r_{\rm peri}\) for the host-dwarf distance works as most
stars are lost near pericentre.

We also use the \textbf{break radius} as defined in
\citet{penarrubia+2009}, marking where the galaxy is still in
disequilibrium after pericentric passage in a highly-eccentric orbit.
The break radius \(r_{\rm break}\) is proportional to the velocity
dispersion, \(\sigma_v\), and time elapsed since pericentre,
\(\Delta t\), \begin{equation}\protect\phantomsection\label{eq:r_break}{
r_{\rm break} = C\,\sigma_{v}\,\Delta t
}\end{equation} where the scaling constant \(C \approx 0.55\) was
derived empirically. \(r_{\rm break}\) describes where the dynamical
timescale is longer than the time since the perturbation, i.e.~the
radius within which the galaxy has had enough time to dynamically relax.

\subsection{A simple tidal simulation}\label{a-simple-tidal-simulation}

To illustrate the effects of tides on an NFW halo due to a larger
galaxy, we consider a toy model. We evolve an NFW subhalo with
\(\rmax=5\,\kpc\) and \(\vmax = 27\kms\) orbiting another NFW static
host halo with \(\rmax=25\,\kpc\) and \(\vmax = 207.4\,\kms\). These
choices are motivated by the inferred structure and masses of dSphs and
the Milky Way, respectively. The stars follow initially an exponential
profile with a scale radius \(R_s=0.25\,\kpc\). We then evolve the orbit
through a pericentre of 10 kpc starting from an apocentre of 100 kpc. We
show the results 170 Myr after the first pericentre. See
Section~\ref{sec:methods} for a more detailed description of our
numerical setup.

Fig.~\ref{fig:idealized_break_radius} illustrates the properties of this
idealized simulation shortly after the first pericentre. The projected
density of stars (right) is relatively undisturbed and spherical in the
centre, but becomes non-isotropic outside the break radius and shows
nascent tidal tails.These tidally disturbed stars appear as an extended,
outer density excess relative to the initial conditions. This excess
appears just outside the break radius. The break radius also marks where
the mean radial velocity of the stars transitions from 0 to positive ---
the galaxy is out of equilibrium outside \(r_{\rm break}\).

To first order, the final density profile of this toy simulation indeed
resembles the outer excess in Sculptor and Ursa Minor. This thesis aims
to investigate if a tidal explanation is viable given the current
properties of each galaxy.

\begin{figure}
\centering
\pandocbounded{\includegraphics[keepaspectratio]{figures/idealized_break_radius.pdf}}
\caption[Break radius validation]{Example density and velocity
distributions of an idealized simulation shortly after pericentre.
\textbf{Top left}: The 2D density profile for the initial and final
simulation with the break radius marked. The break radius of the
simulations is set by the time since pericentre. \textbf{Bottom left}:
the mean radial velocity (dot product of relative position and velocity
relative to dwarf centre) as a function of 2D radius. \textbf{Right}:
The projected 2D stellar density in the \(x\)-\(y\) plane. The green
circle represents the break radius and the grey arrow points towards the
host centre.}\label{fig:idealized_break_radius}
\end{figure}

\section{Thesis outline}\label{thesis-outline}

The goal of this thesis is to review the evidence for an extended
density profile in Ursa Minor and Sculptor, to assess the impact of
tidal effects on each galaxy, and to discuss possible interpretations
for the structure of these galaxies.

In Chapter \ref{sec:observations}, we describe how we compute
observational density profiles following \citet{jensen+2024}. In Chapter
\ref{sec:methods}, we review our simulation methods. Next, we present
our results for the tidal effects on Sculptor and Ursa Minor in Chapter
\ref{sec:results}, We discuss our results, limitations, and implications
in Chapter \ref{sec:discussion}. Finally, Chapter \ref{sec:summary}
summarizes this thesis and discusses future directions for similar work.
