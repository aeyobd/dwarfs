Dwarf galaxies host, in many ways, the most extreme galactic
environments in the universe. These galaxies are typically defined to be
fainter than the Large Magellanic Cloud (LMC), with \(M_V \gtrsim -18\)
or similarly \(M_\star \lesssim 10^9 M_\odot\)
\citep[e.g.,][]{mcconnachie2012, bullock+boylan-kolchin2017}. Because
the galaxy luminosity function increases towards fainter objects, dwarfs
are the most numerous galaxies in the Universe
\citep[e.g.,][]{blanton+2005, mao+2021}. Dwarf galaxies are also highly
\emph{dark-matter dominated}, with mass to light ratios which may exceed
1000 \(M_\odot/ L_\odot\) \citep[e.g.,][]{simon+geha2007, hayashi+2023}.

With the exception of the Magellanic Clouds, most dwarf galaxy
satellites of the Milky Way (MW) are \emph{quenched}, with little to no
recent star formation \citep[e.g.,][]{weisz+2014}. Indeed, most faint MW
satellites contain old stellar populations which are \emph{relics} from
the early universe, consisting of many of the oldest and most metal poor
stars known \citep{simon2019}. Understanding the properties of dwarf
galaxies thus has implications across astronomy, from cosmological
structure formation on small scales to the formation of metal-poor
stellar populations.

In this Chapter, we first describe the general observed properties of
local dwarf galaxies. Next, we summarize our understanding of the
cosmological origin of dwarf galaxies. We later review recent
advancements and pending questions concerning dwarf galaxies, and
introduce the puzzle posed by the extended stellar density profiles of
Sculptor and Ursa Minor. We end with a brief roadmap to the remainder of
this dissertation.

\section{Observations of dwarf
galaxies}\label{observations-of-dwarf-galaxies}

Dwarf galaxies have long raised conundrums for theories of galaxy
formation. The discovery of Fornax and Sculptor in 1938
\citep{shapley1938}\footnote{Technically, the Large and Small Magellanic
  Clouds (LMC, SMC) are also classified as dwarf galaxies, but these
  were likely always known to humans at southern latitudes.}, with no
known analogues at the time, already presented such an enigma. H.
Shapley presented these dwarfs as a new type of \emph{stellar system}
resembling the Magellanic Clouds and globular clusters but did not
attempt to speculate on their exact nature. While dwarf galaxies were
soon understood to be galaxies based on the inferred luminosities and
sizes, their nature remained unclear for decades
\citep[e.g.,][]{hodge1971, gallagher+wyse1994}.

The earliest spectroscopic work hinted that dwarf galaxies may contain
substantial amounts of dark matter. From early velocity dispersion
measurements for dwarf spheroidal (dSph) galaxies, inferred
mass-to-light ratios were at least 10 times larger than for globular
clusters \citep[e.g.,][]{aaronson1983, aaronson+olszewski1987}. While
uncertain initially, these values were later corroborated with larger
and more precise samples \citep[e.g.,][]{hargreaves+1994}. At the time,
several theories were proposed to explain these unusually high
mass-to-light ratios. Examples include: ongoing tidal disruption
inflating inferred velocity dispersions
\citep[e.g.,][]{kuhn+miller1989}, the presence of massive central black
holes \citep[e.g.,][]{strobel+lake1994}, or modified theories of gravity
\citep{milgrom1995}. Over time, however, a consensus developed where the
high mass-to-light ratios of dwarf galaxies was due to the presence of a
dark matter halo \citep[e.g.,][]{dekel+silk1986, wechsler+tinker2018}.
Since then, the properties of dwarf galaxies have played an increasingly
important role in our understanding of the clustering of dark matter on
small scales \citep[e.g.,][]{bullock+boylan-kolchin2017, sales+2022}.

Today, a common definition for a dwarf galaxy is a gravitationally bound
stellar system with dark matter.\footnote{Or, more generally, systems
  inconsistent with Newtonian dynamics of visible matter alone,
  \citet{willman+strader2012}.} In contrast, star clusters (like
globular clusters) have no clear evidence for dark matter. The boundary
between these two classes blurs for faint, compact stellar associations.
Systems with shared characteristics of globular clusters and dwarf
galaxies, and without evidence as to their nature, are known as
``ambiguous'' systems \citep[e.g.,][]{smith+2024}.

Dwarf galaxies span a large range of sizes, luminosities, and
morphologies. Broadly, there are three classes of dwarf galaxies based
on luminosity. Local \textbf{bright dwarfs} with magnitudes
\(-14 \gtrsim M_V \gtrsim  -18\) or stellar masses
\(3\times10^7\,\Mo \lesssim M_\star \lesssim 10^9\,\Mo\) \footnote{assuming
  stellar mass-to-light ratio of \(1 \Mo / L_{\odot}\), may be
  \(\sim 2 \Mo/L_\odot\) for older populations.}, often exhibit
irregular morphologies and recent star formation.
Fig.~\ref{fig:galaxy_images} shows the Large Magellanic Cloud (LMC) as
an example of an irregular, bright dwarf galaxy, where most stars are in
a rotationally-supported disk (seen nearly face-on) with a prominent
bar. \textbf{Classical dwarf spheroidals}\footnote{While formally the
  dwarf galaxy names we discuss contain ``dwarf spheroidal'' (dSph),
  e.g.~Sculptor dSph, we omit this suffix for brevity. Additionally, the
  12 classical dwarf satellites of our Galaxy are (in order of
  decreasing luminosity) Sagittarius, Fornax, Leo I, Sculptor, Antlia
  II, Leo II, Carina, Draco, Ursa Minor, Canes Venatici I, Sextans I,
  and Crater II. Antlia II, Crater II, and Canes Venatici I are the only
  post digital sky surveys additions.} occupy intermediate luminosities
( \(-7.7 \gtrsim M_V  \gtrsim -14\) or
\(10^5\,\Mo \lesssim M_\star \lesssim 3\times10^7\,\Mo\)). Typically,
these systems are old, non-star forming, gas-poor, and spheroidal. All
Milky Way satellites discovered before digital sky surveys are
classicals, and these systems remain among the best studied. The
\textbf{ultra-faint dwarfs} occupy the very faintest magnitude range
(\(M_V \gtrsim -7.7\) or \(M_\star \lesssim 10^5\,\Mo\)). These galaxies
have minuscule stellar masses, typically compact sizes, and very
metal-poor stellar populations \citep[see the review by][]{simon2019}.
Altogether, known dwarf galaxies span more than 15 magnitudes, or over 7
decades in stellar mass.

Most well-studied dwarf galaxies lie near the Milky Way or in its
vicinity, the Local Group of galaxies. The Local Group is defined as the
group consisting of galaxies within \(\sim 1\) Mpc from the Milky
Way-Andromeda centre \citep[e.g.,][ and references
therein]{mcconnachie2012}. Today, we know that the Milky Way system is
teeming with dwarfs, many of which are satellites of either the Milky
Way (MW) or Andromeda (M31). Fig.~\ref{fig:mw_satellite_system} shows
the MW satellite system, including dwarf galaxies, globular clusters,
and ambiguous systems. This nearby population of dwarf galaxies are
opportune for resolved, detailed studies aimed to understand the nature
of these systems.

\begin{figure}
\centering
\includegraphics[width=5.41667in,height=5.41667in]{figures/galaxy_pictures.png}
\caption[Images of dwarf galaxies]{Images of the LMC (DSS2), Fornax
\citep[DES DR2,][]{abbott+2021}, Sculptor (DES DR2), and Ursa Minor
\citep[UNWISE,][with \textit{Gaia} point sources
over-plotted]{lang2014, meisner+lang+schlegel2017, meisner+lang+schlegel2017a}.
The grey ellipse represents the half-light radius for the three dwarf
spheroidals, and the luminosity is derived from the absolute V-band
magnitude of each galaxy.}\label{fig:galaxy_images}
\end{figure}

\begin{figure}
\centering
\includegraphics[width=1\linewidth,height=\textheight,keepaspectratio]{figures/mw_satellites_1.jpg}
\caption[The on-sky distribution of Milky Way satellites]{The location
of MW dwarf galaxies on the sky. We label the classical dwarf galaxies
(green diamonds), fainter dwarfs (blue squares), globular clusters
(orange circles), and ambiguous systems (pink open hexagons). Globular
clusters are more centrally concentrated, but dwarf galaxies are
preferentially found away from the MW disk. Sculptor and Ursa Minor are
highlighted as two dwarfs we study later. The background image is from
ESA/Gaia/DPAC
(https://www.esa.int/ESA\_Multimedia/Images/2018/04/Gaia\_s\_sky\_in\_colour2).
Dwarf galaxies (confirmed), globular clusters, and ambiguous systems are
from the \citet{pace2024} catalogue (version
1.0.3).}\label{fig:mw_satellite_system}
\end{figure}

\section{Dwarf galaxies in a cosmological
context}\label{dwarf-galaxies-in-a-cosmological-context}

We only understand a fraction of the universe's composition. The leading
theory of cosmology, Lambda Cold Dark Matter (\LCDM{}), posits that the
universe is composed of about 68\% dark energy (\(\Lambda\)), 27\% dark
matter (DM), and 5\% regular old baryons\footnote{In a classic,
  astronomer's corruption of jargon, \emph{Baryons} here means
  baryons+leptons, i.e.~any standard model massive fermion.}
\citep{planckcollaboration+2020}. While the composition of dark matter
and dark energy remains elusive, we know their general properties. Dark
energy drives the acceleration of the expansion of the universe on large
scales. We do not discuss dark energy here---it does not substantially
affect the Local Group today. Dark matter, instead, makes up the vast
majority of mass in galaxies. Typically, galaxies have baryonic to dark
matter ratios of between 1:5 to beyond 1:1000 for faint dwarf galaxies
\citep[e.g.,][]{hayashi+2023}.

In \LCDM{}, dark matter is assumed to interact only gravitationally.
Light passes through dark matter unimpeded---in this sense, dark matter
is transparent. Dark matter is also commonly assumed to be \emph{cold},
i.e.~typical velocities much smaller than the speed of light in the
early universe. If dark matter is cold, then it should condense on all
scales, forming non-linear structures (or \emph{halos}) from the size of
galaxy clusters to smaller than the faintest dwarf galaxies.
Implications of dark matter properties include cosmological structural
formation, galaxy structure, and galaxy interactions.

\subsection{\texorpdfstring{Structure formation in
\LCDM{}}{Structure formation in }}\label{structure-formation-in}

The very early universe was almost featureless. Our earliest
observations of the universe stem from the cosmic microwave background
(CMB)---revealing a nearly uniform, isotropic blackbody emission. But
tiny perturbations in the CMB, temperature fluctuations of 1 part in
100,000, reveal the underlying seeds of large-scale cosmological
structure. In an expanding universe, gravitational instability makes CDM
overdensities grow and collapse hierarchically onto larger structures.
Initially, baryonic matter was coupled to radiation and resisted
collapse. Dark matter, only influenced by gravity instead, freely
collapsed into the first structures. Mass perturbations sufficiently
small and overdense become self-gravitating structures, known as
\emph{halos}. After recombination, where electrons combined with atomic
nuclei to form atoms, baryons decoupled from radiation and fell into
dark matter halos, where the condensed at the centre through radiative
energy losses. The densest pockets of baryons later formed the first
stars and galaxies.

Dark matter halos, and their associated galaxies, rarely evolve in
isolation. Instead, \LCDM{} structure formation is \emph{hierarchical}.
Small dark matter halos collapse first and hierarchically merge into
progressively larger halos
\citep[e.g.,][]{white+rees1978, blumenthal+1984, white+frenk1991}.
Hierarchical assembly is evident through the large scale structure of
the universe, remnants of past mergers within the Milky Way, and tidal
disruption of dwarf galaxies and their streams around nearby galaxies.

Small-scale structure formation is sensitive to deviations from \LCDM{}
cosmology \citep[e.g.,][]{bechtol+2022}. One key prediction of \LCDM{}
is that mass perturbations are expected to exist on all scales, and are
largest on the smallest scales, so we would expect the formation of
halos on all scales. Many alternative models, such as warm dark matter,
may smooth out small-scale features and reduce the abundance of small
halos or change their structure \citep[e.g.,][]{lovell+2014}. Dwarf
galaxies, which occupy the smallest dark matter halos, are promising
probes into the behaviour of dark matter on small scales.

\subsection{The structure of cold dark matter halos}\label{sec:NFW}

In \LCDM{} cosmological simulations, dark matter halos are remarkably
self-similar. In \citet{NFW1996, NFW1997}, hereafter NFW, the authors
observe that the spherically-averaged density profiles \(\rho(r)\) are
universally well described by a two-parameter law,
\begin{equation}\protect\phantomsection\label{eq:nfw}{
\rho/\rho_s= \frac{1}{(r/r_s)(1+r/r_s)^2},
}\end{equation} where \(r_s\) is a scale radius and \(\rho_s\) a scale
density. This profile has shown remarkable success at describing \LCDM{}
halos across several orders of magnitude in mass. NFW profiles are
\emph{cuspy}, where the density rises like \(\rho \sim 1/r\) at small
radii \(r \ll r_s\). The steepness of the density profile increases
gradually with radius, and at large radii the density falls off like
\(\rho \sim 1/r^3\). The blue solid curve in Fig.~\ref{fig:nfw_density}
shows an example NFW halo.

The total mass of an NFW profile formally diverges, so halo masses are
conventionally defined using an overdensity criterion. The virial mass,
\(M_{200}\), is defined as the mass within a radius, \(r_{200}\),
containing a mean enclosed density 200 times\footnote{For the collapse
  of a uniform spherical density, the virialized overdensity would be
  \(\Delta = 18\pi^2\approx 178\) for a critical universe
  \(\Omega_m = 1\). This is commonly rounded to \(\Delta = 200\). While
  this parameter may be closer to \(\Delta \approx 100\) for our
  universe, \(\Delta\) also increases with redshift \citep[using eq. 6
  from][]{bryan+norman1998}.} the critical density of the universe:
\begin{equation}{
M_{200} =200\,\frac{4\pi}{3} \ r_{200}^3\ \rho_{\rm crit}, \qquad {\rm where} \quad \rho_{\rm crit}(z) = 3H(z)^2 / 8\pi G,
}\end{equation} and \(H(z)\) is the Hubble constant, which depends on
redshift. Another way of characterizing NFW halos is through the
concentration parameter, \(c=r_{200} / r_s\), which describes how the
characteristic radial scale of the halo compares to the virial radius.
Using this parameter, the scale density is a function of \(c\) alone,
\(\rho_s = (200/3)\,\rho_{\rm crit} c^3 / [\log(1+c) - c/(1+c)]\)
\citep{NFW1996}.

An equivalent, alternative characterization of NFW halos uses their
circular velocity profiles. The circular velocity,
\(\vcirc(r) = \sqrt{G M(r) / r}\), reaches a maximum \(\vmax\) at radius
\(\rmax \approx 2.16258\,r_s\). \(\vmax\) and \(r_{\rm max}\), like
\(M_{200}\) and \(c\), fully specify an NFW halo.

The two parameters of an NFW profile are not independent. Lower-mass
dark matter halos typically collapse earlier, when the universe was
denser. As a result, low mass subhalos tend to be more concentrated
\citep[e.g.,][]{NFW1997}. The relationship between \(M_{200}\) and c, or
the mass-concentration relation, describes the mean trend of
concentration with mass or, equivalently, the dependence of \(\vmax\) on
\(\rmax\) \citep[e.g.,][]{bullock+2001, ludlow+2016}. The left panel of
Fig.~\ref{fig:smhm} illustrates the present-day mass-concentration from
\citet{ludlow+2016}. While concentration tends to decrease with
increasing mass, the relation has substantial scatter. Other parameters
such as the halo spin or shape may affect the scatter of the
mass-concentration relation, but their effect is typically expected to
be small \citep{navarro+2010, dicintio+2013, dutton+maccio2014}.

\begin{figure}
\centering
\includegraphics[width=3.5in,height=\textheight,keepaspectratio]{figures/example_density_profiles.png}
\caption[Example dark matter and stellar density profiles]{Density
profiles in log 3D density versus log 3D radius for stars and dark
matter in a Fornax-like galaxy. The dark matter is more extended and
massive than the star across the entire galaxy. Fornax has a stellar
mass \(M_\star \approx 2.5\times10^7\,\Mo\) with half-light radius
\(0.65\,\kpc\) \citep{munoz+2018, woo+courteau+dekel2008}. The
corresponding cosmological-mean halo for Fornax has \(\vmax=40\,\kpc\)
and \(\rmax=8\,\kpc\), or \(M_{200} = 1\times10^{10}\,\Mo\) and
\(c=12.5\).}\label{fig:nfw_density}
\end{figure}

\begin{figure}
\centering
\pandocbounded{\includegraphics[keepaspectratio]{figures/cosmological_means.pdf}}
\caption[Cosmological halo mass and stellar mass
relations]{\textbf{Left} The NFW halo concentration \(c=r_{200} / r_s\)
as a function of virial mass \(M_{200}\). The solid line with
1-\(\sigma\) shaded region is the mass-concentration relation from
\citet{ludlow+2016} for \(z=0\). \textbf{Right} Stellar mass (top) as a
function of maximum circular velocity. The solid line with the
1-\(\sigma\) shaded region is the relation from \citet{fattahi+2018}
with scatter points simulated central galaxies from \apostle{} in
\citet{fattahi+2018}. The pink star illustrates the location of the
Fornax galaxy, whose density profiles are shown in
Fig.~\ref{fig:nfw_density}. \textbf{add rmax-vmax plot
too}}\label{fig:smhm}
\end{figure}

\subsection{\texorpdfstring{Galaxy formation in
\LCDM{}}{Galaxy formation in }}\label{sec:galaxy_formation}

The observed abundance of galaxies may be compared with the abundance of
\LCDM{} halos to derive constraints regarding which galaxies inhabit
which halos. One simple technique, dubbed ``abundance matching,''
implies a tight relation between the stellar mass of a galaxy and the
mass of the halo it inhabits
\citep{li+white2009, moster+naab+white2013}.

Fig.~\ref{fig:smhm} shows the stellar mass versus halo mass relation
(SMHM, with halo mass represented by \(\vmax\)) predicted by \LCDM{}
cosmological hydrodynamical simulations of Local Group analogues from
the \apostle{} project \citep{sawala+2016}.\footnote{\apostle{}
  simulated Local Group analogues in a \LCDM{} cosmological context with
  the hydrodynamical setup from the \eagle{} simulations
  \citep{crain+2015, schaye+2015}}. While there is some scatter, the
range of predicted \(\vmax\) is fairly narrow across \(\sim 4\) decades
in stellar mass. This figure indicates that the SMHM relation becomes
increasingly steep in the dwarf galaxy regime---many dwarf galaxies are
formed in halos of similar mass. Because lower mass galaxies have
shallower potential wells, feedback becomes more effective at removing
gas. Re-ionization additionally suppresses late star formation in the
faintest galaxies. As a result, the resulting stellar mass of a dwarf
galaxy is highly sensitive to dark matter halo mass.

In \LCDM{} galaxy formation, the majority of mass in a dwarf galaxy
comes from the extended dark matter halo. Fig.~\ref{fig:nfw_density}
shows an example exponential stellar component for the Fornax dwarf
galaxy with its surrounding dark matter NFW halo \citep[with parameters
matching the][ and \citet{fattahi+2018} relations]{ludlow+2016}. Where
the stars are densest, the dark matter remains nearly an order of
magnitude higher in density. Stars make a small contribution to the
gravitational structure of dwarf galaxies---indeed, stars are reasonably
approximated as tracer particles of the underlying dark matter halo. In
addition, the stellar component is typically confined to the central
regions of the dark matter halo.

Several factors affect the SMHM trend, including environment, assembly
history, tidal effects, and the details of galaxy formation. For
example, effects like ram-pressure stripping (removal of gas in the
dwarf galaxy due to pressure from the host's circumgalactic medium) and
tidal removal of gas cause star formation to quench
\citep[e.g.,][]{christensen+2024}. Additionally, the time of formation
(relative to reionization) can influence the resulting stellar content
\citep{kim+2024}. Finally, Galactic tides influence both the dark matter
and stellar mass but in different amounts, adding additional scatter to
the SMHM trend for satellites \citep[e.g.,][]{PNM2008, fattahi+2018}.
Understanding the effects of tides on Local Group dwarf galaxies may
help us understand where and how these galaxies formed in a cosmological
context.

\subsection{Challenges and questions concerning dwarf
galaxies}\label{challenges-and-questions-concerning-dwarf-galaxies}

Observations of dwarf galaxies have been the origin of several disputes
or \emph{small-scale} problems for \LCDM{} \citep[see reviews
by][]{bullock+boylan-kolchin2017, sales+2022}. For example, the mismatch
between the expected number of dwarf galaxies from simulations and the
observed abundance has been known as the \emph{missing satellites
problem}. Additionally, a number of observations suggest that some dwarf
galaxies, although not all, possess dark matter ``cores,''
\citep[e.g.,][]{moore1994, adams+2014, oh+2015, walker+penarrubia2011, read+walker+steger2019},\footnote{In
  detail, gas-phase rotation curves are better able to differentiate
  between cores and cusps, whereas stellar kinematics is less
  constraining.} contrary to the expectation from \LCDM{} of ``cuspy''
inner dark matter profiles \citep{NFW1996, NFW1997}. As a result,
alternative forms of dark matter have been advocated as solutions, such
as Warm or Self-Interacting Dark Matter.

However, some of these tensions have eased as a result of improved
understanding of baryonic physics. For example, recent hydrodynamic
simulations in particular have shown that strong feedback can produce
dark matter cores
\citetext{\citealp[e.g.,][\citet{tollet+2016}]{navarro+eke+frenk1996}; \citealp{fitts+2017}; \citealp{benitez-llambay+2019}; \citealp{orkney+2021}}.
Several open questions remain, concerning, e.g., the plane of
satellites, the details of the sizes and rotation curves of dwarf
galaxies, and the existence and nature of stellar halos and accretion in
dwarf galaxies \citep[e.g.,][]{sales+2022}. Altogether, the numerous
past and ongoing challenges for \LCDM{} in the dwarf galaxy regime
illustrate the opportunity for dwarf galaxies to test the understanding
of galaxy formation and dark matter physics.

\section{The structure of nearby dwarf
galaxies}\label{the-structure-of-nearby-dwarf-galaxies}

\subsection{\texorpdfstring{The \emph{Gaia} space
telescope}{The Gaia space telescope}}\label{the-gaia-space-telescope}

Since Local Group dwarfs are nearby, they are resolved into individual
stars, and therefore we can study these galaxies on a star-by-star
basis. As a result, it is possible to measure the 3D velocity and
position of a star if we can measure its position, distance,
line-of-sight (LOS) velocity, and proper motion. Unfortunately,
determining distances and full 3D velocities is challenging. The most
direct measurement of distance, the parallax, requires precise tracking
of a star's sky position across a year. And while line-of-sight (LOS)
velocities are relatively easily determined from spectroscopy,
tangential velocities, derived from proper motions and distances, are
much more challenging. Typically, measuring proper motions requires
precise (\(\ll\) arcsecond) determinations of small changes in a star's
position over baselines of years to decades. The full 6D position and
velocity information for stars has, until recently, been known for only
a handful of stars.

Launched in 2013, \emph{Gaia} is a space-based, all-sky survey telescope
situated at the Sun-Earth L2 Lagrange point
\citep{gaiacollaboration+2016}. \emph{Gaia} has redefined astrometry,
providing photometry, positions, proper motions, and parallaxes for over
1 billion stars \citep{gaiacollaboration+2021}. While \emph{Gaia}
completed its space-based mission in 2025, two further data releases are
still expected.

Determining absolute parallax measurement is facilitated by the
observation that stars in different regions of the sky are affected by
parallax motion with different phases. By imaging two regions separated
by 106.5 degrees on the same focal plane, \emph{Gaia} measures changes
in relative positions of stars across small and large angles. Combining
measurements from multiple epochs across several years, an absolute
all-sky reference frame is derived from which parallax and proper
motions are derived. In addition to astrometry, \emph{Gaia} measures
photometry in the wide \emph{G} band (330--1050nm) and colours from the
blue photometer (BP, 330--680 nm) and red photometer (RP, 640--1050 nm).
\emph{Gaia} additionally provides low resolution BP-RP spectra and
radial velocity measurements of bright stars \citep[of magnitudes
\(G_{\rm RVS} < 16\),][]{gaiacollaboration+2016}. For our work,
\emph{Gaia}'s most relevant measurements are \(G\) magnitude,
\(G_{\rm BP} - G_{\rm RP}\) colour, \((\alpha, \delta)\) position, and
\((\mu_{\alpha*}, \mu_\delta)\) proper motions.\footnote{The proper
  motions \(\mu_\alpha\) and \(\mu_\delta\) are the apparent rates of
  change in right ascension, \(\alpha\), and declination, \(\delta\),
  typically in units of mili-arcsecond (mas) per year.
  \(\mu_{\alpha*} = \mu_\alpha \cos \delta\) corrects for projection
  effects in \(\alpha\).}

\subsection{\texorpdfstring{\emph{Gaia}'s impact on Milky Way
studies}{Gaia's impact on Milky Way studies}}\label{gaias-impact-on-milky-way-studies}

\emph{Gaia} has revolutionized our understanding of Milky Way structure.
For example, the 6D dynamical measurements and metallicities of MW stars
led to the (re)discovery of past mergers or Milky Way building blocks
like \emph{Gaia}-Sausage Enceladus
\citetext{\citealp[e.g.,][]{helmi+2018}; \citealp{belokurov+2018}; \citealp[but
see also][]{meza+2005}}, out-of-equilibrium structures like the
\emph{Gaia} snail \citep[e.g.,][]{antoja+2018}, and dynamical effects of
the Milky Way's spiral arms and the bar in the solar neighbourhood
\citep[ and references therein]{hunt+vasiliev2025}. In the Milky Way
halo, \emph{Gaia} has helped find and constrain numerous stellar streams
\citep{ibata+malhan+martin2019, bonaca+price-whelan2025}. Altogether,
\emph{Gaia} has revealed the hierarchical formation and complex,
evolving structure of our own Galaxy.

For Milky Way satellites, \emph{Gaia} has improved orbital analysis and
facilitated robust stellar membership determinations. Before
\emph{Gaia}, few galaxies had precisely measured proper motions
\citep[e.g.~using Hubble Space Telescope,][]{piatek+2005, sohn+2017}.
\emph{Gaia} allowed for some of the first systematic and precise
determinations of Milky Way satellite proper motions
\citep{pace+li2019, MV2020a}. While the proper motion uncertainty of a
typical dwarf member star is often large, by combining the proper
motions of 100s or 1000s of stars from \emph{Gaia}, precise average
proper motion measurements can be determined, sometimes only limited by
\emph{Gaia}'s systematic error floor \citep[e.g.,][]{MV2020a}. Proper
motions have thus ushered in a new era for MW satellite dynamical
studies, where we can derive precise orbits for any satellite, assuming
a given MW potential. In addition, \emph{Gaia} helps establish
membership by separating out contaminating MW foreground stars. By
measuring parallaxes and/or proper motions, many more background and
foreground stars can be classified as non-members
\citep[e.g.,][]{battaglia+2022, jensen+2024}.

\subsection{Dwarf galaxy light profiles}\label{sec:exponential_profiles}

Projected luminosity / stellar density profiles efficiently characterize
the radial structure of a galaxy. At its most basic, light profiles
synthesize properties such as the shape, size, and orientation of a
dwarf galaxy. In addition, the details of a stellar density profile can
help interpret a galaxy's assembly and dynamical history
\citep[e.g.,][]{penarrubia+2009, lee+2018, querci+2025}. Note that for
resolved galaxies, these profiles are expressed in stellar count
densities instead of surface brightness.

Four different surface density laws are frequently used to parameterize
dwarf galaxy profiles: Exponential, Plummer, King, or Sérsic profiles
\citep[e.g.,][]{munoz+2018}. The exponential profile is perhaps the
simplest, defined in terms of the central surface density, \(\Sigma_0\),
and projected scale radius, \(R_s\):
\begin{equation}\protect\phantomsection\label{eq:exponential_law}{
\Sigma_{\rm exp} = \Sigma_0\exp(-R / R_s).
}\end{equation} This profile is also often applied to the radial light
distribution of galaxy disks
\citep{devaucouleurs1959a, freeman1970, kent1985}.

To fit globular cluster density profiles, \citet{plummer1911} proposed a
profile based on a self-gravitating polytrope,\footnote{where density
  and pressure are assumed to be related by a power law}
\begin{equation}{
\Sigma_{\rm Pl} = \frac{\Sigma_0}{(1 + (R/R_h)^2)^2},
}\end{equation} where \(\Sigma_0\) is the central surface density and
\(R_h\) is the projected half-light radius. Now mostly superseded by the
King profile for globular clusters, the Plummer model is still a good
fit to many dwarf spheroidals \citep[e.g.,][]{moskowitz+walker2020}.

The \citet{king1962} profile, also a fit to globular clusters, is also
used to describe dwarf galaxies, more so in older literature. Using
three parameters, a core radius \(R_c\), a truncation radius \(R_t\),
and a characteristic density, \(\Sigma_0\), the King profile may be
written as \begin{equation}{
\Sigma_{\rm K} = \Sigma_0\left(\frac{1}{\sqrt{1 + (R/R_c)^2}} - \frac{1}{\sqrt{1+(R_t/R_c)^2}}\right); \qquad \text{for } R<R_t
}\end{equation} and \(\Sigma_{\rm K}=0\) for \(R \geq R_t\). In much of
the older literature, \(R_t\) was interpreted as a ``tidal radius'',
after an analogous interpretation for globular clusters
\citep[e.g.,][]{hodge1961, IH1995}.

Finally, the \citet{sersic1963} profile represents a generalization of
an exponential profile, and describes most dwarf galaxy light profiles
well. Typically parameterized in terms of a half-light radius \(R_h\),
the density at half-light radius \(\Sigma_h\) and a Sérsic index \(n\),
the profile's equation is \begin{equation}{
\Sigma_{\rm S} = \Sigma_h \exp\left[-b_n \,  \left((R/R_h)^{1/n} - 1\right)\right]
}\end{equation} where \(b_n\) the coefficient which solves
\(\Gamma(2n) = 2\gamma(2n, b_n)\) with \(\Gamma\) the Gamma function and
\(\gamma\) the lower incomplete gamma function
\citep{graham+driver2005}. A Sérsic profile with \(n=1\) is equivalent
to an exponential profile, while \(n=4\) recovers
\citet{devaucouleurs1948}'s profile for elliptical galaxies. Although a
Sérsic profile is less commonly applied to dwarf galaxies,
\citet{munoz+2018} advocate for the Sérsic profile since the added
flexibility allows more profiles to be fit with a single law.

While there are no clear theoretical preferences for any of these
profiles, exponential density profiles have been commonly used for dwarf
spheroidal galaxies. \citet{faber+lin1983} were among the first to
demonstrate that an exponential law is a reasonable empirical fit,
theorizing that dwarf spheroidals may have evolved from exponential disk
galaxies\footnote{While studied for far longer, the exponential origin
  of disk galaxies is no better understood. A sample of ideas range from
  scattering of stars \citep{elmegreen+struck2013, wu+2022} to angular
  momentum transport, to disk viscosity-driven radial gas flows
  \citep{lin+pringle1987, wang+lilly2022} or spherical collapse
  \citep{freeman1970}.} and maintained a similar light profile. Later,
\citet{read+gilmore2005} showed that exponential profiles may originate
from mass loss during the evolution of dwarf galaxies. Tides are a
possible mechanism for this transformation---the ``tidal stirring
hypothesis'' \citep{mayer+2001a, klimentowski+2009}. However, the
theoretical origin of exponential disks is unclear.

Many subsequent photometric studies of dwarf spheroidal galaxies have
used exponential fits, finding that exponential and King profiles both
provide good descriptions in many cases
\citep{binggeli+sandage+tarenghi1984, mateo1998, mcconnachie+irwin2006, cicuendez+2018}.
More recently, \citet{moskowitz+walker2020} fit instead generalized
Plummer profiles, but most of their fits would be consistent with a
single-component exponential. As a result, it has become conventional to
assume an exponential density profile to describe dwarf galaxies in
theoretical or observational modelling
\citep[e.g.,][]{kowalczyk+2013, martin+2016, MV2020a, battaglia+2022}.

Outside the MW system, exponential profiles are common, but sometimes
with modifications. For example, many extragalactic dwarf elliptical,
blue compact, and irregular dwarf galaxies are better described with an
exponential profile to which a central cusp or nuclear region is added
\citep{caldwell+bothun1987, noeske+2003}. On the other hand, some
studies find an inner density decrement relative to exponentials
\citep[e.g.,][]{caldwell+1992, makarov+2012}, or that dwarfs are better
fit by two nested exponentials
\citep[e.g.,][]{aparicio+1997, graham+guzman2003, hunter+elmegreen2006, lee+2018}.
It is unclear how these conclusions apply to the dwarf spheroidals of
the Local Group.

Altogether, while there is some variation in the density profiles of
dwarf galaxies, an exponential is an excellent first-order
approximation. Typically, deviations from exponentials are in the
direction of a steeper outer cutoff or changes to the inner slope of a
dwarf galaxy (due, for example, to a nuclear star cluster). Flattened
density profiles in the outer regions are more unusual. Explaining in
detail the origin, similarity, and diversity of dwarf galaxy density
profiles is an open question for theories of dwarf galaxy formation and
evolution.

\subsection{The extended light profiles of Sculptor and Ursa Minor:
Hints of tidal signatures?}\label{sec:scl_umi_obs_tides}

Sculptor (Scl) and Ursa Minor (UMi) appear to be typical dwarf
spheroidal galaxies at first glance (see Fig.~\ref{fig:galaxy_images}).
Tables~\ref{tbl:scl_obs_props}, \ref{tbl:umi_obs_props} describe the
structural parameters of each galaxy. Sculptor, as the first discovered
classical dSph, is even described as a ``prototypical'' dSph
\citep[e.g.,][]{mcconnachie2012}.

However, many works speculated that Scl and UMi were influenced by the
Milky Way's tidal field. Already, \citet{innanen+papp1979} found RR
Lyrae candidate Scl members \citep[from][]{vanagt1978} out to 180' in an
elongated distribution, speculating this to be tidal disruption. Later
density profile determinations noted Scl's elongation and apparent outer
density excess
\citetext{\citealp{eskridge1988}; \citealp{IH1995}; \citealp{walcher+2003}; \citealp{westfall+2006}; \citealp[but
see also][]{coleman+dacosta+bland-hawthorn2005}}. These works often
interpreted these features as evidence of tidal effects
\citep[e.g.,][]{walcher+2003} or sometimes a dwarf galaxy halo
\citep{westfall+2006}.

UMi also attracted similar suspicions. \citet{hargreaves+1994} first
detected a velocity gradient in UMi, suggestive of tidal disruption.
Later, \citet{martinez-delgado+2001} find that stars extend far beyond
the King-profile ``tidal radius'', aligned with UMi's elongation,
consistent with tidally-stripping simulations in
\citet{gomez-flechoso+martinez-delgado2003}. \citet{palma+2003}
furthermore detected S-shaped isophotes and an extended population of
``extratidal'' stars.

Adding to the evidence, \citet{sestito+2023a, sestito+2023b} report a
``kink'' in the density profile of each galaxy. They spectroscopically
follow up some of the most distant stars, finding members as far as 6
and 12 half-light radii from the centre of each dwarf. If these dwarfs
had exponential profiles, like Fornax, then these far-outlying stars
should be much rarer.

Sculptor and Ursa Minor are poorly described by an exponential profile.
The left panel of Fig.~\ref{fig:scl_umi_vs_fornax} shows the density
profiles of Sculptor, Ursa Minor, and Fornax (see
Section~\ref{sec:observations} for details on how these profiles are
measured). Compared to Fornax, both Sculptor and Ursa Minor show an
excess of stars outside \(\log R/R_h\approx 0.4\), which exceeds 100
times the density of the exponential fit at large radii.

A goal of this work is to determine if tidal effects are indeed
responsible for the extended outer light profiles of Sculptor and Ursa
Minor. If tides cannot explain these features, these features may
instead be due to an extended stellar ``halo'' or second component of
the galaxy---suggestive of a complex star formation or assembly history.

\begin{figure}
\centering
\pandocbounded{\includegraphics[keepaspectratio]{./figures/scl_umi_fornax_exp.pdf}}
\caption[Scl and UMi have an extended density profile]{A plot of the
surface density profiles of Sculptor (orange squares), Ursa Minor (red
triangles), and Fornax (green circles) scaled to their half-light radius
and the density at half-light radius (data described in
Section~\ref{sec:observations}). The solid black line is an exponential
profile (Eq.~\ref{eq:exponential_law}). Scl and UMi show a clear
overdensity at large radii.}\label{fig:scl_umi_vs_fornax}
\end{figure}

\begin{table*}[t]
\centering
\caption[Observed Properties of Sculptor]{Observed properties of Sculptor. References are: 1. Muñoz et al. (2018, Sérsic fit), 2. Tran et al. (2022, RR lyrae distance), 3. Alan W. McConnachie and Venn (2020b), 4. Arroyo-Polonio et al. (2024). }
\label{tbl:scl_obs_props}
\begin{tabular}{lll}
\toprule
parameter & value & Source\\
\midrule
$\alpha$ & $15.0183 \pm 0.0012^\circ$ & 1\\
$\delta$ & $-33.7186 \pm 0.0007^\circ$ & ”\\
distance modulus & $19.60 \pm 0.05$ & 2\\
distance & $83.2 \pm 2$ kpc & ”\\
$\mu_{\alpha*}$ & $0.099 \pm 0.002 \pm 0.017$ mas yr$^{-1}$ & 3\\
$\mu_\delta$ & $-0.160 \pm 0.002_{\rm stat} \pm 0.017_{\rm sys}$ mas yr$^{-1}$ & ”\\
LOS velocity & $111.2 \pm 0.3\ {\rm km\,s^{-1}}$ & 4\\
$\sigma_v$ & $9.7\pm0.2\ {\rm km\,s^{-1}}$ & ”\\
$R_h$ & $9.79 \pm 0.04$ arcmin & 1\\
ellipticity & $0.37 \pm 0.01$ & ”\\
position angle & $94\pm1^\circ$ & ”\\
$M_V$ & $-10.82\pm0.14$ & ”\\
\bottomrule
\end{tabular}
\end{table*}

\begin{table*}[t]
\centering
\caption[Observed Properties of Ursa Minor]{Observed properties of Ursa Minor. References are: (1) Muñoz et al. (2018, Sérsic fit), (2) Garofalo et al. (2025) pRR lyrae distance], (3) Alan W. McConnachie and Venn (2020a), (4) Pace et al. (2020), average of MMT and Keck results with systematic uncertainty from Appendix \ref{sec:extra_rv_models} discussion. }
\label{tbl:umi_obs_props}
\begin{tabular}{lll}
\toprule
parameter & value & Source\\
\midrule
$\alpha$ & $ 227.2420 \pm 0.0045$˚ & 1\\
$\delta$ & $67.2221 \pm 0.0016$˚ & ”\\
distance modulus & $19.23 \pm 0.11$ & 2\\
distance & $70.1 \pm 3.6$ kpc & ”\\
$\mu_\alpha*$ & $-0.124 \pm 0.004 \pm 0.017$ mas yr$^{-1}$ & 3\\
$\mu_\delta$ & $0.078 \pm 0.004_{\rm stat} \pm 0.017_{\rm sys}$ mas yr$^{-1}$ & ”\\
LOS velocity & $-245.9 \pm 0.3_{\rm stat} \pm 1_{\rm sys}$ km s$^{-1}$ & 4\\
$\sigma_v$ & $8.6 \pm 0.3$ & ”\\
$R_h$ & $11.62 \pm 0.1$ arcmin & 1\\
ellipticity & $0.55 \pm 0.01$ & ”\\
position angle & $50 \pm 1^\circ$ & ”\\
$M_V$ & $-9.03 \pm 0.05$ & ”\\
\bottomrule
\end{tabular}
\end{table*}

\section{Interpreting tidal signatures}\label{sec:tidal_theory}

The Local Group hosts several examples of ongoing tidal disruption. The
Magellanic stream, a massive, gas-rich feature emanating from the
Magellanic clouds, is believed to arise partially from the MW's tides
\citep{putman+1998, diaz+bekki2012, donghia+fox2016}. Other clear
examples of tidal streams include the Sagittarius stream, the Andromeda
Giant Southern stream, and the Tucana III stream
\citep[e.g.,][]{ibata+gilmore+irwin1994, ibata+2001, li+2018}. These
examples illustrate that hierarchical accretion remains an active
process. Interpreting such observations relies on simulations of tidal
disruption.

Cosmological simulations struggle to resolve tidal effects on dwarfs.
Since many dwarfs are near the resolution limit, they are vulnerable to
artifical disruption
\citep[e.g.,][]{vandenbosch+2018, santos-santos+2025}. To overcome
numerical challenges, idealized simulations model a single subhalo in an
analytic host potential, achieving excellent numerical convergence. For
instance, the simulations we describe later in this thesis reach the
times higher resolution than Aquarius \citep{springel+2008} with 400
times fewer particles. Idealized simulations make numerous
simplifications, neglecting mergers, cosmological context, mass
assembly, and often baryonic physics
\citep[e.g.,][]{hayashi+2003, bullock+johnston2005, klimentowski+2009, ogiya+2019}.
Cosmological simulations appear to predict that tidal disruption is more
common than idealized work or the Milky Way system suggests, but the
origin and role of numerics and assumptions in this discrepancy is
unclear \citep{panithanpaisal+2021, shipp+2023, riley+2024}. We shall
use idealized simulations here to assess tidal effects after the
satellite's infall into the MW halo.

Idealized simulations predict clear properties of tidally disrupting
dwarf spheroidal galaxies. Tidally stripped stars form tidal
streams---approximately uniform-density stellar tails with a bulk
velocity gradient
\citep[e.g.,][]{moore+davis1994, johnston+spergel+hernquist1995, read+2006}.
Most mass loss happens near pericentre, where tides are strongest.
However, the central structure of a dwarf galaxy often remains
undisturbed \citep{oh+lin+aarseth1995, piatek+pryor1995}. For instance,
NFW halos also are found to be nearly immune to tidal disruption
\citep{EP2020}, but cored dark matter halos may disrupt fully and faster
\citep[e.g.,][]{penarrubia+2010, errani+2023a}.

To first order, tidal mass loss peels away the outer layers of a dwarf
galaxy in energy space.
\citet{drakos+taylor+benson2020, drakos+taylor+benson2022, amorisco2021}
showed that tidal effects are nearly entirely described as the removal
of particles above a truncation energy \citep[see
also][]{choi+weinberg+katz2009}. \citet{stucker+2023} generalized this
idea, creating a model for adiabatic tidal mass loss in an isotropic
tidal field. They explain the resilience of NFW halos against full tidal
disruption and origin of well-defined ``tidal tracks'' \citep[as
observed in][]{PNM2008, green+vandenbosch2019, EN2021}.

With precise orbital constraints and improved models of the Milky Way
potential, recent studies has continued to probe the dynamical histories
of individual dwarf galaxies.
\citet{borukhovetskaya+2022, dicintio+2024} both ran simulations tuned
to Fornax, showing that this galaxy's stellar component or globular
clusters are likely not affected by tides. Similarly,
\citet{borukhovetskaya+2022a} analyzed Crater II, showing that the
present-day structure is challenging to reconcile with NFW initial
conditions and Galactic tides. Most relevantly, \citet{iorio+2019}
modelled Sculptor, concluding tides likely do not affect the stellar
component.

Building on this body of work, we will use idealized simulations to
understand the severity of tidal effects on Sculptor and Ursa Minor.

\subsection{Tidal and ``break'' radii}\label{sec:break_radii}

For a given orbit in a given potential, there are characteristic radii
which help gauge the effects of tides on a dwarf galaxy.

The \textbf{Jacobi radius} represents the approximate radius where stars
become unbound for a galaxy in a circular orbit around a host
galaxy.\footnote{The Jacobi radius was derived at least as early as
  \citet{laplace1798}. This radius also bears other names, such as the
  Hill radius \citep[from][]{hill1878}. Likely only named after Jacobi
  for the Jacobi integral \citep{jacobi1836}.} Calculated from an
approximation of the location of the \(L_1\) and \(L_2\) Lagrange
points, the Jacobi radius is where the mean density of the dwarf galaxy
is roughly three times the mean interior density of the host galaxy at
pericentre, or
\begin{equation}\protect\phantomsection\label{eq:r_jacobi}{
3\bar \rho_{\rm MW}(r_{\rm peri}) \approx \bar \rho_{\rm dwarf}(r_J),
}\end{equation} \citep[ eq. 7-84]{BT1987}. If \(r_J\) is comparable to
the visible extent of a galaxy, we should expect to find clear signs of
tidal disturbance. While strictly valid for circular orbits, assuming
\(r_{\rm peri}\) for the host-dwarf distance works as most stars are
lost during pericentric passages.

We also use the \textbf{break radius} as defined in
\citet{penarrubia+2009}, marking the outermost radius within which the
dwarf has been able to achieve equilibrium after pericentric passage in
a highly-eccentric orbit. The break radius \(r_{\rm break}\) is
proportional to the velocity dispersion, \(\sigma_v\), and time elapsed
since pericentre, \(\Delta t\),
\begin{equation}\protect\phantomsection\label{eq:r_break}{
r_{\rm break} = C\,\sigma_{v}\,\Delta t
}\end{equation} where the empirical constant \(C \approx 0.55\).
\(r_{\rm break}\) describes where the dynamical timescale is longer than
the time since the perturbation, i.e.~the radius within which the galaxy
is dynamically relaxed.

\subsection{A simple tidal simulation}\label{a-simple-tidal-simulation}

To illustrate the effects of tides on an NFW halo due to a massive host,
we consider a toy model. We evolve an N-body realization of an NFW
subhalo with \(\rmax=5\,\kpc\) and \(\vmax = 27\kms\) orbiting another
NFW static host halo with \(\rmax=25\,\kpc\) and
\(\vmax = 207.4\,\kms\). These choices are motivated by the inferred
structure and masses of dSphs and the Milky Way, respectively. We
simulate the stellar component by assigning stellar weights to each
particle based on the relative densities of the distribution functions
in energy space (see Section~\ref{sec:painting_stars}). The stars follow
initially a 2D exponential profile with a scale radius
\(R_s=0.25\,\kpc\), embedded within the inner dark matter halo. We then
evolve the orbit through a pericentre of 20 kpc starting from an
apocentre of 100 kpc. See Section~\ref{sec:methods} for a more detailed
description of our numerical setup.

Fig.~\ref{fig:idealized_break_radius} illustrates the properties of this
idealized simulation at the second apocentre, as seen from the centre of
the host. The projected density of stars, is relatively undisturbed and
spherical in the centre, but becomes non-isotropic outside the break
radius and shows nascent tidal tails. These tidally disturbed stars
appear as an extended, outer density excess relative to the initial
conditions. This excess appears just outside the break radius. The break
radius also marks where the mean 3D radial velocity of the stars (with
respect to the galaxy's centre) transitions from 0 to positive --- the
galaxy is out of equilibrium outside \(r_{\rm break}\).

To first order, the final density profile of this toy simulation indeed
resembles the outer excess in Sculptor and Ursa Minor. This thesis aims
to investigate if a tidal explanation like this is viable given the
current properties of each galaxy.

\begin{figure}
\centering
\pandocbounded{\includegraphics[keepaspectratio]{figures/idealized_break_radius.pdf}}
\caption[Example tidal simulation]{Example density and velocity
distributions of an idealized simulation at the second apocentre
(initialized on the first apocentre). \textbf{Top}: The projected 2D
stellar density in the \(y'\)--\(z'\) plane for the initial (left) and
final (right) simulation. \(y'\) is the direction of tangental orbital
motion and \(z'\) is the direction of angular momentum, as seen from the
centre of the host. The dashed green circle represents the break radius
(Eq.~\ref{eq:r_break}) and the blue arrow points in the orbital
direction. \textbf{Bottom left}: The projected stellar density profile
for the initial (dotted) and final (solid) simulation snapshot.
\textbf{Bottom left}: the mean radial velocity (dot product of relative
position and velocity relative to dwarf centre) as a function of
projected radius. The green arrow marks the break radius in both lower
panels.}\label{fig:idealized_break_radius}
\end{figure}

\section{Thesis outline}\label{thesis-outline}

The goal of this thesis is to review the evidence for an extended
density profile in Ursa Minor and Sculptor, to assess the impact of
tidal effects on each galaxy, and to discuss possible interpretations
for the structure of these galaxies.

In Chapter \ref{sec:observations}, we describe how we compute
observational density profiles following \citet{jensen+2024}. In Chapter
\ref{sec:methods}, we review our simulation methods. Next, we present
our results for the tidal effects on Sculptor and Ursa Minor in Chapter
\ref{sec:results}. We discuss the implications of our results and ending
with a summary and outlook in Chapter \ref{sec:discussion}.
