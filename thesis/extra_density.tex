\chapter{The reliability of derived density
profiles}\label{sec:extra_density}

In this section, we detail the J+24 membership algorithm and test the
resulting density profiles for Scl and UMi. We compare profiles derived
with alternative methodologies and literature results. We also present a
non-parametric Bayesian density profile, which reproduces J+24's results
in the inner regions but diverges when the background dominates. In
summary, we find the density profiles are robust up to the
background-limited radius derived here.

\section{Bayesian membership
probabilities}\label{bayesian-membership-probabilities}

To create a high-quality sample, J+24 select stars initially from Gaia
within a 2--4 degree region around each satellite with high-quality
astrometry, reliable photometry, and consistent parallaxes and broadly
consistent proper motions and colours. Stars are removed if they have
excess astrometric noise \citep[\({\rm ruwe} > 1.3\),
see][]{lindegren+2021}, colour excess \(3\sigma\) outside of
expectations \citep[from][]{riello+2021}, proper motions magnitudes
\(>10\,\masyr\) in \(\mu_{\alpha*}\) or \(\mu_\delta\), magnitudes
brighter than the tip of the red giant branch or fainter than \(G=22\),
or colours outside \(-0.5 < G_{\rm BP} - G_{\rm RP} <  2.5\). Photometry
is dereddened with \citet{schlegel+finkbeiner+davis1998} extinction
maps.

J+24 construct satellite likelihoods in spatial, proper motion, and CMD
space following expected satellite properties. J+24 model the spatial
likelihood as either a one or two-component exponential
(Eq.~\ref{eq:exponential_law}). The structural parameters of the inner
component are fixed, and marginalized over if one-component. The outer
profile scale radius and normalization are free parameters.

The PM likelihood is a bivariate gaussian with variance and covariance
equal to each star's proper motions.

J+24 model the CMD as a Padova isochrone \citep{girardi+2002}, with
matching metallicity, 12 Gyr age (or 2 Gyr for Fornax), and Gaussian
width of 0.1 mag plus the \emph{Gaia} colour uncertainty at each
magnitude. The horizontal branch is modelled as a
constant-magnitude\footnote{Specifically, the mean magnitude of the 12
  Gyr, \({\rm [Fe/H]}=-2.2\), Padova isochrone's horizontal branch.}
sequence extending blue of the CMD with the same width as the RGB. The
CMD likelihoods are marginalized over the distance modulus and take the
maximum of the RGB and horizontal branch likelihoods.

The background likelihoods are determined empirically as a kernel
density estimates from stars outside 5\(R_h\) in PM and CMD space. The
spatial background likelihood is uniform.

J+24 derive the distributions of parameters (proper motions, satellite
fraction \(f_{\rm sat}\), and second spatial component if included)
through Monte Carlo Markov chain sampling with broad or
weakly-informative priors. The posterior modes are used to calculate the
final \(P_{\rm sat}\) values.

\section{\texorpdfstring{Possible biases in \emph{Gaia}-derived density
profiles}{Possible biases in Gaia-derived density profiles}}\label{sec:density_extra}

In this section, we test how assumptions in J+24's algorithm or biases
in \emph{Gaia} may affect density profiles. In all cases, density
profiles converge until \(R_{\rm ell} \approx 60\,{\rm arcmin}\), where
background contamination likely dominates.

\textbf{Spatial likelihood.} As J+24 include a spatial likelihood, our
density profiles risks ``double fitting''---the spatial distribution of
stars informs the membership catalogue from which density profiles are
derived. If the assumed spatial likelihood is misrepresentative, the
derived density profiles may bias towards the assumed likelihood.
Fig.~\ref{fig:scl_umi_density_extras} illustrates that selecting the
one- or two-component spatial models affects the outer extent of the
dwarf galaxy. Promisingly, the density excess persists when assuming a
one-component exponential profile. We revisit this concern using
spatially independent methods below.

\textbf{Structural uncertainties}. The assumed structural parameters of
a dwarf (centre, position angle, ellipticity, scale radius) may be
biased or vary radially. As a test, we use a membership sample J+24
derive assuming the outer component is circular. Even when binning in
circular radius, the density profiles remain unchanged (the
\texttt{circ} points in Fig.~\ref{fig:scl_umi_density_extras}). Thus,
our conclusions appear to be robust to an extreme shift in ellipticity.
Reasonable changes in other structural parameters are unlikely to have a
stronger effects.

\textbf{Completeness}. \emph{Gaia} appears to have high but imperfect
completeness, more limited for crowded fields, faint sources
(\(G\gtrsim20\)), and BP-RP magnitudes. In \citet{fabricius+2021}, the
completeness down to \(G\approx 20\) is \(\sim 80\%\) for low-density
globular clusters. As dwarfs are even less dense, their completeness is
likely higher. To test for magnitude-dependent biases, we show density
profiles derived for only the brighter half of member stars in
Fig.~\ref{fig:scl_umi_density_extras}. We find no substantive change
compared to our fiducial density profile. If \emph{Gaia}'s
incompleteness is inhomogeneous, then these variations likely do not
affect our density profiles.

\begin{figure}
\centering
\pandocbounded{\includegraphics[keepaspectratio]{figures/density_methods_extra.pdf}}
\caption[Sculptor density methodology comparison]{Density profiles for
various assumptions for Sculptor. \texttt{2-exp} is the fiducial
double-exponential-likelihood J+24 sample, \texttt{1-exp} instead is a
one-component exponential spatial likelihood, \texttt{simple} uses the
simple position-independent selection criteria, \texttt{circ} is a
2-component bayesian model assuming circular radii, \texttt{bright} only
includes stars brighter than the median magnitude, and \texttt{DELVE} or
\texttt{UNIONS} use photometry from external surveys with background
subtraction. In all cases, the density profiles are nearly identical
until the background-limited regime (grey shaded
region).}\label{fig:scl_umi_density_extras}
\end{figure}

\section{Comparison against alternate
samples}\label{sec:simple_selection}

Next, to compare against simpler methods, we create samples using
absolute filters in the CMD, PM space. We also compare against similarly
selected samples from deep photometry-only data.

For the \texttt{simple} \emph{Gaia} samples, we require high-quality
astrometry (\texttt{ruwe} \textless{} 1.3), parallaxes
\(3\sigma\)-consistent with the dwarf's distance, proper motions with
\(1\masyr\) radius of the galaxy's value, and within the dwarf's
empirical CMD (Fig.~\ref{fig:extra_cmd}).

We also determine density profiles from the deeper photometric surveys.
For Scl, we use DELVE DR2 survey \citep{drlica-wagner+2022}. We select
sources within an ellipse of radius 150 arcminutes, categorized as
likely stars, with reliable \(g\) and \(r\) magnitudes (associated flags
\(\leq 4\)), and within Scl's observed CMD (Fig.~\ref{fig:extra_cmd}).
Ursa Minor is within the Ultraviolet Near-Infrared Optical Northern
Survey \citep[UNIONS,][]{gwyn+2025}. We select sources within an ellipse
of radius 230 arminutes, with no \texttt{FLAGS\_CFIS} set, the
\texttt{s21} and \texttt{s31} both \(<3\) (i.e.~not extended sources),
and within the CMD cut in Fig.~\ref{fig:extra_cmd}.
Fig.~\ref{fig:delve_unions_tangent} shows the resulting spatial
distribution for both samples. The only remarkable structure is Muñoz 1
nearby to Ursa Minor.

Finally, Fig.~\ref{fig:scl_umi_density_extras} compares the density
profiles from each sample. All samples agree until the ``limiting
radius'', where the \texttt{simple} and DELVE/UNIONS samples reach their
respective backgrounds. The very outskirts of J+24's density profiles
may also extend below the background of apparent member stars,
complicating the reliability of the outer density profiles.

\begin{figure}
\centering
\includegraphics[width=0.8\linewidth,height=\textheight,keepaspectratio]{figures/extra_cmd_selection.png}
\caption[Colour-Magnitude sample selection]{Colour-magnitude diagram
cuts used in samples in this section. The likely-member stars in the
inner \(1R_h\) of each dwarf (black points) are used as a guide to
determine the CMD cuts (orange polygons). Light grey points instead show
the full distribution of stars across the field.}\label{fig:extra_cmd}
\end{figure}

\begin{figure}
\centering
\pandocbounded{\includegraphics[keepaspectratio]{figures/delve_unions_tangent.pdf}}
\caption[DELVE and UNIONS spatial distribution of stars]{The
distribution of DELVE and UNIONS selected stars in Scl and UMi as grey
points overdrawn with isophotes. The clump to the lower right of UMi is
Muñoz 1.}\label{fig:delve_unions_tangent}
\end{figure}

\section{A Bayesian density profile}\label{sec:mcmc_hists}

To address the concerns discussed above, we consider here a
non-parametric model to fit the density in each bin. We demonstrate that
the non-parametric model fails to find evidence for stellar density
where J+24 still detects members. Notably, our density profiles diverge
substantially from J+24 for Antlia II. We suggest these differences
arise from background contamination, motivating our ``limiting radii''
representing where densities may become unreliable.

\subsection{Methodology}\label{methodology}

As a non-parametric but similar model to J+24, we consider a piecewise
constant spatial likelihood. In this model, the stars are divided into
radial bins which are then considered independently. Each bin has only
one free parameter, the fraction of satellite stars in that bin,
\(f_{\rm sat}\). This model thus directly derives the density profile
from the data in a single step. If a bin contains insufficient
information to estimate a precise satellite density, than the posterior
\(f_{\rm sat}\) should reflect this uncertainty.

We parameterize \(f_{\rm sat}\) in terms of a log-relative density,
\(\theta\) \begin{equation}{
\begin{split}
\theta &\equiv \log_{10}({\Sigma_{\rm sat}}/{\Sigma_{\rm bg}}) \\
f_{\rm sat} &= \frac{10^{\theta}}{1 + 10^{\theta}}.
\end{split}
}\end{equation} We then adopt a broad uniform prior on \(\theta\) from
-12 to 6. The CMD and PM likelihoods are unchanged from J+24, except the
systemic PM is fixed for efficiency.

To bin stars, we hold fixed J+24's structural parameters and create bins
of the wider of the width \(\Delta \log R=0.05\) or containing the next
20 stars. We then derive posterior \(f_{\rm sat}\) distributions with
MCMC (48 walkers of 1000 steps each, and using the No-U-Turn sampler as
implemented in Turing.jl). The final density profile is directly derived
from \(f_{\rm sat}\) and the number of stars in each bin.

\subsection{Results}\label{results}

\begin{table*}[t]
\centering
\caption[The limiting radii of Gaia-derived density profiles]{For each classical dwarf, the limiting radius $R_{\rm limit}$ in units of $R_h$ and arcminutes. $R_{\rm limit}$ represents where there no longer appears to be evidence of stars in Emph(Str(Gaia)) using the nonparametric MCMC density profiles. }
\label{tbl:mcmc_props}
\begin{tabular}{lll}
\toprule
Galaxy & $R_{\rm limit} / R_h$ & $R_{\rm limit} / '$\\
\midrule
Fornax & 5.25 & 79.1\\
Sculptor & 6.39 & 64.1\\
Leo I & 4.23 & 13.5\\
Ursa Minor & 6.42 & 86.4\\
Leo II & 3.63 & 8.76\\
Carina & 4.16 & 33.3\\
Draco & 3.59 & 27.3\\
Canes Venatici I & 1.95 & 12.5\\
Sextans I & 3.42 & 67.9\\
Crater II & 1.93 & 39.0\\
\bottomrule
\end{tabular}
\end{table*}

Figs.~\ref{fig:mcmc_hists}, \ref{fig:mcmc_hists2} show the derived, MCMC
density profiles as compared to J+24. In general, both methodologies are
consistent. However, J+24 tend to systematically overestimate faint
densities and confidently derive densities where the MCMC model fails to
derive a density estimate.

In the outskirts of satellites, more background / foreground stars may
have consistent PM and CMD properties than satellite members. Improperly
estimating the satellite's density in this ``background-limited'' regime
likely affects the inferred density of members. If the satellite's
density is severely overestimated, then a J+24-like sample may select
many additional background stars. The density of candidates would then
be biased towards the assumed density. As J+24 only use a one or
two-component density profile across the entire dwarf, the density
profile is likely biased where the CMD+PM background dominates. The
nonparametric MCMC model instead does not assume a local density, so
should better represent the underlying satellite density.

Antlia II represents an extreme example---J+24 and the non-parametric
density model systematically disagree across the entire galaxy.
Additionally, the piecewise model derives densities with larger
uncertainties and over a smaller range than J+24. The methodological
divergence here likely arises due to an extraordinary
background/foreground of MW stars. \emph{Gaia} data alone may be unable
to properly constrain the density of such background-contaminated
objects.

To properly compare density profiles before background-limiting effects
become important, we only calculate our profiles in the main text out to
the radii in Table~\ref{tbl:mcmc_props}. We derive these
``background-limited'' radii based on the outermost derived density in
the MCMC non-parametric model with an uncertainty lower than 1 dex. This
closely corresponds to the background density from the ``CMD+PM''
samples in the main text (see Fig.~\ref{fig:scl_observed_profiles}).

\begin{figure}
\centering
\includegraphics[width=1\linewidth,height=\textheight,keepaspectratio]{figures/mcmc_histograms.png}
\caption[Probabilistic density profiles]{A comparison between the MCMC
histogram method and J+24. The MCMC samples in each bin are black
transparent dots (with added jitter), and the J+24 derived density
profiles (i.e., \(P_{\rm sat} < 0.2\)) are orange solid
dots.}\label{fig:mcmc_hists}
\end{figure}

\begin{figure}
\centering
\includegraphics[width=1\linewidth,height=\textheight,keepaspectratio]{figures/mcmc_histograms2.png}
\caption[Probabilistic density profiles
continued]{Fig.~\ref{fig:mcmc_hists} continued.}\label{fig:mcmc_hists2}
\end{figure}

\section{Comparison to literature}\label{comparison-to-literature}

Finally, we compare our density profiles against a sample of
literature-derived density profiles in
Figs.~\ref{fig:scl_lit_profiles}, \ref{fig:umi_lit_profiles}. Deviations
between different profiles are slight, despite the range of methods
across time, and all density profiles extending into the outskirts of
Scl and UMi show a similar overdensity to the ones we find. The extended
stellar profiles of Scl and UMi appear to be a robust result across the
literature.

\begin{figure}
\centering
\pandocbounded{\includegraphics[keepaspectratio]{figures/scl_literatre_profiles.pdf}}
\caption[Sculptor literature density profiles]{A comparison of the Scl
derived density profile and historical works. Unlike most profiles in
this thesis, this density profile is plotted with respect to the
semi-major elliptical radius (\(a = R_{\rm ell} / \sqrt{1-{\rm ell}}\)).
The solid black line is a 2D exponential with corresponding scale radius
to Scl's half-light radius, and the residuals in the bottom panel are
with respect to this profile. References are (in order),
\citet{munoz+2018}; \citet{westfall+2006}; \citet{walcher+2003};
\citet{eskridge1988}; \citet{demers+krautter+kunkel1980}; and
\citet{hodge1961}.}\label{fig:scl_lit_profiles}
\end{figure}

\begin{figure}
\centering
\pandocbounded{\includegraphics[keepaspectratio]{figures/umi_literature_profiles.pdf}}
\caption[Ursa Minor literature density profiles]{Similar to
Fig.~\ref{fig:scl_lit_profiles} except for Ursa Minor. The references
are (in order) \citet[derived using their minor axis
profile]{sato+2025}; \citet{palma+2003}; \citet{martinez-delgado+2001};
\citet{kleyna+1998}; \citet{IH1995}; and
\citet{Hodge1964}.}\label{fig:umi_lit_profiles}
\end{figure}

\section{Summary}\label{summary}

In this Appendix, we discussed the methodological details of the J+24
sample selection algorithm. We consider possible biases due to the
assumed spatial likelihood, \emph{Gaia}'s completeness, and structural
parameters. In all cases, we find these assumptions likely do not cause
major biases in the derived density profiles. However, we show that
J+24's density profiles may become unreliable when dropping below the
background of satellite-like stars. Finally, comparing our density
profiles against the literature, we find our results to be consistent.
We conclude that the detection of an extended density profile in Scl and
UMi is robust to incompleteness, methodology, alternative surveys, and
across the literature.
