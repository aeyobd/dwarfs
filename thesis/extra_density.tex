\chapter{The reliability of derived density
profiles}\label{sec:extra_density}

In this section, we describe the details of J+24's likelihoods. We then
test the reliability of the derived density profiles for Scl and UMi,
comparing methodologies, samples, and to literature. Finally, we present
a comparison of J+24's samples to a non-parametric Bayesian density
profile, finding similar results in the inner regions. Past a ``limiting
radius'', J+24's density profiles become susceptible to the likelihood
specification. However, all methods agree up to the limiting radii. In
conclusion, the density profiles from J+24 are robust except in the very
outer regions.

\section{Bayesian membership
probabilities}\label{bayesian-membership-probabilities}

To create a high-quality sample, J+24 select stars initially from Gaia
within a 2--4 degree circular region centred on the dwarf satisfying:

\begin{itemize}
\tightlist
\item
  Solved astrometry, magnitude, and colour.
\item
  Renormalized unit weight error, \({\rm ruwe} \leq 1.3\). Ensures high
  quality astrometry. \texttt{ruwe} is a measure of the excess
  astrometric noise on fitting a consistent parallax-proper motion
  solution \citep[see][]{lindegren+2021}.
\item
  3\(\sigma\) consistency of measured parallax with dwarf's distance,
  with \citet{lindegren+2021}'s zero-point correction. Note that the
  typical satellite parallax is very small.
\item
  Absolute proper motions, \(\mu_{\alpha*}\), \(\mu_\delta\), less than
  10\(\,{\rm mas\ yr^{-1}}\). This corresponds to tangental velocities
  of \(\gtrsim 500\) km/s at distances larger than 10 kpc.
\item
  Corrected colour excess is within 3\(\sigma\) of the expected
  distribution from \citet{riello+2021}. Removes stars with unreliable
  photometry.
\item
  De-reddened \(G\) magnitude is between
  \(22 > G > G_{\rm TRGB} - 5\sigma_{\rm DM}\). Removes very faint stars
  and stars significantly brighter than the tip of the red giant branch
  (TRGB) magnitude plus the distance modulus uncertainty
  \(\sigma_{\rm DM}\).
\item
  De-reddened colour is between \(-0.5 < {\rm BP - RP} <  2.5\). Removes
  stars substantially outside the expected CMD.
\end{itemize}

Photometry is dereddened with \citet{schlegel+finkbeiner+davis1998}
extinction maps.

J+24 define likelihoods \({\cal L}\) measuring consistency with the MW
stellar background (\({\cal L}_{\rm bg}\)) or the satellite galaxy
(\({\cal L}_{\rm sat}\)). The total likelihoods are the product of a
spatial, PM, and CMD term: \begin{equation}{
{\cal L} = {\cal L}_{\rm space}\ {\cal L}_{\rm PM}\ {\cal L}_{\rm CMD}.
}\end{equation}

Each likelihood is normalized over their respective 2D parameter space
for both the satellite. To control the relative frequency of member and
background stars, \(f_{\rm sat}\) representing the fraction of member
stars in the field. The total likelihood for any star in this model is
the \(f_{\rm sat}\)-weighted sum of the foreground and background,
\begin{equation}\protect\phantomsection\label{eq:Ltot}{
{\cal L}_{\rm tot} = f_{\rm sat}{\cal L}_{\rm sat} + (1-f_{\rm sat}){\cal L}_{\rm bg}.
}\end{equation} Finally, the probability that any star belongs to the
satellite is then given by \begin{equation}{
P_{\rm sat} = 
\frac{f_{\rm sat}\,{\cal L}_{\rm sat}}{{\cal L}_{\rm tot}}
= \frac{f_{\rm sat}{\cal L}_{\rm sat}}{f_{\rm sat}{\cal L}_{\rm sat} + (1-f_{\rm sat}){\cal L}_{\rm bg}}.
}\end{equation}

For the satellite's spatial likelihood, J+24 consider both one-component
and a two-component density models. The one component model is
constructed as a single exponential profile ( surface density
\(\Sigma \propto e^{R_{\rm ell} / R_s}\)), with scale radius \(R_s\).
\(R_s\) is fixed to the equivalent \(R_h\) value from
\citet{munoz+2018}'s Sérsic fit. Additionally, structural uncertainties
(for position angle, ellipticity, and scale radius) are sampled over to
construct the final likelihood map. The two-component model instead adds
a second exponential,
\(\Sigma_\star \propto e^{-R/R_s} + B\,e^{-R/R_{\rm outer}}\). The inner
scale radius is fixed, and the outer scale radius and magnitude of the
second component \(R_{\rm outer}\), \(B\) are free parameters.
Structural property uncertainties are not included in the two-component
model.

The PM likelihood is a bivariate gaussian with variance and covariance
equal to each star's proper motions. J+24 assume the stellar PM errors
are the main source of uncertainty.

The satellite's CMD likelihood is based on a Padova isochrone
\citep{girardi+2002}. The isochrone has a matching metallicity and 12
Gyr age (except 2 Gyr is used for Fornax). The isochrone is given a
Gaussian colour width of 0.1 mag plus the \emph{Gaia} colour uncertainty
at a given magnitude. The horizontal branch is modelled as a constant
magnitude extending blue of the CMD (mean magnitude of -2.2, 12 Gyr HB
stars and a 0.1 mag width plus the mean colour error). A likelihood map
is constructed by sampling the distance modulus in addition to the CMD
width, taking the maximum of RGB and HB likelihoods.

The background likelihoods are instead empirically constructed. Stars
stars outside of 5\(R_h\) passing the quality cuts estimate the
background density in PM and CMD space. The density is a sum of
bivariate gaussians with variances based on Gaia uncertainties (and
covariance for proper motions). The spatial background likelihood is
assumed to be uniform over the field.

J+24 derive \(\mu_{\alpha*}\), \(\mu_\delta\), \(f_{\rm sat}\) (and
\(B\), \(R_{\rm outer}\) for two-component) through an MCMC simulation
with likelihood from Eq.~\ref{eq:Ltot}. Priors are only weakly
informative. The proper motion single component prior is same as
\citet{MV2020a}: a normal distribution with mean 0 and standard
deviation \(100\ \kms\). If using the 2-component model, the prior is
instead a uniform distribution spanning 5\(\sigma\) of single component
case w/ systematic uncertainties. \(f_{\rm sat}\) (and \(B\)) has a
uniform prior from 0 to 1. \(R_{\rm outer}\) has a uniform prior only
restricting \(R_{\rm outer} > R_s\). The mode of each parameter from the
MCMC are then reported and used to calculate the final \(P_{\rm sat}\)
values.

\section{Alternative sample selection}\label{sec:simple_selection}

As J+24's probabilistic model may systematically bias the density
profile (see discussion below), we consider samples selected both from
\emph{Gaia} and external surveys using simple, absolute selection cuts.

For \emph{Gaia} data, we filter stars by four simple selection criteria.
We require \texttt{ruwe} \textless{} 1.3 as above, selecting only stars
with high-quality astrometry. We then require the star's parallax to be
\(3\sigma\)-consistent with the dwarf's distance. Next, we select stars
with proper motions with \(1\masyr\)\hspace{0pt} radius of the galaxy's
measured value. Finally, we filter stars to lie on the dwarf's CMD in
(\(G_{\rm BP} - G_{\rm RP}\),\(G\)), within the vertices defined in
Table~\ref{tbl:colour_cuts}. This polygon is derived based on the
colour-magnitude diagram of stars in the centre of the dwarf galaxy,
where the dwarf dominates the stellar density.

In addition to \emph{Gaia}, we select stars in the Scl field from the
deeper, photometric DELVE survey \citep{drlica-wagner+2022}. We select
sources DELVE data release 2 within an ellipse of radius 150 arcminutes.
We then only include sources which are categorized as likely stars and
sources with reliable (flags \textless=4) \(g\) and \(r\) magnitudes. We
then filter by the \(g\) and \(r\) magnitudes, selecting the empirical
CMD cut using the vertices in Table~\ref{tbl:colour_cuts}.

Ursa Minor is within the Ultraviolet Near-Infrared Optical Northern
Survey \citep[UNIONS][]{gwyn+2025}. We select sources within an ellipse
of radius 230 arminutes. We keep sources with no \texttt{FLAGS\_CFIS}
set and the \texttt{s21} and \texttt{s31} both \(<3\) (i.e.~not extended
sources). We then use the polygon bounded by the vertices in
Table~\ref{tbl:colour_cuts} to select the observed red giant branch,
horizontal branch, and main sequence of the dwarf.

\begin{figure}
\centering
\pandocbounded{\includegraphics[keepaspectratio]{figures/extra_cmd_selection.png}}
\caption[CMD cuts]{Colour-magnitude diagram cuts used in samples in this
section. Orange points are selected, black are non-selected stars within
\(1R_h\), and grey stars are for all stars. \textbf{todo, only plot
black/grey and cmd outline\ldots{}}}
\end{figure}

\begin{figure}
\centering
\pandocbounded{\includegraphics[keepaspectratio]{figures/delve_unions_tangent.pdf}}
\caption[DELVE and UNIONS spatial distribution of stars]{The
distribution of DELVE and UNIONS selected stars in Scl and UMI}
\end{figure}

\section{\texorpdfstring{Uncertainties and possible biases in
\emph{Gaia}-derived density
profiles}{Uncertainties and possible biases in Gaia-derived density profiles}}\label{sec:density_extra}

In this section, we discuss additional tests and verification of the
derived density profiles. In particular, we check that alternative
methodologies do not substantially affect the density profile. In all
cases, the density profiles appear to have excellent convergence out to
\(\log R_{\rm ell} / {\rm arcmin} \approx 1.8\), about the distance
where the background dominates.

\textbf{Spatial likelihood.} J+24's algorithm was designed specifically
to detect the presence of a density excess in satellites and identify
individual stars for spectroscopic follow-up. We are instead interested
in constraining a dwarf galaxy's stellar structure. As such, one
potential problem with using J+24's candidate members is that the
stellar density model is either a single or double exponential. If the
(double) exponential model does not adequately describe a dwarf galaxy,
then J+24's derived density profiles may be systematically biased
towards the assumed (double) exponential profile.

As an example, in the lower panels of Fig.~\ref{fig:scl_density_extras};
Fig.~\ref{fig:umi_density_extras}, the density profile derived from the
fiducial (J+24) sample appears to be well-constrained at densities far
below the CMD+PM or \texttt{simple} model's background density. In
addition, the fiducial and 1-component density profiles deviate, despite
being derived identically besides the spatial likelihood assumption.

These stars are likely selecting stars from the statistical MW
background consistent either galaxy's CMD and PMs, recovering the
assumed density profile. As a result, the reliability of these density
profiles below the CMD+PM background may be questionable. A more robust
analysis, removing this particular density assumption, would be required
to more appropriately represent the knowledge of the density profile as
the background begins to dominate.

We revisit possible biases in spatial likelihood in
Section~\ref{sec:mcmc_hists}.

\textbf{Completeness}. \emph{Gaia} shows high but imperfect
completeness, particularly showing limitations in crowded fields and for
faint sources (\(G\gtrapprox20\)). In \citet{fabricius+2021}, for the
lowest density globular clusters, the completeness down to
\(G\approx 20\) is \(\sim 80\%\). As the lowest density globular
clusters are still far denser than dwarf galaxies, so the completeness
is likely higher for classical dwarfs. \emph{Gaia} is likely more
incomplete in terms of BP-RP photometry (as we use here) and for stars
with small pairwise separations (\(\lesssim 1.4''\)). The influence of
these details is unclear. Nevertheless, \emph{Gaia} should be more
complete for brighter stars.

We find little evidence for magnitude-dependent biases in the density
profiles. In
Figs.~\ref{fig:scl_density_extras}, \ref{fig:umi_density_extras} we show
density profiles derived using only stars brighter than the median
magnitude. We find no substantive differences compared to our fiducial
density profile. If \emph{Gaia}'s incompleteness is inhomogeneous, then
these variations do not appear to affect the density profiles.

If \emph{Gaia} has systematic density biases, the density profile should
vary among different facilities. We show the density profiles derived
from deep, ground-based UNIONS and DEVLE photometry (lower panels of
Figs.~\ref{fig:scl_density_extras}, \ref{fig:umi_density_extras}). For
both Scl and UMi, these alternative surveys result in near-identical
density profiles. Variations between density profiles derived from these
surveys appear to also be with uncertainties.

\textbf{Structural uncertainties}. The assumed structural parameters of
a dwarf may be misrepresentative or vary with radius. J+24 derive
probabilities where the outer stellar component is circular instead of
elliptical. Using the circular 2-component probabilities and binning in
circular radius, we find little difference in the derived density
profiles (the \texttt{circ} points in
Figs.~\ref{fig:scl_density_extras}, \ref{fig:umi_density_extras}). Thus,
our conclusions are robust to extreme variations in the ellipticity.
Reasonable changes in other structural parameters (centre, position
angle) are unlikely to have a more substantial effect.

\begin{figure}
\centering
\pandocbounded{\includegraphics[keepaspectratio]{figures/density_methods_extra.pdf}}
\caption[Sculptor density methodology comparison]{Density profiles for
various assumptions for Sculptor. \texttt{2-exp} is the fiducial
double-exponential-likelihood J+24 sample, \texttt{1-exp} instead is a
one-component exponential spatial likelihood, and \texttt{simple} uses
the simple position-independent selection criteria \texttt{circ} is a
2-component bayesian model assuming circular radii, simple is the series
of simple cuts described, bright is the sample of the brightest half of
stars (scaled by 2).}\label{fig:scl_density_extras}
\end{figure}

\section{A Bayesian density profile}\label{sec:mcmc_hists}

To address the concerns discussed above, we consider here a
non-parametric model to fit the density in each bin.

\begin{table*}[t]
\centering
\caption[Properties of probabilistically-derived density profiles]{For each classical dwarf, we have: the BG-limited radius $R_{\rm limit}$, where the density of stars is no longer reliably derived, the approximate number of member stars in Gaia, and the derived Sérsic indices from the density profiles. }
\label{tbl:mcmc_props}
\begin{tabular}{lllll}
\toprule
Galaxy & $R_{\rm limit} / R_h$ & $R_{\rm limit} / '$ & num stars & $n$\\
\midrule
Fornax & 5.25 & 79.1 & 23154 & $0.794^{+0.034}_{-0.018}$\\
Sculptor & 6.39 & 64.1 & 7024 & $1.281^{+0.04}_{-0.04}$\\
Leo I & 4.23 & 13.5 & 1242 & $0.719^{+0.059}_{-0.053}$\\
Ursa Minor & 6.42 & 86.4 & 2314 & $1.425^{+0.10}_{-0.093}$\\
Leo II & 3.63 & 8.76 & 347 & $0.707^{+0.11}_{-0.092}$\\
Carina & 4.16 & 33.3 & 2389 & $1.025^{+0.08}_{-0.076}$\\
Draco & 3.59 & 27.3 & 1781 & $1.106^{+0.084}_{-0.078}$\\
Canes Venatici I & 1.95 & 12.5 & 156 & $1.18^{+0.36}_{-0.28}$\\
Sextans I & 3.42 & 67.9 & 1830 & $1.136^{+0.10}_{-0.09}$\\
Crater II & 1.93 & 39.0 & 507 & $0.535^{+0.12}_{-0.09}$\\
\bottomrule
\end{tabular}
\end{table*}

\subsection{Methodology}\label{methodology}

This model extends the J+24 framework with two notable differences. (1)
The systemic proper motions are held fixed. This allows for a much more
efficient calculation of the likelihoods. (2) The spatial likelihood for
the satellite is a piecewise constant density function.While this model
theoretically has many more parameters to fit (one for each bin), each
parameter is independently estimated based on only the stars in that
specific bin.

We segregate the data into radial bins and independently fit the model
to only stars in each bin. This fits many small models instead of a
global large model. In detail, this means that, in each bin, the
normalized density of the background and satellite are \emph{equal}. The
only fit parameter is the \(f_{\rm sat}\) in the particular bin.
Therefore, the likelihood in each bin is the same as above except
\({\cal L}_{\rm space} = 1\) for the foreground and background.

In detail, we parameterize \(f_{\rm sat}\) in terms of the log-relative
satellite density. The model construction is then \begin{equation}{
\begin{align}
\theta_i &\equiv \log_{10}({\Sigma_{\rm sat}}/{\Sigma_{\rm bg}}) \\
f_{\rm sat} &= \frac{10^{\theta_i}}{1 + 10^{\theta_i}} \\
\theta_i &\sim {\rm Uniform}(-12, 6) \\
{\cal L}_{\rm tot} &= f_{\rm sat}{\cal L}_{\rm CMD, PM} + (1-f_{\rm sat}){\cal L}_{\rm CMD, PM} \\
\end{align}
}\end{equation}

For each satellite, we hold the structural parameters fixed to those in
J+24, create bins which are the larger of the bin containing 20 stars or
0.05dex in \(\log R_{\rm ell}\). We then sample each chain 1000 steps
for 48 walkers using Turing.jl and the No-U Turns Sampler.

Based on the \(f_{\rm sat}\) values in each bin, the number of members
is just the number of stars in the bin times \(f_{\rm sat}\) for the
bin, allowing the direct derivation of the density profiles from this
model.

We have also briefly explored a model which solves for \(f_{\rm sat}\)
for random realizations of the global structural uncertainties. These do
not appear to influence the results substantially. Another possible
inconsistency is that the half-light radii are estimated from a
different sample. We do briefly rederive structural parameters for the
sample using Sérsic fits (although with a fixed origin).

\subsection{Results}\label{results}

Figs.~\ref{fig:mcmc_hists}, \ref{fig:mcmc_hists2} show the derived, MCMC
histogram density profiles compared to J+24. In general, both
methodologies agree well. However, J+24 tend to systematically
overestimate faint densities and confidently derive densities where the
MCMC model fails. Because J+24 use only one or two components across the
entire dwarf galaxy, the faint regions become sensitive to the spatial
likelihood. Antlia II is an extreme case, where the satellite is hidden
behind a large number of foreground stars. We derive a much more
poorly-constrained and fainter total density profile than J+24, and the
divergence between the MCMC method here and J+24's profile likely shows
that \emph{Gaia} observations are insufficient to properly constrain
Antia II's density profile. This is likely due to uncertainties in the
background density of satellite-like stars in Antlia II.

Once the true density of stars drops below the background of
satellite-like stars, J+24's method will occasionally select consistent
background stars with a density directly dependent on the spatial
likelihood. The uncertainties in the outer regions are likely
underestimated.

To properly compare density profiles before background-limiting effects
become important, we only calculate our profiles in the main text out to
the radii in Table~\ref{tbl:mcmc_props}. We derive these
``background-limited'' radii based on the outermost derived density in
the MCMC histogram with an uncertainty lower than 1 dex. This typically
corresponds to the empirical background from the CMD+PM samples in the
main text.

\begin{figure}
\centering
\pandocbounded{\includegraphics[keepaspectratio]{figures/mcmc_histograms.png}}
\caption[Probabilistic density profiles]{A comparison between the MCMC
histogram method and J+24. The MCMC samples in each bin are black
transparent dots, and the J+24 derived density profiles (with their
\(P_{\rm sat} < 0.2\)) are orange solid dots.}\label{fig:mcmc_hists}
\end{figure}

\begin{figure}
\centering
\pandocbounded{\includegraphics[keepaspectratio]{figures/mcmc_histograms2.png}}
\caption[Probabilistic density profiles continued]{A continuation of
Fig.~\ref{fig:mcmc_hists}.}\label{fig:mcmc_hists2}
\end{figure}

\section{Comparison to literature}\label{comparison-to-literature}

We finally compare our density profiles against a sample of
literature-derived density profiles in
Figs.~\ref{fig:scl_lit_profiles}, \ref{fig:umi_lit_profiles}. The
J+24-derived density profiles are representative of state-of-the-art
density profiles. A few photometric surveys may extend to slightly
fainter densities, but typically with large uncertanties. Deviations
between sources are slight despite the range of methods across time. All
density profiles extending into the outskirts of Scl and UMi show a
similar overdensity to the ones we find. The extended stellar profiles
of Scl and UMi appear to be a robust result across the literature.

\begin{figure}
\centering
\pandocbounded{\includegraphics[keepaspectratio]{figures/scl_literatre_profiles.pdf}}
\caption[Sculptor literature density profiles]{A comparison of the Scl
derived density profile and historical works. Unlike most profiles in
this thesis, this density profile is plotted with respect to the
semi-major elliptical radius (\(a = R_{\rm ell} / \sqrt{1-{\rm ell}}\)).
The solid black line is a 2D exponential with corresponding scale radius
to Scl's half-light radius, and the residuals in the bottom panel are
with respect to this profile. \textbf{remove delve, umi title\ldots,,
combine??}}\label{fig:scl_lit_profiles}
\end{figure}

\begin{figure}
\centering
\pandocbounded{\includegraphics[keepaspectratio]{figures/umi_literature_profiles.pdf}}
\caption[Ursa Minor literature density profiles]{Similar to
Fig.~\ref{fig:scl_lit_profiles} except for Ursa Minor. The references
are (in order) \citet{sato+2025}; \citet{palma+2003};
\citet{martinez-delgado+2001}; \citet{kleyna+1998}; \citet{IH1995}; and
\citet{Hodge1964}.}
\end{figure}

\section{Summary}\label{summary}

In this Appendix, we discussed the methodological details of the J+24
sample selection algorithm. We consider possible biases due to the
assumed spatial likelihood, \emph{Gaia}'s completeness, and structural
parameters. In all cases, we find these assumptions likely do not cause
major biases in the derived density profiles. However, we show that
J+24's density profiles become unreliable when dropping below the
background of satellite-like stars. We address this concern through an
algorithm rederiving the density in elliptical bins. We find excellent
agreement with J+24 for most galaxies out to a background-limited
radius. Finally, comparing our density profiles against the literature,
we find our results to be consistent. We conclude that the detection of
an extended density profile in Scl and UMi is robust to incompleteness,
methodology, alternative surveys, and across the literature.
