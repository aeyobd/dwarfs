\chapter{The reliability of derived density
profiles}\label{sec:extra_density}

In this Appendix, we describe the details of J+24's likelihoods. We then
test the reliability of the derived density profiles of the dwarfs,
comparing methodologies, samples, and to literature. In conclusion, the
density profiles from J+24 converge among methods out to a
background-limited radius, derived here.

\section{Bayesian membership
probabilities}\label{bayesian-membership-probabilities}

To create a high-quality sample, J+24 select stars initially from Gaia
within a 2--4 degree circular region centred on the dwarf satisfying:

\begin{itemize}
\tightlist
\item
  Solved astrometry, magnitude, and colour.
\item
  Renormalized unit weight error, \({\rm ruwe} \leq 1.3\), ensuring high
  quality astrometry. \texttt{ruwe} is a measure of the excess
  astrometric noise on fitting a consistent parallax-proper motion
  solution \citep[see][]{lindegren+2021}.\\
\item
  3\(\sigma\) consistency of measured parallax with dwarf's distance,
  with \citet{lindegren+2021}'s zero-point correction. Note that the
  typical satellite parallax is very small.
\item
  Absolute proper motions, \(\mu_{\alpha*}\), \(\mu_\delta\), less than
  10\(\,{\rm mas\ yr^{-1}}\). This corresponds to tangental velocities
  of \(\gtrsim 500\) km/s at distances larger than 10 kpc.
\item
  Corrected colour excess is within 3\(\sigma\) of the expected
  distribution from \citet{riello+2021}. Removes stars with unreliable
  photometry.
\item
  De-reddened \(G\) magnitude is between
  \(22 > G > G_{\rm TRGB} - 5\sigma_{\rm DM}\). Removes very faint stars
  and stars significantly brighter than the tip of the red giant branch
  (TRGB) magnitude plus the distance modulus uncertainty
  \(\sigma_{\rm DM}\).
\item
  De-reddened colour is between \(-0.5 < {\rm BP - RP} <  2.5\). Removes
  stars substantially outside the expected CMD.
\end{itemize}

Photometry is dereddened with \citet{schlegel+finkbeiner+davis1998}
extinction maps.

J+24 define likelihoods \({\cal L}\) measuring consistency with the MW
stellar background (\({\cal L}_{\rm bg}\)) or the satellite galaxy
(\({\cal L}_{\rm sat}\)). The total likelihoods are the product of a
spatial, PM, and CMD term: \begin{equation}{
{\cal L} = {\cal L}_{\rm space}\ {\cal L}_{\rm PM}\ {\cal L}_{\rm CMD}.
}\end{equation}

Each likelihood is normalized over their respective 2D parameter space
for both the satellite. To control the relative frequency of member and
background stars, \(f_{\rm sat}\) representing the fraction of member
stars in the field. The total likelihood for any star in this model is
the \(f_{\rm sat}\)-weighted sum of the foreground and background,
\begin{equation}\protect\phantomsection\label{eq:Ltot}{
{\cal L}_{\rm tot} = f_{\rm sat}{\cal L}_{\rm sat} + (1-f_{\rm sat}){\cal L}_{\rm bg}.
}\end{equation} Finally, the probability that any star belongs to the
satellite is then given by \begin{equation}{
P_{\rm sat} = 
\frac{f_{\rm sat}\,{\cal L}_{\rm sat}}{{\cal L}_{\rm tot}}
= \frac{f_{\rm sat}{\cal L}_{\rm sat}}{f_{\rm sat}{\cal L}_{\rm sat} + (1-f_{\rm sat}){\cal L}_{\rm bg}}.
}\end{equation}

For the satellite's spatial likelihood, J+24 consider both one-component
and a two-component density models. The one component model is
constructed as a single exponential profile ( surface density
\(\Sigma \propto e^{R_{\rm ell} / R_s}\)), with scale radius \(R_s\).
\(R_s\) is fixed to the equivalent \(R_h\) value from
\citet{munoz+2018}'s Sérsic fit. Additionally, structural uncertainties
(for position angle, ellipticity, and scale radius) are sampled over to
construct the final likelihood map. The two-component model instead adds
a second exponential,
\(\Sigma_\star \propto e^{-R/R_s} + B\,e^{-R/R_{\rm outer}}\). The inner
scale radius is fixed, and the outer scale radius and magnitude of the
second component \(R_{\rm outer}\), \(B\) are free parameters.
Structural property uncertainties are not included in the two-component
model.

The PM likelihood is a bivariate gaussian with variance and covariance
equal to each star's proper motions. J+24 assume the stellar PM errors
are the main source of uncertainty.

The satellite's CMD likelihood is based on a Padova isochrone
\citep{girardi+2002}. The isochrone has a matching metallicity and 12
Gyr age (except 2 Gyr is used for Fornax). The isochrone is given a
Gaussian colour width of 0.1 mag plus the \emph{Gaia} colour uncertainty
at a given magnitude. The horizontal branch is modelled as a constant
magnitude extending blue of the CMD (mean magnitude of -2.2, 12 Gyr HB
stars and a 0.1 mag width plus the mean colour error). A likelihood map
is constructed by sampling the distance modulus in addition to the CMD
width, taking the maximum of RGB and HB likelihoods.

The background likelihoods are instead empirically constructed. Stars
stars outside of 5\(R_h\) passing the quality cuts estimate the
background density in PM and CMD space. The density is a sum of
bivariate gaussians with variances based on Gaia uncertainties (and
covariance for proper motions). The spatial background likelihood is
assumed to be constant.

J+24 derive \(\mu_{\alpha*}\), \(\mu_\delta\), \(f_{\rm sat}\) (and
\(B\), \(R_{\rm outer}\) for two-component) through an MCMC simulation
with likelihood from Eq.~\ref{eq:Ltot}. Priors are only weakly
informative. The proper motion single component prior is same as
\citet{MV2020a}: a normal distribution with mean 0 and standard
deviation \(100\ \kms\). If using the 2-component model, the prior is
instead a uniform distribution spanning 5\(\sigma\) of single component
case w/ systematic uncertainties. \(f_{\rm sat}\) (and \(B\)) has a
uniform prior from 0 to 1. \(R_{\rm outer}\) has a uniform prior only
restricting \(R_{\rm outer} > R_s\). The mode of each parameter from the
MCMC are then reported and used to calculate the final \(P_{\rm sat}\)
values.

\section{Alternative methodologies}\label{alternative-methodologies}

\subsection{Simplistic cuts}\label{simplistic-cuts}

Besides a global background subtraction (as used in
Section~\ref{sec:observations}), absolute cuts in CMD and PM represent
perhaps the next simplest method to construct density profiles. We
create cuts using only the following

\begin{itemize}
\tightlist
\item
  (BP-RP, G) within the vertices defined in Table~\ref{tbl:colour_cuts}
\item
  PM within a \(1\masyr\) radius of the galaxies measured value.
\item
  Parallax within \(3\sigma\) of the dwarf distance
\item
  \texttt{ruwe} \textless{} 1.3
\end{itemize}

\begin{table*}[t]
\centering
\caption[Colour cuts for density profiles.]{Colour cuts used in samples in this section. Each pair of columns contains the vertices of a polygon in the CMD within which we select stars to derive density profiles. }
\label{tbl:colour_cuts}
\begin{tabular}{llllllll}
\toprule
Scl &  &  &  & UMi &  &  & \\
\midrule
Gaia &  & DELVE &  & Gaia &  & Unions & \\
BP-RP & G & g-r & g & BP-RP & G & r-i & r\\
-0.21 & 20.58 & 0.32 & 22.58 & 0.49 & 20.83 & -0.485 & 24.13\\
-0.1 & 20.1 & 0.51 & 23.07 & 1.24 & 20.9 & 0.145 & 23.11\\
0.28 & 19.77 & 0.56 & 21.49 & 1.18 & 19.65 & 0.201 & 21.41\\
0.7 & 19.64 & 0.71 & 19.7 & 1.23 & 18.12 & 0.21 & 20.05\\
1.02 & 18.5 & 0.96 & 18.49 & 1.39 & 17.09 & 0.288 & 17.23\\
1.14 & 17.59 & 0.84 & 18.35 & 1.32 & 16.94 & 0.226 & 17.21\\
1.41 & 16.73 & 0.6 & 19.65 & 0.96 & 18.58 & 0.151 & 19.3\\
1.68 & 15.94 & 0.42 & 21.42 & 0.88 & 19.22 & 0.145 & 19.75\\
1.94 & 15.94 &  &  & 0.4 & 19.52 & 0.026 & 19.86\\
1.91 & 16.8 &  &  & -0.03 & 19.78 & 0.011 & 19.52\\
1.68 & 17.01 &  &  & 0.06 & 20.16 & -0.155 & 19.95\\
1.45 & 17.87 &  &  & 0.83 & 19.72 & -0.22 & 20.34\\
1.32 & 19.16 &  &  & 0.21 & 20.83 & -0.176 & 20.47\\
1.26 & 20.1 &  &  &  &  & -0.098 & 20.33\\
1.35 & 21 &  &  &  &  & 0.008 & 19.84\\
0.46 & 21 &  &  &  &  & 0.142 & 19.75\\
0.82 & 20.27 &  &  &  &  & 0.098 & 21.45\\
0.39 & 20.33 &  &  &  &  & -0.052 & 22.38\\
0.16 & 20.76 &  &  &  &  & -0.388 & 23.42\\
\bottomrule
\end{tabular}
\end{table*}

\subsection{Alternative samples}\label{alternative-samples}

For the DELVE sample \citep{drlica-wagner+2022}, we select everything in
data release 2 between RA of 11 and 19, and declination of -37.7 and
-29.7.

We also consider samples from the Ultraviolet Near-Infrared Optical
Northern Survey \citep[UNIONS][]{gwyn+2025} for Ursa Minor. We select
stars within a 5 degree tangent square of UMi with no
\texttt{FLAGS\_CFIS} set and \texttt{s21} and \texttt{s31} both \(<3\)
(not extended sources).

\subsection{A Bayesian histogram
model}\label{a-bayesian-histogram-model}

To address the concerns discussed above, we consider here a
non-parametric model to fit the density in each bin. This model extends
the J+24 framework with two notable differences. (1) The systemic proper
motions are held fixed. This allows for a much more efficient
calculation of the likelihoods. (2) The spatial likelihood for the
satellite is a piecewise constant density function.While this model
theoretically has many more parameters to fit (one for each bin), each
parameter is independently estimated based on only the stars in that
specific bin.

We segregate the data into radial bins and independently fit the model
to only stars in each bin. This fits many small models instead of a
global large model. In detail, this means that, in each bin, the
normalized density of the background and satellite are \emph{equal}. The
only fit parameter is the \(f_{\rm sat}\) in the particular bin.
Therefore, the likelihood in each bin is the same as above except
\({\cal L}_{\rm space} = 1\) for the foreground and background.

In detail, we parameterize \(f_{\rm sat}\) in terms of the log-relative
satellite density. The model construction is then \begin{equation}{
\theta_i \equiv \log_{10}({\Sigma_{\rm sat}}/{\Sigma_{\rm bg}})
}\end{equation}

\begin{equation}{
f_{\rm sat} = \frac{10^{\theta_i}}{1 + 10^{\theta_i}}
}\end{equation}

\begin{equation}{
\theta_i \sim {\rm Uniform}(-12, 6)
}\end{equation}

\begin{equation}{
{\cal L}_{\rm tot} = f_{\rm sat}{\cal L}_{\rm CMD, PM} + (1-f_{\rm sat}){\cal L}_{\rm CMD, PM}
}\end{equation}

For each satellite, we hold the structural parameters fixed to those in
J+24, create bins which are the larger of the bin containing 20 stars or
0.05dex in \(\log R_{\rm ell}\). We then sample each chain 1000 steps
with 48 realizations using Turing.jl and the No-U Turns Sampler. The
chains appear to converge excellently.

Based on the \(f_{\rm sat}\) values in each bin, the number of members
is just the number of stars in the bin times \(f_{\rm sat}\) for the
bin, allowing the direct derivation of the density profiles from this
model.

We have also briefly explored a model which solves for \(f_{\rm sat}\)
for random realizations of the global structural uncertainties. These do
not appear to influence the results substantially. Another possible
inconsistency is that the half-light radii are estimated from a
different sample. We do briefly rederive structural parameters for the
sample using Sérsic fits (although with a fixed origin).

\section{Comparison of density profiles}\label{sec:density_extra}

In this section, we discuss additional tests and verification of the
derived density profiles. In particular, we check that methodology
(simpler cuts, circularized radii, algorithm version) do not
substantially affect the density profile. We also compile density
profiles presented in the literature as reference. In all cases, the
density profiles appear to have excellent convergence out to
\(\log R_{\rm ell} / {\rm arcmin} \approx 1.8\), about the distance
where the background dominates.

\textbf{Spatial likelihood.} J+24 method was designed in particular to
detect the presence of a density excess and find individual stars at
large radii to be followed up. We are more interested in accurately
quantifying the density profile and size of any perturbations. One
potential problem with using J+24's candidate members is that the
algorithm assumes the density is either described by a single or double
exponential. If this model does not accurately match the actual density
profile of the dwarf galaxy, we want to understand the impact of this
assumption.In particular, in Fig.~\ref{fig:scl_observed_profiles},
notice that the \(P_{\rm sat}\) selection method produces small
errorbars, even when the density is more than 1 dex below the local
background. These stars are likely selecting stars from the statistical
MW background consistent with UMi PM+CMD, recovering the assumed density
profile. As a result, the reliability of these density profiles below
the CMD+PM background may be questionable. A more robust analysis,
removing this particular density assumption, would be required to more
appropriately represent the knowledge of the density profile as the
background begins to dominate.

\textbf{\emph{Gaia} systematics}. While \emph{Gaia} has shown excellent
performance, some notable limitations may introduce problems in our
interpretation and reliability of density profiles. Gaia systematics in
proper motions and parallaxes are typically smaller than the values for
sources of magnitudes \(G\in[18,20]\). Since we use proper motions and
parallaxes as general consistency with the dwarf, and factor in
systematic uncertainties in each case, these effects should not be too
significant. However, the systematic proper motion uncertainties becomes
the dominant source of uncertainty in the derived systemic proper
motions of each galaxy (see
Tables~\ref{tbl:scl_obs_props}, \ref{tbl:umi_obs_props}).

\textbf{Completeness}. \emph{Gaia} shows high but imperfect
completeness, particularly showing limitations in crowded fields and for
faint sources (\(G\gtrapprox20\)). As discussed in
\citet{fabricius+2021}, for the high stellar densities in globular
clusters, the completeness relative to HST varies significantly with the
stellar density. However, the typical stellar densities of dwarf
galaxies are much lower, at about 20 stars/arcmin = 90,000 stars /
degree, lower than the lowest globular cluster densities and safely
below the crowding limit of 750,000 objects/degree for BP/RP photometry.
In \citet{fabricius+2021}, for the lowest density globular clusters, the
completeness down to \(G\approx 20\) is \(\sim 80\%\). Closely separated
stars pose problems for Gaia's on-board processing, as the pixel size is
59x177 mas on the sky. This results in a reduction of stars separated by
less than 1.5'' and especially for stars separated by less than 0.6
arcseconds. The astrometric parameters of closely separated stars
furthermore tends to be of lower quality \citep{fabricius+2021}.
However, even for the denser field of Fornax, only about 3\% of stars
have a neighbour within 2 arc seconds, so multiplicity should not affect
completeness too much (except for unresolved binaries). One potential
issue is that the previous analyses do not account for our cuts on
quality and number of astrometric parameters. These could worsen
completeness, particularly since the BP-RP spectra are more sensitive to
dense fields.

In Figs.~\ref{fig:scl_density_extras}, \ref{fig:umi_density_extras} we
show the results of limiting the density profile to the brightest half
of stars. We find no substantive differences. If \emph{Gaia}'s
incompleteness is inhomogeneous, then the variations in \emph{Gaia}'s
completeness do not strongly affect the density profiles.

\textbf{Structural uncertainties}. J+24 do not account for structural
uncertainties in dwarfs for the two component case. We assume constant
ellipticity and position angle. Dwarf galaxies, in reality, are not
necessarily smooth or have constant ellipticity. J+24 test an
alternative method using circular radii for the extended density
component, and we find these density profiles are very similar to the
fully elliptical case, even when assuming circular bins for the circular
outer component. As such, even reducing the assumed ellipticity from
\(0.37-0.55\) to 0 does not substantially impact the density profiles.

\begin{figure}
\centering
\pandocbounded{\includegraphics[keepaspectratio]{figures/scl_density_methods_extra.pdf}}
\caption[Scl density comparison]{Density profiles for various
assumptions for Sculptor. PSAT is our fiducial 2-component J+24 sample,
circ is a 2-component bayesian model assuming circular radii, simple is
the series of simple cuts described, bright is the sample of the
brightest half of stars (scaled by 2), DELVE is a sample of RGB stars
(background subtracted and rescaled to
match).}\label{fig:scl_density_extras}
\end{figure}

Comparison of density profiles for each J+24 method. The fiducial is a
2-component elliptical model. However, the 1-component is still
elliptical but only contains 1 component and the circular model assumes
a circular outer density profile and bins in circular bins instead of
elliptical bins.

\begin{figure}
\centering
\pandocbounded{\includegraphics[keepaspectratio]{figures/scl_literatre_profiles.pdf}}
\caption[Scl literature density profiles]{A comparison of the Scl
derived density profile and historical works.}
\end{figure}

\subsection{Ursa Minor}\label{ursa-minor}

\begin{figure}
\centering
\pandocbounded{\includegraphics[keepaspectratio]{figures/umi_density_methods_extra.pdf}}
\caption[UMi density comparison]{Similar to
Fig.~\ref{fig:scl_observed_profiles} except for Ursa
Minor.}\label{fig:umi_density_extras}
\end{figure}

\begin{table*}[t]
\centering
\caption[Properties of probabilistically-derived density profiles]{For each classical dwarf, we have: the BG-limited radius $R_{\rm limit}$, where the density of stars is no longer reliably derived, the approximate number of member stars in Gaia, and the derived Sérsic indices from the density profiles. }
\label{tbl:mcmc_props}
\begin{tabular}{lllll}
\toprule
Galaxy & $R_{\rm limit} / R_h$ & $R_{\rm limit} / '$ & num stars & $n$\\
\midrule
Fornax & 5.25 & 79.1 & 23154 & $0.794^{+0.034}_{-0.018}$\\
Sculptor & 6.39 & 64.1 & 7024 & $1.281^{+0.04}_{-0.04}$\\
Leo I & 4.23 & 13.5 & 1242 & $0.719^{+0.059}_{-0.053}$\\
Ursa Minor & 6.42 & 86.4 & 2314 & $1.425^{+0.10}_{-0.093}$\\
Leo II & 3.63 & 8.76 & 347 & $0.707^{+0.11}_{-0.092}$\\
Carina & 4.16 & 33.3 & 2389 & $1.025^{+0.08}_{-0.076}$\\
Draco & 3.59 & 27.3 & 1781 & $1.106^{+0.084}_{-0.078}$\\
Canes Venatici I & 1.95 & 12.5 & 156 & $1.18^{+0.36}_{-0.28}$\\
Sextans I & 3.42 & 67.9 & 1830 & $1.136^{+0.10}_{-0.09}$\\
Crater II & 1.93 & 39.0 & 507 & $0.535^{+0.12}_{-0.09}$\\
\bottomrule
\end{tabular}
\end{table*}

\section{A comparison of density
profiles}\label{a-comparison-of-density-profiles}

Figs.~\ref{fig:mcmc_hists}, \ref{fig:mcmc_hists2} show the derived, MCMC
histogram density profiles s compared to J+24. In general, both
methodologies agree well. However, J+24 tend to systematically
overestimate faint densities and confidently derive densities where the
MCMC model fails. Because J+24 use only one or two components across the
entire dwarf galaxy, the faint regions become sensitive to the spatial
likelihood. Antlia II is an extreme case, where the satellite is hidden
behind a large number of foreground stars. We derive a much more
poorly-constrained and fainter total density profile than J+24, and the
divergence between the MCMC method here and J+24's profile likely shows
that \emph{Gaia} observations are insufficient to properly constrain
Antia II's density profile. This is likely due to uncertainties in the
background density of satellite-like stars in Antlia II.

. Once the true density of stars drops below the background of
satellite-like stars, J+24's method will occasionally select consistent
background stars with a density directly dependent on the spatial
likelihood. The uncertainties in the outer regions are likely
underestimated.

To properly compare density profiles before background-limiting effects
become important, we only calculate our profiles in the main text out to
the radii in Table~\ref{tbl:mcmc_props}. We derive these
``background-limited'' radii based on the outermost derived density in
the MCMC histogram with an uncertainty lower than 1 dex. This typically
corresponds to the empirical background from the CMD+PM samples in the
main text.

\begin{figure}
\centering
\pandocbounded{\includegraphics[keepaspectratio]{figures/mcmc_histograms.png}}
\caption[Probabilistic density profiles]{A comparison between the MCMC
histogram method and J+24. The MCMC samples in each bin are black
transparent dots, and the J+24 derived density profiles (with their
\(P_{\rm sat} < 0.2\)) are orange solid dots.}\label{fig:mcmc_hists}
\end{figure}

\begin{figure}
\centering
\pandocbounded{\includegraphics[keepaspectratio]{figures/mcmc_histograms2.png}}
\caption[Probabilistic density profiles continued]{A continuation of
Fig.~\ref{fig:mcmc_hists}.}\label{fig:mcmc_hists2}
\end{figure}
