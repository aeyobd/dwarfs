\chapter{\texorpdfstring{The Reliability of Projected Number Density
Profiles Derived from
\emph{Gaia}}{The Reliability of Projected Number Density Profiles Derived from Gaia}}\label{sec:extra_density}

In this section, we detail the J+24 membership algorithm and test the
resulting density profiles for Scl and UMi. We compare profiles derived
with alternative methodologies and literature results. We also present a
non-parametric Bayesian density profile, which reproduces J+24's results
in the inner regions but differs when the background dominates. We find
the density profiles are robust up to the background-limited radius
derived here.

\section{Bayesian membership
probabilities}\label{bayesian-membership-probabilities}

To create a high-quality sample, J+24 select stars initially from Gaia
within a 2--4 degree region around each satellite with high-quality
astrometry, reliable photometry, consistent parallaxes, and broadly
consistent proper motions and colours. Stars are removed if they have
excess astrometric noise \citep[\({\rm ruwe} > 1.3\),
see][]{lindegren+2021}, colour excess \(3\sigma\) outside of
expectations \citep[from][]{riello+2021}, proper motions \(>10\,\masyr\)
in \(|\mu_{\alpha*}|\) or \(|\mu_\delta|\), magnitudes brighter than the
tip of the red giant branch\footnote{Derived from the maximum RGB
  magnitude from the associated Padova isochrone plus
  \(5\sigma\)\hspace{0pt} times the distance modulus uncertainty.} or
fainter than \(G=22\), or colours outside
\(-0.5 < G_{\rm BP} - G_{\rm RP} <  2.5\). Photometry is dereddened with
\citet{schlegel+finkbeiner+davis1998} extinction maps.

J+24 construct satellite likelihoods in spatial, proper motion, and CMD
space following expected satellite properties. J+24 model the spatial
likelihood as either a one or two-component exponential
(Eq.~\ref{eq:exponential_law}). The structural parameters of the inner
component are fixed, and marginalized over if one-component. The outer
profile scale radius and normalization are free parameters.

The PM likelihood is a bivariate Gaussian with variance and covariance
equal to each star's proper motions.

J+24 model the CMD as a Padova isochrone \citep{girardi+2002}, with
metallicity matching that obtained spectroscopically, 12 Gyr age (or 2
Gyr for Fornax), and a Gaussian width of 0.1 mag plus the \emph{Gaia}
colour uncertainty at each magnitude. The horizontal branch is modelled
as a constant-magnitude\footnote{Specifically, the mean magnitude of the
  12 Gyr, \({\rm [Fe/H]}=-2.2\), Padova isochrone's horizontal branch.}
sequence extending blue of the CMD with the same width as the RGB. The
CMD likelihoods are marginalized over the distance modulus and take the
maximum of the RGB and horizontal branch likelihoods.

The background likelihoods are determined empirically from kernel
density estimates of stars outside 5\(R_h\) in PM and CMD space. The
spatial background likelihood is uniform.

J+24 derive the distributions of parameters (proper motions, satellite
fraction \(f_{\rm sat}\), and second spatial component if included)
through Monte Carlo Markov chain sampling with broad or
weakly-informative priors. The posterior modes are used to calculate the
final \(P_{\rm sat}\) values.

\section{Independent sample
selection}\label{independent-sample-selection}

To test for possible limitations in the J+24 methodology (discussed
below), we compare their density profiles against samples derived from
absolute cuts in \emph{Gaia} and external photometric surveys.

The \texttt{simple} sample is selected from stars in \emph{Gaia} using
absolute astrometric and quality cuts. We require high-quality
astrometry (\texttt{ruwe} \textless{} 1.3), parallaxes
\(3\sigma\)-consistent with the dwarf's distance, proper motions with
\(1\masyr\) of the systemic value, and colours within the dwarf's
apparent CMD as determined from stars near the dwarf's centre (see
Fig.~\ref{fig:extra_cmd}).

We also determine density profiles from the deeper photometric surveys.
For Scl, we use data from the DECam Local Volume Exploration Survey
(DELVE) DR2 survey \citep{drlica-wagner+2022}. We select sources within
an ellipse of radius 150 arcminutes, categorized as likely stars, with
reliable \(g\) and \(r\) magnitudes (associated flags \(\leq 4\)), and
within Scl's CMD selection (orange polygon in Fig.~\ref{fig:extra_cmd}).
Ursa Minor lies within the Ultraviolet Near-Infrared Optical Northern
Survey \citep[UNIONS,][]{gwyn+2025}. We select sources within an ellipse
of radius 230 armin, with no \texttt{FLAGS\_CFIS} set, both \texttt{s21}
and \texttt{s31} \(<3\) (i.e., not extended sources), and within the CMD
polygon in Fig.~\ref{fig:extra_cmd}.

\begin{figure}
\centering
\includegraphics[width=0.8\linewidth,height=\textheight,keepaspectratio]{figures/extra_cmd_selection.png}
\caption[Colour-Magnitude sample selection]{Colour-magnitude diagram
cuts used in samples in this section. Black points represent stars
within \(1R_h\), light grey shows all field stars passing quality cuts,
and the orange polygon shows the CMD cut. The central, likely-member
stars inform the CMD selections.}\label{fig:extra_cmd}
\end{figure}

\section{\texorpdfstring{Possible biases in \emph{Gaia}-derived density
profiles}{Possible biases in Gaia-derived density profiles}}\label{sec:density_extra}

We now discuss potential biases in J+24's algorithm for \emph{Gaia}
data. We then test how these biases may influence density profiles. In
all cases, density profiles converge within
\(R_{\rm ell} \approx 60\,{\rm arcmin}\), where background contamination
begins to dominate.

\textbf{Spatial likelihood.} Since J+24 include a spatial likelihood,
the resulting density profiles risk ``double fitting''---the same
spatial distribution of stars informs the membership catalogue and the
subsequent density profiles. If the spatial likelihood misrepresents the
underlying satellite structure, then the density profiles may be biased
towards the assumed shape.

\textbf{Structural uncertainties.} The assumed structural parameters of
a dwarf (centre, position angle, ellipticity, scale radius) may be
inaccurate or vary radially. This would change the elliptical radii of
stars (\(R_{\rm ell}\)) and likewise the density profiles.

\textbf{Completeness.} \emph{Gaia}'s completeness appears high but
imperfect, more limited in crowded fields, for faint sources
(\(G\gtrsim20\)), and in \(G_{\rm BP}\)-\(G_{\rm RP}\) magnitudes. In
\citet{fabricius+2021}, the completeness down to \(G\approx 20\) is
\(\sim 80\%\) for low-density globular clusters. As dwarfs are even less
dense, their completeness is likely higher.

The top row of Fig.~\ref{fig:scl_umi_density_extras} compares the
density profiles assuming different spatial likelihoods: the
\texttt{2-exp} and \texttt{1-exp} use two or one-component spatial
likelihoods in J+24, and \texttt{simple} does not assume a spatial
likelihood (described above). All profiles agree within the
background-limited (BG-limited) radius, derived below. The
\texttt{1-exp} profile predicts lower outer densities, yet the outer
density excess in Sculptor and Ursa Minor persists. Thus, the spatial
likelihood can influence the outer regions but has less effect on the
inner regions.

The lower panels of Fig.~\ref{fig:scl_umi_density_extras} test other
possible biases. The \texttt{circ} profile uses circular bins and the
J+24 sample, assuming a circular outer profile. As the profile is nearly
identical to the fiducial model, an extreme change in ellipticity does
not affect our conclusions. The \texttt{bright} profile is derived using
only the brighter half of stars. We find no substantive change compared
to our fiducial density profiles, showing completeness-related magnitude
biases are likely small. Finally, the DELVE and UNIONS density profiles
nearly identically trace the \texttt{2-exp} profile, but both become
background-limited near the BG-limited radius.

In all cases, there is little evidence for biases in \emph{Gaia}'s
completeness or sensitivity to structural parameters. As the spatial
likelihood appears to be the most severe bias, we next consider a more
flexible model.

\begin{figure}
\centering
\pandocbounded{\includegraphics[keepaspectratio]{figures/density_methods_extra.pdf}}
\caption[Density methodology comparison]{Density profiles for various
assumptions for Sculptor (left) and Ursa Minor (right). Each profile is
plotted as points in log surface density (scaled to the \texttt{2-exp}
profile) versus log radius, the black line marks a 2D exponential
profile, and the grey shaded region represents the background-limited
region derived in
Section~\ref{sec:mcmc_hists}.}\label{fig:scl_umi_density_extras}
\end{figure}

\section{A Bayesian density profile}\label{sec:mcmc_hists}

To address the concerns discussed above, we consider here a
non-parametric model to fit the density in each bin. We demonstrate that
the non-parametric model fails to find evidence for stellar density,
where J+24 still detects members. Notably, our density profiles diverge
substantially from J+24 for Antlia II. We suggest these differences
arise from background contamination, motivating our ``limiting radii''
representing where densities may become unreliable.

\subsection{Methodology}\label{methodology}

As a non-parametric but similar model to J+24, we consider a piecewise
constant spatial likelihood. In this model, the stars are divided into
radial bins, which are then considered independently. Each bin has only
one free parameter, the fraction of satellite stars in that bin,
\(f_{\rm sat}\). This model thus directly derives the density profile
from the data in a single step. If a bin contains insufficient
information to estimate a precise satellite density, then the posterior
\(f_{\rm sat}\) should reflect this uncertainty.

We parameterize \(f_{\rm sat}\) in terms of a log-relative density,
\(\theta\) \begin{equation}{
\begin{split}
\theta &\equiv \log_{10}({\Sigma_{\rm sat}}/{\Sigma_{\rm bg}}) \\
f_{\rm sat} &= \frac{10^{\theta}}{1 + 10^{\theta}}.
\end{split}
}\end{equation} We then adopt a broad uniform prior on \(\theta\) from
-12 to 6. The CMD and PM likelihoods are unchanged from J+24, except
that the systemic PM is fixed for efficiency.

To bin stars, we hold fixed J+24's structural parameters and create bins
of the wider of the width \(\Delta \log R=0.05\) or containing the next
20 stars. We then derive posterior \(f_{\rm sat}\) distributions with
MCMC (48 walkers of 1000 steps each, and using the No-U-Turn sampler as
implemented in Turing.jl). The final density profile is directly derived
from \(f_{\rm sat}\) and the number of stars in each bin.

\subsection{Results}\label{results}

\begin{table*}[t]
\centering
\caption[The limiting radii of Gaia-derived density profiles]{For each classical dwarf, the limiting radius $R_{\rm limit}$ in units of $R_h$ and arcminutes. $R_{\rm limit}$ represents where there is no longer evidence of Emph(Str(Gaia)) members using the nonparametric MCMC density profiles. }
\label{tbl:mcmc_props}
\begin{tabular}{lll}
\toprule
Galaxy & $R_{\rm limit} / R_h$ & $R_{\rm limit} / '$\\
\midrule
Fornax & 5.25 & 79.1\\
Sculptor & 6.39 & 64.1\\
Leo I & 4.23 & 13.5\\
Ursa Minor & 6.42 & 86.4\\
Leo II & 3.63 & 8.76\\
Carina & 4.16 & 33.3\\
Draco & 3.59 & 27.3\\
Canes Venatici I & 1.95 & 12.5\\
Sextans I & 3.42 & 67.9\\
Crater II & 1.93 & 39.0\\
\bottomrule
\end{tabular}
\end{table*}

Figs.~\ref{fig:mcmc_hists}, \ref{fig:mcmc_hists2} show the derived MCMC
density profiles as compared to J+24. In general, both methodologies
produce similar densities. However, J+24 tend to systematically
overestimate faint densities and confidently derive densities where the
MCMC model fails to derive a density estimate.

In the outskirts of satellites, more background/foreground stars may
have consistent PM and CMD properties than satellite members. Improperly
estimating the satellite's density in this ``background-limited'' regime
likely affects the inferred density of members. If the satellite's
density is severely overestimated, then a J+24-like sample may select
many additional background stars. The density of candidates would then
be biased towards the assumed density. As J+24 only use a one or
two-component density profile across the entire dwarf, the density
profile is likely biased where the CMD+PM background dominates. The
nonparametric MCMC model instead does not assume a local density, so it
should better represent the underlying satellite density.

Antlia II represents an extreme example---J+24 and the non-parametric
density model systematically disagree across the entire galaxy.
Additionally, the piecewise model derives densities with larger
uncertainties and over a smaller range than J+24. The methodological
divergence here likely arises due to an extraordinary
background/foreground of MW stars. \emph{Gaia} data alone may be unable
to properly constrain the density of such background-contaminated
objects.

To properly compare density profiles before background-limiting effects
become important, we only calculate our profiles in the main text out to
the radii in Table~\ref{tbl:mcmc_props}. We derive these
``background-limited'' radii based on the outermost derived density in
the MCMC non-parametric model with an uncertainty lower than 1 dex. This
closely corresponds to the background density from the ``CMD+PM''
samples in the main text (see Fig.~\ref{fig:scl_observed_profiles}).

\begin{figure}
\centering
\includegraphics[width=1\linewidth,height=\textheight,keepaspectratio]{figures/mcmc_histograms.png}
\caption[Probabilistic density profiles]{A comparison between the MCMC
histogram method and J+24. The MCMC samples in each bin are black
transparent dots (with added jitter), and the J+24 derived density
profiles (i.e., \(P_{\rm sat} < 0.2\)) are orange solid
dots.}\label{fig:mcmc_hists}
\end{figure}

\begin{figure}
\centering
\includegraphics[width=1\linewidth,height=\textheight,keepaspectratio]{figures/mcmc_histograms2.png}
\caption[Probabilistic density profiles
continued]{Fig.~\ref{fig:mcmc_hists} continued.}\label{fig:mcmc_hists2}
\end{figure}

\section{Comparison to literature}\label{comparison-to-literature}

Finally, we compare our density profiles against a sample of
literature-derived density profiles in
Figs.~\ref{fig:scl_lit_profiles}, \ref{fig:umi_lit_profiles}. Deviations
between different profiles are small, despite the range of methods
across time, and all density profiles extending into the outskirts of
Scl and UMi show a similar overdensity to the ones we find. The extended
stellar profiles of Scl and UMi appear to be a robust result across the
literature.

\begin{figure}
\centering
\pandocbounded{\includegraphics[keepaspectratio]{figures/scl_literatre_profiles.pdf}}
\caption[Sculptor literature density profiles]{A comparison of density
profiles of Sculptor for works from the literature. Unlike most profiles
in this thesis, this density profile is plotted with respect to the
semi-major elliptical radius (\(a = R_{\rm ell} / \sqrt{1-{\rm ell}}\)).
The solid black line is a 2D exponential with a corresponding scale
radius to Scl's half-light radius, and the residuals in the bottom panel
are with respect to this profile. References are (in order),
\citet{munoz+2018}; \citet{westfall+2006}; \citet{walcher+2003};
\citet{eskridge1988}; \citet{demers+krautter+kunkel1980}; and
\citet{hodge1961}.}\label{fig:scl_lit_profiles}
\end{figure}

\begin{figure}
\centering
\pandocbounded{\includegraphics[keepaspectratio]{figures/umi_literature_profiles.pdf}}
\caption[Ursa Minor literature density profiles]{Similar to
Fig.~\ref{fig:scl_lit_profiles} except for Ursa Minor. The references
are (in order) \citet[derived using their minor axis
profile]{sato+2025}; \citet{palma+2003}; \citet{martinez-delgado+2001};
\citet{kleyna+1998}; \citet{IH1995}; and
\citet{Hodge1964}.}\label{fig:umi_lit_profiles}
\end{figure}

\section{Summary}\label{summary}

In this Appendix, we discussed the methodological details of the J+24
sample selection algorithm. We consider possible biases due to the
assumed spatial likelihood, \emph{Gaia}'s completeness, and structural
parameters. In all cases, we find that these assumptions likely do not
cause major biases in the derived density profiles. However, we show
that J+24's density profiles may become unreliable when dropping below
the background of satellite-like stars. Finally, comparing our density
profiles against the literature, we find our results to be consistent.
We conclude that the detection of an extended density profile in Scl and
UMi is robust to incompleteness, methodology, alternative surveys, and
across the literature.
