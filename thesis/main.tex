 %! TeX program = xelatex

\DocumentMetadata{lang=en-CA}

\documentclass[12pt,oneside,letterpaper]{report}

\special{pdf:minorversion 7}

\usepackage{amsmath}

\usepackage[T1]{fontenc}
\usepackage{newtxtext}
\usepackage{newtxmath}
% \usepackage{libertine}
% \usepackage{libertinust1math}
\usepackage{microtype}

\usepackage{graphicx}	% Including figure files
\usepackage{booktabs}
\usepackage{longtable}
\graphicspath{{./figures/}} 
\usepackage{isotope}
\usepackage[tableposition=above]{caption}

\usepackage{aas_macros}
\usepackage{natbib}
\usepackage[nottoc,numbib]{tocbibind}
\usepackage{setspace}

\usepackage{lscape} % sideways table

\setcitestyle{aysep={}} 
\setcitestyle{notesep={; }} 


\usepackage{hyperref}
\hypersetup{
    colorlinks=true,
    linkcolor=black,
    filecolor=black,
    urlcolor=black,
    citecolor=black,
    %pdftitle={The Galactic Chemical Evolution of Carbon},
    %pdfauthor={Daniel A Boyea},
    %pdfcreator={Themself},
    %pdfkeywords={astronomy, supernovae, modeling, stellar yields, chemical evolution, galaxies, carbon},
}

\urlstyle{same}


%%%%%%%%%%%%%%% Formatting %%%%%%%%%%%%%%%%%%
\usepackage{titlesec}

\titleformat{\chapter}{\normalfont\normalsize\centering\bf}{\thechapter.}{1em}{}
\titleformat{\section}{\normalfont\normalsize\centering}{\thesection}{1em}{}
\titleformat{\subsection}{\normalfont\normalsize\centering}{\thesubsection}{1em}{}
% \titlespacing*{\chapter}{0pt}{0pt}{40pt} 
% \titlespacing*{\section}{0pt}{0pt}{20pt}
\renewcommand{\contentsname}{Table of Contents}
\AtBeginDocument{
      \renewcommand{\bibsection}{\chapter*{\bibname}}
  }


\usepackage{etoolbox}% http://ctan.org/pkg/etoolbox
\makeatletter
% \patchcmd{<cmd>}{<search>}{<replace>}{<succes>}{<failure>}
\patchcmd{\@chapter}{\addtocontents{lof}{\protect\addvspace{10\p@}}}{}{}{}% LoF
\patchcmd{\@chapter}{\addtocontents{lot}{\protect\addvspace{10\p@}}}{}{}{}% LoT
\makeatother

\pagestyle{plain}
%\counterwithout{figure}{chapter}
%\counterwithout{table}{chapter}
%\counterwithout{equation}{chapter}

\newcommand\T{\rule{0pt}{2.6ex}}       % Top strut
\newcommand\B{\rule[-1.2ex]{0pt}{0pt}} % Bottom strut


\newcommand{\Mo}{{\rm M}_{\odot}}

\newcommand*{\pandocbounded}[1]{#1}

\setkeys{Gin}{keepaspectratio}


\linespread{1.2}
%\usepackage{accsupp}
%\usepackage[accsupp]{axessibility}

%%%%%%%%%%%%%%% Acronyms %%%%%%%%%%%%%%%%%%
\usepackage[acronym,automake,symbols,
nogroupskip,
accsupp,
counter=section,
stylemods={longextra}
]{glossaries-extra}
\makeglossaries


% Ordinary abbreviations
\newcommand{\kms}{{\rm km\,s^{-1}}}
\newcommand{\kmsdeg}{{\rm km\,s^{-1}\,deg^{-1}}}
\newcommand{\masyr}{{\rm mas^{-1}\,yr^{-1}}}
\newcommand{\Gyr}{{\rm Gyr}}
\newcommand{\Msun}{{\rm M_\odot}}

\setabbreviationstyle[acronym]{long-short-sc}
\GlsXtrEnablePreLocationTag{\S }{\S }





\newacronym
[description={Initial mass function. A function describing the mass distribution of newly formed stars. I use a \citet{kroupa01} {\sc imf}, which is described as a piecewise power-law function of $M$.}]
{imf}{imf}{initial mass function}

\newacronym
[description={
Apache Point Observatory Galactic Evolution Experiment. 
A large near-infrared spectroscopic survey of stars in the Milky Way. \citep{apogee17.}}]
{apogee}{apogee}{Apache Point Observatory Galactic Evolution Experiment}





%%%%%%%%%%%%%%%%%%%%%% terms %%%%%%%%%%%%%%%
\newglossaryentry{metallicity}{name={metallicity},
    description={the (mass) fraction of a star or gas which is not made of either H or He. For the sun, the metallicity is $\Zo = 0.014$}
}

\newglossaryentry{yield}{name={yield},
    description={The net production of a new element during a star's lifecycle divided by the star's mass (including winds and supernovae ejecta). }
}

\newglossaryentry{nucleosynthesis}{name={nucleosynthesis},
    description={The synthesis of new elements through fusion inside stars. See section \ref{sec:stel_evo}.}
}



\usepackage[single]{accents}

% useful commands
% below are for statement
\newcommand{\schwa}{ə}
\newcommand{\eng}{\ng{}}
%\newcommand{\overcomma}[1]{${\rm \accentset{\mbox{\normalfont ,}}{#1}}$}
\newcommand{\Lekwnen}{L\schwa{}\overcomma{k}\textsuperscript{w}\schwa{}\ng{}\schwa{}n}
\newcommand{\skipline}{\vspace{\baselineskip}}

\def\overcomma#1{%
  \setbox0=\hbox{#1}% Save the base letter in box0.
  \dimen0=\wd0       % Get its width.
  \vbox{%
    \hbox to \dimen0{\hfil,\hfil}% Create an hbox of the same width with a centered comma.
    \nointerlineskip            % Prevent extra vertical space.
    \box0                       % Place the base letter.
  }%
}

% for pandoc
\newcommand{\tightlist}{%
  \setlength{\itemsep}{0pt}\setlength{\parskip}{0pt}}



\newcommand{\about}[1]{${\sim} #1$}


\title{
    The extended light profiles of the Sculptor and Ursa Minor dwarf galaxies: Natural or tidal?
}
\author{Daniel A. Boyea}


\date{\today}


\begin{document}
% \linespread{1.25}

\pagenumbering{roman}

\makeatletter
\begin{titlepage}
   \begin{center}
       \textbf{\large \@title}\\
        \skipline

        by\\
        \skipline

       \@author\\
       B.Sc. The Ohio State University, 2023\\
       \vspace*{3\baselineskip}
    A Thesis Submitted in Partial Fulfillment of the Requirements for the Degree of\\
    \skipline

    MASTER OF SCIENCE \\
    \skipline

    in the Department of Physics and Astronomy\\
       \vfill
       {\small
       ©\@author, 2025\\
       University of Victoria\\
   }
   \skipline
       {\small
       All rights reserved. This thesis may not be reproduced in whole or in part,
   by photocopy or other means, without the permission of the author.} \\
\skipline
   \end{center}
We acknowledge and respect the \Lekwnen{} (Songhees and X\textsuperscript{w}seps\schwa{}m/
Esquimalt) Peoples on whose territory the university stands, and the
\Lekwnen{} and ${\underaccent{\bar}{\rm W}}$S\'ANE\'C Peoples whose historical relationships with the
land continue to this day. 
\end{titlepage}

\addtocounter{page}{1}


\addcontentsline{toc}{chapter}{Supervisory Committee}
\begin{centering}
\textbf{\@title}\\
\skipline
by\\
\skipline
\@author\\
B.Sc. The Ohio State University, 2023\\
\vspace*{3\baselineskip}
\end{centering}

\subsection*{Supervisory Committee}
\skipline

Dr. Julio Navarro, Supervisor\\
Department of Astronomy \\
\skipline

\noindent Dr. Kim Venn\\
Department of Astronomy\\

\skipline

\noindent External Examiner \\
Outside Department

% Abstract of the paper
\chapter*{Abstract}
\addcontentsline{toc}{chapter}{Abstract}
% context
The satellite galaxies of the Milky Way (MW), with the exception of the
Magellanic clouds, are all dwarf spheroidals (dSphs)---gas-free,
non-rotating stellar systems who's gravity is dominated by dark matter.
The light profiles of dSphs are usually well-approximated by an
exponential law, with a sharp outer cutoff. Yet systems like the
Sculptor (Scl) and Ursa Minor (UMi) dSphs have been found to host member
stars far from the main body, out to \textasciitilde10 effective radii.
The origin of these extended profiles is not clear, and may reflect
either the influence of Galactic tides or instead an innate feature. In
this thesis, I review the excestence of an extended light profiles in
Scl and UMi, validating the Bayesian membership catalogue of
\citet{jensen+2024}. To evaluate if tides indeed can explain these
density features, I conduct idealized N-body simulation of both galaxies
in the tidal field of the Milky Way. I find that neigher dwarf has been
subjected to tidal forces strong enough to create the observed extended
light profiles from initially exponential profiles. One complication is
that Sculptor's orbit is strongly influenced by the presence of the
Large Magellanic Cloud (LMC). Because the LMC's mass is poorly
constrained, Sculptor's long term orbital history is highly uncertain.
Despite this uncertainty, our models still suggest that the combined
tides of the LMC and MW are still unable to explain Scl's outer profile
over the past 5 Gyr. I conclude that the extended light profiles of Scl
and UMi are innate, likely reflecting their formation histories.
Mergers, accretion events, or complex, multi-component star formation
episodes are possible explanations for such an extended outer profile.


\chapter*{Acknowledgements}
\addcontentsline{toc}{chapter}{Acknowledgements}

%Julio Navarro, my supervisor. Rapha and other collegues

%Friends and family and collegues (fill in later).


This work has made use of data from the European Space Agency (ESA) mission
{\it Gaia} (\url{https://www.cosmos.esa.int/gaia}), processed by the {\it Gaia}
Data Processing and Analysis Consortium (DPAC,
\url{https://www.cosmos.esa.int/web/gaia/dpac/consortium}). Funding for the DPAC
has been provided by national institutions, in particular the institutions
participating in the {\it Gaia} Multilateral Agreement. 
\citep{gaiacollaboration+2016, gaiacollaboration+2023}.


This research made use of hips2fits,\footnote{https://alasky.cds.unistra.fr/hips-image-services/hips2fits} a service provided by CDS.


The Digitized Sky Surveys were produced at the Space Telescope Science Institute under U.S. Government grant NAG W-2166. The images of these surveys are based on photographic data obtained using the Oschin Schmidt Telescope on Palomar Mountain and the UK Schmidt Telescope. The plates were processed into the present compressed digital form with the permission of these institutions.

The National Geographic Society --- Palomar Observatory Sky Atlas (POSS-I) was made by the California Institute of Technology with grants from the National Geographic Society.

The Second Palomar Observatory Sky Survey (POSS-II) was made by the California Institute of Technology with funds from the National Science Foundation, the National Geographic Society, the Sloan Foundation, the Samuel Oschin Foundation, and the Eastman Kodak Corporation.

The Oschin Schmidt Telescope is operated by the California Institute of Technology and Palomar Observatory.

The UK Schmidt Telescope was operated by the Royal Observatory Edinburgh, with funding from the UK Science and Engineering Research Council (later the UK Particle Physics and Astronomy Research Council), until 1988 June, and thereafter by the Anglo-Australian Observatory. The blue plates of the southern Sky Atlas and its Equatorial Extension (together known as the SERC-J), as well as the Equatorial Red (ER), and the Second Epoch [red] Survey (SES) were all taken with the UK Schmidt.

All data are subject to the copyright given in the copyright summary. Copyright information specific to individual plates is provided in the downloaded FITS headers.

Supplemental funding for sky-survey work at the ST ScI is provided by the European Southern Observatory.




%% Lists

\tableofcontents
\listoffigures
\listoftables

%\chapter*{Symbols}
%
%
%\setlength{\tabcolsep}{0pt}
%\begin{longtable}{p{0.2\textwidth} p{0.8\textwidth}}
%$\rho$ & 3-dimensional mass density. \\
%
%$\Sigma$ & 2-dimensional density \\
%\end{longtable}
%
%\setlength{\tabcolsep}{6pt}
%\renewcommand*{\arraystretch}{1}
%
%
%\printglossary[type=\acronymtype,nonumberlist]
%\printglossary
%
%
\newpage

\pagenumbering{arabic}



%%%%%%%%%%%%%%%%%%%%%%%%%%%%%%%%%%%%%%%%%%%%%%%%%%

%%%%%%%%%%%%%%%%% BODY OF PAPER %%%%%%%%%%%%%%%%%%

\chapter{Introduction}\label{sec:introduction}
Dwarf galaxies host, in many ways, the most extreme galactic
environments in the universe. These galaxies are typically defined to be
fainter than the Large Magellanic Cloud (LMC), with \(M_V \gtrsim -18\)
or similarly \(M_\star \lesssim 10^9 M_\odot\)
\citep[e.g.,][]{hodge1971, mcconnachie2012}. Because the galaxy
luminosity function increases towards fainter objects, dwarfs are the
most numerous galaxies in the Universe
\citep[e.g.,][]{blanton+2005, mao+2021}. Dwarf galaxies are also highly
\emph{dark-matter dominated}, with mass to light ratios which may exceed
1000 \(M_\odot/ L_\odot\) \citep[e.g.,][]{simon+geha2007, hayashi+2023}.

With the exception of the Magellanic Clouds, most dwarf galaxy
satellites of the Milky Way (MW) are \emph{quenched}, with little to no
recent star formation \citep[e.g.,][]{weisz+2014}. Indeed, most faint MW
satellites contain old stellar populations which are \emph{relics} from
the early universe, consisting of many of the oldest and most metal poor
stars known. Understanding the properties of dwarf galaxies thus has
implications across astronomy, from cosmological structure formation on
small scales to the formation of metal-poor stellar populations.

In this Chapter, we first describe the general observed properties of
local dwarf galaxies. Next, we summarize our understanding of the
cosmological origin of dwarf galaxies. We later review recent
advancements and pending questions concerning dwarf galaxies, and
introduce the puzzle posed by the extended stellar density profiles of
Sculptor and Ursa Minor. We end with a brief roadmap to the remainder of
this dissertation.

\section{Observations of dwarf
galaxies}\label{observations-of-dwarf-galaxies}

Dwarf galaxies have long raised conundrums for theories of galaxy
formation. The discovery of Fornax and Sculptor in 1938
\citep{shapley1938}\footnote{Technically, the Large and Small Magellanic
  Clouds (LMC, SMC) are also classified as dwarf galaxies, but these
  were likely always known to humans at southern latitudes.}, with no
known analogues, already presented such an enigma. Shapley presented
these dwarfs as a new type of \emph{stellar system} resembling the
Magellanic Clouds and globular clusters but did not attempt to speculate
on their exact nature. While dwarf galaxies were quickly understood to
be galaxies based on the inferred luminosities and sizes, their nature
remained unclear for decades
\citep[e.g.,][]{hodge1971, gallagher+wyse1994}.

The earliest spectroscopic work hinted that dwarf galaxies may contain
substantial amounts of dark matter. From early determinations of the
velocity dispersion for the Sculptor (Scl) and Ursa Minor (UMi) dwarf
spheroidal (dSph) galaxies, inferred mass-to-light ratios were at least
10 times larger than the values of globular clusters {[}GCs; e.g.,
\citet{aaronson1983}; \citet{aaronson+olszewski1987}{]}. While rather
uncertain initially, these values were later corroborated with larger
and more precise samples \citep[e.g.,][]{hargreaves+1994}. At the time,
several theories were proposed to explain these unusually high
mass-to-light ratios. Examples include: ongoing tidal disruption
inflating inferred velocity dispersions
\citep[e.g.,][]{kuhn+miller1989}, the presence of massive central black
holes \citep[e.g.,][]{strobel+lake1994}, or modified theories of gravity
\citep{milgrom1995}. Over time, however, a consensus developed where the
high mass-to-light ratios of dwarf galaxies was due to the presence of a
dark matter halo \citep[e.g.,][]{dekel+silk1986, wechsler+tinker2018}.
Since then, the properties of dwarf galaxies have played an increasingly
important role in our understanding of the clustering of dark matter on
small scales \citep[e.g.,][]{sales+2022, bullock+boylan-kolchin2017}.

Today, a common definition for a dwarf galaxy is a gravitationally bound
stellar system with dark matter \citep[or, more specifically, not
consistent with Newtonian dynamics of visible matter
alone,][]{willman+strader2012}. In contrast, star clusters (like GCs)
have no clear evidence for dark matter. The boundary between these two
classes becomes uncertain for faint, compact stellar associations, known
as ``ambiguous'' systems.

Dwarf galaxies span a large range of physical sizes, luminosities, and
morphologies. Broadly, there are three classes of dwarf galaxies based
on luminosity, as illustrated by Fig.~\ref{fig:galaxy_images}. Local
\textbf{bright dwarfs} with magnitudes \(-14 \gtrsim M_V \gtrsim  -18\)
or stellar masses
\(3\times10^7\,\Mo \lesssim M_\star \lesssim 10^9\,\Mo\) \footnote{assuming
  stellar mass-to-light ratio of 1, may be \(\sim 2\) for older
  populations}, often exhibit irregular morphologies and recent star
formation. Fig.~\ref{fig:galaxy_images} shows the Large Magellanic Cloud
(LMC) as an example of an irregular, bright dwarf galaxy, where most
stars are in a rotationally-supported thin disk (seen nearly face-on)
with a prominent bar. \textbf{Classical dwarf spheroidals} occupy
intermediate luminosities ( \(-7.7 \gtrsim M_V  \gtrsim -14\) or
\(10^5\,\Mo \lesssim M_\star \lesssim 3\times10^7\,\Mo\)). Typically,
these systems are old, non-star forming, gas-poor, and spheroidal. All
Milky Way satellites discovered before digital sky surveys are
classicals. The 12 classical dwarf satellites of our Galaxy are (in
order of decreasing luminosity) Sagittarius, Fornax, Leo I, Sculptor,
Antlia II, Leo II, Carina, Draco, Ursa Minor, Canes Venatici I, Sextans
I, and Crater II.\footnote{While formally the dwarf galaxy names we
  discuss contain ``dwarf spheroidal'' (dSph), e.g.~Sculptor dSph, we
  omit this suffix for brevity.} The \textbf{ultra-faint}s dwarfs (UFDs)
occupy the very faintest magnitude range (\(M_V \gtrsim -7.7\) or
\(M_\star \lesssim 10^5\,\Mo\)). These galaxies have minuscule stellar
masses, typically compact sizes, and very metal-poor stellar populations
\citep[see review][]{simon2019}. Altogether, known dwarf galaxies span
more than 15 orders in magnitude, or over 7 decades in stellar mass.

Of local dwarfs, the classical systems still remain among the most
studied galaxies with the best constrained parameters. These dwarfs also
contain large numbers of bright (giant) stars which can be followed up
spectroscopically \citep[e.g.,][]{tolstoy+2023, pace+2020}. As a result,
the determination of fundamental properties such as the position, size,
orientation, distances, proper motions, line-of-sight (LOS) velocities,
and velocity dispersions \(\sigma_\V\), are all relatively well
constrained today. \textbf{revisit this\ldots JFN cuts a similar
version}

Most well-studied dwarf galaxies lie near the Milky Way or in its
vicinity, the Local Group (LG) of galaxies. TheLG is defined as the
group consisting of galaxies within \(\sim 1\) Mpc from the Milky
Way-Andromeda centre \citep[e.g.,][ and references
therein]{mcconnachie2012}. Today, we know that the Milky Way system is
teeming with dwarfs, many of which are satellites of either the Milky
Way or(MW) or Andromeda (M31). Fig.~\ref{fig:mw_satellite_system} shows
the MW satellite system, including dwarf galaxies, globular clusters,
and ambiguous systems. Such a nearby population of dwarf galaxies are
useful for resolved, detailed studies aiming to understand the nature of
these systems.

\begin{figure}
\centering
\includegraphics[width=5.41667in,height=5.41667in]{figures/galaxy_pictures.png}
\caption[Dwarf Galaxy Pictures]{Images of the LMC (DSS2), Fornax
\citep[DES DR2,][]{abbott+2021}, Sculptor (DES DR2), and Ursa Minor
\citep[UNWISE,][with \textit{Gaia} point sources
over-plotted]{lang2014, meisner+lang+schlegel2017, meisner+lang+schlegel2017a}.
The grey ellipse represents the half-light radius for the three dwarf
spheroidals, and the luminosity is derived from the absolute V-band
magnitude of each galaxy.}\label{fig:galaxy_images}
\end{figure}

\begin{figure}
\centering
\includegraphics[width=5.41667in,height=\textheight,keepaspectratio]{figures/mw_satellites_1.jpg}
\caption[Dwarf galaxies sky position]{The location of MW dwarf galaxies
on the sky. We label the classical dwarf galaxies (green diamonds),
fainter dwarfs (blue squares), globular clusters (orange circles), and
ambiguous systems (pink open hexagons). Globular clusters are more
centrally concentrated, but dwarf galaxies are preferentially found away
from the MW disk. Sculptor and Ursa Minor are highlighted as two dwarfs
we study later. The background image is from ESA/Gaia/DPAC
(https://www.esa.int/ESA\_Multimedia/Images/2018/04/Gaia\_s\_sky\_in\_colour2).
Dwarf galaxies (confirmed), globular clusters, and ambiguous systems are
from the \citet{pace2024} catalogue (version
1.0.3).}\label{fig:mw_satellite_system}
\end{figure}

\section{Dwarf galaxies in a cosmological
context}\label{dwarf-galaxies-in-a-cosmological-context}

We only understand a fraction of the universe's composition. The leading
theory of cosmology, Lambda Cold Dark Matter (\LCDM{}), posits that the
universe is composed of about 68\% dark energy (\(\Lambda\)), 27\% dark
matter (DM), and 5\% regular baryons\footnote{\emph{Baryons} here means
  baryons+leptons, i.e.~any standard model massive fermion.}
\citep{planckcollaboration+2020}. While the composition of dark matter
and dark energy remains elusive, we know their general properties. Dark
energy drives the acceleration of the expansion of the universe on large
scales. We do not discuss dark energy here---it does not substantially
affect the Local Group today. Dark matter, instead, makes up the vast
majority of mass in galaxies. Typically, galaxies have baryonic to dark
matter ratios of between 1:5 to beyond 1:1000 for faint dwarf galaxies
\citep[e.g.,][]{hayashi+2023}.

In \LCDM{}, dark matter is assumed to interact only gravitationally.
Light passes through dark matter unimpeded---in this sense, dark matter
is transparent. Dark matter is also commonly assumed to be \emph{cold},
i.e.~typical velocities much smaller than the speed of light in the
early universe. If dark matter is cold, then it should condense on all
scales, forming non-linear structures (or \emph{halos}) from the size of
galaxy clusters to smaller than the faintest dwarf galaxies.
Implications of dark matter properties include cosmological structural
formation, galaxy structure, and galaxy interactions.

\subsection{\texorpdfstring{Structure formation in
\LCDM{}}{Structure formation in }}\label{structure-formation-in}

The very early universe was almost featureless. Our earliest
observations of the universe stem from the cosmic microwave background
(CMB)---revealing a uniform, isotropic, near-perfect blackbody emission.
But tiny perturbations in the CMB, temperature fluctuations of 1 part in
10,000, reveal the underlying seeds of large-scale cosmological
structure. In an expanding universe, gravitational instability makes CDM
overdensities grow and collapse hierarchically onto larger structures.
Initially, baryonic matter was coupled to radiation and resisted
collapse. Dark matter, only influenced by gravity instead, freely
collapsed into the first structures. For mass perturbations both smaller
than the horizon and overdense enough to collapse, these overdensities
of dark matter become self-gravitating, known as \emph{halos}. After
recombination, where electrons combined with atomic nuclei to form
atoms, baryons decoupled from radiation and fell into the dark matter
halos. The densest pockets of baryons later formed the first stars and
galaxies.

Dark matter halos, and their associated galaxies, rarely evolve in
isolation. Instead, structure formation is \emph{hierarchical}. Small
dark matter halos collapse first and hierarchically merge into
progressively larger halos
\citep[e.g.,][]{blumenthal+1984, white+rees1978, white+frenk1991}.
Hierarchical assembly is evident through the large scale structure of
the universe, remnants of past mergers within the Milky Way, and tidal
disruption of dwarf galaxies and their streams around nearby galaxies.

Small-scale structure formation is sensitive to deviations from \LCDM{}
cosmology \citep[e.g.,][]{bechtol+2022}. One key prediction of \LCDM{}
is that mass perturbations are expected to exist on all scales, and are
largest on the smallest scales, so we would expect the formation of
halos on all scales. Many alternative models, such as warm dark matter,
may smooth out small-scale features and reduce the abundance of small
halos or change their structure \citep[e.g.,][]{lovell+2014}. Dwarf
galaxies, which occupy the smallest dark matter halos, are promising
windows into the behaviour of dark matter on small scales.

\subsection{The structure of cold dark matter
halos}\label{the-structure-of-cold-dark-matter-halos}

In \LCDM{} cosmological simulations, dark matter halos are remarkably
self-similar. In \citet{NFW1996, NFW1997}, hereafter NFW, the authors
observe that the spherically-averaged density profiles \(\rho(r)\) are
universally well described by a two-parameter law,
\begin{equation}\protect\phantomsection\label{eq:nfw}{
\rho/\rho_s= \frac{1}{(r/r_s)(1+r/r_s)^2},
}\end{equation} where \(r_s\) is a scale radius and \(\rho_s\) a scale
density. This profile has shown remarkable success in describing \LCDM{}
halos across several orders of magnitude in mass. NFW profiles are
\emph{cuspy}, where the density rises like \(\rho \sim 1/r\) at small
radii \(r \ll r_s\). The steepness of the density profile increases
gradually with radius, and at large radii the density falls off like
\(\rho \sim 1/r^3\).

The total mass of an NFW profile formally diverges, so halo masses are
conventionally defined using an overdensity criterion. The virial mass,
\(M_{200}\), is defined as the mass within a radius, \(r_{200}\),
containing a mean enclosed density 200 times\footnote{For the collapse
  of a uniform spherical density, the virialized overdensity would be
  \(\Delta = 18\pi^2\approx 178\) for a critical universe
  \(\Omega_m = 1\). This is commonly rounded to \(\Delta = 200\). While
  this parameter may be closer to \(\Delta \approx 100\) for our
  universe, \(\Delta\) also increases with redshift \citep[using eq. 6
  from][]{bryan+norman1998}.} the critical density of the universe:
\begin{equation}{
M_{200} =200\,\frac{4\pi}{3} \ r_{200}^3\ \rho_{\rm crit}, \qquad {\rm where} \quad \rho_{\rm crit}(z) = 3H(z)^2 / 8\pi G,
}\end{equation} and \(H(z)\) is the Hubble constant as a function of
redshift. Another way of characterizing NFW halos is through the halo
concentration, \(c=r_{200} / r_s\), which describes how the
characteristic radius scale of the halo compares to the virial radius.
Using this parameter, the scale density is a function of \(c\) alone,
\(\rho_s = (200/3)\,\rho_{\rm crit} c^3 / [\log(1+c) - c/(1+c)]\)
\citep{NFW1996}.

An equivalent, alternative characterization of NFW halos uses their
circular velocity profiles. The circular velocity,
\(\vcirc(r) = \sqrt{G M(r) / r}\), reaches a maximum \(\vmax\) at radius
\(\rmax \approx 2.16258\,r_s\). \(\vmax\) and \(r_{\rm max}\), like
\(M_{200}\) and \(c\), fully specify an NFW halo.

The blue solid curve in Fig.~\ref{fig:nfw_density} shows an example NFW
halo for a galaxy with \(\vmax=40\,\kpc\) and \(\rmax=8\,\kpc\), or
\(M_{200} = 1\times10^10\,\Mo\) and \(c=12.5\), the cosmological mean
for a Fornax stellar mass galaxy.

The two parameters of an NFW profile are not independent. Lower-mass
dark matter halos often collapse earlier, when the universe was denser.
As a result, low mass subhalos tend to be more concentrated
\citep[e.g.,][]{NFW1997}. The relationship between \(M_{200}\) and c, or
the mass-concentration relation, describes the mean trend of
concentration with mass or, equivalently, the dependence of \(\vmax\) on
\(\rmax\) \citep[e.g.,][]{bullock+2001, ludlow+2016}. The left panel of
Fig.~\ref{fig:smhm} illustrates the present-day mass-concentration from
\citet{ludlow+2016}. While concentration tends to decrease with
increasing mass, the relation has substantial scatter. Other parameters
such as the halo spin or shape may affect the scatter of the
mass-concentration relation, but their effect is typically expected to
be small \citep{navarro+2010, dicintio+2013, dutton+maccio2014}.

\begin{figure}
\centering
\includegraphics[width=0.5\linewidth,height=\textheight,keepaspectratio]{figures/example_density_profiles.png}
\caption[Example density profiles]{Density profiles in log density
versus log radius for stars and dark matter in a Fornax-like galaxy. The
dark matter is more extended and massive than the star across the entire
galaxy.}\label{fig:nfw_density}
\end{figure}

\begin{figure}
\centering
\pandocbounded{\includegraphics[keepaspectratio]{figures/cosmological_means.pdf}}
\caption[Stellar-mass halo-mass relation]{\textbf{Left} The NFW halo
concentration \(c=r_{200} / r_s\) as a function of virial mass
\(M_{200}\). The solid line with 1-\(\sigma\) shaded region is the
mass-concentration relation from \citet{ludlow+2016} for \(z=0\).
\textbf{Right} Stellar mass (top) as a function of maximum circular
velocity. The solid line with the 1-\(\sigma\) shaded region is the
relation from \citet{fattahi+2018} with scatter points simulated central
galaxies from \apostle{} in \citet{fattahi+2018}. The pink star
illustrates the location of the Fornax galaxy, whose density profiles
are shown in Fig.~\ref{fig:nfw_density}.}\label{fig:smhm}
\end{figure}

\subsection{\texorpdfstring{Galaxies formation in
\LCDM{}}{Galaxies formation in }}\label{galaxies-formation-in}

The observed abundance of galaxies may be compared with the abundance of
\LCDM{} halos to derive constraints regarding which galaxies inhabit
which halos. This technique, dubbed ``abundance matching,'' implies a
tight relation between the stellar mass of a galaxy and the mass of the
halo it inhabits \citep{li+white2009, moster+naab+white2013}.

Fig.~\ref{fig:smhm} shows the stellar mass versus halo mass relation
(SMHM, with halo mass represented by \(\vmax\)) predicted by the \LCDM{}
cosmological hydrodynamical simulations of Local Group analogues from
the \apostle{} project \citep{sawala+2016}.\footnote{\apostle{}
  simulated Local Group analogues in a \LCDM{} cosmological context with
  the hydrodynamical setup from the \eagle{} simulations
  \citep{crain+2015, schaye+2015}}. While there is some scatter, the
range of predicted \(\vmax\) is fairly narrow across \(\sim 4\) decades
in stellar mass. This figure indicates that the SMHM relation becomes
increasingly steep in the dwarf galaxy regime---many dwarf galaxies are
formed in relatively massive halos. Because lower mass galaxies have
increasingly shallow potential wells, feedback becomes more effective at
removing gas. Re-ionization additionally suppresses late star formation
in the faintest galaxies. As a result, the resulting stellar mass of a
dwarf galaxy is highly sensitive to the dark matter halo mass.

In \LCDM{} galaxy formation, the majority of mass in a dwarf galaxy
comes from the extended dark matter halo. Fig.~\ref{fig:nfw_density}
shows an example exponential stellar component for the Fornax dwarf
galaxy with the expected underlying dark matter halo. This galaxy has a
stellar mass \(M_\star \approx 2.5\times10^7\,\Mo\) with half-light
radius \(0.65\,\kpc\) \citep[ with \citet{woo+courteau+dekel2008}
mass-to-light ratio]{munoz+2018}, and a dark matter halo with
\(\vmax=40\,\kpc\) and \(\rmax=8\,\kpc\). Even when the stars are
densest (the inner regions), the dark matter remains nearly an order of
magnitude higher in density. Stars make a small contribution to the
gravitational structure of dwarf galaxies--indeed, stars are reasonably
approximated as tracer particles of the underlying dark matter halo. In
addition, the stellar component is confined to the central regions of
the dark matter halo.

Several factors affect the SMHM trend, including environment, assembly
history, tidal effects, and the details of galaxy formation. In
particular, effects like ram-pressure stripping (removal of gas in the
dwarf galaxy due to pressure from the host's circumgalactic medium) and
tidal removal of gas cause star formation to quench
\citep[e.g.,][]{christensen+2024}. Additionally, the time of formation
(relative to reionization) can influence the resulting stellar content
\citep{kim+2024}. Finally, tides influence both the dark matter and
stellar mass but in different amounts \citep[e.g.,][]{PNM2008}.
Consequently, tides may reduce the halo mass more than the stellar mass,
adding additional scatter to the SMHM trend, particularly for low-mass
satellites \citep[e.g.,][]{fattahi+2018}. Understanding the effects of
tides on Local Group dwarf galaxies is essential for determining where
and how these galaxies formed in a cosmological context.

\subsection{Open questions and ongoing challenges in dwarf
galaxies}\label{open-questions-and-ongoing-challenges-in-dwarf-galaxies}

Advances in observational techniques have accelerated progress in our
understanding of dwarf galaxies and introduced new questions. Deep
digital photometric surveys have more than quadrupled the number of
known Milky Way dwarf galaxies since 2005 \citep{simon2019}. Upcoming
surveys, like the Vera Rubin Observatory's Legacy Survey for Space and
Time \citep[LSST,][]{ivezic+2019}, will likely identify even fainter
dwarf galaxies. In addition, large aperture multi-object spectrographs
have revealed the detailed and complex inner chemodynamical structure of
dwarf galaxies \citep[e.g.,][]{battaglia+2006, fabrizio+2016}. Beyond
precise structural and kinematic properties of dwarf galaxies, modern
observations allow for the separation of multiple stellar populations,
detailed constraints on the dark matter density profiles, and hints of
tidal disruption or stellar halos. \textbf{does this ¶ (or section)
belong here?}

Observations of dwarf galaxies have been the origin of several disputes
or \emph{small-scale} problems for \LCDM{} \citep[see reviews
by][]{bullock+boylan-kolchin2017, sales+2022}. For example, the mismatch
between the expected number of dwarf galaxies from simulations and the
observed abundance has been known as the \emph{missing satellites
problem}. Additionally, a number of observations suggested that some
dwarf galaxies, although not all, possess dark matter ``cores,''
\citep[e.g.,][]{moore1994, adams+2014, oh+2015, walker+penarrubia2011, read+walker+steger2019},\footnote{In
  detail, gas-phase rotation curves are better able to differentiate
  between cores and cusps, whereas stellar kinematics is less
  constraining.} contrary to the expectation from \LCDM{} of ``cuspy''
inter dark matter profiles \citep{NFW1996, NFW1997}. As a result,
alternative forms of dark matter have been advocated as solutions, such
as Warm or Self-Interacting Dark Matter.

However, these tensions have eased as a result of improved understanding
of baryonic physics. For example, recent hydrodynamic simulations in
particular have shown that strong feedback can produce dark matter cores
\citetext{\citealp[e.g.,][\citet{tollet+2016}]{navarro+eke+frenk1996}; \citealp{fitts+2017}; \citealp{benitez-llambay+2019}; \citealp{orkney+2021}}.
Altogether, the numerous past challenges for \LCDM{} in the dwarf galaxy
regime illustrate the opportunity for dwarf galaxies to test the
understanding of galaxy formation and dark matter physics.

\section{The structure of nearby dwarf
galaxies}\label{the-structure-of-nearby-dwarf-galaxies}

\subsection{\texorpdfstring{The \emph{Gaia} space
telescope}{The Gaia space telescope}}\label{the-gaia-space-telescope}

Since Local Group dwarfs are nearby, they are resolved into individual
stars, and therefore we can study these galaxies on a star-by-star
basis. As a result, it is possible to measure the 3D velocity and
position of a star if we can measure its position, distance,
line-of-sight (LOS) velocity, and proper motion. Unfortunately,
determining distances and full 3D velocities is challenging. The most
direct measurement of distance, parallax, requires precise tracking of a
star's sky position across a year. And while line-of-sight (LOS)
velocities are relatively easily determined from spectroscopy,
tangential velocities, derived from proper motions and distances, are
much more challenging. Typically, measuring proper motions requires
accurate (much less than arcsecond) determinations of small changes in a
star's position over baselines of years to decades. The full 6D position
and velocity information for stars has, until recently, been known for
only a handful of stars.

Launched in 2013, \emph{Gaia} is a space-based, all-sky survey telescope
situated at the Sun-Earth L2 Lagrange point
\citep{gaiacollaboration+2016}. \emph{Gaia} has redefined astrometry,
providing photometry, positions, proper motions, and parallaxes for over
1 billion stars \citep{gaiacollaboration+2021}. While \emph{Gaia}
completed its space-based mission in 2025, two further data releases are
expected.

Determining absolute parallax measurement is facilitated by the
observation that stars in different regions of the sky are affected by
parallax motion with different phases. By imaging two regions separated
by 106.5 degrees on the same focal plane, \emph{Gaia} measures tiny
changes in relative positions of stars across small and large angles.
Combining measurements from multiple epochs across several years, an
absolute all-sky reference frame is derived from which parallax and
proper motions are derived. In addition to astrometry, \emph{Gaia}
measures photometry in the wide \emph{G} band (330-1050nm) and colours
from the blue photometer (BP, 330-680 nm) and red photometer (RP,
640-1050 nm). \emph{Gaia} additionally provides low resolution BP-RP
spectra and radial velocity measurements of bright stars \citep[of
magnitudes \(G_{\rm RVS} < 16\),][]{gaiacollaboration+2016}. For our
work, \emph{Gaia}'s most relevant measurements are \(G\) magnitude,
\(G_{\rm BP} - G_{\rm RP}\) colour, \((\alpha, \delta)\) position, and
\((\mu_{\alpha*}, \mu_\delta)\) proper motions.\footnote{The proper
  motions \(\mu_\alpha\) and \(\mu_\delta\) are the apparent rates of
  change in right ascension, \(\alpha\), and declination, \(\delta\),
  typically in units of mili-arcsecond (mas) per year.
  \(\mu_{\alpha*} = \mu_\alpha \cos \delta\) corrects for projection
  effects in \(\alpha\).}

\subsection{\texorpdfstring{\emph{Gaia}'s impact on Milky Way
studies}{Gaia's impact on Milky Way studies}}\label{gaias-impact-on-milky-way-studies}

\emph{Gaia} has revolutionized our understanding of Milky Way structure.
For example, the 6D dynamical measurements and metallicities of MW stars
led to the (re)discovery of past mergers or Milky Way building blocks
like \emph{Gaia}-Sausage Enceladus
\citetext{\citealp[e.g.,][]{helmi+2018}; \citealp{belokurov+2018}; \citealp[but
see also][]{meza+2005}}, out-of-equilibrium structures like the
\emph{Gaia} snail \citep[e.g.,][]{antoja+2018}, and dynamical effects of
the Milky Way's spiral arms and the bar in the solar neighbourhood
\citep[ and references therein]{hunt+vasiliev2025}. Moving to the Milky
Way halo, \emph{Gaia} has helped find and constrain numerous stellar
streams \citep{ibata+malhan+martin2019, bonaca+price-whelan2025}.
Altogether, \emph{Gaia} has revealed the hierarchical formation and
complex, evolving structure of our own Galaxy.

For Milky Way satellites, \emph{Gaia} has improved orbital analysis and
facilitated stellar membership determination. Before \emph{Gaia}, few
galaxies had precisely measured proper motions, more recently from
Hubble Space Telescope observations \citep[e.g.,][]{sohn+2017}.
\emph{Gaia} allowed the first systematic and precise determinations of
Milky Way satellite proper motions \citep{pace+li2019, MV2020a}. While
the proper motion uncertainty of a typical dwarf member star is often
large, by combining the proper motions of 100s or 1000s of stars from
\emph{Gaia}, precise average proper motion measurements can be
determined, sometimes only limited by \emph{Gaia}'s systematic error
floor \citep[e.g.,][]{MV2020a}. Proper motions have thus ushered in a
new era for MW satellite dynamical studies, where we can derive precise
orbits for any satellite, assuming a given MW potential. In addition,
\emph{Gaia} helps establish membership by separating out contaminating
MW foreground stars. By measuring parallaxes and/or proper motions, many
background and foreground stars can be classified as non-members
\citep[e.g.,][]{battaglia+2022, jensen+2024}.

\subsection{Dwarf galaxy light profiles}\label{sec:exponential_profiles}

Projected luminosity / stellar density profiles efficiently characterize
the radial structure of a galaxy. At its most basic, light profiles
synthesize properties such as the shape, location, size, and orientation
of a dwarf galaxy. In addition, the details of a stellar density profile
can help interpret a galaxy's assembly and dynamical history
\citep[e.g.,][]{penarrubia+2009, lee+2018, querci+2025}. We note that
for resolved galaxies, these profiles are in terms of stellar counts
instead of surface brightness.

Four different surface density laws are frequently used to parameterize
dwarf galaxy profiles: Exponential, Plummer, King, or Sérsic profiles
\citep[e.g.,][]{munoz+2018}. The exponential profile is perhaps the
simplest, defined in terms of the central surface density, \(\Sigma_0\),
and scale radius, \(R_s\):
\begin{equation}\protect\phantomsection\label{eq:exponential_law}{
\Sigma_{\rm exp} = \Sigma_0\exp(-R / R_s).
}\end{equation} This profile is often applied to the radial light
distribution of galaxy disks
\citep{devaucouleurs1959a, freeman1970, kent1985}.

To fit globular cluster density profiles, \citet{plummer1911} proposed a
profile based on a self-gravitating polytrope,\footnote{where density
  and pressure are assumed to be related by a power law}
\begin{equation}{
\Sigma_{\rm Pl} = \frac{\Sigma_0}{(1 + (R/R_h)^2)^2},
}\end{equation} where \(\Sigma_0\) is the central surface density and
\(R_h\) is the projected half-light radius. Now mostly superseded by the
King profile for globular clusters, the Plummer model is still a good
fit to many dwarf spheroidals \citep[e.g.,][]{moskowitz+walker2020}.

The \citet{king1962} profile, also an empirical fit to globular
clusters, is also used to describe dwarf galaxies, especially in older
literature. Using three parameters, a core radius \(R_c\), a truncation
radius \(R_t\), and a characteristic density, \(\Sigma_0\), the King
profile may be written as \begin{equation}{
\Sigma_{\rm K} = \Sigma_0\left(\frac{1}{\sqrt{1 + (R/R_c)^2}} - \frac{1}{\sqrt{1+(R_t/R_c)^2}}\right).
}\end{equation} In much of the older literature, \(R_t\) was interpreted
as a ``tidal radius'', after an analogous interpretation for globular
clusters \citep[e.g.,][]{hodge1961, IH1995}.

Finally, the \citet{sersic1963} profile represents a generalization of
an exponential profile, and describes most dwarf galaxy light profiles
well. Typically parameterized in terms of a half-light radius \(R_h\),
the density at half-light radius \(\Sigma_h\) and a Sérsic index \(n\),
the profile's equation is \begin{equation}{
\Sigma_{\rm S} = \Sigma_h \exp\left[-b_n \,  \left((R/R_h)^{1/n} - 1\right)\right]
}\end{equation} where \(b_n\) is a constant that depends on \(n\). A
Sérsic profile with \(n=1\) is equivalent to an exponential profile,
while \(n=4\) recovers \citet{devaucouleurs1948}'s profile for
elliptical galaxies. Although a Sérsic profile is not always used for
dwarf galaxies, \citet{munoz+2018} advocate for the Sérsic profile since
the added flexibility allows more profiles to be fit with a single law.

While there are no clear theoretical preferences for any of these
profiles, exponential density profiles have been commonly used for dwarf
spheroidal galaxies. \citet{faber+lin1983} were among the first to
demonstrate that an exponential law is a reasonable empirical fit,
theorizing that dwarf spheroidals may have evolved from exponential disk
galaxies and maintained a similar light profile. Later,
\citet{read+gilmore2005} showed that exponential profiles may originate
from mass loss during the evolution of dwarf galaxies. Many subsequent
photometric studies of dwarf spheroidal galaxies have used exponential
fits, finding that exponential and King profiles both provide good
descriptions in many cases
\citep{binggeli+sandage+tarenghi1984, mateo1998, mcconnachie+irwin2006, cicuendez+2018}.
As a result, it has become conventional to assume an exponential density
profile to describe dwarf galaxies in theoretical or observational
modelling
\citep[e.g.,][]{martin+2016, MV2020a, battaglia+2022, kowalczyk+2013}.

Outside Local Group dwarf spheroidals, exponential profiles are fairly
common, but sometimes with modifications. For example, many
extragalactic dwarf elliptical, blue compact, and irregular dwarf
galaxies are better described with an exponential profile to which is
added a central cusp or nuclear region
\citep{caldwell+bothun1987, noeske+2003}. On the other hand, some
studies find an inner density decrement compared to exponentials
\citep[e.g.,][]{caldwell+1992, makarov+2012}.

However, exponential profiles do not fit every dwarf galaxy. For example
\citet{aparicio+1997}, and \citet[for Coma cluster dwarf
ellipticals]{graham+guzman2003}, some people have noted deviations from
this simple empirical rule. Additionally, \citet{hunter+elmegreen2006};
\citet{herrmann+hunter+elmegreen2013};
\citet{herrmann+hunter+elmegreen2016}; and \citet{lee+2018} noted that
at least in a photometric sample of more distant disky, irregular, and
blue compact dwarfs, many or most dwarfs are best fit with two nested
exponential profiles. It is unclear how these conclusions apply to dwarf
spheroidals.

More recently, \citet{moskowitz+walker2020}, have used a Plummer profile
and modified Plummer profiles with steeper outer slopes
(\(\Gamma = d\log \Sigma / d \log R = -8\) rather than \(\Gamma = -4\)).
For most galaxies, they do not have convincing statistical evidence to
prefer one model over another, except for a few galaxies which are
better described by a steep outer slope.

Altogether, while there is some variation in the density profiles of
dwarf galaxies, an exponential is an excellent first-order
approximation. Typically, deviations from exponentials are in the
direction of a steeper outer cutoff or changes to the inner slope of a
dwarf galaxy (due, for example, to a nuclear star cluster). Flattened
density profiles in the outer regions are more unusual. Explaining in
detail the origin, similarity, and diversity of dwarf galaxy density
profiles is an open question for theories of dwarf galaxy formation and
evolution.

\subsection{The extended light profiles of Sculptor and Ursa Minor:
Hints of tidal
signatures?}\label{the-extended-light-profiles-of-sculptor-and-ursa-minor-hints-of-tidal-signatures}

Sculptor and Ursa Minor appear to be typical dwarf spheroidal galaxies
at first glance. Tables~\ref{tbl:scl_obs_props}, \ref{tbl:umi_obs_props}
describe the present-day properties of each galaxy. Sculptor, as the
first discovered classical dSph, is often described as a
``prototypical'' dSph. There has long been speculation that both
Sculptor and Ursa Minor may have been influenced by the Milky Way's
tidal field (see Section~\ref{sec:discussion}).
\citet{sestito+2023a, sestito+2023b}, for example, have recently
reported a ``kink'' in the density profile, beginning around 30 arcmin
for from the centre of each of these systems, that they interpret as
potentially caused by the effects of Galactic tides. They
spectroscopically follow up some of the most distant stars, finding
members as far as 6 and 12 half-light radii from the centre of each
dwarf. If these dwarfs had exponential profiles, like Fornax, then these
far-outlying stars should be much rarer.

Sculptor and Ursa Minor are not well-described by an exponential
profile. The left panel of Fig.~\ref{fig:scl_umi_vs_fornax} shows the
density profiles of Sculptor, Ursa Minor, and Fornax (see
Section~\ref{sec:observations} for details on how these profiles are
measured). Compared to Fornax, both Sculptor and Ursa Minor show an
excess of stars outside \(\log R/R_h\approx 0.4\) and which exceeds 100
times the density of the exponential fit at large radii.

A goal of this work is to determine, in the context of \LCDM{}, if the
effects of the Milky Way (or other satellites) are indeed responsible
for the extended outer light profiles of Sculptor and Ursa Minor. If
tides cannot explain these features, these features may instead be due
to an extended stellar ``halo'' or second component of the
galaxy---suggestive of a complex star formation or assembly history.

\begin{figure}
\centering
\pandocbounded{\includegraphics[keepaspectratio]{./figures/scl_umi_vs_fornax.pdf}}
\caption[Sculptor and Ursa Minor match tidal models]{A plot of the
surface density profiles of Sculptor (orange squares), Ursa Minor (red
triangles), and Fornax (green circles) scaled to their half-light radius
and the density at half-light radius (data described in
Section~\ref{sec:observations}). The solid black line is an exponential
profile (Eq.~\ref{eq:exponential_law}).}\label{fig:scl_umi_vs_fornax}
\end{figure}

\begin{table*}[t]
\centering
\caption[Observed Properties of Sculptor]{Observed properties of Sculptor. References are: 1. Muñoz et al. (2018) Sérsic fits, 2. Tran et al. (2022) RR lyrae distance, 3. Alan W. McConnachie and Venn (2020b), 4. Arroyo-Polonio et al. (2024). }
\label{tbl:scl_obs_props}
\begin{tabular}{lll}
\toprule
parameter & value & Source\\
\midrule
$\alpha$ & $15.0183 \pm 0.0012^\circ$ & 1\\
$\delta$ & $-33.7186 \pm 0.0007^\circ$ & ”\\
distance modulus & $19.60 \pm 0.05$ & 2\\
distance & $83.2 \pm 2$ kpc & ”\\
$\mu_{\alpha*}$ & $0.099 \pm 0.002 \pm 0.017$ mas yr$^{-1}$ & 3\\
$\mu_\delta$ & $-0.160 \pm 0.002_{\rm stat} \pm 0.017_{\rm sys}$ mas yr$^{-1}$ & ”\\
LOS velocity & $111.2 \pm 0.3\ {\rm km\,s^{-1}}$ & 4\\
$\sigma_v$ & $9.7\pm0.2\ {\rm km\,s^{-1}}$ & ”\\
$R_h$ & $9.79 \pm 0.04$ arcmin & 1\\
ellipticity & $0.37 \pm 0.01$ & ”\\
position angle & $94\pm1^\circ$ & ”\\
$M_V$ & $-10.82\pm0.14$ & ”\\
\bottomrule
\end{tabular}
\end{table*}

\begin{table*}[t]
\centering
\caption[Observed Properties of Ursa Minor]{Observed properties of Ursa Minor. References are: (1) Muñoz et al. (2018) Sérsic fits, (2) Garofalo et al. (2025) RR lyrae distance, (3) Alan W. McConnachie and Venn (2020a), (4) Pace et al. (2020), average of MMT and Keck results. }
\label{tbl:umi_obs_props}
\begin{tabular}{lll}
\toprule
parameter & value & Source\\
\midrule
$\alpha$ & $ 227.2420 \pm 0.0045$˚ & 1\\
$\delta$ & $67.2221 \pm 0.0016$˚ & ”\\
distance modulus & $19.23 \pm 0.11$ & 2\\
distance & $70.1 \pm 3.6$ kpc & ”\\
$\mu_\alpha*$ & $-0.124 \pm 0.004 \pm 0.017$ mas yr$^{-1}$ & 3\\
$\mu_\delta$ & $0.078 \pm 0.004_{\rm stat} \pm 0.017_{\rm sys}$ mas yr$^{-1}$ & ”\\
LOS velocity & $-245.9 \pm 0.3_{\rm stat} \pm 1_{\rm sys}$ km s$^{-1}$ & 4\\
$\sigma_v$ & $8.6 \pm 0.3$ & ”\\
$R_h$ & $11.62 \pm 0.1$ arcmin & 1\\
ellipticity & $0.55 \pm 0.01$ & ”\\
position angle & $50 \pm 1^\circ$ & ”\\
$M_V$ & $-9.03 \pm 0.05$ & ”\\
\bottomrule
\end{tabular}
\end{table*}

\section{Interpreting tidal signatures}\label{sec:tidal_theory}

Simulating dwarf galaxies accurately in a cosmological context remains a
challenge. Cosmological simulations can predict the overall abundance of
the most massive dwarf galaxies \citep[e.g.,][]{sawala+2016} and broadly
examine the effects of tides \citep[e.g.,][]{riley+2024}. However,
simulated dwarf galaxies are often near the resolution limit.
Insufficient resolution can lead to artificial disruption of dwarf
galaxies and over-prediction of tidal effects
\citep[e.g.,][]{vandenbosch+2018, santos-santos+2025}. To address this
challenge, idealized simulations are often used to simulate a single
subhalo in an approximate host potential, achieving excellent numerical
convergence. For instance, the simulations we describe later reach 3
times higher resolution than Aquarius \citep{springel+2008} at a
fraction of the computational cost (400 times fewer particles).
Idealized simulations make numerous simplifications, neglecting mergers,
cosmological context, mass assembly, and often baryonic physics. We
shall use idealized simulations here to assess tidal effects after
infall.

Some early simulation work on tidal mass loss of dwarf galaxies
originate from \citet{oh+lin+aarseth1995}; \citet{piatek+pryor1995};
\citet{moore+davis1994}; \citet{johnston+spergel+hernquist1995}.
Already, these works described the relative evolution of stellar and
dark matter mass loss, that most mass loss happens near pericentre, the
formation of symmetric tidal streams, and the minor changes to the
central structure.

Later, \citet{read+2006} showed that tidal tails would likely cause
increasing velocity dispersion profiles, and interpreted the lack of
similar observational features as indicating tides likely did not affect
the stellar component of dwarf galaxies \citep[see
also][]{klimentowski+2009a}.

\citet{hayashi+2003}

\citet{klimentowski+2009} shows that tides can cause a disky dwarf
galaxy to transition into a spheroidal galaxy, and the velocity
dispersion and maximum circular velocity are reasonable tracers of total
mass.

CDM halos also are found to be highly resilient to full disruption
{[}\citet{EP2020}; {]}. In addition, NFW halos evolve along ``tidal
tracks'' {[}e.g, \citet{PNM2008}, \citet{EN2021}; {]}. On the other
hand, cored dark matter halos likely disrupt faster
\citep[e.g.,][]{penarrubia+2010, errani+2023a}.

\citet{green+vandenbosch2019}

Recent, work by
\citet{drakos+taylor+benson2020, drakos+taylor+benson2022, amorisco2021}
showed that tidal effects are nearly entirely described as the removal
of particles below a certain energy \citep[see
also][]{choi+weinberg+katz2009}. \citet{strucker+2023} expanded on this
idea, creating an analytic model for adiabatic tidal mass loss,
confirming that an NFW halo is likely unable to be disrupted and
explaining the evolution of NFW halos along similar tidal tracks.

With precise orbits and a better understanding of the Milky Way
potential, recent work continues to directly probe the dynamical
histories of individual dwarf galaxies.
\citet{borukhovetskaya+2022, dicintio+2024} both ran simulations tuned
to Fornax, showing that this galaxy's stellar component or globular
clusters are likely not affected by tides. \citet{borukhovetskaya+2022a}
also used N-body simulations to analyze Crater II, showing that the
present-day properties of the galaxy are challenging to reconcile with
\LCDM{} initial conditions with tidal evolution. Most relevantly,
\citet{iorio+2019} conducted simulations of Sculptor, concluding tides
likely do not affect this galaxy. Our goal is to apply a similar
framework to Sculptor and Ursa Minor.

\subsection{Tidal and ``break'' radii}\label{sec:break_radii}

For a given orbit in a given potential, there are characteristic radii
which help gauge the effects of tides on a dwarf galaxy model.

The \textbf{Jacobi radius} represents the approximate radius where stars
become unbound for a galaxy in a circular orbit around a host galaxy.
Calculated from an approximation of the location of the \(L_1\) and
\(L_2\) Lagrange points, the Jacobi radius is where the mean density of
the dwarf galaxy is roughly three times the mean interior density of the
host galaxy at pericentre, or
\begin{equation}\protect\phantomsection\label{eq:r_jacobi}{
3\bar \rho_{\rm MW}(r_{\rm peri}) \approx \bar \rho_{\rm dwarf}(r_J),
}\end{equation} \citep[ eq. 7-84]{BT1987}. If \(r_J\) occurs within the
visible extent of a galaxy, we should expect to find relatively clear
signs of tidal disturbance. While strictly valid for circular orbits,
assuming \(r_{\rm peri}\) for the host-dwarf distance works as most
stars are lost near pericentre.

We also use the \textbf{break radius} as defined in
\citet{penarrubia+2009}, marking where the galaxy is still in
disequilibrium after pericentric passage in a highly-eccentric orbit.
The break radius \(r_{\rm break}\) is proportional to the velocity
dispersion, \(\sigma_v\), and time elapsed since pericentre,
\(\Delta t\), \begin{equation}\protect\phantomsection\label{eq:r_break}{
r_{\rm break} = C\,\sigma_{v}\,\Delta t
}\end{equation} where the scaling constant \(C \approx 0.55\) was
derived empirically. \(r_{\rm break}\) describes where the dynamical
timescale is longer than the time since the perturbation, i.e.~the
radius within which the galaxy has had enough time to dynamically relax.

\subsection{A simple tidal simulation}\label{a-simple-tidal-simulation}

To illustrate the effects of tides on an NFW halo due to a larger
galaxy, we consider a toy model. We evolve an NFW subhalo with
\(\rmax=5\,\kpc\) and \(\vmax = 27\kms\) orbiting another NFW static
host halo with \(\rmax=25\,\kpc\) and \(\vmax = 207.4\,\kms\). These
choices are motivated by the inferred structure and masses of dSphs and
the Milky Way, respectively. The stars follow initially an exponential
profile with a scale radius \(R_s=0.25\,\kpc\). We then evolve the orbit
through a pericentre of 10 kpc starting from an apocentre of 100 kpc. We
show the results 170 Myr after the first pericentre. See
Section~\ref{sec:methods} for a more detailed description of our general
simulation setup.

Fig.~\ref{fig:idealized_break_radius} illustrates the properties of this
idealized simulation shortly after the first pericentre. The projected
density of stars (right) is relatively undisturbed and spherical in the
centre, but becomes non-isotropic outside the break radius and shows
nascent tidal tails.These tidally disturbed stars appear as an extended,
outer density excess relative to the initial conditions. This excess
appears just outside the break radius. The break radius also marks where
the mean radial velocity of the stars transitions from 0 to positive ---
the galaxy is out of equilibrium outside \(r_{\rm break}\).

To first order, the final density profile of this toy simulation indeed
resembles the outer excess in Sculptor and Ursa Minor. This thesis aims
to investigate if a tidal explanation is viable given the current
properties of each galaxy.

\begin{figure}
\centering
\pandocbounded{\includegraphics[keepaspectratio]{figures/idealized_break_radius.pdf}}
\caption[Break radius validation]{Example density and velocity
distributions of an idealized simulation shortly after pericentre.
\textbf{Top left}: The 2D density profile for the initial and final
simulation with the break radius marked. The break radius of the
simulations is set by the time since pericentre. \textbf{Bottom left}:
the mean radial velocity (dot product of relative position and velocity
relative to dwarf centre) as a function of 2D radius. \textbf{Right}:
The projected 2D stellar density in the \(x\)-\(y\) plane. The green
circle represents the break radius and the grey arrow points towards the
host centre.}\label{fig:idealized_break_radius}
\end{figure}

\section{Thesis outline}\label{thesis-outline}

The goal of this thesis is to review the evidence for an extended
density profile in Ursa Minor and Sculptor, to assess the impact of
tidal effects on each galaxy, and to discuss possible interpretations
for the structure of these galaxies.

In Chapter \ref{sec:observations}, we describe how we compute
observational density profiles following \citet{jensen+2024}. In Chapter
\ref{sec:methods}, we review our simulation methods. Next, we present
our results for the tidal effects on Sculptor and Ursa Minor in Chapter
\ref{sec:results}, We discuss our results, limitations, and implications
in Chapter \ref{sec:discussion}. Finally, Chapter \ref{sec:summary}
summarizes this work and discuss future directions for similar work in
the field of dwarf galaxies.



\chapter{The Light Profiles of the Classical Dwarf Spheroidals}\label{sec:observations}
As discussed in Section~\ref{sec:exponential_profiles}, the projected
luminosity/stellar mass density profile of dwarf galaxies is generally
well-described by an exponential law (Eq.~\ref{eq:exponential_law}). A
prototypical example is the Fornax dSph --- well-fit by an exponential
over 4.5 decades in surface density. On the other hand, dSphs like
Sculptor or Ursa Minor have profiles which deviate significantly from an
exponential profile fitted to the inner regions, with a clear excess of
stars/light in the outer regions. In this chapter, we critically review
the density profile of the MW classical dwarfs. This thesis concerns the
origin of this excess, with a tidal interpretation explored in
subsequent chapters.

\section{\texorpdfstring{Satellite stellar membership with
\emph{Gaia}}{Satellite stellar membership with Gaia}}\label{satellite-stellar-membership-with-gaia}

Measuring the light profile of a resolved galaxy requires careful
consideration of whether any given star belongs to the system or not.
Without removing contamination from foreground/background sources, faint
features may be lost in the noise or be of uncertain association. When
only photometric data was available, the membership of stars was
ascertained using the colour-magnitude diagram alone \citep[e.g.,
matched filter methods like those used by][]{rockosi+2002}. Now that
\emph{Gaia} data are available, stellar parallax and proper motion are
also available to improve membership assignment.

Here, we use the \citet{jensen+2024}'s (hereafter J+24) membership
probabilities from \emph{Gaia} data. J+24 used a Bayesian framework
incorporating proper motion (PM), colour-magnitude diagram (CMD), and
spatial information to determine the probability that a given star
belongs to the satellite or foreground/background. By accounting for PM
in particular, J+24 produced low contamination samples of candidate
member stars out to large distances from a dwarf galaxy. J+24 extended
the algorithm presented in \citet{MV2020a}; \citet{MV2020b} by
additionally including a secondary, extended spatial component. J+24
detected candidate members out to \textasciitilde10 half-light radii
from the centres of some galaxies (\(R_h\)). Similar recent work has
also included \citet{pace+li2019}; \citet{battaglia+2022};
\citet{pace+erkal+li2022}; \citet{qi+2022}.

J+24's method uses likelihoods, \({\cal L}\), representing the
probability density that a star is consistent with either the
foreground/background, \({\cal L}_{\rm bg}\), or the satellite galaxy,
\({\cal L}_{\rm sat}\). In either case, the likelihoods are the product
of a spatial, PM, and CMD component, \begin{equation}{
{\cal L} = {\cal L}_{\rm space}\ {\cal L}_{\rm PM}\ {\cal L}_{\rm CMD}.
}\end{equation}

For a satellite, the \emph{spatial likelihood} is specified either as a
one- or two-component elliptical exponential profile. A second component
is included only when the preferred amplitude of the second component is
non-zero. The \emph{proper-motion likelihood} quantifies the agreement
of a star's motion with the dwarf galaxy's systemic motion, accounting
for observational uncertainties. The \emph{CMD likelihood} measures the
consistency of a star's \emph{Gaia} \(G\) magnitude and
\(G_{\rm BP}- G_{\rm RP}\) colour with theoretical isochrones for the
galaxy. For the background likelihoods, the spatial likelihood is a
uniform distribution over the field, and the background CMD and PM
likelihoods are constructed empirically from the stars in an annulus far
from the satellite. Each likelihood is normalized as a probability
density over the respective parameter space.

The total likelihood is a mixture model of the satellite and background,
weighted by the fraction of stars in the field belonging to the
satellite, \(f_{\rm sat}\): \begin{equation}{
{\cal L}_{\rm tot} = f_{\rm sat}{\cal L}_{\rm sat} + (1-f_{\rm sat}){\cal L}_{\rm bg}.
}\end{equation} The probability that a given star belongs to the
satellite is then \begin{equation}{
P_{\rm sat} = 
\frac{f_{\rm sat}\,{\cal L}_{\rm sat}}{{\cal L}_{\rm tot}}
= \frac{f_{\rm sat}{\cal L}_{\rm sat}}{f_{\rm sat}{\cal L}_{\rm sat} + (1-f_{\rm sat}){\cal L}_{\rm bg}}.
}\end{equation}

J+24 fit this model using Monte Carlo Markov chain simulations, solving
for the proper motions, satellite membership fraction (\(f_{\rm sat}\)),
and structural properties of the second exponential density profile (if
included). The median parameters from the samples are then used to
calculate the \(P_{\rm sat}\) we use for sample selection.

For our fiducial sample, we adopt a minimum probability of
\(P_{\rm sat} = 0.2\). We do not filter on magnitudes explicitly, but
J+24's quality cuts typically only include stars with \(G < 21\). We use
the \(P_{\rm sat}\) values from the elliptical 2-component runs if a
galaxy shows evidence for an outer component, the 1-component run
otherwise. Most stars have \(P_{\rm  sat}\) values which are nearly 0 or
1, so the exact choice of probability threshold has little effect on the
resulting sample. Even at our relatively generous probability threshold
of 0.2, the purity remains high when validated against spectroscopic
line-of-sight (LOS) velocities (\textasciitilde90\%, J+24). We note,
however, that these purity estimates may be biased. Stars with LOS
velocities are typically brighter and, as a result, have more precise
\emph{Gaia} measurements. However, we find that our conclusions are
unchanged when limiting samples to only the brightest stars. Altogether,
the J+24 method provides a high-quality, low-contamination sample of
dwarf galaxy candidate member stars, which we will now investigate.

\section{The effects of membership
criteria}\label{the-effects-of-membership-criteria}

We analyze stellar distributions in the tangent plane, considering as
well the projected shape of a galaxy. The tangent plane coordinates
\(\xi\) and \(\eta\) are offsets in RA and declination as measured on
the plane tangent relative to the galaxy centre. To account for the
elliptical shape of the galaxy, we use \(R_{\rm ell}\), which we define
as the circularized elliptical radius, \begin{equation}{
R_{\rm ell}^2 = a\,b\,\left(\frac{{\xi'}^2}{a^2} + \frac{{\eta'}^2}{b^2} \right),
}\end{equation} where \(\xi'\) and \(\eta'\) are the tangent plane
coordinate rotated to align with a dwarf galaxy's major and minor axis,
and \(a\) and \(b\) are the semi-major and semi-minor axis of the
galaxy.

We illustrate the likelihood-based membership selection criteria through
a progressive tightening of criteria. First, we consider a
minimally-refined sample, \textbf{all}, which only excludes stars with
poor astrometry, unreliable photometry, or inconsistent parallaxes.
Next, incorporating CMD and PM information, we use the \textbf{CMD+PM}
to test a selection method agnostic to the spatial position. This sample
includes stars where the CMD and PM combined likelihood favours
satellite membership:
\begin{equation}\protect\phantomsection\label{eq:sel_cmd_pm}{
{\cal L}_{\rm CMD,\ sat}\ {\cal L}_{\rm PM,\ sat} > {\cal L}_{\rm CMD,\ bg}\ {\cal L}_{\rm PM,\ bg}.
}\end{equation} Next, our \textbf{fiducial} sample,
\(P_{\rm sat} > 0.2\), includes a spatial likelihood as described in
J+24. Finally, the \textbf{RV members} (for radial velocity) sample adds
line-of-sight velocity information (see Appendix \ref{sec:rv_obs} for
details). While the latter is the best motivated membership sample, we
do not use this sample for stellar density analysis due to its
incompleteness and complex selection function.

Figs.~\ref{fig:scl_selection}, \ref{fig:umi_selection}, \ref{fig:fornax_selection}
show our fiducial sample for three dSphs (Sculptor, Ursa Minor, and
Fornax), as well as impact of different samples, in tangent-plane
coordinates, the CMD, and proper-motion space. The ``all'' sample
extends uniformly across the tangent plane, but includes a substantial
background population. The ``CMD+PM'' sample has a much lower background
density in the tangent plane, revealing the satellite more clearly. The
fiducial sample appears similar to the ``CMD+PM'' sample in the CMD and
PM planes, but excludes improbably distant stars. The ``RV member''
sample also traces out similar distributions in CMD and PM space as the
fiducial sample, with less dispersion, likely reflecting the brighter
magnitudes of stars with spectroscopic follow-up. Each selection
criteria---spatial, CMD, and PM---contributes towards a high-quality
membership assignment.

Based on the fiducial sample's distribution in
Figs.~\ref{fig:scl_selection}, \ref{fig:umi_selection}, \ref{fig:fornax_selection},
Ursa Minor stands out because of its higher ellipticity. Sculptor and
Ursa Minor show a population of members past \(6R_h\) in radius, but
Fornax does not, even though Fornax has many more stars. Fornax also has
a bluer CMD, indicative of more recent star formation. Otherwise, there
are no major morphological differences between these galaxies.

One limitation of the J+24 method is the assumption of a specific
density profile for the dwarf galaxy (a one- or two-component
exponential), which may impact the membership probability of distant
stars. However, there are no clear extensions or overdensities in the
``CMD+PM'' selected sample, which does not include a spatial likelihood.
If even fainter, more extended features exist, they are not clearly
detectable with \emph{Gaia}.

To illustrate the extent of Sculptor and Ursa Minor, we highlight
distant, spectroscopically confirmed members with red-outlined stars in
Figs.~\ref{fig:scl_selection}, \ref{fig:umi_selection}. In particular,
\citet{sestito+2023a}; \citet{sestito+2023b} targeted distant, bright
stars from the J+24 candidate membership list. The most distant stars
confirmed in their work lie at radii \(\gtrsim 10 R_h\) from the centres
of each dwarf. For an exponential profile, 99.95\% of stars fall within
a radius of \(6R_h\), so the mere presence of very distant member stars
provides evidence for deviations from an exponential profile.

\begin{figure}
\centering
\includegraphics[width=0.7\linewidth,height=\textheight,keepaspectratio]{figures/scl_selection.png}
\caption[Sculptor sample selection]{The distributions of various samples
of \emph{Gaia} stars for Sculptor. We plot light grey points for ``all''
field stars (with consistent parallaxes and reliable photometry and
astrometry), turquoise points for ``CMD+PM'' selected stars
(Eq.~\ref{eq:sel_cmd_pm}), blue squares for the ``fiducial'' sample
(with membership probability \(P_{\rm sat} > 0.2\)), and indigo diamonds
for the ``RV members'' sample. We mark the two far-outlier stars from
\citet{sestito+2023a} with rust-outlined indigo stars. \textbf{Top:}
Tangent plane \(\xi, \eta\). The orange ellipses represents 3 and 6
half-light radii. \textbf{Bottom left:} Colour magnitude diagram in Gaia
\(G\) magnitude versus \(G_{\rm BP} - G_{\rm RP}\) colour. We plot a
Padova \(12\,\)Gyr, \({\rm [Fe/H] }=-1.68\) isochone in orange. The
black bar in the top left represents the median colour error.
\textbf{Bottom right:} Proper motion in declination \(\mu_\delta\) vs RA
\(\mu_{\alpha*}\) (corrected). The orange point marks the systemic
\citet{MV2020b} proper motion. The black cross represents the median
proper motion error.}\label{fig:scl_selection}
\end{figure}

\begin{figure}
\centering
\includegraphics[width=0.7\linewidth,height=\textheight,keepaspectratio]{figures/umi_selection.png}
\caption[Ursa Minor sample selection]{Similar to
Fig.~\ref{fig:scl_selection} except for Ursa Minor. We outline
``velocity confirmed'' members outside a radius of \(6R_h\) with red
stars \citep[from][]{sestito+2023b, pace+2020, spencer+2018}. We also
mark the location of Muñoz 1 with a pink
circle.}\label{fig:umi_selection}
\end{figure}

\begin{figure}
\centering
\includegraphics[width=0.7\linewidth,height=\textheight,keepaspectratio]{figures/fornax_selection.png}
\caption[Fornax sample selection]{Similar to
Fig.~\ref{fig:scl_selection} except for Fornax. RV measurements are from
\citet{WMO2009}. Fornax does not show the same extended outer halo of
probable members as Sculptor or Ursa Minor despite having many more
stars.}\label{fig:fornax_selection}
\end{figure}

\section{Density profiles}\label{density-profiles}

We derive density profiles by binning member stars in constant width
bins in \(\log R_{\rm ell}\) of 0.05 dex. We ignore bins interior or
exterior to the first empty bin in either direction. We use symmetric
Poisson uncertainties as ``error bars'' in the density estimate at each
bin.

Fig.~\ref{fig:scl_observed_profiles} show the derived density profiles
for Sculptor, Ursa Minor, and Fornax. We calculate density profiles for
three different samples from above: ``all'', ``CMD+PM'', and the
``fiducial'' sample. In each case, all samples coincide towards the
inner regions of the satellite. However, the density profile from
``all'' plateaus at the total background in the field at radii of 30-60
arcminutes, depending on the galaxy. By restricting stellar membership
with CMD + PM, the background is reduced by 1-2 dex. The ``CMD+PM''
background plateau represents the density of background stars which
could be mistaken as members because of their coincident colours and
proper motions. Finally, the ``all -- background'' profile results from
subtracting the apparent background in the ``all'' profile.

While the density profiles of all samples agree in the inner regions,
they begin to deviate outside \(6 R_h\). Outside this radius, the
fiducial profile extends 1--2 magnitudes below the CMD+PM selection
background. Because the fiducial profile in this region depends
critically on the spatial likelihood, its shape is vulnerable to the
assumed density profile (an exponential in J+24). We explore this in
more detail in Appendix \ref{sec:density_extra}.

Near UMi, there is a small (\(R_h\sim 0.5'\)), likely unassociated,
ultrafaint star cluster, Muñoz 1 \citep{munoz+2012}. The cluster is at a
relative position of \((\xi, \eta) \approx(-42, -15)\) arcminutes,
corresponding to an elliptical radius of 37 arcminutes. This cluster
does not have a bright RGB, so has few stars brighter than a magnitude
\(G=22\). The cluster has little effect on the elliptically-averaged
density profile (see location on Fig.~\ref{fig:scl_observed_profiles}).

Fig.~\ref{fig:classical_dwarfs_densities} compares the fiducial density
profiles of Sculptor, Ursa Minor, Fornax, and other classical dwarf
galaxies. Of the classicals, we exclude Antlia II, due to the extremely
high background, and Sagittarius, which was not included in J+24. The
density profiles are scaled to match at the half-light radius, taken
from \citet{munoz+2018}. All of the classical dwarfs appear to be well
described by an exponential profile in the inner regions. In the outer
regions, however, Sculptor and Ursa Minor deviate, and show a clear
outer excess over an exponential law (solid black line). These galaxies
are better fit by a Plummer law (dashed black line). The deviation from
an exponential grows outwards, and at \(\sim 8 R_h\), may reach 2 orders
of magnitude. The remainder of this thesis will be devoted to assessing
whether the outer excess shown by Scl and UMi are due to Galactic tides.

\begin{figure}
\centering
\pandocbounded{\includegraphics[keepaspectratio]{figures/scl_umi_fnx_density_methods.pdf}}
\caption[Sculptor density profiles]{The density profile of Sculptor,
Ursa Minor, and Fornax for different selection criteria, ploted as log
surface dencity versus log elliptical radius. The samples are: ``all''
selects any high quality and parallax-consistent star (grey
circles),``all--background'' subtracts a uniform background density from
the ``all'' profile (orange pentagons), ``CMD+PM'' select stars
according to CMD and PM only (teal triangles), and ``fiducial'' also
includes spatial information (blue squares). We mark the half-light
radius with a vertical dashed line and the background density with the
horizontal grey line. For Ursa Minor, we show the expected location of
Muñoz 1 stars as a horizontal bar (ranging from plus/minus 3 half-light
radii).}\label{fig:scl_observed_profiles}
\end{figure}

\begin{figure}
\centering
\pandocbounded{\includegraphics[keepaspectratio]{figures/classical_dwarf_profiles.pdf}}
\caption[Classical dwarf density profiles]{The density profiles of
Sculptor, Ursa Minor, and other classical dwarfs compared to Exp2D and
Plummer density profiles. Dwarf galaxies are scaled to the same
half-light radius and density at half-light radius. Sculptor and Ursa
Minor have an excess of stars in the outer regions (past
\(\log R/R_h \sim 0.3\)) compared with other classical
dwarfs.}\label{fig:classical_dwarfs_densities}
\end{figure}


\chapter{Simulation Methods}\label{sec:methods}
In the previous chapters, we have seen that the Scl and UMi classical
dSphs have outer density profiles that appear to deviate from the
exponential law that approximates well all other classical dSphs. Our
main intention is to assess whether such excesses result from the
effects of Galactic tides. To this purpose, we intend to use N-body
simulations of the evolution of CDM halos in a Galactic potential,
constrained to have the orbital parameters consistent with a dwarf's
present-day position and velocity. We shall assume that the Galactic
potential is the static, analytic potential inferred by
\citet{mcmillan2011} from observations of kinematic tracers. We also
assume the potential of each dwarf may be initially approximated by a
cuspy NFW profile. Since the dwarfs in question are heavily dark matter
dominated, we shall use a carefully selected sample of dark matter
particles to mimic and track the evolution of an embedded tracer stellar
component. In this Chapter, I describe the Galactic potential used, the
orbital estimation method, the initial conditions setup, and the N-body
method used.

\section{Orbital estimation}\label{orbital-estimation}

To explore the possible orbits of a dwarf galaxy, we perform a Monte
Carlo sampling of the present-day observables. The present-day position,
distance modulus, LOS velocity, and proper motions are each sampled from
normal distributions given the reported uncertainties in
Tables~\ref{tbl:scl_obs_props}, \ref{tbl:umi_obs_props}. We integrate
each sampled position/velocity back in time for 10 Gyr using \agama{}
\citep{agama}. Dynamical friction is not expected to impact orbits
substantially because of the low masses and large pericentres of the
dwarfs.

\subsection{Galactocentric frame}\label{galactocentric-frame}

To convert observed positions and velocities to Galactocentric
coordinates, we use the Astropy v4 Galactocentric frame
\citep{astropycollaboration+2022}. This frame assumes the Galactic
centre is at position \(\alpha = {\rm 17h\,45m\,37.224s}\),
\(\delta = -28^\circ\,56'\,10.23''\) with proper motions
\(\mu_{\alpha*}=-3.151\pm0.018\ \masyr\) ,
\(\mu_\delta=-5.547\pm0.026 \masyr\) \citep[from the appendix and Table
2 of][]{reid+brunthaler2004}. The Galactic centre is at a distance from
the Sun of \(8.122\pm0.033\,\)kpc with a radial velocity =
\(11 + 1.9 \pm 3\,\kms\) \citep{gravitycollaboration+2018}. The Sun is
assumed to be \(20.8\pm0.3\,\)pc above the disk
\citep{bennett+bovy2019}. Using the procedure outlined in
\citet{drimmel+poggio2018}, the Solar velocity relative to the Galactic
rest frame is then
\(\V_\odot = [-12.9 \pm 3.0, 245.6 \pm 1.4, 7.78 \pm 0.08]\) km/s. The
uncertainties in the reference frame are typically smaller than the
uncertainties on a dwarf galaxy's position and velocity.

\subsection{Milky Way Potential}\label{milky-way-potential}

We adopt the Milky Way potential described in \citet{EP2020}, which is
an analytic approximation to that proposed by \citet{mcmillan2011}.
Fig.~\ref{fig:v_circ_potential} plots the circular velocity profiles of
each component and the total circular velocity profile for our fiducial
profile. The potential includes a stellar bulge, a thin and thick disk,
and a dark matter NFW halo.

The Galactic bulge is described by a \citet{hernquist1990} potential,

\begin{equation}{
\Phi(r) = - \frac{GM}{r + a},
}\end{equation} where \(a=1.3\,{\rm kpc}\) is the scale radius and
\(M=2.1 \times 10^{10}\,\Mo\) is the total mass. The thin and thick
disks are represented with the \citet{miyamoto+nagai1975} cylindrical
potential, \begin{equation}{
\Phi(R, z) = \frac{-GM}{\left(R^2 + \left[a + \sqrt{z^2 + b^2}\right]^{2}\right)^{1/2}},
}\end{equation} where \(a\) is the disc radial scale length, \(b\) is
the scale height, and \(M\) is the total mass of the disk. For the thin
disk, \(a=3.944\,\)kpc, \(b=0.311\,\)kpc, and
\(M=3.944\times10^{10}\,\)M\(_\odot\). For the thick disk,
\(a=4.4\,\)kpc, \(b=0.92\,\)kpc, and \(M=2\times10^{10}\,\)M\(_\odot\).
The halo is an NFW dark matter halo (Eq.~\ref{eq:nfw}) with
\(\rmax = 43.7\,\)kpc and \(\vmax = 191\,\kms\).

\begin{figure}
\centering
\pandocbounded{\includegraphics[keepaspectratio]{figures/v_circ_potential.pdf}}
\caption[Circular velocity of potential]{Circular velocity profile of
\citet{EP2020} potential. The total circular velocity (thick black line)
is composed of an NFW halo (green dashed line), a think and thick
\citet{miyamoto+nagai1975} disk (orange dash-dotted line), and a
\citet{hernquist1990} bulge (blue dotted
line).}\label{fig:v_circ_potential}
\end{figure}

\subsection{Orbits of Sculptor}\label{orbits-of-sculptor}

Sculptor's orbital history is relatively well-constrained.
Fig.~\ref{fig:scl_orbits} illustrates point particle orbits for 100
samples of Sculptor's observed kinematics integrated backwards 5 Gyr in
both Galactocentric coordinate slices (\(x\), \(y\), \(z\)) and in
Galactocentric radius with time. All sampled orbits of Sculptor have
nearly the same morphology---the orbit primarily resides in the
\(y\)--\(z\) plane and has a similar number of periods and
pericentre/apocentre.

To maximize tidal effects, we select an orbit with the \(\sim 3\sigma\)
smallest pericentre among all possible orbits. We achieve this by taking
the median parameters of all orbits with a pericentre less than the
0.0027th quantile pericentre, yielding a pericentre of 43 kpc. Given the
current observations, it is unlikely that Sculptor has a significantly
smaller pericentre than our selected orbit.

We take the first apocentre after a look-back time of 10 Gyr, or at 9.43
Gyr, as the initial conditions for our model of Sculptor, noted in
Table~\ref{tbl:scl_orbits}.

\begin{figure}
\centering
\pandocbounded{\includegraphics[keepaspectratio]{/Users/daniel/thesis/figures/scl_xyzr_orbits.pdf}}
\caption[Sculptor Orbits]{The orbits of Sculptor in a static Milky Way
potential in Galactocentric \(x\), \(y\), and \(z\) coordinates (top)
and in Galactocentric radius \(r\) versus time (bottom). The Milky Way
is at the centre with the disk lying in the \(x\)--\(y\) plane. Our
selected \texttt{smallperi} orbit is plotted in black and light blue
transparent orbits represent the past 5Gyr orbits sampled over Sculptor
observables in Table~\ref{tbl:scl_obs_props}. The orbit of sculptor is
well-constrained in this potential and it is unlikely to achieve a
smaller pericentre than our selected orbit.}\label{fig:scl_orbits}
\end{figure}

\begin{figure}
\centering
\pandocbounded{\includegraphics[keepaspectratio]{/Users/daniel/thesis/figures/umi_xyzr_orbits.pdf}}
\caption[Ursa Minor Orbits]{Similar to Fig.~\ref{fig:scl_orbits}, the
orbits of Ursa Minor in a static Milky Way potential in Galactocentric
\(x\), \(y\), and \(z\) coordinates. In the lower panel, we show the
radius versus time for only three orbits of Ursa
Minor.}\label{fig:umi_orbits}
\end{figure}

\begin{table*}[t]
\centering
\caption[Sculptor Selected Orbits]{Properties of selected orbits for Sculptor. The mean orbit represents the observational mean from Table \ref{tbl:scl_obs_props}. The Smallperi represents instead the $3\sigma$ smallest pericentre, which we use to provide an upper limit on tidal effects. }
\label{tbl:scl_orbits}
\begin{tabular}{lll}
\toprule
Property & Mean & SmallPeri\\
\midrule
distance / kpc & 83.2 & 82.6\\
$\pmra / \masyr$ & 0.099 & 0.134\\
$\pmdec / \masyr$ & -0.160 & -0.198\\
LOS velocity / $\kms$ & 111.2 & 111.2\\
$t_i / \Gyr$ & -8.74 & -9.43\\
$\vec{x}_{i} / \kpc$ & [16.13, 92.47, 39.63] & [-2.49, -42.78, 86.10]\\
$\vec{v}_i / \kms$ & [-2.37, -54.70, 128.96] & [-20.56, -114.83, -57.29]\\
pericentre / kpc & 53 & 43\\
apocentre / kpc & 102 & 96\\
$t_{\rm last\ peri} / {\rm Gyr}$ & -0.45 & -0.46\\
Number of pericentres & 6 & 6\\
\bottomrule
\end{tabular}
\end{table*}

\subsection{Orbits of Ursa Minor}\label{orbits-of-ursa-minor}

Similar to Sculptor, Ursa Minor has a well-constrained orbit.
Fig.~\ref{fig:umi_orbits} shows 100 random point-orbits of Ursa Minor.
Initially, we select an orbit with approximately the \(3\sigma\)
smallest pericentre as in Sculptor.

Ursa Minor's N-body orbit diverges from the point particle orbit. To
ensure the final conditions of the N-body simulation are close to the
intended final position, we iteratively adjust Ursa Minor's initial
conditions. Initially, starting with low-resolution runs, we adjust the
cylindrical actions of the initial orbit by the final difference in
actions at the end of orbital evolution. After the initial actions have
converged (2 iterations), we change the initial action angles by the
final difference in action angles. This method converges within 4
iterations to an orbit agreeing with the observed kinematics of Ursa
Minor.

\begin{table*}[t]
\centering
\caption[Ursa Minor Selected Orbits]{Properties of selected orbits for Ursa Minor. The "smallperi" orbit is the initial point orbit and the "smallperi.5" is the initial orbit for the N-body simulation. }
\label{tbl:umi_orbits}
\begin{tabular}{llll}
\toprule
Property & Mean & SmallPeri.1 & SmallPeri.5\\
\midrule
distance / kpc & $70.1$ & 64.6 & \\
$\pmra / \masyr$ & $-0.124 \pm 0.17$ & -0.158 & \\
$\pmdec / \masyr$ & $0.078\pm0.17$ & 0.05 & \\
$v_{\rm LOS} / \kms$ & $-245.9\pm 1$ & -245.75 & \\
$t_i / \Gyr$ & -8.74 & -9.53 & -9.53\\
$\vec{x}_{i} / \kpc$ & [4.88, -65.11, 50.78] & [-16.48 69.92 21.05] & [17.40, 74.51, 21.34]\\
$\vec{v}_i / \kms$ & [-34.28, 77.11, 101.38] & [16.32, 39.86, -116.99] & [14.27, 48.62, -114.08]\\
pericentre / kpc & 37 & 29 & 28\\
apocentre / kpc & 83 & 75 & 72\\
$t_{\rm last\ peri} / \Gyr$ & $-0.96$ & $-0.80$ & $-0.81$\\
Number of pericentres & 9 & 6 & 6\\
\bottomrule
\end{tabular}
\end{table*}

\section{Initial conditions}\label{initial-conditions}

We use \agama{} \citep{agama} to generate the initial N-body dark matter
halo. We assume galaxies are described by an NFW dark matter potential
(Eq.~\ref{eq:nfw}). We also assume the stars do not contribute to the
potential. The dark matter density is truncated in the outer regions by
\begin{equation}{
\rho_{\rm tNFW} = e^{-(r/r_t)^3}\ \rho_{\rm NFW}(t),
}\end{equation} where we adopt \(r_t = 20 r_s\).

\subsection{Initial dark matter halos for Sculptor and Ursa
Minor}\label{initial-dark-matter-halos-for-sculptor-and-ursa-minor}

From the observed properties of Sculptor and Ursa Minor, we infer
reasonable \LCDM{} initial halo conditions.

Table~\ref{tbl:derived_props} reports our inferred halo and kinematic
properties of Sculptor and Ursa Minor. First, taking the absolute
magnitudes from \citet{munoz+2018} with the mass-to-light ratio from
\citet{woo+courteau+dekel2008} (with \(\sim\) 0.17 dex uncertainty), the
total current stellar mass of Sculptor and Ursa Minor are
\(M_\star \sim 3.1 \times 10^6 \Mo\) and
\(M_\star \sim 7 \times 10^5 \Mo\). Based on the stellar mass-\(\vmax\)
relation \citep[from][]{fattahi+2018}, see also \ref{fig:smhm}, Sculptor
and Ursa Minor's halos should have \(\vmax \approx 31 \,\kms\) and
\(\vmax \approx 27\,\kms\). Finally, using the \citet{ludlow+2016}
\(z=0\) mass-concentration relation, this constraint translates into a
\(\rmax \approx 6 {\rm kpc}\) and \(\rmax \approx 5\,\kpc\) for each
galaxy.

Fig.~\ref{fig:scl_halos} illustrates these estimates visually for both
galaxies. The stellar-mass \(\vmax\) constraint translates to an
estimate of \(\vmax\) only (horizontal band). The mass-concentration
relation describes the relationship between \(\vmax\) and \(\rmax\)
(diagonal linear band). And finally, we include a curved line which
illustrates the halo \(\vmax\) which has a specified initial LOS
velocity dispersion given \(\rmax\)
\(, \vcirc(R_h) / \sqrt{3} \approx \sigma_v\).
\(R_h \approx 0.24\,\kpc\) for both galaxies.

While there is some range in the choice of initial halo, reasonable
changes to the initial halo do not substantially affect the tidal
evolution for either galaxy. The observed velocity dispersion is
directly related to the mass within a half-light radius
\citep[e.g.,][]{wolf+2010}. As a result, halos with the same velocity
dispersion may differ in total mass but should have similar mass within
\(R_h\). So, the tidal effects on stars should be similar for halos with
similar velocity dispersions.

Our selected halos (in Table~\ref{tbl:initial_halos}) for each
simulation run are based on the cosmological constraints and the need to
match the present-day velocity dispersion at the end of the simulation.

\begin{table*}[t]
\centering
\caption[Derived Properties of Sculptor and Ursa Minor]{Inferred properties of the stellar component and halo for Sculptor and Ursa Minor. We record the total luminosity, stellar mass, mass-to-light ratio, dark matter halo $\vmax$ and $\rmax$, and dark matter halo virial mass $M_{200}$ and concentration $c_{\rm NFW}$. }
\label{tbl:derived_props}
\begin{tabular}{lll}
\toprule
parameter & Sculptor & Ursa Minor\\
\midrule
$L_\star$ & $1.8\pm0.2\times10^6\ L_\odot$ & $3.5 \pm 0.1 \times 10^5\,L_\odot$\\
$M_\star$ & $3.1_{-1.0}^{+1.6} \times10^6\ {\rm M}_\odot$ & $7_{-2}^{+3} \times 10^5\,\Mo$\\
$M_\star / L_\star$ & $1.7\times 10^{\pm 0.17}$ & $1.9 \times 10^{\pm 0.17}$\\
$\vmax$ & $31\pm 3\,\kms$ & $27_{-6}^{+7}\,\kms$\\
$\rmax$ & $6 \pm 2$ kpc & $5_{-2}^{+1}$ kpc\\
$M_{200}$ & $0.5 \pm 0.2\times10^{10}\ M_0$ & $3_{-2}^{+4} \times 10^9\,\Mo$\\
$c_{\rm NFW}$ & $13_{-3}^{+4}$ & 14?\\
\bottomrule
\end{tabular}
\end{table*}

\begin{figure}
\centering
\pandocbounded{\includegraphics[keepaspectratio]{figures/initial_halo_selection.pdf}}
\caption[Sculptor initial halos]{Selection of initial halos for Sculptor
and Ursa Minor. The grey line and pink line with shaded regions
represent the \citet{ludlow+2016} mass-concentration relation and
\citet{fattahi+2018} SMHM relation respectively. The curved lines
represent the velocity dispersion of the initial halo given the
present-day half-light radius \citep[via the][ mass
estimator]{wolf+2010}.}\label{fig:scl_halos}
\end{figure}

\begin{table*}[t]
\centering
\caption[Initial halos]{The initial conditions for our initial dark matter halos. }
\label{tbl:initial_halos}
\begin{tabular}{lllll}
\toprule
Halo name & $\rmax$ & $\vmax$ & $M_{200}$ & $c_{\rm NFW}$\\
\midrule
Scl: fiducial & 3.2 & 31 & 0.33 & 21\\
Scl: small & 2.5 & 25 &  & \\
UMi: fiducial & 4 & 38 & 0.62 & 21\\
\bottomrule
\end{tabular}
\end{table*}

\section{Numerical methods}\label{numerical-methods}

To simulate the tidal evolution of galaxies, we use ``N-body''
simulations integrated with the parallel, gravitational-tree program
\gadget{} \citep{gadget4}. The N-body method calculates and evolves the
gravitational accelerations between a large number of collisionless
particles to approximate the dynamical evolution of matter. To
approximate a collisionless system (i.e., without strong
particle-particle gravitational deflections), the force is softened
below a ``softening length.'' Regions which are smaller than the
softening length, or have few particles, are not well-resolved. Tree
codes organize particles into a spatial tree, enabling grouping of the
gravitational forces from nearby particles. A gravitational tree code
substantially reduces the required number of force computations, as
compared to the exact, direct-summation method.

\subsection{Isolation runs and simulation
parameters}\label{isolation-runs-and-simulation-parameters}

To ensure that the initial conditions of the simulation are dynamically
relaxed and well-converged, we run a halo first in isolation using
\gadget{}. Since gravity is scale-free, we use the same isolation run
for all halos. We adopt \(\rmax = 6.0\,\)kpc and \(\vmax = 31\,\kms\)
for the isolation halo based on Sculptor's mean properties. We run this
model for 5 Gyr (about three times the free fall timescale
\(t_{\rm ff} = \vcirc / r \approx 1.5\,\Gyr\) at \(r_{200}=36\,\)kpc).

For our simulation parameters, we adopt a softening length of
\begin{equation}\protect\phantomsection\label{eq:softening_length}{
h_{\rm grav} = 0.014\,{\rm kpc}\left(\frac{r_{\rm max}}{6.0\,{\rm kpc}}\right)\left(\frac{N}{10^7}\right)^{-1/2}.
}\end{equation} See Appendix Section~\ref{sec:extra_convergence} for a
discussion of this choice, which is similar to the \citet{power+2003}
suggested softening. We use the relative tree opening criterion with the
accuracy parameter set to 0.005, and adaptive time stepping with
integration accuracy set to 0.01.

\subsection{Numerical fidelity}\label{numerical-fidelity}

Fig.~\ref{fig:numerical_convergence} illustrates how well our numerical
setup is able to reproduce the desired initial conditions, before and
after running the model in isolation. This figure shows that our
numerical methods are able to approximate well an NFW halo down to an
innermost radius that strongly depends on resolution. The larger the
number of particles, the smaller the radius that is effectively
``resolved'' in a given simulation. For the Sculptor halo shown in this
figure (with \(\rmax = 6.0\,\)kpc and \(\vmax = 31\,\kms\) ), a
simulation with \(10^7\) particles is needed to resolve the innermost
100 pc. For reference, the half-light radius of Sculptor is roughly 100
pc, which means that at least 10 million particles would be needed to
follow faithfully its tidal evolution. Vertical arrows in
Fig.~\ref{fig:numerical_convergence} indicate the ``convergence radius''
defined by \citet[eq.\textasciitilde13]{power+2003} for NFW halos formed
in cosmological N-body simulations. This radius marks the region where
collisional effects driven by the finite number of particles used to
describe the innermost regions of a halo become important. The softening
length (from Eq.~\ref{eq:softening_length}) is typically a few times
smaller than the converged length.

\begin{figure}
\centering
\pandocbounded{\includegraphics[keepaspectratio]{figures/iso_converg_num.pdf}}
\caption[Numerical halo convergence]{Numerical convergence test for
circular velocity as a function of log radius for simulations with
different total numbers of particles in isolation. Residuals in bottom
panel are relative to NFW. The initial conditions are dotted, the
converged radius is marked by arrows \citep[from][ eq. 13]{power+2003},
and the softening length is marked by a vertical bar. Note that a slight
reduction in density starting around \(r = 30\,\kpc\) is expected given
our truncation choice.}\label{fig:numerical_convergence}
\end{figure}

\subsection{Orbital evolution}\label{orbital-evolution}

Next, we evolve the halo in the Galactic potential. We scale the relaxed
snapshot and softening length to match the initial halo in
\ref{tbl:initial_halos}, and shift the snapshot to the initial
conditions from the orbital analysis (see
Tables~\ref{tbl:scl_orbits}, \ref{tbl:umi_orbits}). We then evolve the
full N-body NFW model forward in time in the Galactic potential and
follow it in time until the present time, when the halo is closest to
the observed position of the galaxy.

\subsection{Tidal mass losses}\label{sec:shrinking_spheres}

To accurately follow the evolution of a halo, it is important to
determine the centre of the halo at each time chosen for analysis. We
use a shrinking-spheres centre method inspired by \citet{power+2003}.
First, we start with an initial centre estimate from the last timestep.
Then, we calculate the radius of all particles from the centre, remove
particles with a radius beyond the 0.975 quantile of the centre, and
recalculating the centre of mass. The procedure is repeated until the
selection radius is less than \textasciitilde1kpc or fewer than 0.1\% of
particles remain. After a centre has been chosen, we remove all unbound
particles based on the \gadget{} calculated potential of the halo. For
all future timesteps, we consider only particles retained from the
previous iteration.

The statistical centring uncertainty for the full resolution (\(10^7\)
particle) isolation run is of order 0.003 kpc, but fluctuations are
observed of \(\sim 0.03\,\kpc\). This is about three times the softening
length but is less than the numerically converged radius scale.

\subsection{The stellar component}\label{sec:painting_stars}

We ``paint'' stars onto dark matter particles using the particle-tagging
method \citep[e.g.][]{bullock+johnston2005}, assuming spherical
symmetry. We initially assume exponential stars with
\(R_s = 0.10\,,\kpc\) for both galaxies.

Let \(\Psi\) be the potential (normalized to vanish at infinity) and
\({\cal E}\) the binding energy \({\cal E} = \Psi - 1/2 v^2\). If we
know \(f({\cal E})\), the distribution function (phase-space density in
energy), then we assign a stellar weight to a given particle with energy
\({\cal E}\) using \begin{equation}{
P_\star({\cal E}) = \frac{f_\star({\cal E})}{f_{\rm halo}({\cal E})}.
}\end{equation} While \(f({\cal E})\) is a phase-space density, the
differential energy distribution includes an additional \(g({\cal E})\)
occupation term (BT87). We use Eddington inversion to find the
distribution function, (eq. 4-140b in BT87) \begin{equation}{
f({\cal E}) = \frac{1}{\sqrt{8}\, \pi^2}\left( \int_0^{\cal E} \frac{d^2\rho}{d\Psi^2} \frac{1}{\sqrt{{\cal E} - \Psi}}\ d\Psi + \frac{1}{\sqrt{\cal E}} \left(\frac{d\rho}{d\Psi}\right)_{\Psi=0} \right).
}\end{equation} In practice the right, boundary term is zero as
\(\Psi \to 0\) as \(r\to\infty\), and if \(\rho \propto r^{-n}\) at
large \(r\) and \(\Psi \sim r^{-1}\) then
\(d\rho / d\Psi \sim r^{-n+1}\) which goes to zero provided that
\(n > 1\). We take \(\Psi\) from the underlying assumed analytic dark
matter potential. \(\rho_\star\) can be calculated from the surface
density, \(\Sigma_\star\), via the inverse Abel transform.

We find the stellar profiles created in this manner are stable in the
isolated systems and show excellent agreement with the assumed stellar
density profile.


\chapter{Galactic tidal effects on Sculptor and Ursa Minor}\label{sec:results}
As discussed in Chapters \ref{sec:introduction} and
\ref{sec:observations}, this thesis aims to test whether Galactic tides
are responsible for the extended density profiles of Scl and UMi. In
this Chapter, we analyze tailored N-body simulations, using the methods
described in Chapter \ref{sec:methods}, to assess the tidal impact of
the Galactic potential. To anticipate our main conclusion, we find that
tides drive dark matter loss in both systems but leave their compact
stellar components largely unaffected. The Large Magellanic Cloud (LMC)
has the potential of substantially perturbing Scl's and even UMi's
orbit, yet the resulting tidal effects are still too weak to account for
the extended outer profiles. Our simulations thus demonstrate that
recent tides are unlikely to have altered the stellar structure of Scl
or UMi.

In this Chapter, we consider Scl first, describing tidal effects from
the MW on its dark matter and stellar components. Next, we consider how
accounting for the LMC may affect our conclusions. We then similarly
analyze UMi, considering in turn the dark matter evolution, stellar
evolution, and orbital effects of the LMC.

\section{Tidal effects on Sculptor}\label{tidal-effects-on-sculptor}

\subsection{Evolution of Sculptor's dark matter
halo}\label{evolution-of-sculptors-dark-matter-halo}

As a representation of an extreme tidal history, we initially
investigate the \smallperi{} orbit, described in
Section~\ref{sec:scl_smallperi}, chosen to maximize possible effects of
Galactic tides.

Scl experiences moderate tidal mass loss after 10 Gyrs of evolution.
Fig.~\ref{fig:scl_sim_images} shows the stripping of dark matter and the
formation of diffuse streams trailing and leading Sculptor's orbit.
Because tidal stripping can be described as a gradual removal of the
least bound particles, most mass loss occurs in the outer halo. Instead,
the inner regions of the galaxy may be relatively unaffected.

N-body models may deviate from a point-particle trajectory due to
dynamical ``self-friction'' \citep[e.g.,][]{white1983, miller+2020}.
However, this effect is slight for Scl, which ends near the observed
position, without adjusting the initial conditions (the green point in
Fig.~\ref{fig:scl_sim_images}).

The inner density cusp is tidally resilient.
Fig.~\ref{fig:scl_tidal_track} shows the initial and final circular
velocity profiles, and the evolution of the maximum circular velocity.
The maximum velocity drops from \(31\,\kms\) to \(22\,\kms\), evolving
along the tidal track from \citet{EN2021}. The final circular velocity
profile resembles the initial with an inner cusp, but has a sharper
outer truncation. Quantitatively, the halo loses \(>90\%\) of its
initial mass (see Table~\ref{tbl:scl_sim_results}). However, the inner
structure is not affected, as the Jacobi radius is over 3kpc, outside of
the initial and final \(\rmax\) (see Table~\ref{tbl:scl_sim_results} and
Fig.~\ref{fig:scl_tidal_track}). Thus, tides may remove significant
amounts of mass, but mostly from the outer halo.

\begin{figure}
\centering
\includegraphics[width=1\linewidth,height=\textheight,keepaspectratio]{figures/scl_sim_images.png}
\caption[Sculptor simulation snapshots]{Images of the dark matter
evolution over a selection of past apocentres and the present day.
Limits range from -150 to 150 kpc in the \(y\)--\(z\) (\(\sim\)orbital)
plane, and the colourscale is logarithmic, spanning 5 orders of
magnitude between the maximum and minimum values. The green dot
represents the final expected position of the galaxy, and the solid and
dotted grey curves represent the orbit over one previous or future
radial oscillation, respectively.}\label{fig:scl_sim_images}
\end{figure}

\begin{figure}
\centering
\includegraphics[width=4.5in,height=\textheight,keepaspectratio]{figures/scl_tidal_track.png}
\caption[Sculptor tidal tracks]{Dynamical evolution for the \smallperi{}
model of Sculptor. Dotted and solid lines show the initial and final
circular velocity profiles, and blue and orange lines show the dark
matter and stellar (2D exponential) profiles. The points represent the
evolution of the maximum circular velocity, and the dotted black line
shows the tidal track from \citet{EN2021}. To calculate the velocity
profiles, unbound particles are iteratively removed, recalculating the
potential at each step assuming spherical
symmetry.}\label{fig:scl_tidal_track}
\end{figure}

\begin{table*}[t]
\centering
\caption[Simulation results for Sculptor’s dark matter]{The orbital and dark matter properties for the simulation of Sculptor. The random samples column shows the distributions from point orbits, and the \smallperi{} column contains the results from the N-body simulation. }
\label{tbl:scl_sim_results}
\begin{tabular}{lll}
\toprule
Property & random samples & \smallperi{}\\
\midrule
pericentre & $53\pm3$ & 42\\
apocentre & $102\pm3$ & 94.4\\
time of last pericentre / Gyr & $-0.45 \pm 0.2$ & -0.47\\
number of pericentres & 5–6 & 6\\
Jacobi radius / kpc & $4.5 \pm 0.3$ & 3.5\\
Jacobi radius / arcmin & $186\pm12$ & 148\\
final heliocentric distance / kpc & $83.2\pm2$ & 81.6\\
$\V_\textrm{max, f} / \V_\textrm{max, i}$ &  & 0.695\\
$r_\textrm{max, f} / r_\textrm{max, i}$ &  & 0.406\\
fractional final bound mass &  & 0.0893\\
\bottomrule
\end{tabular}
\end{table*}

\subsection{Evolution of Sculptor's stars}\label{sec:scl_sim_stars}

Tides minimally affect the stellar component of Sculptor in the
\smallperi{} orbit. In Fig.~\ref{fig:scl_smallperi_i_f}, the projected
stellar distribution displays no prominent distortions, and the radial
density profile is nearly unchanged. Only at a surface density
\(\sim10^8\) times fainter than the centre do some faint tidal features
emerge. The total stellar mass lost corresponds to \(\sim 10\) stars in
total (see Table~\ref{tbl:scl_sim_results})---a formidable challenge to
detect even with the best of observations.

This result implies that Scl's extended profile cannot be reproduced by
Galactic tides operating on an initially exponential profile. The weak
effect of tides suggests that the outer profile of Sculptor is innate,
and not the result of tidal evolution. We check this assertion by
choosing a different initial stellar profile which matches the observed
profile and assessing how it evolves on the \smallperi{} orbit. We show
in Fig.~\ref{fig:scl_smallperi_plummer_i_f} that a Plummer profile
(instead of an exponential) provides an adequate fit to Scl's observed
profile. The Plummer model loses more stellar mass and forms more
luminous tidal tails. Observations reaching surface densities \(\sim10\)
times fainter than our data could reveal a stream in this case.
Nevertheless, over the radial extent probed by our data, the stellar
profile remains nearly unchanged by tidal evolution.

The Jacobi and break radii further support the conclusion that tidal
effects should not be apparent in the observed stellar component. As
calculated for this model (see
Tables~\ref{tbl:scl_sim_results}, \ref{tbl:scl_sim_stars_results}), the
break and Jacobi radii both fall outside of \(\sim 100\) arcminutes for
either stellar component. Indeed, the stellar component only begins to
deviate from an exponential profile around this break radius
(Figs.~\ref{fig:scl_smallperi_i_f}, \ref{fig:scl_smallperi_plummer_i_f}).
Since no orbits of Scl produce significantly smaller break or Jacobi
radii, it is unlikely tha any orbit would produce an observable density
excess.

Table~\ref{tbl:scl_sim_stars_results} quantifies the evolution of
stellar properties. The stellar velocity dispersion decreases by only
\(\sim1\,\kms\) and the half-light radius expands by \(\sim 10\%\). This
is consistent with adiabatic expansion due to the reduction of the total
mass \citep[e.g.,][]{stucker+2023}. In addition, the break and Jacobi
radii are \(\gtrsim 100\) arcminutes on the sky---tidal signatures would
be beyond the reach of our data. Altogether, Galactic tides negligibly
impact Scl's stellar component.

\begin{figure}
\centering
\pandocbounded{\includegraphics[keepaspectratio]{figures/scl_smallperi_i_f.pdf}}
\caption[Sculptor initial and final density profiles]{The tidal effects
on Scl's stellar component, for the \smallperi{} orbit with the fiducial
halo and exponential stars with \(R_s=0.10\,\kpc\). \textbf{Top:} the
initial (left) and final (right) 2D projected density of stars on the
sky. The solid circle marks \(6R_h\), the dotted circle the break
radius, and the blue arrow the orbital direction. \textbf{Bottom:} The
initial (dotted) and final (solid) stellar density profiles as compared
to the observed stellar density profile. Arrows mark the half-light
(\(R_h\)), break, and Jacobi radii
(Eqs.~\ref{eq:r_break}, \ref{eq:r_jacobi})
.}\label{fig:scl_smallperi_i_f}
\end{figure}

\begin{figure}
\centering
\pandocbounded{\includegraphics[keepaspectratio]{figures/scl_plummer_i_f.pdf}}
\caption[Sculptor Plummer initial and final density profiles]{Similar to
Fig.~\ref{fig:scl_smallperi_i_f} except for Plummer initial stars with
\(R_h = 0.20\,\kpc\). While a faint stream may be visible with deeper
observations, effects on the stellar profile are
minimal.}\label{fig:scl_smallperi_plummer_i_f}
\end{figure}

\begin{table*}[t]
\centering
\caption[Simulation results for Sculptor’s stars]{The present-day stellar properties for the simulations of Sculptor. In each row, we have the initial stellar velocity dispersion (within 1kpc), the final velocity dispersion, the fraction of stellar mass unbound, the initial half-light radius, the final half-light radius, and the break radius in arcmin and kpc (Eq. \ref{eq:r_break}). }
\label{tbl:scl_sim_stars_results}
\begin{tabular}{lll}
\toprule
Property & Exponential & Plummer\\
\midrule
$\sigma_{\V, i}\,/\,\kms$ & 9.8 & 10.7\\
$\sigma_{\V, f} \,/\,\kms$ & 8.8 & 9.4\\
fractional stellar mass loss & $2.1\times 10^{-6}$ & $0.024$\\
$R_{h, i}\,/\,\kpc$ & 0.169 & 0.202\\
$R_{h, f}\,/\,\kpc$ & 0.189 & 0.227\\
break radius / arcmin & $98$ & $105$\\
break radius / kpc & 2.3 & 2.5\\
\bottomrule
\end{tabular}
\end{table*}

\subsection{Orbital effects of the LMC}\label{sec:scl_lmc}

The Milky Way isn't the only galaxy in town. Recently, work has shown
that the infall of the LMC may substantially affect the Milky Way system
\citep[e.g.,][]{erkal+2019, cautun+2019, garavito-camargo+2021, vasiliev2023}.
With a mass up to one fifth of the MW \citep[e.g.,][]{penarrubia+2015},
the LMC infall affects the MW properties and the orbits of satellites
\citep[see
e.g.,][]{patel+2020, battaglia+2022, correamagnus+vasiliev2022}. In this
section, we examine how the LMC may affect the orbital history of
Sculptor.

We use the \texttt{L3M11} model of the MW and LMC potential from
\citet{vasiliev2024}. The \texttt{L3M11} potential is an evolving
multipole approximation of an N-body simulation including a live MW and
LMC dark matter halo. The potential includes a static MW bulge and disk,
evolving MW and LMC halos, and the MW reflex motion. In their
simulation, the MW was initially a NFW halo with \(r_s=16.5\,\)kpc and
\(M_{\rm 200}= 98.4\times10^{10}\Mo\), and the LMC a NFW halo with
\(r_s=11.7\) and \(M_{200} = 24.6 \times 10^{10} \Mo\). The total
\texttt{L3M11} MW mass is lighter than our initial \citet{EP2020}
potential.

The inclusion of the LMC reshapes Scl's orbital history, as shown in
Fig.~\ref{fig:scl_lmc_orbits_effect}. In the MW-only potential, Scl's
orbit is typical of a long-term MW satellite. However, Scl's closest
approach to the LMC \(~\sim0.1\,\Gyr\) ago affects the long-term
orbit---Scl is inferred to have reach an apocentre of nearly
\(300\,\kpc\). According to this, Scl may be on first or second infall,
depending on the MW and LMC mass.\footnote{The orbital periods may be
  \(\gtrsim 7\,\Gyr\) for the lighter MW model, compared to
  \(\sim 2\,\Gyr\) for the MW-only case} Scl's is orbiting the Milky Way
on a similar plane to the LMC, but in the opposite direction---thus, Scl
is unlikely to be an LMC satellite.

Interestingly, the timing of the LMC encounter implies a break radius
(\(\sim 25'\), from Table~\ref{tbl:scl_lmc_sim_stars}) consistent with
the beginning of Scl's observed density excess (see
Fig.~\ref{fig:classical_dwarfs_densities}, and
Section~\ref{sec:data_density_profiles}). To probe this further, we
select an orbit with the smallest LMC-Scl pericentre (\(20\,\kpc\)) in
the \texttt{L3M11} model, consistent with Scl's present-day position and
velocity. The orbit is selected following the procedure in
Section~\ref{sec:orbital_estimation} (with uncertainties doubled). This
\texttt{LMC-flyby} orbit is integrated back in time \(2\,\Gyr\) ago to
isolate recent tidal effects. We modify Scl's initial halo to have
\(\rmax = 2.5\,\kpc\) and \(\vmax = 25\,\kms\), slightly reducing the
initial stellar velocity dispersion.
Fig.~\ref{fig:scl_lmc_orbits_effect} shows this selected orbit in black
and Table~\ref{tbl:orbit_ics} records the initial conditions.

\begin{table*}[t]
\centering
\caption[Orbits and results for Sculptor in the MW+LMC potential.]{The orbital properties and dark matter evolution for the models including an LMC. Similar to Table \ref{tbl:scl_sim_results} except quantities with respect to the LMC are in parentheses. }
\label{tbl:scl_lmc_sims}
\begin{tabular}{lll}
\toprule
Property & random samples & \texttt{LMC-flyby}\\
\midrule
pericentre & $44\pm 3$ ($29 \pm 2$) & 39 (20)\\
apocentre & $218 \pm 8$ & –\\
time of last pericentre / Gyr & $-0.38\pm0.01$ (-0.11) & -0.33 (-0.10)\\
number of pericentres & 1 (1) & 1 (1)\\
Jacobi radius / kpc & $3.3\pm0.2$ ($3.6\pm0.2$) & 2.8 (2.6)\\
Jacobi radius / arcmin & $136 \pm 9$ ($159\pm5$) & 132 (121)\\
final heliocentric distance / kpc & $83.2\pm2$ & 72.9\\
$\V_\textrm{max, f} / \V_\textrm{max, i}$ &  & 0.928\\
$r_\textrm{max, f} / r_\textrm{max, i}$ &  & 0.763\\
fractional final bound mass &  & 0.5402\\
\bottomrule
\end{tabular}
\end{table*}

\begin{figure}
\centering
\includegraphics[width=1\linewidth,height=\textheight,keepaspectratio]{figures/scl_lmc_xyzr_orbits.png}
\caption[Sculptor orbits with LMC]{Similar to Fig.~\ref{fig:scl_orbits}
except for orbits with (orange) and without (green lines) the inclusion
of an LMC (blue line) in the potential. The bottom row additionally
shows the distance between Scl and the LMC over
time.}\label{fig:scl_lmc_orbits_effect}
\end{figure}

\subsection{Tidal effects from the
LMC}\label{tidal-effects-from-the-lmc}

Perhaps surprisingly, the combined tidal effect of the MW and LMC is
weaker for Scl than in the MW-only case.
Fig.~\ref{fig:scl_lmc_sim_images} shows the dark matter evolution of Scl
and the passage of the LMC. With only one MW pericentre, Scl's dark
matter is less disrupted than in the previous MW-only model. The
subsequent LMC passage modifies Scl's orbit but has otherwise little
effect. The dark matter structure evolves mildly and \(\sim 50\%\) of
mass remains bound (Table~\ref{tbl:scl_lmc_sims}).

Correspondingly, the stellar component is nearly unchanged by the
combined MW and LMC tides. Fig.~\ref{fig:scl_lmc_i_f} shows the
projected stellar distributions and density profiles of this model.
While the break radius is within the observed density profile, tidal
effects are too weak to be detectable. Structural properties of the
stars similarly evolve little (Table~\ref{tbl:scl_lmc_sim_stars}).

The reduced tides of the LMC-including model are likely a result of the
altered orbit of Sculptor. Compared to the MW-only \smallperi{} orbit,
the \texttt{LMC-flyby} model completes fewer orbits and therefore
experiences a reduced net tidal effect. And while the instantaneous
tidal force from the LMC is larger than the MW, Scl does not experience
the LMC tidal field long enough to display disturbances. Furthermore,
the Jacobi radius due to the LMC still falls outside the observed
density profile (Fig.~\ref{fig:scl_lmc_i_f}), and the MW Jacobi radius
is even larger. As a result, tides in an MW and LMC potential are even
weaker overall than for the MW-only orbit.

\begin{figure}
\centering
\includegraphics[width=1\linewidth,height=\textheight,keepaspectratio]{figures/scl_lmc_sim_images.png}
\caption[Sculptor simulation snapshots with LMC]{Similar to
Fig.~\ref{fig:scl_sim_images} for the case where the potential includes
an LMC. The current position and path of the LMC are represented by the
green dot and line, respectively. We also plot the full orbit (over the
past 2Gyr) for both Scl and the LMC, as less than one radial period
happens over this time frame.}\label{fig:scl_lmc_sim_images}
\end{figure}

\begin{figure}
\centering
\pandocbounded{\includegraphics[keepaspectratio]{figures/scl_lmc_i_f.pdf}}
\caption[Sculptor initial and final density with LMC]{Similar to
Fig.~\ref{fig:scl_smallperi_i_f} for the \texttt{LMC-flyby} model. The
Jacobi and break radii here are calculated with respect to the LMC; the
corresponding radii with respect to the MW are larger. With only one MW
pericentre and a recent, rapid LMC encounter, tidal forces do not appear
to affect the stellar distribution.}\label{fig:scl_lmc_i_f}
\end{figure}

\begin{table*}[t]
\centering
\caption[Simulation results for Sculptor’s stars in the MW+LMC potential]{Similar to Table \ref{tbl:scl_sim_stars_results}, but for the properties of the stellar components of the \texttt{LMC-flyby} model of Sculptor. }
\label{tbl:scl_lmc_sim_stars}
\begin{tabular}{lll}
\toprule
Property & Scl: LMC-exponential & LMC-Plummer\\
\midrule
$\sigma_{\V, i}\,/\,\kms$ & 9.0 & 9.4\\
$\sigma_{\V, f} \,/\,\kms$ & 8.8 & 9.2\\
fractional stellar mass loss & $<10^{-12}$ & 0.0013\\
$R_{h, i}$ / kpc & 0.186 & 0.201\\
$R_{h, f}$ / kpc & 0.189 & 0.205\\
break radius & $78'$, 1.6 kpc & $81'$, 1.7 kpc\\
LMC break radius & $23'$, 0.49 kpc & $24'$, 0.52 kpc\\
\bottomrule
\end{tabular}
\end{table*}

\subsection{Summary}\label{summary}

We find, including only the MW potential, that tides only remove dark
matter from the outskirts of Scl. The central cusp and compact stellar
distribution are resilient to tides. Any tidal effects are predicted to
be well outside the reach of current observations. We have also found
that the LMC strongly perturbs Scl's orbit---in this case, Scl may be on
first infall. However, with only 1 pericentre each for the LMC and MW,
the combined tides are weaker than for our initial model. In either
case, we conclude that tides are unlikely to affect Sculptor's stellar
component.

\section{Tidal effects on Ursa Minor}\label{tidal-effects-on-ursa-minor}

\subsection{Evolution of Ursa Minor's dark matter
halo}\label{evolution-of-ursa-minors-dark-matter-halo}

\begin{figure}
\centering
\includegraphics[width=1\linewidth,height=\textheight,keepaspectratio]{figures/umi_sim_images.png}
\caption[Ursa Minor simulation snapshots]{Similar to
Fig.~\ref{fig:scl_sim_images} but for Ursa Minor on the \smallperi{}
orbit. Dark matter evolution is more dramatic than for
Scl.}\label{fig:umi_sim_images}
\end{figure}

\begin{table*}[t]
\centering
\caption[Simulation results for Ursa Minor’s dark matter]{The present-day properties for Ursa Minor’s final dark matter halo. See Table \ref{tbl:scl_sim_results} for details. }
\label{tbl:umi_sim_results}
\begin{tabular}{lll}
\toprule
Property & Random orbits & \smallperi{}\\
\midrule
pericentre & $37\pm3$ & 30\\
apocentre & $83 \pm 4$ & 75\\
time of last pericentre & $-0.96 \pm 0.07$ & -0.80\\
number of pericentres & 7–8 & 8\\
Jacobi radius / kpc & $3.7 \pm 0.3$ & 2.9\\
Jacobi radius / arcmin & $184 \pm 12$ & 156\\
final heliocentric distance & $70.1 \pm 3.6$ & 64.7\\
${\vmax}_f / {\vmax}_i$ &  & 0.511\\
${\rmax}_f / {\rmax}_i$ &  & 0.249\\
fractional dm final mass &  & 0.035\\
\bottomrule
\end{tabular}
\end{table*}

\begin{figure}
\centering
\includegraphics[width=4.5in,height=\textheight,keepaspectratio]{figures/umi_tidal_track.png}
\caption[Ursa Minor tidal tracks]{Similar to
Fig.~\ref{fig:scl_tidal_track} except for Ursa Minor. Ursa Minor loses
substantially more mass than Sculptor.}\label{fig:umi_tidal_track}
\end{figure}

The tidal evolution of Ursa Minor is similar to that of Sculptor in the
MW-only potential. Fig.~\ref{fig:umi_sim_images} shows snapshots of the
DM evolution. UMi loses significantly more DM mass than Scl, forming
substantial dark matter streams encircling the MW several times.

UMi only retains 3\% of its total mass after 9 Gyr
(Table~\ref{tbl:umi_sim_results}). As a result, the final dark matter
component is much smaller than the initial, but still evolves along the
predicted tidal track (Fig.~\ref{fig:umi_tidal_track}). Despite the more
substantial tidal evolution, the Jacobi radius is still large, lying at
\(\sim 4\,\kpc\), well beyond the final \(\rmax\).

Because of UMi's mass loss, the orbit deviates substantially from a
point orbit. Through our orbit-adjustment procedure in
Section~\ref{sec:orbit_corrections}, we recover nearly exactly the
present-day position of Ursa Minor by changing the initial
position/velocity by \(20\,\kpc\) and \(\sim 9\,\kms\). These
adjustments do not significantly affect the qualitative structure or
pericentre of the orbit.

\subsection{Evolution of Ursa Minor's
stars}\label{evolution-of-ursa-minors-stars}

\begin{table*}[t]
\centering
\caption[Simulation results for Ursa Minor’s stars]{Similar to Table \ref{tbl:scl_sim_stars_results}, the present-day stellar properties for the simulation of Ursa Minor for exponential and Plummer stars. }
\label{tbl:umi_sim_stars_results}
\begin{tabular}{lll}
\toprule
Property & \smallperi{}-exp & \smallperi{}-Plummer\\
\midrule
$\sigma_{\V, i}\,/\,\kms$ & 11.0 & 12.1\\
$\sigma_{\V, f}\,/\,\kms$ & 8.8 & 9.1\\
fractional stellar mass loss & $0.00138$ & 0.068\\
$R_{h, i}\,/\,\kpc$ & 0.201 & 0.20\\
$R_{h, f}\,/\,\kpc$ & 0.212 & 0.257\\
break radius & 210 arcmin, 4.0 kpc & 218 arcmin, 4.1 kpc\\
\bottomrule
\end{tabular}
\end{table*}

\begin{figure}
\centering
\pandocbounded{\includegraphics[keepaspectratio]{figures/umi_smallperi_i_f.pdf}}
\caption[Ursa Minor simulated density profiles]{Similar to
Fig.~\ref{fig:scl_smallperi_i_f}: the tidal effects on the stellar
surface density of Ursa Minor for exponential stars on the \smallperi{}
orbit.}\label{fig:umi_smallperi_i_f}
\end{figure}

\begin{figure}
\centering
\pandocbounded{\includegraphics[keepaspectratio]{figures/umi_plummer_i_f.pdf}}
\caption[Ursa Minor Plummer model density]{Similar to
Fig.~\ref{fig:scl_smallperi_i_f}: the tidal effects on the stellar
surface density of Ursa Minor for Plummer stars on the \smallperi{}
orbit.}\label{fig:umi_plummer_i_f}
\end{figure}

Tidal features in UMi's stellar component are still extremely faint,
becoming apparent only outside 100 arcminutes in
Fig.~\ref{fig:umi_smallperi_i_f}. The observed size and velocity
dispersion of Ursa Minor evolve little
(Table~\ref{tbl:umi_sim_stars_results}). For exponential initial
conditions, tidal effects are unlikely to be observable in the near
future.

The break and Jacobi radii fall well outside the observed stellar
profile. Tides would have to be far stronger to affect the observed
stellar component. As a result, the minimal tidal evolution of this
model is not unexpected.

As for Scl (Section~\ref{sec:scl_sim_stars}), we also consider a model
where UMi's stars follow initially a Plummer profile, resembling the
present-day density profile. The stellar evolution of this Plummer
stellar component is similar (Fig.~\ref{fig:umi_plummer_i_f}). However,
because there are more loosely-bound stars, the Plummer model loses
nearly 7\% of its initial stellar mass to tides
(Table~\ref{tbl:umi_sim_stars_results}), and tidal features may be
detectable if we measure densities 2 orders of magnitude fainter than
our present data. We show the properties of a stream in the Appendix
(Fig.~\ref{fig:umi_tidal_stream}), but such a stream is unlikely to be
observable in the near future. We conclude that tides do not strongly
affect the stellar component of this model.

\subsection{Effects of the LMC}\label{effects-of-the-lmc}

\begin{figure}
\centering
\includegraphics[width=1\linewidth,height=\textheight,keepaspectratio]{figures/umi_lmc_xyzr_orbits.png}
\caption[Ursa Minor orbits with LMC]{Orbits of Ursa Minor with (orange)
and without (green) an LMC. The final positions of Ursa Minor and the
LMC are plotted as scatter points, and the solid blue line represents
the LMC trajectory. Note that the LMC mostly increases Ursa Minor's
pericentres and apocentres.}\label{fig:umi_orbits_lmc}
\end{figure}

Fig.~\ref{fig:umi_orbits_lmc} shows the effects of including an LMC on
the orbit of Ursa Minor. Predominantly, the effect is to increase the
orbital period, apocentre, and pericentre. Yet, the orbit remains in a
similar plane and with similar shape. As UMi is on the opposite side of
the Galaxy of the LMC, and has a closest LMC approach of
\(\gtrsim 100\,\kpc\), this is not surprising.

The deviation from the MW-only orbit is mostly due to the LMC-induced
reflex motion of the Milky Way. Because the MW centre is accelerated
towards the LMC and away from UMi, UMi's orbit increases in
characteristic radius.

\subsection{Summary}\label{summary-1}

While tides affect UMi more strongly than Scl, the tidal effects are
insufficient to reshape the observed stellar density profile. Faint
tidal tails may be observable with deeper data. Finally, including the
LMC in the potential further weakens the tides experienced by UMi.

\section{Modelling uncertainties}\label{modelling-uncertainties}

\subsection{Halo structure}\label{halo-structure}

As the above results show, tides only marginally affect the stellar
components of Scl and UMi, even with orbits chosen to have the smallest
observationally-consistent pericentres. While we have only presented
select models, alternative initial conditions do not affect our
qualitative conclusions on tidal effects expected for Scl and UMi in the
Galactic (and LMC) potential.

Although our analysis neglects baryonic physics, Scl and UMi have
predominantly stars older than \(\sim 9\) Gyr
\citep{carrera+2002, deboer+2011, weisz+2014, delosreyes+2022, sato+2025}.
So, gas dynamics are unlikely to affect recent evolution. A
collisionless dark-matter-only simulation should therefore be an
excellent approximation.

Cored or less concentrated dark matter halos disrupt quicker
\citep[e.g.,][]{stucker+2023}. Our fiducial UMi halo, in particular, is
among the least concentrated halos consistent with UMi's velocity
dispersion. Although Scl's fiducial halo is more concentrated, less
concentrated and cored halos evolve similarly (see Appendix
\ref{sec:extra_results}).

Galaxies are rarely perfect isotropic spheres. Sculptor and Ursa Minor
are elliptical, and halos are expected to be radially anisotropic
\citep[e.g.,][]{navarro+2010}. We test non-spherical and anisotropic
models in Appendix \ref{sec:extra_results}, finding that these
assumptions likely do not alter our conclusions.

While alternative initial conditions may influence the total mass
evolution, they should produce a similar final stellar structure. A
system's observed velocity dispersion directly constrains the mean
density within \(R_h\) \citep[e.g.,][]{wolf+2010}. Thus, the tidal force
required to disrupt the stellar component depends mainly on the total
mass inside the stellar half-mass radius, and less on the actual shape
of the density profile of the dark matter inside that radius.

\subsection{Orbital uncertainties}\label{sec:scl_umi_orbit_uncert}

The long-term orbital evolution of satellites are uncertain. Analytic
Milky Way potentials neglect many unknown details, including
triaxiality, mass evolution, and substructure. Due to these
inadequacies, calculated orbits may diverge significantly from the true
orbits of satellites \citep[e.g.,][]{dsouza+bell2022}. The mass-growth
of the Milky Way and dynamical friction imply that orbits were typically
less bound in the past. Orbital energy and angular momentum of subhalos
are not conserved in cosmological N-body simulations. Consequently,
orbits in analytic potentials may overestimate the pericentre and
underestimate the maximum tidal stress \citep[although typically not by
enough to change our
conclusions,][]{santistevan+2023, santistevan+2024}.

As an example, Fig.~\ref{fig:scl_orbit_lmc_uncert} illustrates how
changes to the LMC potential modify the long-term orbital trajectories
of Scl and UMi. More than 4 Gyr ago, the orbits of Scl diverge
substantially. Some orbits are near apocentres of \(\sim 300\,\kpc\)
when others approach pericentres as small as \(\sim 10\,\kpc\). Ursa
Minor's orbit is more stable until the possible previous LMC pericentre.
In some cases, Ursa Minor may have been bound to the LMC.

Motivated by Fig.~\ref{fig:scl_orbit_lmc_uncert}, we examine an extreme
pericentre of \(4\,\kpc\) of Scl with the MW in Appendix
\ref{sec:extra_results}, finding it still insufficient to produce the
observed density profile. Regardless, we conclude our simulated orbits
represent reasonable extremes for \emph{recent} tidal effects. Past
encounters with the LMC are revisited below as a form of
``pre-processing'' in Section~\ref{sec:stellar_halos}.

\begin{figure}
\centering
\includegraphics[width=1\linewidth,height=\textheight,keepaspectratio]{/Users/daniel/thesis/figures/scl_lmc_orbits_mass_effect.png}
\caption[Long term orbital uncertainties]{The long-term orbital history
of Sculptor (\textbf{top}) and Ursa Minor (\textbf{bottom}) is
uncertain. In both panels, light, transparent lines represent randomly
sampled orbits of the satellites (following
Section~\ref{sec:scl_smallperi}) in three different LMC/MW mass models
from \citet{vasiliev2024}. The LMC orbits are in solid, thick lines of
the corresponding colour. The L2M11 has a lighter LMC mass, and the
L3M10 model has a lighter MW mass than our fiducial L3M11 LMC
model.}\label{fig:scl_orbit_lmc_uncert}
\end{figure}

\subsection{Summary}\label{summary-2}

While the long-term tidal evolution is unconstrained, we conclude that
our models are reasonable representations of recent tidal effects. As a
result, recent tides are unlikely to affect the stellar distributions of
Sculptor and Ursa Minor.


%\chapter{The effects of tides on the Ursa Minor Dwarf Spheriodal Galaxy}
%


\chapter{Discussion}\label{sec:discussion}
In the previous chapters, we have shown that (1) Sculptor and Ursa Minor
have unusually extended density profiles for MW classical satellites and
(2) recent tides are an unlikely explanation. Here, we first consider
the reliability and assumptions supporting our conclusions. We then
consider possible formation scenarios of extended density profiles,
including multi-epoch star formation, dwarf mergers and subhalo
encounters, and alternative theories of dwarf galaxy structure. We
finish with an outlook on prospects for disentangling various stellar
halo formation scenarios.

While tides have been a popular explanation for some features of dwarf
galaxies \citep[discussion above, and
e.g.,][]{tsujimoto+shigeyama2002, mayer+2001a}, our work, and many
others, have disfavoured this for Scl, UMi, and many other MW dwarfs.
For example, \citet{read+2006} suggest that the lack of observational
evidence for a rising velocity dispersion profile with radii indicates
most dwarfs are not tidally affected. Later, \citet{penarrubia+2009}
define and use the break radius, Eq.~\ref{eq:r_break}, to show that the
(then understood) orbits of most satellites are inconsistent with a
tidal feature. In addition, \citet{pace+erkal+li2022} use a criterion
based on the observed mean density and orbital pericentre (like the
Jacobi radius) to show most dwarfs are unlikely to be undergoing
significant tidal disruption. Most recently,
\citet{tchiorniy+genina2025} use a similar idealized framework to ours,
with a focus on the inner density, showing tides do not strongly affect
equilibrium assumptions for several classical dwarfs. The general tidal
evolution of our simulations is furthermore consistent with much of the
literature \citep[e.g.,][]{robles+bullock2021, EN2021}. Tides seem to
now play a subdominant role in the evolution of most Milky Way
satellites.

\section{Comparison with prior work}\label{comparison-with-prior-work}

For both Scl and UMi, these galaxies have been studied extensively in
both a theoretical and observational context. While many works
considered mass modelling, star formation histories, chemistry, and
other observational properties, we focus on the observations and models
concerning tides and dynamical evolution here.

However, some cosmological simulations have suggested that tidal effects
may be ubiquitous. \citep{riley+2024, shipp+2023} find that a majority
of satellites form substantial streams, and not due to resolution
effects \citep[see also][]{panithanpaisal+2021}. This result appears to
be in tension with ours and many previous works. Some reasons
cosmological and idealized simulations may give opposite results could
include resolution effects \citep[e.g.,][]{santos-santos+2025},
additional perturbations from dark matter substructure, tidal
``pre-processing'' by other infalling satellites, or the streams may be
below detection limits \citep[e.g.,][]{shipp+2023}. More work is needed
to understand why streams appear to form more readily in cosmological
simulations than is observed in the Milky Way or through idealized
simulations.

Closely related to our work, \citet{iorio+2019} applied idealized N-body
simulations to study tidal effects on Scl. They similarly found weak
tidal effects, even for a dark-matter-free model. However, their orbits
are less well-constrained and they did not consider the LMC.

\subsection{Sculptor}\label{sculptor}

Scl has long been speculated to be disturbed. \citet{innanen+papp1979}
found RR Lyrae candidate Scl members \citep[from][]{vanagt1978} out to
180' in an elongated distribution, speculating this to be tidal
disruption. Many later density profile determinations noted Scl's
elongation and apparent outer density excess similar to J+24's detection
\citep[but see also
\citet{coleman+dacosta+bland-hawthorn2005}]{eskridge1988, IH1995, walcher+2003, westfall+2006}.
Scl's ellipticity and extended density profile were often interpreted as
evidence of tidal debris or sometimes a dwarf galaxy halo.

In addition, Scl hosts at least two populations, as revealed through
photometry \citep{tolstoy+2004}, kinematics
\citep{battaglia+2008, tolstoy+2023, arroyo-polonio+2024}, dynamical
structure \citep{breddels+helmi2014}, age gradients \citep{deboer+2011},
and chemistry \citep{kirby+2009}. These two populations both have radii
smaller than the outer density excess. Scl challenges the idea of a
simple, single-component dwarf spheroidal population and hints at
episodic star formation or hierarchical assembly.

\subsection{Ursa Minor}\label{ursa-minor}

UMi has garnered claims of inner substructure such as stellar or
kinematic ``clumps''. Studies by \citet{olszewski+aaronson1985},
\citet{demers+1995}, \citet{IH1995}, \citet{kleyna+1998},
\citet{battinelli+demers1999}, and \citet{bellazzini+2002} note that
Ursa Minor appears to contain ``clumps'' along its major axis. One clump
has furthermore been shown to be kinematically distinct and colder
\citep[e.g.,][]{pace+2014}. If a star cluster, than the cluster should
disolve in \(\sim 3\) Gyrs provided the UMi's halo is not cored and did
not interact with dark (sub)subhalos---the survival of such substructure
depends on the nature of dark matter \citep{kleyna+2003, lora+2012}.
\citet{wilkinson+2004} additionally find a puzzling drop in the velocity
dispersion in the outskirts. The nature of any clumps or substructure
remains unclear, but \citet{munoz+2018}'s are unable to find signs of
any substructure with modern, deeper photometry.

In addition, several works have found supposed tidal features in UMi.
\citet{hargreaves+1994} first detected a velocity gradient in UMi,
suggestive of tidal disruption. Later, \citet{martinez-delgado+2001}
find that stars extend far beyond the ``tidal radius'' (from a King
profile fit) for Ursa Minor, in the direction of the galaxy's elongation
\citep[see corresponding simulations
by][]{gomez-flechoso+martinez-delgado2003}. \citet{palma+2003} further
showed evidence for S-shaped contours, and an extended population of
``extratidal'' stars. Our density profiles are consistent with these
works. However, strong evidence of tidal disruption has not yet been
found.

\section{Forming a stellar halo}\label{sec:stellar_halos}

As a tidal origin of Scl and UMi's extended density profiles is
disfavoured, we consider alternative processes which may form a dwarf
galaxy ``stellar halo''. Some recent works have also confirmed the
presence of extended, likely non-tidal, density profiles. For example,
\citet{chiti+2021}; \citet{chiti+2023} spectroscopically confirm members
out to \(~9 R_h\) in Tucana II. Given these stars are misaligned with
the orbit and the lack of a velocity gradient, tides seem to be an
unlikely explanation for Tucana II as well. The prevalence of stellar
halos around dwarf galaxies remains an active research topic.

Many dwarf galaxies also host multiple chemodynamical stellar
populations. Typically, older populations are extended, kinematically
hotter, and metal-poo , whereas the younger populations are more
compact, colder, and metal-rich. Examples include Carina
\citep[\citet{fabrizio+2016}, \citet{kordopatis+2016}]{battaglia+2012},
Fornax \citep[\citet{amorisco+evans2012},
\citet{delpino+aparicio+hidalgo2015}]{battaglia+2006}, Sextans
\citep{battaglia+2011, cicuendez+battaglia2018, roederer+2023}, and
Andromeda II
\citep{mcconnachie+arimoto+irwin2007, ho+2012, delpino+2017}. Evidence
of multiple-populations in dwarf galaxies suggests that effects like
galaxy mergers or episodic star formation shape the stellar structure of
dwarf galaxies.

We now explore possible origins of a ``stellar halo'' which may resemble
the extended density profiles in Scl and UMi.

\subsection{Internal processess}\label{internal-processess}

\textbf{Dynamical heating of old stars}. In a galaxy, older stars
generally have higher random velocities (i.e., kinematically ``hotter'')
than younger stars. In dwarf galaxies, several mechanism may heat
stellar components, including supernovae kicks in forming stars,
feedback-driven potential fluctuations
\citep{stinson+2009, maxwell+2012, el-badry+2016, mercado+2021}, or
heating by dark sub-subhalos \citep{penarrubia+2025}. Over time, these
processes naturally produce a more extended, older stellar population.

\textbf{Episodic star formation and feedback.} Dwarf galaxies are
thought to experience bursty star formation--i.e., consisting of several
episodes of intense star formation separated by periods of quiescence
\citep[e.g.,][]{salvadori+ferrara+schneider2008, valcke+derijcke+dejonghe2008, wheeler+2019, azartash-namin+2024}.
Stellar feedback in a dwarf galaxy's shallow potential well drives
oscillations in the star formation rate. Alternatively, re-ionization
may temporarily suspending star formation \citep{benitez-llambay+2015}.
In many simulations, shrinking gas reservoirs form
centrally-concentrated later generations of stars, naturally spawning
multiple populations \citep{kawata+2006, revaz+jablonka2018}. However,
star formation is also sensitive to a dwarf's environment.

\subsection{External processes}\label{external-processes}

\textbf{Induced star formation.} Star formation may quench and reignite
due to an external perturbation. Examples include tidal compression
\citep{mayer+2001a, dong+lin+murray2003}, gaseous filaments
\citep{genina+2019}, dark halos \citep{starkenburg+helmi+sales2016}, or
shocks with the MW corona \citep{wright+2019}. Induced burst would be
more stochastic than internal bursts, possibly explaining diversity in
dwarf galaxy structure.

\textbf{Major mergers.} When galaxies merge, they may leave signatures
such as population gradients and stellar halos. Classical dwarfs have a
\(\sim 10\%\) chance of undergoing a major merger since redshift \(z=1\)
\citep{deason+wetzel+garrison-kimmel2014}. Mergers may preferentially
disperse the lower-mass galaxy's stars, forming a halo and a metallicity
gradient \citep{benitez-llambay+2016}. Intermediate-mass mergers (with
mass-ratios of \(\sim\) 1:5) most effectively populate halos, balancing
stellar mass from larger galaxies with better dispersion of lower-mass
galaxies \citep{deason+2022}. Tuc II's properties are suggested to
originate from a similar merger
\citep{tarumi+yoshida+frebel2021, querci+2025}.

\textbf{Gas-rich mergers.} If a merger occurs between gas-rich galaxies,
triggered star formation may occur in the aftermath
\citep[e.g.,][]{genina+2019}. And II and Phoenix have steep metallicity
gradients and unusual prolate rotation, theorized to result from mergers
of disky dwarfs \citep{lokas+2014, fouquet+2017, cardona-barrero+2021}.

\textbf{Minor mergers / accretion}. Like how the Milky Way's halo is
believed to be built from many minor mergers, dwarf galaxies may form
halos through accretion of yet fainter dwarfs.
\citet{ricotti+polisensky+cleland2022} demonstrated that accretion of
ultra-faint dwarfs may create dwarf stellar halos. A stream detected
around And II further supports the occurrence of mergers among dwarfs
\citep{amorisco+evans+vandeven2014, roederer+2023}. In this scenario,
the halo chemistry would resemble a population of ultra-faint dwarfs.
The occurance of dwarf mergers would further constrain small-scale
galaxy formation.

\textbf{Tidal preprocessing}. Dwarf galaxies may have been
``preprocessed'' by a larger satellite like the LMC
\citep[e.g.,][]{santistevan+2023, riley+2024}. From the orbit
integrations above, it is possible that UMi was once an LMC satellite.
However, the prevalence of preprocessing remains uncertain. Like stellar
heating, preprocessing redistributes already-present stellar
populations.

\textbf{Tidal dwarf galaxies} are cluster-like objects which form in
gas-rich tidal streams created during the merger of two galaxies
\citep[e.g.,][]{mirabel+dottori+lutz1992, bournaud+duc2006}. Tidal
dwarfs may be more susceptible to tides, forming extended density
profiles and appearing to have dark matter
\citep{casas+2012, yang+2014, wang+2024a}. If Scl and UMi are tidal
dwarfs, a stronger velocity (dispersion) gradient and tidal tails should
be detectible.

\subsection{\texorpdfstring{Beyond \LCDM{}}{Beyond }}\label{beyond}

\textbf{Modified Newtonian Dynamics (MOND)}. MOND modifies gravity
instead of using dark matter to explain the rotation curves of galaxies.
In MOND, tides may more strongly affect dwarf galaxies owing to the lack
of dark matter mass and the stronger MW tidal field
\citep{mcgaugh+wolf2010, brada+milgrom2000}. A tidal origin of the
density excess is more likely in this case. If MOND is to recover the
observed velocity dispersions of satellites, then many more dwarfs may
be actively tidally disrupting as well
\citep{casas+2012, yang+2014, wang+2024a}.

\textbf{Self-interacting dark matter (SIDM).} SIDM significantly
complicates tidal evolution. SIDM halos in isolation are not static---by
transferring heat through collisions, these halos first undergo core
formation and then core collapse. Besides structural changes, SIDM adds
that pressure from the host DM halo can change the structure
(e.g.~aiding core collapse) or remove material from the inner galaxy
(analogous to ram pressure stripping) \citep[e.g.,][]{cartonzeng+2024}.
An SIDM halo may be more strongly affected by tides, but the velocity
dispersion makes a tidal disruption in SIDM still unlikely.

\textbf{Fuzzy dark matter} can also heat up stars owing to density
perturbations in the form of interference fringes
\citep[e.g.,][]{el-zant+2020, duttachowdhury+2023}. Similar to other
intrinsic heating methods, this model would likely affect all dwarfs
similarly, so we should detect similar halos around similar dwarfs.

\subsection{Disentangling the origin of a stellar
halo.}\label{disentangling-the-origin-of-a-stellar-halo.}

We just reviewed a number of different, possibly-concurrent explanations
for the formation of a stellar halo. While we leave the nature of Scl
and UMi's extended stellar densities an open question, we can discuss
possible clues to different formation scenarios.

\textbf{Chemistry}. Large samples of detailed chemical abundances have
been invaluable for understanding MW substructure (e.g.~for
\emph{Gaia}-Sausage Enceladus). Chemistry, particularly comparing the
inner and outer regions could test if the halo appears to originate from
a distinct system than the dwarf, differentiating many \emph{internal}
versus \emph{external} scenarios.

\textbf{Star formation histories}. Evidence of significant star
formation episodes or lack thereof may differentiate scenarios which
rely on strong star formation bursts (e.g.~\emph{episodic star formation
history, induced star formation, and gas-rich mergers.})

\textbf{Kinematics.} Particularly, for models relying on recent tidal
disruption, kinematic disequilibrium features should be visible. These
would appear as velocity gradients, increasing velocity dispersions,
outward-biased moving stars, or non-phase mixed structures
\citep[e.g,][]{kroupa1997, read+2006, sanchez-salcedo+hernandez2007}.

\textbf{Deep photometry} may find or rule out signs of dynamical
disequilibrium and tidal tails for tidally susceptible models
\emph{(e.g., MOND and tidal dwarfs.)}

Ongoing, upcoming, and future facilities will be essential for testing
these theories. For example, given the typical faint magnitudes of dSph
stars, chemistry is best accessible through large field-of-view
multi-object spectroscopic instruments on large telescopes (e.g., 4MOST
and extremely large telescopes \citet{skuladottir+2023}). 3D internal
kinematics of dwarfs may require a successor to \emph{Gaia}. However,
using JWST and HST enables precise proper motions for small regions
within dwarf galaxies \citep[e.g.,][]{vitral+2025}, which may be
sufficient to constrain vastly different internal dynamic structures and
ongoing tidal disruption. Finaly, Photometric surveys by Rubin
Observatory and Euclid will most-emminently probe yet fainter magnitudes
around dwarf galaxies and possibly find or constrain stellar halos and
their star formation histories. Altogether, new surveys will likely
uncover novel aspects about the inner workings and outer halos of dwarf
galaxies.

In addition to new observing facilities, the next generation of dwarf
simulations are beginning to answer questions and unravel processes in
the formation of dwarf galaxies. With improved resolution, realism, and
physics, the frequency and effects of various mechanisms discussed can
be better constrained.

\section{Conclusion and outlook}\label{conclusion-and-outlook}

In this thesis, we have investigated the extended, outer density excess
of Sculptor and Ursa Minor and its possible tidal origin. We first
verified that Scl and UMi have unusually extended density profiles,
compared to other dwarf spheroidals. We show that the density profiles
are robust to alternate data criteria, implying that this ``density
excess'' is likely a real feature of each galaxy.

We then investigated if tides were a permissible explanation. By
modelling each galaxy based on cosmological initial conditions, we
showed that tides do not strongly affect either galaxy. The LMC changes
the orbital history of Scl and UMi, and tides become even weaker in a
combined LMC and MW potential. While UMi may form a stellar stream, the
stream is far fainter than is presently detectible. We conclude that
recent tides are unlikely to shape the observed stellar distributions of
Scl and UMi.

Finally, we consider alternative scenarios forming extended density
distribution. We review mechanisms ranging from episodic star formation
histories, to accretion, to departures from \LCDM{}. While a more
precise explanation awaits, upcoming and ongoing surveys will uncover
the stellar populations within dwarf galaxies in unprecedented detail.
Future work will the Milky Way, uncovering the cosmic origins of the
smallest structures forming building blocks of galaxies like ours.


\chapter{Summary and Outlook}\label{sec:summary}

%%%%%%%%%%%%%%%%%%%% REFERENCES %%%%%%%%%%%%%%%%%%

\newpage
\bibliographystyle{aasjournal}
\addcontentsline{toc}{chapter}{Bibliography}
\bibliography{main} 


%%%%%%%%%%%%%%%%% APPENDICES %%%%%%%%%%%%%%%%%%%%%

\appendix

\chapter{Data selection}\label{data-selection}

\section{Describution of algorithm}\label{describution-of-algorithm}

To create a high-quality sample, J+24 select stars initially from Gaia
within a 2--4 degree circular region centred on the dwarf satisfying:

\begin{itemize}
\tightlist
\item
  Solved astrometry, magnitude, and colour.
\item
  Renormalized unit weight error, \({\rm ruwe} \leq 1.3\), ensuring high
  quality astrometry. \texttt{ruwe} is a measure of the excess
  astrometric noise on fitting a consistent parallax-proper motion
  solution \citep[see][]{lindegren+2021}.\\
\item
  3\(\sigma\) consistency of measured parallax with dwarf's distance
  (dwarf parallax is very small; with \citet{lindegren+2021} zero-point
  correction).
\item
  Absolute proper motions, \(\mu_{\alpha*}\), \(\mu_\delta\), less than
  10\(\,{\rm mas\ yr^{-1}}\). (Corresponds to tangental velocities of
  \(\gtrsim 500\) km/s at distances larger than 10 kpc.)
\item
  Corrected colour excess is within 3\(\sigma\) of the expected
  distribution from \citet{riello+2021}. Removes stars with unreliable
  photometry.
\item
  De-reddened \(G\) magnitude is between
  \(22 > G > G_{\rm TRGB} - 5\sigma_{\rm DM}\). Removes very faint stars
  and stars significantly brighter than the tip of the red giant branch
  (TRGB) magnitude plus the distance modulus uncertainty
  \(\sigma_{\rm DM}\).
\item
  Colour is between \(-0.5 < {\rm BP - RP} <  2.5\) (dereddened).
  Removes stars substantially outside the expected CMD.
\end{itemize}

Photometry is dereddened with \citet{schlegel+finkbeiner+davis1998}
extinction maps.

J+24 define likelihoods \({\cal L}\) representing the probability
density that a star is consistent with either the MW stellar background
(\({\cal L}_{\rm bg}\)) or the satellite galaxy
(\({\cal L}_{\rm sat}\)). In either case, the likelihoods are the
product of a spatial, PM, and CMD term: \begin{equation}{
{\cal L} = {\cal L}_{\rm space}\ {\cal L}_{\rm PM}\ {\cal L}_{\rm CMD}.
}\end{equation}

Each likelihood is normalized over their respective 2D parameter space
for both the satellite. To control the relative frequency of member and
background stars, \(f_{\rm sat}\) representing the fraction of member
stars in the field. The total likelihood for any star in this model is
the sum of the satellite and background likelihoods, weighted by their
relative frequencies,
\begin{equation}\protect\phantomsection\label{eq:Ltot}{
{\cal L}_{\rm tot} = f_{\rm sat}{\cal L}_{\rm sat} + (1-f_{\rm sat}){\cal L}_{\rm bg}.
}\end{equation} The probability that any star belongs to the satellite
is then given by \begin{equation}{
P_{\rm sat} = 
\frac{f_{\rm sat}\,{\cal L}_{\rm sat}}{{\cal L}_{\rm tot}}
= \frac{f_{\rm sat}{\cal L}_{\rm sat}}{f_{\rm sat}{\cal L}_{\rm sat} + (1-f_{\rm sat}){\cal L}_{\rm bg}}.
}\end{equation}

For the satellite's spatial likelihood, J+24 consider both one-component
and a two-component density models. The one component model is
constructed as a single exponential profile ( surface density
\(\Sigma \propto e^{R_{\rm ell} / R_s}\)), with scale radius \(R_s\)
fixed to the value in table 1 of \citet{MV2020a} from \citet{munoz+2018}
(for a Sérsic fit). Additionally, structural uncertainties (for position
angle, ellipticity, and scale radius) are sampled over to construct the
final likelihood map. The two-component model instead adds a second
exponential,
\(\Sigma_\star \propto e^{-R/R_s} + B\,e^{-R/R_{\rm outer}}\). The inner
scale radius is fixed, and the outer scale radius and magnitude of the
second component \(R_{\rm outer}\), \(B\) are free parameters.
Structural property uncertainties are not included in the two-component
model.

The PM likelihood is a bivariate gaussian with variance and covariance
equal to each star's proper motions. J+24 assume the stellar PM errors
are the main source of uncertainty.

The satellite's CMD likelihood is based on a Padova isochrone
\citep{girardi+2002}. The isochrone has a matching metallicity and 12
Gyr age (except 2 Gyr is used for Fornax). The (gaussian) colour width
is assumed to be 0.1 mag plus the Gaia colour uncertainty at each
magnitude. The horizontal branch is modelled as a constant magnitude
extending blue of the CMD (mean magnitude of -2.2, 12 Gyr HB stars and a
0.1 mag width plus the mean colour error). A likelihood map is
constructed by sampling the distance modulus in addition to the CMD
width, taking the maximum of RGB and HB likelihoods.

The background likelihoods are instead empirically constructed. Stars
stars outside of 5\(R_h\) passing the quality cuts estimate the
background density in PM and CMD space. The density is a sum of
bivariate gaussians with variances based on Gaia uncertainties (and
covariance for proper motions). The spatial background likelihood is
assumed to be constant.

J+24 derive \(\mu_{\alpha*}\), \(\mu_\delta\), \(f_{\rm sat}\) (and
\(B\), \(R_{\rm outer}\) for two-component) through an MCMC simulation
with likelihood from Eq.~\ref{eq:Ltot}. Priors are only weakly
informative. The proper motion single component prior is same as
\citet{MV2020a}: a normal distribution with mean 0 and standard
deviation \(100\ \kms\). If 2-component spatial, instead is a uniform
distribution spanning 5\(\sigma\) of single component case w/ systematic
uncertainties. \(f_{\rm sat}\) (and \(B\)) has a uniform prior 0--1.
\(R_{\rm outer}\) has a uniform prior only restricting
\(R_{\rm outer} > R_s\). The mode of each parameter from the MCMC are
then reported and used to calculate the final \(P_{\rm sat}\) values.

\section{Additional density tests}\label{sec:density_extra}

In this section, we discuss additional tests and verification of the
derived density profiles. In particular, we check that methodology
(simpler cuts, circularized radii, algorithm version) do not
substantially affect the density profile. We also compile density
profiles presented in the literature as reference. In all cases, the
density profiles appear to have excellent convergence out to
\(\log R_{\rm ell} / {\rm arcmin} \approx 1.8\), about the distance
where the background dominates.

Discuss selection criteria for DELVE and UNIONS samples, literature
comparison, simple selection criteria, MCMC density profiles and when
\citet{jensen+2024} becomes background-limited.

\begin{figure}
\centering
\pandocbounded{\includegraphics[keepaspectratio]{figures/scl_density_methods_extra.pdf}}
\caption[Scl density comparison]{Density profiles for various
assumptions for Sculptor. PSAT is our fiducial 2-component J+24 sample,
circ is a 2-component bayesian model assuming circular radii, simple is
the series of simple cuts described, bright is the sample of the
brightest half of stars (scaled by 2), DELVE is a sample of RGB stars
(background subtracted and rescaled to
match).}\label{fig:scl_density_extras}
\end{figure}

\begin{figure}
\centering
\pandocbounded{\includegraphics[keepaspectratio]{figures/scl_density_methods_j24.pdf}}
\caption[Scl density methods]{Comparison of density profiles for each
J+24 method. The fiducial is a 2-component elliptical model. However,
the 1-component is still elliptical but only contains 1 component and
the circular model assumes a circular outer density profile and bins in
circular bins instead of elliptical
bins.}\label{fig:scl_density_j24_methods}
\end{figure}

\begin{figure}
\centering
\pandocbounded{\includegraphics[keepaspectratio]{figures/umi_density_methods_extra.pdf}}
\caption[UMi density comparison]{Similar to
Fig.~\ref{fig:scl_observed_profiles} except for Ursa
Minor}\label{fig:umi_density_extras}
\end{figure}

\begin{figure}
\centering
\pandocbounded{\includegraphics[keepaspectratio]{figures/umi_density_methods_j24.pdf}}
\caption[UMi density methods]{Similar to
Fig.~\ref{fig:scl_density_j24_methods} except for Ursa
Minor.}\label{fig:umi_density_j24_methods}
\end{figure}

\section{Comparison to Literature}\label{comparison-to-literature}

Here, we compare our density profiles against past derivations of
density profiles for Sculptor and Ursa Minor

\begin{figure}
\centering
\pandocbounded{\includegraphics[keepaspectratio]{figures/analytic_profile_comparison.pdf}}
\caption[Comparison of analytic density profiles]{A comparison of
different parameterizations for dwarf galaxy density profiles. Note that
deviations between profiles only become apparent past 3 R\_h, and only
the Plummer profile, in contrast to more commonly assumed profiles,
deviates by \textasciitilde1 dex positive before 6 Rh. Since this
profile is a far minority in the literature, deviations from exponential
and close relatives are interesting and worth further consideration.}
\end{figure}

\chapter{Radial velocity modeling}\label{sec:rv_obs}

\section{Data selection}\label{data-selection}

For both Sculptor and Ursa Minor, we construct literature samples of
radial velocity measurements. We combine these samples with J+24's
members to produce RV consistent stars and to compute velocity
dispersion, systematic velocities, and test for the appearance of
velocity gradients.

First, we crossmatch all catalogues to J+24 Gaia stars. If a study did
not report GaiaDR3 source ID's, we match to the nearest star within 1-3
arcseconds (see REF Table~\ref{tbl:rv_measurements}). We combine the
mean RV measurement from each study using the inverse-variance weighted
mean \(\bar v\), standard uncertainty \(\delta \bar v\), and (biased)
variance \(s^2\). We remove stars with significant velocity dispersions
as measured between observations in a study or between studies. By using
that \(\chi^2=\frac{s^2}{\delta \bar v^2}\), we remove stars with a
\(\chi^2\) larger than the 99.9th percentile of the \(\chi^2\)
distribution with \(N-1\) measurements. This cut typically removes stars
with reduced chi-squared values
\(\tilde\chi^2  = \frac{s^2}{\nu\,\delta \bar v^2}\gtrsim 7\) (since the
number of measurements is 1-3 typically).

Next, we need to correct the coordinate frames for the solar motion and
on-sky size of the galaxy. We transform the frame into the galactic
standard of rest (GSR). The next step is to account for the slight
differences in the direction of each radial velocity. Let the \(\hat z\)
be the direction from the sun to the dwarf galaxy. Then if \(\phi\) is
the angular distance between the centre of the galaxy and the individual
star, the corrected radial velocity is then \begin{equation}{
v_z = v_{\rm los, gsr}\cos\phi  - v_{\alpha}\cos\theta \sin\phi - v_\delta \sin\theta\sin\phi
}\end{equation} where \(v_{\rm los, gsr}\) is the line of sight velocity
in the GSR frame, \(v_\alpha\) and \(v_\delta\) are the tangental
velocities in RA and Dec, and \(\theta\) is the position angle of the
star with respect to the centre of the dwarf. The correction from both
effects induces an apparent gradient of about \(1.3\,\kmsdeg\) for
Sculptor and less for Ursa Minor \citep[see
also][]{WMO2008, strigari2010}. We add the uncertainty in \(v_z\) from
the distance uncertainty and velocity dispersion in quadrature to the RV
uncertainties for each star. We then use the \(v_z\) values for the
following modelling, however repeating with uncorrected, heliocentric
velocities does not significantly affect the results.

The combined likelihood, including RV information, becomes
\begin{equation}{
{\cal L} = {\cal L}_{\rm space} {\cal L}_{\rm CMD} {\cal L}_{\rm PM} {\cal L}_{\rm RV}
}\end{equation} where we assume that the satellite and background
distributions are Gaussian. Specifically, \begin{equation}{
\begin{split}
{\cal L}_{\rm RV, sat} &= f\left( \frac{v_i -\mu_{v}}{\sqrt{\sigma_{v}^2 + (\delta v_i)^2}}\right) \\
{\cal L}_{\rm RV, bg} &= f\left( v_i /  \sigma_{\rm halo} \right)
\end{split}
}\end{equation} where \(f\) is the probability density of a standard
normal distribution, \(\mu_v\) and \(\sigma_v\) are the systemic
velocity and dispersion of the satellite, and \(\delta v_i\) is the
individual measurement uncertainty. Typically, the velocity dispersion
will dominate the uncertainty budget here. We assume a halo/background
velocity dispersion of a constant \(\sigma_{\rm halo} = 100\,\kms\)
\citep[e.g.][]{brown+2010}.

Similar to above, we retain stars with the resulting membership
probability of greater than 0.2. Because of the additional information
from radial velocities, most stars have probabilities close to 1 or 0 so
the probability cut is not too significant.

We assume priors on systematic velocity and velocity dispersion of
\begin{equation}{
\begin{split}
\mu_{v} &= N(0\,\kms, \sigma_{\rm halo}^2) \\ 
\sigma_{v} &= U(0, 20\,\kms)
\end{split}
}\end{equation} where \(\sigma_{\rm halo} = 100\,{\rm km\,s^{-1}}\) is
the velocity dispersion of the MW halo adopted above, a reasonable
assumption for dwarfs in orbit around the MW.

\section{Results}\label{sec:rv_results}

\begin{figure}
\centering
\pandocbounded{\includegraphics[keepaspectratio]{figures/scl_umi_rv_fits.pdf}}
\caption[LOS velocity fit to Scl.]{Velocity histogram of Scl and UMi in
terms of \(v_z\) (REF). Orange points are from our crossmatched RV
membership sample.}
\end{figure}

For Sculptor, we combine radial velocity measurements from APOGEE,
\citet{sestito+2023a}, \citet{tolstoy+2023}, and \citet{WMO2009}.
\citet{tolstoy+2023} and \citet{WMO2009} provide the bulk of the
measurements. We find that there is no significant velocity shift in
crossmatched stars between catalogues. After crossmatching to high
quality Gaia stars and excluding significant stellar velocity
dispersions, we have a sample of 1918 members.

We derive a systemic velocity for Sculptor of \(111.3\pm0.2\,\kms\)with
velocity dispersion \(9.64\pm0.16\,\kms\). Our values are very
consistent with previous work \citep[e.g.][\citet{arroyo-polonio+2024},
\citet{battaglia+2008}]{walker+2009}. See appendix REF for a more
detailed comparison between different samples and additional tests.

We detect a moderately significant gradient of \(4.3\pm1.3\,\kmsdeg\) at
a position angle of \(-149_{-13}^{+17}\) degrees (see appendix REF).
Several past work has attempted to detect a gradient in Sculptor, but no
consensus has been reached. \citet{arroyo-polonio+2024} detect a
velocity gradient of \(4\pm1.5\,\kmsdeg\) in a similar direction using
the \citet{tolstoy+2023} sample, finding inconclusive statistical
evidence. They additionally suggest a third chemodynamical component of
the galaxy which may bias rotation measurements. \citet{battaglia+2008}
also detect a \(-7.6_{-2.2}^{+3.0}\,\kmsdeg\) velocity gradient along
the major axis, approximately the same direction. Instead,
\citet{strigari2010}; \citet{martinez-garcia+2023} detect no significant
gradient in Sculptor using \citet{WMO2009} sample. Note that
pre-\emph{Gaia} work did not have as strong of a constraint on the
proper motion of Scl, which limits conclusions of the intrinsic velocity
gradient in Scl.

For UMi, we collect radial velocities from, APOGEE,
\citet{sestito+2023b}, \citet{pace+2020}, and \citet{spencer+2018}. We
shifted the velocities of \citet{spencer+2018} (\(-1.1\,\kms\)) and
\citet{pace+2020} (\(+1.1\,\kms\) ) to reach the same scale. We found
183 crossmatched common stars (passing 3\(\sigma\) RV cut, velocity
dispersion cut, and PSAT J+24 \textgreater{} 0.2 w/o velocities). Since
the median difference in velocities in this crossmatch is about 2.2
km/s, we adopt 1 km/s as the approximate systematic error here. Our
final sample includes 831 members.

We derive a mean \(-245.8\pm0.3_{\rm stat}\,\kms\) and velocity
dispersion of \(8.8\pm0.2\,\kms\) for UMi. This is consistent with
\citet{pace+2020} and to a lesser extent with \citet{spencer+2018}. We
do not find evidence for a velocity gradient, consistent with past work
\citep{pace+2020, martinez-garcia+2023}.

\section{Discussion and limitations}\label{discussion-and-limitations}

Our model here is relatively simple. Some things which we note as
systematics or limitations:

\begin{itemize}
\tightlist
\item
  Inter-study systematics and biases. While basic crossmatches and a
  simple velocity shift, combining data from multiple instruments is
  challenging. This appears to be a minor issue (Sculptor) or is
  corrected for (Ursa Minor).
\item
  Misrepresentative uncertainties. Inspection of the variances compared
  to the standard deviations within a study seems to imply that errors
  are accurately reported. APOGEE notes that their RV uncertainties are
  known to be underestimates but are a small proportion of our sample.
\item
  Binarity. While not too large of a change for classical dwarfs, this
  could inflate velocity dispersions of about \(9\,\kms\) by about
  \(1\,\kms\)\citep{spencer+2017}. Thus, our measurement is likely
  slightly inflated given the high binarity fractions measured in these
  systems \citep[\citet{spencer+2018}]{arroyo-polonio+2023}.
\item
  Multiple populations. Both Sculptor and Ursa Minor likely contain
  multiple populations \citep[\citet{pace+2020},
  \citet{tolstoy+2004}]{arroyo-polonio+2024}. Since we only model a
  single population, and each population may have a different extent and
  velocity dispersion, this could result in biased velocity dispersions.
  However, it is unclear how to uniquely determine an overall velocity
  dispersion in a multi-population system.
\item
  Selection effects. RV studies each have their own selection effects,
  which may affect the resulting dispersion, especially if different
  populations or regions of the galaxy have different velocities or
  velocity dispersions. We do not attempt to correct for this.
\end{itemize}

For both Ursa Minor and Sculptor, we also fit models to only data from
individual surveys (see REF). Since the resulting parameters are very
similar, we conclude that many of the systematic uncertainties are
likely smaller than the present errors or that each large survey has
similar biases.

\section{Velocity modelling and
comparisons}\label{velocity-modelling-and-comparisons}

Here, we describe in additional detail, our methods and comparisons for
RV modelling between studies.

Savage-Dickey calculated Bayes factor using Silverman-bandwidth KDE
smoothed samples from posterior/prior.

\begin{table*}[t]
\centering
\caption[Spectroscopic LOS velocity measurements]{Summary of velocity measurements and derived properties. }
\label{tbl:rv_measurements}
\begin{tabular}{lllllllll}
\toprule
 & Study & Instrument & Nspec & Nstar & Ngood & Nmemb & $\delta v_{\rm med}$ & $R_{\rm xmatch}$/arcmin\\
\midrule
Scl & combined &  & 8945 & 2280 & 2096 & 1981 & 0.9 & \\
 & tolstoy+23 & FLAMES & 3311 & 1701 & 1522 & 1482 & 0.65 & –\\
 & sestito+23a & GMOS & 2 & 2 & 2 & 2 & 13 & –\\
 & walker+09 & MMFS & 1818 & 1522 & 1417 & 1328 & 1.8 & 3\\
 & APOGEE & APOGEE & 5082 & 253 & 170 & 164 & 0.5 & –\\
UMi & combined &  & 4714 & 1225 & 1148 & 863 & 2.1 & \\
 & sestito+23b & GRACES & 5 & 5 & 5 & 5 & 1.8 & –\\
 & pace+20 & DEIMOS & 1716 & 1538 & 829 & 678 & 2.5 & 1\\
 & spencer+18 & Hectoshell & 1407 & 970 & 596 & 406 & 0.9 & 2\\
 & APOGEE & APOGEE & 9500 & 279 & 37 & 67 & 0.6 & –\\
\bottomrule
\end{tabular}
\end{table*}

measurement

\begin{table*}[t]
\centering
\caption[Ursa Minor RV fits]{MCMC fits for UMi velocity dispersion. }
\label{tbl:umi_rv_mcmc}
\begin{tabular}{lllll}
\toprule
study & mean & sigma & $\log bf_{\rm sigma}$ & $\log bf_{\rm grad}$\\
\midrule
all & $-245.8\pm0.3$ & $8.8\pm0.2$ & +1.3 & +0.9\\
pace & $-244.6\pm0.4$ & $9.0\pm0.3$ & +0.3 & +0.5\\
spencer & $-246.9\pm0.4$ & $8.8\pm0.3$ & +1.8 & -0.06\\
apogee & $-245.6\pm1.2$ & $10.0_{-0.8}^{+1.0}$ & +1.0 & +0.5\\
\bottomrule
\end{tabular}
\end{table*}

\begin{figure}
\centering
\pandocbounded{\includegraphics[keepaspectratio]{figures/scl_rv_scatter.pdf}}
\caption[Scl velocity sample]{RV members of Sculptor plotted in the
tangent plane coloured by corrected velocity difference from mean
\(v_z - \bar v_z\) . The black ellipse marks the half-light radius in
Fig.~\ref{fig:scl_selection}. The black and green arrows mark the proper
motion (PM, GSR frame) and derived velocity gradient (rot) vectors (to
scale).}
\end{figure}

\begin{figure}
\centering
\pandocbounded{\includegraphics[keepaspectratio]{figures/scl_vel_gradient_scatter.pdf}}
\caption[Scl velocity gradient]{The corrected LOS velocity along the
best fit rotational axis. RV members are black points, the systematic
\(v_z\) is the horizontal grey line, blue lines represent the
(projected) gradient from MCMC samples, and the orange line is a rolling
median (with a window size of 50).}
\end{figure}

Here, we describe some convergence tests to ensure our methods and
results are minimally impacted by numerical limitations and assumptions.
See \citet{power+2003} for a detailed discussion of various assumptions
and parameters used in N-body simulations.

\subsubsection{Softening}\label{softening}

\citet{power+2003} suggest the empirical rule that the ideal softening
(balancing integration time and only compromising resolution in
collisional regime) is \[
h_{grav} = 4 \frac{R_{200}}{\sqrt{N_{200}}}
\]

For our isolation halo (\(M_s=2.7\), \(r_s=2.76\)) and with \(10^7\)
particles, this works out to be \(0.044\,{\rm kpc}\).We adpoted the
slightly smaller softening which was reduced by a factor of
\(\sqrt{10}\) which appears to improve agreement slightly in the
innermost regions. However, this choice likely unnecessarily increases
computation time for the relative gain in accuracy.

\begin{figure}
\centering
\pandocbounded{\includegraphics[keepaspectratio]{/Users/daniel/thesis/figures/iso_converg_softening.png}}
\caption{Softening convergence}\label{fig:softening_convergence}
\end{figure}

\subsubsection{Timestepping and force
accuracy}\label{timestepping-and-force-accuracy}

In general, we use adaptive timestepping and relative opening criteria
for gravitational force computations. To verify that these choices and
associated accuracy parameters minimally impact convergence or speed, we
show a few more isolation runs (using only 1e5 particles)

\begin{itemize}
\tightlist
\item
  constant timestep (\ldots), approximantly half of minimum timestep
  with adaptive timestepping
\item
  geometric opening, with \(\theta = 0.5\).
\item
  strict integration accuracy, (facc = \ldots.)
\end{itemize}

\subsubsection{Alternative methods}\label{alternative-methods}

\begin{itemize}
\tightlist
\item
  FMM
\item
  PMM-tree
\item
  Gadget2
\item
  etc.
\end{itemize}

\begin{figure}
\centering
\pandocbounded{\includegraphics[keepaspectratio]{/Users/daniel/thesis/figures/iso_converg_methods.png}}
\caption{Isolation method convergence}\label{fig:methods_convergence}
\end{figure}

\subsubsection{Fiducial Parameters}\label{fiducial-parameters}

Note that we use code units which assume that \(G=1\) for convenience
and numerical stability. The conversion between code units to physical
units is (for our convention):

\begin{itemize}
\tightlist
\item
  1 length = 1 kpc
\item
  1 mass unit = \(10^{10}\) Msun
\item
  1 velocity unit = 207.4 km/s
\item
  1 time unit = 4.715 Myr
\end{itemize}

Most parameters below are not too relevant or have been discussed or are
merely dealing with cpu and IO details. The changes between simulation
runs primarily affect the integration time, output frequency, and
softening. Otherwise, we leave all other parameters fixed.

\begin{verbatim}
#======IO parameters======

#---Filenames
InitCondFile                initial
OutputDir                   ./out
SnapshotFileBase            snapshot
OutputListFilename          outputs.txt

#---File formats 
ICFormat                    3       # use HDF5
SnapFormat                  3 

#---Mem & CPU limits
TimeLimitCPU                86400
CpuTimeBetRestartFile       7200
MaxMemSize                  2400

#---Time
TimeBegin                   0
TimeMax                    2120      # 10 Gyr

#---Output frequency
OutputListOn                0
TimeBetSnapshot             10
TimeOfFirstSnapshot         0 
TimeBetStatistics           10
NumFilesPerSnapshot         1
MaxFilesWithConcurrentIO    1 


#=======Gravity======


#---Timestep accuracy
ErrTolIntAccuracy           0.01
CourantFac                  0.1     # ignored; for SPH
MaxSizeTimestep             0.5
MinSizeTimestep             0.0 

#---Tree algorithm
TypeOfOpeningCriterion      1       # Relative
ErrTolTheta                 0.5     # mostly used for Barnes-Hut
ErrTolThetaMax              1.0     # (used only for relative)
ErrTolForceAcc              0.005   # (used only for relative)

#---Domain decomposition: should only affect performance
TopNodeFactor                       3.0
ActivePartFracForNewDomainDecomp    0.02

#---Gravitational Softening
SofteningComovingClass0      0.044  # HALO dependent
SofteningMaxPhysClass0       0      # ignored; for cosmological
SofteningClassOfPartType0    0
SofteningClassOfPartType1    0


#=======Miscellanius=======

# probably do not need to change the options below 

#---Unit System
UnitLength_in_cm            1 
UnitMass_in_g               1
UnitVelocity_in_cm_per_s    1 
GravityConstantInternal     1

#---Cosmological Parameters 
ComovingIntegrationOn      0 # no cosmology
Omega0                     0
OmegaLambda                 0 
OmegaBaryon                 0
HubbleParam                 1
Hubble                      100
BoxSize                     0

#---SPH
ArtBulkViscConst             0.8
MinEgySpec                   0
InitGasTemp                  100

#---Initial density estimate (SPH)
DesNumNgb                   64
MaxNumNgbDeviation          1
\end{verbatim}


\end{document}
