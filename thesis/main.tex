 %! TeX program = xelatex

\DocumentMetadata{lang=en-CA}

\documentclass[12pt,oneside,letterpaper]{report}


\usepackage{amsmath}

\usepackage[T1]{fontenc}
\usepackage{newtxtext}
\usepackage{newtxmath}
% \usepackage{libertine}
% \usepackage{libertinust1math}
\usepackage{microtype}

\usepackage{graphicx}	% Including figure files
\usepackage{booktabs}
\usepackage{longtable}
\graphicspath{{./figures/}} 
\usepackage{isotope}
\usepackage[tableposition=above]{caption}

\usepackage{aas_macros}
\usepackage{natbib}
\usepackage[nottoc,numbib]{tocbibind}
\usepackage{setspace}


\setcitestyle{aysep={}} 
\setcitestyle{notesep={; }} 


\usepackage{hyperref}
\hypersetup{
    colorlinks=true,
    linkcolor=black,
    filecolor=black,
    urlcolor=black,
    citecolor=black,
    %pdftitle={The Galactic Chemical Evolution of Carbon},
    %pdfauthor={Daniel A Boyea},
    %pdfcreator={Themself},
    %pdfkeywords={astronomy, supernovae, modeling, stellar yields, chemical evolution, galaxies, carbon},
}

\urlstyle{same}


%%%%%%%%%%%%%%% Formatting %%%%%%%%%%%%%%%%%%
\usepackage{titlesec}

\titleformat{\chapter}{\normalfont\normalsize\centering\bf}{\thechapter.}{1em}{}
\titleformat{\section}{\normalfont\normalsize\centering}{\thesection}{1em}{}
\titleformat{\subsection}{\normalfont\normalsize\centering}{\thesubsection}{1em}{}
% \titlespacing*{\chapter}{0pt}{0pt}{40pt} 
% \titlespacing*{\section}{0pt}{0pt}{20pt}
\renewcommand{\contentsname}{Table of Contents}
\AtBeginDocument{
      \renewcommand{\bibsection}{\chapter*{\bibname}}
  }


\usepackage{etoolbox}% http://ctan.org/pkg/etoolbox
\makeatletter
% \patchcmd{<cmd>}{<search>}{<replace>}{<succes>}{<failure>}
\patchcmd{\@chapter}{\addtocontents{lof}{\protect\addvspace{10\p@}}}{}{}{}% LoF
\patchcmd{\@chapter}{\addtocontents{lot}{\protect\addvspace{10\p@}}}{}{}{}% LoT
\makeatother

\pagestyle{plain}
\counterwithout{figure}{chapter}
\counterwithout{table}{chapter}
\counterwithout{equation}{chapter}

\newcommand\T{\rule{0pt}{2.6ex}}       % Top strut
\newcommand\B{\rule[-1.2ex]{0pt}{0pt}} % Bottom strut



\setkeys{Gin}{}% width=\textwidth}


\linespread{1.2}
%\usepackage{accsupp}
%\usepackage[accsupp]{axessibility}

%%%%%%%%%%%%%%% Acronyms %%%%%%%%%%%%%%%%%%
\usepackage[acronym,automake,symbols,
nogroupskip,
accsupp,
counter=section,
stylemods={longextra}
]{glossaries-extra}
\makeglossaries

\setabbreviationstyle[acronym]{long-short-sc}
\GlsXtrEnablePreLocationTag{\S }{\S }


\newacronym
[ description ={Core collapse supernovae. Massive star explosions. {\sc Ccsne} produce many elements including \gls{alpha}, \gls{fepeak}, and r and \gls{sproc}.}
]
{cc}{ccsne}{core collapse supernovae}

\newacronym
[description={Asymptotic giant branch stars. The {\sc agb} phase is the last phase of \glspl{lowmass}, before stars become white dwarfs. Produces C, N, and heavy \gls{sproc} elements. See \ref{sec:stel_evo}.}.]
{agb}{agb}{asymptotic giant branch}

\newacronym
[description={Type Ia Supernovae. Exploding white dwarfs. Produces \gls{fepeak} and has a long delay time.}]
{ia}{sne ia}{supernovae type Ia}


\newacronym
[description={Galactic chemical evolution.}]
{gce}{gce}{galactic chemical evolution}

\newacronym
[description={Star formation history.}]
{sfh}{sfh}{star formation history}

\newacronym
[description={Single stellar population. A group of stars formed all at the same time.}]
{ssp}{ssp}{single stellar population}

\newacronym
[description={Initial mass function. A function describing the mass distribution of newly formed stars. I use a \citet{kroupa01} {\sc imf}, which is described as a piecewise power-law function of $M$.}]
{imf}{imf}{initial mass function}

\newacronym
[description={
Apache Point Observatory Galactic Evolution Experiment. 
A large near-infrared spectroscopic survey of stars in the Milky Way. \citep{apogee17.}}]
{apogee}{apogee}{Apache Point Observatory Galactic Evolution Experiment}

\newacronym
[
description={Damped Lyman-alpha system. {\sc dla}s are clouds of gas from the 
early universe which are observed through their absorption of quasar spectra. The name 
comes from the strong Lyman-alpha lines ($1216\AA$) due to H absorption.}
]
{dla}{dla}{damped Lyman-alpha systems}





\newcommand{\cc}{\gls{cc}}
\newcommand{\Cc}{\Gls{cc}}
\newcommand{\agb}{\gls{agb}}
\newcommand{\ia}{\gls{ia}}
\newcommand{\sfh}{\gls{sfh}}
\newcommand{\dla}{\gls{dla}}
\newcommand{\ssp}{\gls{ssp}}
\newcommand{\imf}{\gls{imf}}
\newcommand{\gce}{\gls{gce}}
\newcommand{\Gce}{\Gls{gce}}
\newcommand{\apogee}{\gls{apogee}}




%%%%%%%%%%%%%%%%%%%%%% terms %%%%%%%%%%%%%%%
\newglossaryentry{metallicity}{name={metallicity},
    description={the (mass) fraction of a star or gas which is not made of either H or He. For the sun, the metallicity is $\Zo = 0.014$}
}

\newglossaryentry{yield}{name={yield},
    description={The net production of a new element during a star's lifecycle divided by the star's mass (including winds and supernovae ejecta). }
}

\newglossaryentry{nucleosynthesis}{name={nucleosynthesis},
    description={The synthesis of new elements through fusion inside stars. See section \ref{sec:stel_evo}.}
}

\newglossaryentry{multizone}{name={multi-zone},
    description={A chemical evolution model where a galaxy is divided into several zones, each with different stars, gas, and properties. }
}

\newglossaryentry{onezone}{name={one-zone},
    description={A chemical evolution model where the gas is all the same composition, i.e. neglecting spatial variations.}
}

\newglossaryentry{alpha}{name={$\alpha$-element},
    description={Elements which are made up of $\alpha$-particles (He nuclei) throuch the triple-$\alpha$ process (see Eq.~\ref{eq:triple_alpha}). Essentially light, even-numbered elements like O, Mg, and Na.}
}

\newglossaryentry{fepeak}{name={Fe-peak elements},
    description={Fe and nearby elements, produced in \gls{cc} and \gls{ia}.}
}

\newglossaryentry{sproc}{name={s-process elements},
    description={Elements produced through slow neutron captures, typically in \agb\ stars.}
}

\newglossaryentry{rgb}{name={red giant branch},
    description={Red giant branch stars are stars that have completed hydrogen core burning and have expanded in size. See Appendix \ref{sec:stel_evo}.}
}

\newglossaryentry{tdu}{name={third dredge up},
    description={Third dredge up occurs inside \gls{agb} stars. During each thermal pulse, material is {\it dredged up} from the core, changing the chemical abundances of the stellar atmosphere. (While nominally called {\it third dredge up}, there are typically several third dredge ups.). See Appendix \ref{sec:stel_evo}.}
}

\newglossaryentry{fdu}{name={first dredge up},
    description={First dredge up occurs when a \gls{lowmass} enters the \gls{rgb} phase. Material from the core is brought to the surface, increasing N and decreasing C abundances. See Appendix \ref{sec:stel_evo}.}
}

\newglossaryentry{hbb}{name={hot bottom burning},
    description={Hot bottom burning occurs inside \agb\ stars. The base of the convective envelope becomes hot enough for CNO burning to initiate.}
}

\newglossaryentry{subgiant}{name={subgiant},
    description={A star in the process of leaving the main sequence and becoming a \gls{rgb}.}
}

\newglossaryentry{imfave}{name={\gls{imf}-averaged},
    description={Averaged over the initial-mass function ({\sc imf}). The {\imf}-weighted yield is the mass of the newly produced element divided by the mass of star formation for a single stellar population (see \ssp).}
}


\newglossaryentry{dtd}{name={delay time distribution},
    description={The distribution in time of when an element is produced 
    after a star formation event.}
}

\newglossaryentry{massloading}{name={mass loading},
    description={The strength of outflows relative to star formation. See also $\eta$. }
}

\newglossaryentry{lowalpha}{name={low-$\alpha$},
    description={The low-$\alpha$ sequence, as described by Eq.~\ref{eq:high_alpha}. }
}

\newglossaryentry{highalpha}{name={high-$\alpha$},
    description={The high-$\alpha$ sequence, as described by Eq.~\ref{eq:high_alpha}. }
}


\newglossaryentry{lowmass}{name={low-mass star},
    description={Stars with masses $\lesssim 8\,M_\odot$ which end life as a white dwarf. }
}

\newglossaryentry{highmass}{name={high-mass star},
    description={Stars with masses $\lesssim 8\,M_\odot$, which end as a neutron star, black hole, or supernovae.}
}

\newglossaryentry{insideout}{name={{\it insideout}},
    description={Our fiducial star formation history. The rate of star formation is highest towards the center of the galaxy and at earlier times. See Eq.~\ref{eq:inside_out}.}
}

\newglossaryentry{cno}{name={CNO cycle},
    description = { A proton-fusion cycle which occurs in \gls{rgb} stars 
        consisting of a chain of proton captures releasing a He nucleus ($\alpha$-particle) and energy. See Eq.~\ref{eq:cno_cycle}.
}
}



\usepackage[single]{accents}

% useful commands
% below are for statement
\newcommand{\schwa}{ə}
\newcommand{\eng}{\ng{}}
%\newcommand{\overcomma}[1]{${\rm \accentset{\mbox{\normalfont ,}}{#1}}$}
\newcommand{\Lekwnen}{L\schwa{}\overcomma{k}\textsuperscript{w}\schwa{}\ng{}\schwa{}n}
\newcommand{\skipline}{\vspace{\baselineskip}}

\def\overcomma#1{%
  \setbox0=\hbox{#1}% Save the base letter in box0.
  \dimen0=\wd0       % Get its width.
  \vbox{%
    \hbox to \dimen0{\hfil,\hfil}% Create an hbox of the same width with a centered comma.
    \nointerlineskip            % Prevent extra vertical space.
    \box0                       % Place the base letter.
  }%
}

% for pandoc
\newcommand{\tightlist}{%
  \setlength{\itemsep}{0pt}\setlength{\parskip}{0pt}}



\newcommand{\about}[1]{${\sim} #1$}


\title{
    The extended light profiles of the Sculptor and Ursa Minor dwarf galaxies: Tides or nature?
}
\author{Daniel A. Boyea}


\date{\today}


\begin{document}
% \linespread{1.25}

\pagenumbering{roman}

\makeatletter
\begin{titlepage}
   \begin{center}
       \textbf{\large \@title}\\
        \skipline

        by\\
        \skipline

       \@author\\
       B.Sc. The Ohio State University, 2023\\
       \vspace*{3\baselineskip}
    A Thesis Submitted in Partial Fulfillment of the Requirements for the Degree of\\
    \skipline

    MASTER OF SCIENCE \\
    \skipline

    in the Department of Physics and Astronomy\\
       \vfill
       {\small
       ©\@author, 2025\\
       University of Victoria\\
   }
   \skipline
       {\small
       All rights reserved. This thesis may not be reproduced in whole or in part,
   by photocopy or other means, without the permission of the author.} \\
\skipline
   \end{center}
We acknowledge and respect the \Lekwnen{} (Songhees and X\textsuperscript{w}seps\schwa{}m/
Esquimalt) Peoples on whose territory the university stands, and the
\Lekwnen{} and ${\underaccent{\bar}{\rm W}}$S\'ANE\'C Peoples whose historical relationships with the
land continue to this day. 
\end{titlepage}

\addtocounter{page}{1}


\addcontentsline{toc}{chapter}{Supervisory Committee}
\begin{centering}
\textbf{\@title}\\
\skipline
by\\
\skipline
\@author\\
B.Sc. The Ohio State University, 2023\\
\vspace*{3\baselineskip}
\end{centering}

\subsection*{Supervisory Committee}
\skipline

Dr. Julio Navarro, Supervisor\\
Department of Astronomy \\
\skipline

\noindent Dr. Kim Venn\\
Department of Astronomy\\

\skipline

\noindent External Examiner \\
Outside Department

% Abstract of the paper
\chapter*{Abstract}
\addcontentsline{toc}{chapter}{Abstract}
% context
Sculptor and Ursa minor are some weird bois.


\chapter*{Acknowledgements}
\addcontentsline{toc}{chapter}{Acknowledgements}

%Julio Navarro, my supervisor. Rapha and other collegues

%Friends and family and collegues (fill in later).


This work has made use of data from the European Space Agency (ESA) mission
{\it Gaia} (\url{https://www.cosmos.esa.int/gaia}), processed by the {\it Gaia}
Data Processing and Analysis Consortium (DPAC,
\url{https://www.cosmos.esa.int/web/gaia/dpac/consortium}). Funding for the DPAC
has been provided by national institutions, in particular the institutions
participating in the {\it Gaia} Multilateral Agreement. 
\citep{gaiacollaboration+2016, gaiacollaboration+2023}.


This research made use of hips2fits,\footnote{https://alasky.cds.unistra.fr/hips-image-services/hips2fits} a service provided by CDS.


The Digitized Sky Surveys were produced at the Space Telescope Science Institute under U.S. Government grant NAG W-2166. The images of these surveys are based on photographic data obtained using the Oschin Schmidt Telescope on Palomar Mountain and the UK Schmidt Telescope. The plates were processed into the present compressed digital form with the permission of these institutions.

The National Geographic Society --- Palomar Observatory Sky Atlas (POSS-I) was made by the California Institute of Technology with grants from the National Geographic Society.

The Second Palomar Observatory Sky Survey (POSS-II) was made by the California Institute of Technology with funds from the National Science Foundation, the National Geographic Society, the Sloan Foundation, the Samuel Oschin Foundation, and the Eastman Kodak Corporation.

The Oschin Schmidt Telescope is operated by the California Institute of Technology and Palomar Observatory.

The UK Schmidt Telescope was operated by the Royal Observatory Edinburgh, with funding from the UK Science and Engineering Research Council (later the UK Particle Physics and Astronomy Research Council), until 1988 June, and thereafter by the Anglo-Australian Observatory. The blue plates of the southern Sky Atlas and its Equatorial Extension (together known as the SERC-J), as well as the Equatorial Red (ER), and the Second Epoch [red] Survey (SES) were all taken with the UK Schmidt.

All data are subject to the copyright given in the copyright summary. Copyright information specific to individual plates is provided in the downloaded FITS headers.

Supplemental funding for sky-survey work at the ST ScI is provided by the European Southern Observatory.




%% Lists

\tableofcontents
\listoffigures
\listoftables

%\chapter*{Symbols}
%
%
%\setlength{\tabcolsep}{0pt}
%\begin{longtable}{p{0.2\textwidth} p{0.8\textwidth}}
%$\rho$ & 3-dimensional mass density. \\
%
%$\Sigma$ & 2-dimensional density \\
%\end{longtable}
%
%\setlength{\tabcolsep}{6pt}
%\renewcommand*{\arraystretch}{1}
%
%
%\printglossary[type=\acronymtype,nonumberlist]
%\printglossary
%
%
\newpage

\pagenumbering{arabic}



%%%%%%%%%%%%%%%%%%%%%%%%%%%%%%%%%%%%%%%%%%%%%%%%%%

%%%%%%%%%%%%%%%%% BODY OF PAPER %%%%%%%%%%%%%%%%%%

\chapter{Introduction}\label{sec:introduction}
Dwarf galaxies host, in many ways, the most extreme galactic
environments in the universe. These galaxies are typically defined to be
fainter than the Large Magellanic Cloud (LMC), with \(M_V \gtrsim -18\)
or similarly \(M_\star \lesssim 10^9 M_\odot\)
\citep[e.g.,][]{mcconnachie2012, bullock+boylan-kolchin2017}. Because
the galaxy luminosity function increases towards fainter objects, dwarfs
are the most numerous of galaxies
\citep[e.g.,][]{blanton+2005, mao+2021}. Dwarf galaxies are also highly
\emph{dark-matter dominated}, with mass to light ratios which may exceed
1000 \(M_\odot/ L_\odot\) \citep[implying \(\sim1000\) times more dark
matter than stellar mass, e.g.,][]{simon+geha2007, hayashi+2023}.

Except for the Magellanic Clouds, most dwarf galaxy satellites of the
Milky Way (MW) are \emph{quenched}, with little to no recent star
formation \citep[e.g.,][]{weisz+2014}. Indeed, most faint MW satellites
contain stellar populations which are \emph{relics} from the early
universe, consisting of many of the oldest and most metal-poor stars
known \citep{simon2019}. Understanding the properties of dwarf galaxies
thus has implications across astronomy, from small-scale cosmological
structure formation to the origins of the first stars.

In this Chapter, we first describe the general observed properties of
local dwarf galaxies. Next, we summarize our understanding of the
cosmological origin of dwarf galaxies. We later review recent
advancements and pending questions concerning dwarf galaxies, and
introduce the puzzle posed by the extended stellar density profiles of
Sculptor and Ursa Minor. Then, we discuss the theory of tidal evolution.
We end with a brief roadmap to the remainder of this dissertation.

\section{Observations of dwarf
galaxies}\label{observations-of-dwarf-galaxies}

Dwarf galaxies have long raised conundrums for theories of galaxy
formation. The discovery of Fornax and Sculptor in 1938
\citep{shapley1938}\footnote{Technically, the Large and Small Magellanic
  Clouds (LMC, SMC) are also classified as dwarf galaxies, but these
  were likely always known to humans at southern latitudes.}, with no
known analogues at the time, already presented an enigma. H. Shapley
presented these dwarfs as a new type of \emph{stellar system} resembling
the Magellanic Clouds and globular clusters but did not attempt to
speculate on their nature. While dwarf galaxies were soon understood to
be galaxies based on the inferred luminosities and sizes, their exact
nature remained unclear for decades
\citep[e.g.,][]{hodge1971, gallagher+wyse1994}.

The earliest spectroscopic work hinted that dwarf galaxies may contain
substantial am-ounts of dark matter. From velocity dispersion
measurements for dwarf spheroidal (dSph) galaxies, inferred
mass-to-light ratios were at least 10 times larger than for globular
clusters \citep[e.g.,][]{aaronson1983, aaronson+olszewski1987}. While
uncertain initially, these values were later corroborated with larger
and more precise samples \citep[e.g.,][]{hargreaves+1994}. At the time,
several theories were proposed to explain these unusually high
mass-to-light ratios. Examples include: ongoing tidal disruption
inflating inferred velocity dispersions
\citep[e.g.,][]{kuhn+miller1989}, the presence of massive central black
holes \citep[e.g.,][]{strobel+lake1994}, or modified theories of gravity
\citep{milgrom1995}. Over time, a consensus developed where the high
mass-to-light ratios of dwarf galaxies were due to the presence of a
massive dark matter halo
\citep[e.g.,][]{dekel+silk1986, wechsler+tinker2018}. Since then, the
properties of dwarf galaxies have played an increasingly important role
in our understanding of the clustering of dark matter on small scales
\citep[e.g.,][]{bullock+boylan-kolchin2017, sales+2022}.

Today, a common definition for a (dwarf) galaxy is that of a
gravitationally bound stellar system with dark matter.\footnote{Or, more
  generally, systems inconsistent with Newtonian dynamics of visible
  matter alone \citep{willman+strader2012}.} In contrast, star clusters
(such as globular clusters) have no clear evidence for dark matter. The
boundary between these two classes blurs for faint, compact stellar
associations. Systems with characteristics of both globular clusters and
dwarf galaxies are known as ``ambiguous'' systems
\citep[e.g.,][]{smith+2024}.

Dwarf galaxies span a large range of sizes, luminosities, and
morphologies. Broadly, there are two morphological classes of dwarf
galaxies, although additional types exist outside the Milky Way system.
\textbf{Dwarf irregular galaxies} (dIrr) are defined by complex
morphologies and recent star formation. Around the Milky Way, only the
two brightest satellites, the Magellanic Clouds, are classified as dwarf
irregulars. Fig.~\ref{fig:galaxy_images} shows the Large Magellanic
Cloud (LMC), where most stars are in a rotationally-supported disk (seen
nearly face-on) with a prominent bar. In contrast, \textbf{dwarf
spheroidals (dSph)}\footnote{While formally the dwarf galaxy names we
  discuss contain ``dwarf spheroidal'' (dSph), e.g., Sculptor dSph, we
  omit this suffix for brevity. Additionally, the 12 classical dwarf
  satellites of our Galaxy are (in order of decreasing luminosity)
  Sagittarius, Fornax, Leo I, Sculptor, Antlia II, Leo II, Carina,
  Draco, Ursa Minor, Canes Venatici I, Sextans I, and Crater II. The
  only post-digital sky survey additions are Antlia II, Crater II, and
  Canes Venatici I.} are old, approximately spheroidal,
dispersion-supported, non-star-forming, and gas-poor systems. These
galaxies can be further divided into classical and ultra-faint dwarfs.
\textbf{Classical dSphs} have \(M_V \lesssim -7.7\) or
\(M_\star \gtrsim 10^5\,\Mo\). All Milky Way dSph galaxies discovered
before digital sky surveys are classical dwarfs, and these systems
remain among the best studied. \textbf{Ultra-faint dwarf galaxies},
fainter than \(M_V \approx -7.7\), have minuscule stellar masses and
very metal-poor stellar populations \citep[see the review
by][]{simon2019}. Altogether, known dwarf galaxies span more than 15
absolute magnitudes, or over 6 decades in stellar mass.

Most well-studied dwarf galaxies lie in the vicinity of the Milky Way,
the \emph{Local Group} of galaxies. The Local Group is defined as the
group consisting of galaxies within \(\sim 1\) Mpc from the MW-Andromeda
centre \citep[e.g.,][ and references therein]{mcconnachie2012}. Today,
we know that the Local Group is teeming with dwarfs, many of which are
satellites of either the MW or Andromeda (M31).
Fig.~\ref{fig:mw_satellite_system} shows the MW satellite system,
including dwarf galaxies, globular clusters, and ambiguous systems. This
nearby population of dwarf galaxies is amenable to resolved studies
aimed at investigating their detailed history and structure.

\begin{figure}
\centering
\includegraphics[width=1\linewidth,height=\textheight,keepaspectratio]{figures/galaxy_pictures.png}
\caption[Images of dwarf galaxies]{Images of the LMC \citep[Digitized
Sky Survey II,][]{lasker+1996}, Fornax \citep[DES DR2,][]{abbott+2021},
Sculptor (DES DR2), and Ursa Minor \citep[UNWISE,][with \textit{Gaia}
point sources
overplotted]{lang2014, meisner+lang+schlegel2017, meisner+lang+schlegel2017a}.\footnotemark{}
The grey ellipse represents the half-light radius for the three dwarf
spheroidals, and the luminosity is derived from the absolute V-band
magnitude of each galaxy.}\label{fig:galaxy_images}
\end{figure}
\footnotetext{Created with hips2fits
  (https://alasky.cds.unistra.fr/hips-image-services/hips2fits), a
  service provided by CDS.}

\begin{figure}
\centering
\includegraphics[width=1\linewidth,height=\textheight,keepaspectratio]{figures/mw_satellites_1.jpg}
\caption[The on-sky distribution of Milky Way satellites]{The location
of MW satellites on the sky. We label the classical dwarf galaxies
(green diamonds), fainter dwarfs (blue squares), globular clusters
(orange circles), and ambiguous systems (pink open hexagons). Globular
clusters are more centrally concentrated, but dwarf galaxies are
preferentially found away from the MW disk. Sculptor and Ursa Minor are
highlighted as two dwarfs we study later. The background image is from
ESA/Gaia/DPAC.\footnotemark{} Dwarf galaxies (confirmed), globular
clusters, and ambiguous systems are from the \citet{pace2024} catalogue
(version 1.0.3).}\label{fig:mw_satellite_system}
\end{figure}
\footnotetext{https://www.esa.int/ESA\_Multimedia/Images/2018/04/Gaia\_s\_sky\_in\_colour2}

\section{Dwarf galaxies in a cosmological
context}\label{dwarf-galaxies-in-a-cosmological-context}

We only understand a fraction of the universe's composition. The leading
theory of cosmology, Lambda Cold Dark Matter (\LCDM{}), posits that the
universe is composed of about 68\% dark energy (\(\Lambda\)), 27\% dark
matter (DM), and 5\% baryons\footnote{In a classic astronomer's
  corruption of jargon, \emph{Baryons} here means baryons and leptons,
  i.e., protons, neutrons, and electrons.}
\citep{planckcollaboration+2020}. While the composition of dark matter
and dark energy remains elusive, we know their general properties. Dark
energy drives the acceleration of the expansion of the universe on large
scales. We do not discuss dark energy here---it does not substantially
affect the Local Group today. Dark matter, instead, makes up the vast
majority of mass in galaxies. Typically, galaxies have baryonic-to-dark
matter ratios of between 1:5 to beyond 1:1000 for faint dwarf galaxies
\citep[e.g.,][]{hayashi+2023}.

In \LCDM{}, dark matter is assumed to interact only gravitationally.
Light and matter pass through dark matter unimpeded---in this sense,
dark matter is transparent. Dark matter is also assumed to be
\emph{cold}, i.e., with typical velocities much smaller than the speed
of light in the early universe. If dark matter is cold, then it should
condense on all scales, from the size of galaxy clusters to smaller than
the faintest dwarf galaxies. Implications of dark matter properties
include cosmological structure, galaxy formation, and galaxy
interactions.

\subsection{\texorpdfstring{Structure formation in
\LCDM{}}{Structure formation in }}\label{structure-formation-in}

The very early universe was almost featureless. Our earliest
observations of the universe stem from the cosmic microwave background
(CMB)---displaying a nearly uniform, isotropic blackbody emission
\citep[e.g.,][]{ryden2016}. But tiny perturbations in the CMB,
temperature fluctuations of 1 part in 100,000, reveal the underlying
seeds of large-scale cosmological structure. In an expanding universe,
gravitational instability makes CDM overdensities grow and collapse
hierarchically onto larger structures. Initially, baryonic matter was
coupled to radiation and resisted collapse. Dark matter, only influenced
by gravity, freely collapsed into the first structures. Mass
perturbations sufficiently small and overdense become self-gravitating
structures, known as \emph{halos} \citep[e.g.,][]{galaxiesbook}. After
recombination, where electrons combined with atomic nuclei to form
atoms, baryons decoupled from radiation and fell into dark matter halos,
where they condensed at the centre through radiative energy losses. The
densest pockets of baryons later formed the first stars and galaxies.

Dark matter halos and their associated galaxies rarely evolve in
isolation. Instead, \LCDM{} structure formation is \emph{hierarchical}.
Small dark matter halos collapse first and hierarchically merge into
progressively larger halos
\citep[e.g.,][]{white+rees1978, blumenthal+1984, white+frenk1991}.
Hierarchical assembly is evident through the large-scale structure of
the universe, remnants of past mergers within the Milky Way, and tidal
disruption of dwarf galaxies and their streams around nearby galaxies.

Small-scale structure formation is sensitive to deviations from \LCDM{}
\citep[e.g.,][]{bechtol+2022}. One key prediction of \LCDM{} is that
mass perturbations are expected to exist on all scales, and are largest
on the smallest scales, so we would expect the formation of halos on all
scales. Many alternative models, such as warm dark matter, may smooth
out small-scale features and reduce the abundance of small halos or
change their structure \citep[e.g.,][]{lovell+2014}. Dwarf galaxies,
which occupy the smallest dark matter halos capable of hosting a
luminous component, are promising probes into the behaviour of dark
matter on small scales.

\subsection{The structure of cold dark matter halos}\label{sec:NFW}

In \LCDM{} cosmological simulations, dark matter halos are remarkably
self-similar. \citet{NFW1996, NFW1997} observe that the
spherically-averaged density profiles \(\rho(r)\) are universally well
described by a two-parameter law,
\begin{equation}\protect\phantomsection\label{eq:nfw}{
\rho/\rho_s= \frac{1}{(r/r_s)(1+r/r_s)^2},
}\end{equation} where \(r_s\) is a scale radius and \(\rho_s\) a scale
density. This profile, known by the author's initials NFW, has shown
remarkable success at describing \LCDM{} halos across several orders of
magnitude in mass. NFW profiles are \emph{cuspy}, where the density
rises like \(\rho \sim 1/r\) at small radii \(r \ll r_s\). The steepness
of the density profile increases gradually with radius, and at large
radii the density falls off like \(\rho \sim 1/r^3\). The solid blue
curve in Fig.~\ref{fig:nfw_density} shows an example NFW halo.

The total mass of an NFW profile formally diverges, so halo masses are
conventionally defined using an overdensity criterion. The virial mass,
\(M_{200}\), is defined as the mass within a radius, \(r_{200}\),
containing a mean enclosed density 200 times\footnote{For the collapse
  of a uniform spherical density, the virialized overdensity would be
  \(\Delta = 18\pi^2\approx 178\) for a critical universe
  \(\Omega_m = 1\). This is commonly rounded to \(\Delta = 200\). While
  this parameter may be closer to \(\Delta \approx 100\) for our
  universe, \(\Delta\) also increases with redshift \citep[see, e.g.,
  eq. 6 from][]{bryan+norman1998}.} the critical density of the
universe: \begin{equation}{
M_{200} =200\,\frac{4\pi}{3} \ r_{200}^3\ \rho_{\rm crit}, \qquad {\rm where} \quad \rho_{\rm crit}(z) = 3H(z)^2 / 8\pi G,
}\end{equation} and \(H(z)\) is the Hubble constant, which depends on
redshift. Another way of characterizing NFW halos is through the
concentration parameter, \(c=r_{200} / r_s\), which describes how the
characteristic radial scale of the halo compares to the virial radius.
Using this parameter, the scale density is a function of \(c\) alone,
\(\rho_s = (200/3)\,\rho_{\rm crit} c^3 / [\log(1+c) - c/(1+c)]\)
\citep{NFW1996}.

An equivalent, alternative characterization of NFW halos uses their
circular velocity profiles. The circular velocity,
\(\vcirc(r) = \sqrt{G M(r) / r}\), reaches a maximum \(\vmax\) at radius
\(\rmax \approx 2.16258\,r_s\). \(\vmax\) and \(r_{\rm max}\), like
\(M_{200}\) and \(c\), fully specify an NFW halo.

The two parameters of an NFW profile are not independent. Lower-mass
dark matter halos typically collapse earlier, when the universe was
denser. As a result, low mass subhalos tend to be more concentrated
\citep[e.g.,][]{NFW1997}. The relationship between \(M_{200}\) and c, or
the mass-concentration relation, describes the mean trend of
concentration with mass or, equivalently, the dependence of \(\vmax\) on
\(\rmax\) \citep[e.g.,][]{bullock+2001, ludlow+2014}. The left panel of
Fig.~\ref{fig:smhm} illustrates the present-day mass-concentration from
\citet{ludlow+2016}. While concentration tends to decrease with
increasing mass, the relation has substantial scatter. Other parameters,
such as the halo spin or shape, may affect the scatter of the
mass-concentration relation, but their effect is typically expected to
be small \citep{navarro+2010, dicintio+2013, dutton+maccio2014}.

\begin{figure}
\centering
\includegraphics[width=3.5in,height=\textheight,keepaspectratio]{figures/example_density_profiles.png}
\caption[Example dark matter and stellar density profiles]{Density
profiles in log 3D density versus log 3D radius for stars and dark
matter in a Fornax-like galaxy. The dark matter is more extended and
massive than the star across the entire galaxy. This galaxy has a
stellar mass \(M_\star \approx 2.5\times10^7\,\Mo\) with half-light
radius \(0.65\,\kpc\) following a projected exponential surface
brightness profile (deprojected into 3D density). The corresponding
cosmological-mean halo has \(\vmax=40\,\kpc\) and \(\rmax=8\,\kpc\), or
\(M_{200} = 1\times10^{10}\,\Mo\) and
\(c=12.5\).}\label{fig:nfw_density}
\end{figure}

\begin{figure}
\centering
\pandocbounded{\includegraphics[keepaspectratio]{figures/cosmological_means.pdf}}
\caption[Cosmological mass-concentration and stellar mass-halo mass
relations]{\textbf{Left} The NFW halo concentration \(c=r_{200} / r_s\)
as a function of virial mass \(M_{200}\). The solid line with
1-\(\sigma\) shaded region is the mass-concentration relation from
\citet{ludlow+2016} for \(z=0\). \textbf{Middle}: Equivalent to the left
except in terms of the halo maximum circular velocity, \(\vmax\), and
radius where the velocity is maximized, \(\rmax\). \textbf{Right}
Stellar mass (top) as a function of maximum circular velocity. The solid
line with the 1-\(\sigma\) shaded region is the relation from
\citet{fattahi+2018} with scatter points simulated central galaxies from
\apostle{} in \citet{fattahi+2018}. The pink star illustrates the
location of the Fornax galaxy, whose density profiles are shown in
Fig.~\ref{fig:nfw_density}.}\label{fig:smhm}
\end{figure}

\subsection{\texorpdfstring{Galaxy formation in
\LCDM{}}{Galaxy formation in }}\label{sec:galaxy_formation}

The observed abundance of galaxies may be compared with the abundance of
\LCDM{} halos to derive constraints regarding which galaxies inhabit
which halos. One simple technique, dubbed ``abundance matching,''
assumes a tight relation between the stellar mass of a galaxy and the
mass of the halo it inhabits
\citep{li+white2009, moster+naab+white2013}.

The right-hand panel of Fig.~\ref{fig:smhm} shows the stellar mass
versus halo mass relation (SMHM, with halo mass represented by
\(\vmax\)) predicted by \LCDM{} cosmological hydrodynamical simulations
of Local Group analogues from the \apostle{} project
\citep{sawala+2016}.\footnote{\apostle{} simulated Local Group analogues
  in a \LCDM{} cosmological context with the hydrodynamical setup from
  the \eagle{} simulations \citep{crain+2015, schaye+2015}.} While there
is some scatter, the range of predicted \(\vmax\) is fairly narrow
across \(\sim 4\) decades in stellar mass. This figure indicates that
the SMHM relation becomes increasingly steep in the dwarf galaxy
regime---many dwarf galaxies are formed in halos of similar mass.
Because lower mass galaxies have shallower potential wells, the
energetic output of evolving stars and possibly supermassive black holes
(i.e., ``feedback'') becomes more effective at removing gas.
Reionization additionally suppresses late star formation in the faintest
galaxies. As a result, the resulting stellar mass of a dwarf galaxy is
highly sensitive to the details of halo assembly and evolution.

In \LCDM{} galaxy formation, the majority of mass in a dwarf galaxy
comes from the extended dark matter halo. Fig.~\ref{fig:nfw_density}
shows an example exponential stellar component for the Fornax dwarf
galaxy with its surrounding dark matter NFW halo \citep[with parameters
matching][]{ludlow+2016, fattahi+2018}. Where the stars are densest, the
dark matter remains nearly an order of magnitude higher in density.
Stars make a small contribution to the gravitational structure of dwarf
galaxies---indeed, stars are reasonably approximated as tracer particles
of the underlying dark matter halo. In addition, the stellar component
is typically confined to the central regions of the dark matter halo.

Several factors affect the SMHM trend, including environment, assembly
history, tidal effects, and the details of galaxy formation. For
example, effects like ram-pressure stripping (removal of gas in the
dwarf galaxy due to pressure from the host's circumgalactic medium) and
tidal removal of gas cause star formation to quench
\citep[e.g.,][]{christensen+2024}. Additionally, the time of formation
(relative to reionization) can influence the resulting stellar content
\citep{kim+2024}. Finally, Galactic tides reduce both the dark matter
and stellar mass but in different amounts, adding additional scatter to
the SMHM trend for satellites \citep[e.g.,][]{PNM2008, fattahi+2018}.
Understanding the effects of tides on Local Group dwarf galaxies may
help us understand where and how these galaxies formed in a cosmological
context.

\subsection{Challenges and questions concerning dwarf
galaxies}\label{challenges-and-questions-concerning-dwarf-galaxies}

Observations of dwarf galaxies have been the origin of several disputes
or \emph{small-scale} problems for \LCDM{} \citep[see reviews
by][]{bullock+boylan-kolchin2017, sales+2022}. For example, the mismatch
between the number of dwarf galaxies and the predicted abundance of
\LCDM{} halos has been known as the \emph{missing satellites problem}.
Additionally, several observations suggest that some dwarf galaxies,
although not all, possess dark matter ``cores,''
\citep[e.g.,][]{moore1994, adams+2014, oh+2015, walker+penarrubia2011, read+walker+steger2019},
contrary to the expectation from \LCDM{} of ``cuspy'' inner dark matter
profiles \citep{NFW1996, NFW1997}. As a result, alternative forms of
dark matter have been advocated as solutions, such as Warm or
Self-Interacting Dark Matter.

However, some of these tensions have eased as a result of improved
understanding of baryonic physics. For example, recent hydrodynamic
simulations, in particular, have shown that strong feedback can produce
dark matter cores
\citetext{\citealp[e.g.,][\citet{tollet+2016}]{navarro+eke+frenk1996}; \citealp{fitts+2017}; \citealp{benitez-llambay+2019}; \citealp{orkney+2021}}.
Several open questions remain, concerning, e.g., the nearly planar
distribution of luminous Milky Way satellites, the details of the sizes
and rotation curves of dwarf galaxies, and the existence and nature of
stellar halos in dwarf galaxies \citep[e.g.,][]{sales+2022}. Altogether,
the numerous past and ongoing challenges for \LCDM{} in the dwarf galaxy
regime illustrate the opportunity for dwarf galaxies to our the
understanding of galaxy formation and dark matter physics.

\section{The structure of nearby dwarf
galaxies}\label{the-structure-of-nearby-dwarf-galaxies}

\subsection{\texorpdfstring{The \emph{Gaia}
mission}{The Gaia mission}}\label{the-gaia-mission}

Since Local Group dwarfs are nearby, they are resolved into individual
stars, and therefore, we can study these galaxies on a star-by-star
basis. As a result, it is possible to measure the 3D velocity and
position of a star if we can measure its position, distance,
line-of-sight (LOS) velocity, and proper motion. Unfortunately,
determining distances and full 3D velocities is challenging. The most
direct measurement of distance, the parallax, requires precise tracking
of a star's sky position across a year. And while line-of-sight (LOS)
velocities are relatively easily determined from spectroscopy,
tangential velocities, derived from proper motions and distances, are
much more challenging. Typically, measuring proper motions requires
precise (\(\ll\) arcsecond) determinations of small changes in a star's
position over baselines of years to decades. The full 6D position and
velocity information for stars has, until recently, been known for only
a handful of stars.

Launched in 2013, \emph{Gaia} is a space-based, all-sky survey telescope
situated at the Sun-Earth L2 Lagrange point
\citep{gaiacollaboration+2016}. \emph{Gaia} has redefined astrometry,
providing photometry, positions, proper motions, and parallaxes for over
1 billion stars \citep{gaiacollaboration+2021}. While \emph{Gaia}
completed its space-based mission in 2025, two further data releases are
still expected.

Determining absolute parallax measurement is facilitated by the
observation that stars in different regions of the sky are affected by
parallax motion with different phases. By imaging two regions separated
by 106.5 degrees on the same focal plane, \emph{Gaia} measures changes
in the relative positions of stars across small and large angles.
Combining measurements from multiple epochs across several years, an
absolute all-sky reference frame is derived from which parallax and
proper motions are calculated. In addition to astrometry, \emph{Gaia}
measures photometry in the wide \emph{G} band (330--1050nm) and colours
from the blue photometer (BP, 330--680 nm) and red photometer (RP,
640--1050 nm). \emph{Gaia} additionally provides low-resolution BP-RP
spectra and radial velocity measurements of bright stars \citep[of
magnitudes \(G_{\rm RVS} < 16\),][]{gaiacollaboration+2016}. For our
work, \emph{Gaia}'s most relevant measurements are \(G\) magnitude,
\(G_{\rm BP} - G_{\rm RP}\) colour, \((\alpha, \delta)\) position, and
\((\mu_{\alpha*}, \mu_\delta)\) proper motions.\footnote{The proper
  motions \(\mu_\alpha\) and \(\mu_\delta\) are the apparent rates of
  change in right ascension, \(\alpha\), and declination, \(\delta\),
  typically in units of milli-arcsecond (mas) per year.
  \(\mu_{\alpha*} = \mu_\alpha \cos \delta\) corrects for projection
  effects in \(\alpha\).}

\subsection{\texorpdfstring{\emph{Gaia}'s impact on Milky Way
studies}{Gaia's impact on Milky Way studies}}\label{gaias-impact-on-milky-way-studies}

\emph{Gaia} has revolutionized our understanding of Milky Way structure.
For example, the 6D dynamical measurements and metallicities of MW stars
led to the (re)discovery of past mergers or Milky Way building blocks
like \emph{Gaia}-Sausage Enceladus
\citetext{\citealp[e.g.,][]{helmi+2018}; \citealp{belokurov+2018}; \citealp[but
see also][]{meza+2005}}, out-of-equilibrium structures like the
\emph{Gaia} snail \citep[e.g.,][]{antoja+2018}, and dynamical effects of
the Milky Way's spiral arms and the bar in the solar neighbourhood
\citep[ and references therein]{hunt+vasiliev2025}. In the Milky Way
halo, \emph{Gaia} has helped find and constrain numerous stellar streams
\citep{ibata+malhan+martin2019, bonaca+price-whelan2025}. Altogether,
\emph{Gaia} has revealed the hierarchical formation and complex,
evolving structure of our own Galaxy.

For Milky Way satellites, \emph{Gaia} has improved orbital analysis and
facilitated robust stellar membership determinations. Before
\emph{Gaia}, few galaxies had precisely measured proper motions
\citep[e.g., using the Hubble Space
Telescope,][]{piatek+2005, sohn+2017}. \emph{Gaia} allowed for some of
the first systematic and precise determinations of Milky Way satellite
proper motions
\citep{gaiacollaboration+2018, simon2018, fritz+2018, pace+li2019, MV2020a}.
While the proper motion uncertainty of a typical dwarf member star is
often large, by combining the proper motions of 100s or 1000s of stars
from \emph{Gaia}, precise average proper motion measurements can be
determined, sometimes only limited by \emph{Gaia}'s systematic error
floor \citep[e.g.,][]{MV2020a}. Proper motions have thus ushered in a
new era for MW satellite dynamical studies, where we can derive precise
orbits for any satellite, assuming a given MW potential. In addition,
\emph{Gaia} helps establish membership by filtering contaminating MW
foreground stars. By measuring parallaxes and/or proper motions, many
more background and foreground stars can be classified as non-members
\citep[e.g.,][]{battaglia+2022, jensen+2024}.

\subsection{Dwarf galaxy light profiles}\label{sec:exponential_profiles}

Projected luminosity/stellar density profiles efficiently characterize
the radial structure of a galaxy. At its most basic, light profiles
synthesize properties such as the shape, size, and orientation of a
dwarf galaxy. In addition, the details of a stellar density profile can
help interpret a galaxy's assembly and dynamical history
\citep[e.g.,][]{penarrubia+2009, lee+2018, querci+2025}. Note that for
resolved galaxies, these profiles are expressed in stellar count
densities instead of surface brightness.

Four different surface density laws are frequently used to parameterize
dwarf galaxy profiles: Exponential, Plummer, King, or Sérsic profiles
\citep[e.g.,][]{munoz+2018}. The exponential profile is perhaps the
simplest, defined in terms of the central surface density, \(\Sigma_0\),
and projected scale radius, \(R_s\):
\begin{equation}\protect\phantomsection\label{eq:exponential_law}{
\Sigma_{\rm exp} = \Sigma_0\exp(-R / R_s).
}\end{equation} This profile is also often applied to the radial light
distribution of galaxy disks
\citep{devaucouleurs1959a, freeman1970, kent1985}.

To fit globular cluster density profiles, \citet{plummer1911} proposed a
profile based on a self-gravitating polytrope,\footnote{where density
  and pressure are assumed to be related by a power law}
\begin{equation}{
\Sigma_{\rm Pl} = \frac{\Sigma_0}{(1 + (R/R_h)^2)^2},
}\end{equation} where \(\Sigma_0\) is the central surface density and
\(R_h\) is the projected half-light radius. Now mostly superseded by the
King profile for globular clusters, the Plummer model is still a good
fit to many dwarf spheroidals \citep[e.g.,][]{moskowitz+walker2020}.

The \citet{king1962} profile, also a fit to globular clusters, is also
used to describe dwarf galaxies, more so in older literature. Using
three parameters, a core radius \(R_c\), a truncation radius \(R_t\),
and a characteristic density, \(\Sigma_0\), the King profile may be
written as \begin{equation}{
\Sigma_{\rm K} = \Sigma_0\left(\frac{1}{\sqrt{1 + (R/R_c)^2}} - \frac{1}{\sqrt{1+(R_t/R_c)^2}}\right); \qquad \text{for } R<R_t
}\end{equation} and \(\Sigma_{\rm K}=0\) for \(R \geq R_t\). In much of
the older literature, \(R_t\) was interpreted as a ``tidal radius,''
after an analogous interpretation for globular clusters
\citep[e.g.,][]{hodge1961, IH1995}.

Finally, the \citet{sersic1963} profile represents a generalization of
an exponential profile and describes most dwarf galaxy light profiles
well. Typically parameterized in terms of a half-light radius \(R_h\),
the density at half-light radius \(\Sigma_h\), and a Sérsic index \(n\),
the profile's equation is \begin{equation}{
\Sigma_{\rm S} = \Sigma_h \exp\left[-b_n \,  \left((R/R_h)^{1/n} - 1\right)\right]
}\end{equation} where the coefficient \(b_n\) solves
\(\Gamma(2n) = 2\gamma(2n, b_n)\) with \(\Gamma\) the Gamma function and
\(\gamma\) the lower incomplete gamma function
\citep{graham+driver2005}. A Sérsic profile with \(n=1\) is equivalent
to an exponential profile, while \(n=4\) recovers
\citepos{devaucouleurs1948} profile for elliptical galaxies. Although a
Sérsic profile is less commonly applied to dwarf galaxies,
\citet{munoz+2018} advocate for the Sérsic profile since the added
flexibility allows more profiles to be fit with a single law.

While there are no clear theoretical preferences for any of these
profiles, exponential density profiles have been commonly used for dwarf
spheroidal galaxies. \citet{faber+lin1983} were among the first to
demonstrate that an exponential law is a reasonable empirical fit,
theorizing that dwarf spheroidals may have evolved from exponential disk
galaxies and maintained a similar light profile. Later,
\citet{read+gilmore2005} showed that exponential profiles may originate
from mass loss during the evolution of dwarf galaxies. Tides are a
possible mechanism for this transformation---the \emph{tidal stirring
hypothesis} \citep{mayer+2001a, klimentowski+2009}. However, the
theoretical origin of exponential disks is also unknown.\footnote{While
  studied for far longer, the exponential origin of disk galaxies is no
  better understood. Ideas range from scattering of stars
  \citep{elmegreen+struck2013, wu+2022} to angular momentum transport,
  to disk viscosity-driven radial gas flows
  \citep{lin+pringle1987, wang+lilly2022} or spherical collapse
  \citep{freeman1970}.}

Many subsequent photometric studies of dwarf spheroidal galaxies have
used exponential fits, finding that exponential and King profiles both
provide good descriptions in many cases
\citep{binggeli+sandage+tarenghi1984, mateo1998, mcconnachie+irwin2006, cicuendez+2018}.
More recently, \citet{moskowitz+walker2020} fit instead generalized
Plummer profiles, but most of their fits would be consistent with a
single-component exponential.\footnote{A single-component exponential
  profile is close to their ``Steeper'' profile in
  \citet{moskowitz+walker2020} over the range of their data. However,
  most of their galaxies have little evidence to prefer the Steeper or
  Plummer density profile.} As a result, it has become conventional to
assume an exponential density profile to describe dwarf galaxies in
theoretical or observational modelling
\citep[e.g.,][]{kowalczyk+2013, martin+2016, MV2020a, battaglia+2022}.

Dwarf galaxies outside the Local Group commonly follow exponential
profiles, but sometimes with modifications. For example, many
extragalactic dwarf elliptical, blue compact dwarf, and irregular dwarf
galaxies are better described with an exponential profile to which a
central cusp or nuclear region is added
\citep{caldwell+bothun1987, noeske+2003}. On the other hand, some
studies find an inner density decrement relative to exponentials
\citep[e.g.,][]{caldwell+1992, makarov+2012}, or that dwarfs are better
fit by two nested exponentials
\citep[e.g.,][]{aparicio+1997, graham+guzman2003, hunter+elmegreen2006, lee+2018}.
It is unclear how these conclusions apply to the dwarf spheroidals of
the Local Group.

Altogether, while there is some variation in the density profiles of
dwarf galaxies, an exponential is an excellent first-order
approximation. Typically, deviations from exponentials are in the
direction of a steeper outer cutoff or changes to the inner slope of a
dwarf galaxy (due, for example, to a nuclear star cluster). Flattened
density profiles in the outer regions are more unusual. Explaining in
detail the origin, similarity, and diversity of dwarf galaxy density
profiles is an open question for theories of dwarf galaxy formation and
evolution.

\subsection{The extended light profiles of Sculptor and Ursa Minor:
Hints of tidal signatures?}\label{sec:scl_umi_obs_tides}

Sculptor (Scl) and Ursa Minor (UMi) appear to be typical dwarf
spheroidal galaxies at first glance (see Fig.~\ref{fig:galaxy_images}).
Tables~\ref{tbl:scl_obs_props}, \ref{tbl:umi_obs_props} describe the
structural parameters of each galaxy. Sculptor, as the first discovered
classical dSph, is even described as a ``prototypical'' dSph
\citep[e.g.,][]{mcconnachie2012}.

However, many studies have speculated that Scl and UMi have been
influenced by the Milky Way's tidal field. Already,
\citet{innanen+papp1979} found RR Lyrae candidate Scl members
\citep[from][]{vanagt1978} out to \(180'\) in an elongated distribution,
speculating this to be tidal disruption. Later density profile
determinations noted Scl's elongation and apparent outer density excess
\citetext{\citealp{eskridge1988}; \citealp{IH1995}; \citealp{walcher+2003}; \citealp{westfall+2006}; \citealp[but
see also][]{coleman+dacosta+bland-hawthorn2005}}. These studies often
interpreted these features as evidence of tidal effects
\citep[e.g.,][]{walcher+2003} or sometimes a ``halo''of stars
surrounding the dwarf \citep{westfall+2006}.

UMi has attracted similar suspicions. \citet{hargreaves+1994} first
detected a velocity gradient in UMi, suggestive of tidal disruption.
Later, \citet{martinez-delgado+2001} found stars far beyond the
King-profile ``tidal radius,'' aligned with UMi's elongation, consistent
with tidally-stripping simulations in
\citet{gomez-flechoso+martinez-delgado2003}. \citet{palma+2003}
furthermore detected S-shaped isophotes and an extended population of
``extratidal'' stars.

Adding to the evidence, \citet{sestito+2023a, sestito+2023b} report a
``kink'' in the density profile of each galaxy. They spectroscopically
follow up distant stars, finding members as far as 6 and 12 half-light
radii from the centre of each dwarf. If these dwarfs had exponential
profiles, like Fornax, then these far-outlying stars should be much
rarer.

Sculptor and Ursa Minor are poorly described by an exponential profile.
The left panel of Fig.~\ref{fig:scl_umi_vs_fornax} shows the density
profiles of Sculptor, Ursa Minor, and Fornax (see
Section~\ref{sec:observations} for details on how these profiles are
measured). Compared to Fornax, both Sculptor and Ursa Minor show an
excess of stars outside \(\log R/R_h\approx 0.4\), implying densities
which exceed 100 times the density of the exponential fit at large
radii.

A goal of this work is to determine if tidal effects are indeed
responsible for the extended outer light profiles of Sculptor and Ursa
Minor. If tides cannot explain these features, these features may
instead be due to an extended stellar ``halo'' or second component of
the galaxy---suggestive of a complex star formation or assembly history.

\begin{figure}
\centering
\pandocbounded{\includegraphics[keepaspectratio]{./figures/scl_umi_fornax_exp.pdf}}
\caption[The extended stellar profiles of Sculptor and Ursa
Minor]{Surface density profiles of Sculptor (orange squares), Ursa Minor
(red triangles), and Fornax (green circles) scaled to their half-light
radius and the density at half-light radius (data described in
Section~\ref{sec:observations}). The solid black line is an exponential
profile (Eq.~\ref{eq:exponential_law}). Scl and UMi show a clear excess
over an exponential at large radii.}\label{fig:scl_umi_vs_fornax}
\end{figure}

\begin{table*}[t]
\centering
\caption[Observed properties of Sculptor]{Observed properties of Sculptor. References are: (1) Muñoz et al. (2018, Sérsic fit), (2) Tran et al. (2022, RR lyrae distance), (3) Alan W. McConnachie and Venn (2020b), (4) Arroyo-Polonio et al. (2024). }
\label{tbl:scl_obs_props}
\begin{tabular}{lll}
\toprule
parameter & value & Source\\
\midrule
$\alpha$ & $15.0183 \pm 0.0012^\circ$ & 1\\
$\delta$ & $-33.7186 \pm 0.0007^\circ$ & 1\\
distance modulus & $19.60 \pm 0.05$ & 2\\
distance & $83.2 \pm 2$ kpc & 2\\
$\mu_{\alpha*}$ & $0.099 \pm 0.002 \pm 0.017$ mas yr$^{-1}$ & 3\\
$\mu_\delta$ & $-0.160 \pm 0.002_{\rm stat} \pm 0.017_{\rm sys}$ mas yr$^{-1}$ & 3\\
LOS velocity & $111.2 \pm 0.3\ {\rm km\,s^{-1}}$ & 4\\
$\sigma_v$ & $9.7\pm0.2\ {\rm km\,s^{-1}}$ & 4\\
$R_h$ & $9.79 \pm 0.04$ arcmin & 1\\
ellipticity & $0.37 \pm 0.01$ & 1\\
position angle & $94\pm1^\circ$ & 1\\
$M_V$ & $-10.82\pm0.14$ & 1\\
\bottomrule
\end{tabular}
\end{table*}

\begin{table*}[t]
\centering
\caption[Observed properties of Ursa Minor]{Observed properties of Ursa Minor. References are: (1) Muñoz et al. (2018, Sérsic fit), (2) Garofalo et al. (2025, RR lyrae distance), (3) Alan W. McConnachie and Venn (2020a), (4) Pace et al. (2020), average of MMT and Keck results with systematic uncertainty from Appendix \ref{sec:extra_rv_models} discussion. }
\label{tbl:umi_obs_props}
\begin{tabular}{lll}
\toprule
parameter & value & Source\\
\midrule
$\alpha$ & $ 227.2420 \pm 0.0045$˚ & 1\\
$\delta$ & $67.2221 \pm 0.0016$˚ & 1\\
distance modulus & $19.23 \pm 0.11$ & 2\\
distance & $70.1 \pm 3.6$ kpc & 2\\
$\mu_\alpha*$ & $-0.124 \pm 0.004 \pm 0.017$ mas yr$^{-1}$ & 3\\
$\mu_\delta$ & $0.078 \pm 0.004_{\rm stat} \pm 0.017_{\rm sys}$ mas yr$^{-1}$ & 3\\
LOS velocity & $-245.9 \pm 0.3_{\rm stat} \pm 1_{\rm sys}$ km s$^{-1}$ & 4\\
$\sigma_v$ & $8.6 \pm 0.3$ & 4\\
$R_h$ & $11.62 \pm 0.1$ arcmin & 1\\
ellipticity & $0.55 \pm 0.01$ & 1\\
position angle & $50 \pm 1^\circ$ & 1\\
$M_V$ & $-9.03 \pm 0.05$ & 1\\
\bottomrule
\end{tabular}
\end{table*}

\section{Interpreting tidal signatures}\label{sec:tidal_theory}

The Local Group hosts several examples of ongoing tidal disruption. The
Magellanic stream, a massive, gas-rich feature emanating from the
Magellanic clouds, is believed to arise partially from the MW's tides
\citep{putman+1998, diaz+bekki2012, donghia+fox2016}. Other clear
examples of tidal streams include the Sagittarius stream, the Andromeda
Giant Southern stream, and the Tucana III stream
\citep[e.g.,][]{ibata+gilmore+irwin1994, ibata+2001, li+2018}. These
examples illustrate that hierarchical accretion remains an active
process. Interpreting such observations relies on simulations of tidal
disruption.

Cosmological simulations struggle to resolve tidal effects on dwarfs.
Since many dwarfs are near the resolution limit, they are vulnerable to
artificial disruption
\citep[e.g.,][]{vandenbosch+2018, santos-santos+2025}. To overcome
numerical challenges, idealized simulations model a single subhalo in an
analytic host potential, achieving excellent numerical convergence. For
example, the simulations we describe later reach three times higher
resolution than Aquarius \citep{springel+2008} with 400 times fewer
particles. Idealized simulations make numerous simplifications,
neglecting mergers, cosmological context, mass assembly, and often
baryonic physics
\citep[e.g.,][]{hayashi+2003, bullock+johnston2005, klimentowski+2009, ogiya+2019}.
Cosmological simulations appear to predict that tidal disruption is more
common than idealized simulations of MW dwarfs suggest, but the role of
numerics and assumptions in this discrepancy are unclear
\citep{panithanpaisal+2021, shipp+2023, riley+2024}. We shall use
idealized simulations here to assess tidal effects after the satellite's
infall into the MW halo.

Idealized simulations predict clear properties of tidally disrupting
dwarf spheroidal galaxies. Tidally stripped stars form \emph{tidal
streams}---stellar tails with a bulk velocity gradient
\citep[e.g.,][]{moore+davis1994, johnston+spergel+hernquist1995, read+2006}.
Most mass loss happens near pericentre, where tides are strongest.
However, the central structure of a dwarf galaxy often remains
undisturbed \citep{oh+lin+aarseth1995, piatek+pryor1995}. For instance,
NFW halos are also found to be resilient to full tidal disruption
\citep{EP2020}, but cored dark matter halos may disrupt fully and faster
\citep[e.g.,][]{penarrubia+2010, errani+2023a}.

To first order, tidal mass loss peels away the outer layers of a dwarf
galaxy in energy space.
\citet{drakos+taylor+benson2020, drakos+taylor+benson2022, amorisco2021}
showed that tidal effects are nearly entirely described as the removal
of particles above a truncation energy \citep[see
also][]{choi+weinberg+katz2009}. \citet{stucker+2023} generalized this
idea, creating a model for adiabatic tidal mass loss in an isotropic
tidal field. Their model explains the resilience of NFW halos against
full tidal disruption and the origin of well-defined ``tidal tracks''
\citep[as observed in][]{PNM2008, green+vandenbosch2019, EN2021}.

With precise orbital constraints and improved models of the Milky Way
potential, recent studies have continued to probe the dynamical
histories of individual dwarf galaxies.
\citet{battaglia+sollima+nipoti2015, borukhovetskaya+2022, dicintio+2024}
ran simulations tuned to Fornax, showing that this galaxy's stellar
component or globular clusters are likely not affected by tides.
Similarly, \citet{borukhovetskaya+2022a} analyzed Crater II, showing
that the present-day structure is challenging to reconcile with NFW
initial conditions and Galactic tides. Most relevantly,
\citet{iorio+2019} also tailored simulations to Scl, finding weak
Galactic tidal influence.

Building on this body of work, we will use idealized simulations to
understand the severity of tidal effects on Sculptor and Ursa Minor.

\subsection{Tidal and ``break'' radii}\label{sec:break_radii}

For a given orbit in a given potential, there are characteristic radii
which help gauge the effects of tides on a dwarf galaxy.

The \textbf{Jacobi radius} represents the approximate radius where stars
become unbound for a galaxy in a circular orbit around a host
galaxy.\footnote{The Jacobi radius was derived at least as early as
  \citet{laplace1798}. This radius also bears other names, such as the
  Hill radius \citep[from][]{hill1878}. Likely only named after Jacobi
  for the Jacobi integral \citep{jacobi1836}.} Calculated from an
approximation of the location of the \(L_1\) and \(L_2\) Lagrange
points, the Jacobi radius is where the mean density of the dwarf galaxy
is roughly three times the mean interior density of the host galaxy at
pericentre, or
\begin{equation}\protect\phantomsection\label{eq:r_jacobi}{
3\bar \rho_{\rm MW}(r_{\rm peri}) \approx \bar \rho_{\rm dwarf}(r_J),
}\end{equation} \citep[ eq. 7-84]{BT1987}. If \(r_J\) is comparable to
the visible extent of a galaxy, we should expect to find clear signs of
tidal disturbance. While strictly valid for circular orbits, assuming
\(r_{\rm peri}\) for the host-dwarf distance works as most stars are
lost during pericentric passages.

We also use the \textbf{break radius} as defined in
\citet{penarrubia+2009}, marking the outermost radius within which the
dwarf has been able to achieve equilibrium after pericentric passage in
a highly eccentric orbit. The break radius \(r_{\rm break}\) is
proportional to the velocity dispersion, \(\sigma_v\), and time elapsed
since pericentre, \(\Delta t\),
\begin{equation}\protect\phantomsection\label{eq:r_break}{
r_{\rm break} = C\,\sigma_{v}\,\Delta t
}\end{equation} where the empirical constant is \(C \approx 0.55\).
\(r_{\rm break}\) describes where the dynamical timescale becomes longer
than the time since the perturbation, i.e., the radius within which the
galaxy is dynamically relaxed.

\subsection{A simple tidal simulation}\label{a-simple-tidal-simulation}

To illustrate the effects of tides on an NFW halo due to a massive host,
we consider a toy model. We evolve an N-body realization of an NFW
subhalo with \(\rmax=5\,\kpc\) and \(\vmax = 27\,\kms\) orbiting a
static NFW host halo with \(\rmax=25\,\kpc\) and
\(\vmax = 207.4\,\kms\). These choices are motivated by the inferred
structure and masses of dSphs and the Milky Way, respectively. We
simulate the stellar component by assigning stellar weights to each
particle based on the relative densities of the distribution functions
in energy space (see Section~\ref{sec:painting_stars}). The stars
initially follow a 2D exponential profile with a scale radius
\(R_s=0.25\,\kpc\) and total mass \(5\times10^5\,\Mo\), embedded within
the inner dark matter halo. We start the model at apocentre on an orbit
with a pericentre of \(20\,\kpc\) and apocentre of \(100\,\kpc\). See
Section~\ref{sec:methods} for a more detailed description of our
numerical setup.

Fig.~\ref{fig:idealized_break_radius} illustrates the properties of this
idealized simulation at the second apocentre, as seen from the centre of
the host. The projected density of stars is relatively undisturbed and
spherical in the centre, but becomes non-isotropic outside the break
radius and shows nascent tidal tails. These tidally disturbed stars
appear as an extended, outer density excess relative to the initial
conditions. This excess appears just outside the break radius. The break
radius also marks where the mean 3D radial velocity of the stars (with
respect to the galaxy's centre) becomes non-zero --- the galaxy is out
of equilibrium outside \(r_{\rm break}\).

To first order, the final density profile of this toy simulation indeed
resembles the outer excess in Sculptor and Ursa Minor, motivating this
thesis.

\begin{figure}
\centering
\pandocbounded{\includegraphics[keepaspectratio]{figures/idealized_break_radius.pdf}}
\caption[Example tidal simulation]{Example density and velocity
distributions of an idealized dwarf galaxy model undergoing tidal
stripping. \textbf{Top}: The projected 2D stellar density in the
\(y'\)--\(z'\) plane for the initial (left) and final (right)
simulation. \(y'\) is the direction of tangential orbital motion, and
\(z'\) is the direction of orbital angular momentum, as measured from
the centre of the host. The dashed green circle represents the break
radius (Eq.~\ref{eq:r_break}) and the blue arrow points in the orbital
direction. \textbf{Bottom left}: The projected stellar density profile
for the initial (dotted) and final (solid) simulation snapshot.
\textbf{Bottom right}: the mean 3D radial velocity (the dot product of
relative position and velocity relative to dwarf centre) as a function
of projected radius. The green arrow marks the break radius in both
lower panels.}\label{fig:idealized_break_radius}
\end{figure}

\section{Thesis outline}\label{thesis-outline}

The goal of this thesis is to review the evidence for an extended
density profile in Ursa Minor and Sculptor, to assess the impact of
tidal effects on each galaxy, and to discuss possible interpretations
for the structure of these galaxies.

In Chapter \ref{sec:observations}, we describe how we compute
observational density profiles following \citet{jensen+2024}. In Chapter
\ref{sec:methods}, we review our simulation methods. Next, we present
our results for the tidal effects on Sculptor and Ursa Minor in Chapter
\ref{sec:results}. We discuss the implications of our results and end
with a summary and outlook in Chapter \ref{sec:discussion}.



\chapter{The Light Profiles of the Classical Dwarf Spheroidals}\label{sec:observations}
\section{Gaia Membership Selection}\label{gaia-membership-selection}

Gaia provides unprecedented accuracy in proper motions and magnitudes.
Gaia data is uniquely excellent to produce low-contamination samples of
likely member stars belonging to satellites. Here, we breifly describe
J+24's membership estimation and discuss how this informs our
observational knoledge of each galaxies density profile. In general,
J+24 use a Bayesian framework incorporating proper motion (PM),
colour-magnitude diagram (CMD), and spatial information to determine the
probability that a given star belongs to the satellite. J+24 extends
\citet{MV2020a} (see also \citet{pace+li2019}, \citet{battaglia+2022},
\citet{pace+erkal+li2022}, etc.).

J+24 select stars initially from Gaia within a 2--4 degree circular
region centred on the dwarf satisfying:

\begin{itemize}
\tightlist
\item
  Solved parallax, proper motions, colour, and magnitudes.
\item
  High quality astrometry (\texttt{ruwe\ \textless{}=\ 1.3})
\item
  3\(\sigma\) consistency of measured parallax with dwarf distance +
  uncertainty (typically near zero; with \citet{lindegren+2018}
  zero-point correction).
\item
  Absolute RA and Dec proper motions less than
  10\(\,{\rm mas\ yr^{-1}}\) (corresponding to tangental velocities of
  \(\gtrsim 500\) km/s at distances larger than 10 kpc.)
\item
  No colour excess (\citet{lindegren+2018} equation C.2)
\item
  G \textless{} 22 and less than 5\(\sigma\) above TRGB, and between
  -0.5 and 2.5 in Bp - Rp.
\end{itemize}

Photometry is dereddened using \citet{schlegel+1988} extenction.

J+24 calculate the probability that any star belongs to either the
satellite or the MW background as \[
P_{\rm sat} = \frac{f_{\rm sat}{\cal L}_{\rm sat}}{f_{\rm sat}{\cal L}_{\rm sat} + (1-f_{\rm sat}){\cal L}_{\rm bg}}
\]

where the satellite (sat) and background (bg) likelihoods are simply the
product of the PM, CMD, and spatial components: \[
{\cal L} = {\cal L}_{\rm space}\ {\cal L}_{\rm PM}\ {\cal L}_{\rm CMD}
\]

The satellite likelihood is constructed as

\begin{itemize}
\tightlist
\item
  CMD: The CMD is the lowest metallicity isochrone from Padova
  \citep{girardi+2002} with age 12 Gyr with a colour width of 0.1 mag
  plus the Gaia colour uncertainty at each magnitude. The HB is modelled
  as a constant magnitude extending blue of the CMD. The HB magnitude is
  the mean magnitude of HB stars from most metal poor isochrone with a
  0.1 mag width plus the mean colour error. A likelihood map is
  constructed by sampling the distance modulus in addition to the CMD
  width, taking the maximum of RGB and HB likelihoods.
\item
  Spatial: A single exponential
  (\(\Sigma \propto e^{R_{\rm ell} / R_s}\)) accounting for structural
  uncertainty (sampled over position angle, ellipticity, and half light
  radius).
\item
  Alternative spatial: For Scl and UMi, this is instead a double
  exponential
  \(\Sigma_\star \propto e^{-R/R_s} + B\,e^{-R/R_{\rm outer}}\) where
  the inner exponential remains fixed. Structural parameter
  uncertainties are not accounted for.
\item
  PM. A bivariate gaussian with variance and covariance equal to each
  star's proper motions. Each star's proper motions uncertainty are
  assumed to be the dominant uncertainty.
\end{itemize}

The background likelihood is constructed as:

\begin{itemize}
\tightlist
\item
  CMD : Constructed as a density map using the other quality-selected
  stars outside of \(5R_h\) in the catalogue. The map is a sum of
  bivariate gaussians for each star with standard deviations based on
  the Gaia uncertainties.
\item
  PM: same as CMD except in PM space.
\item
  Spatial: a constant likelihood.
\end{itemize}

Note that each likelihood map is normalized over the respective 2D
parameter space. In order to represent the difference in frequency of
background and forground stars, \(f_{\rm sat}\) represents the field
fraction of member stars.

In J+24, a MCMC simulation is ran using the above likelihood to solve
for the following parameters

\begin{itemize}
\tightlist
\item
  Systemic proper motions \(\mu_\alpha\), \(\mu_\delta\). Single
  component prior is same as \citet{MV2020}: a normal distribution with
  mean 0 and standard deviation 100 km/s. If 2-component spatial,
  instead is a uniform distribution spanning 5\(\sigma\) of single
  component case w/ systematic uncertainties.
\item
  \(f_{\rm sat}\) density normalization. Prior is a uniform distribution
  between 0 and 1.
\item
  Spatial component parameters \(B\) is uniform from 0-1 and
  \(R_{\rm outer}\) is uniform and greater than \(R_s\) for extended
  profiles (Scl and UMi here.)
\end{itemize}

The mode of each parameter from the MCMC are then used to calculate the
final \(P_{\rm sat}\) values we use here.

We adopt a probability cut of \(P_{\rm sat} = 0.2\) as our fiducial
sample. Most stars are assigned probabilities close to 0 or 1, so the
choice of probability threshhold is not too significant. Additionally,
even for a probability cut of 0.2, the purity of the resulting sample
with RV measurements is very high (\textasciitilde90\%, J+24). (Note:
there is likely a high systematic bias in using stars with RV
measurements to measure purity. Fainter stars have poorer measurements
and distant stars are less likely to have been targeted. )

\subsection{Resulting Samples}\label{resulting-samples}

In figures fig.~\ref{fig:sculptor_selection},
fig.~\ref{fig:umi_selction}, we illustrate the resulting samples from
the algorithm. In each case, each criteria plays a roll: proper motions
are centred around the dwarf systemic motion, the CMD is well defined,
and stars only within a few \(R_h\) are included. We also plot the RV
members found in general and in \citet{sestito+2024},
\citet{sestito+2024b}.

The tangent plane, \(\xi\), \(\eta\), is the projection that

We also illustrate the approximate result of removing the spatial
component from the likelihood. We define the CMD+PM selection as stars
satisfying \[
{\cal L}_{\rm CMD,\ sat}\ {\cal L}_{\rm PM,\ sat} > {\cal L}_{\rm CMD,\ bg}\ {\cal L}_{\rm PM,\ bg}
\]

These stars appear similar to the fiducial (probable members) sample,
but instead also appear as an approximately uniform distribution across
the entire field. This illustrates the approximate background of stars
which may be confused as members. Additionally, since there is no clear
spatial structure in the CMD+PM sample, it is unlikely that there are
tidal tails detectable with Gaia. Not shown here, we also try a variety
of simpler, absolute cuts and thresholds, finding no extended structure
beyond what is detected in J+24.

This means that at least at the level of where the background density
dominates, we can exclude models which produce tidal tails brighter than
a density of
\(\Sigma_\star \approx 10^{-2}\,\text{Gaia-stars\ arcmin}^{-2} \approx 10^{-6} \, {\rm M_\odot\ kpc^{-2}}\)
(TODO assuming a distance of \ldots{} and stellar mass of \ldots).

\begin{figure}
\centering
\pandocbounded{\includegraphics[keepaspectratio]{figures/scl_selection.pdf}}
\caption[Sculptor selection criteria]{The selection criteria for Scl
members. Probable members (2-component) are orange, and all field stars
(satisfying quality criteria) are in light grey. \textbf{Top:} Tangent
plane. \textbf{Bottom left:} Colour magnitude diagram. \textbf{Bottom
right:} Proper motion.}\label{fig:sculptor_selection}
\end{figure}

\begin{figure}
\centering
\pandocbounded{\includegraphics[keepaspectratio]{figures/umi_selection.pdf}}
\caption[Ursa Minor Selection]{Similar to
fig.~\ref{fig:sculptor_selection} except for Ursa Minor. UMi features a
very extended density profile with some stars \textasciitilde{} 6\(R_h\)
including a RV member. UMi is also highly elliptical compared to other
classical dwarfs.}\label{fig:umi_selection}
\end{figure}

\subsection{Density Profiles}\label{density-profiles}

Our primary observational constraint is the density profile of a dwarf
galaxy.

To derive density profiles, we use 0.05 dex bins in log radius (i.e.~the
bins are derived from 10\^{}(minimum(logR):0.05:maximum(logR))). The
density in each bin is then (from Poisson statistics) \[
\Sigma_b = N_{\rm memb} / A_{\rm bin} \pm \sqrt{N_{\rm memb}} / A_{\rm bin}
\] where \(N_{\rm memb}\) is the number of members in the bin and
\(A_{\rm bin}\) is the area of the bin's annulus in 2D. As discussed
below, these uncertainties underrepresent the true uncertainty on
multiple accounts. We retain Poisson errors for simplicity here.

In Figures fig.~\ref{fig:scl_observed_profiles},
fig.~\ref{fig:umi_observed_profiles}, we show the derived density
profiles for each galaxy for samples similar to in the selection plots
above. In each case, all samples are the same towards the inner regions
of the satellite, illustrating that these density profiles are dominated
by satellite stars in the centre. The sample containing all stars
reaches a plateau at the total background in the field. However,
restricting stars to being most likely satellite members by CMD + PM,
the background is much lower. This plateau likely represents the real
background of background stars which could be mistaken as members.
Finally, we have the probable members and bg-subtracted densities. BG
subtracted is based on the \texttt{all} density profile, subtracting the
mean background density for stars beyond the last point of the BG
subtracted profile.

Note that the probable members (fiducial) density profile continues to
confidently estimate the density profile below the CMD+PM likely star
background. These points are likely unreliable (see discussion below).
However, before this point, both the BG subtracted and probable members
density profiles are strikingly similar. Assumptions about the details
of the likelihood and spatial dependence have marginal influence on the
resulting density profile when the satellite is higher density than the
background.

\begin{figure}
\centering
\pandocbounded{\includegraphics[keepaspectratio]{figures/scl_density_methods.pdf}}
\caption[Sculptor density profiles]{The density profile of Sculptor for
different selection criteria. \emph{probable members} selects stars with
PSAT \textgreater{} 0.2 considering PM, CMD, and spatial, \emph{CMD+PM}
select stars more likely to be members according to CMD and PM only,
\emph{all} selects any high quality star, and \emph{BG subtracted} is
the background-subtracted density derived from high-quality
stars.}\label{fig:scl_observed_profiles}
\end{figure}

\begin{figure}
\centering
\pandocbounded{\includegraphics[keepaspectratio]{figures/umi_density_methods.pdf}}
\caption[Ursa Minor density profiles]{Similar to
fig.~\ref{fig:scl_observed_profiles} except for Ursa
Minor.}\label{fig:umi_observed_profiles}
\end{figure}

\begin{figure}
\centering
\pandocbounded{\includegraphics[keepaspectratio]{figures/fornax_density_methods.pdf}}
\caption[Fornax density profiles]{Similar to
fig.~\ref{fig:scl_observed_profiles} except for
Fornax.}\label{fig:fornax_observed_profiles}
\end{figure}

\section{Comparison of the Classical
dwarfs}\label{comparison-of-the-classical-dwarfs}

Classical dwarfs are often the brightest dwarfs in the sky.c The density
profiles of classical dwarf galaxies is thus well measured, enabling
detailed comparisons.

\begin{table*}[t]
\centering
\begin{tabular}{lll}
\toprule
galaxy & R\_h (exp inner 3Rs) & num cand\\
\midrule
Fornax & $17.8\pm0.6$ & 23,154\\
Leo I & $3.7\pm0.2$ & 1,242\\
Sculptor & $9.4\pm0.3$ & 6,888\\
Leo II & $2.4\pm0.3$ & 347\\
Carina & $8.7\pm0.4$ & 2,389\\
Sextans I & $20.2 \pm0.9$ & 1,830\\
Ursa Minor & $11.7 \pm 0.5$ & 2,122\\
Draco & $7.3\pm0.3$ & 1,781\\
\bottomrule
\end{tabular}
\end{table*}

Using the same methods above, we select members from J+24's stellar
probabilities. We use the one-component exponential density profiles
with

Figure: Density profiles for each dwarf galaxy. Here, we use the
1-component exponential stellar probabilities from J+24. Dwarf galaxies
are scaled by our derived \(R_h\) values.

\subsection{Density Profile Reliability and
Uncertainties}\label{density-profile-reliability-and-uncertainties}

\begin{itemize}
\tightlist
\item
  How well do we know the density profiles?
\item
  What uncertainties affect derived density profiles?
\item
  Can we determine if Gaia, structural, or algorithmic systematics
  introduce important errors in derived density profiles?
\item
  Using J+24 data, we validate

  \begin{itemize}
  \tightlist
  \item
    Check that PSAT, magnitude, no-space do not affect density profile
    shape too significantly
  \end{itemize}
\item
  Our ``high quality'' members all have \textgreater{} 50 member stars
  and do not depend too highly on the spatial component, mostly
  corresponding to the classical dwarfs
\end{itemize}

J+24's algorithm takes spatial position into account, assuming either a
one or two component exponential density profile. When deriving a
density profile, this assumption may influence the derived density
profile, especially when the galaxy density is fainter than the
background of similar appearing stars. To remedy this and estimate where
the background begins to take over, we also explore a cut based on the
likelihood ratio of only the CMD and PM components. This is in essence
assuming that the spatial position of a star contains no information on
it's membership probability (a uniform distribution like the background)

\subsection{Caveats}\label{caveats}

The J+24 method was designed to determine high probability members for
spectroscopic followup in particular. Note that we instead care about
retrieving a reliable density profile.

In particular, in fig.~\ref{fig:umi_observed_profiles}, notice that the
PSAT method produces artifically small errorbars even when the density
is \textgreater1dex below the local background. These stars are likely
selecting stars from the statistical MW background consistent with UMi
PM / CMD, recovering the assumed density profile. As a result, the
reliability of faint features in these density profiles is questionable
and a more robust analysis, removing this particular density assumtion,
would be required to more appropriately represent the knowledge of the
density profile as the background begins to dominate.

J+24 do not account for structural uncertainties in dwarfs. This is a
not insignificant source of uncertainty in the derived density profile

We assume constant ellipticity and position angle. Dwarf galaxies, in
reality, are not necessarily smooth and constant.

\section{Comparison and conclusions}\label{comparison-and-conclusions}

To illustrate the differences between each dwarf galaxy, in
fig.~\ref{fig:classical_dwarfs_densities}, we compare Scl, UMi, and Fnx
against exponential and plummer density profiles (\textbf{TODO: state
these somewhere}). While all dwarfs have marginal differences in the
inner regions, each dwarf diverges in the outer regions relative to an
exponential. In particular, while Fnx is underdense, Scl and UMi are
both overdense, approximately fitting a Plummer density profile instead.

In summary, we have used J+24 data to derive the density profiles for
Fornax, Sculptor, and Ursa Minor. In each case, the density profile is
robust against different selection criteria. Both Sculptor (Ursa Minor)
show strong (weak) evidence for deviations from an exponential profile.
We will explore a tidal explanation for these features in this work.

\begin{figure}
\centering
\pandocbounded{\includegraphics[keepaspectratio]{figures/scl_umi_fornax_exp_fit.pdf}}
\caption[Classical dwarf density profiles]{The density profiles of
Sculptor, Ursa Minor, and Fornax compared to Exp2D and Plummer density
profiles. Dwarf galaxies are scaled to the same half-light radius and
density at half-light radius (fit from the inner 3 scale radii
exponential recursively. )}\label{fig:classical_dwarfs_densities}
\end{figure}

\section{Radial velocity modeling}\label{radial-velocity-modeling}

For both Sculptor and Ursa Minor, we construct samples of radial
velocity measurements and cross match each sample to produce estimates
of the total velocity dispersion and line of sight velocity for each
galaxy.

\subsection{Methodology}\label{methodology}

First, we crossmatch all catalogues to J+24 Gaia stars. If a study did
not report GaiaDR3 source ID's, we match to the nearest star within 2
arcseconds. We exclude stars not matched to Gaia for simplicity.

We combine the mean RV measurement from each study using the
inverse-variance weighted mean and standard uncertainty. \[
\bar v = \frac{1}{\sum w_i}\sum_i w_i\ v_{i} \\
\delta v = \sqrt{\frac{1}{\sum_i w_{i}^2}}
\] where \(w_i = 1/s_i^2\), and we estimate the inter-study standard
deviation \[
s^2 = \frac{1}{\sum w_i} \sum_i w_i (v_{i} - \bar v)^2
\] We remove stars with significant velocity dispersions as measured
between or within a study: \[
s < 5\,\delta v\,\sqrt{n}
\] where \(s, \delta v, n\) are the standard deviation, standard, error,
and number of measurements for both a study (with multiple epochs) and
inter-study comparisons.

The combined RV likelihood is then \[
{\cal L} = {\cal L}_{\rm space} {\cal L}_{\rm CMD} {\cal L}_{\rm PM} {\cal L}_{\rm RV}
\] where \[
{\cal L}_{\rm RV, sat} = N(\mu_{v}, \sigma_{v}^2 + (\delta v_i)^2) \\
{\cal L}_{\rm RV, bg} = N(0, \sigma_{\rm halo}^2)
\] where \(\mu_v\) and \(\sigma_v\) are the systemic velocity and
dispersion, and \(\delta v_i\) is the individual measurement
uncertainty. Typically, the velocity dispersion will dominate the
uncertainty budget here. We assume a halo/background velocity dispersion
of a constant \$\sigma\_\{\rm halo\} = 100 \$ km/s
\citep[e.g.][]{brown+2010}.

Similar to above, we retain stars with the resulting membership
probability of greater than 0.2.

Finally, we need to correct the coordinate frames for the solar motion
and on-sky size of the galaxy. The first step is to subtract out the
solar motion from each radial velocity, corresponding to a typical
gradient of \textasciitilde3 km/s across the field. The next step is to
account for the slight differences in the direction of each radial
velocity. Define the \(\hat z\) direction to point parallel to the
direction from the sun to the centre of the dwarf galaxy. Then if
\(\phi\) is the angular distance between the centre of the galaxy and
the individual star, the corrected radial velocity is then \[
v_z = v_{\rm los, gsr}\cos\phi  - v_{\alpha}\cos\theta \sin\phi - v_\delta \sin\theta\sin\phi
\] where
\(v_{\rm tan, R} = d(\mu_{\alpha*}\cos\theta + \mu_\delta \sin \theta)\)
is the radial component of the proper motion with respect to the centre
of the galaxy.This correction is of the order \(v_{\rm tan}\theta\) so
induces a gradient of about \(1 km/s/degree\) for sculptor
\citep[see][]{WMO2008}. The uncertainty is then the velocity uncertainty
plus the distance uncertainties times the PM uncertainty from above. We
then use the \(v_z\) values for the following modelling, however
repeating with plain, solar-frame velocities does not substantially
affect the results too much .

For the priors on the satellite velocity dispersion and systematic
velocity, we use \[
\mu_{v} = N(0, \sigma_{\rm halo}^2) \\
\sigma_{v} = U(0, 20\,{\rm km\,s^{-1}})
\] where \(\sigma_{\rm halo} = 100\,{\rm km\,s^{-1}}\) is the velocity
dispersion of the MW halo adopted above, a reasonable assumption for
dwarfs in orbit around the MW.

\subsection{Sculptor}\label{sculptor}

\begin{table*}[t]
\centering
\caption{Summary of velocity measurements and derived properties.}
\label{tbl:Summary-of-velocity-measurements-and-derived-properties}
\begin{tabular}{llllllll}
\toprule
 & Study & Instrument & Memb & Rep. & N$\sigma > s$ & $\bar v$ & $\sigma_v$\\
\midrule
Scl &  &  &  &  &  &  & \\
 & tolstoy+23 & VLT/FLAMES & 200 (1604) & 500 &  &  & 9.6\\
 & sestito+23a & GMOS & 2000 &  &  &  & -\\
 & walker+09 & MMFS & 2000 &  &  &  & 9.5\\
 & APOGEE & APOGEE & 200 & 200 &  & 8 & 9\\
UMi &  &  &  &  &  &  & \\
 & sestito+23b & GRACES & 6 &  &  &  & \\
 & pace+20 & Keck/DEIMOS &  &  &  & -245 & 8.6\\
 & spencer+18 & MMT/Hectoshell &  &  &  & -247 & \\
 & APOGEE & APOGEE & 100 & 100 &  &  & \\
\bottomrule
\end{tabular}
\end{table*}

For Sculptor, we combine radial velocity measurements from APOGEE,
\citet{sestito+2023a}, \citet{tolstoy+2023}, and \citet{WMO2009}.
\citet{tolstoy+2023} and \citet{WMO2009} provide the bulk of the
measurements. We find that there is no significant velocity shift in
crossmatched stars between catalogues. After crossmatching to Gaia and
excluding significant inter-study dispersions, we have a sample of XXXX
members.

We derive a systemic velocity of \(111.2\pm0.2\) km/s with velocity
dispersion \(9.67\pm0.16\) km/s. Our values are very consistent with
previous work \citep[e.g.][\citet{arroyo-polonio+2024},
\citet{tolstoy+2023}]{WMO2009}.

Finally, we attempt to fit a linear velocity gradient to Scl by adding
parameters for the gradient in \(\xi\) ande \(\eta\). We derive a
gradient of xxx and xxx (see Figure\textasciitilde FIG.), noting that
this direction is different and higher in magnitude than the proper
motion of Scl. Comparing the density difference, we find a bayes factor
of -15. Compared to past work, \citet{battaglia+2008},
\citet{arroyo-polonio+2021}.

\begin{figure}
\centering
\pandocbounded{\includegraphics[keepaspectratio]{figures/scl_vlos_xi_eta.pdf}}
\caption[Scl velocity sample]{A plot of the corrected los velocities for
Scl binned in tangent plane coordinates. We detect a slight rotational
gradient towards the bottom right. \textbf{TODO} add a histogram of
velocities here too with fits, maybe with respect to distance.}
\end{figure}

\begin{figure}
\centering
\pandocbounded{\includegraphics[keepaspectratio]{figures/scl_vel_gradient.pdf}}
\caption[Scl velocity gradient]{A velocity gradient in Sculptor! The
arrow marks the gradient induced by Scl's proper motion on the sky.
\textbf{TODO}: plot of velocity of Scl in direction of gradient.}
\end{figure}

\emph{Is it interesting that the velocity dispersion of Scl seems to
increase significantly with Rell?}

\subsection{Ursa Minor}\label{ursa-minor}

For UMi, we collect radial velocities from, APOGEE,
\citet{sestito+2024b}, \citet{pace+2020}, and \citet{spencer+2018}.

We shifted the velocities of \citet{spencer+2018} (\(-0.9\) km/s) and
\citet{pace+2020} (\(+1.0\) km/s) to reach the same scale based on a
crossmatch of about 200 common stars. Since the mean difference in
velocities in this crossmatch is of order 1.9 km/s, we adopt this as the
systematic LOS velocity error.

UMi interestingly has a more structured observational pattern, but does
not appear to have any significant velocity substructure.

\begin{figure}
\centering
\pandocbounded{\includegraphics[keepaspectratio]{figures/umi_vlos_xi_eta.pdf}}
\caption[UMi velocity sample]{A plot of the corrected los velocities for
UMi binned in tangent plane coordinates. There is no clear rotation or
velocity gradient here. Interestingly, many velocity members are as far
as 100 arcmin away.}
\end{figure}

\subsection{Discussion and
limitations}\label{discussion-and-limitations}

Our model here is relatively simple. Some things which we note as
systematics and are challenging to account fully for are

\begin{itemize}
\tightlist
\item
  Inter-study systematics and biases. While basic crossmatches and a
  simple velocity shift, combining data from multiple instruments is
  challenging.
\item
  Inappropriate uncertainty reporting. Inspection of the variances
  compared to the standard deviations within a study seems to imply that
  errors are accurately reported. APOGEE notes that their RV
  uncertainties are known to be underestimates.
\item
  Binarity. While not too large of a change for classical dwarfs, this
  could inflate velocity dispersions of about 9 km/s by about 1 km/s
  \citet{spencer+2017}. Thus, our measurement is likely slightly
  inflated given the high binarity fractions measured in classical
  dwarfs \citep[\citet{spencer+2018}]{arroyo-polonio+2023}.
\item
  Selection effects. RV studies each have their own selection effects. I
  do not know how to correct for this.
\end{itemize}

Because the derived parameters are similar for the two different larger
surveys we consider for UMi and Scl, we note that many of these effects
are likely not too significant (with the exception of the systemic
motion of UMi.)

\section{Appendix / Extra Notes}\label{appendix-extra-notes}

\subsection{Additional density profile
tests}\label{additional-density-profile-tests}

\begin{figure}
\centering
\pandocbounded{\includegraphics[keepaspectratio]{figures/scl_density_methods_extra.pdf}}
\caption[Density profiles]{Density profiles for various assumptions for
Sculptor. PSAT is our fiducial 2-component J+24 sample, circ is a
2-component bayesian model assuming circular radii, simple is the series
of simple cuts described in Appendix ?, bright is the sample of the
brightest half of stars (scaled by 2), DELVE is a sample of RGB stars
(background subtracted and rescaled to
match).}\label{fig:sculptor_observed_profiles}
\end{figure}

Note that a full rigorous statistical analysis would require a
simulation study of injecting dwarfs into Gaia and assessing the
reliability of various methods of membership and density profiles. This
is beyond the scope of this thesis.

\begin{verbatim}
SELECT TOP 1000
       *
FROM delve_dr2.objects
WHERE 11 < ra
and ra < 19
and -37.7 < dec
and dec < -29.7
\end{verbatim}


\chapter{Simulation Methods}\label{sec:methods}
In this section, we discuss our methods used for the following two
chapters. We utilize an analytic, static, axisymmetric Milky Way
potential based on \citet{mcmillan2011}. Our N-body simulation assume
collisionless, cold dark matter and are integrated with Gadget-4. We
check for numerical convergence and stability of initial conditions in
isolation.

\section{Milky Way potential}\label{milky-way-potential}

We adopt the Milky Way potential described in \citet{EP2020}, an
analytic approximation of \citet{mcmillan2011}.
Fig.~\ref{fig:v_circ_potential} plots the circular velocity profiles of
each component and the total circular velocity profile for our fiducial
profile. The potential consisting of a stellar bulge, a thin and thick
disk, and a dark matter NFW halo. Among typical orbits for Sculptor and
Ursa Minor, the dark matter halo dominates the potential.

The potential is specified as follows. The galactic bulge is described
by a \citet{hernquist1990} potential,

\begin{equation}\protect\phantomsection\label{eq:hernquist}{
\Phi(r) = - \frac{GM}{r + a},
}\end{equation} where \(a=1.3\,{\rm kpc}\) is the scale radius and
\(M=2.1 \times 10^{10}\,\Mo\) is the total mass. The thin and thick
disks are represented with the \citet{miyamoto+nagai1975} cylindrical
potential:

\begin{equation}\protect\phantomsection\label{eq:mn75}{
\Phi(R, z) = \frac{-GM}{\left(R^2 + \left[a + \sqrt{z^2 + b^2}\right]^{2}\right)^{1/2}}
}\end{equation}

where \(a\) is the disc radial scale length, \(b\) is the scale height,
and \(M\) is the total mass of the disk. For the thin disk,
\(a=3.944\,\)kpc, \(b=0.311\,\)kpc,
\(M=3.944\times10^{10}\,\)M\(_\odot\). For the thick disk,
\(a=4.4\,\)kpc, \(b=0.92\,\)kpc, and \(M=2\times10^{10}\,\)M\(_\odot\).
The halo is a NFW dark matter halo (Eq.~\ref{eq:nfw}) with
\(M_s=79.5\times10^{10}\,\)M\(_\odot\). and \(r_s = 20.2\,\)kpc.

\begin{figure}
\centering
\pandocbounded{\includegraphics[keepaspectratio]{figures/v_circ_potential.png}}
\caption[Circular velocity of potential]{Circular velocity profile of
\citet{EP2020} potential.}\label{fig:v_circ_potential}
\end{figure}

\section{Orbital estimation}\label{orbital-estimation}

To estimate the possible orbits of a dwarf galaxy, we perform a Monte
Carlo sampling of the present-day observables. The present-day position,
distance modulus, LOS velocity, and proper motions are each sampled from
normal distributions given the reported uncertainties in
Tables~\ref{tbl:scl_obs_props}, \ref{tbl:umi_obs_props}. We integrate
each sampled observable back in time for 10 Gyr in Gadget as massless
point particles outputting positions every 5Myr (with otherwise similar
parameters to n-body runs below.) When selecting the initial position
and velocity of an N-body model, we select the position and velocity of
the first apocentre occurring after 10 Gyr ago.

We assume the galaxy is represented as a point particle for this
analysis. In detail, (self) dynamical friction likely influences the
orbit, but a point particle is an adequate description.

To convert from Gaia to galactocentric coordinates, we use the Astropy
v4 galactocentric frame \citep{astropycollaboration+2022}. This frame
assumes the galactic centre appears at position
\(\alpha = {\rm 17h45m37.224s}\), \(\delta = -28°56'10.23''\) with
proper motions \(\mu_{\alpha*}=-3.151\pm0.018\ \masyr\) ,
\(\mu_\delta=-5.547\pm0.026 \masyr\) (from the appendix and Table 2 of
\citet{reid+brunthaler2004}). The galactic centre distance is
\(8.122\pm0.033\,\)kpc with a radial velocity = \(11 + 1.9 \pm 3\) km/s.
from \citet{gravitycollaboration+2018}. Finally, adding that the sun is
\(20.8\pm0.3\,\)pc above the disk from \citet{bennett+bovy2019}, and
using the procedure outlined in \citet{drimmel+poggio2018}, the solar
velocity relative to the galactic rest frame is `v\_sun' =
\([-12.9 \pm 3.0, 245.6 \pm 1.4, 7.78 \pm 0.08]\) km/s. The
uncertainties in the reference frame are typically smaller than the
uncertainties on the dwarf galaxy's position and velocity.

\section{N-Body modelling}\label{n-body-modelling}

\subsection{Initial conditions}\label{initial-conditions}

We use Agama \citep{agama} to generate initial conditions. We initially
assume galaxies are described by an NFW dark matter potential
Eq.~\ref{eq:nfw} and the stars are merely collisionless tracers embedded
in this potential (added on in post-processing). The density is a
cubic-exponentially truncated with a profile
\begin{equation}\protect\phantomsection\label{eq:trunc_nfw}{
\rho_{\rm tNFW} = e^{-(r/r_t)^3}\ \rho_{\rm NFW}(t)
}\end{equation} where we adopt \(r_t = 20 r_s\) or approximately
\(r_{200}\) for our Sculptor-like fiducial halo. Using \(r_{200}\) for
\(r_t\) would depend on the chosen scale of the halo. So our adopted
\(r_t\) is an approximate upper limit of \(r_{200}\) for typical dwarf
galaxy halos \citep{ludlow+2016}. It is unlikely that the outer density
profile of loosely bound particles past \(20\) kpc affects the tidal
evolution of a subhalo.

\subsection{Isolation runs and simulation
parameters}\label{isolation-runs-and-simulation-parameters}

To ensure that the initial conditionss of the simulation are dynamically
relaxed and well-converged, we run the simulation in isolation (no
external potential) for 5 Gyr (or about 3 times the crossing timescale
at the virial radius). Our fiducial isolation halo uses \(r_s=2.76\) kpc
and \(M_s = 0.29 \times 10^{10}\) Msun, but can be easily rescaled for
any length or mass scale.

For our simulation parameters, we adopt a softening length of
\begin{equation}{
h_{\rm grav} = 0.014 \left(\frac{r_s}{2.76\,{\rm kpc}}\right)\left(\frac{N}{10^7}\right)^{-1/2}.
}\end{equation} See appendix Section~\ref{sec:extra_convergence} for a
discussion of this choice, which is similar to the \citet{power+2003}
suggested softening. We use the relative tree opening criterion with the
accuracy parameter set to 0.005, and adaptive time stepping with
integration accuracy set to 0.01.

\subsubsection{Numerical convergence}\label{numerical-convergence}

As discussed in \citet{power+2003}, the region where density is
converged is related to the region which becomes collisionally relaxed
over the time of the universe (so about 5-10 Gyr). (i.e.~where the
collisionless assumption breaks down). The relaxation timescale is given
by \begin{equation}\protect\phantomsection\label{eq:t_relax}{
t_{\rm relax}(r) = t_{\rm circ}(r) \frac{N(r)}{8\,\ln N(r)}
= {t_{\rm circ}(r_{200})} \frac{\sqrt{200}}{8} \frac{N(r)}{\ln N(r)} \left(\frac{\bar \rho (r)}{\rho_{\textrm crit}}\right)^{-1/2}
}\end{equation} where \(N(r)\) is the number of particles within \(r\),
\(t_{\rm circ}\) is the circular velocity timescale, and \(\bar \rho\)
is the mean interior density \citep{power+2003}. The converged radius
\(r_{\rm relax}(t)\) is the radius above which \(t_{\rm relax} > t\).
Typically, \(r_{\rm relax}(10\Gyr)\) is about 6-10 times our adopted
softening length, increasing with particle number. As such, at full
resolution, we can only trust density profiles down to \(\sim10\) times
our softening length, sufficient to resolve stellar density profiles.

\begin{figure}
\centering
\pandocbounded{\includegraphics[keepaspectratio]{figures/iso_converg_num.png}}
\caption[Numerical halo convergence]{Numerical convergence test for
circular velocity as a function of log radius for simulations with
different total numbers of particles in isolation. Residuals in bottom
panel are relative to NFW. The initial conditions are dotted and the
converged radius is marked by arrows (Eq.~\ref{eq:t_relax}). Note that a
slight reduction in density starting around \$r = 30 \$kpc is expected
given our truncation choice.}\label{fig:numerical_convergance}
\end{figure}

\subsection{Orbital runs}\label{orbital-runs}

To perform the simulations of a given galaxy in a given potential, we
centre the isolation run's final snapshot
(Section~\ref{sec:shrinking_spheres}) and place the dwarf galaxy in the
specified orbit in the given potential. We typically run the simulation
for 10 Gyr, which allows us to orbit slightly past the expected initial
conditions.

\section{Post processing}\label{post-processing}

\subsection{Centring}\label{sec:shrinking_spheres}

Shrinking spheres centres inspired by \citet{power+2003}

\begin{itemize}
\tightlist
\item
  Recursively shrink radius by 0.975 quantile and recalculating centroid
  until radius is less than \textasciitilde1kpc or fewer than 0.1\% of
  particles remain.
\item
  Remove bound particles (using instantanious potential).
\item
  Use previous snapshot centre and acceleration to predict new centre
  for next snapshot and only include particles included in previous
  centring timestep.
\end{itemize}

The statistical centring uncertainty for the 1e7 particle isolation run
is of order 0.003 kpc, however oscillations in the centre are of order
0.03 kpc. This is about three times the softening length but is less
than the numerically converged radius scale.

\subsection{Velocity profiles}\label{velocity-profiles}

\begin{itemize}
\tightlist
\item
  Circular velocity is computed assuming spherical symmetry and only
  shown for every 200th particle (ranked from the centre outwards)
\item
  \(v_{\rm circ\ max}\) and \(r_{\rm circ,\ max}\) is found by
  least-squares fitting the NFW functional velocity form to the points
  of the velocity profile that have the 10\% highest velocities. This is
  not a necessarily good fit, especially as the halo becomes stripped,
  but accurate enough to find a reliable maximum.
\end{itemize}

\subsection{Stellar probabilities}\label{stellar-probabilities}

We ``paint'' stars onto dark matter particles using the particle tagging
method \citep[e.g.][]{bullock+johnston2005}, assuming spherical
symmetry. Let \(\Psi\) be the potential (normalized to vanish at
infinity) and \({\cal E}\) is the binding energy
\({\cal E} = \Psi - 1/2 v^2\). If we know \(f({\cal E})\), the
distribution function (phase-space density in energy), then we assign
the stellar weight for a given particle with energy \({\cal E}\) is

\begin{equation}{
P_\star({\cal E}) = \frac{f_\star({\cal E})}{f_{\rm halo}({\cal E})}.
}\end{equation} While \(f({\cal E})\) is a phase-space density, the
differential energy distribution includes an additional \(g({\cal E})\)
occupation term (BT87). We use Eddington inversion to find the
distribution function, (eq. 4-140b in BT87)

\begin{equation}{
f({\cal E}) = \frac{1}{\sqrt{8}\, \pi^2}\left( \int_0^{\cal E} \frac{d^2\rho}{d\Psi^2} \frac{1}{\sqrt{{\cal E} - \Psi}}\ d\Psi + \frac{1}{\sqrt{\cal E}} \left(\frac{d\rho}{d\Psi}\right)_{\Psi=0} \right).
}\end{equation}

In practice the right, boundary term is zero as \(\Psi \to 0\) as
\(r\to\infty\), and if \(\rho \propto r^{-n}\) at large \(r\) and
\(\Psi \sim r^{-1}\) then \(d\rho / d\Psi \sim r^{-n+1}\) which goes to
zero provided that \(n > 1\). We take \(\Psi\) from the underlying
assumed analytic dark matter potential. \(\rho_\star\) can be calculated
from the surface density, \(\Sigma_\star\), via the inverse Abel
transform. In this work, we consider both Plummer and exponential
profiles, as described in

\subsection{Stellar density profiles and velocity
dispersion}\label{stellar-density-profiles-and-velocity-dispersion}

Stellar velocity dispersion is calculated for all stars within a 1kpc
sphere of the centre. We assume
\(\sigma_{\rm los} = \sqrt{\sigma_x^2 + \sigma_y^2 + \sigma_z^2}/ \sqrt 3\),
i.e.~isotropic velocity dispersion.

We calculate density profiles similar to stars and assume a constant bin
width in \(\log R\) of \(2 {\rm IQR} / \sqrt[3]{N}\) (Freedman-Diaconis
prescription). \(R\) may be either the cylendrical radius in \(x-y\) or
the on-sky project \(\xi, \eta\) tangent plane coordinates.

Because our dwarfs are assumed to be spherical/isotropic, we retain this
assumption when calculating predicted density profiles.


\chapter{The Effects of Tides}\label{sec:results}
As discussed in Chapters \ref{sec:introduction}, \ref{sec:observations},
we test whether Galactic tides can form the extended density profiles of
Scl and UMi. In this Chapter, we analyze tailored N-body simulations,
using the methods described in Chapter \ref{sec:methods}, to assess
tidal impacts on each galaxy. We find that tides drive dark matter loss
in both systems but leave their compact stellar components largely
unaffected. The Large Magellanic Cloud (LMC) perturbs the orbit of Scl,
and somewhat UMi, yet the resulting tidal effects are weaker than in a
Milky Way-only potential. Our simulations demonstrate that recent tides
are unlikely to alter the stellar structure of Scl or UMi.

In this Chapter, we consider Scl first, describing tidal effects from
the MW on dark matter and stellar components, and then consider the
effects from the LMC. We then analyze UMi in a similar manner,
considering in turn the dark matter evolution, stellar evolution, and
orbital effects of the LMC.

\section{Tidal effects on Sculptor}\label{tidal-effects-on-sculptor}

\subsection{Dark matter evolution}\label{dark-matter-evolution}

As a representation of the most extreme tidal history, we initially
investigate the \smallperi{} Scl model.

Scl experiences moderate tidal mass loss. Fig.~\ref{fig:scl_sim_images}
shows the stripping of dark matter and formation of diffuse streams
trailing and leading Sculptor's orbit. Because tidal stripping can be
described as removal of the lowest energy particles, most mass loss
occurs in the outer halo.

N-body models may deviate from a point-particle trajectory due to
dynamical self-friction \citep[e.g.,][]{miller+2020}. However, this
effect is slight for Scl, which ends near the observed position, without
adjusting the initial conditions (the green point in
Fig.~\ref{fig:scl_sim_images}).

The inner density cusp is tidally resilient.
Fig.~\ref{fig:scl_tidal_track} shows the initial and final circular
velocity profiles, and the evolution of the maximum circular velocity.
The maximum velocity drops from \(31\,\kms\) to \(22\,\kms\), evolving
along the tidal track from \citet{EN2021}. The final circular velocity
profile resembles the initial with an inner cusp, but has a sharper
outer truncation. Quantitatively, the halo loses \(>90\%\) its initial
mass, as listed in Table~\ref{tbl:scl_sim_results}. However, the inner
structure is not expected to be affected, as the Jacobi radius is over
3kpc, outside of the initial and final \(\rmax\). Thus, tides may remove
significant mass, but mostly from the outer halo.

\begin{figure}
\centering
\pandocbounded{\includegraphics[keepaspectratio]{figures/scl_sim_images.png}}
\caption[Sculptor simulation snapshots]{Images of the dark matter
evolution over a selection of past apocentres and the present day.
Limits range from -150 to 150 kpc in the \(y\)--\(z\) (\(\sim\)orbital)
plane, and the colourscale is logarithmic spanning 5 orders of magnitude
between the maximum and minimum values. The green dot represents the
final expected position of the galaxy and the solid and dotted grey
curves represents the orbit over one previous or future radial
oscillation respectively.}\label{fig:scl_sim_images}
\end{figure}

\begin{figure}
\centering
\pandocbounded{\includegraphics[keepaspectratio]{figures/scl_tidal_track.png}}
\caption[Sculptor tidal tracks]{Dark matter evolution for the
\smallperi{} model of Sculptor. Blue solid and orange dashed lines show
the initial and final circular velocity profiles. The points represent
the evolution of the maximum circular velocity, and the dotted black
line shows the tidal track from \citet{EN2021}. To calculate the
velocity profiles, unbound particles are iteratively removed,
recalculating the potential at each step assuming spherical
symmetry.}\label{fig:scl_tidal_track}
\end{figure}

\begin{table*}[t]
\centering
\caption[Simulation results for Sculptor’s dark matter]{The orbital and dark matter properties for the simulation of Sculptor. The random samples column shows the distributions from point orbits, and the \smallperi{} column contains the results from the N-body simulation. }
\label{tbl:scl_sim_results}
\begin{tabular}{lll}
\toprule
Property & random samples & \smallperi{}\\
\midrule
pericentre & $53\pm3$ & 29\\
apocentre & $102\pm3$ & 94.4\\
time of last pericentre / Gyr & $-0.45 \pm 0.2$ & -0.47\\
number of pericentres & 6 & 6\\
Jacobi radius / kpc & $4.5 \pm 0.3$ & 3.5\\
Jacobi radius / arcmin & $186\pm12$ & 101\\
final heliocentric distance / kpc & $83.2\pm2$ & 81.6\\
$\V_\textrm{max, f} / \V_\textrm{max, i}$ &  & 0.695\\
$r_\textrm{max, f} / r_\textrm{max, i}$ &  & 0.406\\
fractional final bound mass &  & 0.0893\\
\bottomrule
\end{tabular}
\end{table*}

\subsection{Stellar evolution}\label{stellar-evolution}

Tides minimally affect the stellar component of Sculptor in the
\smallperi{} orbit. In Fig.~\ref{fig:scl_smallperi_i_f}, the projected
stellar distribution displays no prominent distortions, and the radial
density profile is nearly unchanged. Only at a surface density
\(\sim10^8\) times fainter than the centre may faint tidal features
emerge. The total stellar mass lost corresponds to \(\sim 10\) stars in
total (see Table~\ref{tbl:scl_sim_results})---a formidable challenge
with the best of observations.

More extended stellar profiles are more sensitive to tides. With Plummer
initial conditions, as shown in
Fig.~\ref{fig:scl_smallperi_plummer_i_f}, the model loses more stellar
mass and forms more luminous tidal tails. Observations reaching surface
densities \(\sim10\) times fainter than our data could reveal a stream
in this case. Nevertheless, the radial extent probed by our data remains
nearly unchanged by tidal evolution.

Table~\ref{tbl:scl_sim_stars_results} quantifies the evolution of
stellar properties. The stellar velocity dispersion decreases by only
\(\sim1\,\kms\) and the half-light radius expands by \(\sim 10\%\). This
is consistent with adiabatic expansion due to the reduction of the total
mass \citep[e.g.,][]{stucker+2023}. In addition, the break and Jacobi
radii are \(\gtrsim 100\) arcminutes on the sky---tidal signatures would
be beyond the reach of our data. Altogether, Galactic tides negligibly
impact Scl's stellar component.

\begin{figure}
\centering
\pandocbounded{\includegraphics[keepaspectratio]{figures/scl_smallperi_i_f.pdf}}
\caption[Sculptor initial and final density profiles]{The tidal effects
on Scl's stellar component, for the \smallperi{} orbit with the fiducial
halo and exponential stars with \(R_s=0.10\,\kpc\). \textbf{Top:} the
initial (left) and final (right) 2D projected density of stars on the
sky. The solid circle marks \(6R_h\), the dotted circle the break
radius, and the (dotted) line the past (future) orbit. \textbf{Bottom:}
The initial (dotted) and final (solid) stellar density profiles as
compared to the observed stellar density profile. Arrows mark the break
and Jacobi radii (Eqs.~\ref{eq:r_break}, \ref{eq:r_jacobi})
.}\label{fig:scl_smallperi_i_f}
\end{figure}

\begin{figure}
\centering
\pandocbounded{\includegraphics[keepaspectratio]{figures/scl_plummer_i_f.pdf}}
\caption[Sculptor Plummer initial and final density profiles]{Similar to
Fig.~\ref{fig:scl_smallperi_i_f} except for Plummer initial stars with
\(R_h = 0.20\,\kpc\). While a faint stream may be visible with deeper
observations, effects on the stellar profile are
minimal.}\label{fig:scl_smallperi_plummer_i_f}
\end{figure}

\begin{table*}[t]
\centering
\caption[Simulation results for Sculptor’s stars]{The present-day stellar properties for the simulations of Sculptor. In each row, we have the initial stellar velocity dispersion (within 1kpc), the final velocity dispersion, the fraction of stellar mass unbound, the initial half-light radius, the final half-light radius, and the break radius in arcmin and kpc (Eq. \ref{eq:r_break}). }
\label{tbl:scl_sim_stars_results}
\begin{tabular}{lll}
\toprule
Property & Exponential & Plummer\\
\midrule
$\sigma_{\V, i}\,/\,\kms$ & 9.8 & 10.7\\
$\sigma_{\V, f} \,/\,\kms$ & 8.8 & 9.4\\
fractional stellar mass loss & $2.1\times 10^{-6}$ & $0.024$\\
$R_{h, i}\,/\,\kpc$ & 0.169 & 0.202\\
$R_{h, f}\,/\,\kpc$ & 0.189 & 0.227\\
break radius / arcmin & $98$ & $105$\\
break radius / kpc & 2.3 & 2.5\\
\bottomrule
\end{tabular}
\end{table*}

\subsection{Orbital effects of the LMC}\label{sec:scl_lmc}

The Milky Way isn't the only galaxy in town. Recently, work has shown
that the infall of the LMC may substantially affect the Milky Way system
\citep[e.g.,][]{erkal+2019, cautun+2019, garavito-camargo+2021, vasiliev2023}.
With a mass up to one fifth of the MW \citep[e.g.,][]{penarrubia+2015},
the LMC infall affects conclusions about the MW properties and the
orbits of satellites \citep[see
e.g.,][]{patel+2020, battaglia+2022, correamagnus+vasiliev2022}. In this
section, we examine how the LMC affects the orbital history of Sculptor.

We use the \texttt{L3M11} model of the MW and LMC potential from
\citet{vasiliev2024}. The \texttt{L3M11} potential is an evolving
multipole approximation of an N-body simulation including a live MW and
LMC dark matter halo. The potential includes a static MW bulge and disk,
evolving MW and LMC halos, and the MW reflex motion. In their
simulation, the MW was initially a NFW halo with \(r_s=16.5\,\)kpc and
\(M_{\rm 200}= 98.4\times10^{10}\Mo\) , and the LMC a NFW halo with
\(r_s=11.7\) and \(M_{200} = 24.6 \times 10^{10} \Mo\). The total
\texttt{L3M11} MW mass is lighter than our initial \citet{EP2020}
potential. Notably, this model has a previous LMC pericentre
\(\sim6\,\Gyr\) ago.

The inclusion of the LMC reshapes Scl's orbital history, as shown in
Fig.~\ref{fig:scl_lmc_orbits_effect}. In the MW-only potential, Scl's
orbit is typical of a long-term MW satellite. However, Scl's passage by
the LMC \(~\sim0.1\,\Gyr\) ago affects the long-term orbit---Scl instead
reaches an apocentre of nearly \(300\,\kpc\). Scl may even be on first
infall, depending on the MW and LMC mass. Scl's orbit is
counter-rotating to the LMC, so Scl is unlikely to be a LMC satellite.

The timing of the LMC encounter implies a break radius consistent with
the observed density kink. We select the orbit with the
\(6\sigma\)-smallest LMC pericentre. The orbit is selected following the
procedure in Section~\ref{sec:orbital_estimation} except with
observational uncertainties doubled. Results are similar when selecting
for a small MW pericentre instead. This \texttt{LMC-flyby} orbit, is
integrated back in time \(2\,\Gyr\) ago to isolate recent tidal effects.
We modify Scl's initial halo to have \(\rmax = 2.5\,\kpc\) and
\(\vmax = 25\,\kms\), slightly reducing the initial stellar velocity
dispersion. Fig.~\ref{fig:scl_lmc_orbits_effect} shows this selected
orbit in black and Table~\ref{tbl:orbit_ics} records the initial
conditions.

\begin{table*}[t]
\centering
\caption[Orbits and results for Scl in the LMC potential.]{The orbital properties and dark matter evolution for the models including an LMC. Similar to Table \ref{tbl:scl_sim_results} except quantities with respect to the LMC are in parentheses. }
\label{tbl:scl_lmc_sims}
\begin{tabular}{lll}
\toprule
Property & random samples & \texttt{LMC-flyby}\\
\midrule
pericentre & $44\pm 3$ ($29 \pm 2$) & 39 (20)\\
apocentre & $218 \pm 8$ & –\\
time of last pericentre / Gyr & $-0.38\pm0.01$ (-0.11) & -0.33 (-0.10)\\
number of pericentres & 1 (2) & 1 (1)\\
Jacobi radius / kpc & $4.1\pm0.3$ ($4.5\pm0.2$) & 2.8 (2.6)\\
Jacobi radius / arcmin & $168 \pm 11$ ($186\pm6$) & 132 (121)\\
final heliocentric distance / kpc & $83.2\pm2$ & 72.9\\
$\V_\textrm{max, f} / \V_\textrm{max, i}$ &  & 0.928\\
$r_\textrm{max, f} / r_\textrm{max, i}$ &  & 0.763\\
fractional final bound mass &  & 0.5402\\
\bottomrule
\end{tabular}
\end{table*}

\begin{figure}
\centering
\pandocbounded{\includegraphics[keepaspectratio]{figures/scl_lmc_xyzr_orbits.png}}
\caption[Sculptor orbits with LMC]{Similar to Fig.~\ref{fig:scl_orbits}
except for orbits with (orange) and without (green lines) the inclusion
of an LMC (blue line) in the potential. The bottom row additionally
shows the distance between Scl and the LMC over
time.}\label{fig:scl_lmc_orbits_effect}
\end{figure}

\subsection{Tidal effects from the
LMC}\label{tidal-effects-from-the-lmc}

The combined tidal force of the MW and LMC are even weaker for Scl than
in the MW-only case. Fig.~\ref{fig:scl_lmc_sim_images} shows the dark
matter evolution of Scl and the passage of the LMC. With only one MW
pericentre, Scl's dark matter is less disrupted than the previous
MW-only model. The subsequent LMC passage has little effect. As a
result, the dark matter structure evolves mildly and \(\sim 50\%\) of
mass remains bound (Table~\ref{tbl:scl_lmc_sims}).

Correspondingly, the stellar component is nearly unchanged.
Fig.~\ref{fig:scl_lmc_i_f} shows the projected stellar distributions and
density profiles of this model. While the break radius is within the
observed density profile, tidal effects are too weak to be detectible.
Structural properties of the stars similarly evolve little
(Table~\ref{tbl:scl_lmc_sim_stars}). While the instantaneous tidal force
from the LMC is larger than the MW, Scl does not experience the LMC
tidal field long enough to display disturbances.

\begin{figure}
\centering
\pandocbounded{\includegraphics[keepaspectratio]{figures/scl_lmc_sim_images.png}}
\caption[Sculptor simulation snapshots with LMC]{Similar to
Fig.~\ref{fig:scl_sim_images} except for the case where the potential
includes an LMC. The current position and path of the LMC are
represented by the green dot and line respectively. We also plot the
full orbit (over the past 2Gyr) for both Scl and the LMC, as less than
one radial period happens over this time
frame.}\label{fig:scl_lmc_sim_images}
\end{figure}

\begin{figure}
\centering
\pandocbounded{\includegraphics[keepaspectratio]{figures/scl_lmc_i_f.pdf}}
\caption[Sculptor initial and final density with LMC]{Similar to
Fig.~\ref{fig:scl_smallperi_i_f} except for the \texttt{LMC-flyby}
model. With only one MW pericentre and a recent, rapid LMC encounter,
tidal forces do not appear to affect the stellar
distribution.}\label{fig:scl_lmc_i_f}
\end{figure}

\begin{table*}[t]
\centering
\caption[Simulation results for Sculptor’s stars in the LMC+MW potential]{Similar to Table \ref{tbl:scl_sim_stars_results}, but for the properties of the stellar components of the \texttt{LMC-flyby} model of Sculptor. }
\label{tbl:scl_lmc_sim_stars}
\begin{tabular}{lll}
\toprule
Property & Scl: LMC-exponential & LMC-Plummer\\
\midrule
$\sigma_{\V, i}\,/\,\kms$ & 9.0 & 9.4\\
$\sigma_{\V, f} \,/\,\kms$ & 8.8 & 9.2\\
fractional stellar mass loss & $<10^{-12}$ & 0.0013\\
$R_{h, i}$ / kpc & 0.186 & 0.201\\
$R_{h, f}$ / kpc & 0.189 & 0.205\\
break radius & $77'$, 1.6 kpc & $80'$, 1.7 kpc\\
LMC break radius & $23'$, 0.49 kpc & $24'$, 0.52 kpc\\
\bottomrule
\end{tabular}
\end{table*}

\subsection{Summary}\label{summary}

We find, including only the MW, that tides only remove dark matter from
the outskirts of Scl. The central cusp and compact stellar distribution
are resilient to tides and any tidal effects are well-outside the reach
of current observations. We then find that the LMC strongly perturbs
Scl's orbit---in this case, Scl may be on first infall. However, with
only 2 pericentres, the combined tides of the LMC and MW tides are
weaker than for our initial model. In either case, tides are unlikely to
affect Sculptor's stellar component.

\section{Tidal effects on Ursa Minor}\label{tidal-effects-on-ursa-minor}

The tidal evolution of Ursa Minor is similar to Sculptor in the MW-only
case. Fig.~\ref{fig:umi_sim_images} shows snapshots of the DM evolution.
UMi loses more DM mass, with only about 3\% of the total mass remaining
(see Fig.~\ref{fig:umi_tidal_track} and
Table~\ref{tbl:umi_sim_results}). However, tidal features in the
observed stellar components are still extremely faint, appearing outside
100 arcminutes in Fig.~\ref{fig:umi_smallperi_i_f}. Tides only
moderately affect the observed size and velocity dispersion of Ursa
Minor (Table~\ref{tbl:umi_sim_stars_results}).

A stream may just be beyond observational limits if Ursa Minor's stars
were initially more extended. As seen in Fig.~\ref{fig:umi_plummer_i_f},
the tidal tails are more luminous than for Sculptor and would begin near
the outermost density measurement. The properties of the stream are
shown in Fig.~\ref{fig:umi_tidal_stream}. The stream stars become
apparent outside of about 2 degrees, or the Jacobi radius. As expected
for tidal tails, the stream shows gradients in each observable, with a
dispersion in each observable similar to that of the dwarf progenitor.

\begin{table*}[t]
\centering
\caption[Simulation results for Ursa Minor’s dark matter]{The present-day properties for Ursa Minor’s final dark matter halo. See Table \ref{tbl:scl_sim_results} for details. }
\label{tbl:umi_sim_results}
\begin{tabular}{lll}
\toprule
Property & Random orbits & \smallperi{}\\
\midrule
pericentre & $37\pm3$ & 30\\
apocentre & $83 \pm 4$ & 75\\
time of last pericentre & $-0.97 \pm 0.07$ & -0.80\\
number of pericentres & $\sim 8$ & 8\\
Jacobi radius / kpc & $3.7 \pm 0.3$ & 2.9\\
Jacobi radius / arcmin & $184 \pm 12$ & 156\\
final heliocentric distance & $70.1 \pm 3.6$ & 64.7\\
${\vmax}_f / {\vmax}_i$ &  & 0.511\\
${\rmax}_f / {\rmax}_i$ &  & 0.249\\
fractional dm final mass &  & 0.035\\
\bottomrule
\end{tabular}
\end{table*}

\begin{table*}[t]
\centering
\caption[Simulation results for Ursa Minor’s stars]{Similar to Table \ref{tbl:scl_sim_stars_results}, the present-day stellar properties for the simulation of Ursa Minor for exponential and Plummer stars. }
\label{tbl:umi_sim_stars_results}
\begin{tabular}{lll}
\toprule
Property & smallperi-exp & smallperi-Plummer\\
\midrule
$\sigma_{\V, i}\,/\,\kms$ & 10.0 & 10.9\\
$\sigma_{\V, f}\,/\,\kms$ & 8.2 & 8.5\\
fractional stellar mass loss & $0.00015$ & 0.039\\
$R_{h, i}\,/\,\kpc$ & 0.135 & 0.151\\
$R_{h, f}\,/\,\kpc$ & 0.169 & 0.191\\
break radius & 197 arcmin, 3.7 kpc & 204 arcmin, 3.8 kpc\\
\bottomrule
\end{tabular}
\end{table*}

\begin{figure}
\centering
\pandocbounded{\includegraphics[keepaspectratio]{figures/umi_sim_images.png}}
\caption[Ursa Minor simulation snapshots]{Similar to
Fig.~\ref{fig:scl_sim_images} except for Ursa Minor on the \smallperi{}
orbit.}\label{fig:umi_sim_images}
\end{figure}

\begin{figure}
\centering
\pandocbounded{\includegraphics[keepaspectratio]{figures/umi_tidal_track.png}}
\caption[Ursa Minor tidal tracks]{Similar to
Fig.~\ref{fig:scl_tidal_track} except for Ursa Minor. Ursa Minor looses
substantially more mass than Sculptor.}\label{fig:umi_tidal_track}
\end{figure}

\begin{figure}
\centering
\pandocbounded{\includegraphics[keepaspectratio]{figures/umi_smallperi_i_f.pdf}}
\caption[Ursa Minor simulated density profiles]{The tidal effects on the
stellar surface density of Ursa Minor for exponential stars on the
\smallperi{} orbit.}\label{fig:umi_smallperi_i_f}
\end{figure}

\begin{figure}
\centering
\pandocbounded{\includegraphics[keepaspectratio]{figures/umi_plummer_i_f.pdf}}
\caption[Ursa Minor Plummer model density]{The tidal effects on the
stellar surface density of Ursa Minor for Plummer stars on the
\smallperi{} orbit.}\label{fig:umi_plummer_i_f}
\end{figure}

\begin{figure}
\centering
\pandocbounded{\includegraphics[keepaspectratio]{figures/umi_sim_stream.pdf}}
\caption[Ursa Minor predicted stream]{The properties of the stream
around the UMi \smallperi{} orbit with Plummer stars. The panels are all
as a function of \(\xi'\), the distance along the stream as defined by
the current GSR proper motion vector. The top panels show the GSR proper
motions in RA and Dec, and the bottom two show the distance and GSR
radial velocities. To sample the stream, we randomly draw 100,000
samples from the snapshot based on the stellar weights. A detectible
gradient in \(\mu_{\alpha*}\) and LOS velocity should be detectible if
the stream is tracked across several
degrees.}\label{fig:umi_tidal_stream}
\end{figure}

\subsection{Effects of the LMC}\label{effects-of-the-lmc}

Fig.~\ref{fig:umi_orbits_lmc} shows the effects of including an LMC on
the orbit of Ursa Minor. Predominantly, the effect is to increase the
orbital period, apocentre and pericentre. However, the orbit remains in
a similar plane and with similar morphology. As UMi is on the opposite
side of the Galaxy of the LMC, this is not unexpected. The deviation
from the MW-only orbit is mostly due to the LMC-induced reflex motion of
the Milky Way.

\begin{figure}
\centering
\pandocbounded{\includegraphics[keepaspectratio]{figures/umi_lmc_xyzr_orbits.png}}
\caption[Ursa Minor orbits with LMC]{Orbits of Ursa Minor with (orange)
and without (green) an LMC. The final positions of Ursa Minor and the
LMC are plotted as scatter points and the solid blue line represents the
LMC trajectory. Note that the LMC mostly increases Ursa Minor's
pericentres and apocentres.}\label{fig:umi_orbits_lmc}
\end{figure}

\subsection{Summary}\label{summary-1}

While tides affect UMi more strongly than Scl, the tidal effects are
insufficient to reshape the observed stellar density profile. Faint
tidal tails may be observable with deeper data. Finally, including the
LMC in the potential further weakens the tides experience by UMi.

\section{Modeling uncertainties}\label{modeling-uncertainties}

\subsection{Modelling assumptions}\label{modelling-assumptions}

As Section~\ref{sec:results} shows, tides marginally affected the
stellar components of Scl and UMi, even under extreme orbits. While
Section~\ref{sec:results} only presents select models, alternative
initial conditions do not affect our qualitative conclusions on tidal
effects.

We neglect baryonic physics. Since Scl and UMi have predominantly stars
older than \(\sim 9\) Gyr
\citep{carrera+2002, deboer+2011, weisz+2014, delosreyes+2022, sato+2025},
gas dynamics are unlikely to affect recent evolution. A collisionless
dark-matter only simulation should be an excellent approximation.

Cored or less concentrated dark matter halos disrupt quicker
\citep[e.g.,][]{stucker+2023}. Our fiducial UMi halo in particular is
among the least concentrated halos consistent with UMi's velocity
dispersion. Although Scl's fiducial halo is more concentrated, less
concentrated and cored halos evolve similarly (see
Section~\ref{sec:extra_results}).

Galaxies are rarely perfect isotropic spheres. Sculptor and Ursa Minor
are elliptical, and halos are expected to be radially anisotropic
\citep[e.g.,][]{navarro+2010}. We test example non-spherical and
anisotropic models in Section~\ref{sec:extra_results}, finding that
these assumptions likely do not alter our conclusions.

A system's observed velocity dispersion directly constrains the mean
density within \(R_h\) \citep[e.g.,][]{wolf+2010}. Thus, the tidal force
required to disrupt the stellar component does not strongly depend on
the inner halo structure (via the Jacobi radius). Alternative initial
conditions may influence the total mass evolution but would produce a
similar final stellar structure.

\subsection{Orbital uncertainties}\label{orbital-uncertainties}

The long-term orbits of satellites are uncertain. Analytic Milky Way
potentials neglect many unknown details, including triaxiality, halo
twisting, mass evolution, and substructure. As a result, calculated
orbits may diverge significantly from the true orbits of satellites
\citep[e.g.,][]{dsouza+bell2022}. The mass-growth of the Milky Way and
dynamical friction implies that orbits are typically less bound in the
past (less affected by tides). However, orbital energy and angular
momentum are not conserved in cosmological simulations. Orbits in
analytic potentials may overestimate the pericentre and underestimate
the maximum tidal stress \citep[although typically not by enough to
change our conclusions,][]{santistevan+2023, santistevan+2024}.

Fig.~\ref{fig:scl_orbit_lmc_uncert} illustrates the uncertainties of the
long-term orbits of Scl and UMi. Scl distance could range from
4--\(300\,\kpc\) any time more than 4Gyr ago. And, if the LMC had a
previous pericentre, UMi may have been an LMC satellite
\citep[see][]{vasiliev2024}. Notably, these potentials are still
idealized in many ways.

Motivated by Fig.~\ref{fig:scl_orbit_lmc_uncert}, we examine an extreme
pericentre of \(4\,\kpc\) of Scl with the MW in Appendix
\ref{sec:extra_results}, finding it still insufficient to produce the
observed density profile. We conclude our simulated orbits represent
reasonable extremes for \emph{recent} tidal effects. Past encounters
with the LMC are revisited below as a form of ``pre-processing'' in
Section~\ref{sec:stellar_halos}.

\begin{figure}
\centering
\pandocbounded{\includegraphics[keepaspectratio]{/Users/daniel/thesis/figures/scl_lmc_orbits_mass_effect.png}}
\caption[Sculptor Orbits with LMC]{The long-term orbital history of
Sculptor (\textbf{top}) and UMi (\textbf{bottom}) are uncertain. In both
panels, light, transparent lines represent randomly-sampled orbit of the
satellites (alla ref) in three different LMC/MW mass models from
\citet{vasiliev2024}. The LMC orbits are in solid, thick lines of the
corresponding colour. The L2M11 has a lighter LMC mass, and the L3M10
model has a lighter MW mass than our fiducial L3M11 LMC
model.}\label{fig:scl_orbit_lmc_uncert}
\end{figure}

\subsection{Summary}\label{summary-2}

While the long-term tidal evolution is unconstrained, we conclude that
our models are reasonable representations of recent tidal effects. As a
result, recent tides are unlikely to affect the stellar distributions of
Sculptor or Ursa Minor.


%\chapter{The effects of tides on the Ursa Minor Dwarf Spheriodal Galaxy}
%\input{ursa_minor}

\chapter{Discussion}\label{sec:discussion}
In the previous chapters, we have shown that (1) Sculptor and Ursa Minor
have unusually extended density profiles for MW classical satellites and
(2) recent tides are an unlikely explanation. Here, we first consider
the reliability and assumptions supporting our conclusions. We then
consider possible formation scenarios of extended density profiles,
including multi-epoch star formation, dwarf mergers and subhalo
encounters, and alternative theories of dwarf galaxy structure. We
finish with an outlook on prospects for disentangling various stellar
halo formation scenarios.

\section{Comparison with prior work}\label{comparison-with-prior-work}

For both Scl and UMi, these galaxies have been studied extensively in
both a theoretical and observational context. While many works
considered mass modelling, star formation histories, chemistry, and
other observational properties, we focus on the observations and models
concerning tides and dynamical evolution here.

While tides have been a popular explanation for some features of dwarf
galaxies \citep[discussion above, and
e.g.,][]{tsujimoto+shigeyama2002, mayer+2001a}, our work, and many
others, have disfavoured this for Scl, UMi, and many other MW dwarfs.
For example, \citet{read+2006} suggest that the lack of observational
evidence for a rising velocity dispersion profile with radii indicates
most dwarfs are not tidally affected. Later, \citet{penarrubia+2009}
define and use the break radius, Eq.~\ref{eq:r_break}, to show that the
(then understood) orbits of most satellites are inconsistent with a
tidal feature. In addition, \citet{pace+erkal+li2022} use a criterion
based on the observed mean density and orbital pericentre (like the
Jacobi radius) to show most dwarfs are unlikely to be undergoing
significant tidal disruption. Most recently,
\citet{tchiorniy+genina2025} use a similar idealized framework to ours,
with a focus on the inner density, showing tides do not strongly affect
equilibrium assumptions for several classical dwarfs. The general tidal
evolution of our simulations is furthermore consistent with much of the
literature \citep[e.g.,][]{robles+bullock2021, EN2021}. Tides seem to
now play a subdominant role in the evolution of most Milky Way
satellites.

However, some cosmological simulations have suggested that tidal effects
may be ubiquitous. \citep{riley+2024, shipp+2023} find that a majority
of satellites form substantial streams, and not due to resolution
effects \citep[see also][]{panithanpaisal+2021}. This result appears to
be in tension with ours and many previous works. Some reasons
cosmological and idealized simulations may give opposite results could
include resolution effects \citep[e.g.,][]{santos-santos+2025},
additional perturbations from dark matter substructure, tidal
``pre-processing'' by other infalling satellites, or the streams may be
below detection limits \citep[e.g.,][]{shipp+2023}. More work is needed
to understand why streams appear to form more readily in cosmological
simulations than is observed in the Milky Way or through idealized
simulations.

\subsection{Sculptor}\label{sculptor}

Scl has long been speculated to be disturbed. \citet{innanen+papp1979}
found RR Lyrae candidate Scl members \citep[from][]{vanagt1978} out to
180' in an elongated distribution, speculating this to be tidal
disruption. Many later density profile determinations noted Scl's
elongation and apparent outer density excess similar to J+24's detection
\citep[but see also
\citet{coleman+dacosta+bland-hawthorn2005}]{eskridge1988, IH1995, walcher+2003, westfall+2006}.
Scl's ellipticity and extended density profile were often interpreted as
evidence of tidal debris or sometimes a dwarf galaxy halo.

In addition, Scl hosts at least two populations, as revealed through
photometry \citep{tolstoy+2004}, kinematics
\citep{battaglia+2008, tolstoy+2023, arroyo-polonio+2024}, dynamical
structure \citep{breddels+helmi2014}, age gradients \citep{deboer+2011},
and chemistry \citep{kirby+2009}. These two populations both have radii
smaller than the outer density excess. Scl challenges the idea of a
simple, single-component dwarf spheroidal population and hints at
episodic star formation or hierarchical assembly.

Our work is inconsistent with claims of observational tidal disruption.
Closely related to our work, \citet{iorio+2019} applied idealized N-body
simulations to study tidal effects on Scl. They similarly found weak
tidal effects, even for a dark-matter-free model. However, their orbits
are less well-constrained and they did not consider the LMC. If the
extended density profile of Scl is formed through another mechanism
which also creates multiple populations (see discussion below), then
both observational features would become consistent with such an
explanation.

\subsection{Ursa Minor}\label{ursa-minor}

UMi has garnered claims of inner substructure such as stellar or
kinematic ``clumps''. Studies by \citet{olszewski+aaronson1985},
\citet{demers+1995}, \citet{IH1995}, \citet{kleyna+1998},
\citet{battinelli+demers1999}, and \citet{bellazzini+2002} note that
Ursa Minor appears to contain ``clumps'' along its major axis. One clump
has furthermore been shown to be kinematically distinct and colder
\citep[e.g.,][]{pace+2014}. If a star cluster, than the cluster should
disolve in \(\sim 3\) Gyrs provided the UMi's halo is not cored and did
not interact with dark (sub)subhalos---the survival of such substructure
depends on the nature of dark matter \citep{kleyna+2003, lora+2012}.
\citet{wilkinson+2004} additionally find a puzzling drop in the velocity
dispersion in the outskirts. The nature of any clumps or substructure
remains unclear, but \citet{munoz+2018}'s are unable to find signs of
any substructure with modern, deeper photometry.

In addition, several works have found supposed tidal features in UMi.
\citet{hargreaves+1994} first detected a velocity gradient in UMi,
suggestive of tidal disruption. Later, \citet{martinez-delgado+2001}
find that stars extend far beyond the ``tidal radius'' (from a King
profile fit) for Ursa Minor, in the direction of the galaxy's elongation
\citep[see corresponding simulations
by][]{gomez-flechoso+martinez-delgado2003}. \citet{palma+2003} further
showed evidence for S-shaped contours, and an extended population of
``extratidal'' stars. Our density profiles are consistent with these
works. However, strong evidence of tidal disruption has not yet been
found.

As our models disagree with a tidal interpretation for the density
profile of Ursa Minor, the observed density excess likely originate from
another process. However, also find no evidence for a velocity gradient
with current spectroscopic samples Section~\ref{sec:extra_rv_models}.
While the origin of inner substructure in UMi is beyond the scope of
this work, this structure, if real, points towards a complex origin of
Ursa Minor.

\section{Forming a stellar halo}\label{sec:stellar_halos}

As we disfavour a tidal origin of Scl and UMi's extended density
profiles, we consider alternative processes which may form a dwarf
galaxy ``stellar halo''. Some recent works have also confirmed the
presence of extended, likely non-tidal, density profiles. Tucana II is
another dwarf with spectroscopic confirmation of outer ``halo'' stars
out to \(9R_h\) and inconsistent with a tidal origin
\citep{chiti+2021, chiti+2023}. The prevalence of stellar halos around
dwarf galaxies remains an active research topic.

Many dwarf galaxies also host multiple chemodynamical stellar
populations. Typically, older populations are extended, kinematically
hotter, and metal-poo, whereas the younger populations are more compact,
colder, and metal-rich. Examples include Carina
\citep[\citet{fabrizio+2016}, \citet{kordopatis+2016}]{battaglia+2012},
Fornax \citep[\citet{amorisco+evans2012},
\citet{delpino+aparicio+hidalgo2015}]{battaglia+2006}, Sextans
\citep{battaglia+2011, cicuendez+battaglia2018, roederer+2023}, and
Andromeda II
\citep{mcconnachie+arimoto+irwin2007, ho+2012, delpino+2017}. Evidence
of multiple-populations in dwarf galaxies suggests that effects like
galaxy mergers or episodic star formation shape the stellar structure of
dwarf galaxies.

We now explore possible origins of a ``stellar halo'' which may resemble
the extended density profiles in Scl and UMi.

\subsection{Internal processess}\label{internal-processess}

\textbf{Dynamical heating of old stars}. In a galaxy, older stars
generally have higher random velocities (i.e., kinematically ``hotter'')
than younger stars. In dwarf galaxies, several mechanism may heat
stellar components, including supernovae kicks in forming stars,
feedback-driven potential fluctuations
\citep{stinson+2009, maxwell+2012, el-badry+2016, mercado+2021}, or
heating by dark sub-subhalos \citep{penarrubia+2025}. Over time, these
processes naturally produce a more extended, older stellar population.
If internal dynamical heating explains stellar halos, then similar
stellar halos should exist around similar dwarfs.

\textbf{Episodic star formation and feedback.} Dwarf galaxies are
thought to experience bursty star formation--i.e., consisting of several
episodes of intense star formation separated by periods of quiescence
\citep[e.g.,][]{salvadori+ferrara+schneider2008, valcke+derijcke+dejonghe2008, wheeler+2019, azartash-namin+2024}.
Stellar feedback in a dwarf galaxy's shallow potential well drives
oscillations in the star formation rate. Alternatively, re-ionization
may temporarily suspending star formation \citep{benitez-llambay+2015}.
In many simulations, shrinking gas reservoirs form
centrally-concentrated later generations of stars, naturally spawning
multiple populations \citep{kawata+2006, revaz+jablonka2018}. If
episodic or vigorous star formation helps create stellar halos, then
corresponding evidence of bursty star formation histories should be
apparent.

\subsection{External processes}\label{external-processes}

\textbf{Induced star formation.} Star formation may quench and reignite
due to an external perturbation. Examples include tidal compression
\citep{mayer+2001a, dong+lin+murray2003}, gaseous filaments
\citep{genina+2019}, dark halos \citep{starkenburg+helmi+sales2016}, or
shocks with the MW corona \citep{wright+2019}. Induced burst would be
more stochastic than internal bursts, possibly explaining diversity in
dwarf galaxy structure.

\textbf{Major mergers.} When galaxies merge, they may leave signatures
such as population gradients and stellar halos. Classical dwarfs have a
\(\sim 10\%\) chance of undergoing a major merger since redshift \(z=1\)
\citep{deason+wetzel+garrison-kimmel2014}. Mergers may preferentially
disperse the lower-mass galaxy's stars, forming a halo and a metallicity
gradient \citep{benitez-llambay+2016}. Intermediate-mass mergers (with
mass-ratios of \(\sim\) 1:5) most effectively populate halos, balancing
stellar mass from larger galaxies with better dispersion of lower-mass
galaxies \citep{deason+2022}. Tuc II's properties are suggested to
originate from a similar merger
\citep{tarumi+yoshida+frebel2021, querci+2025}. Major mergers should
produce multiple chemodynamical populations consistent with two
distinct, but similar mass dwarf galaxies.

\textbf{Gas-rich mergers.} If a merger occurs between gas-rich galaxies,
triggered star formation may occur in the aftermath
\citep[e.g.,][]{genina+2019}. And II and Phoenix have steep metallicity
gradients and unusual prolate rotation, theorized to result from mergers
of disky dwarfs \citep{lokas+2014, fouquet+2017, cardona-barrero+2021}.
Gas-rich mergers should have at least 3-distinct chemodynamical
populations, one from each galaxy and one from the induced starburst.

\textbf{Minor mergers / accretion}. Like how the Milky Way's halo is
believed to be built from many minor mergers, dwarf galaxies may form
halos through accretion of yet fainter dwarfs.
\citet{ricotti+polisensky+cleland2022} demonstrated that accretion of
ultra-faint dwarfs may create dwarf stellar halos. A stream detected
around And II further supports the existence of mergers among dwarfs
\citep{amorisco+evans+vandeven2014, roederer+2023}. The occurrence of
dwarf mergers would further constrain small-scale galaxy formation. In
this scenario, the halo chemistry would resemble a population of
ultra-faint dwarfs.

\textbf{Tidal preprocessing}. Dwarf galaxies may have been
``preprocessed'' by a larger satellite like the LMC
\citep[e.g.,][]{santistevan+2023, riley+2024}. From the orbit
integrations above, it is possible that UMi was once an LMC satellite.
However, the prevalence of preprocessing remains uncertain. Like stellar
heating, preprocessing redistributes already-present stellar populations
without creating new populations, but would occur more stochastically.
Distant, unbound stream stars may be found around a preprocessed dwarf.

\textbf{Tidal dwarf galaxies} are cluster-like objects which form in
gas-rich tidal streams created during the merger of two galaxies
\citep[e.g.,][]{mirabel+dottori+lutz1992, bournaud+duc2006}. Tidal
dwarfs may be more susceptible to tides, forming extended density
profiles and appearing to have dark matter
\citep{casas+2012, yang+2014, wang+2024a}. If Scl and UMi are tidal
dwarfs, a stronger velocity (dispersion) gradient and tidal tails should
be detectible.

\subsection{\texorpdfstring{Beyond \LCDM{}}{Beyond }}\label{beyond}

\textbf{Modified Newtonian Dynamics (MOND)}. MOND modifies gravity
instead of using dark matter to explain the rotation curves of galaxies.
In MOND, tides may more strongly affect dwarf galaxies owing to the lack
of dark matter mass and the stronger MW tidal field
\citep{mcgaugh+wolf2010, brada+milgrom2000}. A tidal origin of the
density excess is more likely in this case. If MOND is to recover the
observed velocity dispersions of satellites, then many more dwarfs may
be actively tidally disrupting
\citetext{\citealp{mcgaugh+wolf2010}; \citealp[but see
also][]{sanchez-salcedo+hernandez2007}}.

\textbf{Self-interacting dark matter (SIDM).} SIDM significantly
complicates tidal evolution. SIDM halos in isolation are not static---by
transferring heat through collisions, these halos first undergo core
formation and then core collapse. Besides structural changes, SIDM adds
that pressure from the host DM halo can change the structure
(e.g.~aiding core collapse) or remove material from the inner galaxy
(analogous to ram pressure stripping) \citep[e.g.,][]{cartonzeng+2024}.
An SIDM halo may be more strongly affected by tides, but the velocity
dispersion makes a tidal disruption in SIDM still unlikely.

\textbf{Fuzzy dark matter} can also heat up stars owing to density
perturbations in the form of interference fringes
\citep[e.g.,][]{el-zant+2020, duttachowdhury+2023}. Similar to other
intrinsic heating methods, this model would likely affect all dwarfs
similarly, so we should detect similar halos around similar dwarfs.

\subsection{Disentangling the origin of a stellar
halo.}\label{disentangling-the-origin-of-a-stellar-halo.}

We just reviewed a number of different, possibly-concurrent explanations
for the formation of a stellar halo. While we leave the nature of Scl
and UMi's extended stellar densities an open question, we can discuss
possible clues to different formation scenarios.

\emph{Chemistry}. Large samples of detailed chemical abundances have
been invaluable for understanding MW substructure (e.g.~for
\emph{Gaia}-Sausage Enceladus). Chemistry, particularly comparing the
inner and outer regions could test if the halo appears to originate from
a distinct system than the dwarf, differentiating many \emph{internal}
versus \emph{external} scenarios.

\emph{Star formation histories}. Evidence of significant star formation
episodes or lack thereof may differentiate scenarios which rely on
strong star formation bursts (e.g.~\emph{episodic star formation
history, induced star formation, and gas-rich mergers.})

\emph{Kinematics}. Particularly, for models relying on recent tidal
disruption, kinematic disequilibrium features should be visible. These
would appear as velocity gradients, increasing velocity dispersions,
outward-biased moving stars, or non-phase mixed structures
\citep[e.g,][]{kroupa1997, read+2006, sanchez-salcedo+hernandez2007}.

\emph{Deep photometry} may find or rule out signs of dynamical
disequilibrium and tidal tails for tidally susceptible models
\emph{(e.g., MOND and tidal dwarfs.)} In addition, photometry will help
constrain the prevalence and nature of extended features in dwarf
galaxies.

Ongoing, upcoming, and future facilities will be essential for testing
these theories. For example, given the typical faint magnitudes of dSph
stars, chemistry is best accessible through large field-of-view
multi-object spectroscopic instruments on large telescopes (e.g., 4MOST
and extremely large telescopes \citet{skuladottir+2023}). 3D internal
kinematics of dwarfs may require a successor to \emph{Gaia}. However,
using JWST and HST enables precise proper motions for small regions
within dwarf galaxies \citep[e.g.,][]{vitral+2025}, which may be
sufficient to constrain vastly different internal dynamic structures and
ongoing tidal disruption. Finaly, Photometric surveys by Rubin
Observatory and Euclid will most-emminently probe yet fainter magnitudes
around dwarf galaxies and possibly find or constrain stellar halos and
their star formation histories. Altogether, new surveys will likely
uncover novel aspects about the inner workings and outer halos of dwarf
galaxies.

In addition to new observing facilities, the next generation of dwarf
simulations are beginning to answer questions and unravel processes in
the formation of dwarf galaxies. With improved resolution, realism, and
physics, the frequency and effects of various mechanisms discussed can
be better constrained.

\section{Conclusion and outlook}\label{conclusion-and-outlook}

In this thesis, we have investigated the extended, outer density excess
of Sculptor and Ursa Minor and its possible tidal origin. We first
verified that Scl and UMi have unusually extended density profiles,
compared to other dwarf spheroidals. We show that the density profiles
are robust to alternate data criteria, implying that this ``density
excess'' is likely a real feature of each galaxy.

We then investigated if tides were a permissible explanation. By
modelling each galaxy based on cosmological initial conditions, we
showed that tides do not strongly affect either galaxy. The LMC changes
the orbital history of Scl and UMi, and tides become even weaker in a
combined LMC and MW potential. While UMi may form a stellar stream, the
stream is far fainter than is presently detectible. We conclude that
recent tides are unlikely to shape the observed stellar distributions of
Scl and UMi.

Finally, we consider alternative scenarios forming extended density
distribution. We review mechanisms ranging from episodic star formation
histories, to accretion, to departures from \LCDM{}. While a more
precise explanation awaits, upcoming and ongoing surveys will uncover
the stellar populations within dwarf galaxies in unprecedented detail.
Future work will the Milky Way, uncovering the cosmic origins of the
smallest structures forming building blocks of galaxies like ours.


\chapter{Summary and Outlook}\label{sec:summary}

%%%%%%%%%%%%%%%%%%%% REFERENCES %%%%%%%%%%%%%%%%%%

\newpage
\bibliographystyle{aasjournal}
\addcontentsline{toc}{chapter}{Bibliography}
\bibliography{main} 


%%%%%%%%%%%%%%%%% APPENDICES %%%%%%%%%%%%%%%%%%%%%

\appendix

\chapter{Data selection}\label{data-selection}

\section{Describution of algorithm}\label{describution-of-algorithm}

To create a high-quality sample, J+24 select stars initially from Gaia
within a 2--4 degree circular region centred on the dwarf satisfying:

\begin{itemize}
\tightlist
\item
  Solved astrometry, magnitude, and colour.
\item
  Renormalized unit weight error, \({\rm ruwe} \leq 1.3\), ensuring high
  quality astrometry. \texttt{ruwe} is a measure of the excess
  astrometric noise on fitting a consistent parallax-proper motion
  solution \citep[see][]{lindegren+2021}.\\
\item
  3\(\sigma\) consistency of measured parallax with dwarf's distance
  (dwarf parallax is very small; with \citet{lindegren+2021} zero-point
  correction).
\item
  Absolute proper motions, \(\mu_{\alpha*}\), \(\mu_\delta\), less than
  10\(\,{\rm mas\ yr^{-1}}\). (Corresponds to tangental velocities of
  \(\gtrsim 500\) km/s at distances larger than 10 kpc.)
\item
  Corrected colour excess is within 3\(\sigma\) of the expected
  distribution from \citet{riello+2021}. Removes stars with unreliable
  photometry.
\item
  De-reddened \(G\) magnitude is between
  \(22 > G > G_{\rm TRGB} - 5\sigma_{\rm DM}\). Removes very faint stars
  and stars significantly brighter than the tip of the red giant branch
  (TRGB) magnitude plus the distance modulus uncertainty
  \(\sigma_{\rm DM}\).
\item
  Colour is between \(-0.5 < {\rm BP - RP} <  2.5\) (dereddened).
  Removes stars substantially outside the expected CMD.
\end{itemize}

Photometry is dereddened with \citet{schlegel+finkbeiner+davis1998}
extinction maps.

J+24 define likelihoods \({\cal L}\) representing the probability
density that a star is consistent with either the MW stellar background
(\({\cal L}_{\rm bg}\)) or the satellite galaxy
(\({\cal L}_{\rm sat}\)). In either case, the likelihoods are the
product of a spatial, PM, and CMD term: \begin{equation}{
{\cal L} = {\cal L}_{\rm space}\ {\cal L}_{\rm PM}\ {\cal L}_{\rm CMD}.
}\end{equation}

Each likelihood is normalized over their respective 2D parameter space
for both the satellite. To control the relative frequency of member and
background stars, \(f_{\rm sat}\) representing the fraction of member
stars in the field. The total likelihood for any star in this model is
the sum of the satellite and background likelihoods, weighted by their
relative frequencies,
\begin{equation}\protect\phantomsection\label{eq:Ltot}{
{\cal L}_{\rm tot} = f_{\rm sat}{\cal L}_{\rm sat} + (1-f_{\rm sat}){\cal L}_{\rm bg}.
}\end{equation} The probability that any star belongs to the satellite
is then given by \begin{equation}{
P_{\rm sat} = 
\frac{f_{\rm sat}\,{\cal L}_{\rm sat}}{{\cal L}_{\rm tot}}
= \frac{f_{\rm sat}{\cal L}_{\rm sat}}{f_{\rm sat}{\cal L}_{\rm sat} + (1-f_{\rm sat}){\cal L}_{\rm bg}}.
}\end{equation}

For the satellite's spatial likelihood, J+24 consider both one-component
and a two-component density models. The one component model is
constructed as a single exponential profile ( surface density
\(\Sigma \propto e^{R_{\rm ell} / R_s}\)), with scale radius \(R_s\)
fixed to the value in table 1 of \citet{MV2020a} from \citet{munoz+2018}
(for a Sérsic fit). Additionally, structural uncertainties (for position
angle, ellipticity, and scale radius) are sampled over to construct the
final likelihood map. The two-component model instead adds a second
exponential,
\(\Sigma_\star \propto e^{-R/R_s} + B\,e^{-R/R_{\rm outer}}\). The inner
scale radius is fixed, and the outer scale radius and magnitude of the
second component \(R_{\rm outer}\), \(B\) are free parameters.
Structural property uncertainties are not included in the two-component
model.

The PM likelihood is a bivariate gaussian with variance and covariance
equal to each star's proper motions. J+24 assume the stellar PM errors
are the main source of uncertainty.

The satellite's CMD likelihood is based on a Padova isochrone
\citep{girardi+2002}. The isochrone has a matching metallicity and 12
Gyr age (except 2 Gyr is used for Fornax). The (gaussian) colour width
is assumed to be 0.1 mag plus the Gaia colour uncertainty at each
magnitude. The horizontal branch is modelled as a constant magnitude
extending blue of the CMD (mean magnitude of -2.2, 12 Gyr HB stars and a
0.1 mag width plus the mean colour error). A likelihood map is
constructed by sampling the distance modulus in addition to the CMD
width, taking the maximum of RGB and HB likelihoods.

The background likelihoods are instead empirically constructed. Stars
stars outside of 5\(R_h\) passing the quality cuts estimate the
background density in PM and CMD space. The density is a sum of
bivariate gaussians with variances based on Gaia uncertainties (and
covariance for proper motions). The spatial background likelihood is
assumed to be constant.

J+24 derive \(\mu_{\alpha*}\), \(\mu_\delta\), \(f_{\rm sat}\) (and
\(B\), \(R_{\rm outer}\) for two-component) through an MCMC simulation
with likelihood from Eq.~\ref{eq:Ltot}. Priors are only weakly
informative. The proper motion single component prior is same as
\citet{MV2020a}: a normal distribution with mean 0 and standard
deviation \(100\ \kms\). If 2-component spatial, instead is a uniform
distribution spanning 5\(\sigma\) of single component case w/ systematic
uncertainties. \(f_{\rm sat}\) (and \(B\)) has a uniform prior 0--1.
\(R_{\rm outer}\) has a uniform prior only restricting
\(R_{\rm outer} > R_s\). The mode of each parameter from the MCMC are
then reported and used to calculate the final \(P_{\rm sat}\) values.

\section{Additional density tests}\label{additional-density-tests}

In this section, we discuss additional tests and verification of the
derived density profiles. In particular, we check that methodology
(simpler cuts, circularized radii, algorithm version) do not
substantially affect the density profile. We also compile density
profiles presented in the literature as reference. In all cases, the
density profiles appear to have excellent convergence out to
\(\log R_{\rm ell} / {\rm arcmin} \approx 1.8\), about the distance
where the background dominates.

Discuss selection criteria for DELVE and UNIONS samples, literature
comparison, simple selection criteria, MCMC density profiles and when
\citet{jensen+2024} becomes background-limited.

\begin{figure}
\centering
\pandocbounded{\includegraphics[keepaspectratio]{figures/scl_density_methods_extra.pdf}}
\caption[Scl density comparison]{Density profiles for various
assumptions for Sculptor. PSAT is our fiducial 2-component J+24 sample,
circ is a 2-component bayesian model assuming circular radii, simple is
the series of simple cuts described, bright is the sample of the
brightest half of stars (scaled by 2), DELVE is a sample of RGB stars
(background subtracted and rescaled to
match).}\label{fig:scl_density_extras}
\end{figure}

\begin{figure}
\centering
\pandocbounded{\includegraphics[keepaspectratio]{figures/scl_density_methods_j24.pdf}}
\caption[Scl density methods]{Comparison of density profiles for each
J+24 method. The fiducial is a 2-component elliptical model. However,
the 1-component is still elliptical but only contains 1 component and
the circular model assumes a circular outer density profile and bins in
circular bins instead of elliptical
bins.}\label{fig:scl_density_j24_methods}
\end{figure}

\begin{figure}
\centering
\pandocbounded{\includegraphics[keepaspectratio]{figures/umi_density_methods_extra.pdf}}
\caption[UMi density comparison]{Similar to
Fig.~\ref{fig:scl_observed_profiles} except for Ursa
Minor}\label{fig:umi_density_extras}
\end{figure}

\begin{figure}
\centering
\pandocbounded{\includegraphics[keepaspectratio]{figures/umi_density_methods_j24.pdf}}
\caption[UMi density methods]{Similar to
Fig.~\ref{fig:scl_density_j24_methods} except for Ursa
Minor.}\label{fig:umi_density_j24_methods}
\end{figure}

\section{Comparison to Literature}\label{comparison-to-literature}

Here, we compare our density profiles against past derivations of
density profiles for Sculptor and Ursa Minor

\begin{figure}
\centering
\pandocbounded{\includegraphics[keepaspectratio]{figures/analytic_profile_comparison.pdf}}
\caption[Comparison of analytic density profiles]{A comparison of
different parameterizations for dwarf galaxy density profiles. Note that
deviations between profiles only become apparent past 3 R\_h, and only
the Plummer profile, in contrast to more commonly assumed profiles,
deviates by \textasciitilde1 dex positive before 6 Rh. Since this
profile is a far minority in the literature, deviations from exponential
and close relatives are interesting and worth further consideration.}
\end{figure}

\chapter{Radial velocity modeling}\label{sec:rv_obs}

\section{Data selection}\label{data-selection}

For both Sculptor and Ursa Minor, we construct literature samples of
radial velocity measurements. We combine these samples with J+24's
members to produce RV consistent stars and to compute velocity
dispersion, systematic velocities, and test for the appearance of
velocity gradients.

First, we crossmatch all catalogues to J+24 Gaia stars. If a study did
not report GaiaDR3 source ID's, we match to the nearest star within 1-3
arcseconds (see REF Table~\ref{tbl:rv_measurements}). We combine the
mean RV measurement from each study using the inverse-variance weighted
mean \(\bar v\), standard uncertainty \(\delta \bar v\), and (biased)
variance \(s^2\). We remove stars with significant velocity dispersions
as measured between observations in a study or between studies. By using
that \(\chi^2=\frac{s^2}{\delta \bar v^2}\), we remove stars with a
\(\chi^2\) larger than the 99.9th percentile of the \(\chi^2\)
distribution with \(N-1\) measurements. This cut typically removes stars
with reduced chi-squared values
\(\tilde\chi^2  = \frac{s^2}{\nu\,\delta \bar v^2}\gtrsim 7\) (since the
number of measurements is 1-3 typically).

Next, we need to correct the coordinate frames for the solar motion and
on-sky size of the galaxy. We transform the frame into the galactic
standard of rest (GSR). The next step is to account for the slight
differences in the direction of each radial velocity. Let the \(\hat z\)
be the direction from the sun to the dwarf galaxy. Then if \(\phi\) is
the angular distance between the centre of the galaxy and the individual
star, the corrected radial velocity is then \begin{equation}{
v_z = v_{\rm los, gsr}\cos\phi  - v_{\alpha}\cos\theta \sin\phi - v_\delta \sin\theta\sin\phi
}\end{equation} where \(v_{\rm los, gsr}\) is the line of sight velocity
in the GSR frame, \(v_\alpha\) and \(v_\delta\) are the tangental
velocities in RA and Dec, and \(\theta\) is the position angle of the
star with respect to the centre of the dwarf. The correction from both
effects induces an apparent gradient of about \(1.3\,\kmsdeg\) for
Sculptor and less for Ursa Minor \citep[see
also][]{WMO2008, strigari2010}. We add the uncertainty in \(v_z\) from
the distance uncertainty and velocity dispersion in quadrature to the RV
uncertainties for each star. We then use the \(v_z\) values for the
following modelling, however repeating with uncorrected, heliocentric
velocities does not significantly affect the results.

The combined likelihood, including RV information, becomes
\begin{equation}{
{\cal L} = {\cal L}_{\rm space} {\cal L}_{\rm CMD} {\cal L}_{\rm PM} {\cal L}_{\rm RV}
}\end{equation} where we assume that the satellite and background
distributions are Gaussian. Specifically, \begin{equation}{
\begin{split}
{\cal L}_{\rm RV, sat} &= f\left( \frac{v_i -\mu_{v}}{\sqrt{\sigma_{v}^2 + (\delta v_i)^2}}\right) \\
{\cal L}_{\rm RV, bg} &= f\left( v_i /  \sigma_{\rm halo} \right)
\end{split}
}\end{equation} where \(f\) is the probability density of a standard
normal distribution, \(\mu_v\) and \(\sigma_v\) are the systemic
velocity and dispersion of the satellite, and \(\delta v_i\) is the
individual measurement uncertainty. Typically, the velocity dispersion
will dominate the uncertainty budget here. We assume a halo/background
velocity dispersion of a constant \(\sigma_{\rm halo} = 100\,\kms\)
\citep[e.g.][]{brown+2010}.

Similar to above, we retain stars with the resulting membership
probability of greater than 0.2. Because of the additional information
from radial velocities, most stars have probabilities close to 1 or 0 so
the probability cut is not too significant.

We assume priors on systematic velocity and velocity dispersion of
\begin{equation}{
\begin{split}
\mu_{v} &= N(0\,\kms, \sigma_{\rm halo}^2) \\ 
\sigma_{v} &= U(0, 20\,\kms)
\end{split}
}\end{equation} where \(\sigma_{\rm halo} = 100\,{\rm km\,s^{-1}}\) is
the velocity dispersion of the MW halo adopted above, a reasonable
assumption for dwarfs in orbit around the MW.

\section{Results}\label{sec:rv_results}

\begin{figure}
\centering
\pandocbounded{\includegraphics[keepaspectratio]{figures/scl_umi_rv_fits.pdf}}
\caption[LOS velocity fit to Scl.]{Velocity histogram of Scl and UMi in
terms of \(v_z\) (REF). Orange points are from our crossmatched RV
membership sample.}
\end{figure}

For Sculptor, we combine radial velocity measurements from APOGEE,
\citet{sestito+2023a}, \citet{tolstoy+2023}, and \citet{WMO2009}.
\citet{tolstoy+2023} and \citet{WMO2009} provide the bulk of the
measurements. We find that there is no significant velocity shift in
crossmatched stars between catalogues. After crossmatching to high
quality Gaia stars and excluding significant stellar velocity
dispersions, we have a sample of 1918 members.

We derive a systemic velocity for Sculptor of \(111.3\pm0.2\,\kms\)with
velocity dispersion \(9.64\pm0.16\,\kms\). Our values are very
consistent with previous work \citep[e.g.][\citet{arroyo-polonio+2024},
\citet{battaglia+2008}]{walker+2009}. See appendix REF for a more
detailed comparison between different samples and additional tests.

We detect a moderately significant gradient of \(4.3\pm1.3\,\kmsdeg\) at
a position angle of \(-149_{-13}^{+17}\) degrees (see appendix REF).
Several past work has attempted to detect a gradient in Sculptor, but no
consensus has been reached. \citet{arroyo-polonio+2024} detect a
velocity gradient of \(4\pm1.5\,\kmsdeg\) in a similar direction using
the \citet{tolstoy+2023} sample, finding inconclusive statistical
evidence. They additionally suggest a third chemodynamical component of
the galaxy which may bias rotation measurements. \citet{battaglia+2008}
also detect a \(-7.6_{-2.2}^{+3.0}\,\kmsdeg\) velocity gradient along
the major axis, approximately the same direction. Instead,
\citet{strigari2010}; \citet{martinez-garcia+2023} detect no significant
gradient in Sculptor using \citet{WMO2009} sample. Note that
pre-\emph{Gaia} work did not have as strong of a constraint on the
proper motion of Scl, which limits conclusions of the intrinsic velocity
gradient in Scl.

For UMi, we collect radial velocities from, APOGEE,
\citet{sestito+2023b}, \citet{pace+2020}, and \citet{spencer+2018}. We
shifted the velocities of \citet{spencer+2018} (\(-1.1\,\kms\)) and
\citet{pace+2020} (\(+1.1\,\kms\) ) to reach the same scale. We found
183 crossmatched common stars (passing 3\(\sigma\) RV cut, velocity
dispersion cut, and PSAT J+24 \textgreater{} 0.2 w/o velocities). Since
the median difference in velocities in this crossmatch is about 2.2
km/s, we adopt 1 km/s as the approximate systematic error here. Our
final sample includes 831 members.

We derive a mean \(-245.8\pm0.3_{\rm stat}\,\kms\) and velocity
dispersion of \(8.8\pm0.2\,\kms\) for UMi. This is consistent with
\citet{pace+2020} and to a lesser extent with \citet{spencer+2018}. We
do not find evidence for a velocity gradient, consistent with past work
\citep{pace+2020, martinez-garcia+2023}.

\section{Discussion and limitations}\label{discussion-and-limitations}

Our model here is relatively simple. Some things which we note as
systematics or limitations:

\begin{itemize}
\tightlist
\item
  Inter-study systematics and biases. While basic crossmatches and a
  simple velocity shift, combining data from multiple instruments is
  challenging. This appears to be a minor issue (Sculptor) or is
  corrected for (Ursa Minor).
\item
  Misrepresentative uncertainties. Inspection of the variances compared
  to the standard deviations within a study seems to imply that errors
  are accurately reported. APOGEE notes that their RV uncertainties are
  known to be underestimates but are a small proportion of our sample.
\item
  Binarity. While not too large of a change for classical dwarfs, this
  could inflate velocity dispersions of about \(9\,\kms\) by about
  \(1\,\kms\)\citep{spencer+2017}. Thus, our measurement is likely
  slightly inflated given the high binarity fractions measured in these
  systems \citep[\citet{spencer+2018}]{arroyo-polonio+2023}.
\item
  Multiple populations. Both Sculptor and Ursa Minor likely contain
  multiple populations \citep[\citet{pace+2020},
  \citet{tolstoy+2004}]{arroyo-polonio+2024}. Since we only model a
  single population, and each population may have a different extent and
  velocity dispersion, this could result in biased velocity dispersions.
  However, it is unclear how to uniquely determine an overall velocity
  dispersion in a multi-population system.
\item
  Selection effects. RV studies each have their own selection effects,
  which may affect the resulting dispersion, especially if different
  populations or regions of the galaxy have different velocities or
  velocity dispersions. We do not attempt to correct for this.
\end{itemize}

For both Ursa Minor and Sculptor, we also fit models to only data from
individual surveys (see REF). Since the resulting parameters are very
similar, we conclude that many of the systematic uncertainties are
likely smaller than the present errors or that each large survey has
similar biases.

\section{Velocity modelling and
comparisons}\label{velocity-modelling-and-comparisons}

Here, we describe in additional detail, our methods and comparisons for
RV modelling between studies.

Savage-Dickey calculated Bayes factor using Silverman-bandwidth KDE
smoothed samples from posterior/prior.

\begin{table*}[t]
\centering
\caption[Spectroscopic LOS velocity measurements]{Summary of velocity measurements and derived properties. }
\label{tbl:rv_measurements}
\begin{tabular}{lllllllll}
\toprule
 & Study & Instrument & Nspec & Nstar & Ngood & Nmemb & $\delta v_{\rm med}$ & $R_{\rm xmatch}$/arcmin\\
\midrule
Scl & combined &  & 8945 & 2280 & 2096 & 1981 & 0.9 & \\
 & tolstoy+23 & FLAMES & 3311 & 1701 & 1522 & 1482 & 0.65 & –\\
 & sestito+23a & GMOS & 2 & 2 & 2 & 2 & 13 & –\\
 & walker+09 & MMFS & 1818 & 1522 & 1417 & 1328 & 1.8 & 3\\
 & APOGEE & APOGEE & 5082 & 253 & 170 & 164 & 0.5 & –\\
UMi & combined &  & 4714 & 1225 & 1148 & 863 & 2.1 & \\
 & sestito+23b & GRACES & 5 & 5 & 5 & 5 & 1.8 & –\\
 & pace+20 & DEIMOS & 1716 & 1538 & 829 & 678 & 2.5 & 1\\
 & spencer+18 & Hectoshell & 1407 & 970 & 596 & 406 & 0.9 & 2\\
 & APOGEE & APOGEE & 9500 & 279 & 37 & 67 & 0.6 & –\\
\bottomrule
\end{tabular}
\end{table*}

measurement

\begin{table*}[t]
\centering
\caption[Ursa Minor RV fits]{MCMC fits for UMi velocity dispersion. }
\label{tbl:umi_rv_mcmc}
\begin{tabular}{lllll}
\toprule
study & mean & sigma & $\log bf_{\rm sigma}$ & $\log bf_{\rm grad}$\\
\midrule
all & $-245.8\pm0.3$ & $8.8\pm0.2$ & +1.3 & +0.9\\
pace & $-244.6\pm0.4$ & $9.0\pm0.3$ & +0.3 & +0.5\\
spencer & $-246.9\pm0.4$ & $8.8\pm0.3$ & +1.8 & -0.06\\
apogee & $-245.6\pm1.2$ & $10.0_{-0.8}^{+1.0}$ & +1.0 & +0.5\\
\bottomrule
\end{tabular}
\end{table*}

\begin{figure}
\centering
\pandocbounded{\includegraphics[keepaspectratio]{figures/scl_rv_scatter.pdf}}
\caption[Scl velocity sample]{RV members of Sculptor plotted in the
tangent plane coloured by corrected velocity difference from mean
\(v_z - \bar v_z\) . The black ellipse marks the half-light radius in
Fig.~\ref{fig:scl_selection}. The black and green arrows mark the proper
motion (PM, GSR frame) and derived velocity gradient (rot) vectors (to
scale).}
\end{figure}

\begin{figure}
\centering
\pandocbounded{\includegraphics[keepaspectratio]{figures/scl_vel_gradient_scatter.pdf}}
\caption[Scl velocity gradient]{The corrected LOS velocity along the
best fit rotational axis. RV members are black points, the systematic
\(v_z\) is the horizontal grey line, blue lines represent the
(projected) gradient from MCMC samples, and the orange line is a rolling
median (with a window size of 50).}
\end{figure}

\chapter{Numerical convergence and
parameters}\label{sec:extra_convergence}

Here, we describe some convergence tests to ensure our methods and
results are minimally impacted by numerical limitations and assumptions.
See \citet{power+2003} for a detailed discussion of various assumptions
and parameters used in N-body simulations.

\section{Softening}\label{softening}

One challenge of N-body integration is close, collisional encounters,
assuming point particles, cause divergences in the local force. However,
this should not occur in a \emph{collisionless} simulation. As a result,
most collisionless N-body codes adopt a gravitational softening, a
length scale below which the force of gravity begins to decrease between
point particles.

\citet{power+2003} empirically suggest that the ideal softening is
\begin{equation}{
h_{\rm grav} = 4 \frac{R_{200}}{\sqrt{N_{200}}},
}\end{equation}

where \(h_{\rm grav}\) is the softening length, and \(N_{200}\) is the
number of particles within \(R_{200}\). This choice balances integration
time and only compromises resolution in collisional regime.

For our isolation halo (\(M_s=2.7\), \(r_s=2.76\)) and with \(10^7\)
particles, this works out to be \(0.044\,{\rm kpc}\).We adpoted the
slightly smaller softening which was reduced by a factor of
\(\sqrt{10}\) which appears to improve agreement slightly in the
innermost regions.

\begin{figure}
\centering
\pandocbounded{\includegraphics[keepaspectratio]{figures/iso_converg_softening.png}}
\caption{Softening convergence}\label{fig:softening_convergence}
\end{figure}

\section{Time stepping and force
accuracy}\label{time-stepping-and-force-accuracy}

In general, we use adaptive timestepping and relative opening criteria
for gravitational force computations. To verify that these choices and
associated accuracy parameters minimally impact convergence or speed, we
show a few more isolation runs (using only 1e5 particles)

\begin{itemize}
\tightlist
\item
  constant timestep (\ldots), approximantly half of minimum timestep
  with adaptive timestepping
\item
  geometric opening, with \(\theta = 0.5\).
\item
  strict integration accuracy, (facc = \ldots.)
\end{itemize}

\begin{figure}
\centering
\pandocbounded{\includegraphics[keepaspectratio]{figures/iso_converg_methods.png}}
\caption{Isolation method convergence}\label{fig:methods_convergence}
\end{figure}

\section{Fiducial parameters}\label{fiducial-parameters}

Note that we use code units which assume that \(G=1\) for convenience
and numerical stability. The conversion between code units to physical
units is (for our convention):

\begin{itemize}
\tightlist
\item
  1 length = 1 kpc
\item
  1 mass unit = \(10^{10}\) Msun
\item
  1 velocity unit = 207.4 km/s
\item
  1 time unit = 4.715 Myr
\end{itemize}

Most parameters below are not too relevant or have been discussed or are
merely dealing with cpu and IO details. The changes between simulation
runs primarily affect the integration time, output frequency, and
softening. Otherwise, we leave all other parameters fixed.

\begin{verbatim}
#======IO parameters======

#---Filenames
InitCondFile                initial
OutputDir                   ./out
SnapshotFileBase            snapshot
OutputListFilename          outputs.txt

#---File formats 
ICFormat                    3       # use HDF5
SnapFormat                  3 

#---Mem & CPU limits
TimeLimitCPU                86400
CpuTimeBetRestartFile       7200
MaxMemSize                  2400

#---Time
TimeBegin                   0
TimeMax                     2120     # 10 Gyr

#---Output frequency
OutputListOn                0
TimeBetSnapshot             10
TimeOfFirstSnapshot         0 
TimeBetStatistics           10
NumFilesPerSnapshot         1
MaxFilesWithConcurrentIO    1 


#=======Gravity======


#---Timestep accuracy
ErrTolIntAccuracy           0.01
CourantFac                  0.1     # ignored; for SPH
MaxSizeTimestep             0.5
MinSizeTimestep             0.0 

#---Tree algorithm
TypeOfOpeningCriterion      1       # Relative
ErrTolTheta                 0.5     # mostly used for Barnes-Hut
ErrTolThetaMax              1.0     # (used only for relative)
ErrTolForceAcc              0.005   # (used only for relative)

#---Domain decomposition: should only affect performance
TopNodeFactor                       3.0
ActivePartFracForNewDomainDecomp    0.02

#---Gravitational Softening
SofteningComovingClass0     0.044  # HALO dependent
SofteningMaxPhysClass0      0      # ignored; for cosmological
SofteningClassOfPartType0   0
SofteningClassOfPartType1   0


#=======Miscellanius=======

# probably do not need to change the options below 

#---Unit System
UnitLength_in_cm            1 
UnitMass_in_g               1
UnitVelocity_in_cm_per_s    1 
GravityConstantInternal     1

#---Cosmological Parameters 
ComovingIntegrationOn         0 # no cosmology
Omega0                      0
OmegaLambda                 0 
OmegaBaryon                 0
HubbleParam                 1
Hubble                      100
BoxSize                     0

#---SPH
ArtBulkViscConst            0.8
MinEgySpec                  0
InitGasTemp                 100

#---Initial density estimate (SPH)
DesNumNgb                   64
MaxNumNgbDeviation          1
\end{verbatim}


\end{document}
