\chapter{Data selection}\label{data-selection}

\section{Describution of algorithm}\label{describution-of-algorithm}

To create a high-quality sample, J+24 select stars initially from Gaia
within a 2--4 degree circular region centred on the dwarf satisfying:

\begin{itemize}
\tightlist
\item
  Solved astrometry, magnitude, and colour.
\item
  Renormalized unit weight error, \({\rm ruwe} \leq 1.3\), ensuring high
  quality astrometry. \texttt{ruwe} is a measure of the excess
  astrometric noise on fitting a consistent parallax-proper motion
  solution \citep[see][]{lindegren+2021}.\\
\item
  3\(\sigma\) consistency of measured parallax with dwarf's distance
  (dwarf parallax is very small; with \citet{lindegren+2021} zero-point
  correction).
\item
  Absolute proper motions, \(\mu_{\alpha*}\), \(\mu_\delta\), less than
  10\(\,{\rm mas\ yr^{-1}}\). (Corresponds to tangental velocities of
  \(\gtrsim 500\) km/s at distances larger than 10 kpc.)
\item
  Corrected colour excess is within 3\(\sigma\) of the expected
  distribution from \citet{riello+2021}. Removes stars with unreliable
  photometry.
\item
  De-reddened \(G\) magnitude is between
  \(22 > G > G_{\rm TRGB} - 5\sigma_{\rm DM}\). Removes very faint stars
  and stars significantly brighter than the tip of the red giant branch
  (TRGB) magnitude plus the distance modulus uncertainty
  \(\sigma_{\rm DM}\).
\item
  Colour is between \(-0.5 < {\rm BP - RP} <  2.5\) (dereddened).
  Removes stars substantially outside the expected CMD.
\end{itemize}

Photometry is dereddened with \citet{schlegel+finkbeiner+davis1998}
extinction maps.

J+24 define likelihoods \({\cal L}\) representing the probability
density that a star is consistent with either the MW stellar background
(\({\cal L}_{\rm bg}\)) or the satellite galaxy
(\({\cal L}_{\rm sat}\)). In either case, the likelihoods are the
product of a spatial, PM, and CMD term: \begin{equation}{
{\cal L} = {\cal L}_{\rm space}\ {\cal L}_{\rm PM}\ {\cal L}_{\rm CMD}.
}\end{equation}

Each likelihood is normalized over their respective 2D parameter space
for both the satellite. To control the relative frequency of member and
background stars, \(f_{\rm sat}\) representing the fraction of member
stars in the field. The total likelihood for any star in this model is
the sum of the satellite and background likelihoods, weighted by their
relative frequencies, \begin{equation}{
{\cal L}_{\rm tot} = f_{\rm sat}{\cal L}_{\rm sat} + (1-f_{\rm sat}){\cal L}_{\rm bg}.
}\end{equation} The probability that any star belongs to the satellite
is then given by \begin{equation}{
P_{\rm sat} = 
\frac{f_{\rm sat}\,{\cal L}_{\rm sat}}{{\cal L}_{\rm tot}}
= \frac{f_{\rm sat}{\cal L}_{\rm sat}}{f_{\rm sat}{\cal L}_{\rm sat} + (1-f_{\rm sat}){\cal L}_{\rm bg}}.
}\end{equation}

For the satellite's spatial likelihood, J+24 consider both one-component
and a two-component density models. The one component model is
constructed as a single exponential profile ( surface density
\(\Sigma \propto e^{R_{\rm ell} / R_s}\)), with scale radius \(R_s\)
fixed to the value in table 1 of \citet{MV2020a} from \citet{munoz+2018}
(for a Sérsic fit). Additionally, structural uncertainties (for position
angle, ellipticity, and scale radius) are sampled over to construct the
final likelihood map. The two-component model instead adds a second
exponential,
\(\Sigma_\star \propto e^{-R/R_s} + B\,e^{-R/R_{\rm outer}}\). The inner
scale radius is fixed, and the outer scale radius and magnitude of the
second component \(R_{\rm outer}\), \(B\) are free parameters.
Structural property uncertainties are not included in the two-component
model.

The PM likelihood is a bivariate gaussian with variance and covariance
equal to each star's proper motions. J+24 assume the stellar PM errors
are the main source of uncertainty.

The satellite's CMD likelihood is based on a Padova isochrone
\citep{girardi+2002}. The isochrone has a matching metallicity and 12
Gyr age (except 2 Gyr is used for Fornax). The (gaussian) colour width
is assumed to be 0.1 mag plus the Gaia colour uncertainty at each
magnitude. The horizontal branch is modelled as a constant magnitude
extending blue of the CMD (mean magnitude of -2.2, 12 Gyr HB stars and a
0.1 mag width plus the mean colour error). A likelihood map is
constructed by sampling the distance modulus in addition to the CMD
width, taking the maximum of RGB and HB likelihoods.

The background likelihoods are instead empirically constructed. Stars
stars outside of 5\(R_h\) passing the quality cuts estimate the
background density in PM and CMD space. The density is a sum of
bivariate gaussians with variances based on Gaia uncertainties (and
covariance for proper motions). The spatial background likelihood is
assumed to be constant.

J+24 derive \(\mu_{\alpha*}\), \(\mu_\delta\), \(f_{\rm sat}\) (and
\(B\), \(R_{\rm outer}\) for two-component) through an MCMC simulation
with likelihood from Eq.~\ref{eq:Ltot}. Priors are only weakly
informative. The proper motion single component prior is same as
\citet{MV2020a}: a normal distribution with mean 0 and standard
deviation \(100\ \kms\). If 2-component spatial, instead is a uniform
distribution spanning 5\(\sigma\) of single component case w/ systematic
uncertainties. \(f_{\rm sat}\) (and \(B\)) has a uniform prior 0--1.
\(R_{\rm outer}\) has a uniform prior only restricting
\(R_{\rm outer} > R_s\). The mode of each parameter from the MCMC are
then reported and used to calculate the final \(P_{\rm sat}\) values.

\section{Additional density tests}\label{additional-density-tests}

In this section, we discuss additional tests and verification of the
derived density profiles. In particular, we check that methodology
(simpler cuts, circularized radii, algorithm version) do not
substantially affect the density profile. We also compile density
profiles presented in the literature as reference. In all cases, the
density profiles appear to have excellent convergence out to
\(\log R_{\rm ell} / {\rm arcmin} \approx 1.8\), about the distance
where the background dominates.

Discuss selection criteria for DELVE and UNIONS samples, literature
comparison, simple selection criteria, MCMC density profiles and when
\citet{jensen+2024} becomes background-limited.

\begin{figure}
\centering
\pandocbounded{\includegraphics[keepaspectratio]{figures/scl_density_methods_extra.pdf}}
\caption[Scl density comparison]{Density profiles for various
assumptions for Sculptor. PSAT is our fiducial 2-component J+24 sample,
circ is a 2-component bayesian model assuming circular radii, simple is
the series of simple cuts described, bright is the sample of the
brightest half of stars (scaled by 2), DELVE is a sample of RGB stars
(background subtracted and rescaled to
match).}\label{fig:scl_density_extras}
\end{figure}

\begin{figure}
\centering
\pandocbounded{\includegraphics[keepaspectratio]{figures/scl_density_methods_j24.pdf}}
\caption[Scl density methods]{Comparison of density profiles for each
J+24 method. The fiducial is a 2-component elliptical model. However,
the 1-component is still elliptical but only contains 1 component and
the circular model assumes a circular outer density profile and bins in
circular bins instead of elliptical
bins.}\label{fig:scl_density_j24_methods}
\end{figure}

\begin{figure}
\centering
\pandocbounded{\includegraphics[keepaspectratio]{figures/umi_density_methods_extra.pdf}}
\caption[UMi density comparison]{Similar to
Fig.~\ref{fig:scl_observed_profiles} except for Ursa
Minor}\label{fig:umi_density_extras}
\end{figure}

\begin{figure}
\centering
\pandocbounded{\includegraphics[keepaspectratio]{figures/umi_density_methods_j24.pdf}}
\caption[UMi density methods]{Similar to
Fig.~\ref{fig:scl_density_j24_methods} except for Ursa
Minor.}\label{fig:umi_density_j24_methods}
\end{figure}

\section{Comparison to Literature}\label{comparison-to-literature}

Here, we compare our density profiles against past derivations of
density profiles for Sculptor and Ursa Minor

\begin{figure}
\centering
\pandocbounded{\includegraphics[keepaspectratio]{figures/analytic_profile_comparison.pdf}}
\caption[Comparison of analytic density profiles]{A comparison of
different parameterizations for dwarf galaxy density profiles. Note that
deviations between profiles only become apparent past 3 R\_h, and only
the Plummer profile, in contrast to more commonly assumed profiles,
deviates by \textasciitilde1 dex positive before 6 Rh. Since this
profile is a far minority in the literature, deviations from exponential
and close relatives are interesting and worth further consideration.}
\end{figure}
