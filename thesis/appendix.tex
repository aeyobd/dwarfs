\section{Appendix / Extra Notes}\label{appendix-extra-notes}

\subsection{Additional density profile
tests}\label{additional-density-profile-tests}

\begin{figure}
\centering
\pandocbounded{\includegraphics[keepaspectratio]{figures/scl_density_methods_extra.pdf}}
\caption[Density profiles]{Density profiles for various assumptions for
Sculptor. PSAT is our fiducial 2-component J+24 sample, circ is a
2-component bayesian model assuming circular radii, simple is the series
of simple cuts described in Appendix ?, bright is the sample of the
brightest half of stars (scaled by 2), DELVE is a sample of RGB stars
(background subtracted and rescaled to
match).}\label{fig:sculptor_observed_profiles}
\end{figure}

Note that a full rigorous statistical analysis would require a
simulation study of injecting dwarfs into Gaia and assessing the
reliability of various methods of membership and density profiles. This
is beyond the scope of this thesis.

\begin{verbatim}
SELECT TOP 1000
       *
FROM delve_dr2.objects
WHERE 11 < ra
and ra < 19
and -37.7 < dec
and dec < -29.7
\end{verbatim}

\subsection{Velocity fit parameters.}\label{velocity-fit-parameters.}

Savage-Dickey calculated bayes factor using Silverman-bandwidth KDE
smoothed samples from posterior/prior.

\emph{Is it interesting that the velocity dispersion of Scl seems to
increase significantly with Rell?}

\begin{table*}[t]
\centering
\caption{MCMC fits for different RV datasets for Scl amoung 3 different models.}
\label{tbl:MCMC-fits-for-different-RV-datasets-for-Scl-amoung-3-different-models}
\begin{tabular}{lllllll}
\toprule
study & mean & sigma & $\partial \log\sigma / \partial \log R$ & $\partial v_z / \partial x$ (km/s/deg) & $\theta_{\rm grad} / ^{\circ}$ & $\log bf$\\
\midrule
all &  &  &  &  &  & \\
 & $111.18\pm0.23$ & $9.71\pm0.17$ & - & - & - & 0\\
 & $111.19 \pm 0.23$ & $9.68\pm0.17$ & - & $4.3\pm1.3$ & $-149_{-13}^{+17}$ & -1.7\\
 & $111.16\pm0.23$ & $9.73\pm0.17$ & $0.056\pm0.021$ & - & - & -0.9\\
tolstoy+23 &  &  &  &  &  & \\
 & $111.3 \pm 0.3$ & $9.80 \pm 0.18$ & - & - & - & 0\\
 & $111.3\pm0.3$ & $9.78\pm0.18$ & – & $4.3\pm1.4$ & $-154_{-13}^{+19}$ & -1.4\\
 & $111.2 \pm 0.3$ & $9.74\pm0.19$ & $0.085 \pm 0.023$ & – & – & -4.5\\
walker+09 &  &  &  &  &  & \\
 & $111.0\pm0.3$ & $9.57\pm0.21$ & – & – & – & 0\\
 & $111.1\pm0.3$ & $9.53\pm0.21$ & - & $5.2_{-1.6}^{+1.8}$ & $-134_{-16}^{+23}$ & -2.4\\
 & $111.0\pm0.3$ & $9.61\pm0.21$ & $0.03\pm0.03$ & – & – & +1.6\\
apogee &  &  &  &  &  & \\
 & $109.9\pm0.8$ & $8.3\pm0.6$ & – & – & – & –\\
 & $109.9\pm0.8$ & $8.3\pm0.6$ & – & $6\pm3$ & $-151_{-36}^{+44}$ & +0.4\\
 & $109.9\pm0.8$ & $8.3\pm0.7$ & $0.05\pm0.08$ & – & – & +1.1\\
\bottomrule
\end{tabular}
\end{table*}

\begin{table*}[t]
\centering
\caption{MCMC fits for UMi velocity dispersion.}
\label{tbl:MCMC-fits-for-UMi-velocity-dispersion}
\begin{tabular}{lllll}
\toprule
study & mean & sigma & $\log bf_{\rm sigma}$ & $\log bf_{\rm grad}$\\
\midrule
all & $-245.9\pm0.3$ & $8.76\pm0.24$ & +1.5 & +2.2\\
pace & $-244.5\pm0.4$ & $9.1\pm0.3$ & +0.2 & +1.1\\
spencer & $-246.9\pm0.4$ & $8.8\pm0.3$ & +1.8 & -0.3\\
apogee & $-248.2\pm1.6$ & $9.0_{-1.1}^{+1.3}$ & +0.8 & +0.8\\
\bottomrule
\end{tabular}
\end{table*}

\begin{table*}[t]
\centering
\caption{Summary of velocity measurements and derived properties. sestito+2023a number of members depends on spatial model used.}
\label{tbl:Summary-of-velocity-measurements-and-derived-properties-sestito+2023a-number-of-members-depends-on-spatial-model-used}
\begin{tabular}{llllllll}
\toprule
 & Study & Instrument & Nspec & Nstar & Ngood & Nmemb & $\delta v_{\rm med}$\\
\midrule
Scl & combined &  & 8945 & 2280 & 2034 & 1918 & 0.9\\
 & tolstoy+23 & FLAMES & 3311 & 1701 & 1522 & 1480 & 0.65\\
 & sestito+23a & GMOS & 2 & 2 & 2 & 2 & 13\\
 & walker+09 & MMFS & 1818 & 1522 & 1417 & 1329 & 1.8\\
 & APOGEE & APOGEE & 5082 & 253 & 102 & 98 & 0.6\\
UMi & combined &  & 4714 & 1225 & 1148 & 831 & 2.3\\
 & sestito+23b & GRACES & 5 & 5 & 5 & 5 & 1.8\\
 & pace+20 & DEIMOS & 1716 & 1538 & 829 & 682 & 2.5\\
 & spencer+18 & Hectoshell & 1407 & 970 & 596 & 406 & 0.9\\
 & APOGEE & APOGEE & 9500 & 279 & 37 & 32 & 0.9\\
\bottomrule
\end{tabular}
\end{table*}

\section{Numerical Convergence}\label{numerical-convergence}

Here, we describe some convergence tests to ensure our methods and
results are minimally impacted by numerical limitations and assumptions.
See \citet{power+2003} for a detailed discussion of various assumptions
and parameters used in N-body simulations.

\subsubsection{Softening}\label{softening}

\citet{power+2003} suggest the empirical rule that the ideal softening
(balancing integration time and only compromising resolution in
collisional regime) is \[
h_{grav} = 4 \frac{R_{200}}{\sqrt{N_{200}}}
\]

For our isolation halo (\(M_s=2.7\), \(r_s=2.76\)) and with \(10^7\)
particles, this works out to be \(0.044\,{\rm kpc}\).We adpoted the
slightly smaller softening which was reduced by a factor of
\(\sqrt{10}\) which appears to improve agreement slightly in the
innermost regions. However, this choice likely unnecessarily increases
computation time for the relative gain in accuracy.

\begin{figure}
\centering
\pandocbounded{\includegraphics[keepaspectratio]{/Users/daniel/thesis/figures/iso_converg_softening.png}}
\caption{Softening convergence}\label{fig:softening_convergence}
\end{figure}

\subsubsection{Timestepping and force
accuracy}\label{timestepping-and-force-accuracy}

In general, we use adaptive timestepping and relative opening criteria
for gravitational force computations. To verify that these choices and
associated accuracy parameters minimally impact convergence or speed, we
show a few more isolation runs (using only 1e5 particles)

\begin{itemize}
\tightlist
\item
  constant timestep (\ldots), approximantly half of minimum timestep
  with adaptive timestepping
\item
  geometric opening, with \(\theta = 0.5\).
\item
  strict integration accuracy, (facc = \ldots.)
\end{itemize}

\subsubsection{Alternative methods}\label{alternative-methods}

\begin{itemize}
\tightlist
\item
  FMM
\item
  PMM-tree
\item
  Gadget2
\item
  etc.
\end{itemize}

\begin{figure}
\centering
\pandocbounded{\includegraphics[keepaspectratio]{/Users/daniel/thesis/figures/iso_converg_methods.png}}
\caption{Isolation method convergence}\label{fig:methods_convergence}
\end{figure}

\subsubsection{Fiducial Parameters}\label{fiducial-parameters}

Note that we use code units which assume that \(G=1\) for convenience
and numerical stability. The conversion between code units to physical
units is (for our convention):

\begin{itemize}
\tightlist
\item
  1 length = 1 kpc
\item
  1 mass unit = \(10^{10}\) Msun
\item
  1 velocity unit = 207.4 km/s
\item
  1 time unit = 4.715 Myr
\end{itemize}

Most parameters below are not too relevant or have been discussed or are
merely dealing with cpu and IO details. The changes between simulation
runs primarily affect the integration time, output frequency, and
softening. Otherwise, we leave all other parameters fixed.

\begin{verbatim}
#======IO parameters======

#---Filenames
InitCondFile                initial
OutputDir                   ./out
SnapshotFileBase            snapshot
OutputListFilename          outputs.txt

#---File formats 
ICFormat                    3       # use HDF5
SnapFormat                  3 

#---Mem & CPU limits
TimeLimitCPU                86400
CpuTimeBetRestartFile       7200
MaxMemSize                  2400

#---Time
TimeBegin                   0
TimeMax                    2120      # 10 Gyr

#---Output frequency
OutputListOn                0
TimeBetSnapshot             10
TimeOfFirstSnapshot         0 
TimeBetStatistics           10
NumFilesPerSnapshot         1
MaxFilesWithConcurrentIO    1 


#=======Gravity======


#---Timestep accuracy
ErrTolIntAccuracy           0.01
CourantFac                  0.1     # ignored; for SPH
MaxSizeTimestep             0.5
MinSizeTimestep             0.0 

#---Tree algorithm
TypeOfOpeningCriterion      1       # Relative
ErrTolTheta                 0.5     # mostly used for Barnes-Hut
ErrTolThetaMax              1.0     # (used only for relative)
ErrTolForceAcc              0.005   # (used only for relative)

#---Domain decomposition: should only affect performance
TopNodeFactor                       3.0
ActivePartFracForNewDomainDecomp    0.02

#---Gravitational Softening
SofteningComovingClass0      0.044  # HALO dependent
SofteningMaxPhysClass0       0      # ignored; for cosmological
SofteningClassOfPartType0    0
SofteningClassOfPartType1    0


#=======Miscellanius=======

# probably do not need to change the options below 

#---Unit System
UnitLength_in_cm            1 
UnitMass_in_g               1
UnitVelocity_in_cm_per_s    1 
GravityConstantInternal     1

#---Cosmological Parameters 
ComovingIntegrationOn      0 # no cosmology
Omega0                     0
OmegaLambda                 0 
OmegaBaryon                 0
HubbleParam                 1
Hubble                      100
BoxSize                     0

#---SPH
ArtBulkViscConst             0.8
MinEgySpec                   0
InitGasTemp                  100

#---Initial density estimate (SPH)
DesNumNgb                   64
MaxNumNgbDeviation          1
\end{verbatim}
