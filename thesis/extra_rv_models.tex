\chapter{LOS velocity modeling}\label{sec:rv_obs}

In this section, we analyze the observed line-of-sight (LOS) velocity
distributions for Sculptor (Scl) and Ursa Minor (UMi). Our goal is to
understand if the galaxies show evidence of tidal features---for
example, a velocity gradient or rising outer velocity dispersion. Our
derived systemic velocities and dispersions are consistent with
literature. We find weak evidence for a velocity gradient in Scl and a
rising outer velocity dispersion. However, Scl's velocity gradient is
misaligned with its proper motion, likely inconsistent with a tidal
interpretation. Scl's velocity dispersion also mostly rises within the
effective radius, more likely a feature of inner structure than tidal
disruption. We find no evidence of a gradient in mean velocity or
velocity dispersion in UMi. We conclude that Scl and UMi do not show
clear features of tidal disruption given current velocity observations.

\section{Data selection}\label{data-selection}

For both Sculptor and Ursa Minor, we construct literature samples of LOS
velocity measurements. We combine these samples with J+24's Bayesian
likelihoods to produce LOS velocity-informed membership probabilities.
We select stars with moderate to large membership probabilities for
future analysis.

\begin{table*}[t]
\centering
\caption[Spectroscopic LOS velocity measurements]{Summary of velocity measurements and derived properties. }
\label{tbl:rv_measurements}
\begin{tabular}{llllllll}
\toprule
 & Study & Nspec & Nstar & Ngood & Nmemb & $\delta v_{\rm med}$ & $R_{\rm xmatch}$\\
\midrule
Scl & combined & 8945 & 2280 & 2096 & 1981 & 0.9 & \\
 & tolstoy+23 & 3311 & 1701 & 1522 & 1482 & 0.65 & –\\
 & sestito+23a & 2 & 2 & 2 & 2 & 13 & –\\
 & walker+09 & 1818 & 1522 & 1417 & 1328 & 1.8 & 3”\\
 & APOGEE & 5082 & 253 & 170 & 164 & 0.5 & –\\
UMi & combined & 4714 & 1225 & 1148 & 863 & 2.1 & \\
 & sestito+23b & 5 & 5 & 5 & 5 & 1.8 & –\\
 & pace+20 & 1716 & 1538 & 829 & 678 & 2.5 & 1”\\
 & spencer+18 & 1407 & 970 & 596 & 406 & 0.9 & 2”\\
 & APOGEE & 9500 & 279 & 37 & 67 & 0.6 & –\\
\bottomrule
\end{tabular}
\end{table*}

First, we crossmatch all catalogues to J+24 Gaia stars. If a study did
not report GaiaDR3 source ID's, we match to the nearest star within 1--3
arcseconds (see REF Table~\ref{tbl:rv_measurements}). We combine the
mean RV measurement from each study using the inverse-variance weighted
mean \(\bar v\), standard uncertainty \(\delta \bar v\), and (biased)
variance \(s^2\).

We next remove stars with significant velocity dispersions as measured
between observations in a study or between studies, to filter out
unreliable measurements or possible spectroscopic binaries. By using
that \(\chi^2=\frac{s^2}{\delta \bar v^2}\), we remove stars with a
\(\chi^2\) larger than the 99.9th percentile of the \(\chi^2\)
distribution with \(N-1\) measurements. This cut typically removes stars
with reduced chi-squared values
\(\tilde\chi^2  = \frac{s^2}{\nu\,\delta \bar v^2}\gtrsim 7\) (as a star
typically fewer than four velocity measurements).

We correct the velocities for the solar motion and on-sky size of the
galaxy. First, we transform the velocities into the galactic standard of
rest (GSR), a reference frame centred at the Sun's position but at rest
with respect to the Galactic centre. Next, we transform velocities to
correct for the apparent gradient induced by the dwarf's proper motion.
Let the \(\hat z\) be the direction from the sun to the dwarf galaxy.
Then if \(\phi\) is the angular distance between the centre of the
galaxy and the individual star, the corrected radial velocity is then
\begin{equation}{
v_{\rm gsr}' = v_{\rm los, gsr}\cos\phi  - v_{\alpha}\cos\theta \sin\phi - v_\delta \sin\theta\sin\phi
}\end{equation} \{eq:v\_z\} where \(v_{\rm los, gsr}\) is the LOS
velocity in the GSR frame, \(v_\alpha\) and \(v_\delta\) are the
tangental velocities in RA and Dec in the GSR frame, and \(\theta\) is
the position angle of the star with respect to the centre of the dwarf.
The correction from both effects induces an apparent gradient of about
\(1.3\,\kmsdeg\) for Sculptor and less for Ursa Minor \citep[see
also][]{WMO2008, strigari2010}. We add the uncertainty in \(v_z\) from
the distance uncertainty and velocity dispersion in quadrature to the RV
uncertainties for each star. We use the \(v_z\) values for the following
modelling.

We build on J+24's likelihood framework to add velocity information (see
Section~\ref{sec:extra_density}). The total likelihood of a star
belonging to the satellite or foreground/background population is
\begin{equation}{
{\cal L} = {\cal L}_{\rm space} {\cal L}_{\rm CMD} {\cal L}_{\rm PM} {\cal L}_{\rm los},
}\end{equation} where we have added \({\cal L}_{\rm los}\), LOS-velocity
likelihood term. We assume that the satellite and background
\(v_{\rm los, gsr}\) distributions are Gaussian. Specifically,
\begin{equation}{
\begin{split}
{\cal L}_{\rm los, sat} &= f\left( \frac{v_i -\mu_{v}}{\sqrt{\sigma_{v}^2 + (\delta v_i)^2}}\right) \\
{\cal L}_{\rm los, bg} &= f\left( v_i /  \sigma_{\rm halo} \right)
\end{split}
}\end{equation} where \(f\) is the probability density of a standard
normal distribution, \(\mu_v\) and \(\sigma_v\) are the systemic
velocity and dispersion of the satellite, and \(\delta v_i\) is the
individual measurement uncertainty. Typically, the velocity dispersion
dominates the uncertainty budget. We assume a halo/background velocity
dispersion of a constant \(\sigma_{\rm halo} = 100\,\kms\)
\citep[e.g.][]{brown+2010}.

Similar to above, we retain stars with the resulting membership
probability of greater than 0.2. Like in J+24, most stars have
probabilities of nearly 1 or 0, so the precise choice of cut marginally
affects the resulting sample.

\section{MCMC modeling}\label{mcmc-modeling}

To model for possible gradients. We use Monte-Carlo Markov chains (MCMC)
to calculate the posterior distributions of our models of the intrinsic
velocity dispersion and possible velocity gradients in Scl and UMi.

The likelihood is based on the distribution of stars with respect to the
model's predicted \(\mu_\V\) and \(\sigma_\V\) at each star's position
in the tangent plane \(\xi\) and \(\eta\). Specifically, the log
likelihood is \begin{equation}{
\log {\cal L} = \sum_i \frac{1}{\sigma_i} \log f\left(\frac{\V_i - \mu_i}{\sigma_i}\right),
}\end{equation} where \(f\) is the probability distribution of a
standard normal distribution, \(\V_i\) is the velocity of the \(i\)th
star, and \(\mu_i\) and \(\sigma_i\) are the model's predicted mean and
velocity dispersion at the \(i\)th star's position \(\xi_i\) and
\(\eta_i\).

We consider a single model designed to derive systemic velocity and the
velocity dispersion, detect velocity gradients, and detect a rising
velocity dispersion. The mean velocity at a given point on the sky,
\(\mu\), is assumed to be, \begin{equation}{
\mu(\xi, \eta) = \mu_0 + a\,\xi + b\,\eta
}\end{equation}

for tangent-plane coordinates \(\xi\) and \(\eta\), systemtic velocity
\(\mu_0\), and velocity gradient slopes \(a\) and \(b\). The velocity
dispersion at a given position, \(\sigma\), is assumed to depend as a
power-law on elliptical radius \(R_{\rm ell}\) alone: \begin{equation}{
\log \sigma = \log \sigma_0 + c\,\log(R_{\rm ell} / R_h)
}\end{equation} where \(\sigma_0\) is the system's velocity dispersion
at \(R_h\), and \(c\) is the velocity dispersion gradient slope.

Our model has parameters with priors \begin{equation}{
\mu_{0} \sim N(0, \sigma^2_{\rm halo}) \,\kms\\
\sigma_{0} \sim U(0, 20)\ \kms \\
a \sim N(0, 6^2)\ \kms\,{\rm deg}^{-1}\\
b \sim N(0, 6^2)\ \kms\,{\rm deg}^{-1}\\
c\sim N(0, 0.3^2)\ {\rm dex} \\
}\end{equation} where \(\sigma_{\rm halo} = 100\,{\rm km\,s^{-1}}\) is
the velocity dispersion of the MW halo adopted above,
\(N(\mu, \sigma^2)\) is a standard normal distribution with mean \(\mu\)
and variance \(\sigma^2\), and \(U(l, h)\) is a uniform distribution
between \(l\) and \(h\).

Bayes factors quantify the relative evidence between Bayesian models. We
can use Bayes factors to compare the above model to simpler models
without gradients in \(\sigma\) or \(\mu\). Assuming that a
gradient-free model represents the case \(a=b=0\) (gradient free) or
\(c=0\) (fixed dispersion) models, we can quickly calculate the relative
Bayes factors without running another model. We use the Savage-Dickey
method \citep{dickey+lientz1970}, where the Bayes factor is then the
relative density of posterior versus prior samples when \(a=b=0\)
(gradient free) or \(c=0\) (fixed dispersion). In order to calculate
these densities, we use a kernel density estimator on the posterior
samples with a Silvermann bandwidth in each dimension.

While this is a more complex model, we find each parameter agrees with
models of reduced complexity.

\section{Results}\label{sec:rv_results}

\begin{figure}
\centering
\pandocbounded{\includegraphics[keepaspectratio]{figures/scl_umi_rv_fits.pdf}}
\caption[LOS velocity fit to Scl]{Velocity histogram of Scl and UMi in
terms of projected-correction GSR LOS velocity (Eq.~\ref{eq:v_z}). Black
points with error bars are from the crossmatched observed sample, and
green lines represent MCMC fits to the velocity dispersion.}
\end{figure}

\begin{landscape}
\begin{table*}[t]
\centering
\caption[Line-of-sight velocity fits]{MCMC fits for different RV datasets for Sculptor among 3 different models. The first row contains the priors ($N$ for normal with mean and variance, $U$ for uniform distributions). }
\label{tbl:scl_rv_mcmc}
\begin{tabular}{lllllllll}
\toprule
galaxy & study & $\mu$ & $\sigma$ & $\partial \log\sigma / \partial \log R$ & $\partial v_z / \partial x$  & $\theta_{\rm sigma}$ & $\log bf_{\rm grad}$ & $\log bf_{\rm grad}$\\
units & --- & $\kms$ & $\kms$ & dex & $\kms\,{\rm dec}^{-1}$ & deg & --- & --- \\
\midrule
Prior & --- & $N(0, 100^2)$ & $U(0, 20)$ & $N(0, 0.3^2)$ \\
\midrule
Scl \\
& all  & $111.3\pm0.2$ & $9.63\pm0.16$ & $0.07\pm0.02$ & $4.4 \pm 1.4$& $-146_{-14}^{+18}$ & -3.4 & -2.7\\
& tolstoy+23 & $111.2 \pm 0.3$ & $9.70\pm0.18$ & $0.083 \pm 0.023$ & $4.4\pm1.5$ & $-154_{-15}^{18}$ & -4.1 & -0.9 \\
& walker+09 & $111.1\pm0.3$ & $9.5\pm0.2$ & $0.05\pm0.03$ & $5.2\pm1.8$ & $-135_{-17}^{+23}$ & +0.7 & -1.7 \\
& apogee  & $111.2\pm0.7$ & $8.4\pm0.5$ & $0.06\pm0.06$ & $5.4_{-2.4}^{2.7}$ & $-127_{-36}^{+49}$ & +1.1 & +0.2 \\
\midrule
UMi \\
& all & $-245.8\pm0.3$ & $8.8\pm0.2$ & $0.04 \pm 0.03$ & $4 \pm 2$ & $-210_{-17}^{+21}$ & +1.4 & +1.1 \\
& pace+20 & $-244.5\pm0.4$ & $9.1\pm0.3$ & $0.08 \pm 0.05$ & $5\pm3$ & $-216 \pm 25$ & +0.5 & +0.7 \\
& spencer+18 & $-247.0\pm0.4$ & $8.7\pm0.3$ & $-0.004 \pm 0.05$ & $6 \pm 3$ & $-214\pm20$ & +1.8 & -0.2 \\
& apogee & $-245.9\pm1.3$ & $10.0\pm1.0$ & $0.04 \pm 0.98$ & $6_{-3}^{+4}$ & $-200\pm50$ & +1.0 & +0.5 \\
\bottomrule
\end{tabular}
\end{table*}

\end{landscape}


\begin{figure}
\centering
\pandocbounded{\includegraphics[keepaspectratio]{figures/scl_rv_scatter_gradient.png}}
\caption[Scl velocity gradient]{\textbf{Top} members of Sculptor plotted
in the tangent plane coloured by corrected velocity difference from mean
\(v_z - \bar v_z\) . The black ellipse marks the half-light radius in
Fig.~\ref{fig:scl_selection}. The black and green arrows mark the proper
motion (PM, GSR frame) and derived velocity gradient (rot) vectors (to
scale). \textbf{Bottom}: The corrected LOS velocity along the best fit
rotational axis. RV members are black points, the systematic \(v_z\) is
the horizontal grey line, blue lines represent the (projected) gradient
from MCMC samples, and the orange line is a rolling median (with a
window size of 50).}\label{fig:scl_velocity_gradient_scatter}
\end{figure}

\begin{figure}
\centering
\pandocbounded{\includegraphics[keepaspectratio]{figures/sigma_v_gradient.pdf}}
\caption[Possible gradients in the velocity dispersion]{The observed
velocity dispersion gradient (black) in 10 equal number bins and the
derived slopes from the model fitting. Scl shows moderate evidence for
an increasing velocity dispersion with radius. UMi's evidence for a
radial gradient is weaker, and Scl's velocity dispersion appears to be
flat outside of \(R_h\).}
\end{figure}

For Sculptor, we combine radial velocity measurements from APOGEE,
\citet{sestito+2023a}, \citet{tolstoy+2023}, and \citet{WMO2009}.
\citet{tolstoy+2023} and \citet{WMO2009} provide the bulk of the
measurements. We find that there is no significant velocity shift in
crossmatched stars between catalogues. After crossmatching to high
quality Gaia stars and excluding significant stellar velocity
dispersions, we have a sample of 1918 members.

We derive a systemic velocity for Sculptor of \(111.3\pm0.2\,\kms\)with
velocity dispersion \(9.64\pm0.16\,\kms\). Our values are very
consistent with previous work \citep[e.g.][\citet{arroyo-polonio+2024},
\citet{battaglia+2008}]{walker+2009}.

We detect a moderately significant gradient of \(4.3\pm1.3\,\kmsdeg\) at
a position angle of \(-149_{-13}^{+17}\) degrees. Several past work has
attempted to detect a gradient in Sculptor, but no consensus has been
reached. \citet{arroyo-polonio+2024} detect a velocity gradient of
\(4\pm1.5\,\kmsdeg\) in a similar direction using the
\citet{tolstoy+2023} sample, finding inconclusive statistical evidence.
They additionally suggest a third chemodynamical component of the galaxy
which may bias rotation measurements. \citet{battaglia+2008} also detect
a \(-7.6_{-2.2}^{+3.0}\,\kmsdeg\) velocity gradient along the major
axis, approximately the same direction. Instead, \citet{strigari2010};
\citet{martinez-garcia+2023} detect no significant gradient in Sculptor
using \citet{WMO2009} sample. Note that pre-\emph{Gaia} work did not
have as strong of a constraint on the proper motion of Scl, which limits
conclusions of the intrinsic velocity gradient in Scl.

Fig.~\ref{fig:scl_velocity_gradient_scatter} plots the combined velocity
sample and the MCMC samples for the velocity gradient. While some
samples have a consistent gradient with 0, most samples have a positive
velocity gradient, consistent with the rolling median trend.

The velocity gradient in Scl is misaligned with the proper motion

For UMi, we collect radial velocities from, APOGEE,
\citet{sestito+2023b}, \citet{pace+2020}, and \citet{spencer+2018}. We
shifted the velocities of \citet{spencer+2018} (\(-1.1\,\kms\)) and
\citet{pace+2020} (\(+1.1\,\kms\) ) to reach the same scale. We found
183 crossmatched common stars (passing 3\(\sigma\) RV cut, velocity
dispersion cut, and PSAT J+24 \textgreater{} 0.2 w/o velocities). Since
the median difference in velocities in this crossmatch is about 2.2
km/s, we adopt 1 km/s as the approximate systematic error here. Our
final sample includes 831 members.

We derive a mean \(-245.8\pm0.3_{\rm stat}\,\kms\) and velocity
dispersion of \(8.8\pm0.2\,\kms\) for UMi. This is consistent with
\citet{pace+2020} and to a lesser extent with \citet{spencer+2018}. We
do not find evidence for a velocity gradient, consistent with past work
\citep{pace+2020, martinez-garcia+2023}.

For both Ursa Minor and Sculptor, we also fit models to only data from
individual surveys. Since the resulting parameters are very similar, we
conclude that many of the systematic uncertainties are likely smaller
than the present errors or that each large survey has similar biases.

\section{Discussion and limitations}\label{discussion-and-limitations}

Our model here is relatively simple. Some things which we note as
systematics or limitations:

\emph{Inter-study systematics and biases}. While basic crossmatches and
a simple velocity shift, combining data from multiple instruments is
challenging. This appears to be a minor issue (Sculptor) or is corrected
for (Ursa Minor).

\emph{Misrepresentative uncertainties}. Inspection of the variances
compared to the standard deviations within a study seems to imply that
errors are accurately reported. APOGEE notes that their RV uncertainties
are known to be underestimates but are a small proportion of our sample.
We find that most samples report internally-consistent uncertainties for
multiple observations of the same star.

\emph{Binarity}. Stars in binary systems can inflate the inferred
system's velocity dispersion. For example, a system with an apparent
velocity dispersion of \(9\,\kms\) may have a true velocity dispersion
of \(8\,\kms\) \citep{spencer+2017}. Thus, our measurement is likely
slightly inflated given the high binary fractions measured in these
systems \citep{spencer+2018, arroyo-polonio+2023, gration+2025}.

\emph{Multiple populations}. Both Sculptor and Ursa Minor likely contain
multiple populations \citep[\citet{pace+2020},
\citet{tolstoy+2004}]{arroyo-polonio+2024}. Since we only model a single
population, and each population may have a different extent and velocity
dispersion, this could result in biased velocity dispersions. However,
it is unclear how to uniquely determine an overall velocity dispersion
in a multi-population system.

\emph{Selection effects}. RV studies each have their own selection
effects, which may affect the resulting dispersion, especially if
different populations or regions of the galaxy have different velocities
or velocity dispersions. We do not attempt to correct for this. Given
that samples extend through the half-light radius of each system, we do
not expect current samples to miss large features except in the very
outskirts.

Could we observe velocity gradients instead in \emph{Gaia}.
Unfortunately, given Scl and UMi's distances (\(\gtrsim 70\,\kpc\))
\emph{Gaia's} systemic proper motion uncertainty of \(0.017\,\masyr\)
corresponds to a velocity of \(5\,\kms\), making the detection of
gradients of amplitude less than \(5\,\kms\) challenging. In addition,
the typical uncertainties of member stars of Scl and UMi are even
larger. While in future data releases, \emph{Gaia} uncertainties may
permit study in the internal motion of dwarfs, the current \emph{Gaia}
proper motions are not yet precise enough.

\section{Summary}\label{summary}

In this section, we analyze literature samples of LOS velocity
measurements for both Scl and UMi. In each case, we find systemic LOS
velocities and velocity dispersions consistent with past work. We detect
weak evidence for a velocity gradient in Scl, however the gradient is
mis-aligned with the proper motion of Scl so is unlikely to be of tidal
origin. In both galaxies, the velocity dispersion profile is consistent
with a constant velocity dispersion with radius. The lack of
observational evidence for ongoing tidal disruption in the velocity
distribution of stars in Scl and UMi further supports our
interpretations in the main text.
