\chapter{Line-of-sight velocities: Sample selection and
modelling}\label{sec:extra_rv_models}

In this section, we analyze the observed line-of-sight (LOS) velocity
distributions for Sculptor (Scl) and Ursa Minor (UMi). Our aim is to
test for kinematic tidal signatures by combining literature LOS
velocities with J+24's membership framework. Our derived systemic
velocities and dispersions are consistent with literature. We find weak
evidence for a velocity gradient in Scl. As the gradient is misaligned
with the orbit, the gradient more likely reflects intrinsic rotation
than tidal disruption. Scl's velocity dispersion also rises within the
effective radius, but predominantly in the inner regions, contrary to
tidal disruption. We find no evidence of a gradient in mean velocity or
velocity dispersion in UMi. We conclude that Scl and UMi do not show
clear features of tidal disruption given current velocity observations.

\section{Data processing and
selection}\label{data-processing-and-selection}

We compile LOS velocities from several spectroscopic surveys. For
Sculptor, we combine \citet{tolstoy+2023}; \citet{walker+2009};
\citet{sestito+2023a}; and \apogee{} \citep[DR17,][]{abdurrouf+2022}.
For Ursa Minor, we combine \citet{spencer+2018}; \citet{pace+2020};
\citet{sestito+2023b}; and \apogee{}. We then crossmatch all catalogues
to J+24 Gaia stars. If a study did not report GaiaDR3 source ID's, we
match to the nearest star within 1--3 arcseconds. We combine
measurements of the same star using inverse-variance weighting.

To reduce likely binaries, we remove stars with significant velocity
dispersions.\footnote{Specifically, using that
  \(\chi^2=\frac{s^2}{\delta \bar v^2}\), we remove stars with a
  \(\chi^2\) larger than the 99.9th percentile of the \(\chi^2\)
  distribution with \(N-1\) measurements.}

We build on J+24's likelihood by adding multiplicative terms in the
total likelihood for the velocity consistency (see
Section~\ref{sec:extra_density}). We assume that the satellite and
background \(v_{\rm los, gsr}\) distributions are Gaussian. We assume a
halo/background velocity dispersion of a constant
\(\sigma_{\rm halo} = 100\,\kms\) with mean 0
\citep[e.g.][]{brown+2010}. We select stars with velocity-informed
satellite membership probabilities of greater than 0.2. For Scl, we find
1918 unique members and UMi, 831.

For UMi, we shifted the velocities of \citet{spencer+2018}
(\(-1.1\,\kms\)) and \citet{pace+2020} (\(+1.1\,\kms\) ) to account for
a systematic velocity offset. Otherwise all studies appear to agree.

We correct the velocities for the solar motion and on-sky size of the
galaxy. We transform the velocities into the galactic standard of rest
(GSR, i.e.~same location as ICRS but velocities relative to the galactic
centre), and then transform velocities to correct for the apparent
gradient induced by the dwarf's proper motion.\footnote{Specifically,
  sub} We define \(v_{\rm gsr}'\) to be velocities in the The correction
from both effects induces an apparent gradient of about \(1.3\,\kmsdeg\)
for Sculptor and less for Ursa Minor \citep[see
also][]{WMO2008, strigari2010}. The uncertainty on this velocity
correction are less than the individual star uncertainties and our
derived velocity gradient.

\section{MCMC modeling}\label{mcmc-modeling}

We fit Monte-Carlo Markov chain (MCMC) models to solve for the systemic
velocity, velocity dispersion, and possible gradient. We assume that the
galaxy follows a planar velocity gradient. The mean velocity at a given
point on the sky, \(\mu\), is assumed to be,

\begin{equation}{
\mu(\xi, \eta) = \mu_0 + a\,\xi + b\,\eta
}\end{equation}

for tangent-plane coordinates \(\xi\) and \(\eta\), systemic velocity
\(\mu_0\), and velocity gradient slopes \(a\) and \(b\). The velocity
dispersion at a given position, \(\sigma\), is assumed to depend as a
power-law on elliptical radius \(R_{\rm ell}\) alone: \begin{equation}{
\log \sigma = \log \sigma_0 + c\,\log(R_{\rm ell} / R_h)
}\end{equation} where \(\sigma_0\) is the system's velocity dispersion
at \(R_h\), and \(c\) is the velocity dispersion gradient slope. We use
weakly-informative priors, as described in Table~\ref{tbl:scl_rv_mcmc},
and with \(a, b \sim N(0, 6^2)\,\kmsdeg\).

\section{Results}\label{sec:rv_results}

\begin{figure}
\centering
\pandocbounded{\includegraphics[keepaspectratio]{figures/scl_umi_rv_fits.pdf}}
\caption[LOS velocity fit to Scl]{Velocity histogram of Scl and UMi in
terms of projected-correction GSR LOS velocity (Eq.~\ref{eq:v_z}). Black
points with error bars are from the crossmatched observed sample, and
green lines represent MCMC fits to the velocity dispersion.}
\end{figure}

Figure: The observed velocity dispersion gradient (black) in 10 equal
number bins and the derived slopes from the model fitting. Scl shows
moderate evidence for an increasing velocity dispersion with radius.
UMi's evidence for a radial gradient is weaker, and Scl's velocity
dispersion appears to be flat outside of \(R_h\).

\begin{landscape}
\begin{table*}[t]
\centering
\caption[Line-of-sight velocity fits]{MCMC fits for different RV datasets for Sculptor among 3 different models. The first row contains the priors ($N$ for normal with mean and variance, $U$ for uniform distributions). }
\label{tbl:scl_rv_mcmc}
\begin{tabular}{lllllllll}
\toprule
galaxy & study & $\mu$ & $\sigma$ & $\partial \log\sigma / \partial \log R$ & $\partial v_z / \partial x$  & $\theta_{\rm sigma}$ & $\log bf_{\rm grad}$ & $\log bf_{\rm grad}$\\
units & --- & $\kms$ & $\kms$ & dex & $\kms\,{\rm dec}^{-1}$ & deg & --- & --- \\
\midrule
Prior & --- & $N(0, 100^2)$ & $U(0, 20)$ & $N(0, 0.3^2)$ \\
\midrule
Scl \\
& all  & $111.3\pm0.2$ & $9.63\pm0.16$ & $0.07\pm0.02$ & $4.4 \pm 1.4$& $-146_{-14}^{+18}$ & -3.4 & -2.7\\
& tolstoy+23 & $111.2 \pm 0.3$ & $9.70\pm0.18$ & $0.083 \pm 0.023$ & $4.4\pm1.5$ & $-154_{-15}^{18}$ & -4.1 & -0.9 \\
& walker+09 & $111.1\pm0.3$ & $9.5\pm0.2$ & $0.05\pm0.03$ & $5.2\pm1.8$ & $-135_{-17}^{+23}$ & +0.7 & -1.7 \\
& apogee  & $111.2\pm0.7$ & $8.4\pm0.5$ & $0.06\pm0.06$ & $5.4_{-2.4}^{2.7}$ & $-127_{-36}^{+49}$ & +1.1 & +0.2 \\
\midrule
UMi \\
& all & $-245.8\pm0.3$ & $8.8\pm0.2$ & $0.04 \pm 0.03$ & $4 \pm 2$ & $-210_{-17}^{+21}$ & +1.4 & +1.1 \\
& pace+20 & $-244.5\pm0.4$ & $9.1\pm0.3$ & $0.08 \pm 0.05$ & $5\pm3$ & $-216 \pm 25$ & +0.5 & +0.7 \\
& spencer+18 & $-247.0\pm0.4$ & $8.7\pm0.3$ & $-0.004 \pm 0.05$ & $6 \pm 3$ & $-214\pm20$ & +1.8 & -0.2 \\
& apogee & $-245.9\pm1.3$ & $10.0\pm1.0$ & $0.04 \pm 0.98$ & $6_{-3}^{+4}$ & $-200\pm50$ & +1.0 & +0.5 \\
\bottomrule
\end{tabular}
\end{table*}

\end{landscape}


\begin{figure}
\centering
\pandocbounded{\includegraphics[keepaspectratio]{figures/scl_rv_scatter_gradient.png}}
\caption[Scl velocity gradient]{\textbf{Top} members of Sculptor plotted
in the tangent plane coloured by corrected velocity difference from mean
\(v_z - \bar v_z\) . The black ellipse marks the half-light radius in
Fig.~\ref{fig:scl_selection}. The black and green arrows mark the proper
motion (PM, GSR frame) and derived velocity gradient (rot) vectors (to
scale). \textbf{Bottom}: The corrected LOS velocity along the best fit
rotational axis. RV members are black points, the systematic \(v_z\) is
the horizontal grey line, blue lines represent the (projected) gradient
from MCMC samples, and the orange line is a rolling median (with a
window size of 50).}\label{fig:scl_velocity_gradient_scatter}
\end{figure}

We derive a systemic velocity for Sculptor of \(111.3\pm0.2\,\kms\)with
velocity dispersion \(9.64\pm0.16\,\kms\). Our values are very
consistent with previous work \citep[e.g.][\citet{arroyo-polonio+2024},
\citet{battaglia+2008}]{walker+2009}. We detect a moderately significant
gradient of \(4.3\pm1.3\,\kmsdeg\) at a position angle of
\(-149_{-13}^{+17}\) degrees, similar to past detections
\citetext{\citealp[e.g.,][]{arroyo-polonio+2024}; \citealp{battaglia+2008}; \citealp[but
see also][]{strigari2010}; \citealp{martinez-garcia+2023}}.

Fig.~\ref{fig:scl_velocity_gradient_scatter} plots the combined velocity
sample and the MCMC samples for the velocity gradient. While some
samples have a consistent gradient with 0, most samples have a positive
velocity gradient, consistent with the rolling median trend. The
velocity gradient in Scl is misaligned with the proper motion

We derive a mean \(-245.8\pm0.3_{\rm stat}\,\kms\) and velocity
dispersion of \(8.8\pm0.2\,\kms\) for UMi. We do not find evidence for a
velocity gradient. This is consistent with previous work
\citetext{\citealp{pace+2020}; \citealp[somewhat
with][]{spencer+2018}; \citealp{martinez-garcia+2023}}.

\section{Discussion and caveats}\label{discussion-and-caveats}

\emph{Binarity}. Stars in binary systems can inflate the inferred
system's velocity dispersion. For Scl and UMi with high measured
velocity dispersion, binaries likely add \(\sim 1\,\kms\) to the
velocity dispersion \citep{spencer+2018, gration+2025}.

\emph{Multiple populations}. Both Sculptor and Ursa Minor likely contain
multiple populations with different kinematics\citep[\citet{pace+2020},
\citet{tolstoy+2004}]{arroyo-polonio+2024}. We do not model these
separately---it is unclear how to define a velocity dispersion in such a
case.

\emph{Inter-study biases}. While basic crossmatches and a simple
velocity shift, combining data from multiple instruments is challenging.
Studies may also have biased selection effects or misrepresentative
uncertainties, biasing our results. However, besides the \(\sim1\,\kms\)
velocity offset in Ursa Minor, our results are consistent across
samples.

\emph{Tangental motions.} Precise star-by-star proper motions would
further test for kinematic disequilibrium. Presently, \emph{Gaia}
uncertainties are too large to permit such studies for Sculptor and Ursa
Minor.\footnote{Specifically, typical proper motion uncertntanties for
  faint, member stars of Scl and UMi are around \(0.5\,\masyr\),
  corresponding to velocities \(\sim 100\,\kms\).}

\section{Summary}\label{summary}

In this section, we analyze literature samples of LOS velocity
measurements for both Scl and UMi. In each case, we find systemic LOS
velocities and velocity dispersions consistent with past work. We detect
weak evidence for a velocity gradient in Scl, however the gradient is
mis-aligned with the proper motion of Scl so is unlikely to be of tidal
origin. In both galaxies, the velocity dispersion profile is consistent
with a constant velocity dispersion with radius. The lack of
observational evidence for ongoing tidal disruption in the velocity
distribution of stars in Scl and UMi further supports our
interpretations in the main text.
