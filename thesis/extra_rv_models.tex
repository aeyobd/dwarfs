\chapter{Line-of-Sight Velocities: Sample Selection and
Modelling}\label{sec:extra_rv_models}

In this section, we analyze the observed line-of-sight (LOS) velocity
distributions for Sculptor (Scl) and Ursa Minor (UMi). We aim to test
for kinematic tidal signatures by combining literature LOS velocities
with J+24's membership framework. Our derived systemic velocities and
dispersions are consistent with the literature. We find weak evidence
for a velocity gradient in Scl. As the gradient is misaligned with the
orbit, the gradient more likely reflects intrinsic rotation than tidal
disruption. Scl's velocity dispersion also rises within the effective
radius, but predominantly in the inner regions, contrary to tidal
disruption. We find no evidence of a gradient in mean velocity or
velocity dispersion in UMi. We conclude that Scl and UMi do not show
clear features of tidal disruption given current velocity observations.

\section{Data processing and
selection}\label{data-processing-and-selection}

We compile LOS velocities from several spectroscopic surveys. For
Sculptor, we combine \citet{tolstoy+2023}; \citet{walker+2009};
\citet{sestito+2023a}; and APOGEE \citep[DR17,][]{abdurrouf+2022}. For
Ursa Minor, we combine \citet{spencer+2018}; \citet{pace+2020};
\citet{sestito+2023b}; and APOGEE. We then cross-match all catalogues to
J+24 Gaia stars. If a study did not report \emph{Gaia} DR3 source IDs,
we match to the nearest star within 1--3 arcseconds. We combine
measurements of the same star using inverse-variance weighting. To
reduce likely binaries, we remove stars with significant velocity
dispersions.\footnote{Specifically, using that
  \(\chi^2=\frac{s^2}{\delta \bar v^2}\), we remove stars with a
  \(\chi^2\) larger than the 99.9th percentile of the \(\chi^2\)
  distribution with \(N-1\) measurements.}

We build on J+24's likelihood by adding multiplicative terms in the
total likelihood for the velocity consistency (see
Section~\ref{sec:the_algorithm}). We assume that the satellite and
background \(v_{\rm los, gsr}\) distributions are Gaussian in the
Galactic Standard of Rest (GSR, i.e., same location as ICRS but
velocities relative to the galactic centre). For the satellite, we adopt
a mean and standard deviation based on
Tables~\ref{tbl:scl_obs_props}, \ref{tbl:umi_obs_props}, and, for the
background, mean \(0\,\kms\) and dispersion
\(\sigma_{\rm halo} = 100\,\kms\) \citep[e.g.][]{brown+2010}. We select
stars with velocity-informed satellite membership probabilities of
greater than 0.2. For Scl, we find 1918 unique members and UMi, 831.

For UMi, we shifted the velocities of \citet{spencer+2018}
(\(-1.1\,\kms\)) and \citet{pace+2020} (\(+1.1\,\kms\) ) to account for
a systematic velocity offset. Otherwise, all studies appear to be on a
similar velocity scale.

We correct the velocities for the solar motion and the on-sky size of
the galaxy. We transform the velocities into the GSR and correct for the
apparent gradient induced by the dwarf's proper motion
\citep[see,][]{WMO2008, strigari2010}. We define \(v_{\rm gsr}'\) to be
velocities in the GSR frame, subtracting the PM-induced gradient. The
correction from both effects induces an apparent gradient of about
\(1.3\,\kmsdeg\) for Sculptor and less for Ursa Minor. The uncertainty
on this velocity correction is less than the individual star
uncertainties and any derived velocity gradients.

\section{Monte Carlo Markov chain
modelling}\label{monte-carlo-markov-chain-modelling}

We fit Monte Carlo Markov chain (MCMC) models to solve for the systemic
velocity, velocity dispersion, and possible gradients in velocities or
velocity dispersions. We assume that the galaxy follows a planar
velocity gradient. The mean velocity at a given point on the sky,
\(\mu\), is assumed to be,

\begin{equation}{
\mu(\xi, \eta) = \mu_0 + a\,\xi + b\,\eta,
}\end{equation} for tangent plane coordinates \(\xi\) and \(\eta\),
systemic velocity \(\mu_0\), and velocity gradient slopes \(a\) and
\(b\). The velocity dispersion at a given position, \(\sigma\), is
assumed to depend as a power-law on elliptical radius \(R_{\rm ell}\)
alone: \begin{equation}{
\log \sigma = \log \sigma_0 + c\,\log(R_{\rm ell} / R_h)
}\end{equation} where \(\sigma_0\) is the system's velocity dispersion
at \(R_h\), and \(c\) is the velocity dispersion gradient slope. We use
weakly-informative priors, as described in Table~\ref{tbl:scl_rv_mcmc},
and with \(a, b \sim N(0, 6^2)\,\kmsdeg\).

\section{Results}\label{sec:rv_results}

\begin{figure}
\centering
\pandocbounded{\includegraphics[keepaspectratio]{figures/scl_umi_rv_fits.pdf}}
\caption[Velocity dispersion fits]{\textbf{Left:} Velocity histogram of
Scl and UMi in terms of projected-correction LOS velocity. Black points
with error bars are a histogram of LOS-velocity members, and green lines
represent MCMC fits to the velocity dispersion. \textbf{Right:} The
observed velocity dispersion gradient (black points) in 10 equal number
bins and the derived slopes for MCMC samples (green lines). Scl shows
moderate evidence for an increasing velocity dispersion with radius, but
Scl's velocity dispersion appears to flatten outside of \(R_h\). UMi's
evidence for a radial gradient is weaker.}\label{fig:rv_hists}
\end{figure}

\begin{landscape}
\begin{table*}[t]
\centering
\caption[Line-of-sight velocity fits]{MCMC fits for different RV datasets for Sculptor among 3 different models. The first row contains the priors ($N$ for normal with mean and variance, $U$ for uniform distributions). }
\label{tbl:scl_rv_mcmc}
\begin{tabular}{lllllllll}
\toprule
galaxy & study & $\mu$ & $\sigma$ & $\partial \log\sigma / \partial \log R$ & $\partial v_z / \partial x$  & $\theta_{\rm sigma}$ & $\log bf_{\rm grad}$ & $\log bf_{\rm grad}$\\
units & --- & $\kms$ & $\kms$ & dex & $\kms\,{\rm dec}^{-1}$ & deg & --- & --- \\
\midrule
Prior & --- & $N(0, 100^2)$ & $U(0, 20)$ & $N(0, 0.3^2)$ \\
\midrule
Scl \\
& all  & $111.3\pm0.2$ & $9.63\pm0.16$ & $0.07\pm0.02$ & $4.4 \pm 1.4$& $-146_{-14}^{+18}$ & -3.4 & -2.7\\
& tolstoy+23 & $111.2 \pm 0.3$ & $9.70\pm0.18$ & $0.083 \pm 0.023$ & $4.4\pm1.5$ & $-154_{-15}^{18}$ & -4.1 & -0.9 \\
& walker+09 & $111.1\pm0.3$ & $9.5\pm0.2$ & $0.05\pm0.03$ & $5.2\pm1.8$ & $-135_{-17}^{+23}$ & +0.7 & -1.7 \\
& apogee  & $111.2\pm0.7$ & $8.4\pm0.5$ & $0.06\pm0.06$ & $5.4_{-2.4}^{2.7}$ & $-127_{-36}^{+49}$ & +1.1 & +0.2 \\
\midrule
UMi \\
& all & $-245.8\pm0.3$ & $8.8\pm0.2$ & $0.04 \pm 0.03$ & $4 \pm 2$ & $-210_{-17}^{+21}$ & +1.4 & +1.1 \\
& pace+20 & $-244.5\pm0.4$ & $9.1\pm0.3$ & $0.08 \pm 0.05$ & $5\pm3$ & $-216 \pm 25$ & +0.5 & +0.7 \\
& spencer+18 & $-247.0\pm0.4$ & $8.7\pm0.3$ & $-0.004 \pm 0.05$ & $6 \pm 3$ & $-214\pm20$ & +1.8 & -0.2 \\
& apogee & $-245.9\pm1.3$ & $10.0\pm1.0$ & $0.04 \pm 0.98$ & $6_{-3}^{+4}$ & $-200\pm50$ & +1.0 & +0.5 \\
\bottomrule
\end{tabular}
\end{table*}

\end{landscape}


\begin{figure}
\centering
\pandocbounded{\includegraphics[keepaspectratio]{figures/scl_rv_scatter_gradient.pdf}}
\caption[A possible velocity gradient in Sculptor]{\textbf{Top} members
of Sculptor plotted in the tangent plane, coloured by corrected velocity
difference from mean \(v_{\rm gsr}' - \bar v_{\rm gsr}'\). The black
ellipse marks the half-light radius in Fig.~\ref{fig:scl_selection}. The
black and green arrows mark the proper motion (PM, GSR frame) and
derived velocity gradient (rot) vectors (to scale). \textbf{Bottom}: The
corrected LOS velocity along the most likely rotational axis. RV members
are black points, the systematic \(v_{\rm gsr}''\) is the horizontal
grey line, blue lines represent the (projected) gradient from MCMC
samples, and the orange line is a rolling median (with a window size of
50).}\label{fig:scl_velocity_gradient_scatter}
\end{figure}

\begin{figure}
\centering
\pandocbounded{\includegraphics[keepaspectratio]{figures/umi_rv_scatter_gradient.pdf}}
\caption[velocity]{Similar to
Fig.~\ref{fig:scl_velocity_gradient_scatter} except for Ursa Minor. Ursa
Minor does not have clear evidence of a velocity
gradient.}\label{fig:umi_velocity_gradient_scatter}
\end{figure}

We derive a systemic velocity for Sculptor of \(111.3\pm0.2\,\kms\) with
velocity dispersion \(9.64\pm0.16\,\kms\) (see Fig.~\ref{fig:rv_hists}).
Our values are very consistent with previous work
\citep[e.g.,][]{walker+2009, arroyo-polonio+2024, battaglia+2008}. We
detect a moderately significant gradient of \(4.3\pm1.3\,\kmsdeg\) at a
position angle of \(-149_{-13}^{+17}\) degrees, similar to past
detections
\citetext{\citealp[e.g.,][]{arroyo-polonio+2024}; \citealp{battaglia+2008}; \citealp[but
see also][]{strigari2010}; \citealp{martinez-garcia+2023}}.

Fig.~\ref{fig:scl_velocity_gradient_scatter} plots the combined velocity
sample and the MCMC samples for the velocity gradient. While some
samples have a gradient consistent with 0, most samples have a positive
velocity gradient, consistent with the rolling median trend. The
velocity gradient in Scl is misaligned with the proper motion.

We derive a mean \(-245.8\pm0.3_{\rm stat}\,\kms\) and velocity
dispersion of \(8.8\pm0.2\,\kms\) for UMi (see Fig.~\ref{fig:rv_hists}).
We do not find evidence for a velocity gradient. This is consistent with
previous work \citetext{\citealp{pace+2020}; \citealp[somewhat
with][]{spencer+2018}; \citealp{martinez-garcia+2023}}.

\section{Discussion and caveats}\label{discussion-and-caveats}

\emph{Binarity}. Binary star systems may inflate the inferred system's
velocity dispersion. For Scl and UMi with high measured velocity
dispersion, binaries likely add \(\sim 1\,\kms\) to the velocity
dispersion \citep{spencer+2018, gration+2025}.

\emph{Multiple populations}. Both Sculptor and Ursa Minor likely contain
multiple populations with different kinematics (see
Section~\ref{sec:peculiarities}). We do not model these separately---it
is unclear how to define a velocity dispersion in such a case.

\emph{Inter-study biases}. While basic crossmatches and a simple
velocity shift, combining data from multiple instruments is challenging.
Studies may also have biased selection effects or misrepresentative
uncertainties, biasing our results. However, besides the \(\sim1\,\kms\)
velocity offset in Ursa Minor, our results are consistent across
samples.

\emph{Tangential motions.} Precise star-by-star proper motions would
further test for kinematic disequilibrium. Presently, \emph{Gaia}
uncertainties are too large to permit such studies for Sculptor and Ursa
Minor.\footnote{Specifically, typical proper motion uncertainties for
  faint, member stars of Scl and UMi are around \(0.5\,\masyr\),
  corresponding to velocities \(\sim 100\,\kms\).}

\section{Summary}\label{summary}

In this section, we analyze literature samples of LOS velocity
measurements for both Scl and UMi. In each case, we find systemic
velocities and dispersions consistent with past work. We detect weak
evidence for a velocity gradient in Scl, but the gradient is misaligned
with the proper motion of Scl, so it is unlikely to be of tidal origin.
Sculptor also shows evidence for a mildly rising velocity dispersion
with radius; however, the outer regions appear to be flat in velocity
dispersion. The lack of observational evidence for ongoing tidal
disruption in the velocity distribution of stars in Scl and UMi further
supports our interpretations in the main text.
