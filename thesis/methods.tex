\section{Simulation Methods}\label{simulation-methods}

In this section, we discuss our simulation methods from orbital
estimation to initial conditions to simulation parameters and software.

We use Agama \citep{vasiliev} for potential specification and initial
conditions, and Gadget-4 \citep{springle2021} for orbit integration and
dark matter simulations (with a custom patch to integrate with Agama).

\subsection{Potential}\label{potential}

Our fiducial potential is as described in \citet{EP2020}. This potential
is an analytic approximation of \citet{McMillan2011}, consisting of a
stellar Bulge, disk, and a dark matter NFW halo. The specific parameters
are

\subsubsection{Bulge}\label{bulge}

Hernquist potential with \(r_s=1.3\), mass = 2.1.

The \citet{hernquist1990} density profile for the galactic bulge is
parameterized in terms of a characteristic mass, \(M\), and radius
\(a\).

Density Profile \[
\rho(r) = \frac{M}{2\pi} \frac{a}{r} \frac{1}{(r+a)^3}
\]

\subsubsection{Disk}\label{disk}

\begin{itemize}
\tightlist
\item
  Thin disk. miyamoto+nagai disk with parameters \(a=3.944\),
  \(b=0.311\), \(M=3.944\),
\item
  Thick disk. miyamoto+nagai disk with parameters \(a=4.4\), \(b=0.92\),
  and \(M=2\)
\end{itemize}

\[
\Phi(R, z) = \frac{-GM}{\left(R^2 + \left[a + \sqrt{z^2 + b^2}\right]^{2}\right)^{1/2}}
\]

for convenience, we define the helper variables,
\(S = \sqrt{z^2 + b^2}\) and \(D = \sqrt{R^2 + (a+S)^2}\), so the
potential simplifies to \[
\Phi(R, z) = \frac{-GM}{D}
\]

\[
\rho = \frac{b^2 M}{4\pi} \frac{aR^2 + (a+3r)(a + r)^2}{S^3 D^5}
\]

\subsubsection{Halo}\label{halo}

Mvir=115, r=20.2, c=9.545

\begin{itemize}
\tightlist
\item
  NFW Dark matter halo. \(M_s=79.5\) (double check). and \(r_s = 20.2\).
\end{itemize}

Variations to the potential of the inner disk (exclusion of a bar)
should minimally affect our results as no orbit we consider reaches less
than \textasciitilde15 kpc of the MW centre.

\textbf{Figure:} Circular velocity profile of \citet{EP2020} potential.

\subsection{Orbital Estimation}\label{orbital-estimation}

To estimate the orbits of a dwarf galaxy, we perform a Monte Carlo
sampling of the present-day observables. We integrate each sampled
observable back in time for 10 Gyr in Gadget as massless point
particles.

We use the astropy v4 galactocentric frame. This frame assumes
\ldots\ldots{}

\subsection{Initial conditions}\label{initial-conditions}

We use Agama (cite) to generate initial conditions. We initially assume
galaxies are described by an NFW dark matter potential and the stars are
merely collisionless tracers embedded in this potential

\subsubsection{Stellar Probabilities}\label{stellar-probabilities}

We assign stellar probabilities via Eddington inversion using the
following method:

\begin{itemize}
\tightlist
\item
  \(\Psi\) is taken as known from the underlying assumed analytic dark
  matter potential.
\item
  \(\rho_\star\) is
\end{itemize}

\subsubsection{Stellar profiles}\label{stellar-profiles}

We consider several different stellar profiles in this work.

\paragraph{Exp2D}\label{exp2d}

The 2D exponential stellar profile is given by \[
\Sigma_\star(R) = A e^{-R / R_s}
\] where \(A\) is some normalization and \(R_s\) is the exponential
scale radius. The 3D deprojected profile is found through abel inversion
(e.g.~Rapha\ldots), i.e. \[
\rho_\star (r) = \int dR = xxx
\]

\paragraph{Plummer}\label{plummer}

\subsection{Dark Matter simulations}\label{dark-matter-simulations}

See \citet{power+2003} for a discussion of this.

The gravitational softening is based on: \[
h_{grav} = 4 \frac{R_{200}}{\sqrt{N_{200}}}
\]

I in fact find that dividing the softening by a factor of sqrt(10) gives
slightly more precise results with a negligible increase in compute
time.

\section{Appendix: Convergence tests and Simulation
Parameters}\label{appendix-convergence-tests-and-simulation-parameters}

Here, we describe some convergence tests to ensure our methods and
results are minimally impacted by numerical limitations and assumptions.
See \citet{power2003} for a detailed discussion of various assumptions
and parameters used in N-body simulations.

\subsubsection{Particle Number.}\label{particle-number.}

\subsubsection{Timestepping}\label{timestepping}

\subsubsection{Gravitational methods}\label{gravitational-methods}

\subsubsection{Gravitational accuracy}\label{gravitational-accuracy}

\subsubsection{Fiducial Parameters}\label{fiducial-parameters}

\begin{verbatim}
# -----------------------------------IO---------------------------------------

#---Filenames
InitCondFile                initial
OutputDir                   ./out
SnapshotFileBase            snapshot
OutputListFilename          outputs.txt

#---File formats 
ICFormat                    3       # use HDF5
SnapFormat                  3 

#---Mem & CPU limits
TimeLimitCPU                86400   # TUNE
CpuTimeBetRestartFile       7200    # TUNE 
MaxMemSize                  2400    # TUNE to system (MB)

#---Time
TimeBegin                   0
TimeMax                       2120      # 10 Gyr

#---Output frequency
OutputListOn                0
TimeBetSnapshot             10      # change as needed
TimeOfFirstSnapshot         0       # MATCH to TimeBegin
TimeBetStatistics           10      # change as needed 
NumFilesPerSnapshot         1
MaxFilesWithConcurrentIO    1 


# ----------------------------------Gravity-----------------------------------

# MATCH all the below if continuing run

#---Timestep accuracy
ErrTolIntAccuracy           0.01    # MATCH
CourantFac                  0.1     # ignored; for SPH
MaxSizeTimestep             0.5     # MATCH
MinSizeTimestep             0.0 

#---Tree algorithm
TypeOfOpeningCriterion      1       # 0: Barnes-Hut, 1: Relative
ErrTolTheta                 0.5     # mostly used for Barnes-Hut
ErrTolThetaMax              1.0     # (used only for relative)
ErrTolForceAcc              0.005   # (used only for relative)

#---Domain decomposition: should only affect performance
TopNodeFactor                       3.0
ActivePartFracForNewDomainDecomp    0.02

#---Gravitational Softening
SofteningComovingClass0      # FILL IN
SofteningMaxPhysClass0       0      # ignored; for cosmological
SofteningClassOfPartType0    0
SofteningClassOfPartType1    0


#--------------------------------Misc-------------------------------------

# probably do not need to change the options below 

#---Unit System
UnitLength_in_cm            1 
UnitMass_in_g               1
UnitVelocity_in_cm_per_s    1 
GravityConstantInternal     1

#---Cosmological Parameters 
ComovingIntegrationOn     0 # no cosmology
Omega0                    0
OmegaLambda                 0 
OmegaBaryon                 0
HubbleParam                 1
Hubble                      100
BoxSize                     0

#---SPH
ArtBulkViscConst             0.8
MinEgySpec                   0
InitGasTemp                  100

#---Initial density estimate (SPH)
DesNumNgb                   64
MaxNumNgbDeviation          1
\end{verbatim}
