In this section, we discuss our methods used for the following two
chapters. We utilize an analytic, static, axisymmetric Milky Way
potential based on \citet{mcmillan2011}. Our N-body simulation assume
collisionless, cold dark matter and are integrated with Gadget-4. We
check for numerical convergence and stability of initial conditions in
isolation.

\section{Milky Way potential}\label{milky-way-potential}

We adopt the Milky Way potential described in \citet{EP2020}, an
analytic approximation of \citet{mcmillan2011}.
Fig.~\ref{fig:v_circ_potential} plots the circular velocity profiles of
each component and the total circular velocity profile for our fiducial
profile. The potential consisting of a stellar bulge, a thin and thick
disk, and a dark matter NFW halo. Among typical orbits for Sculptor and
Ursa Minor, the dark matter halo dominates the potential.

The potential is specified as follows. The galactic bulge is described
by a \citet{hernquist1990} potential,

\begin{equation}\protect\phantomsection\label{eq:hernquist}{
\Phi(r) = - \frac{GM}{r + a},
}\end{equation} where \(a=1.3\,{\rm kpc}\) is the scale radius and
\(M=2.1 \times 10^{10}\,\Mo\) is the total mass. The thin and thick
disks are represented with the \citet{miyamoto+nagai1975} cylindrical
potential:

\begin{equation}\protect\phantomsection\label{eq:mn75}{
\Phi(R, z) = \frac{-GM}{\left(R^2 + \left[a + \sqrt{z^2 + b^2}\right]^{2}\right)^{1/2}}
}\end{equation}

where \(a\) is the disc radial scale length, \(b\) is the scale height,
and \(M\) is the total mass of the disk. For the thin disk,
\(a=3.944\,\)kpc, \(b=0.311\,\)kpc,
\(M=3.944\times10^{10}\,\)M\(_\odot\). For the thick disk,
\(a=4.4\,\)kpc, \(b=0.92\,\)kpc, and \(M=2\times10^{10}\,\)M\(_\odot\).
The halo is a NFW dark matter halo (Eq.~\ref{eq:nfw}) with
\(M_s=79.5\times10^{10}\,\)M\(_\odot\). and \(r_s = 20.2\,\)kpc.

\begin{figure}
\centering
\pandocbounded{\includegraphics[keepaspectratio]{figures/v_circ_potential.png}}
\caption[Circular velocity of potential]{Circular velocity profile of
\citet{EP2020} potential.}\label{fig:v_circ_potential}
\end{figure}

\section{Orbital estimation}\label{orbital-estimation}

To estimate the possible orbits of a dwarf galaxy, we perform a Monte
Carlo sampling of the present-day observables. The present-day position,
distance modulus, LOS velocity, and proper motions are each sampled from
normal distributions given the reported uncertainties in
Tables~\ref{tbl:scl_obs_props}, \ref{tbl:umi_obs_props}. We integrate
each sampled observable back in time for 10 Gyr in Gadget as massless
point particles outputting positions every 5Myr (with otherwise similar
parameters to n-body runs below.) When selecting the initial position
and velocity of an N-body model, we select the position and velocity of
the first apocentre occurring after 10 Gyr ago.

We assume the galaxy is represented as a point particle for this
analysis. In detail, (self) dynamical friction likely influences the
orbit, but a point particle is an adequate description.

To convert from Gaia to galactocentric coordinates, we use the Astropy
v4 galactocentric frame \citep{astropycollaboration+2022}. This frame
assumes the galactic centre appears at position
\(\alpha = {\rm 17h45m37.224s}\), \(\delta = -28°56'10.23''\) with
proper motions \(\mu_{\alpha*}=-3.151\pm0.018\ \masyr\) ,
\(\mu_\delta=-5.547\pm0.026 \masyr\) (from the appendix and Table 2 of
\citet{reid+brunthaler2004}). The galactic centre distance is
\(8.122\pm0.033\,\)kpc with a radial velocity = \(11 + 1.9 \pm 3\) km/s.
from \citet{gravitycollaboration+2018}. Finally, adding that the sun is
\(20.8\pm0.3\,\)pc above the disk from \citet{bennett+bovy2019}, and
using the procedure outlined in \citet{drimmel+poggio2018}, the solar
velocity relative to the galactic rest frame is `v\_sun' =
\([-12.9 \pm 3.0, 245.6 \pm 1.4, 7.78 \pm 0.08]\) km/s. The
uncertainties in the reference frame are typically smaller than the
uncertainties on the dwarf galaxy's position and velocity.

\section{N-Body modelling}\label{n-body-modelling}

\subsection{Initial conditions}\label{initial-conditions}

We use Agama \citep{agama} to generate initial conditions. We initially
assume galaxies are described by an NFW dark matter potential
Eq.~\ref{eq:nfw} and the stars are merely collisionless tracers embedded
in this potential (added on in post-processing). The density is a
cubic-exponentially truncated with a profile
\begin{equation}\protect\phantomsection\label{eq:trunc_nfw}{
\rho_{\rm tNFW} = e^{-(r/r_t)^3}\ \rho_{\rm NFW}(t)
}\end{equation} where we adopt \(r_t = 20 r_s\) or approximately
\(r_{200}\) for our Sculptor-like fiducial halo. Using \(r_{200}\) for
\(r_t\) would depend on the chosen scale of the halo. So our adopted
\(r_t\) is an approximate upper limit of \(r_{200}\) for typical dwarf
galaxy halos \citep{ludlow+2016}. It is unlikely that the outer density
profile of loosely bound particles past \(20\) kpc affects the tidal
evolution of a subhalo.

\subsection{Isolation runs and simulation
parameters}\label{isolation-runs-and-simulation-parameters}

To ensure that the initial conditionss of the simulation are dynamically
relaxed and well-converged, we run the simulation in isolation (no
external potential) for 5 Gyr (or about 3 times the crossing timescale
at the virial radius). Our fiducial isolation halo uses \(r_s=2.76\) kpc
and \(M_s = 0.29 \times 10^{10}\) Msun, but can be easily rescaled for
any length or mass scale.

For our simulation parameters, we adopt a softening length of
\begin{equation}{
h_{\rm grav} = 0.014 \left(\frac{r_s}{2.76\,{\rm kpc}}\right)\left(\frac{N}{10^7}\right)^{-1/2}.
}\end{equation} See appendix Section~\ref{sec:extra_convergence} for a
discussion of this choice, which is similar to the \citet{power+2003}
suggested softening. We use the relative tree opening criterion with the
accuracy parameter set to 0.005, and adaptive time stepping with
integration accuracy set to 0.01.

\subsubsection{Numerical convergence}\label{numerical-convergence}

As discussed in \citet{power+2003}, the region where density is
converged is related to the region which becomes collisionally relaxed
over the time of the universe (so about 5-10 Gyr). (i.e.~where the
collisionless assumption breaks down). The relaxation timescale is given
by \begin{equation}\protect\phantomsection\label{eq:t_relax}{
t_{\rm relax}(r) = t_{\rm circ}(r) \frac{N(r)}{8\,\ln N(r)}
= {t_{\rm circ}(r_{200})} \frac{\sqrt{200}}{8} \frac{N(r)}{\ln N(r)} \left(\frac{\bar \rho (r)}{\rho_{\textrm crit}}\right)^{-1/2}
}\end{equation} where \(N(r)\) is the number of particles within \(r\),
\(t_{\rm circ}\) is the circular velocity timescale, and \(\bar \rho\)
is the mean interior density \citep{power+2003}. The converged radius
\(r_{\rm relax}(t)\) is the radius above which \(t_{\rm relax} > t\).
Typically, \(r_{\rm relax}(10\Gyr)\) is about 6-10 times our adopted
softening length, increasing with particle number. As such, at full
resolution, we can only trust density profiles down to \(\sim10\) times
our softening length, sufficient to resolve stellar density profiles.

\begin{figure}
\centering
\pandocbounded{\includegraphics[keepaspectratio]{figures/iso_converg_num.png}}
\caption[Numerical halo convergence]{Numerical convergence test for
circular velocity as a function of log radius for simulations with
different total numbers of particles in isolation. Residuals in bottom
panel are relative to NFW. The initial conditions are dotted and the
converged radius is marked by arrows (Eq.~\ref{eq:t_relax}). Note that a
slight reduction in density starting around \$r = 30 \$kpc is expected
given our truncation choice.}\label{fig:numerical_convergance}
\end{figure}

\subsection{Orbital runs}\label{orbital-runs}

To perform the simulations of a given galaxy in a given potential, we
centre the isolation run's final snapshot
(Section~\ref{sec:shrinking_spheres}) and place the dwarf galaxy in the
specified orbit in the given potential. We typically run the simulation
for 10 Gyr, which allows us to orbit slightly past the expected initial
conditions.

\section{Post processing}\label{post-processing}

\subsection{Centring}\label{sec:shrinking_spheres}

Shrinking spheres centres inspired by \citet{power+2003}

\begin{itemize}
\tightlist
\item
  Recursively shrink radius by 0.975 quantile and recalculating centroid
  until radius is less than \textasciitilde1kpc or fewer than 0.1\% of
  particles remain.
\item
  Remove bound particles (using instantanious potential).
\item
  Use previous snapshot centre and acceleration to predict new centre
  for next snapshot and only include particles included in previous
  centring timestep.
\end{itemize}

The statistical centring uncertainty for the 1e7 particle isolation run
is of order 0.003 kpc, however oscillations in the centre are of order
0.03 kpc. This is about three times the softening length but is less
than the numerically converged radius scale.

\subsection{Velocity profiles}\label{velocity-profiles}

\begin{itemize}
\tightlist
\item
  Circular velocity is computed assuming spherical symmetry and only
  shown for every 200th particle (ranked from the centre outwards)
\item
  \(v_{\rm circ\ max}\) and \(r_{\rm circ,\ max}\) is found by
  least-squares fitting the NFW functional velocity form to the points
  of the velocity profile that have the 10\% highest velocities. This is
  not a necessarily good fit, especially as the halo becomes stripped,
  but accurate enough to find a reliable maximum.
\end{itemize}

\subsection{Stellar probabilities}\label{stellar-probabilities}

We ``paint'' stars onto dark matter particles using the particle tagging
method \citep[e.g.][]{bullock+johnston2005}, assuming spherical
symmetry. Let \(\Psi\) be the potential (normalized to vanish at
infinity) and \({\cal E}\) is the binding energy
\({\cal E} = \Psi - 1/2 v^2\). If we know \(f({\cal E})\), the
distribution function (phase-space density in energy), then we assign
the stellar weight for a given particle with energy \({\cal E}\) is

\begin{equation}{
P_\star({\cal E}) = \frac{f_\star({\cal E})}{f_{\rm halo}({\cal E})}.
}\end{equation} While \(f({\cal E})\) is a phase-space density, the
differential energy distribution includes an additional \(g({\cal E})\)
occupation term (BT87). We use Eddington inversion to find the
distribution function, (eq. 4-140b in BT87)

\begin{equation}{
f({\cal E}) = \frac{1}{\sqrt{8}\, \pi^2}\left( \int_0^{\cal E} \frac{d^2\rho}{d\Psi^2} \frac{1}{\sqrt{{\cal E} - \Psi}}\ d\Psi + \frac{1}{\sqrt{\cal E}} \left(\frac{d\rho}{d\Psi}\right)_{\Psi=0} \right).
}\end{equation}

In practice the right, boundary term is zero as \(\Psi \to 0\) as
\(r\to\infty\), and if \(\rho \propto r^{-n}\) at large \(r\) and
\(\Psi \sim r^{-1}\) then \(d\rho / d\Psi \sim r^{-n+1}\) which goes to
zero provided that \(n > 1\). We take \(\Psi\) from the underlying
assumed analytic dark matter potential. \(\rho_\star\) can be calculated
from the surface density, \(\Sigma_\star\), via the inverse Abel
transform. In this work, we consider both Plummer and exponential
profiles, as described in

\subsection{Stellar density profiles and velocity
dispersion}\label{stellar-density-profiles-and-velocity-dispersion}

Stellar velocity dispersion is calculated for all stars within a 1kpc
sphere of the centre. We assume
\(\sigma_{\rm los} = \sqrt{\sigma_x^2 + \sigma_y^2 + \sigma_z^2}/ \sqrt 3\),
i.e.~isotropic velocity dispersion.

We calculate density profiles similar to stars and assume a constant bin
width in \(\log R\) of \(2 {\rm IQR} / \sqrt[3]{N}\) (Freedman-Diaconis
prescription). \(R\) may be either the cylendrical radius in \(x-y\) or
the on-sky project \(\xi, \eta\) tangent plane coordinates.

Because our dwarfs are assumed to be spherical/isotropic, we retain this
assumption when calculating predicted density profiles.
