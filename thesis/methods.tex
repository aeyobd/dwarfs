In the previous chapters, we have seen that the Scl and UMi classical
dSphs have outer density profiles that appear to deviate from the
exponential law that approximates well all other classical dSphs. Our
main intention is to assess whether such excesses result from the
effects of Galactic tides. To this purpose, we intend to use N-body
simulations of the evolution of CDM halos in a Galactic potential,
constrained to have the orbital parameters consistent with a dwarf's
present-day position and velocity. We shall assume that the Galactic
potential is the static, analytic potential inferred by
\citet{mcmillan2011} from observations of kinematic tracers. We also
assume the potential of each dwarf may be initially approximated by a
cuspy NFW profile. Since the dwarfs in question are heavily dark matter
dominated, we shall use a carefully selected sample of dark matter
particles to mimic and track the evolution of an embedded tracer stellar
component. In this Chapter, I describe the Galactic potential used, the
orbital estimation method, the initial conditions setup, and the N-body
method used.

\section{Orbital estimation}\label{orbital-estimation}

To explore the possible orbits of a dwarf galaxy, we perform a Monte
Carlo sampling of the present-day observables. The present-day position,
distance modulus, LOS velocity, and proper motions are each sampled from
normal distributions given the reported uncertainties in
Tables~\ref{tbl:scl_obs_props}, \ref{tbl:umi_obs_props}. We integrate
each sampled position/velocity back in time for 10 Gyr using \agama{}
\citep{agama}. Dynamical friction is not expected to impact orbits
substantially because of the low masses and large pericentres of the
dwarfs.

\subsection{Galactocentric frame}\label{galactocentric-frame}

To convert observed positions and velocities to Galactocentric
coordinates, we use the Astropy v4 Galactocentric frame
\citep{astropycollaboration+2022}. This frame assumes the Galactic
centre is at position \(\alpha = {\rm 17h\,45m\,37.224s}\),
\(\delta = -28^\circ\,56'\,10.23''\) with proper motions
\(\mu_{\alpha*}=-3.151\pm0.018\ \masyr\) ,
\(\mu_\delta=-5.547\pm0.026 \masyr\) \citep[from the appendix and Table
2 of][]{reid+brunthaler2004}. The Galactic centre is at a distance from
the Sun of \(8.122\pm0.033\,\)kpc with a radial velocity =
\(11 + 1.9 \pm 3\,\kms\) \citep{gravitycollaboration+2018}. The Sun is
assumed to be \(20.8\pm0.3\,\)pc above the disk
\citep{bennett+bovy2019}. Using the procedure outlined in
\citet{drimmel+poggio2018}, the Solar velocity relative to the Galactic
rest frame is then
\(\V_\odot = [-12.9 \pm 3.0, 245.6 \pm 1.4, 7.78 \pm 0.08]\) km/s. The
uncertainties in the reference frame are typically smaller than the
uncertainties on a dwarf galaxy's position and velocity.

\subsection{Milky Way Potential}\label{milky-way-potential}

We adopt the Milky Way potential described in \citet{EP2020}, which is
an analytic approximation to that proposed by \citet{mcmillan2011}.
Fig.~\ref{fig:v_circ_potential} plots the circular velocity profiles of
each component and the total circular velocity profile for our fiducial
profile. The potential includes a stellar bulge, a thin and thick disk,
and a dark matter NFW halo.

The Galactic bulge is described by a \citet{hernquist1990} potential,

\begin{equation}{
\Phi(r) = - \frac{GM}{r + a},
}\end{equation} where \(a=1.3\,{\rm kpc}\) is the scale radius and
\(M=2.1 \times 10^{10}\,\Mo\) is the total mass. The thin and thick
disks are represented with the \citet{miyamoto+nagai1975} cylindrical
potential, \begin{equation}{
\Phi(R, z) = \frac{-GM}{\left(R^2 + \left[a + \sqrt{z^2 + b^2}\right]^{2}\right)^{1/2}},
}\end{equation} where \(a\) is the disc radial scale length, \(b\) is
the scale height, and \(M\) is the total mass of the disk. For the thin
disk, \(a=3.944\,\)kpc, \(b=0.311\,\)kpc, and
\(M=3.944\times10^{10}\,\)M\(_\odot\). For the thick disk,
\(a=4.4\,\)kpc, \(b=0.92\,\)kpc, and \(M=2\times10^{10}\,\)M\(_\odot\).
The halo is an NFW dark matter halo (Eq.~\ref{eq:nfw}) with
\(\rmax = 43.7\,\)kpc and \(\vmax = 191\,\kms\).

\begin{figure}
\centering
\pandocbounded{\includegraphics[keepaspectratio]{figures/v_circ_potential.pdf}}
\caption[Circular velocity of potential]{Circular velocity profile of
\citet{EP2020} potential. The total circular velocity (thick black line)
is composed of an NFW halo (green dashed line), a think and thick
\citet{miyamoto+nagai1975} disk (orange dash-dotted line), and a
\citet{hernquist1990} bulge (blue dotted
line).}\label{fig:v_circ_potential}
\end{figure}

\subsection{Orbits of Sculptor}\label{orbits-of-sculptor}

Sculptor's orbital history is relatively well-constrained.
Fig.~\ref{fig:scl_orbits} illustrates point particle orbits for 100
samples of Sculptor's observed kinematics integrated backwards 5 Gyr in
both Galactocentric coordinate slices (\(x\), \(y\), \(z\)) and in
Galactocentric radius with time. All sampled orbits of Sculptor have
nearly the same morphology---the orbit primarily resides in the
\(y\)--\(z\) plane and has a similar number of periods and
pericentre/apocentre.

To maximize tidal effects, we select an orbit with the \(\sim 3\sigma\)
smallest pericentre among all possible orbits. We achieve this by taking
the median parameters of all orbits with a pericentre less than the
0.0027th quantile pericentre, yielding a pericentre of 43 kpc. Given the
current observations, it is unlikely that Sculptor has a significantly
smaller pericentre than our selected orbit.

We take the first apocentre after a look-back time of 10 Gyr, or at 9.43
Gyr, as the initial conditions for our model of Sculptor, noted in
Table~\ref{tbl:scl_orbits}.

\begin{figure}
\centering
\pandocbounded{\includegraphics[keepaspectratio]{/Users/daniel/thesis/figures/scl_xyzr_orbits.pdf}}
\caption[Sculptor Orbits]{The orbits of Sculptor in a static Milky Way
potential in Galactocentric \(x\), \(y\), and \(z\) coordinates (top)
and in Galactocentric radius \(r\) versus time (bottom). The Milky Way
is at the centre with the disk lying in the \(x\)--\(y\) plane. Our
selected \texttt{smallperi} orbit is plotted in black and light blue
transparent orbits represent the past 5Gyr orbits sampled over Sculptor
observables in Table~\ref{tbl:scl_obs_props}. The orbit of sculptor is
well-constrained in this potential and it is unlikely to achieve a
smaller pericentre than our selected orbit.}\label{fig:scl_orbits}
\end{figure}

\begin{figure}
\centering
\pandocbounded{\includegraphics[keepaspectratio]{/Users/daniel/thesis/figures/umi_xyzr_orbits.pdf}}
\caption[Ursa Minor Orbits]{Similar to Fig.~\ref{fig:scl_orbits}, the
orbits of Ursa Minor in a static Milky Way potential in Galactocentric
\(x\), \(y\), and \(z\) coordinates. In the lower panel, we show the
radius versus time for only three orbits of Ursa
Minor.}\label{fig:umi_orbits}
\end{figure}

\begin{table*}[t]
\centering
\caption[Sculptor Selected Orbits]{Properties of selected orbits for Sculptor. The mean orbit represents the observational mean from Table \ref{tbl:scl_obs_props}. The Smallperi represents instead the $3\sigma$ smallest pericentre, which we use to provide an upper limit on tidal effects. }
\label{tbl:scl_orbits}
\begin{tabular}{lll}
\toprule
Property & Mean & SmallPeri\\
\midrule
distance / kpc & 83.2 & 82.6\\
$\pmra / \masyr$ & 0.099 & 0.134\\
$\pmdec / \masyr$ & -0.160 & -0.198\\
LOS velocity / $\kms$ & 111.2 & 111.2\\
$t_i / \Gyr$ & -8.74 & -9.43\\
$\vec{x}_{i} / \kpc$ & [16.13, 92.47, 39.63] & [-2.49, -42.78, 86.10]\\
$\vec{v}_i / \kms$ & [-2.37, -54.70, 128.96] & [-20.56, -114.83, -57.29]\\
pericentre / kpc & 53 & 43\\
apocentre / kpc & 102 & 96\\
$t_{\rm last\ peri} / {\rm Gyr}$ & -0.45 & -0.46\\
Number of pericentres & 6 & 6\\
\bottomrule
\end{tabular}
\end{table*}

\subsection{Orbits of Ursa Minor}\label{orbits-of-ursa-minor}

Similar to Sculptor, Ursa Minor has a well-constrained orbit.
Fig.~\ref{fig:umi_orbits} shows 100 random point-orbits of Ursa Minor.
Initially, we select an orbit with approximately the \(3\sigma\)
smallest pericentre as in Sculptor.

Ursa Minor's N-body orbit diverges from the point particle orbit. To
ensure the final conditions of the N-body simulation are close to the
intended final position, we iteratively adjust Ursa Minor's initial
conditions. Initially, starting with low-resolution runs, we adjust the
cylindrical actions of the initial orbit by the final difference in
actions at the end of orbital evolution. After the initial actions have
converged (2 iterations), we change the initial action angles by the
final difference in action angles. This method converges within 4
iterations to an orbit agreeing with the observed kinematics of Ursa
Minor.

\begin{table*}[t]
\centering
\caption[Ursa Minor Selected Orbits]{Properties of selected orbits for Ursa Minor. The "smallperi" orbit is the initial point orbit and the "smallperi.5" is the initial orbit for the N-body simulation. }
\label{tbl:umi_orbits}
\begin{tabular}{llll}
\toprule
Property & Mean & SmallPeri.1 & SmallPeri.5\\
\midrule
distance / kpc & $70.1$ & 64.6 & \\
$\pmra / \masyr$ & $-0.124 \pm 0.17$ & -0.158 & \\
$\pmdec / \masyr$ & $0.078\pm0.17$ & 0.05 & \\
$v_{\rm LOS} / \kms$ & $-245.9\pm 1$ & -245.75 & \\
$t_i / \Gyr$ & -8.74 & -9.53 & -9.53\\
$\vec{x}_{i} / \kpc$ & [4.88, -65.11, 50.78] & [-16.48 69.92 21.05] & [17.40, 74.51, 21.34]\\
$\vec{v}_i / \kms$ & [-34.28, 77.11, 101.38] & [16.32, 39.86, -116.99] & [14.27, 48.62, -114.08]\\
pericentre / kpc & 37 & 29 & 28\\
apocentre / kpc & 83 & 75 & 72\\
$t_{\rm last\ peri} / \Gyr$ & $-0.96$ & $-0.80$ & $-0.81$\\
Number of pericentres & 9 & 6 & 6\\
\bottomrule
\end{tabular}
\end{table*}

\section{Initial conditions}\label{initial-conditions}

We use \agama{} \citep{agama} to generate the initial N-body dark matter
halo. We assume galaxies are described by an NFW dark matter potential
(Eq.~\ref{eq:nfw}). We also assume the stars do not contribute to the
potential. The dark matter density is truncated in the outer regions by
\begin{equation}{
\rho_{\rm tNFW} = e^{-(r/r_t)^3}\ \rho_{\rm NFW}(t),
}\end{equation} where we adopt \(r_t = 20 r_s\).

\subsection{Initial dark matter halos for Sculptor and Ursa
Minor}\label{initial-dark-matter-halos-for-sculptor-and-ursa-minor}

From the observed properties of Sculptor and Ursa Minor, we infer
reasonable \LCDM{} initial halo conditions.

Table~\ref{tbl:derived_props} reports our inferred halo and kinematic
properties of Sculptor and Ursa Minor. First, taking the absolute
magnitudes from \citet{munoz+2018} with the mass-to-light ratio from
\citet{woo+courteau+dekel2008} (with \(\sim\) 0.17 dex uncertainty), the
total current stellar mass of Sculptor and Ursa Minor are
\(M_\star \sim 3.1 \times 10^6 \Mo\) and
\(M_\star \sim 7 \times 10^5 \Mo\). Based on the stellar mass-\(\vmax\)
relation \citep[from][]{fattahi+2018}, see also \ref{fig:smhm}, Sculptor
and Ursa Minor's halos should have \(\vmax \approx 31 \,\kms\) and
\(\vmax \approx 27\,\kms\). Finally, using the \citet{ludlow+2016}
\(z=0\) mass-concentration relation, this constraint translates into a
\(\rmax \approx 6 {\rm kpc}\) and \(\rmax \approx 5\,\kpc\) for each
galaxy.

Fig.~\ref{fig:scl_halos} illustrates these estimates visually for both
galaxies. The stellar-mass \(\vmax\) constraint translates to an
estimate of \(\vmax\) only (horizontal band). The mass-concentration
relation describes the relationship between \(\vmax\) and \(\rmax\)
(diagonal linear band). And finally, we include a curved line which
illustrates the halo \(\vmax\) which has a specified initial LOS
velocity dispersion given \(\rmax\)
\(, \vcirc(R_h) / \sqrt{3} \approx \sigma_v\).
\(R_h \approx 0.24\,\kpc\) for both galaxies.

While there is some range in the choice of initial halo, reasonable
changes to the initial halo do not substantially affect the tidal
evolution for either galaxy. The observed velocity dispersion is
directly related to the mass within a half-light radius
\citep[e.g.,][]{wolf+2010}. As a result, halos with the same velocity
dispersion may differ in total mass but should have similar mass within
\(R_h\). So, the tidal effects on stars should be similar for halos with
similar velocity dispersions.

Our selected halos (in Table~\ref{tbl:initial_halos}) for each
simulation run are based on the cosmological constraints and the need to
match the present-day velocity dispersion at the end of the simulation.

\begin{table*}[t]
\centering
\caption[Derived Properties of Sculptor and Ursa Minor]{Inferred properties of the stellar component and halo for Sculptor and Ursa Minor. We record the total luminosity, stellar mass, mass-to-light ratio, dark matter halo $\vmax$ and $\rmax$, and dark matter halo virial mass $M_{200}$ and concentration $c_{\rm NFW}$. }
\label{tbl:derived_props}
\begin{tabular}{lll}
\toprule
parameter & Sculptor & Ursa Minor\\
\midrule
$L_\star$ & $1.8\pm0.2\times10^6\ L_\odot$ & $3.5 \pm 0.1 \times 10^5\,L_\odot$\\
$M_\star$ & $3.1_{-1.0}^{+1.6} \times10^6\ {\rm M}_\odot$ & $7_{-2}^{+3} \times 10^5\,\Mo$\\
$M_\star / L_\star$ & $1.7\times 10^{\pm 0.17}$ & $1.9 \times 10^{\pm 0.17}$\\
$\vmax$ & $31\pm 3\,\kms$ & $27_{-6}^{+7}\,\kms$\\
$\rmax$ & $6 \pm 2$ kpc & $5_{-2}^{+1}$ kpc\\
$M_{200}$ & $0.5 \pm 0.2\times10^{10}\ M_0$ & $3_{-2}^{+4} \times 10^9\,\Mo$\\
$c_{\rm NFW}$ & $13_{-3}^{+4}$ & 14?\\
\bottomrule
\end{tabular}
\end{table*}

\begin{figure}
\centering
\pandocbounded{\includegraphics[keepaspectratio]{figures/initial_halo_selection.pdf}}
\caption[Sculptor initial halos]{Selection of initial halos for Sculptor
and Ursa Minor. The grey line and pink line with shaded regions
represent the \citet{ludlow+2016} mass-concentration relation and
\citet{fattahi+2018} SMHM relation respectively. The curved lines
represent the velocity dispersion of the initial halo given the
present-day half-light radius \citep[via the][ mass
estimator]{wolf+2010}.}\label{fig:scl_halos}
\end{figure}

\begin{table*}[t]
\centering
\caption[Initial halos]{The initial conditions for our initial dark matter halos. }
\label{tbl:initial_halos}
\begin{tabular}{lllll}
\toprule
Halo name & $\rmax$ & $\vmax$ & $M_{200}$ & $c_{\rm NFW}$\\
\midrule
Scl: fiducial & 3.2 & 31 & 0.33 & 21\\
Scl: small & 2.5 & 25 &  & \\
UMi: fiducial & 4 & 38 & 0.62 & 21\\
\bottomrule
\end{tabular}
\end{table*}

\section{Numerical methods}\label{numerical-methods}

To simulate the tidal evolution of galaxies, we use ``N-body''
simulations integrated with the parallel, gravitational-tree program
\gadget{} \citep{gadget4}. The N-body method calculates and evolves the
gravitational accelerations between a large number of collisionless
particles to approximate the dynamical evolution of matter. To
approximate a collisionless system (i.e., without strong
particle-particle gravitational deflections), the force is softened
below a ``softening length.'' Regions which are smaller than the
softening length, or have few particles, are not well-resolved. Tree
codes organize particles into a spatial tree, enabling grouping of the
gravitational forces from nearby particles. A gravitational tree code
substantially reduces the required number of force computations, as
compared to the exact, direct-summation method.

\subsection{Isolation runs and simulation
parameters}\label{isolation-runs-and-simulation-parameters}

To ensure that the initial conditions of the simulation are dynamically
relaxed and well-converged, we run a halo first in isolation using
\gadget{}. Since gravity is scale-free, we use the same isolation run
for all halos. We adopt \(\rmax = 6.0\,\)kpc and \(\vmax = 31\,\kms\)
for the isolation halo based on Sculptor's mean properties. We run this
model for 5 Gyr (about three times the free fall timescale
\(t_{\rm ff} = \vcirc / r \approx 1.5\,\Gyr\) at \(r_{200}=36\,\)kpc).

For our simulation parameters, we adopt a softening length of
\begin{equation}\protect\phantomsection\label{eq:softening_length}{
h_{\rm grav} = 0.014\,{\rm kpc}\left(\frac{r_{\rm max}}{6.0\,{\rm kpc}}\right)\left(\frac{N}{10^7}\right)^{-1/2}.
}\end{equation} See Appendix Section~\ref{sec:extra_convergence} for a
discussion of this choice, which is similar to the \citet{power+2003}
suggested softening. We use the relative tree opening criterion with the
accuracy parameter set to 0.005, and adaptive time stepping with
integration accuracy set to 0.01.

\subsection{Numerical fidelity}\label{numerical-fidelity}

Fig.~\ref{fig:numerical_convergence} illustrates how well our numerical
setup is able to reproduce the desired initial conditions, before and
after running the model in isolation. This figure shows that our
numerical methods are able to approximate well an NFW halo down to an
innermost radius that strongly depends on resolution. The larger the
number of particles, the smaller the radius that is effectively
``resolved'' in a given simulation. For the Sculptor halo shown in this
figure (with \(\rmax = 6.0\,\)kpc and \(\vmax = 31\,\kms\) ), a
simulation with \(10^7\) particles is needed to resolve the innermost
100 pc. For reference, the half-light radius of Sculptor is roughly 100
pc, which means that at least 10 million particles would be needed to
follow faithfully its tidal evolution. Vertical arrows in
Fig.~\ref{fig:numerical_convergence} indicate the ``convergence radius''
defined by \citet[eq.\textasciitilde13]{power+2003} for NFW halos formed
in cosmological N-body simulations. This radius marks the region where
collisional effects driven by the finite number of particles used to
describe the innermost regions of a halo become important. The softening
length (from Eq.~\ref{eq:softening_length}) is typically a few times
smaller than the converged length.

\begin{figure}
\centering
\pandocbounded{\includegraphics[keepaspectratio]{figures/iso_converg_num.pdf}}
\caption[Numerical halo convergence]{Numerical convergence test for
circular velocity as a function of log radius for simulations with
different total numbers of particles in isolation. Residuals in bottom
panel are relative to NFW. The initial conditions are dotted, the
converged radius is marked by arrows \citep[from][ eq. 13]{power+2003},
and the softening length is marked by a vertical bar. Note that a slight
reduction in density starting around \(r = 30\,\kpc\) is expected given
our truncation choice.}\label{fig:numerical_convergence}
\end{figure}

\subsection{Orbital evolution}\label{orbital-evolution}

Next, we evolve the halo in the Galactic potential. We scale the relaxed
snapshot and softening length to match the initial halo in
\ref{tbl:initial_halos}, and shift the snapshot to the initial
conditions from the orbital analysis (see
Tables~\ref{tbl:scl_orbits}, \ref{tbl:umi_orbits}). We then evolve the
full N-body NFW model forward in time in the Galactic potential and
follow it in time until the present time, when the halo is closest to
the observed position of the galaxy.

\subsection{Tidal mass losses}\label{sec:shrinking_spheres}

To accurately follow the evolution of a halo, it is important to
determine the centre of the halo at each time chosen for analysis. We
use a shrinking-spheres centre method inspired by \citet{power+2003}.
First, we start with an initial centre estimate from the last timestep.
Then, we calculate the radius of all particles from the centre, remove
particles with a radius beyond the 0.975 quantile of the centre, and
recalculating the centre of mass. The procedure is repeated until the
selection radius is less than \textasciitilde1kpc or fewer than 0.1\% of
particles remain. After a centre has been chosen, we remove all unbound
particles based on the \gadget{} calculated potential of the halo. For
all future timesteps, we consider only particles retained from the
previous iteration.

The statistical centring uncertainty for the full resolution (\(10^7\)
particle) isolation run is of order 0.003 kpc, but fluctuations are
observed of \(\sim 0.03\,\kpc\). This is about three times the softening
length but is less than the numerically converged radius scale.

\subsection{The stellar component}\label{sec:painting_stars}

We ``paint'' stars onto dark matter particles using the particle-tagging
method \citep[e.g.][]{bullock+johnston2005}, assuming spherical
symmetry. We initially assume exponential stars with
\(R_s = 0.10\,,\kpc\) for both galaxies.

Let \(\Psi\) be the potential (normalized to vanish at infinity) and
\({\cal E}\) the binding energy \({\cal E} = \Psi - 1/2 v^2\). If we
know \(f({\cal E})\), the distribution function (phase-space density in
energy), then we assign a stellar weight to a given particle with energy
\({\cal E}\) using \begin{equation}{
P_\star({\cal E}) = \frac{f_\star({\cal E})}{f_{\rm halo}({\cal E})}.
}\end{equation} While \(f({\cal E})\) is a phase-space density, the
differential energy distribution includes an additional \(g({\cal E})\)
occupation term (BT87). We use Eddington inversion to find the
distribution function, (eq. 4-140b in BT87) \begin{equation}{
f({\cal E}) = \frac{1}{\sqrt{8}\, \pi^2}\left( \int_0^{\cal E} \frac{d^2\rho}{d\Psi^2} \frac{1}{\sqrt{{\cal E} - \Psi}}\ d\Psi + \frac{1}{\sqrt{\cal E}} \left(\frac{d\rho}{d\Psi}\right)_{\Psi=0} \right).
}\end{equation} In practice the right, boundary term is zero as
\(\Psi \to 0\) as \(r\to\infty\), and if \(\rho \propto r^{-n}\) at
large \(r\) and \(\Psi \sim r^{-1}\) then
\(d\rho / d\Psi \sim r^{-n+1}\) which goes to zero provided that
\(n > 1\). We take \(\Psi\) from the underlying assumed analytic dark
matter potential. \(\rho_\star\) can be calculated from the surface
density, \(\Sigma_\star\), via the inverse Abel transform.

We find the stellar profiles created in this manner are stable in the
isolated systems and show excellent agreement with the assumed stellar
density profile.
