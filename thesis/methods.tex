\section{Simulation Methods}\label{simulation-methods}

In this section, we discuss our simulation methods from orbital
estimation to initial conditions to simulation parameters and software.

We use Agama \citep{agama} for potential specification and initial
conditions, and Gadget-4 \citep{gadget4} for orbit integration and dark
matter simulations (with a custom patch to integrate with Agama).

\subsection{Potential}\label{potential}

Our fiducial potential is as described in \citet{EP2020}. This potential
is an analytic approximation of \citet{mcmillan2011}, consisting of a
stellar Bulge, disk, and a dark matter NFW halo.
fig.~\ref{fig:v_circ_potential} plots the circular velocity profiles.

The \citet{hernquist1990} density profile for the galactic bulge is
parameterized in terms of a characteristic mass, \(M\), and radius
\(a\).

\[
\rho(r) = \frac{M}{2\pi} \frac{a}{r} \frac{1}{(r+a)^3}
\] Hernquist potential with \(r_s=1.3\), mass = 2.1.

\begin{itemize}
\tightlist
\item
  Thin disk. miyamoto+nagai disk with parameters \(a=3.944\),
  \(b=0.311\), \(M=3.944\),
\item
  Thick disk. miyamoto+nagai disk with parameters \(a=4.4\), \(b=0.92\),
  and \(M=2\)
\end{itemize}

\[
\Phi(R, z) = \frac{-GM}{\left(R^2 + \left[a + \sqrt{z^2 + b^2}\right]^{2}\right)^{1/2}}
\]

\begin{itemize}
\tightlist
\item
  NFW Dark matter halo. \(M_s=79.5\) (double check). and \(r_s = 20.2\).
\end{itemize}

Variations to the potential of the inner disk (exclusion of a bar)
should minimally affect our results as no orbit we consider reaches less
than \textasciitilde15 kpc of the MW centre. We exclude the mass
evolution of the halo from this analysis. Over 10 Gyr, this would be
fairly significant (\textasciitilde2x in MW mass) but since we want to
determine the upper limit of tidal effects, it is safe to neglect this.

\begin{figure}
\centering
\includegraphics{figures/v_circ_potential.pdf}
\caption{Circular velocity of potential}\label{fig:v_circ_potential}
\end{figure}

\textbf{Figure:} Circular velocity profile of \citet{EP2020} potential.
We also show the MW and LMC potential from \citet{vasiliev2024} for
their L3M11 model.

\subsection{Orbital Estimation}\label{orbital-estimation}

To estimate the orbits of a dwarf galaxy, we perform a Monte Carlo
sampling of the present-day observables. We integrate each sampled
observable back in time for 10 Gyr in Gadget as massless point particles
outputing every 0.5Myr (with otherwise similar parameters to n-body runs
below.)

To convert from Gaia to galactocentric coordinates, we use the astropy
v4 galactocentric frame. This frame assumes an

\begin{itemize}
\tightlist
\item
  J2000 `ra' = 17h45m37.224s, `dec' = -28°56'10.23'' (appendix),
  `pmracosdec=-3.151±0.018 mas/yr', `pmdec=-5.547±0.026' (Table 2). from
  \citet{reid+brunthaler2004}
\item
  distance = 8.122 ± 0.033 kpc, solar radial velocity = 11 + 1.9 ± 3
  km/s. from \citet{gravitycollaboration+2018}.
\item
  `z\_sun' = 20.8±0.3pc from bennett+bovy2019
\item
  `v\_sun' is slightly more complicated, relying on parameters from
  above and using the procedure described in \citet{drimmel+poggio2018}.
  This results in

  \begin{itemize}
  \tightlist
  \item
    `v\_sun' = {[}-12.9 ± 3.0, 245.6 ± 1.4, 7.78 ± 0.08{]} km/s
  \end{itemize}
\end{itemize}

\subsection{Initial conditions}\label{initial-conditions}

We use Agama \citep{agama} to generate initial conditions. We initially
assume galaxies are described by an NFW dark matter potential (REF) and
the stars are merely collisionless tracers embedded in this potential.
We adopt a truncation radius of 100 times the scale radius (likely
larger than necessary, but should not affect results.)

\subsubsection{Stellar Probabilities}\label{stellar-probabilities}

We assign stellar probabilities via Eddington inversion using the
following method:

\begin{itemize}
\tightlist
\item
  \(\Psi\) is taken as known from the underlying assumed analytic dark
  matter potential.
\item
  \(\rho_\star\) is pre-specified or calculated via abel integration /
  deprojection
\item
  \(f(\epsilon)\) (the energy distribution function) is determined by
  Edington inversion
\end{itemize}

To find the distribution function, we use Eddington inversion (eq.
4-140b in BT87) \[
f({\cal E}) = \frac{1}{\sqrt{8}\, \pi^2}\left( \int_0^{\cal E} \frac{d^2\rho}{d\Psi^2} \frac{1}{\sqrt{{\cal E} - \Psi}}\ d\Psi + \frac{1}{\sqrt{\cal E}} \left(\frac{d\rho}{d\Psi}\right)_{\Psi=0} \right)
\]

where \({\cal E}\) is the binding energy, and \(\Psi\) is the potential
normalized to go to zero at infinity. In practice the right, boundary
term is zero as \(\Psi \to 0\) as \(r\to\infty\), and if
\(\rho \propto r^{-n}\) at large \(r\) and \(\Psi \sim r^{-1}\) then
\(d\rho / d\Psi \sim r^{-n+1}\) which goes to zero provided that
\(n > 1\).

We consider both a Plummer and 2-dimensional exponential (Exp2D) stellar
profile:

\paragraph{Exp2D}\label{exp2d}

The 2D exponential stellar profile is given by \[
\Sigma_\star(R) = A e^{-R / R_s}
\] where \(A\) is some normalization and \(R_s\) is the exponential
scale radius. The 3D deprojected profile is found through abel inversion
(e.g.~Rapha\ldots), i.e. \[
\rho_\star (r) =- \frac{1}{\pi}\int_r^\infty \frac{d\Sigma}{dR} \frac{1}{\sqrt{R^2 - r^2}} dR  = \frac{\Sigma_0}{\pi R_s^2}\,K_0(r/R_s)
\] where \(K_0\) is the 0th order modified Bessel function of the second
type.

\paragraph{Plummer}\label{plummer}

A Plummer profile is defined by \[
\Sigma(R) = \frac{M}{4\pi R_s^2} \left(1 + \frac{R^2}{R_s^2}\right)^{-2} ,
\] where \(M\) is the total mass and \(R_s\) is the characteristic scale
radius. The density is \[
\rho(r) = \frac{3M}{4\pi\,R_s^3} \left(1 + \frac{r^2}{R_s^2}\right)^{-5/2}.
\]

\subsection{Dark Matter simulations}\label{dark-matter-simulations}

We adopt a softening length of \[
h_{\rm grav} = 0.014 \left(\frac{r_s}{2.76\,{\rm kpc}}\right)\left(\frac{N}{10^7}\right)^{-1/2}
\] see appendix for a discussion of this choice, which is similar to the
\citet{power+2003} suggested softening.

\subsubsection{Analysis}\label{analysis}

Shrinking spheres centres inspired by \citet{power+2003}

\begin{itemize}
\tightlist
\item
  Recursively shrink radius by 0.975 quantile and recalculating centroid
  until radius is less than \textasciitilde1kpc
\item
  Remove bound particles (using instantanious potential).
\item
  Use previous snapshot centre and acceleration to predict new centre
  for next snapshot.
\end{itemize}

\section{Appendix: Convergence tests and Simulation
Parameters}\label{appendix-convergence-tests-and-simulation-parameters}

Here, we describe some convergence tests to ensure our methods and
results are minimally impacted by numerical limitations and assumptions.
See \citet{power+2003} for a detailed discussion of various assumptions
and parameters used in N-body simulations.

\subsubsection{Softening}\label{softening}

\citet{power+2003} suggest the empirical rule that the ideal softening
(balancing integration time and only compromising resolution in
collisional regime) is \[
h_{grav} = 4 \frac{R_{200}}{\sqrt{N_{200}}}
\]

For our isolation halo (\(M_s=2.7\), \(r_s=2.76\)) and with \(10^7\)
particles, this works out to be \(0.044\,{\rm kpc}\).We adpoted the
slightly smaller softening which was reduced by a factor of
\(\sqrt{10}\) based on initial numerical experiments. However, this
choice is likely too small and a larger softening would minimally impact
our results but improve computation speed.

\subsubsection{Particle Number}\label{particle-number}

As discussed in \citet{power+2003}, the region where density is
converged is related to the region which becomes collisionally relaxed
over the time of the universe (so about 5-10 Gyr). (i.e.~where the
collisionless assumption breaks down). The relaxation timescale is given
by \(t_{\rm relax} = t_{\rm circ} N(r) / 8\ln N(r)\) or otherwise, \[
t_{\rm relax}(r) := t_{\rm circ}(r) \frac{N(r)}{8\,\ln N(r)}
= {t_{\rm circ}(r_{200})} \frac{\sqrt{200}}{8} \frac{N(r)}{\ln N(r)} \left(\frac{\bar \rho (r)}{\rho_{\textrm crit}}\right)^{-1/2}
\] This works out to be about 6-10 times (increasing with particle
number) our adopted softening length for NFW halos given our
assumptions. As such, at full resolution, we can only trust density
profiles down to \(10\epsilon\) or about
\(0.14\,{\rm kpc} (r_s/2.76\,{\rm kpc})\), just enough to resolve
stellar density profiles.

Note that by decreasing the truncation radius, the effective particle
number is improved, so future experiments could be less generous with
this parameter to improve computational performance (order 30\% better
resolution).

\begin{figure}
\centering
\includegraphics{figures/iso_converg_num.pdf}
\caption[num convergence]{Numerical convergence test for circular
velocity as a function of log radius for simulations with different
total numbers of particles in isolation. Residuals in lower panel are
relative to NFW. The initial conditions are dotted and the converged
radius is marked by arrows (REF).}
\end{figure}

\subsubsection{Timestepping and force
accuracy}\label{timestepping-and-force-accuracy}

In general, we use adaptive timestepping and relative opening criteria
for gravitational force computations. To verify that these choices and
associated accuracy parameters minimally impact convergence or speed, we
show a few more isolation runs (using only 1e5 particles)

\begin{itemize}
\tightlist
\item
  constant timestep (\ldots), approximantly half of minimum timestep
  with adaptive timestepping
\item
  geometric opening, with \(\theta = 0.5\).
\item
  strict integration accuracy, (facc = \ldots.)
\end{itemize}

\subsubsection{Alternative methods}\label{alternative-methods}

\begin{itemize}
\tightlist
\item
  FMM
\item
  PMM-tree
\item
  Gadget2
\item
  etc.
\end{itemize}

\subsubsection{Fiducial Parameters}\label{fiducial-parameters}

Note that we use code units which assume that \(G=1\) for convenience
and numerical stability. The conversion between code units to physical
units is (for our convention):

\begin{itemize}
\tightlist
\item
  1 length = 1 kpc
\item
  1 mass unit = \(10^{10}\) Msun
\item
  1 velocity unit = 207.4 km/s
\item
  1 time unit = 4.715 Myr
\end{itemize}

Most parameters below are not too relevant or have been discussed or are
merely dealing with cpu and IO details. The changes between simulation
runs primarily affect the integration time, output frequency, and
softening. Otherwise, we leave all other parameters fixed.

\begin{verbatim}
#======IO parameters======

#---Filenames
InitCondFile                initial
OutputDir                   ./out
SnapshotFileBase            snapshot
OutputListFilename          outputs.txt

#---File formats 
ICFormat                    3       # use HDF5
SnapFormat                  3 

#---Mem & CPU limits
TimeLimitCPU                86400
CpuTimeBetRestartFile       7200
MaxMemSize                  2400

#---Time
TimeBegin                   0
TimeMax                       2120      # 10 Gyr

#---Output frequency
OutputListOn                0
TimeBetSnapshot             10
TimeOfFirstSnapshot         0 
TimeBetStatistics           10
NumFilesPerSnapshot         1
MaxFilesWithConcurrentIO    1 


#=======Gravity======


#---Timestep accuracy
ErrTolIntAccuracy           0.01
CourantFac                  0.1     # ignored; for SPH
MaxSizeTimestep             0.5
MinSizeTimestep             0.0 

#---Tree algorithm
TypeOfOpeningCriterion      1       # Relative
ErrTolTheta                 0.5     # mostly used for Barnes-Hut
ErrTolThetaMax              1.0     # (used only for relative)
ErrTolForceAcc              0.005   # (used only for relative)

#---Domain decomposition: should only affect performance
TopNodeFactor                       3.0
ActivePartFracForNewDomainDecomp    0.02

#---Gravitational Softening
SofteningComovingClass0      0.044  # HALO dependent
SofteningMaxPhysClass0       0      # ignored; for cosmological
SofteningClassOfPartType0    0
SofteningClassOfPartType1    0


#=======Miscellanius=======

# probably do not need to change the options below 

#---Unit System
UnitLength_in_cm            1 
UnitMass_in_g               1
UnitVelocity_in_cm_per_s    1 
GravityConstantInternal     1

#---Cosmological Parameters 
ComovingIntegrationOn     0 # no cosmology
Omega0                    0
OmegaLambda                 0 
OmegaBaryon                 0
HubbleParam                 1
Hubble                      100
BoxSize                     0

#---SPH
ArtBulkViscConst             0.8
MinEgySpec                   0
InitGasTemp                  100

#---Initial density estimate (SPH)
DesNumNgb                   64
MaxNumNgbDeviation          1
\end{verbatim}
