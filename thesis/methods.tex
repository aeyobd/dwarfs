In the previous chapters, we have seen that the Scl and UMi classical
dSphs have outer density profiles that appear to deviate from the
exponential law that approximates well all other classical dSphs. Our
main intention is to assess whether such deviations result from the
effects of Galactic tides. To this purpose, we use N-body simulations of
the evolution of CDM halos in a Galactic potential, constrained to have
the orbital parameters consistent with a dwarf's present-day position
and velocity. We shall assume that, over the past 10 Gyr, the Galactic
potential is the static, analytic potential inferred by
\citet{mcmillan2011} from observations of kinematic tracers. We also
assume that the potential of each dwarf may be initially approximated by
a cuspy NFW profile. Since the dwarfs in question are heavily dark
matter dominated, we shall use a carefully selected sample of dark
matter particles to emulate the evolution of an embedded tracer stellar
component. In this Chapter, we describe our choice of Galactic
potential, orbital estimation, initial conditions setup, and N-body
methods.

\section{Orbital estimation}\label{sec:orbital_estimation}

To explore the possible orbits of a dwarf galaxy, we perform a Monte
Carlo sampling of the present-day observables. The present-day position,
distance modulus, LOS velocity, and proper motions are each sampled from
normal distributions given the reported uncertainties in
Tables~\ref{tbl:scl_obs_props}, \ref{tbl:umi_obs_props}. We integrate
each sampled position/velocity back in time for 10 Gyr using \agama{}
\citep{agama}. Dynamical friction is not expected to impact orbits
substantially because of the low masses and large pericentres of the
dwarfs, so we assume a single point-mass particle for the backwards
integration.

\subsection{Galactocentric frame}\label{galactocentric-frame}

To convert observed positions and velocities to Galactocentric
coordinates, we use the Astropy v4 Galactocentric frame
\citep{astropycollaboration+2022}. Our Cartesian Galactocentric
coordinates here assume the Galactic centre is at
\([x, y, z] = [0,0,0]\), where \(x\) is the direction from the sun to
the Galactic centre, \(y\) is the direction of the motion of the Local
Standard of Rest, and \(z\) is the direction perpendicular to the
Galactic plane. The coordinate frame is also right-handed, such that the
\(z\)-angular momentum of the sun is negative (since the sun is at
\(x<0\)). In this frame, the solar position is
\([-8.122 \pm 0.033, 0, 0.0208 \pm 0.003]\, \kpc\)
\citep{gravitycollaboration+2018, bennett+bovy2019} and the solar
velocity is \(\V_\odot = [-12.9 \pm 3.0, 245.6 \pm 1.4, 7.78 \pm 0.08]\)
km/s
\citep{reid+brunthaler2004, drimmel+poggio2018, gravitycollaboration+2018}.
The uncertainties in this reference frame are typically smaller than the
uncertainties on a dwarf galaxy's distance and tangental velocity.

\subsection{Milky Way Potential}\label{milky-way-potential}

We adopt the Milky Way potential described in \citet{EP2020}, which is
an analytic approximation to that proposed by \citet{mcmillan2011}.
Fig.~\ref{fig:v_circ_potential} plots the circular velocity profiles of
each component and the total circular velocity profile for this
potential. The potential includes a stellar bulge, a thin and thick
disk, and a dark matter NFW halo.

The Galactic bulge is described by a \citet{hernquist1990} potential,

\begin{equation}{
\Phi(r) = - \frac{GM}{r + a},
}\end{equation} where \(a=1.3\,{\rm kpc}\) is the scale radius and
\(M=2.1 \times 10^{10}\,\Mo\) is the total mass. The thin and thick
disks are represented with the \citet{miyamoto+nagai1975} cylindrical
potential, \begin{equation}{
\Phi(R, z) = \frac{-GM}{\left(R^2 + \left[a + \sqrt{z^2 + b^2}\right]^{2}\right)^{1/2}},
}\end{equation} where \(a\) is the disc radial scale length, \(b\) is
the scale height, and \(M\) is the total mass of the disk. For the thin
disk, \(a=3.944\,\)kpc, \(b=0.311\,\)kpc, and
\(M=3.944\times10^{10}\,\)M\(_\odot\). For the thick disk,
\(a=4.4\,\)kpc, \(b=0.92\,\)kpc, and \(M=2\times10^{10}\,\)M\(_\odot\).
The halo is an NFW dark matter halo (Eq.~\ref{eq:nfw}) with
\(\rmax = 43.7\,\)kpc and \(\vmax = 191\,\kms\), or
\(M_{200} = 126.6\times 10^{10}\,\Mo\) and \(r_s=20.2\,\kpc\).

\begin{figure}
\centering
\includegraphics[width=0.8\linewidth,height=\textheight,keepaspectratio]{figures/v_circ_potential.png}
\caption[Circular velocity of the Milky Way potential]{Circular velocity
profile of \citet{EP2020} potential. The total circular velocity (thick
black line) is composed of an NFW halo (green dashed line), thin and
thick \citet{miyamoto+nagai1975} disks (orange dash-dotted line), and a
\citet{hernquist1990} bulge (blue dotted
line).}\label{fig:v_circ_potential}
\end{figure}

\subsection{Sculptor's Orbit}\label{sec:scl_smallperi}

Sculptor's orbit in the assumed potential is relatively
well-constrained. Fig.~\ref{fig:scl_orbits} illustrates point particle
orbits for 100 samples of Sculptor's observed kinematics integrated
backwards for \(10\,\Gyr\) in both Galactocentric Cartesian coordinates
(\(x\), \(y\), \(z\)) and in Galactocentric radius with time. These
orbits sample the uncertainties in distance, proper motion, and radial
velocity, as given in Table~\ref{tbl:scl_obs_props}. All sampled orbits
of Sculptor have nearly the same morphology---the orbit primarily
resides in the \(y\)--\(z\) plane and completes a similar number of
periods and pericentric and apocentric passages.

To maximize tidal effects, we select an orbit with the \(\sim 3\sigma\)
smallest pericentre among all possible orbits integrated backwards for
\(10\,\Gyr\). We achieve this by taking the median parameters of all
orbits with a pericentre less than the 0.0027th quantile pericentre,
yielding a pericentre of 43 kpc. Given the current observations, it is
unlikely that Sculptor has had a significantly smaller pericentre than
our selected orbit, which we refer to as the \smallperi{} orbit. We take
the first apocentre after a look-back time of 10 Gyr, or at
\(\sim9.1\,\Gyr\), as the initial conditions for our model of Sculptor,
noted in Table~\ref{tbl:orbit_ics}.

\begin{figure}
\centering
\pandocbounded{\includegraphics[keepaspectratio]{/Users/daniel/thesis/figures/scl_xyzr_orbits.pdf}}
\caption[Sculptor's possible orbits]{The orbits of Sculptor in a static
Milky Way potential in Galactocentric \(x\), \(y\), and \(z\)
coordinates (top) and in Galactocentric radius \(r\) versus time
(bottom). The Milky Way is at the centre with the disk lying in the
\(x\)--\(y\) plane. Our selected \smallperi{} orbit is plotted in black
and light green transparent orbits represent the past 10Gyr orbits
sampled over Sculptor observables in Table~\ref{tbl:scl_obs_props}. The
orbit of Sculptor is well-constrained in this potential and it is
unlikely to achieve a smaller pericentre than the \smallperi{}
orbit.}\label{fig:scl_orbits}
\end{figure}

\begin{figure}
\centering
\pandocbounded{\includegraphics[keepaspectratio]{/Users/daniel/thesis/figures/umi_xyzr_orbits.pdf}}
\caption[Ursa Minor's possible orbits]{Similar to
Fig.~\ref{fig:scl_orbits}, the orbits of Ursa Minor in a static Milky
Way potential in Galactocentric \(x\), \(y\), and \(z\) coordinates. In
the lower panel, we show the radius versus time for only three orbits of
Ursa Minor, representing the \smallperi{} point-mass orbit (black), the
mean orbit, and the orbit with the \(3\sigma\)-largest
pericentre.}\label{fig:umi_orbits}
\end{figure}

\begin{table*}[t]
\centering
\caption[Orbit initial conditions]{The orbital initial conditions for models presented in this work. The observables represent the medians from orbital integration used to derived the orbits. Instead, the initial position and velocity represents the initialization of the actual N-body model. The \smallperi{} represents instead the $3\sigma$ smallest pericentre, which we use to provide an upper limit on tidal effects. We describe the \texttt{LMC-flyby} orbit in Section \ref{sec:scl_lmc}. }
\label{tbl:orbit_ics}
\begin{tabular}{llll}
\toprule
Property & Scl: \smallperi{} & Scl: \verb|LMC-flyby| & Umi: \smallperi{}\\
\midrule
distance / kpc & 82.6 & 72.9 & 64.6\\
$\pmra / \masyr$ & 0.134 & 0.137 & -0.158\\
$\pmdec / \masyr$ & -0.198 & -0.157 & 0.050\\
LOS velocity / $\kms$ & 111.2 & 111.2 & -245.75\\
$t_i / \Gyr$ & -9.17 & -2.0 & -9.67\\
${x}_{i} / \kpc$ & -2.49 & 4.30 & -17.40\\
${y}_{i} / \kpc$ & -42.78 & 138.89 & 74.51\\
${z}_{i} / \kpc$ & 86.10 & 125.88 & 21.34\\
$\V_{x\,i} / \kms$ & -20.56 & 6.89 & 14.27\\
$\V_{y\,i} / \kms$ & -114.83 & -56.83 & 48.62\\
$\V_{z\,i} / \kms$ & -57.29 & 52.09 & -114.08\\
\bottomrule
\end{tabular}
\end{table*}

\subsection{Ursa Minor's Orbit}\label{sec:orbit_corrections}

Similar to Sculptor, Ursa Minor has a well-constrained orbit in the
assumed MW potential. Fig.~\ref{fig:umi_orbits} shows 100 random
point-mass orbits of Ursa Minor. As for Sculptor, we select an orbit
with approximately the \(3\sigma\) smallest pericentre, the Ursa Minor
\(\smallperi{}\) orbit (see Fig.~\ref{fig:umi_orbits} and
Table~\ref{tbl:orbit_ics}).

We shall see later that the orbit of Ursa Minor's N-body model differs
from the point particle orbit because of the effects of tidal mass loss.
To ensure the final conditions of the N-body simulation are close to the
intended final position, we iteratively adjust Ursa Minor's initial
conditions. Initially, starting with low-resolution runs, we adjust the
cylindrical actions of the initial orbit by the final difference in
actions at the end of orbital evolution. After the initial actions have
converged (2 iterations), we change the initial action angles by the
final difference in action angles. This method converges within 4
iterations to an orbit agreeing with the observed kinematics of Ursa
Minor. Since Sculptor's orbit is less strongly affected by tides, we do
not carry out this correction for Sculptor.

\section{Initial conditions}\label{initial-conditions}

We use \agama{} \citep{agama} to generate the initial N-body dark matter
halo. We assume galaxies are described by a spherical, isotropic NFW
dark matter potential (Eq.~\ref{eq:nfw}). We also assume the stars do
not contribute to the potential. The dark matter density is truncated in
the outer regions by
\begin{equation}\protect\phantomsection\label{eq:trunc_nfw}{
\rho_{\rm tNFW}(r) = e^{-(r/r_t)^3}\ \rho_{\rm NFW}(r),
}\end{equation} where we adopt \(r_t = 20 r_s\).

\subsection{Initial dark matter halos for Sculptor and Ursa
Minor}\label{initial-dark-matter-halos-for-sculptor-and-ursa-minor}

The observed half-light radius and velocity dispersion of Sculptor and
Ursa Minor, together with the mass-concentration relation of \LCDM{}
halos, determines the properties of the N-body models adopted for each
dwarf.

Table~\ref{tbl:derived_props} lists the our inferred halo and kinematic
properties of Sculptor and Ursa Minor. First, taking the absolute
magnitudes from \citet{munoz+2018} with the mass-to-light ratio from
\citet{woo+courteau+dekel2008}, the total current stellar mass of
Sculptor and Ursa Minor are \(M_\star \approx 3.1 \times 10^6 \Mo\) and
\(M_\star \approx 7 \times 10^5 \Mo\), respectively. Based on the
stellar mass-\(\vmax\) relation \citep[from][]{fattahi+2018}, Sculptor
and Ursa Minor's halos should have \(\vmax \approx 31 \,\kms\) and
\(\vmax \approx 27\,\kms\). Finally, using the \citet{ludlow+2016}
\(z=0\) mass-concentration relation, this constraint translates into a
\(\rmax \approx 6 {\rm kpc}\) and \(\rmax \approx 5\,\kpc\) for each
galaxy.

The observed velocity dispersion constrains the total mass within the
stellar half-light radius. From the \citet{wolf+2010} mass estimator,
the mass contained within the 3D half-light radius \(r_h\) is
\begin{equation}\protect\phantomsection\label{eq:wolf_mass}{
M(r_h) \approx \frac{3}{G} \sigma_\V^2\,r_h
}\end{equation} for a velocity dispersion \(\sigma_\V\). For Scl and
UMi, the cosmological mean parameters above would predict
\(\sigma_{\V, i}  \approx 8.5\,\kms\) and \(8.0\,\kms\) assuming
\(r_h=240\,\)pc. This is below the observed values of \(9.7\,\kms\) and
\(8.7\,\kms\), and tidal evolution will further reduce the dispersion.
To match the observed \(\sigma_\V\) at the end of the simulation, we
choose \(\vmax = 31\,\kms\) and \(\rmax=3.2\,\kpc\) for Scl and
\(\vmax=38\,\kms\) and \(\rmax=4\,\kpc\) for UMi.

While there is some range in the choice of initial halo, reasonable
changes to the initial halo do not substantially affect the tidal
evolution for either galaxy. So, the tidal effects on stars should be
similar for halos with similar velocity dispersions.

\begin{table*}[t]
\centering
\caption[Derived Properties of Sculptor and Ursa Minor]{Inferred properties of the stellar component and halo for Sculptor and Ursa Minor. We record the total luminosity, stellar mass, mass-to-light ratio, dark matter halo $\vmax$ and $\rmax$, and dark matter halo virial mass $M_{200}$ and concentration $c_{\rm NFW}$. Uncertainties are the 16-84th percentile derived using Monte-Carlo sampling, assuming 0.035 and 0.1 dex uncertainties in the $\vmax(M_\star)$ and $c(M_{200})$ relations. }
\label{tbl:derived_props}
\begin{tabular}{lll}
\toprule
parameter & Sculptor & Ursa Minor\\
\midrule
$L_\star$ & $1.8\pm0.2\times10^6\ L_\odot$ & $3.5 \pm 0.1 \times 10^5\,L_\odot$\\
$M_\star$ & $3.1_{-1.0}^{+1.6} \times10^6\ {\rm M}_\odot$ & $7_{-2}^{+3} \times 10^5\,\Mo$\\
$M_\star / L_\star$ & $1.7\times 10^{\pm 0.17}$ & $1.9 \times 10^{\pm 0.17}$\\
$\vmax$ & $31\pm 3\,\kms$ & $27_{-2}^{+3}\,\kms$\\
$\rmax$ & $6 \pm 2$ kpc & $5_{-1}^{+2}$ kpc\\
$M_{200}$ & $0.5 \pm 0.2\times10^{10}\ M_0$ & $0.3_{-0.1}^{+0.2} \times 10^{10}\,\Mo$\\
$c_{\rm NFW}$ & $13_{-3}^{+4}$ & $13.5_{-3}^{+4}$\\
\bottomrule
\end{tabular}
\end{table*}

\begin{figure}
\centering
\pandocbounded{\includegraphics[keepaspectratio]{figures/initial_halo_selection.pdf}}
\caption[Initial halo selection]{Selection of initial halos for Sculptor
and Ursa Minor. The grey line and pink line with shaded regions
represent the \citet{ludlow+2016} mass-concentration relation and
\citet{fattahi+2018} SMHM relation respectively. The curved lines
represent the velocity dispersion of the initial halo given the
present-day half-light radius via
Eq.~\ref{eq:wolf_mass}.}\label{fig:scl_halos}
\end{figure}

\section{Numerical methods}\label{numerical-methods}

\subsection{\texorpdfstring{The N-body code:
\gadget{}}{The N-body code: }}\label{the-n-body-code}

To simulate the tidal evolution of galaxies, we use N-body simulations
integrated with the parallel, gravitational-tree program \gadget{}
\citep{gadget4}. The N-body method calculates and evolves the
gravitational accelerations between a large number of collisionless
particles to approximate the dynamical evolution of matter. To
approximate a collisionless system (i.e., without strong
particle-particle gravitational deflections), the force is tapered below
a ``softening length.'' Resolution is limited by the number of particles
and the softening length. Tree codes organize particles into a spatial
tree, enabling grouping of the gravitational forces from nearby
particles. A gravitational tree code substantially reduces the required
number of force computations, as compared to the exact, direct-summation
method.

\subsection{Isolation runs and simulation
parameters}\label{isolation-runs-and-simulation-parameters}

To ensure that the initial conditions of the simulation are dynamically
relaxed and well-converged, we run a halo first in isolation using
\gadget{}. Since gravity is scale-free, we use the same isolation run
for all halos and rescale the results to the desired values of size and
mass. We adopt a fiducial value of \(\rmax = 6.0\,\)kpc and
\(\vmax = 31\,\kms\) for the isolation halo based on Sculptor's mean
properties. We run this model for 5 Gyr (about one-half crossing time
\(t_{\rm cross} = 2\pi\,r /\vcirc  \approx 9\,\Gyr\) at
\(r_{200}=36\,\)kpc).

For our simulation parameters, we adopt a softening length of
\begin{equation}\protect\phantomsection\label{eq:softening_length}{
h_{\rm grav} = 0.014\,{\rm kpc}\left(\frac{r_{\rm max}}{6.0\,{\rm kpc}}\right)\left(\frac{N}{10^7}\right)^{-1/2},
}\end{equation} where the force is softened by a spline kernel
\citetext{\citealp[eqs. 70--71
in][]{springel+yoshida+white2001}; \citealp[but with the spline
characteristic radius \(2.8h_{\rm grav}\), see][]{gadget4}}. See
Appendix \ref{sec:extra_convergence} for a discussion of this choice and
our simulation parameters, which is similar to the \citet{power+2003}
suggested softening.

\subsection{Numerical fidelity}\label{numerical-fidelity}

Fig.~\ref{fig:numerical_convergence} illustrates how well our numerical
setup is able to reproduce the desired initial conditions, before and
after running the model in isolation. This figure shows that our
numerical methods are able to approximate well an NFW halo down to a
given innermost radius that strongly depends on resolution. The larger
the number of particles, the smaller the radius that is effectively
``resolved'' in a given simulation. For the Sculptor-like halo shown in
this figure (with \(\rmax = 6.0\,\)kpc and \(\vmax = 31\,\kms\)), a
simulation with \(10^7\) particles is needed to resolve the innermost
100 pc. For reference, the half-light radius of Sculptor is roughly 100
pc, which means that at least 10 million particles are needed to follow
faithfully its tidal evolution. Vertical arrows in
Fig.~\ref{fig:numerical_convergence} indicate the ``convergence radius''
defined by \citet[eq.\textasciitilde13]{power+2003} for NFW halos formed
in cosmological N-body simulations. This radius marks the region where
collisional effects driven by the finite number of particles used to
describe the innermost regions of a halo become important over a Hubble
time in a cosmological simulation. The softening length (from
Eq.~\ref{eq:softening_length}) is typically a few times smaller than the
converged length.

\begin{figure}
\centering
\pandocbounded{\includegraphics[keepaspectratio]{figures/iso_converg_num.pdf}}
\caption[Numerical convergence of the N-body simulation]{Numerical
convergence test for circular velocity as a function of log radius for
simulations with different total numbers of particles in isolation.
Residuals in bottom panel are relative to NFW. The initial conditions
are dotted, the converged radius is marked by arrows \citep[eq.
13,][]{power+2003}, and the softening length is marked by a vertical
bar. Note that a slight reduction in density starting around
\(r = 30\,\kpc\) is expected given our outer truncation
choice.}\label{fig:numerical_convergence}
\end{figure}

\subsection{Orbital evolution}\label{orbital-evolution}

Next, we evolve the halo in the Galactic potential. We scale the relaxed
snapshot and softening length to match the initial halo, and shift the
snapshot to the initial conditions inferred from the orbital analysis
(see Table~\ref{tbl:orbit_ics}). We then evolve the full N-body NFW
model forward in time in the Galactic potential until the present time,
when the halo is closest to its present-day observed position in the MW
halo.

\subsection{Halo centring}\label{sec:shrinking_spheres}

To accurately follow the evolution of a halo, it is important to
determine the centre of the self-bound halo remnant at each time chosen
for analysis. We use a shrinking-spheres centre method inspired by
\citet{power+2003}. First, we start with an initial centre estimate
using all bound particles from the last timestep. Then, we calculate the
radius of all particles from the centre, remove particles with a radius
beyond the 0.975 quantile of the centre, and recalculate the centre of
mass. The procedure is repeated until the selection radius is less than
\textasciitilde1kpc or fewer than 0.1\% of particles remain. After a
centre has been chosen, we remove all unbound particles based on the
\gadget{} calculated potential of the halo. For all future timesteps, we
consider only particles retained from the previous iteration.

The statistical centring uncertainty for the full resolution (\(10^7\)
particle) isolation run is of order 0.003 kpc, but fluctuations are
observed of \(\sim 0.03\,\kpc\). This is about three times the softening
length but is less than the numerically converged radial scale.

\subsection{Sculptor and Ursa Minor's initial stellar
components}\label{sec:painting_stars}

We ``paint'' stars onto dark matter particles using the particle-tagging
method \citep[e.g.,][]{bullock+johnston2005}, assuming spherical
symmetry. We initially assume stars follow a projected exponential law
(Eq.~\ref{eq:exponential_law}) with \(R_s = 0.10\,\kpc\) for both
galaxies. The tagging method assigns a probability to each dark matter
particle, which is proportional to the ``light-to-mass'' ratio required
to match the assumed stellar light profile. We briefly describe the
procedure next, but refer interested readers to \citet{EP2020}.

Let \(\Psi\) be the potential (normalized to vanish at infinity) and
\({\cal E}\) the binding energy, \({\cal E} = \Psi - 1/2 v^2\). If we
know the distribution function \(f({\cal E})\) (the phase-space density
as a function of energy), then we assign a stellar weight to a given
particle with energy \({\cal E}\) using \begin{equation}{
P_\star({\cal E}) = \frac{f_\star({\cal E})}{f_{\rm halo}({\cal E})}.
}\end{equation} While \(f({\cal E})\) is a phase-space density, the
differential energy distribution includes an additional \(g({\cal E})\)
occupation term \citep{BT1987}. We use Eddington inversion to find the
distribution function, \citep[eq. 4-140b in][]{BT1987} \begin{equation}{
f({\cal E}) = \frac{1}{\sqrt{8}\, \pi^2}\left( \int_0^{\cal E} \frac{d^2\rho}{d\Psi^2} \frac{1}{\sqrt{{\cal E} - \Psi}}\ d\Psi + \frac{1}{\sqrt{\cal E}} \left(\frac{d\rho}{d\Psi}\right)_{\Psi=0} \right).
}\end{equation} In practice the right, boundary term is zero.\footnote{If,
  at large \(r\), \(\rho \propto r^{-n}\) with \(n>1\), so
  \(\Psi \sim r^{-1}\) then \(d\rho / d\Psi \sim r^{-n+1}\).} We take
\(\Psi\) from the underlying assumed analytic dark matter potential.
\(\rho_\star\) can be calculated from the surface density,
\(\Sigma_\star\), via the inverse Abel transform.

Fig.~\ref{fig:scl_umi_initial_isolation} shows the initial circular
velocity profiles of stars and dark matter, demonstrating that these
galaxies are stable in isolation and the stellar component would
contribute negligibly to the total mass (\(M \sim \V_{\rm circ}^2 R\)).
Our cosmologically-motivated, observationally-based, and
numerically-converged initial conditions enable us to accurately
consider tidal effects in the next Chapter.

\begin{figure}
\centering
\pandocbounded{\includegraphics[keepaspectratio]{figures/initial_velocity.pdf}}
\caption[Initial halo velocity profiles]{The initial circular velocity
profiles of dark matter (blue) and stars (orange) for Sculptor and Ursa
Minor. The initial conditions are dotted and the isolation evolved
profiles are solid. The green cross marks the present day half-light
radius and velocity dispersion, and the black band represents the
1-sigma mean density of the MW at pericentre across orbits. Initial
conditions are stable in isolation and mass is dominated by dark
matter.}\label{fig:scl_umi_initial_isolation}
\end{figure}
