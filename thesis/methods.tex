In this section, we discuss our simulation methods from orbital
estimation and initial conditions to post-processing and software.

We use Agama \citep{agama} for potential specification and initial
conditions, and Gadget-4 \citep{gadget4} for orbit integration and dark
matter simulations (with a custom patch to integrate with Agama).

\section{Milky Way Potential}\label{milky-way-potential}

fig.~\ref{fig:v_circ_potential} plots the circular velocity profiles of
each component of our fiducially potential. We adopt the potential
described in \citet{EP2020}, an analytic approximation of
\citet{mcmillan2011}. The potential consisting of a stellar Bulge, a
thin and thick disk, and a dark matter NFW halo. Especially in the
regime where Sculptor and Ursa Minor orbit, the dark matter halo is the
dominant component of the potential. The details of the stellar disk and
bulge are less important.

The galactic bulge is described by a \citet{hernquist1990} potential.
The potential is

\begin{equation}
\Phi(r) = - \frac{GM}{r + a}
\end{equation} where \(a=1.3\,{\rm kpc}\) is the scale radius and
\(M=2.1 \times 10^{10}\,\Mo\) is the total mass.

The thin and thick disks are represented with the
\citet{miyamoto+nagai1975} cylindrical potential: \begin{equation}
\Phi(R, z) = \frac{-GM}{\left(R^2 + \left[a + \sqrt{z^2 + b^2}\right]^{2}\right)^{1/2}}
\end{equation}

where \(a\) is the disc radial scale length, \(b\) is the scale height,
and \(M\) is the total mass of the disk. For the thin disk,
\(a=3.944\,\)kpc, \(b=0.311\,\)kpc,
\(M=3.944\times10^{10}\,\)M\(_\odot\). For the thick disk,
\(a=4.4\,\)kpc, \(b=0.92\,\)kpc, and \(M=2\times10^{10}\,\)M\(_\odot\).

The halo is a NFW dark matter halo (REF) with
\(M_s=79.5\times10^{10}\,\)M\(_\odot\). and \(r_s = 20.2\,\)kpc.

Variations to the potential of the inner disk (exclusion of a bar)
should minimally affect our results as no orbit we consider reaches less
than \textasciitilde15 kpc of the MW centre. We exclude the mass
evolution of the halo from this analysis. Over \(10\,\)Gyr, this would
be fairly significant (factor of \(\sim 2\)in MW mass, REF) but since we
want to determine the upper limit of tidal effects, it is safe to
neglect this.

Finally, in chapter REF, we consider the influence of the large Milky
Cloud on the orbits and evolution of Scl. We adopt the
\citet{vasiliev2024} multipole approximation of an N-body simulation of
the LMC and MW. Their initial conditions are

\begin{itemize}
\tightlist
\item
  MW halo:
\item
  MW bulge (static):
\item
  MW disk (static):
\item
  LMC halo:
\end{itemize}

\begin{figure}
\centering
\pandocbounded{\includegraphics[keepaspectratio]{figures/v_circ_potential.pdf}}
\caption[Circular velocity of potential]{Circular velocity profile of
\citet{EP2020} potential. We also show the MW and LMC potential from
\citet{vasiliev2024} for their L3M11 model.}\label{fig:v_circ_potential}
\end{figure}

\section{Orbital Estimation}\label{orbital-estimation}

To estimate the orbits of a dwarf galaxy, we perform a Monte Carlo
sampling of the present-day observables. We integrate each sampled
observable back in time for 10 Gyr in Gadget as massless point particles
outputting positions every 5Myr (with otherwise similar parameters to
n-body runs below.)

To convert from Gaia to galactocentric coordinates, we use the
\citet{astropy} v4 galactocentric frame. This frame assumes the galactic
centre (Sagittarius A*) appears:

\begin{itemize}
\tightlist
\item
  J2000 `ra' = 17h45m37.224s, `dec' = -28°56'10.23'',
  \(\mu_{\alpha*}=-3.151\pm0.018\) mas/yr, \(\mu_\delta=-5.547\pm0.026\)
  from the appendix and Table 2 of \citet{reid+brunthaler2004}.
\item
  distance = \(8.122\pm0.033\,\)kpc, solar radial velocity =
  \(11 + 1.9 \pm 3\) km/s. from \citet{gravitycollaboration+2018}.
\end{itemize}

Finally, adding that the sun is \(20.8\pm0.3\,\)pc above the disk from
bennett+bovy2019, and using the procedure described in
\citet{drimmel+poggio2018}, the solar velocity relative to the galactic
rest frame is

\begin{itemize}
\tightlist
\item
  `v\_sun' = \([-12.9 \pm 3.0, 245.6 \pm 1.4, 7.78 \pm 0.08]\) km/s
\end{itemize}

We select the initial conditions from the location of the first
apocentre of a selected orbit.

While we assume that the galaxy can be described as a point particle,
not subject to (internal) dynamical friction, this approximation works
well if the galaxy is not strongly affected by tides (like Scl).
Analytic approximations of dynamical friction tend to be inadequate to
more accurately model the N-body trajectory.

\section{N-Body Modelling}\label{n-body-modelling}

\subsection{Initial conditions}\label{initial-conditions}

We use Agama \citep{agama} to generate initial conditions. We initially
assume galaxies are described by an NFW dark matter potential (REF) and
the stars are merely collisionless tracers embedded in this potential
(added on in post-processing). The density is exponentially truncated
with a profile
\begin{equation}\protect\phantomsection\label{eq:trunc_nfw}{
\rho_{\rm tNFW} = e^{-r/r_t}\ \rho_{\rm NFW}(t)
}\end{equation} where we adopt \(r_t = 100 r_s\) or approximately 10
times \(r_{200}\).

\subsection{Isolation runs and simulation
parameters}\label{isolation-runs-and-simulation-parameters}

To ensure that the initial conditionss of the simulation are dynamically
relaxed and well-converged, we run the simulation in isolation (no
external potential) for 5 Gyr (or approximately the crossing timescale).
Our fiducial isolation halo uses \(r_s=2.76\) kpc and
\(M_s = 0.29 \times 10^{10}\) Msun, but can be easily rescaled for any
length or mass scale.

We adopt a softening length of \begin{equation}
h_{\rm grav} = 0.014 \left(\frac{r_s}{2.76\,{\rm kpc}}\right)\left(\frac{N}{10^7}\right)^{-1/2}.
\end{equation} See appendix REF for a discussion of this choice, which
is similar to the \citet{power+2003} suggested softening.

We use the relative tree opening criterion with the accuracy parameter
set to 0.005, and adaptive time stepping with integration accuracy set
to 0.01.

\subsubsection{Numerical convergence}\label{numerical-convergence}

As discussed in \citet{power+2003}, the region where density is
converged is related to the region which becomes collisionally relaxed
over the time of the universe (so about 5-10 Gyr). (i.e.~where the
collisionless assumption breaks down). The relaxation timescale is given
by \(t_{\rm relax} = t_{\rm circ} N(r) / 8\ln N(r)\) or otherwise,
\begin{equation}
t_{\rm relax}(r) := t_{\rm circ}(r) \frac{N(r)}{8\,\ln N(r)}
= {t_{\rm circ}(r_{200})} \frac{\sqrt{200}}{8} \frac{N(r)}{\ln N(r)} \left(\frac{\bar \rho (r)}{\rho_{\textrm crit}}\right)^{-1/2}
\end{equation} This works out to be about 6-10 times (increasing with
particle number) our adopted softening length for NFW halos given our
assumptions. As such, at full resolution, we can only trust density
profiles down to \(\sim10\epsilon\), just enough to resolve stellar
density profiles.

Note that by decreasing the truncation radius, the effective particle
number is improved, so future experiments could be less generous with
this parameter to improve computational performance (order 30\% better
resolution).

\begin{figure}
\centering
\pandocbounded{\includegraphics[keepaspectratio]{figures/iso_converg_num.pdf}}
\caption[Numerical halo convergence]{Numerical convergence test for
circular velocity as a function of log radius for simulations with
different total numbers of particles in isolation. Residuals in lower
panel are relative to NFW. The initial conditions are dotted and the
converged radius is marked by arrows
(REF).}\label{fig:numerical_convergance}
\end{figure}

\subsection{Orbital runs}\label{orbital-runs}

To perform the simulations of a given galaxy in a given potential, we
centre the isolation run's final snapshot and place the dwarf galaxy in
the specified orbit in the given potential. We typically run the
simulation for 10 Gyr, which allows us to orbit slightly past the
expected initial conditions.

\section{Post Processing}\label{post-processing}

\subsection{Centring}\label{centring}

Shrinking spheres centres inspired by \citet{power+2003}

\begin{itemize}
\tightlist
\item
  Recursively shrink radius by 0.975 quantile and recalculating centroid
  until radius is less than \textasciitilde1kpc or fewer than 0.1\% of
  particles remain.
\item
  Remove bound particles (using instantanious potential).
\item
  Use previous snapshot centre and acceleration to predict new centre
  for next snapshot and only include particles included in previous
  centring timestep.
\end{itemize}

The statistical centring uncertainty for the 1e7 particle isolation run
is of order 0.003 kpc, however oscillations in the centre are of order
0.03 kpc. This is about three times the softening length but is less
than the numerically converged radius scale.

\subsection{Velocity profiles}\label{velocity-profiles}

\begin{itemize}
\tightlist
\item
  Circular velocity is computed assuming spherical symmetry and only
  shown for every 200th particle (ranked from the centre outwards)
\end{itemize}

\begin{equation}
v_{\rm circ}(r) = \sqrt{\frac{GM(r)}{r}}
\end{equation}

\begin{itemize}
\tightlist
\item
  \(v_{\rm circ\ max}\) and \(r_{\rm circ,\ max}\) is found by
  least-squares fitting the NFW functional velocity form to the points
  of the velocity profile that have the 10\% highest velocities. This is
  not a necessarily good fit, especially as the halo becomes stripped,
  but accurate enough to find a reliable maximum.
\end{itemize}

\subsection{Stellar Probabilities}\label{stellar-probabilities}

We assign stellar probabilities (assuming spherical, isotropic symmetry)
via Eddington inversion.

Let \(\Psi\) be the potential (normalized to vanish at infinity) and
\({\cal E}\) is the binding energy \({\cal E} = \Psi - 1/2 v^2\). If we
know \(f({\cal E})\), the distribution function (phase-space density in
energy), then we assign the stellar weight for a given particle with
energy \({\cal E}\) is \begin{equation}
P_\star({\cal E}) = \frac{f_\star({\cal E})}{f_{\rm halo}({\cal E})}.
\end{equation} While \(f({\cal E})\) is a phase-space density, the
differential energy distribution includes an additional \(g({\cal E})\)
occupation term (BTXXX).

We use Eddington inversion to find the distribution function, (eq.
4-140b in BT87) \begin{equation}
f({\cal E}) = \frac{1}{\sqrt{8}\, \pi^2}\left( \int_0^{\cal E} \frac{d^2\rho}{d\Psi^2} \frac{1}{\sqrt{{\cal E} - \Psi}}\ d\Psi + \frac{1}{\sqrt{\cal E}} \left(\frac{d\rho}{d\Psi}\right)_{\Psi=0} \right).
\end{equation}

In practice the right, boundary term is zero as \(\Psi \to 0\) as
\(r\to\infty\), and if \(\rho \propto r^{-n}\) at large \(r\) and
\(\Psi \sim r^{-1}\) then \(d\rho / d\Psi \sim r^{-n+1}\) which goes to
zero provided that \(n > 1\).

\(\Psi\) is taken as known from the underlying assumed analytic dark
matter potential. \(\rho_\star\) is pre-specified or calculated via Abel
integration / deprojection

We consider both a Plummer and 2-dimensional exponential (Exp2D) stellar
models:

\subsubsection{Exp2D}\label{exp2d}

The 2D exponential stellar profile is given by \begin{equation}
\Sigma_\star(R) = A e^{-R / R_s}
\end{equation} where \(A\) is some normalization and \(R_s\) is the
exponential scale radius. The 3D deprojected profile is found through
abel inversion (e.g.~Rapha\ldots), i.e. \begin{equation}
\rho_\star (r) =- \frac{1}{\pi}\int_r^\infty \frac{d\Sigma}{dR} \frac{1}{\sqrt{R^2 - r^2}} dR  = \frac{\Sigma_0}{\pi R_s^2}\,K_0(r/R_s)
\end{equation} where \(K_0\) is the 0th order modified Bessel function
of the second type.

\subsubsection{Plummer}\label{plummer}

A Plummer profile is defined by \begin{equation}
\Sigma(R) = \frac{M}{4\pi R_s^2} \left(1 + \frac{R^2}{R_s^2}\right)^{-2} ,
\end{equation} where \(M\) is the total mass and \(R_s\) is the
characteristic scale radius. The density is \begin{equation}
\rho(r) = \frac{3M}{4\pi\,R_s^3} \left(1 + \frac{r^2}{R_s^2}\right)^{-5/2}.
\end{equation}

\subsection{Stellar density profiles and velocity
dispersion}\label{stellar-density-profiles-and-velocity-dispersion}

\begin{itemize}
\tightlist
\item
  Stellar velocity dispersion is calculated for all stars within 1kpc
  (3D) of the centre by assuming
  \(\sigma_{\rm los} = \sqrt{\sigma_x^2 + \sigma_y^2 + \sigma_z^2}/ \sqrt 3\),
  i.e.~isotropic velocity dispersion
\item
  We calculate density profiles similar to stars and assume a constant
  bin width in \(\log R\) of \(2 {\rm IQR} / \sqrt[3]{N}\)
  (Freedman-Diaconis prescription). \(R\) may be either the cylendrical
  radius in \(x-y\) or the on-sky project \(\xi, \eta\) tangent plane
  coordinates.
\item
  Because our dwarfs are assumed to be spherical/isotropic, we retain
  this assumption when calculating predicted density profiles.
\end{itemize}
