In the previous chapters, we have seen that the Scl and UMi classical
dSphs have outer density profiles that appear to deviate from the
exponential law that approximates well all other classical dSphs. Our
main intension is to assess whether such excesses result from the
effects of Galactic tides. To this purpose, we intend to use N-body
simulations of the evolution of CDM halos in a Galactic potential,
constrained to have the orbital parameters consistent with a dwarf's
present-day position and velocity. We shall assume that the Galactic
potential is the static, analytic potential inferred by
\citet{mcmillan+2011} from observations of kinematic tracers. We also
assume the potential of each dwarf may be initially approximated by a
cuspy NFW profile. Since the dwarfs in question are heavily dark matter
dominated, we shall use a carefully selected sample of dark matter
particles to mimic and track the evolution of an embedded tracer stellar
component. In this Chapter, I describe the Galactic potential used, the
orbital estimation method, the initial conditions setup, and the N-body
method used.

\section{Orbital estimation}\label{orbital-estimation}

To choose the possible orbits of a dwarf galaxy, we perform a Monte
Carlo sampling of the present-day observables. The present-day position,
distance modulus, LOS velocity, and proper motions are each sampled from
normal distributions given the reported uncertainties in
Tables~\ref{tbl:scl_obs_props}, \ref{tbl:umi_obs_props}. We integrate
each sampled observable back in time for 10 Gyr using \agama [@agama].
Dynamical is not expected to impact orbits substantially because of the
low masses of the dwarfs, and because the dwarfs orbit mainly in the
periphery of the Galaxy. When selecting the initial position and
velocity of an N-body model, we select the position and velocity of the
first apocentre occurring 10 Gyr ago.

To convert from Gaia to Galactocentric coordinates, we use the Astropy
v4 Galactocentric frame \citep{astropycollaboration+2022}. This frame
assumes the Galactic centre is at position
\(\alpha = {\rm 17h\,45m\,37.224s}\),
\(\delta = -28^\circ\,56'\,10.23''\) with proper motions
\(\mu_{\alpha*}=-3.151\pm0.018\ \masyr\) ,
\(\mu_\delta=-5.547\pm0.026 \masyr\) (from the appendix and Table 2 of
\citet{reid+brunthaler2004}). The Galactic centre distance from the Sun
is \(8.122\pm0.033\,\)kpc with a radial velocity =
\(11 + 1.9 \pm 3\,\kms\), from \citet{gravitycollaboration+2018}.
Finally, adding that the Sun is \(20.8\pm0.3\,\)pc above the disk from
\citet{bennett+bovy2019}, and using the procedure outlined in
\citet{drimmel+poggio2018}, the Solar velocity relative to the Galactic
rest frame is
\(\v_\odot = [-12.9 \pm 3.0, 245.6 \pm 1.4, 7.78 \pm 0.08]\) km/s. The
uncertainties in the reference frame are typically smaller than the
uncertainties on a dwarf galaxy's position and velocity.

\subsection{Milky Way Potential}\label{milky-way-potential}

We adopt the Milky Way potential described in \citet{EP2020}, which is
an analytic approximation to that proposed by \citet{mcmillan2011}.
Fig.~\ref{fig:v_circ_potential} plots the circular velocity profiles of
each component and the total circular velocity profile for our fiducial
profile. The potential includes a stellar bulge, a thin and thick disk,
and a dark matter NFW halo.

The galactic bulge is described by a \citet{hernquist1990} potential,

\begin{equation}{
\Phi(r) = - \frac{GM}{r + a},
}\end{equation} where \(a=1.3\,{\rm kpc}\) is the scale radius and
\(M=2.1 \times 10^{10}\,\Mo\) is the total mass. The thin and thick
disks are represented with the \citet{miyamoto+nagai1975} cylindrical
potential:

\begin{equation}{
\Phi(R, z) = \frac{-GM}{\left(R^2 + \left[a + \sqrt{z^2 + b^2}\right]^{2}\right)^{1/2}}
}\end{equation}

where \(a\) is the disc radial scale length, \(b\) is the scale height,
and \(M\) is the total mass of the disk. For the thin disk,
\(a=3.944\,\)kpc, \(b=0.311\,\)kpc,
\(M=3.944\times10^{10}\,\)M\(_\odot\). For the thick disk,
\(a=4.4\,\)kpc, \(b=0.92\,\)kpc, and \(M=2\times10^{10}\,\)M\(_\odot\).
The halo is a NFW dark matter halo (Eq.~\ref{eq:nfw}) with
\(r_{\rm max} = 43.7\,\)kpc and \(\V_{\rm max} = 191\,\kms\).

\begin{figure}
\centering
\pandocbounded{\includegraphics[keepaspectratio]{figures/v_circ_potential.pdf}}
\caption[Circular velocity of potential]{Circular velocity profile of
\citet{EP2020} potential.}\label{fig:v_circ_potential}
\end{figure}

\subsection{Orbits of Sculptor}\label{orbits-of-sculptor}

Next, we calculate the range of possible orbits of Sculptor in our
fiducial Milky Way potential. Fig.~\ref{fig:scl_orbits} illustrates
point particle orbits for 100 samples of Sculptor's observed kinematics
integrated backwards 5 Gyr in both galactocentric coordinate slices (x,
y, z) and in galactocentric radius with time. In terms of the x, y, z
planes, note that all sampled orbits of Sculptor have nearly the same
morphology---they each orbit mostly in the y-z plane, passing through
the same number of pericentres and maintaining similar orbits across
5Gyr. Sculptor's orbital history is relatively well-constrained.

We select an orbit with the \(\sim 3\sigma\) smallest pericentre among
all possible orbits. We achieve this by taking the median parameters of
all orbits with a pericentre less than the 0.0027th quantile pericentre,
yielding a pericentre of 43 kpc.\footnote{only slightly smaller than the
  0.00135th quantile pericentre} It is unlikely given the current
observations that Sculptor has a smaller pericentre. Since tidal effects
depend most strongly on the pericentre of an orbit, this orbit should
maximize the tidal force from the Milky Way.

We take the first apocentre after a lookback time of 10 Gyr, or at 9.43
Gyr, as the initial conditions for our model of Sculptor.
Table~\ref{tbl:scl_orbits} notes the initial conditions for our
simulation.

\begin{figure}
\centering
\pandocbounded{\includegraphics[keepaspectratio]{/Users/daniel/thesis/figures/scl_xyzr_orbits.pdf}}
\caption[Sculptor Orbits]{The orbits of Sculptor in a static Milky Way
potential in galactocentric \(x\), \(y\), and \(z\) coordinates. The
Milky Way is at the centre with the disk lying in the \(x\)--\(y\)
plane. Our selected \texttt{smallperi} orbit is plotted in black and
light blue transparent orbits represent the past 5Gyr orbits sampled
over Sculptor observables in Table~\ref{tbl:scl_obs_props}. The orbit of
sculptor is well-constrained in this potential and it is unlikely to
achieve a smaller pericentre than our selected
orbit.}\label{fig:scl_orbits}
\end{figure}

\begin{table*}[t]
\centering
\caption[Sculptor Selected Orbits]{Properties of selected orbits for Sculptor. The mean orbit represents the observational mean from Table \ref{tbl:scl_obs_props}. The Smallperi represents instead the $3\sigma$ smallest pericentre, which we use to provide an upper limit on tidal effects. }
\label{tbl:scl_orbits}
\begin{tabular}{llll}
\toprule
Property & Mean & SmallPeri & LMC\\
\midrule
distance & 83.2 & 82.6 & \\
pmra & 0.099 & 0.134 & \\
pmdec & -0.160 & -0.198 & \\
Vlos & 111.2 & 111.2 & \\
$t_i$ & -8.74 & -9.43 & \\
$\hat{x}_{i}$ & [16.13, 92.47, 39.63] & [-2.49, -42.78, 86.10] & \\
$\vec{v}_i$ & [-2.37, -54.70, 128.96] & [-20.56, -114.83, -57.29] & \\
pericentre & 53 & 43 & \\
apocentre & 102 & 96 & \\
$t_{\rm last\ peri}$ & -0.45 & -0.46 & \\
Numer of peris &  &  & \\
\bottomrule
\end{tabular}
\end{table*}

\subsection{Orbits of Ursa Minor}\label{orbits-of-ursa-minor}

\begin{figure}
\centering
\pandocbounded{\includegraphics[keepaspectratio]{/Users/daniel/thesis/figures/umi_xyzr_orbits.pdf}}
\caption{Ursa Minor Orbits}
\end{figure}

\subsection{}\label{section}

ra = 227.242 dec = 67.2221 distance = 64.6 pmra = -0.158 pmdec = 0.05
radial\_velocity = -245.75

t\_i = -9.53 pericentre = 29.64 apocentre = 74.88 t last peri = -0.80
x\_i = {[}-16.48 69.92 21.05{]} v\_i = {[}16.32 39.86 -116.99{]}

\section{Initial conditions}\label{initial-conditions}

We use \agama [@agama] to generate initial conditions. We initially
assume galaxies are described by an NFW dark matter potential
Eq.~\ref{eq:nfw} and the stars are merely collisionless tracers embedded
in this potential (added on in post-processing). The density is a
cubic-exponentially truncated with a profile \begin{equation}{
\rho_{\rm tNFW} = e^{-(r/r_t)^3}\ \rho_{\rm NFW}(t)
}\end{equation} where we adopt \(r_t = 20 r_s\) or approximately
\(r_{200}\) for our Sculptor-like fiducial halo. Using \(r_{200}\) for
\(r_t\) would depend on the chosen scale of the halo. So our adopted
\(r_t\) is an approximate upper limit of \(r_{200}\) for typical dwarf
galaxy halos \citep{ludlow+2016}. It is unlikely that the outer density
profile of loosely bound particles past \(20\) kpc affects the tidal
evolution of a subhalo.

\subsection{Halos for Sculptor}\label{halos-for-sculptor}

From the observed properties of Sculptor, we can infer reasonable
\(\Lambda\)CMD dark matter halo hosts. Table~\ref{tbl:scl_derived_props}
reports our inferred halo and kinematic properties of Sculptor. First,
taking the absolute magnitude from \citet{munoz+2018} with the
mass-to-light ratio from \citet{woo+courteau+dekel2008} (1.7 with
\(\sim\) 0.17 dex uncertainty), the total current stellar mass of
Sculptor is \(M_\star \sim 3.1 \times 10^6 \Mo\). Based on the stellar
mass-\(v_{\rm max}\) relation \citep[from][]{fattahi+2018}, see also
\ref{fig:smhm}, the halo should have
\(v_{\rm max} \sim 31 \pm 3 \,\kms\). Finally, using the
\citet{ludlow+2016} \(z=0\) mass-concentration relation, this constraint
translates into a \(r_{\rm max} \sim 6 \pm 2\,{\rm kpc}\), or
alternatively \(M_{200} = 5\times10^9\) with \(c=13\).

Fig.~\ref{fig:scl_halos} illustrates these estimates visually. The
stellar-mass \(v_{\rm max}\) constraint translates to an estimate of
\(v_{\rm max}\) only (horizontal band). The mass-concentration relation
describes the relationship between \(v_{\rm max}\) and \(r_{\rm max}\),
an approximately diagonal linear band. And finally, we include a curved
line which illustrates where the initial velocity dispersion is
approximantly larger than the present-day observed dispersion
\(v_{\rm circ}(R_h) / \sqrt{3} \sim \sigma_v \gtrsim 9\,\kms\).

The intersection of the constraints on the halo leads us to select an
initial halo, \texttt{compact}, which is more compact than the mean but
with a observationally consistent initial velocity dispersion. We also
have the \texttt{light} halo which occupies the intersection between
3\(\sigma\) lowest mass and the velocity dispersion constraints, which
we come back to in Section~\ref{sec:scl_lmc}.

While there is some range in the choice of initial halo, reasonable
changes to the initial halo do not substantially affect the tidal
evolution for Sculptor. The velocity dispersion, in particular,
translates to an estimate of the mass within \(r_h\). As a result, halos
with the same velocity dispersion only differ in total mass, but the
density slope and circular velocity near the stellar component is
similar. These halos will have a similarly affected stellar component.

\begin{table*}[t]
\centering
\caption[Derived Properties of Sculptor]{To derive total mass, we use the absolute magnitude from Muñoz et al. (2018) (see also Table \ref{tbl:scl_obs_props}) along with the stellar mass to light ratio (1.7 with 0.17 dex uncertainty) from Woo, Courteau, and Dekel (2008). Sculptor’s total stellar mass is then $M_\star \sim 3.1_{-1.0}^{1.6} \times 10^6\,\Mo$. From the stellar mass to dark matter halo’s characteristic velocity $v_{\rm circ}$ in Fattahi et al. (2018), we expect Sculptor to have $v_{\rm circ} \approx 31 \pm 3 \kms$. }
\label{tbl:scl_derived_props}
\begin{tabular}{lll}
\toprule
parameter & value & reference\\
\midrule
$L_\star$ & $1.8\pm0.2\times10^6\ L_\odot$ & \\
$M_\star$ & $3.1_{-1.0}^{+1.6} \times10^6\ {\rm M}_\odot$ & \\
$M_\star / L_\star$ & $1.7\times 10^{\pm 0.17}$ & Woo, Courteau, and Dekel (2008)\\
$v_{\rm circ, max}$ & $31\pm 3\,\kms$ & relation from Fattahi et al. (2018)\\
$r_{\rm circ, max}$ & $6 \pm 2$ kpc & $v_{\rm circ}$ with Ludlow et al. (2016) mass-concentration relation\\
$M_{200}$ & $0.5 \pm 0.2\times10^{10}\ M_0$ & \\
$c_{\rm NFW}$ & $13.1_{-2.8}^{+3.6}$ & \\
$r_{\rm break, obs}$ & $25 \pm 5$ arcmin & (sestito+2024a?)\\
$t_{\rm break, obs}$ & $110\pm30$ Myr & $\sigma_v$, $r_{\rm break}$ with @\\
\bottomrule
\end{tabular}
\end{table*}

\begin{figure}
\centering
\pandocbounded{\includegraphics[keepaspectratio]{/Users/daniel/thesis/figures/scl_initial_halos.pdf}}
\caption[Sculptor initial halos]{The suggested halos of Sculptor
compared to cosmological predictions. \textbf{TODO: simplify
\(\sigma_v\) calculation}.}\label{fig:scl_halos}
\end{figure}

\begin{table*}[t]
\centering
\caption[Initial halos of Sculptor]{The parameters for our initial Sculptor halos. }
\label{tbl:scl_ini_halos}
\begin{tabular}{lllll}
\toprule
Halo name & Rmax & Vmax & M200 & c\\
\midrule
compact & 3.2 & 31 & 0.33 & 21\\
lmc & 4.2 & 31 & 0.39 & 17\\
small & 2.5 & 25 &  & \\
\bottomrule
\end{tabular}
\end{table*}

\subsection{Halos for Ursa Minor}\label{halos-for-ursa-minor}

\begin{table*}[t]
\centering
\caption[Derived Properties of Ursa Minor]{Derived properties of Ursa Minor. }
\label{tbl:umi_derived_props}
\begin{tabular}{lll}
\toprule
parameter & value & reference\\
\midrule
$L_\star$ & $3.5 \pm 0.1 \times 10^5\,L_\odot$ & \\
$M_\star$ & $7_{-2}^{+3} \times 10^5\,\Mo$ & \\
$M_\star / L_\star$ & 1.9 (pm 0.17 dex) & Woo, Courteau, and Dekel (2008)\\
$M_{200}$ & $3_{-2}^{+4} \times 10^9\,\Mo$ & \\
$c$ & 14? & \\
$v_{\rm circ, max}$ & $27_{-6}^{+7}\,\kms$ & \\
$r_{\rm circ, max}$ & $5_{-2}^{+1}$ kpc & \\
$r_{\rm break, obs}$ & $30 \pm 5$ arcmin & \\
$t_{\rm break, obs}$ & $120\pm30$ Myr & \\
\bottomrule
\end{tabular}
\end{table*}

\begin{table*}[t]
\centering
\caption[Ursa Minor Initial Halos]{Initial halos for Ursa Minor. }
\label{tbl:umi_ini_halos}
\begin{tabular}{lllll}
\toprule
Halo name & Rmax & Vmax & M200 & c\\
\midrule
fiducial & 5 & 37 & 0.67 & 17\\
compact & 4 & 38 & 0.62 & 21\\
\bottomrule
\end{tabular}
\end{table*}

\begin{figure}
\centering
\pandocbounded{\includegraphics[keepaspectratio]{/Users/daniel/thesis/figures/umi_initial_halos.pdf}}
\caption[Ursa Minor initial halos]{The suggested halos of Ursa Minor
compared to cosmological predictions.}
\end{figure}

\subsection{}\label{section-1}

\section{N-Body modelling}\label{n-body-modelling}

Modelling gravitational evolution for large systems requires special
methods. Perhaps the simplest method to compute the evolution of dark
matter is through \emph{N-body simulations}. A dark matter halo is
represented as a large number of dark matter particles (bodies). Each
body is essentially a Monte Carlo sample of the underlying phase-space
distribution. Note that dark matter (and galaxies) are often assumed to
be \emph{collisionless}---particles are not strongly affected by close,
\emph{collisional} gravitational encounters which substantially change
the momenta of involved bodies. In contrast, star clusters are often
collisional so neglecting these encounters may not be a reasonable
approximation. While we use individual gravitating bodies in N-body
simulations, the Newtonian gravitational force is softened to be a
Plummer sphere as to limit strongly collisional encounters.

Naively, the Newtonian gravitational force requires adding together the
forces from each particle on each particle, with a computational cost
that scales quadratically with the number of particles, or \(O(N^2)\).
With this method, simulating a large number of particles, such as
\(10^6\), would require \(10^{12}\) force evaluations at each time step,
making cosmological and high-resolution studies unfeasible. However,
only long-range gravitational interactions tend to be important for CDM,
so we can utilize the \emph{tree method} to compute the gravitational
force vastly more efficiently.

The first gravitational tree code was introduced in
\citet{barnes+hut1986}, and is still in use today. We utilize the
massively parallel code \emph{Gadget 4} \citep{gadget4}. Particles are
spatially split into an \emph{octotree}. The tree construction stars
with one large node, a box containing all of the particles. If there is
more than one particle in a box/node , the box is then divided into 8
more nodes (halving the side length in each dimension) and this step is
repeated until each node only contains 1 particle. With this
heirarchichal organization, if a particle is sufficiently far away from
a node, then the force is well approximated by the force from the centre
of mass of the node. As such, each force calculation only requires a
walk through the tree, only descending farther into the tree as
necessary to retain accuracy. The total force calculations reduce from
\(O(N^2)\) to \(O(N\,\log N)\), representing orders of magnitude
speedup. Modern codes such as \emph{Gadget} utilize other performance
tricks, such as splitting particles across many supercomputer nodes,
efficient memory storage, adaptive time stepping, and parallel file
writing to retain fast performance for large scale simulations, forming
the foundation for many cosmological simulation codes.

\textbf{could be 1 paragraph}

\subsection{Isolation runs and simulation
parameters}\label{isolation-runs-and-simulation-parameters}

To ensure that the initial conditionss of the simulation are dynamically
relaxed and well-converged, we run the simulation in isolation (no
external potential) for 5 Gyr (or about 3 times the crossing timescale
at the virial radius). Our fiducial isolation halo uses \(r_s=2.76\) kpc
and \(M_s = 0.29 \times 10^{10}\,\Mo\) , but can be easily rescaled for
any length or mass scale.

For our simulation parameters, we adopt a softening length of
\begin{equation}{
h_{\rm grav} = 0.014 \left(\frac{r_s}{2.76\,{\rm kpc}}\right)\left(\frac{N}{10^7}\right)^{-1/2}.
}\end{equation} See Appendix Section~\ref{sec:extra_convergence} for a
discussion of this choice, which is similar to the \citet{power+2003}
suggested softening. We use the relative tree opening criterion with the
accuracy parameter set to 0.005, and adaptive time stepping with
integration accuracy set to 0.01.

\subsection{Numerical fidelity}\label{numerical-fidelity}

Fig.~\ref{fig:numerical_convergence} illustrates how well our numerical
setup is able to reproduce the desired initial conditions, before and
after running the model in isolation. Circular velocity is computed
assuming spherical symmetry and only shown for every 200th particle
(ranked from the centre outwards). With increasing particle number, the
circular velocity profile maintains close agreement with the expected
NFW velocity profile until \(r_{\rm relax}\), as marked by arrows. The
reduction of mass (and consequently circular velocity) interior to
\(r_{\rm relax}\) is likely due to collisional effects which would
continue to reduce with higher resolution. Typically,
\(r_{\rm relax}(10\Gyr)\) is about 6-10 times our adopted softening
length, increasing with particle number. As such, at full resolution, we
can only trust density profiles down to \(\sim10\) times the softening
length, sufficient to resolve stellar density profiles.

\begin{figure}
\centering
\pandocbounded{\includegraphics[keepaspectratio]{figures/iso_converg_num.pdf}}
\caption[Numerical halo convergence]{Numerical convergence test for
circular velocity as a function of log radius for simulations with
different total numbers of particles in isolation. Residuals in bottom
panel are relative to NFW. The initial conditions are dotted and the
converged radius is marked by arrows (Eq.~\ref{eq:t_relax}). Note that a
slight reduction in density starting around \$r = 30 \$kpc is expected
given our truncation choice. \textbf{Add
softening}}\label{fig:numerical_convergence}
\end{figure}

\subsection{Orbital evolution}\label{orbital-evolution}

To perform the simulations of a given galaxy in a given potential, we
centre the isolation run's final snapshot
(Section~\ref{sec:shrinking_spheres}) and place the dwarf galaxy in the
specified orbit in the given potential. We typically run the simulation
for 10 Gyr, which allows us to orbit slightly past the expected initial
conditions.

\subsection{Centring}\label{sec:shrinking_spheres}

Accurately determining the centre of the subhalo at each timestep is
essential to most analysis. We use a shrinking-spheres centre method
inspired by \citet{power+2003}. First, we start with an initial centre
estimate from the last timestep. Then, we calculate the radius of all
particles from the centre, remove particles with a radius beyond the
0.975 quantile of the centre, and recalculating centroid until radius is
less than \textasciitilde1kpc or fewer than 0.1\% of particles remain.
Finally, we remove all unbound particles based on the instantaneous
N-body potential with all particles. For all future timesteps, we use
the snapshot centre's position, velocity, and acceleration to predict
the location of the next centre. We also consider only particles
retained from the previous iteration.

The statistical centring uncertainty for the full resolution (\(10^7\)
particle) isolation run is of order 0.003 kpc, but oscillations in the
centre are of order 0.03 kpc. This is about three times the softening
length but is less than the numerically converged radius scale.

\subsection{Stellar probabilities}\label{stellar-probabilities}

We ``paint'' stars onto dark matter particles using the particle tagging
method \citep[e.g.][]{bullock+johnston2005}, assuming spherical
symmetry. Let \(\Psi\) be the potential (normalized to vanish at
infinity) and \({\cal E}\) is the binding energy
\({\cal E} = \Psi - 1/2 v^2\). If we know \(f({\cal E})\), the
distribution function (phase-space density in energy), then we assign
the stellar weight for a given particle with energy \({\cal E}\) is

\begin{equation}{
P_\star({\cal E}) = \frac{f_\star({\cal E})}{f_{\rm halo}({\cal E})}.
}\end{equation} While \(f({\cal E})\) is a phase-space density, the
differential energy distribution includes an additional \(g({\cal E})\)
occupation term (BT87). We use Eddington inversion to find the
distribution function, (eq. 4-140b in BT87)

\begin{equation}{
f({\cal E}) = \frac{1}{\sqrt{8}\, \pi^2}\left( \int_0^{\cal E} \frac{d^2\rho}{d\Psi^2} \frac{1}{\sqrt{{\cal E} - \Psi}}\ d\Psi + \frac{1}{\sqrt{\cal E}} \left(\frac{d\rho}{d\Psi}\right)_{\Psi=0} \right).
}\end{equation}

In practice the right, boundary term is zero as \(\Psi \to 0\) as
\(r\to\infty\), and if \(\rho \propto r^{-n}\) at large \(r\) and
\(\Psi \sim r^{-1}\) then \(d\rho / d\Psi \sim r^{-n+1}\) which goes to
zero provided that \(n > 1\). We take \(\Psi\) from the underlying
assumed analytic dark matter potential. \(\rho_\star\) can be calculated
from the surface density, \(\Sigma_\star\), via the inverse Abel
transform.

We find the stellar profiles created in this manner are stable in the
isolated systems and retail excellent agreement with the assumed stellar
density profile.
