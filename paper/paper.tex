%-----------------------------------------------------------------------
%                                                                 aa.tex
% AA vers. 9.3, LaTeX class for Astronomy & Astrophysics
% Demonstration file
%                                                       (c) EDP Sciences
%-----------------------------------------------------------------------
%
%\documentclass[referee]{aa}    % for a referee version
%\documentclass[onecolumn]{aa}  % for a paper on 1 column  
%\documentclass[longauth]{aa}   % for long lists of authors and/or affiliations. 
                                % This command displays the first eight authors on page 1
                                % and shift the whole list after the references.
                                % Ensure to separate each author with the \and command (see below)
%\documentclass[letter]{aa}     % for the letters
%\documentclass[bibyear]{aa}    % if the references are not structured
                                % according to the author-year natbib style

\documentclass{aa}  

\usepackage{graphicx}
\usepackage{txfonts}
\usepackage{lipsum}
\usepackage{subcaption}         % necessary for continued figures, example in section 3
                                % and appendix
\usepackage{lscape}             % to rotate a single page table, example in appendix.
                                % For landscape tables, see the longtable examples.
\usepackage{placeins}           % useful with \FloatBarrier, to keep 
                                % onecolumn floats from drifting to the next section
                                
%%%%%%%%%%%%%%%%%%%%%%%%%%%%%%%%%%%%%%%%
\usepackage[]{hyperref}
\hypersetup{
    colorlinks=true,
    linkcolor=blue,
    filecolor=magenta,
    urlcolor=blue,
    citecolor=blue,
}


% extra commands
\newcommand{\V}{{\rm v}}
\newcommand{\vmax}{\V_{\rm max}}
\newcommand{\vcirc}{\V_{\rm circ}}
\newcommand{\rmax}{r_{\rm max}}
\newcommand{\kpc}{{\rm kpc}}
\newcommand{\Gyr}{{\rm Gyr}}
\newcommand{\kms}{{\rm km\,s^{-1}}}
\newcommand{\masyr}{{\rm mas\,yr^{-1}}}
\newcommand{\Mo}{M_\odot}

\newcommand{\agama}{{\tt Agama}}
\newcommand{\gadget}{{\tt Gadget-4}}
\newcommand{\smallperi}{{\tt smallperi}}
\newcommand{\LCDM}{$\Lambda$CDM}


%%% citepos command
\makeatletter
\DeclareRobustCommand\citepos
  {\begingroup
   \let\NAT@nmfmt\NAT@posfmt% ...except with a different name format
   \NAT@swafalse\let\NAT@ctype\z@\NAT@partrue
   \@ifstar{\NAT@fulltrue\NAT@citetp}{\NAT@fullfalse\NAT@citetp}}

\let\NAT@orig@nmfmt\NAT@nmfmt
\def\NAT@posfmt#1{\NAT@orig@nmfmt{#1's}}

\makeatother




\begin{document}

%%%%%%%%%%%%%%%%%%%%%%%%%%%%%%%%%%%%%%%%
% if you use custom commands in your title,
% ensure to check your title when submitting!
%%%%%%%%%%%%%%%%%%%%%%%%%%%%%%%%%%%%%%%%
\title{The extended stellar densities of the Sculptor and Ursa Minor dwarf galaxies}

\subtitle{Innate nature or tidal nurture?}

%%%%%%%%%%%%%%%%%%%%%%%%%%%%%%%%%%%%%%%% % Please separate each author with the \and command
%
% Please do not include ORCIDs next to author names.
% Only ORCIDs authenticated by individual authors in EDPS
% editorial system will be taken into account.
% ORCIDs included here will be removed.
%%%%%%%%%%%%%%%%%%%%%%%%%%%%%%%%%%%%%%%%

   \author{D. A. Boyea\inst{1}
        \and others
        }

        \institute{%
            Department of Physics \& Astronomy, University of Victoria, Finnerty Road, Victoria, British Columbia V8A 1A1, Canada
             \email{danielaboyea@gmail.com}
             \thanks{Shows the usage of elements in the institute field}
            \and others\\ }

   \date{Received December XX, 20XX}

% \abstract{}{}{}{}{}
% 5 {} token are mandatory
 
  \abstract
  % context heading (optional)
  % {} leave it empty if necessary  
   {Most dwarf spheroidals (dSph) of the Milky Way follow an exponential surface density profile. However, Sculptor (Scl) and Ursa Minor (UMi) dSphs show an excess of stars at large radii.}
  % aims heading (mandatory)
   {We aim to determine if tides may explain the excess of stars in Sculptor and Ursa Minor.} 
  % methods heading (mandatory)
   {We develop N-body models simulating tidal effects in a Milky Way and Large Magellanic Cloud (LMC) potential. }
  % results heading (mandatory)
   {Neighet galaxy is strongly tidally affected. Because the LMC perturbs Scl's orbit, Scl has few pericentres and the stellar component barely evolves. UMi loses about 90\% of its dark matter mass but the stellar component would not be observationally affected. }
  % conclusions heading (optional), leave it empty if necessary
   {We conclude that the excess of stars likely reflects a divergent assembly history as compared to other local dSphs, possibly explained by past mergers or star formation bursts. }

   \keywords{giant planet formation --
                $\kappa$-mechanism --
                stability of gas spheres
               }

   \maketitle


%%%%%%%%%%%%%%%%%%%%%%%%%%%%%%%%%%%%%%%%%%%%%%%%%%%%%%%%%%%%%%
\section{Introduction}

Since \citet{faber+lin1983}, exponential density profiles have been commonly used for dwarf
spheroidal galaxies \citealt{binggeli+sandage+tarenghi1984, mateo1998, mcconnachie+irwin2006, cicuendez+2018, kowalczyk+2013, martin+2016, MV2020a, battaglia+2022}.
However, Sculptor and Ursa Minor, among Milky Way satellites, show strong deviations from an exponential profile (figure X), often speculated to be a result of tidal interactions
\citep[e.g.,][]{innanen+papp1979, exkridge1988, IH1995, martinez-delgado+2001, walcher+2003, palma+2003} or sometimes a ``halo''of stars
surrounding the dwarf \citep{westfall+2006}.
Most recently, \citet{sestito+2023a, sestito+2023b} report a
``kink'' in the density profile of each galaxy. They spectroscopically
follow up distant stars, finding members as far as 6 and 12 half-light
radii from the centre of each dwarf. If these dwarfs had exponential
profiles, like Fornax, then these far-outlying stars should be much
rarer.

The Local Group hosts several examples of ongoing tidal disruption. The
Magellanic stream, a massive, gas-rich feature emanating from the
Magellanic clouds, is believed to arise partially from the MW's tides
\citep{putman+1998, diaz+bekki2012, donghia+fox2016}. Other clear
examples of tidal streams include the Sagittarius stream, the Andromeda
Giant Southern stream, and the Tucana III stream
\citep[e.g.,][]{ibata+gilmore+irwin1994, ibata+2001, li+2018}. These
examples illustrate that hierarchical accretion remains an active
process. Interpreting such observations relies on simulations of tidal
disruption.

Cosmological simulations struggle to resolve tidal effects on dwarfs.
Since many dwarfs are near the resolution limit, they are vulnerable to
artificial disruption
\citep[e.g.,][]{vandenbosch+2018, santos-santos+2025}. To overcome
numerical challenges, idealized simulations model a single subhalo in an
analytic host potential, achieving excellent numerical convergence \citep[e.g.,][]{hayashi+2003, bullock+johnston2005, klimentowski+2009, ogiya+2019}.

Idealized simulations predict clear properties of tidally disrupting
dwarf spheroidal galaxies. Tidally stripped stars form \emph{tidal
streams}---stellar tails with a bulk velocity gradient
\citep[e.g.,][]{moore+davis1994, johnston+spergel+hernquist1995, read+2006}.
Most mass loss happens near pericentre, where tides are strongest.
However, the central structure of a dwarf galaxy often remains
undisturbed \citep{oh+lin+aarseth1995, piatek+pryor1995}. For instance,
NFW halos are also found to be resilient to full tidal disruption
\citep{EP2020}, but cored dark matter halos may disrupt fully and faster
\citep[e.g.,][]{penarrubia+2010, errani+2023a}.
To first order, tidal mass loss peels away the outer layers of a dwarf
galaxy in energy space.
\citet{drakos+taylor+benson2020, drakos+taylor+benson2022, amorisco2021}
showed that tidal effects are nearly entirely described as the removal
of particles above a truncation energy \citep[see
also][]{choi+weinberg+katz2009}. \citet{stucker+2023} generalized this
idea, creating a model for adiabatic tidal mass loss in an isotropic
tidal field. Their model explains the resilience of NFW halos against
full tidal disruption and the origin of well-defined ``tidal tracks''
\citep[as observed in][]{PNM2008, green+vandenbosch2019, EN2021}.

With precise orbital constraints and improved models of the Milky Way
potential, recent studies have continued to probe the dynamical
histories of individual dwarf galaxies.
\citet{battaglia+sollima+nipoti2015, borukhovetskaya+2022, dicintio+2024}
both ran simulations tuned to Fornax, showing that this galaxy's stellar
component or globular clusters are likely not affected by tides.
Similarly, \citet{borukhovetskaya+2022a} analyzed Crater II, showing
that the present-day structure is challenging to reconcile with NFW
initial conditions and Galactic tides. Most relevantly,
\citet{iorio+2019} also tailored simulations to Scl, finding weak
Galactic tidal influence.

The \textbf{Jacobi radius} represents the approximate radius where stars
become unbound for a galaxy in a circular orbit around a host
galaxy.\footnote{The Jacobi radius was derived at least as early as
  \citet{laplace1798}. This radius also bears other names, such as the
  Hill radius \citep[from][]{hill1878}. Likely only named after Jacobi
  for the Jacobi integral \citep{jacobi1836}.} Calculated from an
approximation of the location of the \(L_1\) and \(L_2\) Lagrange
points, the Jacobi radius is where the mean density of the dwarf galaxy
is roughly three times the mean interior density of the host galaxy at
pericentre, or
\begin{equation}\protect\phantomsection\label{eq:r_jacobi}{
3\bar \rho_{\rm MW}(r_{\rm peri}) \approx \bar \rho_{\rm dwarf}(r_J),
}\end{equation} \citep[ eq. 7-84]{BT1987}. If \(r_J\) is comparable to
the visible extent of a galaxy, we should expect to find clear signs of
tidal disturbance. While strictly valid for circular orbits, assuming
\(r_{\rm peri}\) for the host-dwarf distance works as most stars are
lost during pericentric passages.

We also use the \textbf{break radius} as defined in
\citet{penarrubia+2009}, marking the outermost radius within which the
dwarf has been able to achieve equilibrium after pericentric passage in
a highly eccentric orbit. The break radius \(r_{\rm break}\) is
proportional to the velocity dispersion, \(\sigma_v\), and time elapsed
since pericentre, \(\Delta t\),
\begin{equation}\protect\phantomsection\label{eq:r_break}{
r_{\rm break} = C\,\sigma_{v}\,\Delta t
}\end{equation} where the empirical constant is \(C \approx 0.55\).
\(r_{\rm break}\) describes where the dynamical timescale becomes longer
than the time since the perturbation, i.e., the radius within which the
galaxy is dynamically relaxed.


Building on this body of work, we will use idealized simulations to
understand the severity of tidal effects on Sculptor and Ursa Minor.

\begin{table}[t]
\centering
\caption[Observed Properties of Sculptor]{Observed properties of Sculptor. References are: (1) Muñoz et al. (2018, Sérsic fit), (2) Tran et al. (2022, RR lyrae distance), (3) Alan W. McConnachie and Venn (2020b), (4) Arroyo-Polonio et al. (2024). }
\label{tbl:scl_obs_props}
\begin{tabular}{lll}
\hline
parameter & value & Source\\
\hline
$\alpha / ^\circ$ & $15.0183 \pm 0.0012$ & 1\\
$\delta / ^\circ$ & $-33.7186 \pm 0.0007$ & ”\\
distance modulus & $19.60 \pm 0.05$ & 2\\
distance & $83.2 \pm 2$ kpc & ”\\
$\mu_{\alpha*} / \masyr$  & $0.099 \pm 0.002 \pm 0.017$ & 3\\
$\mu_\delta / \masyr$ & $-0.160 \pm 0.002_{\rm stat} \pm 0.017_{\rm sys}$  & ”\\
$\V_{\rm los}$ / ${\rm km\,s^{-1}}$ & $111.2 \pm 0.3\ $ & 4\\
$\sigma_v$ / ${\rm km\,s^{-1}}$ & $9.7\pm0.2\ $ & ”\\
$R_h/'$ & $9.79 \pm 0.04$ & 1\\
ellipticity & $0.37 \pm 0.01$ & ”\\
position angle & $94\pm1^\circ$ & ”\\
$M_V$ & $-10.82\pm0.14$ & ”\\
\hline
\end{tabular}
\end{table}

\begin{table}[t]
\centering
\caption[Observed Properties of Ursa Minor]{Observed properties of Ursa Minor. References are: (1) Muñoz et al. (2018, Sérsic fit), (2) Garofalo et al. (2025, RR lyrae distance), (3) Alan W. McConnachie and Venn (2020a), (4) Pace et al. (2020), average of MMT and Keck results with systematic uncertainty based on the difference between the MMT and Keck means. }
\label{tbl:umi_obs_props}
\begin{tabular}{lll}
\hline
parameter & value & Source\\
\hline
$\alpha/^\circ$ & $ 227.2420 \pm 0.0045$ & 1\\
$\delta/^\circ$ & $67.2221 \pm 0.0016$ & 1\\
distance modulus & $19.23 \pm 0.11$ & 2\\
distance/kpc & $70.1 \pm 3.6$ & ”\\
$\mu_\alpha*$ / mas yr$^{-1}$  & $-0.124 \pm 0.004 \pm 0.017$ & 3\\
$\mu_\delta$ / mas yr$^{-1}$  & $0.078 \pm 0.004_{\rm stat} \pm 0.017_{\rm sys}$ & ”\\
$\V_{\rm los}$ / km s$^{-1}$  &$-245.9 \pm 0.3_{\rm stat} \pm 1_{\rm sys}$ & 4\\
$\sigma_v$ & $8.6 \pm 0.3$ & 4\\
$R_h$ & $11.62 \pm 0.1$ & 1\\
ellipticity & $0.55 \pm 0.01$ & 1\\
position angle & $50 \pm 1^\circ$ & 1\\
$M_V$ & $-9.03 \pm 0.05$ & 1\\
\hline
\end{tabular}
\end{table}




\section{{\it Gaia} density profiles}

Reducing background-contamination is essential to determine faint structures around dwarf galaxies. Before {\it Gaia}, most studies relied on colour-magnitude selections to select stars more likely to be members. Now with {\it Gaia}, proper motions and parallax information can substantially reduce the background of likely stars, producing low-contamination membership lists for dwarf galaxies.

Here, we use the \citepos[hereafter J+24]{jensen+2024} membership
probabilities from \emph{Gaia} data. J+24 used a Bayesian framework
incorporating proper motion (PM), colour-magnitude diagram (CMD), and
spatial information to determine the probability that a given star
belongs to the satellite or foreground/background. By accounting for PM
in particular, J+24 produced low contamination samples of candidate
member stars out to large distances from a dwarf galaxy. J+24 extended
the algorithm presented in \citet{MV2020a, MV2020b} by additionally
including a secondary, extended spatial component. J+24 detected
candidate members out to \textasciitilde10 half-light radii from the
centres of some galaxies (\(R_h\)). Similar recent work includes
\citet{pace+li2019}; \citet{battaglia+2022}; \citet{pace+erkal+li2022};
\citet{qi+2022}.


For our fiducial sample, we adopt a minimum probability of
\(P_{\rm sat} = 0.2\). We do not filter on magnitudes explicitly, but
J+24's quality cuts typically only include stars with \(G < 21\). We use
the \(P_{\rm sat}\) values from the elliptical 2-component runs if a
galaxy shows evidence for an outer component, the 1-component run
otherwise. Most stars have \(P_{\rm  sat}\) values which are nearly 0 or
1, so the exact choice of probability threshold has little effect on the
resulting sample. Even at our relatively generous probability threshold
of 0.2, the purity remains high when validated against spectroscopic
line-of-sight (LOS) velocities (\textasciitilde90\%, J+24).\footnote{This
  would indicate that the J+24 model probabilities are mis-calibrated.
  However, most LOS surveys of dwarf galaxies select brighter stars
  (which have better \emph{Gaia} measurements) and likely members,
  complicating the interpretation of this purity estimate.} However, we
find that our conclusions are unchanged when limiting samples to only
the brightest stars. Altogether, the J+24 method provides a
high-quality, low-contamination sample of dwarf galaxy candidate member
stars, which we will now investigate in further detail.


Fig.~\ref{fig:observed_density_profiles} compares the fiducial density
profiles of Sculptor, Ursa Minor, Fornax, and other classical dwarf
galaxies. Of the classicals, we exclude Antlia II, due to the extremely
high background, and Sagittarius, which was not included in J+24. The
density profiles are scaled to match at the half-light radius, taken
from \citet{munoz+2018}. All of the classical dwarfs appear to be well
described by an exponential profile in the inner regions.\footnote{Comparing
  density profiles of dwarf galaxies is complicated by variations in the
  effective depth between galaxies. However, the deviations between Scl,
  UMi, and other galaxies are apparent even where the data is complete
  across all dwarfs.} In the outer regions, however, Sculptor and Ursa
Minor deviate and show a clear outer excess over an exponential law
(solid black line). These galaxies are better fit by a Plummer law
(dashed black line). The deviation from an exponential grows outwards,
and at \(\sim 8 R_h\), may reach 2 orders of magnitude. 


\begin{figure}
    \centering
    \includegraphics{figures/classical_dwarf_profiles.pdf}
    {\caption{The density profiles of classical dwarfs, using data from \citet{jensen+2024}.}}
    \label{fig:observed_density_profiles}
\end{figure}

\section{Methods}

\subsection{Milky Way potential and reference frame}\label{milky-way-potential}
To convert observed positions and velocities to Galactocentric
coordinates, we use the Astropy v4 Galactocentric frame
\citep{astropycollaboration+2022}. Our Cartesian Galactocentric
coordinates here assume the Galactic centre is at
\([x, y, z] = [0,0,0]\), where \(x\) is the direction from the sun to
the Galactic centre, \(y\) is the direction of the motion of the Local
Standard of Rest, and \(z\) is the direction perpendicular to the
Galactic plane. The coordinate frame is also right-handed, such that the
\(z\)-angular momentum of the sun is negative (since the sun is at
\(x<0\)). In this frame, the solar position is
\([-8.122 \pm 0.033, 0, 0.0208 \pm 0.003]\, \kpc\)
\citep{gravitycollaboration+2018, bennett+bovy2019} and the solar
velocity is \(\V_\odot = [-12.9 \pm 3.0, 245.6 \pm 1.4, 7.78 \pm 0.08]\)
km/s
\citep{reid+brunthaler2004, drimmel+poggio2018, gravitycollaboration+2018}.
The uncertainties in this reference frame are typically smaller than the
uncertainties on a dwarf galaxy's distance and tangential velocity.


We adopt the Milky Way potential described in \citet{EP2020}, which is
an analytic approximation to that proposed by \citet{mcmillan2011}.
The Galactic bulge is described by a \citet{hernquist1990} potential,
\begin{equation}{
    \Phi(r) = - \frac{GM}{r + a},
}\end{equation} 
where \(a=1.3\,{\rm kpc}\) is the scale radius and
\(M=2.1 \times 10^{10}\,\Mo\) is the total mass. The thin and thick
disks are represented with the \citet{miyamoto+nagai1975} cylindrical
potential, 
\begin{equation}{
    \Phi(R, z) = \frac{-GM}{\sqrt{R^2 + \left(a + \sqrt{z^2 + b^2}\right)^{2}}},
}\end{equation} 
where \(a\) is the disc radial scale length, \(b\) is
the scale height, and \(M\) is the total mass of the disk. For the thin
disk, \(a=3.944\,\)kpc, \(b=0.311\,\)kpc, and
\(M=3.944\times10^{10}\,\)M\(_\odot\). For the thick disk,
\(a=4.4\,\)kpc, \(b=0.92\,\)kpc, and \(M=2\times10^{10}\,\)M\(_\odot\).
The halo is an NFW \citep{NFW1996, 1997} dark matter halo (Eq.~\ref{eq:nfw}) 
\begin{equation}\protect\phantomsection\label{eq:nfw}{
\rho/\rho_s= \frac{1}{(r/r_s)(1+r/r_s)^2},
}\end{equation}
where $r_s$ and $\rho_s$ are the scale mass and density. We parameterize the NFW halo instead in terms of $\rmax$ and $\vmax$, the maximum circular velocity and radius of maximum circular velocity.  
In terms of these parameters, \(\rmax = 43.7\,\)kpc and \(\vmax = 191\,\kms\), or
\(M_{200} = 126.6\times 10^{10}\,\Mo\) and \(r_s=20.2\,\kpc\).


\subsection{Orbits}\label{sec:scl_smallperi}

To explore the possible orbits of a dwarf galaxy, we perform a Monte
Carlo sampling of the present-day observables. The present-day position,
distance modulus, LOS velocity, and proper motions are each sampled from
normal distributions given the reported uncertainties in
Tables~\ref{tbl:scl_obs_props}, \ref{tbl:umi_obs_props}. We integrate
each sampled position/velocity back in time for 10 Gyr using \agama{}
\citep{agama}. Dynamical friction is not expected to impact orbits
substantially, so we assume a single point-mass particle for the backwards
integration.
Figs.~\ref{fig:scl_orbits},~\ref{fig:umi_orbits} shows 100 samples of these orbits.


To maximize tidal effects, we select an orbit with the \(\sim 6\sigma\)
smallest pericentre among all possible orbits integrated backwards for
\(10\,\Gyr\). We achieve this by taking the median parameters of all
orbits with a pericentre less than the 0.0027th quantile pericentre,
yielding a pericentre of 43 kpc. Given the current observations, it is
unlikely that Sculptor has had a significantly smaller pericentre than
our selected orbit, which we refer to as the \smallperi{} orbit. We take
the first apocentre after a look-back time of 10 Gyr, or at
\(\sim9.1\,\Gyr\), as the initial conditions for our model of Sculptor,
noted in Table~\ref{tbl:orbit_ics}.



We iteratively adjust Ursa Minor's initial
conditions to better match the present-day position. Initially, starting with low-resolution runs, we adjust the
cylindrical actions of the initial orbit by the final difference in
actions at the end of orbital evolution. After the initial actions have
converged (2 iterations), we change the initial action angles by the
final difference in action angles. This method converges within 4
iterations to an orbit agreeing with the observed kinematics of Ursa
Minor. Since Sculptor's orbit is less strongly affected by tides, we do
not carry out this correction for Sculptor.

\begin{table}[t]
\centering
\caption{The orbital initial conditions for models presented in this work. The observables record the intended final position of the N-body model derived from point-mass orbit quantiles. Instead, the initial position and velocity represent the initial conditions for the actual N-body model. The \smallperi{} orbits represents the $3\sigma$-smallest-pericentre orbit, which we use to provide an upper limit on tidal effects. We describe the \texttt{LMC-flyby} orbit in Section \ref{sec:scl_lmc}. }
\label{tbl:orbit_ics}
\begin{tabular}{llll}
\hline
Property & Scl & Scl-\verb|LMC| & UMi\\
\hline
$\vmax / \kms$ & 31 & 25 & 38 \\
$\rmax / \kpc$ & 3.2 & 2.5 & 4.0 \\
$R_{s, \rm exp}$ & 0.10 & 0.10 \\
$R_{s, \rm Plummer}$ & 0.20 & 0.20 \\
$N_\star$ \\
\hline
distance $/$ kpc & 82.6 & 73.1 & 64.6\\
$\mu_{\alpha*} / \masyr$ & 0.134 & 0.137 & -0.158\\
$\mu_\delta / \masyr$ & -0.198 & -0.156 & 0.050\\
LOS velocity $/$ $\kms$ & 111.2 & 111.2 & -245.75\\
\hline
$t_i / \Gyr$ & -9.17 & -2.0 & -9.67\\
${x}_{i} / \kpc$ & -2.49 & 4.30 & -17.40\\
${y}_{i} / \kpc$ & -42.78 & 138.89 & 74.51\\
${z}_{i} / \kpc$ & 86.10 & 125.88 & 21.34\\
$\V_{x,\,i} / \kms$ & -20.56 & 6.89 & 14.27\\
$\V_{y,\,i} / \kms$ & -114.83 & -56.83 & 48.62\\
$\V_{z,\,i} / \kms$ & -57.29 & 52.09 & -114.08\\
\hline
\end{tabular}
\end{table}



\subsection{Initial conditions}\label{initial-conditions}

We use \agama{} \citep{agama} to generate the initial N-body dark matter
halo. We assume galaxies are described by a spherical, isotropic NFW
dark matter potential (Eq.~\ref{eq:nfw}). We also assume the stars do
not contribute to the potential. The dark matter density is truncated in
the outer regions by
\begin{equation}\protect\phantomsection\label{eq:trunc_nfw}{
    \rho_{\rm tNFW}(r) = e^{-(r/r_t)^3}\ \rho_{\rm NFW}(r),
}\end{equation}
where we adopt \(r_t = 20 r_s\).


We select cosmologically-motivated initial dark matter halos. First, taking the absolute
magnitudes from \citet{munoz+2018} with the mass-to-light ratio from
\citet{woo+courteau+dekel2008}, the total current stellar mass of
Sculptor and Ursa Minor are \(M_\star \approx 3.1 \times 10^6 \Mo\) and
\(M_\star \approx 7 \times 10^5 \Mo\), respectively. Based on the
stellar mass-\(\vmax\) relation \citep[from][]{fattahi+2018}, Sculptor
and Ursa Minor's halos should have \(\vmax \approx 31 \,\kms\) and
\(\vmax \approx 27\,\kms\). Finally, using the \citet{ludlow+2016}
\(z=0\) mass-concentration relation, this constraint translates into a
\(\rmax \approx 6\,\kpc\) and \(\rmax \approx 5\,\kpc\) for each galaxy.
The cosmological means predict too low a velocity dispersion
(for Scl and
UMi, \(\sigma_{\V, i}  \approx 8.5\,\kms\) and \(8.0\,\kms\) given their observed half-light radius). 
To match the observed \(\sigma_\V\) at the end of tidal evolution, we
choose \(\vmax = 31\,\kms\) and \(\rmax=3.2\,\kpc\) for Scl and
\(\vmax=38\,\kms\) and \(\rmax=4\,\kpc\) for UMi.


\subsection{N-body methods}\label{isolation-runs-and-simulation-parameters}

To ensure that the initial conditions of the simulation are dynamically
relaxed, we run a halo first in isolation using
\gadget{}. Since gravity is scale-free, we use the same isolation run
for all halos and rescale the results to the desired values of size and
mass. We adopt a fiducial value of \(\rmax = 6.0\,\)kpc and
\(\vmax = 31\,\kms\) for the isolation halo based on Sculptor's mean
properties. We run this model for 5 Gyr (about one-half crossing time
\(t_{\rm cross} = 2\pi\,r /\vcirc  \approx 9\,\Gyr\) at
\(r_{200}=36\,\)kpc).

For our simulation parameters, we adopt a softening length of
\begin{equation}\protect\phantomsection\label{eq:softening_length}{
    h_{\rm grav} = 0.014\,{\rm kpc}\left(\frac{r_{\rm max}}{6.0\,{\rm kpc}}\right)\left(\frac{N}{10^7}\right)^{-1/2},
}\end{equation} 
for a halo with \(N\) particles.
Our isolation halo is converged down to a radius $r\approx 100\,{\rm pc}$ (see also the convergence radius from \citealt{power+2003}).

Next, we evolve the halo in the Galactic potential. We scale the relaxed
snapshot and softening length to match the initial halo, and shift the
snapshot to the initial conditions inferred from the orbital analysis
(see Table~\ref{tbl:orbit_ics}). We then evolve the full N-body NFW
model forward in time in the Galactic potential until the present time,
when the halo is closest to its present-day observed position in the MW
halo.\footnote{Specifically, the snapshot which best agrees (as measured
  by \(\chi^2\)) with the intended final position (in
  Table~\ref{tbl:orbit_ics}) and the observed position and velocity
  uncertainties. We use \(R_h\) as the uncertainty in \(\alpha\) and
  \(\delta\).}


We determine the halo centre using a shrinking-spheres centre method inspired by \citet{power+2003}.
First, we start with an initial centre estimate using all bound
particles from the previous snapshot. Then, we calculate the distance of
all particles from the centre, remove particles with a distance beyond
the 0.975 quantile of the centre, and recalculate the centre of mass.
The procedure is repeated until the selection radius is less than
\textasciitilde1kpc or fewer than 0.1\% of particles remain. We then
remove all unbound particles based on the \gadget{} calculated potential
of the halo and recalculate the centre. For all future timesteps, we
consider only particles retained from the previous iteration.

\subsection{Sculptor and Ursa Minor's initial stellar
components}\label{sec:painting_stars}

We ``paint'' stars onto dark matter particles using the particle-tagging
method \citep[e.g.,][]{bullock+johnston2005}, assuming spherical
symmetry. We initially assume stars follow a projected exponential law
(Eq.~\ref{eq:exponential_law}) with \(R_s = 0.10\,\kpc\) for both
galaxies. The tagging method assigns a probability to each dark matter
particle, which is proportional to the ``light-to-mass'' ratio required
to match the assumed stellar light profile. We briefly describe the
procedure next, but refer interested readers to \citet{EP2020}.

Let \(\Psi\) be the potential (normalized to vanish at infinity) and
\({\cal E}\) the binding energy, \({\cal E} = \Psi - 1/2 v^2\). If we
know the distribution function\footnote{i.e., the phase-space density of
  particles. Note that since \(f\) is formally defined in 6D phase
  space, there is an additional ``density of states'' term in order to
  calculate the histogram of particles with \({\cal E}\),
  \(dM/d{\cal E}\) \citep[section 4.4.5,][]{BT1987}.} \(f({\cal E})\),
then we assign a stellar weight to a given particle with energy
\({\cal E}\) using \begin{equation}{
P_\star({\cal E}) = \frac{f_\star({\cal E})}{f_{\rm halo}({\cal E})}.
}\end{equation} We use Eddington inversion to find the distribution
function, 
\begin{equation}{
    f({\cal E}) = \frac{1}{\sqrt{8}\, \pi^2}\left( \int_0^{\cal E} \frac{d^2\rho}{d\Psi^2} \frac{1}{\sqrt{{\cal E} - \Psi}}\ d\Psi + \frac{1}{\sqrt{\cal E}} \left(\frac{d\rho}{d\Psi}\right)_{\Psi=0} \right)
}
\end{equation} 
\citep[eq. 4-140b in][]{BT1987}. In practice the right,
boundary term is zero. We
take \(\Psi\) from the underlying assumed analytic dark matter
potential. \(\rho_\star\) can be calculated from the surface density,
\(\Sigma_\star\), via the inverse Abel transform.


\begin{figure}
    \centering
    \includegraphics{figures/scl_lmc_xyzr_orbits.pdf}
    \caption{
        Orbits of Sculptor with and without the LMC.
    }
    \label{fig:scl_orbits}
\end{figure}


\begin{figure}
    \centering
    \includegraphics{figures/umi_xyzr_orbits.pdf}
    \caption{
        Orbits of Ursa Minor
    }
    \label{fig:umi_orbits}
\end{figure}


\section{Results}

\begin{itemize}
    \item Scl minimally evolves as a result of having one pericentre
    \item The LMC critically changes Scl's tidal history.
    \item UMi loses DM mass but the stellar component, as observed, would not show tidal signatures
    \item Break and Jacobi radii suport these interpretations. Stronger tidal evolution is unlikely under any other orbits in these potentials.
\end{itemize}

\subsection{Sculptor}
\begin{table}[t]
\centering
\caption[Orbits and results for Sculptor in the MW+LMC potential.]{The orbital properties and dark matter evolution for the models including an LMC. Similar to Table \ref{tbl:scl_sim_results} except quantities with respect to the LMC are in parentheses. }
\label{tbl:scl_lmc_sims}
\begin{tabular}{lll}
\hline
Property & random samples & Scl\\
\hline
pericentre $/$ kpc & $44\pm 3$ ($29 \pm 2$) & 39 (20)\\
apocentre $/$ kpc & $218 \pm 8$ & –\\
time of last pericentre $/$ Gyr & $-0.38\pm0.01$ ($-0.11$) & $-0.33$ ($-0.10$)\\
number of pericentres & 1 (1) & 1 (1)\\
Jacobi radius $/$ kpc & $3.3\pm0.2$ ($3.6\pm0.2$) & 2.8 (2.6)\\
Jacobi radius $/$ arcmin & $136 \pm 9$ ($159\pm5$) & 132 (121)\\
final heliocentric distance $/$ kpc & $83.2\pm2$ & 72.9\\
$\V_\textrm{max, f} / \V_\textrm{max, i}$ &  & 0.928\\
$r_\textrm{max, f} / r_\textrm{max, i}$ &  & 0.763\\
fractional final bound mass &  & 0.5402\\
\hline
\end{tabular}
\end{table}



\begin{figure*}
    \centering
    \includegraphics{figures/scl_lmc_sim_images.png}
    \caption{Snapshots of Scl in the LMC and MW potential.}
    \label{fig:scl_sim_images}
\end{figure*}

\begin{figure}
    \centering
    \includegraphics{figures/scl_density_i_f.pdf}
    \caption{Simulated initial and final stellar density profiles for a tidal simulation of Scl in the MW+LMC potential} 
    \label{fig:scl_density_i_f}
\end{figure}


\subsection{Ursa Minor}
\begin{figure*}
    \centering
    \includegraphics{figures/umi_sim_images.png}
    \caption{Snapshots of UMi's tidal evolution in the MW potential}
    \label{fig:umi_sim_images}
\end{figure*}

\begin{figure}
    \centering
    \includegraphics{figures/umi_density_i_f.pdf}
    \caption{Similar to \ref{fig:scl_density_i_f} except for UMi.} 
\end{figure}


%%%%%%%%%%%%%%%%%%%%%%%%%%%%%%%%%%%%%%%%%%%%%%%%%%%%%%%%%%%%%%
\section{Discussion}

\subsection{Comparison with prior work}\label{comparison-with-prior-work}


A few studies have modelled Sculptor or Ursa Minor specifically, finding
similar conclusions. \citet{iorio+2019} applied idealized N-body
simulations to study tidal effects on Scl. They similarly found weak
tidal effects, even for a dark-matter-free model. Most recently,
\citet{tchiorniy+genina2025} also used idealized simulations tuned to
five classical dwarfs with a focus on the inner density, concluding that
tides do not strongly affect equilibrium assumptions or the stellar
component.

Our work extends these models, adding updated structural properties, a
broader range of orbital histories, and the influence of the LMC.
Despite these considerations, we still reach similar conclusions: tidal
effects do not shape Scl or UMi's stellar component. A non-tidal process
likely underlies the origin of these galaxies' outer density excess.


Besides their extended stellar density profiles, Sculptor and Ursa Minor
display other peculiarities that may hold clues to their formation. Scl
and UMi both host at least two distinct chemodynamic populations, as
revealed through their photometric or metallicity-velocity
structure\footnote{Other examples of galaxies with multiple populations
  include Carina \citep{battaglia+2012, fabrizio+2016, kordopatis+2016},
  Fornax
  \citep{battaglia+2006, amorisco+evans2012, delpino+aparicio+hidalgo2015},
  Sextans
  \citep{battaglia+2011, cicuendez+battaglia2018, roederer+2023}, and
  Andromeda II
  \citep{mcconnachie+arimoto+irwin2007, ho+2012, delpino+2017}.}
\citep{tolstoy+2004, battaglia+2008, pace+2020}. The inner population is
younger, higher metallicity, and dynamically colder, whereas the outer
population is older, lower metallicity, and dynamically hotter.
These populations may map onto the observed density excess for Ursa Minor, and \citepos{arroyo-polonio+2024} third population would reside within the density excess.

%Ursa Minor has also shown evidence for possible inner substructure, such
%as stellar or kinematic ``clumps''
%\citep[e.g.,][]{olszewski+aaronson1985, demers+1995, kleyna+1998, battinelli+demers1999, bellazzini+2002}.
%While one clump was re-detected kinematically in \citet{pace+2014},
%\citet{munoz+2018} find no evidence of substructure with modern
%photometry. 

\subsection{Discussion and Conclusions}\label{sec:stellar_halos}
In this work, we have investigated the extended, outer density excess
of Sculptor and Ursa Minor and its possible tidal origin. We have shown
that the density profiles are robust, implying that Scl and UMi contain
a true density excess relative to other classical dwarfs.

We then investigated whether tides are a possible explanation. By
modelling each galaxy based on cosmological initial conditions, we
showed that tides do not strongly affect either galaxy. The LMC changes
the orbital history of Scl and UMi, but tides become even weaker in a
combined LMC and MW potential. We conclude that recent tides are
unlikely to shape the outer radial distribution of stars in Scl and UMi.

\textbf{Episodic star formation.} Star formation may quench and
reignite, creating successive stellar generations with differing
distributions. External star formation triggers include tidal
compression \citep{mayer+2001a, dong+lin+murray2003}, collisions with
gaseous filaments \citep{genina+2019}, perturbations from dark halos
\citep{starkenburg+helmi+sales2016}, or shocks with the MW corona
\citep{wright+2019}. More common mechanisms, like feedback or
reionization-driven quenching, may also form multiple stellar
generations
\citep{kawata+2006, benitez-llambay+2015, revaz+jablonka2018}. However,
such processes would not explain why extended stellar populations appear
to be non-universal. If an extended population was formed during a
separate star formation event, then the star formation history should
contain evidence of a corresponding burst.

\textbf{Major mergers.} Dwarf galaxy mergers may be relatively common
\citep{deason+wetzel+garrison-kimmel2014}. After a merger, stars from
the lower-mass galaxy are preferentially dispersed, forming an extended
stellar component and population gradient
\citep{benitez-llambay+2016, deason+2022}. If the galaxies contain gas,
the merger can also trigger new star formation, forming a younger
population of stars \citep[e.g.,][]{genina+2019}. A few local dwarfs are
suspected to have undergone a major merger, including Tucana II,
Andromeda II, and Phoenix
\citep{lokas+2014, fouquet+2017, tarumi+yoshida+frebel2021, cardona-barrero+2021, querci+2025}.
A galaxy having undergone a major merger should harbour at least two or
three populations with distinct origin.

Just as the Milky Way's halo is likely built from many minor mergers,
dwarf galaxies may have built accreted ``stellar halos'' composed of
even fainter satellites \citep{ricotti+polisensky+cleland2022}. Few
examples of dwarf-dwarf accretion are known, but a possible stream
around And II or chemical peculiarities around Sextans I may be such
instances \citep{amorisco+evans+vandeven2014, roederer+2023}. In this
scenario, the extended stellar populations would contain chemical
signatures from several distinct ultra-faint dwarf galaxies.

\textbf{Tidal preprocessing}. Some dwarf galaxies may have been tidally
``preprocessed'' by a massive satellite like the LMC
\citep[e.g.,][]{santistevan+2023, riley+2024}. Tidal preprocessing
redistributes already-present stellar populations and may mimic a
stellar halo. Key evidence suggestive of preprocessing may include
distant dwarf stars or stellar streams.

\textbf{Dark-matter-free dwarfs.} Tidal dwarfs are galaxies that formed
in gas-rich tidal streams during the merger of two larger galaxies
\citep[e.g.,][]{mirabel+dottori+lutz1992, bournaud+duc2006}. Without a
dark matter halo, tidal dwarfs may be more susceptible to tides, forming
extended density profiles and inflating the velocity dispersion
\citep{casas+2012, yang+2014, wang+2024a}.

Similarly, in Modified Newtonian Dynamics (MOND), galaxies do not
contain dark matter. Instead, the gravitational force law is altered to
explain rotation curves. Dwarfs would similarly experience stronger
tidal effects in MOND \citep{mcgaugh+wolf2010, brada+milgrom2000}.

In both cases, tides more plausibly produce extended density profiles.
However, recovering the observed velocity dispersions often requires
ongoing tidal disruption
\citetext{\citealp{mcgaugh+wolf2010}; \citealp[but see
also][]{sanchez-salcedo+hernandez2007}}. Current data do not show
evidence of such features {\bf Ref}---future data would be needed to uncover
signs of disruption to support this hypothesis.

\textbf{Dynamical heating.} Old stars in dwarf galaxies may be
dynamically hotter than younger stars due to processes including stellar
feedback
\citep{stinson+2009, maxwell+2012, el-badry+2016, mercado+2021},
sub-subhalo interactions \citep{penarrubia+2025}, or even fuzzy dark
matter interference fringes
\citep[e.g.,][]{el-zant+2020, duttachowdhury+2023}. However, most of
these processes should operate similarly across dwarf
galaxies---extended stellar populations should be more common if these
processes are important.

We have reviewed a number of different, possibly concurrent explanations
for the formation of extended populations. While we leave the nature of
Scl and UMi's extended stellar densities an open question, we can
discuss possible clues to different formation scenarios.
\emph{Precise chemistry}. Chemical abundances, comparing the inner and
outer regions, can test if the halo was formed within the dwarf or
accreted. If external in origin, chemistry could reveal the properties
of any past mergers.
\emph{Detailed kinematics}. For models relying on recent tidal
disruption, kinematic disequilibrium features should be visible. These
would appear as velocity gradients, increasing velocity dispersions,
outward-biased moving stars, or non-phase mixed structures
\citep[e.g,][]{kroupa1997, read+2006, sanchez-salcedo+hernandez2007}. In
addition, measurements of the ellipticity and anisotropy would help
verify or alter our understanding of the mass structure and whether
DM-free models may be permissible.
\emph{Deep photometry} may find or rule out signs of dynamical
disequilibrium and tidal tails for tidally susceptible models (e.g.,
\emph{pre-processing} and \emph{DM-free dwarfs}). In addition,
photometry will help constrain the prevalence and nature of extended
features in dwarf galaxies.
\emph{Star formation histories} can be derived through photometry or
chemistry. Evidence of significant star formation episodes or lack
thereof may differentiate scenarios that rely on strong star formation
bursts (e.g., \emph{episodic star formation history, induced star
formation,} and \emph{gas-rich mergers}).

Both upcoming observatories and novel simulation methods will be instrumental to uncovering the possible origins of the extended stellar densities in Sculptor and Ursa Minor.


%%%%%%%%%%%%%%%%%%%%%%%%%%%%%%%%%%%%%%%%%%%%%%%%%%%%%%%%%%%%%%
\section{Conclusions}



%%%%%%%%%%%%%%%%%%%%%%%%%%%%%%%%%%%%%%%%%%%%%%%%%%%%%%%%%%%%%%
\begin{acknowledgements}
\end{acknowledgements}



\bibliographystyle{bibtex/aa.bst} % style aa.bst
\bibliography{paper.bib} % your references Yourfile.bib


\end{document}
